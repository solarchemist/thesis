% Created by tikzDevice version 0.12.4 on 2023-12-15 02:04:03
% !TEX encoding = UTF-8 Unicode
\RequirePackage{luatex85}
\documentclass{LuaUUThesis}
\nonstopmode

\usepackage{tikz}
\usepackage{luatex85}
\usepackage[active,tightpage,psfixbb]{preview}
\usepackage{fontspec}
\setsansfont{Linux Biolinum O}
\setmainfont[Ligatures=TeX]{TeX Gyre Pagella}\linespread{1.05}
\usepackage{mathtools}
\usepackage{unicode-math}
\setmathfont{TeX Gyre Pagella Math}
\setmonofont[Scale=0.85]{DejaVu Sans Mono}
\PreviewEnvironment{pgfpicture}
\setlength\PreviewBorder{0pt}
\usepackage{booktabs}
\usepackage{ctable}
\usepackage[backend=biber]{biblatex}
\usepackage{xfrac}
\usetikzlibrary{ arrows, backgrounds, calc, decorations, decorations.markings, decorations.pathmorphing, decorations.pathreplacing, decorations.shapes, decorations.text, fadings, fit, positioning, shapes }
\usepackage[ minimal=true, modules={ thermodynamics, units, orbital, reactions, redox, reactants, scheme} ]{chemmacros}
\chemsetup{ formula=chemformula, chemformula/frac-style=nicefrac, reactions/list-entry={}, reactants/printreactants-style=xltabular, reactants/initiate=true }
\NewChemLatin{\exsitu}{ex~situ}
\NewChemLatin{\Exsitu}{Ex~situ}
\NewChemLatin{\Insitu}{In~situ}
\NewChemParticle{\proton}{H+}
\NewChemParticle{\electron}{e-}
\NewChemParticle{\hole}{h+}
\NewChemParticle{\photon}{\ensuremath{h\nu}}
\NewChemParticle{\oxide}{O^{2-}}
\NewChemParticle{\hydroxide}{OH-}
\NewChemParticle{\HydRad}{^.OH}
\NewChemParticle{\hydroxyl}{OH}
\NewChemParticle{\zincox}{ZnO}
\NewChemParticle{\ZnO}{ZnO}
\NewChemParticle{\ironox}{Fe2O3}
\NewChemParticle{\tinox}{SnO2}
\NewChemParticle{\EtOH}{EtOH}
\NewChemParticle{\LiOH}{LiOH * H2O}
\NewChemParticle{\CdS}{CdS}
\NewChemParticle{\hydrogen}{H2}
\NewChemParticle{\oxygen}{O2}
\NewChemParticle{\carbonox}{CO2}
\NewChemParticle{\carbondiox}{CO2}
\NewChemParticle\ZnAc{Zn(OAc)2}
\NewChemParticle\ZnOAc{Zn(OAc)2}
\NewChemParticle{\hematite}{\chemalpha{}‐Fe2O3}
\usepackage{chemfig}
\setchemfig{atom sep=2em}
\definesubmol\nobond{-[,1.2,,,draw=none]}
\renewcommand*\printatom[1]{\ensuremath{\mathsf{#1}}}
\usepackage{siunitx}
\sisetup{ uncertainty-mode=compact, reset-text-family=false, text-family-to-math=false, text-series-to-math=false }
\DeclareSIUnit{\counts}{cts}
\DeclareSIUnit{\cps}{cps}
\DeclareSIUnit{\lines}{lines}
\DeclareSIUnit{\vsAgCl}{\vs~\ch{Ag}/\ch{AgCl}}
\DeclareSIUnit{\vsSHE}{\vs~SHE}
\DeclareSIUnit{\vsNHE}{\vs~NHE}
\DeclareSIUnit{\vsSCE}{\vs~SCE}
\DeclareSIUnit{\vsAVS}{\vs~AVS}
\DeclareSIUnit{\voltAgCl}{\volt\vsAgCl}
\DeclareSIUnit{\voltSHE}{\volt\vsSHE}
\DeclareSIUnit{\voltNHE}{\volt\vsNHE}
\DeclareSIUnit{\voltSCE}{\volt\vsSCE}
\DeclareSIUnit{\voltAVS}{\volt\vsAVS}
\DeclareSIUnit{\rpm}{rpm}
\DeclareSIUnit{\day}{day}
\DeclareSIUnit{\days}{days}
\DeclareSIUnit{\year}{year}
\DeclareSIUnit{\yearswe}{år}
\DeclareSIUnit{\years}{years}
\DeclareSIUnit{\yr}{year}
\DeclareSIUnit{\yrs}{year}
\DeclareSIUnit{\TJ}{\tera\joule}
\DeclareSIUnit{\GJ}{\giga\joule}
\DeclareSIUnit{\MJ}{\mega\joule}
\DeclareSIUnit{\kJ}{\kilo\joule}
\DeclareSIUnit{\TW}{\tera\watt}
\DeclareSIUnit[inter-unit-product={}]{\TWh}{\tera\watt\hour}
\DeclareSIUnit[inter-unit-product={}]{\MWh}{\mega\watt\hour}
\DeclareSIUnit[inter-unit-product={}]{\Wh}{\watt\hour}
\DeclareSIUnit{\MB}{\mega\byte}
\DeclareSIUnit{\GB}{\giga\byte}
\DeclareSIUnit{\TB}{\tera\byte}
\DeclareSIUnit{\USD}{USD}
\DeclareSIUnit{\EUR}{EUR}
\DeclareSIUnit{\SEK}{SEK}
\DeclareSIUnit{\foot}{ft}
\DeclareSIUnit{\sq}{sq}
\DeclareSIUnit{\astronomicalunit}{AU}
\DeclareSIUnit{\AU}{\astronomicalunit}
\DeclareSIUnit{\ppm}{ppm}
\DeclareSIUnit{\ppb}{ppb}
\DeclareSIUnit{\angstrom}{Å}
\usepackage{xifthen}
\usepackage{xparse}
\ExplSyntaxOn \NewExpandableDocumentCommand{\stringcase}{mO{}m}{ \str_case_e:nnF { #1 } { #3 } { #2 } } \ExplSyntaxOff  \NewDocumentCommand{\NonResonantModesShort}{m}{ \stringcase{#1}[\textbf{??}]{ {E2l}{\textit{E}\textsubscript{\textup{2l}}} {B1l}{\textit{B}\textsubscript{\textup{1l}}} {A1TO}{\textit{A}\textsubscript{\textup{1(TO)}}} {E1TO}{\textit{E}\textsubscript{\textup{1(TO)}}} {E2h}{\textit{E}\textsubscript{\textup{2h}}} {E1LO}{\textit{E}\textsubscript{\textup{1(LO)}}} {B1h}{\textit{B}\textsubscript{\textup{1h}}} {A1LO}{\textit{A}\textsubscript{\textup{1(LO)}}} {2E2l}{\textup{2}\textit{E}\textsubscript{\textup{2l}}} {2E2h}{\textup{2}\textit{E}\textsubscript{\textup{2h}}} {2E1LO}{\textup{2}\textit{E}\textsubscript{\textup{1(LO)}}} {2A1LO}{\textup{2}\textit{A}\textsubscript{\textup{1(LO)}}} } }  \NewDocumentCommand{\NonResonantModesLong}{m}{ \stringcase{#1}[\textbf{??}]{ {E2l}{\textit{E}\textsubscript{\textup{2(low)}}} {B1l}{\textit{B}\textsubscript{\textup{1(low)}}} {A1TO}{\textit{A}\textsubscript{\textup{1(TO)}}} {E1TO}{\textit{E}\textsubscript{\textup{1(TO)}}} {E2h}{\textit{E}\textsubscript{\textup{2(high)}}} {E1LO}{\textit{E}\textsubscript{\textup{1(LO)}}} {B1h}{\textit{B}\textsubscript{\textup{1(high)}}} {A1LO}{\textit{A}\textsubscript{\textup{1(LO)}}} {2E2l}{\textup{2}\textit{E}\textsubscript{\textup{2(low)}}} {2E2h}{\textup{2}\textit{E}\textsubscript{\textup{2(high)}}} {2E1LO}{\textup{2}\textit{E}\textsubscript{\textup{1(LO)}}} {2A1LO}{\textup{2}\textit{A}\textsubscript{\textup{1(LO)}}} } }  \NewDocumentCommand{\NonResonantModesLogic}{m O{short}}{ \IfNoValueTF{#2}{ \NonResonantModesShort{#1} }{ \ifthenelse{\equal{#2}{long}}{ \NonResonantModesLong{#1} }{ \ifthenelse{\equal{#2}{short}}{ \NonResonantModesShort{#1} }{ \NonResonantModesShort{} } } } }  \NewDocumentCommand{\nonresmode}{m o O{short} O{sum}}{ \IfNoValueTF{#2}{ \IfNoValueTF{#3}{ \NonResonantModesLogic{#1} }{ \ifthenelse{\isempty{#3}}{ \NonResonantModesLogic{#1} }{ \NonResonantModesLogic{#1}[#3] } } }{ \ifthenelse{\isempty{#2}}{ \IfNoValueTF{#3}{ \NonResonantModesLogic{#1} }{ \ifthenelse{\isempty{#3}}{ \NonResonantModesLogic{#1} }{ \NonResonantModesLogic{#1}[#3] } } }{ \ifthenelse{\equal{#4}{diff}}{ \ifthenelse{\isempty{#3}}{ \NonResonantModesLogic{#1}\ensuremath{\mkern0mu-\mkern0mu}\NonResonantModesLogic{#2} }{ \ifthenelse{\equal{#3}{long}}{ \NonResonantModesLogic{#1}[#3]\ensuremath{\mkern1.5mu-\mkern1.5mu}\NonResonantModesLogic{#2}[#3] }{ \NonResonantModesLogic{#1}\ensuremath{\mkern1.5mu-\mkern1.5mu}\NonResonantModesLogic{#2} } } }{ \ifthenelse{\isempty{#3}}{ \NonResonantModesLogic{#1}\ensuremath{\mkern0mu+\mkern0mu}\NonResonantModesLogic{#2} }{ \ifthenelse{\equal{#3}{long}}{ \NonResonantModesLogic{#1}[#3]\ensuremath{\mkern1.5mu+\mkern1.5mu}\NonResonantModesLogic{#2}[#3] }{ \NonResonantModesLogic{#1}\ensuremath{\mkern1.5mu+\mkern1.5mu}\NonResonantModesLogic{#2} } } } } } }   \NewDocumentCommand{\Eg}{mo}{ \ifthenelse{\isempty{#1}}{ \ensuremath{E_\text{g}}}{ \IfNoValueTF{#2}{ \ensuremath{E_\text{g}{\,=\,}\qty{#1}{\eV}}}{ \ifthenelse{\isempty{#2}}{ \ensuremath{E_\text{g}{\,=\,}\qty{#1}{\eV}}}{ \ensuremath{E_\text{g}{\,=\,}\qty{#1}{#2}}}}}}   \NewDocumentCommand{\Rsq}{om}{ \IfNoValueTF{#1}{ \ensuremath{R^2{\,=\,}\num{#2}}}{ \ensuremath{R_\text{#1}^2{\,=\,}\num{#2}}}}   \NewDocumentCommand{\density}{mo}{ \ifthenelse{\isempty{#1}}{ \ensuremath{\rho}}{ \IfNoValueTF{#2}{ \ensuremath{\rho{\,=\,}\qty{#1}{\gram\per\cubic\cm}}}{ \ifthenelse{\isempty{#2}}{ \ensuremath{\rho{\,=\,}\qty{#1}{\gram\per\cubic\cm}}}{ \ensuremath{\rho{\,=\,}\qty{#1}{#2}}}}}}   \NewDocumentCommand{\miller}{m}{ \num{#1} }
\newcommand{\SweaveOpts}[1]{}  % do not interfere with LaTeX
\newcommand{\SweaveInput}[1]{} % because they are not real TeX commands
\newcommand{\Sexpr}[1]{}       % will only be parsed by R


% printedition: configure thesis for physical printing (webedition by default)
% openright: force chapters to start on right-hand page (important for print edition)
% titles: show current chapter and section in top outer corner of each page (like a proper book)
% note that cleveref language switching does not work when languages are specified
% as babel package options, they need to be here. The main language should be loaded last.
% https://tex.stackexchange.com/a/20991/10824
% https://tex.stackexchange.com/questions/248341/how-to-make-a-custom-crefformat-dependent-on-language-chosen-with-babel/252847

%\input{assets/preamble/preamble.Rnw}


% For "LaTeX Workshop" (VSCodium extension) to do its autocomplete magic,
% our bib-files must be included here, in the main Rnw
% That also means \usepackage{biblatex} must be included in tikzLualatexPackages
\addbibresource{assets/references/library.bib}


\begin{document}

\begin{tikzpicture}[x=1pt,y=1pt]
\definecolor{fillColor}{RGB}{255,255,255}
\path[use as bounding box,fill=fillColor,fill opacity=0.00] (0,0) rectangle ( 55.65,152.49);
\begin{scope}
\path[clip] (  0.00,  0.00) rectangle ( 55.65,152.49);
\definecolor{drawColor}{RGB}{255,255,255}
\definecolor{fillColor}{RGB}{255,255,255}

\path[draw=drawColor,line width= 0.5pt,line join=round,line cap=round,fill=fillColor] (  0.00,  0.00) rectangle ( 55.65,152.49);
\end{scope}
\begin{scope}
\path[clip] (  6.75, 24.71) rectangle ( 36.98,147.99);
\definecolor{fillColor}{RGB}{255,255,255}

\path[fill=fillColor] (  6.75, 24.71) rectangle ( 36.98,147.99);
\definecolor{drawColor}{RGB}{169,153,133}

\path[draw=drawColor,line width= 0.6pt,line join=round] ( 18.70, 31.33) --
	( 19.47, 56.72) --
	( 16.34, 77.04) --
	( 15.74, 99.05) --
	( 18.12,119.36) --
	( 33.53,141.37);

\path[draw=drawColor,line width= 1.1pt,line join=round] ( 35.60,140.35) --
	( 35.60,142.39);

\path[draw=drawColor,line width= 1.1pt,line join=round] ( 35.60,141.37) --
	( 31.46,141.37);

\path[draw=drawColor,line width= 1.1pt,line join=round] ( 31.46,140.35) --
	( 31.46,142.39);

\path[draw=drawColor,line width= 1.1pt,line join=round] ( 22.61, 30.31) --
	( 22.61, 32.34);

\path[draw=drawColor,line width= 1.1pt,line join=round] ( 22.61, 31.33) --
	( 14.78, 31.33);

\path[draw=drawColor,line width= 1.1pt,line join=round] ( 14.78, 30.31) --
	( 14.78, 32.34);

\path[draw=drawColor,line width= 1.1pt,line join=round] ( 19.03, 98.03) --
	( 19.03,100.06);

\path[draw=drawColor,line width= 1.1pt,line join=round] ( 19.03, 99.05) --
	( 12.45, 99.05);

\path[draw=drawColor,line width= 1.1pt,line join=round] ( 12.45, 98.03) --
	( 12.45,100.06);

\path[draw=drawColor,line width= 1.1pt,line join=round] ( 22.66, 55.71) --
	( 22.66, 57.74);

\path[draw=drawColor,line width= 1.1pt,line join=round] ( 22.66, 56.72) --
	( 16.27, 56.72);

\path[draw=drawColor,line width= 1.1pt,line join=round] ( 16.27, 55.71) --
	( 16.27, 57.74);

\path[draw=drawColor,line width= 1.1pt,line join=round] ( 18.32, 76.02) --
	( 18.32, 78.05);

\path[draw=drawColor,line width= 1.1pt,line join=round] ( 18.32, 77.04) --
	( 14.36, 77.04);

\path[draw=drawColor,line width= 1.1pt,line join=round] ( 14.36, 76.02) --
	( 14.36, 78.05);

\path[draw=drawColor,line width= 1.1pt,line join=round] ( 28.11,118.35) --
	( 28.11,120.38);

\path[draw=drawColor,line width= 1.1pt,line join=round] ( 28.11,119.36) --
	(  8.12,119.36);

\path[draw=drawColor,line width= 1.1pt,line join=round] (  8.12,118.35) --
	(  8.12,120.38);
\definecolor{fillColor}{RGB}{169,153,133}

\path[draw=drawColor,line width= 0.4pt,line join=round,line cap=round,fill=fillColor] ( 33.53,141.37) circle (  1.96);

\path[draw=drawColor,line width= 0.4pt,line join=round,line cap=round,fill=fillColor] ( 18.70, 31.33) circle (  1.96);

\path[draw=drawColor,line width= 0.4pt,line join=round,line cap=round,fill=fillColor] ( 15.74, 99.05) circle (  1.96);

\path[draw=drawColor,line width= 0.4pt,line join=round,line cap=round,fill=fillColor] ( 19.47, 56.72) circle (  1.96);

\path[draw=drawColor,line width= 0.4pt,line join=round,line cap=round,fill=fillColor] ( 16.34, 77.04) circle (  1.96);

\path[draw=drawColor,line width= 0.4pt,line join=round,line cap=round,fill=fillColor] ( 18.12,119.36) circle (  1.96);
\definecolor{drawColor}{gray}{0.20}

\path[draw=drawColor,line width= 0.5pt,line join=round,line cap=round] (  6.75, 24.71) rectangle ( 36.98,147.99);
\end{scope}
\begin{scope}
\path[clip] (  0.00,  0.00) rectangle ( 55.65,152.49);
\definecolor{drawColor}{gray}{0.20}

\path[draw=drawColor,line width= 0.5pt,line join=round] ( 36.98, 31.33) --
	( 39.23, 31.33);

\path[draw=drawColor,line width= 0.5pt,line join=round] ( 36.98, 56.72) --
	( 39.23, 56.72);

\path[draw=drawColor,line width= 0.5pt,line join=round] ( 36.98, 77.04) --
	( 39.23, 77.04);

\path[draw=drawColor,line width= 0.5pt,line join=round] ( 36.98, 99.05) --
	( 39.23, 99.05);

\path[draw=drawColor,line width= 0.5pt,line join=round] ( 36.98,119.36) --
	( 39.23,119.36);

\path[draw=drawColor,line width= 0.5pt,line join=round] ( 36.98,141.37) --
	( 39.23,141.37);
\end{scope}
\begin{scope}
\path[clip] (  0.00,  0.00) rectangle ( 55.65,152.49);
\definecolor{drawColor}{RGB}{0,0,0}

\node[text=drawColor,anchor=base west,inner sep=0pt, outer sep=0pt, scale=  0.68] at ( 41.03, 29.00) {35};

\node[text=drawColor,anchor=base west,inner sep=0pt, outer sep=0pt, scale=  0.68] at ( 41.03, 54.40) {50};

\node[text=drawColor,anchor=base west,inner sep=0pt, outer sep=0pt, scale=  0.68] at ( 41.03, 74.71) {62};

\node[text=drawColor,anchor=base west,inner sep=0pt, outer sep=0pt, scale=  0.68] at ( 41.03, 96.72) {75};

\node[text=drawColor,anchor=base west,inner sep=0pt, outer sep=0pt, scale=  0.68] at ( 41.03,117.04) {87};

\node[text=drawColor,anchor=base west,inner sep=0pt, outer sep=0pt, scale=  0.68] at ( 41.03,139.05) {100};
\end{scope}
\begin{scope}
\path[clip] (  0.00,  0.00) rectangle ( 55.65,152.49);
\definecolor{drawColor}{gray}{0.20}

\path[draw=drawColor,line width= 0.5pt,line join=round] ( 14.06, 22.46) --
	( 14.06, 24.71);

\path[draw=drawColor,line width= 0.5pt,line join=round] ( 27.24, 22.46) --
	( 27.24, 24.71);
\end{scope}
\begin{scope}
\path[clip] (  0.00,  0.00) rectangle ( 55.65,152.49);
\definecolor{drawColor}{RGB}{0,0,0}

\node[text=drawColor,anchor=base,inner sep=0pt, outer sep=0pt, scale=  0.68] at ( 14.06, 16.01) {0.36};

\node[text=drawColor,anchor=base,inner sep=0pt, outer sep=0pt, scale=  0.68] at ( 27.24, 16.01) {0.42};
\end{scope}
\begin{scope}
\path[clip] (  0.00,  0.00) rectangle ( 55.65,152.49);
\definecolor{drawColor}{RGB}{0,0,0}

\node[text=drawColor,anchor=base,inner sep=0pt, outer sep=0pt, scale=  0.90] at ( 21.86,  6.25) {Q$(I)$};
\end{scope}
\end{tikzpicture}

\end{document}

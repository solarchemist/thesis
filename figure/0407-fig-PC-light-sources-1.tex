% Created by tikzDevice version 0.12.4 on 2023-12-15 02:01:10
% !TEX encoding = UTF-8 Unicode
\RequirePackage{luatex85}
\documentclass{LuaUUThesis}
\nonstopmode

\usepackage{tikz}
\usepackage{luatex85}
\usepackage[active,tightpage,psfixbb]{preview}
\usepackage{fontspec}
\setsansfont{Linux Biolinum O}
\setmainfont[Ligatures=TeX]{TeX Gyre Pagella}\linespread{1.05}
\usepackage{mathtools}
\usepackage{unicode-math}
\setmathfont{TeX Gyre Pagella Math}
\setmonofont[Scale=0.85]{DejaVu Sans Mono}
\PreviewEnvironment{pgfpicture}
\setlength\PreviewBorder{0pt}
\usepackage{booktabs}
\usepackage{ctable}
\usepackage[backend=biber]{biblatex}
\usepackage{xfrac}
\usetikzlibrary{ arrows, backgrounds, calc, decorations, decorations.markings, decorations.pathmorphing, decorations.pathreplacing, decorations.shapes, decorations.text, fadings, fit, positioning, shapes }
\usepackage[ minimal=true, modules={ thermodynamics, units, orbital, reactions, redox, reactants, scheme} ]{chemmacros}
\chemsetup{ formula=chemformula, chemformula/frac-style=nicefrac, reactions/list-entry={}, reactants/printreactants-style=xltabular, reactants/initiate=true }
\NewChemLatin{\exsitu}{ex~situ}
\NewChemLatin{\Exsitu}{Ex~situ}
\NewChemLatin{\Insitu}{In~situ}
\NewChemParticle{\proton}{H+}
\NewChemParticle{\electron}{e-}
\NewChemParticle{\hole}{h+}
\NewChemParticle{\photon}{\ensuremath{h\nu}}
\NewChemParticle{\oxide}{O^{2-}}
\NewChemParticle{\hydroxide}{OH-}
\NewChemParticle{\HydRad}{^.OH}
\NewChemParticle{\hydroxyl}{OH}
\NewChemParticle{\zincox}{ZnO}
\NewChemParticle{\ZnO}{ZnO}
\NewChemParticle{\ironox}{Fe2O3}
\NewChemParticle{\tinox}{SnO2}
\NewChemParticle{\EtOH}{EtOH}
\NewChemParticle{\LiOH}{LiOH * H2O}
\NewChemParticle{\CdS}{CdS}
\NewChemParticle{\hydrogen}{H2}
\NewChemParticle{\oxygen}{O2}
\NewChemParticle{\carbonox}{CO2}
\NewChemParticle{\carbondiox}{CO2}
\NewChemParticle\ZnAc{Zn(OAc)2}
\NewChemParticle\ZnOAc{Zn(OAc)2}
\NewChemParticle{\hematite}{\chemalpha{}‐Fe2O3}
\usepackage{chemfig}
\setchemfig{atom sep=2em}
\definesubmol\nobond{-[,1.2,,,draw=none]}
\renewcommand*\printatom[1]{\ensuremath{\mathsf{#1}}}
\usepackage{siunitx}
\sisetup{ uncertainty-mode=compact, reset-text-family=false, text-family-to-math=false, text-series-to-math=false }
\DeclareSIUnit{\counts}{cts}
\DeclareSIUnit{\cps}{cps}
\DeclareSIUnit{\lines}{lines}
\DeclareSIUnit{\vsAgCl}{\vs~\ch{Ag}/\ch{AgCl}}
\DeclareSIUnit{\vsSHE}{\vs~SHE}
\DeclareSIUnit{\vsNHE}{\vs~NHE}
\DeclareSIUnit{\vsSCE}{\vs~SCE}
\DeclareSIUnit{\vsAVS}{\vs~AVS}
\DeclareSIUnit{\voltAgCl}{\volt\vsAgCl}
\DeclareSIUnit{\voltSHE}{\volt\vsSHE}
\DeclareSIUnit{\voltNHE}{\volt\vsNHE}
\DeclareSIUnit{\voltSCE}{\volt\vsSCE}
\DeclareSIUnit{\voltAVS}{\volt\vsAVS}
\DeclareSIUnit{\rpm}{rpm}
\DeclareSIUnit{\day}{day}
\DeclareSIUnit{\days}{days}
\DeclareSIUnit{\year}{year}
\DeclareSIUnit{\yearswe}{år}
\DeclareSIUnit{\years}{years}
\DeclareSIUnit{\yr}{year}
\DeclareSIUnit{\yrs}{year}
\DeclareSIUnit{\TJ}{\tera\joule}
\DeclareSIUnit{\GJ}{\giga\joule}
\DeclareSIUnit{\MJ}{\mega\joule}
\DeclareSIUnit{\kJ}{\kilo\joule}
\DeclareSIUnit{\TW}{\tera\watt}
\DeclareSIUnit[inter-unit-product={}]{\TWh}{\tera\watt\hour}
\DeclareSIUnit[inter-unit-product={}]{\MWh}{\mega\watt\hour}
\DeclareSIUnit[inter-unit-product={}]{\Wh}{\watt\hour}
\DeclareSIUnit{\MB}{\mega\byte}
\DeclareSIUnit{\GB}{\giga\byte}
\DeclareSIUnit{\TB}{\tera\byte}
\DeclareSIUnit{\USD}{USD}
\DeclareSIUnit{\EUR}{EUR}
\DeclareSIUnit{\SEK}{SEK}
\DeclareSIUnit{\foot}{ft}
\DeclareSIUnit{\sq}{sq}
\DeclareSIUnit{\astronomicalunit}{AU}
\DeclareSIUnit{\AU}{\astronomicalunit}
\DeclareSIUnit{\ppm}{ppm}
\DeclareSIUnit{\ppb}{ppb}
\DeclareSIUnit{\angstrom}{Å}
\usepackage{xifthen}
\usepackage{xparse}
\ExplSyntaxOn \NewExpandableDocumentCommand{\stringcase}{mO{}m}{ \str_case_e:nnF { #1 } { #3 } { #2 } } \ExplSyntaxOff  \NewDocumentCommand{\NonResonantModesShort}{m}{ \stringcase{#1}[\textbf{??}]{ {E2l}{\textit{E}\textsubscript{\textup{2l}}} {B1l}{\textit{B}\textsubscript{\textup{1l}}} {A1TO}{\textit{A}\textsubscript{\textup{1(TO)}}} {E1TO}{\textit{E}\textsubscript{\textup{1(TO)}}} {E2h}{\textit{E}\textsubscript{\textup{2h}}} {E1LO}{\textit{E}\textsubscript{\textup{1(LO)}}} {B1h}{\textit{B}\textsubscript{\textup{1h}}} {A1LO}{\textit{A}\textsubscript{\textup{1(LO)}}} {2E2l}{\textup{2}\textit{E}\textsubscript{\textup{2l}}} {2E2h}{\textup{2}\textit{E}\textsubscript{\textup{2h}}} {2E1LO}{\textup{2}\textit{E}\textsubscript{\textup{1(LO)}}} {2A1LO}{\textup{2}\textit{A}\textsubscript{\textup{1(LO)}}} } }  \NewDocumentCommand{\NonResonantModesLong}{m}{ \stringcase{#1}[\textbf{??}]{ {E2l}{\textit{E}\textsubscript{\textup{2(low)}}} {B1l}{\textit{B}\textsubscript{\textup{1(low)}}} {A1TO}{\textit{A}\textsubscript{\textup{1(TO)}}} {E1TO}{\textit{E}\textsubscript{\textup{1(TO)}}} {E2h}{\textit{E}\textsubscript{\textup{2(high)}}} {E1LO}{\textit{E}\textsubscript{\textup{1(LO)}}} {B1h}{\textit{B}\textsubscript{\textup{1(high)}}} {A1LO}{\textit{A}\textsubscript{\textup{1(LO)}}} {2E2l}{\textup{2}\textit{E}\textsubscript{\textup{2(low)}}} {2E2h}{\textup{2}\textit{E}\textsubscript{\textup{2(high)}}} {2E1LO}{\textup{2}\textit{E}\textsubscript{\textup{1(LO)}}} {2A1LO}{\textup{2}\textit{A}\textsubscript{\textup{1(LO)}}} } }  \NewDocumentCommand{\NonResonantModesLogic}{m O{short}}{ \IfNoValueTF{#2}{ \NonResonantModesShort{#1} }{ \ifthenelse{\equal{#2}{long}}{ \NonResonantModesLong{#1} }{ \ifthenelse{\equal{#2}{short}}{ \NonResonantModesShort{#1} }{ \NonResonantModesShort{} } } } }  \NewDocumentCommand{\nonresmode}{m o O{short} O{sum}}{ \IfNoValueTF{#2}{ \IfNoValueTF{#3}{ \NonResonantModesLogic{#1} }{ \ifthenelse{\isempty{#3}}{ \NonResonantModesLogic{#1} }{ \NonResonantModesLogic{#1}[#3] } } }{ \ifthenelse{\isempty{#2}}{ \IfNoValueTF{#3}{ \NonResonantModesLogic{#1} }{ \ifthenelse{\isempty{#3}}{ \NonResonantModesLogic{#1} }{ \NonResonantModesLogic{#1}[#3] } } }{ \ifthenelse{\equal{#4}{diff}}{ \ifthenelse{\isempty{#3}}{ \NonResonantModesLogic{#1}\ensuremath{\mkern0mu-\mkern0mu}\NonResonantModesLogic{#2} }{ \ifthenelse{\equal{#3}{long}}{ \NonResonantModesLogic{#1}[#3]\ensuremath{\mkern1.5mu-\mkern1.5mu}\NonResonantModesLogic{#2}[#3] }{ \NonResonantModesLogic{#1}\ensuremath{\mkern1.5mu-\mkern1.5mu}\NonResonantModesLogic{#2} } } }{ \ifthenelse{\isempty{#3}}{ \NonResonantModesLogic{#1}\ensuremath{\mkern0mu+\mkern0mu}\NonResonantModesLogic{#2} }{ \ifthenelse{\equal{#3}{long}}{ \NonResonantModesLogic{#1}[#3]\ensuremath{\mkern1.5mu+\mkern1.5mu}\NonResonantModesLogic{#2}[#3] }{ \NonResonantModesLogic{#1}\ensuremath{\mkern1.5mu+\mkern1.5mu}\NonResonantModesLogic{#2} } } } } } }   \NewDocumentCommand{\Eg}{mo}{ \ifthenelse{\isempty{#1}}{ \ensuremath{E_\text{g}}}{ \IfNoValueTF{#2}{ \ensuremath{E_\text{g}{\,=\,}\qty{#1}{\eV}}}{ \ifthenelse{\isempty{#2}}{ \ensuremath{E_\text{g}{\,=\,}\qty{#1}{\eV}}}{ \ensuremath{E_\text{g}{\,=\,}\qty{#1}{#2}}}}}}   \NewDocumentCommand{\Rsq}{om}{ \IfNoValueTF{#1}{ \ensuremath{R^2{\,=\,}\num{#2}}}{ \ensuremath{R_\text{#1}^2{\,=\,}\num{#2}}}}   \NewDocumentCommand{\density}{mo}{ \ifthenelse{\isempty{#1}}{ \ensuremath{\rho}}{ \IfNoValueTF{#2}{ \ensuremath{\rho{\,=\,}\qty{#1}{\gram\per\cubic\cm}}}{ \ifthenelse{\isempty{#2}}{ \ensuremath{\rho{\,=\,}\qty{#1}{\gram\per\cubic\cm}}}{ \ensuremath{\rho{\,=\,}\qty{#1}{#2}}}}}}   \NewDocumentCommand{\miller}{m}{ \num{#1} }
\newcommand{\SweaveOpts}[1]{}  % do not interfere with LaTeX
\newcommand{\SweaveInput}[1]{} % because they are not real TeX commands
\newcommand{\Sexpr}[1]{}       % will only be parsed by R


% printedition: configure thesis for physical printing (webedition by default)
% openright: force chapters to start on right-hand page (important for print edition)
% titles: show current chapter and section in top outer corner of each page (like a proper book)
% note that cleveref language switching does not work when languages are specified
% as babel package options, they need to be here. The main language should be loaded last.
% https://tex.stackexchange.com/a/20991/10824
% https://tex.stackexchange.com/questions/248341/how-to-make-a-custom-crefformat-dependent-on-language-chosen-with-babel/252847

%\input{assets/preamble/preamble.Rnw}


% For "LaTeX Workshop" (VSCodium extension) to do its autocomplete magic,
% our bib-files must be included here, in the main Rnw
% That also means \usepackage{biblatex} must be included in tikzLualatexPackages
\addbibresource{assets/references/library.bib}


\begin{document}

\begin{tikzpicture}[x=1pt,y=1pt]
\definecolor{fillColor}{RGB}{255,255,255}
\path[use as bounding box,fill=fillColor,fill opacity=0.00] (0,0) rectangle (274.63,105.51);
\begin{scope}
\path[clip] (  0.00,  0.00) rectangle (274.63,105.51);
\definecolor{drawColor}{RGB}{255,255,255}
\definecolor{fillColor}{RGB}{255,255,255}

\path[draw=drawColor,line width= 0.5pt,line join=round,line cap=round,fill=fillColor] (  0.00,  0.00) rectangle (274.63,105.51);
\end{scope}
\begin{scope}
\path[clip] ( 16.95, 65.71) rectangle (141.29,101.01);
\definecolor{fillColor}{RGB}{255,255,255}

\path[fill=fillColor] ( 16.95, 65.71) rectangle (141.29,101.01);
\definecolor{fillColor}{RGB}{248,118,109}

\path[fill=fillColor] ( 24.27, 65.84) --
	( 24.27, 65.84) --
	( 24.27, 65.84) --
	( 24.27, 65.84) --
	( 24.27, 65.84) --
	( 24.34, 65.86) --
	( 24.34, 65.86) --
	( 24.34, 65.86) --
	( 24.42, 65.89) --
	( 24.42, 65.89) --
	( 24.42, 65.89) --
	( 24.42, 65.89) --
	( 24.42, 65.89) --
	( 24.42, 65.89) --
	( 24.50, 65.95) --
	( 24.50, 65.95) --
	( 24.50, 65.95) --
	( 24.51, 65.96) --
	( 24.51, 65.96) --
	( 24.51, 65.96) --
	( 24.55, 66.06) --
	( 24.55, 66.06) --
	( 24.55, 66.06) --
	( 24.57, 66.09) --
	( 24.57, 66.09) --
	( 24.57, 66.09) --
	( 24.63, 66.19) --
	( 24.63, 66.19) --
	( 24.63, 66.19) --
	( 24.65, 66.17) --
	( 24.65, 66.17) --
	( 24.65, 66.17) --
	( 24.72, 66.11) --
	( 24.72, 66.11) --
	( 24.72, 66.11) --
	( 24.75, 66.09) --
	( 24.75, 66.09) --
	( 24.75, 66.09) --
	( 24.80, 66.05) --
	( 24.80, 66.05) --
	( 24.80, 66.05) --
	( 24.83, 65.99) --
	( 24.83, 65.99) --
	( 24.83, 65.99) --
	( 24.87, 65.96) --
	( 24.87, 65.96) --
	( 24.87, 65.96) --
	( 24.95, 65.94) --
	( 24.95, 65.94) --
	( 24.95, 65.94) --
	( 25.02, 65.92) --
	( 25.02, 65.92) --
	( 25.02, 65.92) --
	( 25.10, 65.92) --
	( 25.10, 65.92) --
	( 25.10, 65.92) --
	( 25.12, 65.92) --
	( 25.12, 65.92) --
	( 25.12, 65.92) --
	( 25.17, 65.92) --
	( 25.17, 65.92) --
	( 25.17, 65.92) --
	( 25.25, 65.89) --
	( 25.25, 65.89) --
	( 25.25, 65.89) --
	( 25.32, 65.90) --
	( 25.32, 65.90) --
	( 25.32, 65.90) --
	( 25.40, 65.88) --
	( 25.40, 65.88) --
	( 25.40, 65.88) --
	( 25.47, 65.90) --
	( 25.47, 65.90) --
	( 25.47, 65.90) --
	( 25.55, 65.89) --
	( 25.55, 65.89) --
	( 25.55, 65.89) --
	( 25.62, 65.89) --
	( 25.62, 65.89) --
	( 25.62, 65.89) --
	( 25.70, 65.86) --
	( 25.70, 65.86) --
	( 25.70, 65.86) --
	( 25.77, 65.89) --
	( 25.77, 65.89) --
	( 25.77, 65.89) --
	( 25.80, 65.88) --
	( 25.80, 65.88) --
	( 25.80, 65.88) --
	( 25.85, 65.86) --
	( 25.85, 65.86) --
	( 25.85, 65.86) --
	( 25.92, 65.86) --
	( 25.92, 65.86) --
	( 25.92, 65.86) --
	( 26.00, 65.90) --
	( 26.00, 65.90) --
	( 26.00, 65.90) --
	( 26.07, 65.88) --
	( 26.07, 65.88) --
	( 26.07, 65.88) --
	( 26.15, 65.88) --
	( 26.15, 65.88) --
	( 26.15, 65.88) --
	( 26.22, 65.89) --
	( 26.22, 65.89) --
	( 26.22, 65.89) --
	( 26.30, 65.88) --
	( 26.30, 65.88) --
	( 26.30, 65.88) --
	( 26.38, 65.90) --
	( 26.38, 65.90) --
	( 26.38, 65.90) --
	( 26.45, 65.92) --
	( 26.45, 65.92) --
	( 26.45, 65.92) --
	( 26.53, 65.90) --
	( 26.53, 65.90) --
	( 26.53, 65.90) --
	( 26.60, 65.93) --
	( 26.60, 65.93) --
	( 26.60, 65.93) --
	( 26.68, 65.91) --
	( 26.68, 65.91) --
	( 26.68, 65.91) --
	( 26.75, 65.93) --
	( 26.75, 65.93) --
	( 26.75, 65.93) --
	( 26.83, 65.92) --
	( 26.83, 65.92) --
	( 26.83, 65.92) --
	( 26.90, 65.89) --
	( 26.90, 65.89) --
	( 26.90, 65.89) --
	( 26.98, 65.92) --
	( 26.98, 65.92) --
	( 26.98, 65.92) --
	( 27.05, 65.94) --
	( 27.05, 65.94) --
	( 27.05, 65.94) --
	( 27.06, 65.94) --
	( 27.06, 65.94) --
	( 27.06, 65.94) --
	( 27.13, 65.92) --
	( 27.13, 65.92) --
	( 27.13, 65.92) --
	( 27.20, 65.94) --
	( 27.20, 65.94) --
	( 27.20, 65.94) --
	( 27.28, 65.92) --
	( 27.28, 65.92) --
	( 27.28, 65.92) --
	( 27.35, 65.89) --
	( 27.35, 65.89) --
	( 27.35, 65.89) --
	( 27.39, 65.90) --
	( 27.39, 65.90) --
	( 27.39, 65.90) --
	( 27.43, 65.92) --
	( 27.43, 65.92) --
	( 27.43, 65.92) --
	( 27.50, 65.91) --
	( 27.50, 65.91) --
	( 27.50, 65.91) --
	( 27.58, 65.91) --
	( 27.58, 65.91) --
	( 27.58, 65.91) --
	( 27.65, 65.91) --
	( 27.65, 65.91) --
	( 27.65, 65.91) --
	( 27.66, 65.91) --
	( 27.66, 65.91) --
	( 27.66, 65.91) --
	( 27.73, 65.89) --
	( 27.73, 65.89) --
	( 27.73, 65.89) --
	( 27.80, 65.94) --
	( 27.80, 65.94) --
	( 27.80, 65.94) --
	( 27.88, 65.97) --
	( 27.88, 65.97) --
	( 27.88, 65.97) --
	( 27.95, 65.95) --
	( 27.95, 65.95) --
	( 27.95, 65.95) --
	( 28.03, 65.97) --
	( 28.03, 65.97) --
	( 28.03, 65.97) --
	( 28.03, 65.97) --
	( 28.03, 65.97) --
	( 28.03, 65.97) --
	( 28.10, 65.98) --
	( 28.10, 65.98) --
	( 28.10, 65.98) --
	( 28.18, 65.99) --
	( 28.18, 65.99) --
	( 28.18, 65.99) --
	( 28.19, 65.99) --
	( 28.19, 65.99) --
	( 28.19, 65.99) --
	( 28.25, 65.98) --
	( 28.25, 65.98) --
	( 28.25, 65.98) --
	( 28.33, 65.96) --
	( 28.33, 65.96) --
	( 28.33, 65.96) --
	( 28.39, 65.93) --
	( 28.39, 65.93) --
	( 28.39, 65.93) --
	( 28.40, 65.93) --
	( 28.40, 65.93) --
	( 28.40, 65.93) --
	( 28.48, 65.93) --
	( 28.48, 65.93) --
	( 28.48, 65.93) --
	( 28.56, 65.92) --
	( 28.56, 65.92) --
	( 28.56, 65.92) --
	( 28.63, 65.92) --
	( 28.63, 65.92) --
	( 28.63, 65.92) --
	( 28.71, 65.94) --
	( 28.71, 65.94) --
	( 28.71, 65.94) --
	( 28.78, 65.93) --
	( 28.78, 65.93) --
	( 28.78, 65.93) --
	( 28.86, 65.92) --
	( 28.86, 65.92) --
	( 28.86, 65.92) --
	( 28.93, 65.93) --
	( 28.93, 65.93) --
	( 28.93, 65.93) --
	( 29.01, 65.93) --
	( 29.01, 65.93) --
	( 29.01, 65.93) --
	( 29.08, 65.94) --
	( 29.08, 65.94) --
	( 29.08, 65.94) --
	( 29.16, 65.93) --
	( 29.16, 65.93) --
	( 29.16, 65.93) --
	( 29.20, 65.93) --
	( 29.20, 65.93) --
	( 29.20, 65.93) --
	( 29.23, 65.93) --
	( 29.23, 65.93) --
	( 29.23, 65.93) --
	( 29.31, 65.94) --
	( 29.31, 65.94) --
	( 29.31, 65.94) --
	( 29.38, 65.93) --
	( 29.38, 65.93) --
	( 29.38, 65.93) --
	( 29.40, 65.94) --
	( 29.40, 65.94) --
	( 29.40, 65.94) --
	( 29.46, 65.95) --
	( 29.46, 65.95) --
	( 29.46, 65.95) --
	( 29.53, 65.96) --
	( 29.53, 65.96) --
	( 29.53, 65.96) --
	( 29.61, 65.93) --
	( 29.61, 65.93) --
	( 29.61, 65.93) --
	( 29.68, 65.93) --
	( 29.68, 65.93) --
	( 29.68, 65.93) --
	( 29.76, 65.95) --
	( 29.76, 65.95) --
	( 29.76, 65.95) --
	( 29.83, 65.94) --
	( 29.83, 65.94) --
	( 29.83, 65.94) --
	( 29.91, 65.96) --
	( 29.91, 65.96) --
	( 29.91, 65.96) --
	( 29.93, 65.96) --
	( 29.93, 65.96) --
	( 29.93, 65.96) --
	( 29.98, 65.96) --
	( 29.98, 65.96) --
	( 29.98, 65.96) --
	( 30.06, 66.00) --
	( 30.06, 66.00) --
	( 30.06, 66.00) --
	( 30.13, 66.02) --
	( 30.13, 66.02) --
	( 30.13, 66.02) --
	( 30.21, 66.01) --
	( 30.21, 66.01) --
	( 30.21, 66.01) --
	( 30.28, 66.07) --
	( 30.28, 66.07) --
	( 30.28, 66.07) --
	( 30.36, 66.09) --
	( 30.36, 66.09) --
	( 30.36, 66.09) --
	( 30.41, 66.09) --
	( 30.41, 66.09) --
	( 30.41, 66.09) --
	( 30.43, 66.10) --
	( 30.43, 66.10) --
	( 30.43, 66.10) --
	( 30.51, 66.16) --
	( 30.51, 66.16) --
	( 30.51, 66.16) --
	( 30.58, 66.22) --
	( 30.58, 66.22) --
	( 30.58, 66.22) --
	( 30.66, 66.28) --
	( 30.66, 66.28) --
	( 30.66, 66.28) --
	( 30.73, 66.31) --
	( 30.73, 66.31) --
	( 30.73, 66.31) --
	( 30.73, 66.31) --
	( 30.73, 66.31) --
	( 30.73, 66.31) --
	( 30.81, 66.40) --
	( 30.81, 66.40) --
	( 30.81, 66.40) --
	( 30.88, 66.50) --
	( 30.88, 66.50) --
	( 30.88, 66.50) --
	( 30.94, 66.57) --
	( 30.94, 66.57) --
	( 30.94, 66.57) --
	( 30.96, 66.60) --
	( 30.96, 66.60) --
	( 30.96, 66.60) --
	( 31.03, 66.71) --
	( 31.03, 66.71) --
	( 31.03, 66.71) --
	( 31.10, 66.85) --
	( 31.10, 66.85) --
	( 31.10, 66.85) --
	( 31.14, 66.90) --
	( 31.14, 66.90) --
	( 31.14, 66.90) --
	( 31.18, 66.98) --
	( 31.18, 66.98) --
	( 31.18, 66.98) --
	( 31.26, 67.10) --
	( 31.26, 67.10) --
	( 31.26, 67.10) --
	( 31.30, 67.19) --
	( 31.30, 67.19) --
	( 31.30, 67.19) --
	( 31.33, 67.36) --
	( 31.33, 67.36) --
	( 31.33, 67.36) --
	( 31.41, 67.80) --
	( 31.41, 67.80) --
	( 31.41, 67.80) --
	( 31.42, 67.89) --
	( 31.42, 67.89) --
	( 31.42, 67.89) --
	( 31.46, 67.59) --
	( 31.46, 67.59) --
	( 31.46, 67.59) --
	( 31.48, 67.69) --
	( 31.48, 67.69) --
	( 31.48, 67.69) --
	( 31.56, 68.07) --
	( 31.56, 68.07) --
	( 31.56, 68.07) --
	( 31.58, 68.21) --
	( 31.58, 68.21) --
	( 31.58, 68.21) --
	( 31.63, 68.36) --
	( 31.63, 68.36) --
	( 31.63, 68.36) --
	( 31.70, 68.66) --
	( 31.70, 68.66) --
	( 31.70, 68.66) --
	( 31.71, 68.67) --
	( 31.71, 68.67) --
	( 31.71, 68.67) --
	( 31.74, 69.08) --
	( 31.74, 69.49) --
	( 31.74, 69.49) --
	( 31.78, 69.73) --
	( 31.78, 69.73) --
	( 31.78, 69.74) --
	( 31.85, 70.29) --
	( 31.85, 70.29) --
	( 31.85, 70.29) --
	( 31.87, 70.38) --
	( 31.87, 70.38) --
	( 31.87, 70.38) --
	( 31.93, 70.09) --
	( 31.93, 70.09) --
	( 31.93, 70.09) --
	( 31.95, 70.00) --
	( 31.95, 70.00) --
	( 31.95, 70.00) --
	( 32.00, 70.31) --
	( 32.00, 70.31) --
	( 32.00, 70.31) --
	( 32.08, 70.74) --
	( 32.08, 70.74) --
	( 32.08, 70.74) --
	( 32.11, 70.90) --
	( 32.11, 71.48) --
	( 32.11, 71.48) --
	( 32.15, 71.68) --
	( 32.15, 71.68) --
	( 32.15, 71.68) --
	( 32.23, 72.01) --
	( 32.23, 72.01) --
	( 32.23, 72.01) --
	( 32.23, 72.01) --
	( 32.23, 72.01) --
	( 32.23, 72.01) --
	( 32.27, 72.60) --
	( 32.27, 73.15) --
	( 32.27, 73.15) --
	( 32.30, 73.59) --
	( 32.30, 73.59) --
	( 32.30, 73.59) --
	( 32.38, 74.53) --
	( 32.38, 74.53) --
	( 32.38, 74.53) --
	( 32.39, 74.69) --
	( 32.39, 74.69) --
	( 32.39, 74.69) --
	( 32.43, 73.98) --
	( 32.43, 73.98) --
	( 32.43, 73.98) --
	( 32.45, 73.72) --
	( 32.45, 73.72) --
	( 32.45, 73.72) --
	( 32.47, 73.55) --
	( 32.47, 73.55) --
	( 32.47, 73.54) --
	( 32.47, 73.52) --
	( 32.47, 73.52) --
	( 32.47, 73.53) --
	( 32.53, 75.22) --
	( 32.53, 75.22) --
	( 32.53, 75.22) --
	( 32.55, 75.94) --
	( 32.55, 76.47) --
	( 32.55, 76.47) --
	( 32.59, 75.30) --
	( 32.59, 75.30) --
	( 32.59, 75.30) --
	( 32.60, 75.65) --
	( 32.60, 75.65) --
	( 32.60, 75.66) --
	( 32.67, 78.23) --
	( 32.67, 78.23) --
	( 32.67, 78.23) --
	( 32.68, 78.09) --
	( 32.68, 78.09) --
	( 32.68, 78.09) --
	( 32.71, 77.03) --
	( 32.71, 77.75) --
	( 32.71, 77.75) --
	( 32.75, 77.95) --
	( 32.75, 77.95) --
	( 32.75, 77.95) --
	( 32.83, 78.32) --
	( 32.83, 78.32) --
	( 32.83, 78.32) --
	( 32.90, 78.66) --
	( 32.90, 78.66) --
	( 32.90, 78.66) --
	( 32.96, 78.92) --
	( 32.96, 79.53) --
	( 32.96, 79.53) --
	( 32.98, 80.61) --
	( 32.98, 80.61) --
	( 32.98, 80.61) --
	( 33.00, 81.66) --
	( 33.00, 82.27) --
	( 33.00, 82.27) --
	( 33.05, 82.66) --
	( 33.05, 82.66) --
	( 33.05, 82.66) --
	( 33.08, 82.85) --
	( 33.08, 82.85) --
	( 33.08, 82.85) --
	( 33.12, 80.07) --
	( 33.12, 83.27) --
	( 33.12, 83.27) --
	( 33.13, 82.75) --
	( 33.13, 82.75) --
	( 33.13, 82.75) --
	( 33.16, 80.95) --
	( 33.16, 80.95) --
	( 33.16, 80.95) --
	( 33.20, 83.42) --
	( 33.20, 82.77) --
	( 33.20, 82.77) --
	( 33.20, 82.78) --
	( 33.20, 82.78) --
	( 33.20, 82.78) --
	( 33.28, 83.06) --
	( 33.28, 83.06) --
	( 33.28, 83.06) --
	( 33.28, 83.08) --
	( 33.28, 82.40) --
	( 33.28, 82.40) --
	( 33.35, 82.30) --
	( 33.35, 82.30) --
	( 33.35, 82.30) --
	( 33.43, 82.21) --
	( 33.43, 82.21) --
	( 33.43, 82.21) --
	( 33.44, 82.18) --
	( 33.44, 82.18) --
	( 33.44, 82.18) --
	( 33.50, 82.07) --
	( 33.50, 82.07) --
	( 33.50, 82.07) --
	( 33.58, 81.99) --
	( 33.58, 81.99) --
	( 33.58, 81.99) --
	( 33.64, 81.92) --
	( 33.64, 81.92) --
	( 33.64, 81.92) --
	( 33.65, 81.92) --
	( 33.65, 81.92) --
	( 33.65, 81.92) --
	( 33.73, 81.80) --
	( 33.73, 81.80) --
	( 33.73, 81.80) --
	( 33.80, 81.70) --
	( 33.80, 81.70) --
	( 33.80, 81.70) --
	( 33.85, 81.66) --
	( 33.85, 81.12) --
	( 33.85, 81.12) --
	( 33.88, 81.28) --
	( 33.88, 81.28) --
	( 33.88, 81.28) --
	( 33.89, 81.35) --
	( 33.89, 81.35) --
	( 33.89, 81.35) --
	( 33.95, 81.17) --
	( 33.95, 81.17) --
	( 33.95, 81.17) --
	( 34.03, 80.95) --
	( 34.03, 80.95) --
	( 34.03, 80.95) --
	( 34.05, 80.89) --
	( 34.05, 80.89) --
	( 34.05, 80.89) --
	( 34.09, 80.48) --
	( 34.09, 80.48) --
	( 34.09, 80.48) --
	( 34.10, 80.46) --
	( 34.10, 80.46) --
	( 34.10, 80.46) --
	( 34.17, 80.33) --
	( 34.17, 80.33) --
	( 34.17, 80.33) --
	( 34.21, 80.24) --
	( 34.21, 80.24) --
	( 34.21, 80.24) --
	( 34.25, 79.81) --
	( 34.25, 79.81) --
	( 34.25, 79.81) --
	( 34.25, 79.81) --
	( 34.25, 79.52) --
	( 34.25, 79.52) --
	( 34.32, 79.34) --
	( 34.32, 79.34) --
	( 34.32, 79.34) --
	( 34.40, 79.17) --
	( 34.40, 79.17) --
	( 34.40, 79.17) --
	( 34.41, 79.14) --
	( 34.41, 78.76) --
	( 34.41, 78.76) --
	( 34.47, 78.61) --
	( 34.47, 78.61) --
	( 34.47, 78.61) --
	( 34.53, 78.41) --
	( 34.53, 78.41) --
	( 34.53, 78.41) --
	( 34.55, 78.31) --
	( 34.55, 78.31) --
	( 34.55, 78.31) --
	( 34.61, 77.93) --
	( 34.61, 77.42) --
	( 34.61, 77.42) --
	( 34.62, 77.38) --
	( 34.62, 77.38) --
	( 34.62, 77.38) --
	( 34.70, 77.00) --
	( 34.70, 77.00) --
	( 34.70, 77.00) --
	( 34.70, 76.98) --
	( 34.70, 76.98) --
	( 34.70, 76.98) --
	( 34.74, 76.57) --
	( 34.74, 76.57) --
	( 34.74, 76.57) --
	( 34.77, 76.43) --
	( 34.77, 76.43) --
	( 34.77, 76.43) --
	( 34.82, 76.28) --
	( 34.82, 76.28) --
	( 34.82, 76.28) --
	( 34.85, 75.89) --
	( 34.85, 75.89) --
	( 34.85, 75.89) --
	( 34.86, 75.77) --
	( 34.86, 75.77) --
	( 34.86, 75.77) --
	( 34.87, 75.73) --
	( 34.87, 75.73) --
	( 34.87, 75.73) --
	( 34.92, 75.50) --
	( 34.92, 75.50) --
	( 34.92, 75.50) --
	( 34.98, 75.32) --
	( 34.98, 74.38) --
	( 34.98, 74.38) --
	( 35.00, 74.51) --
	( 35.00, 74.51) --
	( 35.00, 74.51) --
	( 35.06, 74.89) --
	( 35.06, 74.89) --
	( 35.06, 74.89) --
	( 35.07, 74.78) --
	( 35.07, 74.78) --
	( 35.07, 74.78) --
	( 35.15, 74.10) --
	( 35.15, 74.10) --
	( 35.15, 74.10) --
	( 35.18, 73.81) --
	( 35.18, 73.81) --
	( 35.18, 73.81) --
	( 35.22, 73.68) --
	( 35.22, 73.68) --
	( 35.22, 73.68) --
	( 35.30, 73.45) --
	( 35.30, 73.45) --
	( 35.30, 73.45) --
	( 35.34, 73.30) --
	( 35.34, 73.30) --
	( 35.34, 73.30) --
	( 35.37, 72.94) --
	( 35.37, 72.94) --
	( 35.37, 72.94) --
	( 35.38, 72.79) --
	( 35.38, 72.79) --
	( 35.38, 72.79) --
	( 35.45, 72.54) --
	( 35.45, 72.54) --
	( 35.45, 72.54) --
	( 35.50, 72.28) --
	( 35.50, 72.28) --
	( 35.50, 72.28) --
	( 35.52, 72.19) --
	( 35.52, 72.19) --
	( 35.52, 72.19) --
	( 35.58, 71.87) --
	( 35.58, 71.87) --
	( 35.58, 71.87) --
	( 35.60, 71.84) --
	( 35.60, 71.84) --
	( 35.60, 71.84) --
	( 35.67, 71.59) --
	( 35.67, 71.59) --
	( 35.67, 71.59) --
	( 35.71, 71.45) --
	( 35.71, 71.45) --
	( 35.71, 71.45) --
	( 35.74, 71.29) --
	( 35.74, 71.29) --
	( 35.74, 71.29) --
	( 35.82, 71.00) --
	( 35.82, 71.00) --
	( 35.82, 71.00) --
	( 35.82, 70.98) --
	( 35.82, 70.98) --
	( 35.82, 70.98) --
	( 35.83, 70.97) --
	( 35.83, 70.97) --
	( 35.83, 70.97) --
	( 35.87, 70.50) --
	( 35.87, 70.50) --
	( 35.87, 70.50) --
	( 35.89, 70.45) --
	( 35.89, 70.45) --
	( 35.89, 70.45) --
	( 35.97, 70.32) --
	( 35.97, 70.32) --
	( 35.97, 70.32) --
	( 36.04, 70.20) --
	( 36.04, 70.20) --
	( 36.04, 70.20) --
	( 36.11, 70.10) --
	( 36.11, 70.10) --
	( 36.11, 70.10) --
	( 36.12, 70.01) --
	( 36.12, 70.01) --
	( 36.12, 70.01) --
	( 36.15, 69.64) --
	( 36.15, 69.64) --
	( 36.15, 69.64) --
	( 36.19, 69.40) --
	( 36.19, 69.40) --
	( 36.19, 69.40) --
	( 36.23, 69.19) --
	( 36.23, 69.19) --
	( 36.23, 69.19) --
	( 36.27, 69.13) --
	( 36.27, 69.13) --
	( 36.27, 69.13) --
	( 36.34, 69.00) --
	( 36.34, 69.00) --
	( 36.34, 69.00) --
	( 36.42, 68.86) --
	( 36.42, 68.86) --
	( 36.42, 68.86) --
	( 36.43, 68.83) --
	( 36.43, 68.45) --
	( 36.43, 68.45) --
	( 36.49, 68.38) --
	( 36.49, 68.38) --
	( 36.49, 68.38) --
	( 36.57, 68.27) --
	( 36.57, 68.27) --
	( 36.57, 68.27) --
	( 36.64, 68.21) --
	( 36.64, 68.21) --
	( 36.64, 68.21) --
	( 36.71, 68.10) --
	( 36.71, 68.10) --
	( 36.71, 68.10) --
	( 36.76, 68.03) --
	( 36.76, 68.03) --
	( 36.76, 68.03) --
	( 36.79, 67.97) --
	( 36.79, 67.97) --
	( 36.79, 67.97) --
	( 36.86, 67.85) --
	( 36.86, 67.85) --
	( 36.86, 67.85) --
	( 36.92, 67.74) --
	( 36.92, 67.74) --
	( 36.92, 67.74) --
	( 36.94, 67.70) --
	( 36.94, 67.70) --
	( 36.94, 67.70) --
	( 37.01, 67.59) --
	( 37.01, 67.59) --
	( 37.01, 67.59) --
	( 37.09, 67.48) --
	( 37.09, 67.48) --
	( 37.09, 67.48) --
	( 37.16, 67.35) --
	( 37.16, 67.35) --
	( 37.16, 67.35) --
	( 37.16, 67.35) --
	( 37.16, 67.35) --
	( 37.16, 67.35) --
	( 37.24, 67.22) --
	( 37.24, 67.22) --
	( 37.24, 67.22) --
	( 37.31, 67.06) --
	( 37.31, 67.06) --
	( 37.31, 67.06) --
	( 37.32, 67.04) --
	( 37.32, 67.04) --
	( 37.32, 67.04) --
	( 37.39, 66.95) --
	( 37.39, 66.95) --
	( 37.39, 66.95) --
	( 37.46, 66.85) --
	( 37.46, 66.85) --
	( 37.46, 66.85) --
	( 37.54, 66.78) --
	( 37.54, 66.78) --
	( 37.54, 66.78) --
	( 37.61, 66.69) --
	( 37.61, 66.69) --
	( 37.61, 66.69) --
	( 37.61, 66.69) --
	( 37.61, 66.69) --
	( 37.61, 66.69) --
	( 37.65, 66.65) --
	( 37.65, 66.65) --
	( 37.65, 66.65) --
	( 37.69, 66.62) --
	( 37.69, 66.62) --
	( 37.69, 66.62) --
	( 37.76, 66.57) --
	( 37.76, 66.57) --
	( 37.76, 66.57) --
	( 37.83, 66.52) --
	( 37.83, 66.52) --
	( 37.83, 66.52) --
	( 37.91, 66.46) --
	( 37.91, 66.46) --
	( 37.91, 66.46) --
	( 37.97, 66.40) --
	( 37.97, 66.40) --
	( 37.97, 66.40) --
	( 37.98, 66.39) --
	( 37.98, 66.39) --
	( 37.98, 66.39) --
	( 38.06, 66.38) --
	( 38.06, 66.38) --
	( 38.06, 66.38) --
	( 38.13, 66.33) --
	( 38.13, 66.33) --
	( 38.13, 66.33) --
	( 38.21, 66.28) --
	( 38.21, 66.28) --
	( 38.21, 66.28) --
	( 38.28, 66.26) --
	( 38.28, 66.26) --
	( 38.28, 66.26) --
	( 38.33, 66.23) --
	( 38.33, 66.23) --
	( 38.33, 66.23) --
	( 38.36, 66.22) --
	( 38.36, 66.22) --
	( 38.36, 66.22) --
	( 38.43, 66.17) --
	( 38.43, 66.17) --
	( 38.43, 66.17) --
	( 38.51, 66.15) --
	( 38.51, 66.15) --
	( 38.51, 66.15) --
	( 38.58, 66.12) --
	( 38.58, 66.12) --
	( 38.58, 66.12) --
	( 38.65, 66.11) --
	( 38.65, 66.11) --
	( 38.65, 66.11) --
	( 38.73, 66.09) --
	( 38.73, 66.09) --
	( 38.73, 66.09) --
	( 38.80, 66.05) --
	( 38.80, 66.05) --
	( 38.80, 66.05) --
	( 38.82, 66.05) --
	( 38.82, 66.05) --
	( 38.82, 66.05) --
	( 38.88, 66.02) --
	( 38.88, 66.02) --
	( 38.88, 66.02) --
	( 38.95, 66.03) --
	( 38.95, 66.03) --
	( 38.95, 66.03) --
	( 38.99, 66.00) --
	( 38.99, 66.00) --
	( 38.99, 66.00) --
	( 39.03, 65.97) --
	( 39.03, 65.97) --
	( 39.03, 65.97) --
	( 39.10, 65.98) --
	( 39.10, 65.98) --
	( 39.10, 65.98) --
	( 39.14, 65.97) --
	( 39.14, 65.97) --
	( 39.14, 65.97) --
	( 39.18, 66.01) --
	( 39.18, 66.01) --
	( 39.18, 66.01) --
	( 39.25, 66.09) --
	( 39.25, 66.09) --
	( 39.25, 66.09) --
	( 39.33, 66.18) --
	( 39.33, 66.18) --
	( 39.33, 66.18) --
	( 39.40, 66.29) --
	( 39.40, 66.29) --
	( 39.40, 66.29) --
	( 39.47, 66.35) --
	( 39.47, 66.35) --
	( 39.47, 66.35) --
	( 39.47, 66.27) --
	( 39.47, 66.27) --
	( 39.47, 66.27) --
	( 39.51, 66.03) --
	( 39.51, 66.03) --
	( 39.51, 66.03) --
	( 39.55, 66.19) --
	( 39.55, 66.19) --
	( 39.55, 66.19) --
	( 39.62, 66.42) --
	( 39.62, 66.42) --
	( 39.62, 66.42) --
	( 39.67, 66.59) --
	( 39.67, 66.59) --
	( 39.67, 66.59) --
	( 39.70, 66.97) --
	( 39.70, 66.97) --
	( 39.70, 66.97) --
	( 39.71, 67.11) --
	( 39.71, 67.11) --
	( 39.71, 67.11) --
	( 39.77, 66.96) --
	( 39.77, 66.96) --
	( 39.77, 66.96) --
	( 39.79, 66.91) --
	( 39.79, 66.91) --
	( 39.79, 66.91) --
	( 39.83, 66.32) --
	( 39.83, 66.32) --
	( 39.83, 66.32) --
	( 39.85, 66.43) --
	( 39.85, 66.43) --
	( 39.85, 66.43) --
	( 39.87, 66.56) --
	( 39.87, 66.56) --
	( 39.87, 66.56) --
	( 39.91, 66.12) --
	( 39.91, 66.12) --
	( 39.91, 66.12) --
	( 39.92, 66.10) --
	( 39.92, 66.10) --
	( 39.92, 66.10) --
	( 40.00, 66.04) --
	( 40.00, 66.04) --
	( 40.00, 66.04) --
	( 40.03, 66.00) --
	( 40.03, 66.00) --
	( 40.03, 66.00) --
	( 40.07, 65.98) --
	( 40.07, 65.98) --
	( 40.07, 65.98) --
	( 40.14, 65.97) --
	( 40.14, 65.97) --
	( 40.14, 65.97) --
	( 40.22, 65.98) --
	( 40.22, 65.98) --
	( 40.22, 65.98) --
	( 40.29, 65.98) --
	( 40.29, 65.98) --
	( 40.29, 65.98) --
	( 40.35, 65.95) --
	( 40.35, 65.95) --
	( 40.35, 65.95) --
	( 40.37, 65.95) --
	( 40.37, 65.95) --
	( 40.37, 65.95) --
	( 40.44, 65.93) --
	( 40.44, 65.93) --
	( 40.44, 65.93) --
	( 40.52, 65.97) --
	( 40.52, 65.97) --
	( 40.52, 65.97) --
	( 40.52, 65.97) --
	( 40.52, 65.97) --
	( 40.52, 65.97) --
	( 40.59, 65.96) --
	( 40.59, 65.96) --
	( 40.59, 65.96) --
	( 40.67, 65.95) --
	( 40.67, 65.95) --
	( 40.67, 65.95) --
	( 40.74, 65.96) --
	( 40.74, 65.96) --
	( 40.74, 65.96) --
	( 40.81, 65.98) --
	( 40.81, 65.98) --
	( 40.81, 65.98) --
	( 40.89, 65.95) --
	( 40.89, 65.95) --
	( 40.89, 65.95) --
	( 40.92, 65.95) --
	( 40.92, 65.95) --
	( 40.92, 65.95) --
	( 40.96, 65.96) --
	( 40.96, 65.96) --
	( 40.96, 65.96) --
	( 41.04, 65.97) --
	( 41.04, 65.97) --
	( 41.04, 65.97) --
	( 41.11, 65.97) --
	( 41.11, 65.97) --
	( 41.11, 65.97) --
	( 41.19, 65.99) --
	( 41.19, 65.99) --
	( 41.19, 65.99) --
	( 41.26, 65.99) --
	( 41.26, 65.99) --
	( 41.26, 65.99) --
	( 41.29, 66.00) --
	( 41.29, 66.00) --
	( 41.29, 66.00) --
	( 41.33, 66.02) --
	( 41.33, 66.02) --
	( 41.33, 66.02) --
	( 41.41, 66.03) --
	( 41.41, 66.03) --
	( 41.41, 66.03) --
	( 41.48, 66.02) --
	( 41.48, 66.02) --
	( 41.48, 66.02) --
	( 41.56, 66.04) --
	( 41.56, 66.04) --
	( 41.56, 66.04) --
	( 41.63, 66.08) --
	( 41.63, 66.08) --
	( 41.63, 66.08) --
	( 41.67, 66.08) --
	( 41.67, 66.08) --
	( 41.67, 66.08) --
	( 41.71, 66.07) --
	( 41.71, 66.07) --
	( 41.71, 66.07) --
	( 41.78, 66.08) --
	( 41.78, 66.08) --
	( 41.78, 66.08) --
	( 41.86, 66.11) --
	( 41.86, 66.11) --
	( 41.86, 66.11) --
	( 41.93, 66.12) --
	( 41.93, 66.12) --
	( 41.93, 66.12) --
	( 42.00, 66.19) --
	( 42.00, 66.19) --
	( 42.00, 66.19) --
	( 42.05, 66.16) --
	( 42.05, 66.16) --
	( 42.05, 66.16) --
	( 42.08, 66.15) --
	( 42.08, 66.15) --
	( 42.08, 66.15) --
	( 42.15, 66.24) --
	( 42.15, 66.24) --
	( 42.15, 66.24) --
	( 42.21, 66.26) --
	( 42.21, 66.26) --
	( 42.21, 66.09) --
	( 42.21, 66.09) --
	( 42.21, 66.09) --
	( 42.15, 66.07) --
	( 42.15, 66.07) --
	( 42.15, 66.07) --
	( 42.08, 65.98) --
	( 42.08, 65.98) --
	( 42.08, 65.98) --
	( 42.05, 66.00) --
	( 42.05, 66.00) --
	( 42.05, 66.00) --
	( 42.00, 66.02) --
	( 42.00, 66.02) --
	( 42.00, 66.02) --
	( 41.93, 65.95) --
	( 41.93, 65.95) --
	( 41.93, 65.95) --
	( 41.86, 65.95) --
	( 41.86, 65.95) --
	( 41.86, 65.95) --
	( 41.78, 65.92) --
	( 41.78, 65.92) --
	( 41.78, 65.92) --
	( 41.71, 65.91) --
	( 41.71, 65.91) --
	( 41.71, 65.91) --
	( 41.67, 65.92) --
	( 41.67, 65.92) --
	( 41.67, 65.92) --
	( 41.63, 65.92) --
	( 41.63, 65.92) --
	( 41.63, 65.92) --
	( 41.56, 65.88) --
	( 41.56, 65.88) --
	( 41.56, 65.88) --
	( 41.48, 65.86) --
	( 41.48, 65.86) --
	( 41.48, 65.86) --
	( 41.41, 65.87) --
	( 41.41, 65.87) --
	( 41.41, 65.87) --
	( 41.33, 65.87) --
	( 41.33, 65.87) --
	( 41.33, 65.87) --
	( 41.29, 65.85) --
	( 41.29, 65.85) --
	( 41.29, 65.85) --
	( 41.26, 65.84) --
	( 41.26, 65.84) --
	( 41.26, 65.84) --
	( 41.19, 65.84) --
	( 41.19, 65.84) --
	( 41.19, 65.84) --
	( 41.11, 65.82) --
	( 41.11, 65.82) --
	( 41.11, 65.82) --
	( 41.04, 65.83) --
	( 41.04, 65.83) --
	( 41.04, 65.83) --
	( 40.96, 65.82) --
	( 40.96, 65.82) --
	( 40.96, 65.82) --
	( 40.92, 65.81) --
	( 40.92, 65.81) --
	( 40.92, 65.81) --
	( 40.89, 65.80) --
	( 40.89, 65.80) --
	( 40.89, 65.80) --
	( 40.81, 65.83) --
	( 40.81, 65.83) --
	( 40.81, 65.83) --
	( 40.74, 65.81) --
	( 40.74, 65.81) --
	( 40.74, 65.81) --
	( 40.67, 65.79) --
	( 40.67, 65.79) --
	( 40.67, 65.79) --
	( 40.59, 65.81) --
	( 40.59, 65.81) --
	( 40.59, 65.81) --
	( 40.52, 65.80) --
	( 40.52, 65.80) --
	( 40.52, 65.80) --
	( 40.52, 65.80) --
	( 40.52, 65.80) --
	( 40.52, 65.80) --
	( 40.44, 65.76) --
	( 40.44, 65.76) --
	( 40.44, 65.76) --
	( 40.37, 65.78) --
	( 40.37, 65.78) --
	( 40.37, 65.78) --
	( 40.35, 65.78) --
	( 40.35, 65.78) --
	( 40.35, 65.78) --
	( 40.29, 65.80) --
	( 40.29, 65.80) --
	( 40.29, 65.80) --
	( 40.22, 65.79) --
	( 40.22, 65.79) --
	( 40.22, 65.79) --
	( 40.14, 65.77) --
	( 40.14, 65.77) --
	( 40.14, 65.77) --
	( 40.07, 65.77) --
	( 40.07, 65.77) --
	( 40.07, 65.77) --
	( 40.03, 65.78) --
	( 40.03, 65.78) --
	( 40.03, 65.78) --
	( 40.00, 65.79) --
	( 40.00, 65.79) --
	( 40.00, 65.79) --
	( 39.92, 65.78) --
	( 39.92, 65.78) --
	( 39.92, 65.78) --
	( 39.91, 65.78) --
	( 39.91, 65.78) --
	( 39.91, 65.78) --
	( 39.87, 65.78) --
	( 39.87, 65.78) --
	( 39.87, 65.78) --
	( 39.85, 65.79) --
	( 39.85, 65.79) --
	( 39.85, 65.79) --
	( 39.83, 65.79) --
	( 39.83, 65.79) --
	( 39.83, 65.79) --
	( 39.79, 65.80) --
	( 39.79, 65.80) --
	( 39.79, 65.80) --
	( 39.77, 65.80) --
	( 39.77, 65.80) --
	( 39.77, 65.80) --
	( 39.71, 65.78) --
	( 39.71, 65.78) --
	( 39.71, 65.78) --
	( 39.70, 65.77) --
	( 39.70, 65.77) --
	( 39.70, 65.77) --
	( 39.67, 65.76) --
	( 39.67, 65.76) --
	( 39.67, 65.76) --
	( 39.62, 65.75) --
	( 39.62, 65.75) --
	( 39.62, 65.75) --
	( 39.55, 65.77) --
	( 39.55, 65.77) --
	( 39.55, 65.77) --
	( 39.51, 65.77) --
	( 39.51, 65.77) --
	( 39.51, 65.77) --
	( 39.47, 65.76) --
	( 39.47, 65.76) --
	( 39.47, 65.76) --
	( 39.47, 65.77) --
	( 39.47, 65.77) --
	( 39.47, 65.77) --
	( 39.40, 65.79) --
	( 39.40, 65.79) --
	( 39.40, 65.79) --
	( 39.33, 65.77) --
	( 39.33, 65.77) --
	( 39.33, 65.77) --
	( 39.25, 65.77) --
	( 39.25, 65.77) --
	( 39.25, 65.77) --
	( 39.18, 65.78) --
	( 39.18, 65.78) --
	( 39.18, 65.78) --
	( 39.14, 65.78) --
	( 39.14, 65.78) --
	( 39.14, 65.78) --
	( 39.10, 65.78) --
	( 39.10, 65.78) --
	( 39.10, 65.78) --
	( 39.03, 65.74) --
	( 39.03, 65.74) --
	( 39.03, 65.74) --
	( 38.99, 65.76) --
	( 38.99, 65.76) --
	( 38.99, 65.76) --
	( 38.95, 65.78) --
	( 38.95, 65.78) --
	( 38.95, 65.78) --
	( 38.88, 65.75) --
	( 38.88, 65.75) --
	( 38.88, 65.75) --
	( 38.82, 65.76) --
	( 38.82, 65.76) --
	( 38.82, 65.76) --
	( 38.80, 65.76) --
	( 38.80, 65.76) --
	( 38.80, 65.76) --
	( 38.73, 65.77) --
	( 38.73, 65.77) --
	( 38.73, 65.77) --
	( 38.65, 65.76) --
	( 38.65, 65.76) --
	( 38.65, 65.76) --
	( 38.58, 65.75) --
	( 38.58, 65.75) --
	( 38.58, 65.75) --
	( 38.51, 65.75) --
	( 38.51, 65.75) --
	( 38.51, 65.75) --
	( 38.43, 65.74) --
	( 38.43, 65.74) --
	( 38.43, 65.74) --
	( 38.36, 65.77) --
	( 38.36, 65.77) --
	( 38.36, 65.77) --
	( 38.33, 65.77) --
	( 38.33, 65.77) --
	( 38.33, 65.77) --
	( 38.28, 65.77) --
	( 38.28, 65.77) --
	( 38.28, 65.77) --
	( 38.21, 65.76) --
	( 38.21, 65.76) --
	( 38.21, 65.76) --
	( 38.13, 65.78) --
	( 38.13, 65.78) --
	( 38.13, 65.78) --
	( 38.06, 65.79) --
	( 38.06, 65.79) --
	( 38.06, 65.79) --
	( 37.98, 65.77) --
	( 37.98, 65.77) --
	( 37.98, 65.77) --
	( 37.97, 65.77) --
	( 37.97, 65.77) --
	( 37.97, 65.77) --
	( 37.91, 65.78) --
	( 37.91, 65.78) --
	( 37.91, 65.78) --
	( 37.83, 65.78) --
	( 37.83, 65.78) --
	( 37.83, 65.78) --
	( 37.76, 65.78) --
	( 37.76, 65.78) --
	( 37.76, 65.78) --
	( 37.69, 65.76) --
	( 37.69, 65.76) --
	( 37.69, 65.76) --
	( 37.65, 65.76) --
	( 37.65, 65.76) --
	( 37.65, 65.76) --
	( 37.61, 65.77) --
	( 37.61, 65.77) --
	( 37.61, 65.77) --
	( 37.61, 65.77) --
	( 37.61, 65.77) --
	( 37.61, 65.77) --
	( 37.54, 65.78) --
	( 37.54, 65.78) --
	( 37.54, 65.78) --
	( 37.46, 65.76) --
	( 37.46, 65.76) --
	( 37.46, 65.76) --
	( 37.39, 65.77) --
	( 37.39, 65.77) --
	( 37.39, 65.77) --
	( 37.32, 65.78) --
	( 37.32, 65.78) --
	( 37.32, 65.78) --
	( 37.31, 65.78) --
	( 37.31, 65.78) --
	( 37.31, 65.78) --
	( 37.24, 65.78) --
	( 37.24, 65.78) --
	( 37.24, 65.78) --
	( 37.16, 65.75) --
	( 37.16, 65.75) --
	( 37.16, 65.75) --
	( 37.16, 65.75) --
	( 37.16, 65.75) --
	( 37.16, 65.75) --
	( 37.09, 65.76) --
	( 37.09, 65.76) --
	( 37.09, 65.76) --
	( 37.01, 65.75) --
	( 37.01, 65.75) --
	( 37.01, 65.75) --
	( 36.94, 65.75) --
	( 36.94, 65.75) --
	( 36.94, 65.75) --
	( 36.92, 65.75) --
	( 36.92, 65.75) --
	( 36.92, 65.75) --
	( 36.86, 65.77) --
	( 36.86, 65.77) --
	( 36.86, 65.77) --
	( 36.79, 65.75) --
	( 36.79, 65.75) --
	( 36.79, 65.75) --
	( 36.76, 65.76) --
	( 36.76, 65.76) --
	( 36.76, 65.76) --
	( 36.71, 65.77) --
	( 36.71, 65.77) --
	( 36.71, 65.77) --
	( 36.64, 65.78) --
	( 36.64, 65.78) --
	( 36.64, 65.78) --
	( 36.57, 65.75) --
	( 36.57, 65.75) --
	( 36.57, 65.75) --
	( 36.49, 65.77) --
	( 36.49, 65.77) --
	( 36.49, 65.77) --
	( 36.43, 65.76) --
	( 36.43, 65.76) --
	( 36.43, 65.76) --
	( 36.42, 65.76) --
	( 36.42, 65.76) --
	( 36.42, 65.76) --
	( 36.34, 65.76) --
	( 36.34, 65.76) --
	( 36.34, 65.76) --
	( 36.27, 65.76) --
	( 36.27, 65.76) --
	( 36.27, 65.76) --
	( 36.23, 65.76) --
	( 36.23, 65.76) --
	( 36.23, 65.76) --
	( 36.19, 65.75) --
	( 36.19, 65.75) --
	( 36.19, 65.75) --
	( 36.15, 65.76) --
	( 36.15, 65.76) --
	( 36.15, 65.76) --
	( 36.12, 65.77) --
	( 36.12, 65.77) --
	( 36.12, 65.77) --
	( 36.11, 65.77) --
	( 36.11, 65.77) --
	( 36.11, 65.77) --
	( 36.04, 65.75) --
	( 36.04, 65.75) --
	( 36.04, 65.75) --
	( 35.97, 65.74) --
	( 35.97, 65.74) --
	( 35.97, 65.74) --
	( 35.89, 65.75) --
	( 35.89, 65.75) --
	( 35.89, 65.75) --
	( 35.87, 65.76) --
	( 35.87, 65.76) --
	( 35.87, 65.76) --
	( 35.83, 65.76) --
	( 35.83, 65.76) --
	( 35.83, 65.76) --
	( 35.82, 65.76) --
	( 35.82, 65.76) --
	( 35.82, 65.76) --
	( 35.82, 65.77) --
	( 35.82, 65.77) --
	( 35.82, 65.77) --
	( 35.74, 65.75) --
	( 35.74, 65.75) --
	( 35.74, 65.75) --
	( 35.71, 65.76) --
	( 35.71, 65.76) --
	( 35.71, 65.76) --
	( 35.67, 65.78) --
	( 35.67, 65.78) --
	( 35.67, 65.78) --
	( 35.60, 65.77) --
	( 35.60, 65.77) --
	( 35.60, 65.77) --
	( 35.58, 65.77) --
	( 35.58, 65.77) --
	( 35.58, 65.77) --
	( 35.52, 65.73) --
	( 35.52, 65.73) --
	( 35.52, 65.73) --
	( 35.50, 65.74) --
	( 35.50, 65.74) --
	( 35.50, 65.74) --
	( 35.45, 65.76) --
	( 35.45, 65.76) --
	( 35.45, 65.76) --
	( 35.38, 65.77) --
	( 35.38, 65.77) --
	( 35.38, 65.77) --
	( 35.37, 65.77) --
	( 35.37, 65.77) --
	( 35.37, 65.77) --
	( 35.34, 65.77) --
	( 35.34, 65.77) --
	( 35.34, 65.77) --
	( 35.30, 65.77) --
	( 35.30, 65.77) --
	( 35.30, 65.77) --
	( 35.22, 65.75) --
	( 35.22, 65.75) --
	( 35.22, 65.75) --
	( 35.18, 65.74) --
	( 35.18, 65.74) --
	( 35.18, 65.74) --
	( 35.15, 65.74) --
	( 35.15, 65.74) --
	( 35.15, 65.74) --
	( 35.07, 65.76) --
	( 35.07, 65.76) --
	( 35.07, 65.76) --
	( 35.06, 65.76) --
	( 35.06, 65.76) --
	( 35.06, 65.76) --
	( 35.00, 65.76) --
	( 35.00, 65.76) --
	( 35.00, 65.76) --
	( 34.98, 65.76) --
	( 34.98, 65.76) --
	( 34.98, 65.76) --
	( 34.92, 65.73) --
	( 34.92, 65.73) --
	( 34.92, 65.73) --
	( 34.87, 65.77) --
	( 34.87, 65.77) --
	( 34.87, 65.77) --
	( 34.86, 65.77) --
	( 34.86, 65.77) --
	( 34.86, 65.77) --
	( 34.85, 65.78) --
	( 34.85, 65.78) --
	( 34.85, 65.78) --
	( 34.82, 65.77) --
	( 34.82, 65.77) --
	( 34.82, 65.77) --
	( 34.77, 65.75) --
	( 34.77, 65.75) --
	( 34.77, 65.75) --
	( 34.74, 65.75) --
	( 34.74, 65.75) --
	( 34.74, 65.75) --
	( 34.70, 65.74) --
	( 34.70, 65.74) --
	( 34.70, 65.74) --
	( 34.70, 65.74) --
	( 34.70, 65.74) --
	( 34.70, 65.74) --
	( 34.62, 65.75) --
	( 34.62, 65.75) --
	( 34.62, 65.75) --
	( 34.61, 65.75) --
	( 34.61, 65.75) --
	( 34.61, 65.75) --
	( 34.55, 65.74) --
	( 34.55, 65.74) --
	( 34.55, 65.74) --
	( 34.53, 65.74) --
	( 34.53, 65.74) --
	( 34.53, 65.74) --
	( 34.47, 65.78) --
	( 34.47, 65.78) --
	( 34.47, 65.78) --
	( 34.41, 65.75) --
	( 34.41, 65.75) --
	( 34.41, 65.75) --
	( 34.40, 65.75) --
	( 34.40, 65.75) --
	( 34.40, 65.75) --
	( 34.32, 65.75) --
	( 34.32, 65.75) --
	( 34.32, 65.75) --
	( 34.25, 65.76) --
	( 34.25, 65.76) --
	( 34.25, 65.76) --
	( 34.25, 65.76) --
	( 34.25, 65.76) --
	( 34.25, 65.76) --
	( 34.21, 65.78) --
	( 34.21, 65.78) --
	( 34.21, 65.78) --
	( 34.17, 65.80) --
	( 34.17, 65.80) --
	( 34.17, 65.80) --
	( 34.10, 65.76) --
	( 34.10, 65.76) --
	( 34.10, 65.76) --
	( 34.09, 65.76) --
	( 34.09, 65.76) --
	( 34.09, 65.76) --
	( 34.05, 65.75) --
	( 34.05, 65.75) --
	( 34.05, 65.75) --
	( 34.03, 65.75) --
	( 34.03, 65.75) --
	( 34.03, 65.75) --
	( 33.95, 65.75) --
	( 33.95, 65.75) --
	( 33.95, 65.75) --
	( 33.89, 65.75) --
	( 33.89, 65.75) --
	( 33.89, 65.75) --
	( 33.88, 65.75) --
	( 33.88, 65.75) --
	( 33.88, 65.75) --
	( 33.85, 65.75) --
	( 33.85, 65.75) --
	( 33.85, 65.75) --
	( 33.80, 65.74) --
	( 33.80, 65.74) --
	( 33.80, 65.74) --
	( 33.73, 65.76) --
	( 33.73, 65.76) --
	( 33.73, 65.76) --
	( 33.65, 65.80) --
	( 33.65, 65.80) --
	( 33.65, 65.80) --
	( 33.64, 65.79) --
	( 33.64, 65.79) --
	( 33.64, 65.79) --
	( 33.58, 65.77) --
	( 33.58, 65.77) --
	( 33.58, 65.77) --
	( 33.50, 65.74) --
	( 33.50, 65.74) --
	( 33.50, 65.74) --
	( 33.44, 65.76) --
	( 33.44, 65.76) --
	( 33.44, 65.76) --
	( 33.43, 65.77) --
	( 33.43, 65.77) --
	( 33.43, 65.77) --
	( 33.35, 65.76) --
	( 33.35, 65.76) --
	( 33.35, 65.76) --
	( 33.28, 65.77) --
	( 33.28, 65.77) --
	( 33.28, 65.77) --
	( 33.28, 65.77) --
	( 33.28, 65.77) --
	( 33.28, 65.77) --
	( 33.20, 65.77) --
	( 33.20, 65.77) --
	( 33.20, 65.77) --
	( 33.20, 65.77) --
	( 33.20, 65.77) --
	( 33.20, 65.77) --
	( 33.16, 65.76) --
	( 33.16, 65.76) --
	( 33.16, 65.76) --
	( 33.13, 65.76) --
	( 33.13, 65.76) --
	( 33.13, 65.76) --
	( 33.12, 65.76) --
	( 33.12, 65.76) --
	( 33.12, 65.76) --
	( 33.08, 65.76) --
	( 33.08, 65.76) --
	( 33.08, 65.76) --
	( 33.05, 65.76) --
	( 33.05, 65.76) --
	( 33.05, 65.76) --
	( 33.00, 65.75) --
	( 33.00, 65.75) --
	( 33.00, 65.75) --
	( 32.98, 65.75) --
	( 32.98, 65.75) --
	( 32.98, 65.75) --
	( 32.96, 65.75) --
	( 32.96, 65.75) --
	( 32.96, 65.75) --
	( 32.90, 65.75) --
	( 32.90, 65.75) --
	( 32.90, 65.75) --
	( 32.83, 65.78) --
	( 32.83, 65.78) --
	( 32.83, 65.78) --
	( 32.75, 65.77) --
	( 32.75, 65.77) --
	( 32.75, 65.77) --
	( 32.71, 65.76) --
	( 32.71, 65.76) --
	( 32.71, 65.76) --
	( 32.68, 65.76) --
	( 32.68, 65.76) --
	( 32.68, 65.76) --
	( 32.67, 65.76) --
	( 32.67, 65.76) --
	( 32.67, 65.76) --
	( 32.60, 65.76) --
	( 32.60, 65.76) --
	( 32.60, 65.76) --
	( 32.59, 65.76) --
	( 32.59, 65.76) --
	( 32.59, 65.76) --
	( 32.55, 65.76) --
	( 32.55, 65.76) --
	( 32.55, 65.76) --
	( 32.53, 65.77) --
	( 32.53, 65.77) --
	( 32.53, 65.77) --
	( 32.47, 65.75) --
	( 32.47, 65.75) --
	( 32.47, 65.75) --
	( 32.47, 65.75) --
	( 32.47, 65.75) --
	( 32.47, 65.75) --
	( 32.45, 65.75) --
	( 32.45, 65.75) --
	( 32.45, 65.75) --
	( 32.43, 65.75) --
	( 32.43, 65.75) --
	( 32.43, 65.75) --
	( 32.39, 65.75) --
	( 32.39, 65.75) --
	( 32.39, 65.75) --
	( 32.38, 65.75) --
	( 32.38, 65.75) --
	( 32.38, 65.75) --
	( 32.30, 65.77) --
	( 32.30, 65.77) --
	( 32.30, 65.77) --
	( 32.27, 65.77) --
	( 32.27, 65.77) --
	( 32.27, 65.77) --
	( 32.23, 65.76) --
	( 32.23, 65.76) --
	( 32.23, 65.76) --
	( 32.23, 65.76) --
	( 32.23, 65.76) --
	( 32.23, 65.76) --
	( 32.15, 65.74) --
	( 32.15, 65.74) --
	( 32.15, 65.74) --
	( 32.11, 65.75) --
	( 32.11, 65.75) --
	( 32.11, 65.75) --
	( 32.08, 65.75) --
	( 32.08, 65.75) --
	( 32.08, 65.75) --
	( 32.00, 65.75) --
	( 32.00, 65.75) --
	( 32.00, 65.75) --
	( 31.95, 65.76) --
	( 31.95, 65.76) --
	( 31.95, 65.76) --
	( 31.93, 65.77) --
	( 31.93, 65.77) --
	( 31.93, 65.77) --
	( 31.87, 65.76) --
	( 31.87, 65.76) --
	( 31.87, 65.76) --
	( 31.85, 65.76) --
	( 31.85, 65.76) --
	( 31.85, 65.76) --
	( 31.78, 65.75) --
	( 31.78, 65.75) --
	( 31.78, 65.75) --
	( 31.74, 65.76) --
	( 31.74, 65.76) --
	( 31.74, 65.76) --
	( 31.71, 65.78) --
	( 31.71, 65.78) --
	( 31.71, 65.78) --
	( 31.70, 65.78) --
	( 31.70, 65.78) --
	( 31.70, 65.78) --
	( 31.63, 65.73) --
	( 31.63, 65.73) --
	( 31.63, 65.73) --
	( 31.58, 65.74) --
	( 31.58, 65.74) --
	( 31.58, 65.74) --
	( 31.56, 65.75) --
	( 31.56, 65.75) --
	( 31.56, 65.75) --
	( 31.48, 65.77) --
	( 31.48, 65.77) --
	( 31.48, 65.77) --
	( 31.46, 65.77) --
	( 31.46, 65.77) --
	( 31.46, 65.77) --
	( 31.42, 65.78) --
	( 31.42, 65.78) --
	( 31.42, 65.78) --
	( 31.41, 65.78) --
	( 31.41, 65.78) --
	( 31.41, 65.78) --
	( 31.33, 65.74) --
	( 31.33, 65.74) --
	( 31.33, 65.74) --
	( 31.30, 65.74) --
	( 31.30, 65.74) --
	( 31.30, 65.74) --
	( 31.26, 65.73) --
	( 31.26, 65.73) --
	( 31.26, 65.73) --
	( 31.18, 65.76) --
	( 31.18, 65.76) --
	( 31.18, 65.76) --
	( 31.14, 65.77) --
	( 31.14, 65.77) --
	( 31.14, 65.77) --
	( 31.10, 65.77) --
	( 31.10, 65.77) --
	( 31.10, 65.77) --
	( 31.03, 65.76) --
	( 31.03, 65.76) --
	( 31.03, 65.76) --
	( 30.96, 65.77) --
	( 30.96, 65.77) --
	( 30.96, 65.77) --
	( 30.94, 65.77) --
	( 30.94, 65.77) --
	( 30.94, 65.77) --
	( 30.88, 65.76) --
	( 30.88, 65.76) --
	( 30.88, 65.76) --
	( 30.81, 65.75) --
	( 30.81, 65.75) --
	( 30.81, 65.75) --
	( 30.73, 65.76) --
	( 30.73, 65.76) --
	( 30.73, 65.76) --
	( 30.73, 65.76) --
	( 30.73, 65.76) --
	( 30.73, 65.76) --
	( 30.66, 65.78) --
	( 30.66, 65.78) --
	( 30.66, 65.78) --
	( 30.58, 65.77) --
	( 30.58, 65.77) --
	( 30.58, 65.77) --
	( 30.51, 65.76) --
	( 30.51, 65.76) --
	( 30.51, 65.76) --
	( 30.43, 65.75) --
	( 30.43, 65.75) --
	( 30.43, 65.75) --
	( 30.41, 65.75) --
	( 30.41, 65.75) --
	( 30.41, 65.75) --
	( 30.36, 65.77) --
	( 30.36, 65.77) --
	( 30.36, 65.77) --
	( 30.28, 65.77) --
	( 30.28, 65.77) --
	( 30.28, 65.77) --
	( 30.21, 65.73) --
	( 30.21, 65.73) --
	( 30.21, 65.73) --
	( 30.13, 65.76) --
	( 30.13, 65.76) --
	( 30.13, 65.76) --
	( 30.06, 65.77) --
	( 30.06, 65.77) --
	( 30.06, 65.77) --
	( 29.98, 65.75) --
	( 29.98, 65.75) --
	( 29.98, 65.75) --
	( 29.93, 65.76) --
	( 29.93, 65.76) --
	( 29.93, 65.76) --
	( 29.91, 65.77) --
	( 29.91, 65.77) --
	( 29.91, 65.77) --
	( 29.83, 65.75) --
	( 29.83, 65.75) --
	( 29.83, 65.75) --
	( 29.76, 65.76) --
	( 29.76, 65.76) --
	( 29.76, 65.76) --
	( 29.68, 65.75) --
	( 29.68, 65.75) --
	( 29.68, 65.75) --
	( 29.61, 65.75) --
	( 29.61, 65.75) --
	( 29.61, 65.75) --
	( 29.53, 65.78) --
	( 29.53, 65.78) --
	( 29.53, 65.78) --
	( 29.46, 65.77) --
	( 29.46, 65.77) --
	( 29.46, 65.77) --
	( 29.40, 65.76) --
	( 29.40, 65.76) --
	( 29.40, 65.76) --
	( 29.38, 65.76) --
	( 29.38, 65.76) --
	( 29.38, 65.76) --
	( 29.31, 65.76) --
	( 29.31, 65.76) --
	( 29.31, 65.76) --
	( 29.23, 65.76) --
	( 29.23, 65.76) --
	( 29.23, 65.76) --
	( 29.20, 65.76) --
	( 29.20, 65.76) --
	( 29.20, 65.76) --
	( 29.16, 65.76) --
	( 29.16, 65.76) --
	( 29.16, 65.76) --
	( 29.08, 65.77) --
	( 29.08, 65.77) --
	( 29.08, 65.77) --
	( 29.01, 65.76) --
	( 29.01, 65.76) --
	( 29.01, 65.76) --
	( 28.93, 65.76) --
	( 28.93, 65.76) --
	( 28.93, 65.76) --
	( 28.86, 65.75) --
	( 28.86, 65.75) --
	( 28.86, 65.75) --
	( 28.78, 65.76) --
	( 28.78, 65.76) --
	( 28.78, 65.76) --
	( 28.71, 65.77) --
	( 28.71, 65.77) --
	( 28.71, 65.77) --
	( 28.63, 65.75) --
	( 28.63, 65.75) --
	( 28.63, 65.75) --
	( 28.56, 65.75) --
	( 28.56, 65.75) --
	( 28.56, 65.75) --
	( 28.48, 65.76) --
	( 28.48, 65.76) --
	( 28.48, 65.76) --
	( 28.40, 65.76) --
	( 28.40, 65.76) --
	( 28.40, 65.76) --
	( 28.39, 65.76) --
	( 28.39, 65.76) --
	( 28.39, 65.76) --
	( 28.33, 65.77) --
	( 28.33, 65.77) --
	( 28.33, 65.77) --
	( 28.25, 65.76) --
	( 28.25, 65.76) --
	( 28.25, 65.76) --
	( 28.19, 65.75) --
	( 28.19, 65.75) --
	( 28.19, 65.75) --
	( 28.18, 65.75) --
	( 28.18, 65.75) --
	( 28.18, 65.75) --
	( 28.10, 65.75) --
	( 28.10, 65.75) --
	( 28.10, 65.75) --
	( 28.03, 65.75) --
	( 28.03, 65.75) --
	( 28.03, 65.75) --
	( 28.03, 65.75) --
	( 28.03, 65.75) --
	( 28.03, 65.75) --
	( 27.95, 65.74) --
	( 27.95, 65.74) --
	( 27.95, 65.74) --
	( 27.88, 65.78) --
	( 27.88, 65.78) --
	( 27.88, 65.78) --
	( 27.80, 65.76) --
	( 27.80, 65.76) --
	( 27.80, 65.76) --
	( 27.73, 65.74) --
	( 27.73, 65.74) --
	( 27.73, 65.74) --
	( 27.66, 65.76) --
	( 27.66, 65.76) --
	( 27.66, 65.76) --
	( 27.65, 65.76) --
	( 27.65, 65.76) --
	( 27.65, 65.76) --
	( 27.58, 65.76) --
	( 27.58, 65.76) --
	( 27.58, 65.76) --
	( 27.50, 65.76) --
	( 27.50, 65.76) --
	( 27.50, 65.76) --
	( 27.43, 65.76) --
	( 27.43, 65.76) --
	( 27.43, 65.76) --
	( 27.39, 65.75) --
	( 27.39, 65.75) --
	( 27.39, 65.75) --
	( 27.35, 65.73) --
	( 27.35, 65.73) --
	( 27.35, 65.73) --
	( 27.28, 65.76) --
	( 27.28, 65.76) --
	( 27.28, 65.76) --
	( 27.20, 65.77) --
	( 27.20, 65.77) --
	( 27.20, 65.77) --
	( 27.13, 65.75) --
	( 27.13, 65.75) --
	( 27.13, 65.75) --
	( 27.06, 65.77) --
	( 27.06, 65.77) --
	( 27.06, 65.77) --
	( 27.05, 65.77) --
	( 27.05, 65.77) --
	( 27.05, 65.77) --
	( 26.98, 65.75) --
	( 26.98, 65.75) --
	( 26.98, 65.75) --
	( 26.90, 65.72) --
	( 26.90, 65.72) --
	( 26.90, 65.72) --
	( 26.83, 65.76) --
	( 26.83, 65.76) --
	( 26.83, 65.76) --
	( 26.75, 65.77) --
	( 26.75, 65.77) --
	( 26.75, 65.77) --
	( 26.68, 65.75) --
	( 26.68, 65.75) --
	( 26.68, 65.75) --
	( 26.60, 65.78) --
	( 26.60, 65.78) --
	( 26.60, 65.78) --
	( 26.53, 65.75) --
	( 26.53, 65.75) --
	( 26.53, 65.75) --
	( 26.45, 65.78) --
	( 26.45, 65.78) --
	( 26.45, 65.78) --
	( 26.38, 65.75) --
	( 26.38, 65.75) --
	( 26.38, 65.75) --
	( 26.30, 65.74) --
	( 26.30, 65.74) --
	( 26.30, 65.74) --
	( 26.22, 65.75) --
	( 26.22, 65.75) --
	( 26.22, 65.75) --
	( 26.15, 65.75) --
	( 26.15, 65.75) --
	( 26.15, 65.75) --
	( 26.07, 65.75) --
	( 26.07, 65.75) --
	( 26.07, 65.75) --
	( 26.00, 65.77) --
	( 26.00, 65.77) --
	( 26.00, 65.77) --
	( 25.92, 65.74) --
	( 25.92, 65.74) --
	( 25.92, 65.74) --
	( 25.85, 65.74) --
	( 25.85, 65.74) --
	( 25.85, 65.74) --
	( 25.80, 65.76) --
	( 25.80, 65.76) --
	( 25.80, 65.76) --
	( 25.77, 65.77) --
	( 25.77, 65.77) --
	( 25.77, 65.77) --
	( 25.70, 65.73) --
	( 25.70, 65.73) --
	( 25.70, 65.73) --
	( 25.62, 65.77) --
	( 25.62, 65.77) --
	( 25.62, 65.77) --
	( 25.55, 65.76) --
	( 25.55, 65.76) --
	( 25.55, 65.76) --
	( 25.47, 65.77) --
	( 25.47, 65.77) --
	( 25.47, 65.77) --
	( 25.40, 65.75) --
	( 25.40, 65.75) --
	( 25.40, 65.75) --
	( 25.32, 65.76) --
	( 25.32, 65.76) --
	( 25.32, 65.76) --
	( 25.25, 65.75) --
	( 25.25, 65.75) --
	( 25.25, 65.75) --
	( 25.17, 65.77) --
	( 25.17, 65.77) --
	( 25.17, 65.77) --
	( 25.12, 65.77) --
	( 25.12, 65.77) --
	( 25.12, 65.77) --
	( 25.10, 65.77) --
	( 25.10, 65.77) --
	( 25.10, 65.77) --
	( 25.02, 65.75) --
	( 25.02, 65.75) --
	( 25.02, 65.75) --
	( 24.95, 65.75) --
	( 24.95, 65.75) --
	( 24.95, 65.75) --
	( 24.87, 65.75) --
	( 24.87, 65.75) --
	( 24.87, 65.75) --
	( 24.83, 65.77) --
	( 24.83, 65.77) --
	( 24.83, 65.77) --
	( 24.80, 65.79) --
	( 24.80, 65.79) --
	( 24.80, 65.79) --
	( 24.75, 65.77) --
	( 24.75, 65.77) --
	( 24.75, 65.77) --
	( 24.72, 65.76) --
	( 24.72, 65.76) --
	( 24.72, 65.76) --
	( 24.65, 65.75) --
	( 24.65, 65.75) --
	( 24.65, 65.75) --
	( 24.63, 65.75) --
	( 24.63, 65.75) --
	( 24.63, 65.75) --
	( 24.57, 65.77) --
	( 24.57, 65.77) --
	( 24.57, 65.77) --
	( 24.55, 65.77) --
	( 24.55, 65.77) --
	( 24.55, 65.77) --
	( 24.51, 65.76) --
	( 24.51, 65.76) --
	( 24.51, 65.76) --
	( 24.50, 65.76) --
	( 24.50, 65.76) --
	( 24.50, 65.76) --
	( 24.42, 65.74) --
	( 24.42, 65.74) --
	( 24.42, 65.74) --
	( 24.42, 65.74) --
	( 24.42, 65.74) --
	( 24.42, 65.74) --
	( 24.34, 65.74) --
	( 24.34, 65.74) --
	( 24.34, 65.74) --
	( 24.27, 65.76) --
	( 24.27, 65.76) --
	( 24.27, 65.76) --
	( 24.27, 65.76) --
	( 24.27, 65.76) --
	( 24.27, 65.76) --
	cycle;
\definecolor{drawColor}{RGB}{248,118,109}

\path[draw=drawColor,line width= 0.6pt,line join=round] ( 24.27, 65.84) --
	( 24.27, 65.84) --
	( 24.27, 65.84) --
	( 24.27, 65.84) --
	( 24.27, 65.84) --
	( 24.34, 65.86) --
	( 24.34, 65.86) --
	( 24.34, 65.86) --
	( 24.42, 65.89) --
	( 24.42, 65.89) --
	( 24.42, 65.89) --
	( 24.42, 65.89) --
	( 24.42, 65.89) --
	( 24.42, 65.89) --
	( 24.50, 65.95) --
	( 24.50, 65.95) --
	( 24.50, 65.95) --
	( 24.51, 65.96) --
	( 24.51, 65.96) --
	( 24.51, 65.96) --
	( 24.55, 66.06) --
	( 24.55, 66.06) --
	( 24.55, 66.06) --
	( 24.57, 66.09) --
	( 24.57, 66.09) --
	( 24.57, 66.09) --
	( 24.63, 66.19) --
	( 24.63, 66.19) --
	( 24.63, 66.19) --
	( 24.65, 66.17) --
	( 24.65, 66.17) --
	( 24.65, 66.17) --
	( 24.72, 66.11) --
	( 24.72, 66.11) --
	( 24.72, 66.11) --
	( 24.75, 66.09) --
	( 24.75, 66.09) --
	( 24.75, 66.09) --
	( 24.80, 66.05) --
	( 24.80, 66.05) --
	( 24.80, 66.05) --
	( 24.83, 65.99) --
	( 24.83, 65.99) --
	( 24.83, 65.99) --
	( 24.87, 65.96) --
	( 24.87, 65.96) --
	( 24.87, 65.96) --
	( 24.95, 65.94) --
	( 24.95, 65.94) --
	( 24.95, 65.94) --
	( 25.02, 65.92) --
	( 25.02, 65.92) --
	( 25.02, 65.92) --
	( 25.10, 65.92) --
	( 25.10, 65.92) --
	( 25.10, 65.92) --
	( 25.12, 65.92) --
	( 25.12, 65.92) --
	( 25.12, 65.92) --
	( 25.17, 65.92) --
	( 25.17, 65.92) --
	( 25.17, 65.92) --
	( 25.25, 65.89) --
	( 25.25, 65.89) --
	( 25.25, 65.89) --
	( 25.32, 65.90) --
	( 25.32, 65.90) --
	( 25.32, 65.90) --
	( 25.40, 65.88) --
	( 25.40, 65.88) --
	( 25.40, 65.88) --
	( 25.47, 65.90) --
	( 25.47, 65.90) --
	( 25.47, 65.90) --
	( 25.55, 65.89) --
	( 25.55, 65.89) --
	( 25.55, 65.89) --
	( 25.62, 65.89) --
	( 25.62, 65.89) --
	( 25.62, 65.89) --
	( 25.70, 65.86) --
	( 25.70, 65.86) --
	( 25.70, 65.86) --
	( 25.77, 65.89) --
	( 25.77, 65.89) --
	( 25.77, 65.89) --
	( 25.80, 65.88) --
	( 25.80, 65.88) --
	( 25.80, 65.88) --
	( 25.85, 65.86) --
	( 25.85, 65.86) --
	( 25.85, 65.86) --
	( 25.92, 65.86) --
	( 25.92, 65.86) --
	( 25.92, 65.86) --
	( 26.00, 65.90) --
	( 26.00, 65.90) --
	( 26.00, 65.90) --
	( 26.07, 65.88) --
	( 26.07, 65.88) --
	( 26.07, 65.88) --
	( 26.15, 65.88) --
	( 26.15, 65.88) --
	( 26.15, 65.88) --
	( 26.22, 65.89) --
	( 26.22, 65.89) --
	( 26.22, 65.89) --
	( 26.30, 65.88) --
	( 26.30, 65.88) --
	( 26.30, 65.88) --
	( 26.38, 65.90) --
	( 26.38, 65.90) --
	( 26.38, 65.90) --
	( 26.45, 65.92) --
	( 26.45, 65.92) --
	( 26.45, 65.92) --
	( 26.53, 65.90) --
	( 26.53, 65.90) --
	( 26.53, 65.90) --
	( 26.60, 65.93) --
	( 26.60, 65.93) --
	( 26.60, 65.93) --
	( 26.68, 65.91) --
	( 26.68, 65.91) --
	( 26.68, 65.91) --
	( 26.75, 65.93) --
	( 26.75, 65.93) --
	( 26.75, 65.93) --
	( 26.83, 65.92) --
	( 26.83, 65.92) --
	( 26.83, 65.92) --
	( 26.90, 65.89) --
	( 26.90, 65.89) --
	( 26.90, 65.89) --
	( 26.98, 65.92) --
	( 26.98, 65.92) --
	( 26.98, 65.92) --
	( 27.05, 65.94) --
	( 27.05, 65.94) --
	( 27.05, 65.94) --
	( 27.06, 65.94) --
	( 27.06, 65.94) --
	( 27.06, 65.94) --
	( 27.13, 65.92) --
	( 27.13, 65.92) --
	( 27.13, 65.92) --
	( 27.20, 65.94) --
	( 27.20, 65.94) --
	( 27.20, 65.94) --
	( 27.28, 65.92) --
	( 27.28, 65.92) --
	( 27.28, 65.92) --
	( 27.35, 65.89) --
	( 27.35, 65.89) --
	( 27.35, 65.89) --
	( 27.39, 65.90) --
	( 27.39, 65.90) --
	( 27.39, 65.90) --
	( 27.43, 65.92) --
	( 27.43, 65.92) --
	( 27.43, 65.92) --
	( 27.50, 65.91) --
	( 27.50, 65.91) --
	( 27.50, 65.91) --
	( 27.58, 65.91) --
	( 27.58, 65.91) --
	( 27.58, 65.91) --
	( 27.65, 65.91) --
	( 27.65, 65.91) --
	( 27.65, 65.91) --
	( 27.66, 65.91) --
	( 27.66, 65.91) --
	( 27.66, 65.91) --
	( 27.73, 65.89) --
	( 27.73, 65.89) --
	( 27.73, 65.89) --
	( 27.80, 65.94) --
	( 27.80, 65.94) --
	( 27.80, 65.94) --
	( 27.88, 65.97) --
	( 27.88, 65.97) --
	( 27.88, 65.97) --
	( 27.95, 65.95) --
	( 27.95, 65.95) --
	( 27.95, 65.95) --
	( 28.03, 65.97) --
	( 28.03, 65.97) --
	( 28.03, 65.97) --
	( 28.03, 65.97) --
	( 28.03, 65.97) --
	( 28.03, 65.97) --
	( 28.10, 65.98) --
	( 28.10, 65.98) --
	( 28.10, 65.98) --
	( 28.18, 65.99) --
	( 28.18, 65.99) --
	( 28.18, 65.99) --
	( 28.19, 65.99) --
	( 28.19, 65.99) --
	( 28.19, 65.99) --
	( 28.25, 65.98) --
	( 28.25, 65.98) --
	( 28.25, 65.98) --
	( 28.33, 65.96) --
	( 28.33, 65.96) --
	( 28.33, 65.96) --
	( 28.39, 65.93) --
	( 28.39, 65.93) --
	( 28.39, 65.93) --
	( 28.40, 65.93) --
	( 28.40, 65.93) --
	( 28.40, 65.93) --
	( 28.48, 65.93) --
	( 28.48, 65.93) --
	( 28.48, 65.93) --
	( 28.56, 65.92) --
	( 28.56, 65.92) --
	( 28.56, 65.92) --
	( 28.63, 65.92) --
	( 28.63, 65.92) --
	( 28.63, 65.92) --
	( 28.71, 65.94) --
	( 28.71, 65.94) --
	( 28.71, 65.94) --
	( 28.78, 65.93) --
	( 28.78, 65.93) --
	( 28.78, 65.93) --
	( 28.86, 65.92) --
	( 28.86, 65.92) --
	( 28.86, 65.92) --
	( 28.93, 65.93) --
	( 28.93, 65.93) --
	( 28.93, 65.93) --
	( 29.01, 65.93) --
	( 29.01, 65.93) --
	( 29.01, 65.93) --
	( 29.08, 65.94) --
	( 29.08, 65.94) --
	( 29.08, 65.94) --
	( 29.16, 65.93) --
	( 29.16, 65.93) --
	( 29.16, 65.93) --
	( 29.20, 65.93) --
	( 29.20, 65.93) --
	( 29.20, 65.93) --
	( 29.23, 65.93) --
	( 29.23, 65.93) --
	( 29.23, 65.93) --
	( 29.31, 65.94) --
	( 29.31, 65.94) --
	( 29.31, 65.94) --
	( 29.38, 65.93) --
	( 29.38, 65.93) --
	( 29.38, 65.93) --
	( 29.40, 65.94) --
	( 29.40, 65.94) --
	( 29.40, 65.94) --
	( 29.46, 65.95) --
	( 29.46, 65.95) --
	( 29.46, 65.95) --
	( 29.53, 65.96) --
	( 29.53, 65.96) --
	( 29.53, 65.96) --
	( 29.61, 65.93) --
	( 29.61, 65.93) --
	( 29.61, 65.93) --
	( 29.68, 65.93) --
	( 29.68, 65.93) --
	( 29.68, 65.93) --
	( 29.76, 65.95) --
	( 29.76, 65.95) --
	( 29.76, 65.95) --
	( 29.83, 65.94) --
	( 29.83, 65.94) --
	( 29.83, 65.94) --
	( 29.91, 65.96) --
	( 29.91, 65.96) --
	( 29.91, 65.96) --
	( 29.93, 65.96) --
	( 29.93, 65.96) --
	( 29.93, 65.96) --
	( 29.98, 65.96) --
	( 29.98, 65.96) --
	( 29.98, 65.96) --
	( 30.06, 66.00) --
	( 30.06, 66.00) --
	( 30.06, 66.00) --
	( 30.13, 66.02) --
	( 30.13, 66.02) --
	( 30.13, 66.02) --
	( 30.21, 66.01) --
	( 30.21, 66.01) --
	( 30.21, 66.01) --
	( 30.28, 66.07) --
	( 30.28, 66.07) --
	( 30.28, 66.07) --
	( 30.36, 66.09) --
	( 30.36, 66.09) --
	( 30.36, 66.09) --
	( 30.41, 66.09) --
	( 30.41, 66.09) --
	( 30.41, 66.09) --
	( 30.43, 66.10) --
	( 30.43, 66.10) --
	( 30.43, 66.10) --
	( 30.51, 66.16) --
	( 30.51, 66.16) --
	( 30.51, 66.16) --
	( 30.58, 66.22) --
	( 30.58, 66.22) --
	( 30.58, 66.22) --
	( 30.66, 66.28) --
	( 30.66, 66.28) --
	( 30.66, 66.28) --
	( 30.73, 66.31) --
	( 30.73, 66.31) --
	( 30.73, 66.31) --
	( 30.73, 66.31) --
	( 30.73, 66.31) --
	( 30.73, 66.31) --
	( 30.81, 66.40) --
	( 30.81, 66.40) --
	( 30.81, 66.40) --
	( 30.88, 66.50) --
	( 30.88, 66.50) --
	( 30.88, 66.50) --
	( 30.94, 66.57) --
	( 30.94, 66.57) --
	( 30.94, 66.57) --
	( 30.96, 66.60) --
	( 30.96, 66.60) --
	( 30.96, 66.60) --
	( 31.03, 66.71) --
	( 31.03, 66.71) --
	( 31.03, 66.71) --
	( 31.10, 66.85) --
	( 31.10, 66.85) --
	( 31.10, 66.85) --
	( 31.14, 66.90) --
	( 31.14, 66.90) --
	( 31.14, 66.90) --
	( 31.18, 66.98) --
	( 31.18, 66.98) --
	( 31.18, 66.98) --
	( 31.26, 67.10) --
	( 31.26, 67.10) --
	( 31.26, 67.10) --
	( 31.30, 67.19) --
	( 31.30, 67.19) --
	( 31.30, 67.19) --
	( 31.33, 67.36) --
	( 31.33, 67.36) --
	( 31.33, 67.36) --
	( 31.41, 67.80) --
	( 31.41, 67.80) --
	( 31.41, 67.80) --
	( 31.42, 67.89) --
	( 31.42, 67.89) --
	( 31.42, 67.89) --
	( 31.46, 67.59) --
	( 31.46, 67.59) --
	( 31.46, 67.59) --
	( 31.48, 67.69) --
	( 31.48, 67.69) --
	( 31.48, 67.69) --
	( 31.56, 68.07) --
	( 31.56, 68.07) --
	( 31.56, 68.07) --
	( 31.58, 68.21) --
	( 31.58, 68.21) --
	( 31.58, 68.21) --
	( 31.63, 68.36) --
	( 31.63, 68.36) --
	( 31.63, 68.36) --
	( 31.70, 68.66) --
	( 31.70, 68.66) --
	( 31.70, 68.66) --
	( 31.71, 68.67) --
	( 31.71, 68.67) --
	( 31.71, 68.67) --
	( 31.74, 69.08) --
	( 31.74, 69.49) --
	( 31.74, 69.49) --
	( 31.78, 69.73) --
	( 31.78, 69.73) --
	( 31.78, 69.74) --
	( 31.85, 70.29) --
	( 31.85, 70.29) --
	( 31.85, 70.29) --
	( 31.87, 70.38) --
	( 31.87, 70.38) --
	( 31.87, 70.38) --
	( 31.93, 70.09) --
	( 31.93, 70.09) --
	( 31.93, 70.09) --
	( 31.95, 70.00) --
	( 31.95, 70.00) --
	( 31.95, 70.00) --
	( 32.00, 70.31) --
	( 32.00, 70.31) --
	( 32.00, 70.31) --
	( 32.08, 70.74) --
	( 32.08, 70.74) --
	( 32.08, 70.74) --
	( 32.11, 70.90) --
	( 32.11, 71.48) --
	( 32.11, 71.48) --
	( 32.15, 71.68) --
	( 32.15, 71.68) --
	( 32.15, 71.68) --
	( 32.23, 72.01) --
	( 32.23, 72.01) --
	( 32.23, 72.01) --
	( 32.23, 72.01) --
	( 32.23, 72.01) --
	( 32.23, 72.01) --
	( 32.27, 72.60) --
	( 32.27, 73.15) --
	( 32.27, 73.15) --
	( 32.30, 73.59) --
	( 32.30, 73.59) --
	( 32.30, 73.59) --
	( 32.38, 74.53) --
	( 32.38, 74.53) --
	( 32.38, 74.53) --
	( 32.39, 74.69) --
	( 32.39, 74.69) --
	( 32.39, 74.69) --
	( 32.43, 73.98) --
	( 32.43, 73.98) --
	( 32.43, 73.98) --
	( 32.45, 73.72) --
	( 32.45, 73.72) --
	( 32.45, 73.72) --
	( 32.47, 73.55) --
	( 32.47, 73.55) --
	( 32.47, 73.54) --
	( 32.47, 73.52) --
	( 32.47, 73.52) --
	( 32.47, 73.53) --
	( 32.53, 75.22) --
	( 32.53, 75.22) --
	( 32.53, 75.22) --
	( 32.55, 75.94) --
	( 32.55, 76.47) --
	( 32.55, 76.47) --
	( 32.59, 75.30) --
	( 32.59, 75.30) --
	( 32.59, 75.30) --
	( 32.60, 75.65) --
	( 32.60, 75.65) --
	( 32.60, 75.66) --
	( 32.67, 78.23) --
	( 32.67, 78.23) --
	( 32.67, 78.23) --
	( 32.68, 78.09) --
	( 32.68, 78.09) --
	( 32.68, 78.09) --
	( 32.71, 77.03) --
	( 32.71, 77.75) --
	( 32.71, 77.75) --
	( 32.75, 77.95) --
	( 32.75, 77.95) --
	( 32.75, 77.95) --
	( 32.83, 78.32) --
	( 32.83, 78.32) --
	( 32.83, 78.32) --
	( 32.90, 78.66) --
	( 32.90, 78.66) --
	( 32.90, 78.66) --
	( 32.96, 78.92) --
	( 32.96, 79.53) --
	( 32.96, 79.53) --
	( 32.98, 80.61) --
	( 32.98, 80.61) --
	( 32.98, 80.61) --
	( 33.00, 81.66) --
	( 33.00, 82.27) --
	( 33.00, 82.27) --
	( 33.05, 82.66) --
	( 33.05, 82.66) --
	( 33.05, 82.66) --
	( 33.08, 82.85) --
	( 33.08, 82.85) --
	( 33.08, 82.85) --
	( 33.12, 80.07) --
	( 33.12, 83.27) --
	( 33.12, 83.27) --
	( 33.13, 82.75) --
	( 33.13, 82.75) --
	( 33.13, 82.75) --
	( 33.16, 80.95) --
	( 33.16, 80.95) --
	( 33.16, 80.95) --
	( 33.20, 83.42) --
	( 33.20, 82.77) --
	( 33.20, 82.77) --
	( 33.20, 82.78) --
	( 33.20, 82.78) --
	( 33.20, 82.78) --
	( 33.28, 83.06) --
	( 33.28, 83.06) --
	( 33.28, 83.06) --
	( 33.28, 83.08) --
	( 33.28, 82.40) --
	( 33.28, 82.40) --
	( 33.35, 82.30) --
	( 33.35, 82.30) --
	( 33.35, 82.30) --
	( 33.43, 82.21) --
	( 33.43, 82.21) --
	( 33.43, 82.21) --
	( 33.44, 82.18) --
	( 33.44, 82.18) --
	( 33.44, 82.18) --
	( 33.50, 82.07) --
	( 33.50, 82.07) --
	( 33.50, 82.07) --
	( 33.58, 81.99) --
	( 33.58, 81.99) --
	( 33.58, 81.99) --
	( 33.64, 81.92) --
	( 33.64, 81.92) --
	( 33.64, 81.92) --
	( 33.65, 81.92) --
	( 33.65, 81.92) --
	( 33.65, 81.92) --
	( 33.73, 81.80) --
	( 33.73, 81.80) --
	( 33.73, 81.80) --
	( 33.80, 81.70) --
	( 33.80, 81.70) --
	( 33.80, 81.70) --
	( 33.85, 81.66) --
	( 33.85, 81.12) --
	( 33.85, 81.12) --
	( 33.88, 81.28) --
	( 33.88, 81.28) --
	( 33.88, 81.28) --
	( 33.89, 81.35) --
	( 33.89, 81.35) --
	( 33.89, 81.35) --
	( 33.95, 81.17) --
	( 33.95, 81.17) --
	( 33.95, 81.17) --
	( 34.03, 80.95) --
	( 34.03, 80.95) --
	( 34.03, 80.95) --
	( 34.05, 80.89) --
	( 34.05, 80.89) --
	( 34.05, 80.89) --
	( 34.09, 80.48) --
	( 34.09, 80.48) --
	( 34.09, 80.48) --
	( 34.10, 80.46) --
	( 34.10, 80.46) --
	( 34.10, 80.46) --
	( 34.17, 80.33) --
	( 34.17, 80.33) --
	( 34.17, 80.33) --
	( 34.21, 80.24) --
	( 34.21, 80.24) --
	( 34.21, 80.24) --
	( 34.25, 79.81) --
	( 34.25, 79.81) --
	( 34.25, 79.81) --
	( 34.25, 79.81) --
	( 34.25, 79.52) --
	( 34.25, 79.52) --
	( 34.32, 79.34) --
	( 34.32, 79.34) --
	( 34.32, 79.34) --
	( 34.40, 79.17) --
	( 34.40, 79.17) --
	( 34.40, 79.17) --
	( 34.41, 79.14) --
	( 34.41, 78.76) --
	( 34.41, 78.76) --
	( 34.47, 78.61) --
	( 34.47, 78.61) --
	( 34.47, 78.61) --
	( 34.53, 78.41) --
	( 34.53, 78.41) --
	( 34.53, 78.41) --
	( 34.55, 78.31) --
	( 34.55, 78.31) --
	( 34.55, 78.31) --
	( 34.61, 77.93) --
	( 34.61, 77.42) --
	( 34.61, 77.42) --
	( 34.62, 77.38) --
	( 34.62, 77.38) --
	( 34.62, 77.38) --
	( 34.70, 77.00) --
	( 34.70, 77.00) --
	( 34.70, 77.00) --
	( 34.70, 76.98) --
	( 34.70, 76.98) --
	( 34.70, 76.98) --
	( 34.74, 76.57) --
	( 34.74, 76.57) --
	( 34.74, 76.57) --
	( 34.77, 76.43) --
	( 34.77, 76.43) --
	( 34.77, 76.43) --
	( 34.82, 76.28) --
	( 34.82, 76.28) --
	( 34.82, 76.28) --
	( 34.85, 75.89) --
	( 34.85, 75.89) --
	( 34.85, 75.89) --
	( 34.86, 75.77) --
	( 34.86, 75.77) --
	( 34.86, 75.77) --
	( 34.87, 75.73) --
	( 34.87, 75.73) --
	( 34.87, 75.73) --
	( 34.92, 75.50) --
	( 34.92, 75.50) --
	( 34.92, 75.50) --
	( 34.98, 75.32) --
	( 34.98, 74.38) --
	( 34.98, 74.38) --
	( 35.00, 74.51) --
	( 35.00, 74.51) --
	( 35.00, 74.51) --
	( 35.06, 74.89) --
	( 35.06, 74.89) --
	( 35.06, 74.89) --
	( 35.07, 74.78) --
	( 35.07, 74.78) --
	( 35.07, 74.78) --
	( 35.15, 74.10) --
	( 35.15, 74.10) --
	( 35.15, 74.10) --
	( 35.18, 73.81) --
	( 35.18, 73.81) --
	( 35.18, 73.81) --
	( 35.22, 73.68) --
	( 35.22, 73.68) --
	( 35.22, 73.68) --
	( 35.30, 73.45) --
	( 35.30, 73.45) --
	( 35.30, 73.45) --
	( 35.34, 73.30) --
	( 35.34, 73.30) --
	( 35.34, 73.30) --
	( 35.37, 72.94) --
	( 35.37, 72.94) --
	( 35.37, 72.94) --
	( 35.38, 72.79) --
	( 35.38, 72.79) --
	( 35.38, 72.79) --
	( 35.45, 72.54) --
	( 35.45, 72.54) --
	( 35.45, 72.54) --
	( 35.50, 72.28) --
	( 35.50, 72.28) --
	( 35.50, 72.28) --
	( 35.52, 72.19) --
	( 35.52, 72.19) --
	( 35.52, 72.19) --
	( 35.58, 71.87) --
	( 35.58, 71.87) --
	( 35.58, 71.87) --
	( 35.60, 71.84) --
	( 35.60, 71.84) --
	( 35.60, 71.84) --
	( 35.67, 71.59) --
	( 35.67, 71.59) --
	( 35.67, 71.59) --
	( 35.71, 71.45) --
	( 35.71, 71.45) --
	( 35.71, 71.45) --
	( 35.74, 71.29) --
	( 35.74, 71.29) --
	( 35.74, 71.29) --
	( 35.82, 71.00) --
	( 35.82, 71.00) --
	( 35.82, 71.00) --
	( 35.82, 70.98) --
	( 35.82, 70.98) --
	( 35.82, 70.98) --
	( 35.83, 70.97) --
	( 35.83, 70.97) --
	( 35.83, 70.97) --
	( 35.87, 70.50) --
	( 35.87, 70.50) --
	( 35.87, 70.50) --
	( 35.89, 70.45) --
	( 35.89, 70.45) --
	( 35.89, 70.45) --
	( 35.97, 70.32) --
	( 35.97, 70.32) --
	( 35.97, 70.32) --
	( 36.04, 70.20) --
	( 36.04, 70.20) --
	( 36.04, 70.20) --
	( 36.11, 70.10) --
	( 36.11, 70.10) --
	( 36.11, 70.10) --
	( 36.12, 70.01) --
	( 36.12, 70.01) --
	( 36.12, 70.01) --
	( 36.15, 69.64) --
	( 36.15, 69.64) --
	( 36.15, 69.64) --
	( 36.19, 69.40) --
	( 36.19, 69.40) --
	( 36.19, 69.40) --
	( 36.23, 69.19) --
	( 36.23, 69.19) --
	( 36.23, 69.19) --
	( 36.27, 69.13) --
	( 36.27, 69.13) --
	( 36.27, 69.13) --
	( 36.34, 69.00) --
	( 36.34, 69.00) --
	( 36.34, 69.00) --
	( 36.42, 68.86) --
	( 36.42, 68.86) --
	( 36.42, 68.86) --
	( 36.43, 68.83) --
	( 36.43, 68.45) --
	( 36.43, 68.45) --
	( 36.49, 68.38) --
	( 36.49, 68.38) --
	( 36.49, 68.38) --
	( 36.57, 68.27) --
	( 36.57, 68.27) --
	( 36.57, 68.27) --
	( 36.64, 68.21) --
	( 36.64, 68.21) --
	( 36.64, 68.21) --
	( 36.71, 68.10) --
	( 36.71, 68.10) --
	( 36.71, 68.10) --
	( 36.76, 68.03) --
	( 36.76, 68.03) --
	( 36.76, 68.03) --
	( 36.79, 67.97) --
	( 36.79, 67.97) --
	( 36.79, 67.97) --
	( 36.86, 67.85) --
	( 36.86, 67.85) --
	( 36.86, 67.85) --
	( 36.92, 67.74) --
	( 36.92, 67.74) --
	( 36.92, 67.74) --
	( 36.94, 67.70) --
	( 36.94, 67.70) --
	( 36.94, 67.70) --
	( 37.01, 67.59) --
	( 37.01, 67.59) --
	( 37.01, 67.59) --
	( 37.09, 67.48) --
	( 37.09, 67.48) --
	( 37.09, 67.48) --
	( 37.16, 67.35) --
	( 37.16, 67.35) --
	( 37.16, 67.35) --
	( 37.16, 67.35) --
	( 37.16, 67.35) --
	( 37.16, 67.35) --
	( 37.24, 67.22) --
	( 37.24, 67.22) --
	( 37.24, 67.22) --
	( 37.31, 67.06) --
	( 37.31, 67.06) --
	( 37.31, 67.06) --
	( 37.32, 67.04) --
	( 37.32, 67.04) --
	( 37.32, 67.04) --
	( 37.39, 66.95) --
	( 37.39, 66.95) --
	( 37.39, 66.95) --
	( 37.46, 66.85) --
	( 37.46, 66.85) --
	( 37.46, 66.85) --
	( 37.54, 66.78) --
	( 37.54, 66.78) --
	( 37.54, 66.78) --
	( 37.61, 66.69) --
	( 37.61, 66.69) --
	( 37.61, 66.69) --
	( 37.61, 66.69) --
	( 37.61, 66.69) --
	( 37.61, 66.69) --
	( 37.65, 66.65) --
	( 37.65, 66.65) --
	( 37.65, 66.65) --
	( 37.69, 66.62) --
	( 37.69, 66.62) --
	( 37.69, 66.62) --
	( 37.76, 66.57) --
	( 37.76, 66.57) --
	( 37.76, 66.57) --
	( 37.83, 66.52) --
	( 37.83, 66.52) --
	( 37.83, 66.52) --
	( 37.91, 66.46) --
	( 37.91, 66.46) --
	( 37.91, 66.46) --
	( 37.97, 66.40) --
	( 37.97, 66.40) --
	( 37.97, 66.40) --
	( 37.98, 66.39) --
	( 37.98, 66.39) --
	( 37.98, 66.39) --
	( 38.06, 66.38) --
	( 38.06, 66.38) --
	( 38.06, 66.38) --
	( 38.13, 66.33) --
	( 38.13, 66.33) --
	( 38.13, 66.33) --
	( 38.21, 66.28) --
	( 38.21, 66.28) --
	( 38.21, 66.28) --
	( 38.28, 66.26) --
	( 38.28, 66.26) --
	( 38.28, 66.26) --
	( 38.33, 66.23) --
	( 38.33, 66.23) --
	( 38.33, 66.23) --
	( 38.36, 66.22) --
	( 38.36, 66.22) --
	( 38.36, 66.22) --
	( 38.43, 66.17) --
	( 38.43, 66.17) --
	( 38.43, 66.17) --
	( 38.51, 66.15) --
	( 38.51, 66.15) --
	( 38.51, 66.15) --
	( 38.58, 66.12) --
	( 38.58, 66.12) --
	( 38.58, 66.12) --
	( 38.65, 66.11) --
	( 38.65, 66.11) --
	( 38.65, 66.11) --
	( 38.73, 66.09) --
	( 38.73, 66.09) --
	( 38.73, 66.09) --
	( 38.80, 66.05) --
	( 38.80, 66.05) --
	( 38.80, 66.05) --
	( 38.82, 66.05) --
	( 38.82, 66.05) --
	( 38.82, 66.05) --
	( 38.88, 66.02) --
	( 38.88, 66.02) --
	( 38.88, 66.02) --
	( 38.95, 66.03) --
	( 38.95, 66.03) --
	( 38.95, 66.03) --
	( 38.99, 66.00) --
	( 38.99, 66.00) --
	( 38.99, 66.00) --
	( 39.03, 65.97) --
	( 39.03, 65.97) --
	( 39.03, 65.97) --
	( 39.10, 65.98) --
	( 39.10, 65.98) --
	( 39.10, 65.98) --
	( 39.14, 65.97) --
	( 39.14, 65.97) --
	( 39.14, 65.97) --
	( 39.18, 66.01) --
	( 39.18, 66.01) --
	( 39.18, 66.01) --
	( 39.25, 66.09) --
	( 39.25, 66.09) --
	( 39.25, 66.09) --
	( 39.33, 66.18) --
	( 39.33, 66.18) --
	( 39.33, 66.18) --
	( 39.40, 66.29) --
	( 39.40, 66.29) --
	( 39.40, 66.29) --
	( 39.47, 66.35) --
	( 39.47, 66.35) --
	( 39.47, 66.35) --
	( 39.47, 66.27) --
	( 39.47, 66.27) --
	( 39.47, 66.27) --
	( 39.51, 66.03) --
	( 39.51, 66.03) --
	( 39.51, 66.03) --
	( 39.55, 66.19) --
	( 39.55, 66.19) --
	( 39.55, 66.19) --
	( 39.62, 66.42) --
	( 39.62, 66.42) --
	( 39.62, 66.42) --
	( 39.67, 66.59) --
	( 39.67, 66.59) --
	( 39.67, 66.59) --
	( 39.70, 66.97) --
	( 39.70, 66.97) --
	( 39.70, 66.97) --
	( 39.71, 67.11) --
	( 39.71, 67.11) --
	( 39.71, 67.11) --
	( 39.77, 66.96) --
	( 39.77, 66.96) --
	( 39.77, 66.96) --
	( 39.79, 66.91) --
	( 39.79, 66.91) --
	( 39.79, 66.91) --
	( 39.83, 66.32) --
	( 39.83, 66.32) --
	( 39.83, 66.32) --
	( 39.85, 66.43) --
	( 39.85, 66.43) --
	( 39.85, 66.43) --
	( 39.87, 66.56) --
	( 39.87, 66.56) --
	( 39.87, 66.56) --
	( 39.91, 66.12) --
	( 39.91, 66.12) --
	( 39.91, 66.12) --
	( 39.92, 66.10) --
	( 39.92, 66.10) --
	( 39.92, 66.10) --
	( 40.00, 66.04) --
	( 40.00, 66.04) --
	( 40.00, 66.04) --
	( 40.03, 66.00) --
	( 40.03, 66.00) --
	( 40.03, 66.00) --
	( 40.07, 65.98) --
	( 40.07, 65.98) --
	( 40.07, 65.98) --
	( 40.14, 65.97) --
	( 40.14, 65.97) --
	( 40.14, 65.97) --
	( 40.22, 65.98) --
	( 40.22, 65.98) --
	( 40.22, 65.98) --
	( 40.29, 65.98) --
	( 40.29, 65.98) --
	( 40.29, 65.98) --
	( 40.35, 65.95) --
	( 40.35, 65.95) --
	( 40.35, 65.95) --
	( 40.37, 65.95) --
	( 40.37, 65.95) --
	( 40.37, 65.95) --
	( 40.44, 65.93) --
	( 40.44, 65.93) --
	( 40.44, 65.93) --
	( 40.52, 65.97) --
	( 40.52, 65.97) --
	( 40.52, 65.97) --
	( 40.52, 65.97) --
	( 40.52, 65.97) --
	( 40.52, 65.97) --
	( 40.59, 65.96) --
	( 40.59, 65.96) --
	( 40.59, 65.96) --
	( 40.67, 65.95) --
	( 40.67, 65.95) --
	( 40.67, 65.95) --
	( 40.74, 65.96) --
	( 40.74, 65.96) --
	( 40.74, 65.96) --
	( 40.81, 65.98) --
	( 40.81, 65.98) --
	( 40.81, 65.98) --
	( 40.89, 65.95) --
	( 40.89, 65.95) --
	( 40.89, 65.95) --
	( 40.92, 65.95) --
	( 40.92, 65.95) --
	( 40.92, 65.95) --
	( 40.96, 65.96) --
	( 40.96, 65.96) --
	( 40.96, 65.96) --
	( 41.04, 65.97) --
	( 41.04, 65.97) --
	( 41.04, 65.97) --
	( 41.11, 65.97) --
	( 41.11, 65.97) --
	( 41.11, 65.97) --
	( 41.19, 65.99) --
	( 41.19, 65.99) --
	( 41.19, 65.99) --
	( 41.26, 65.99) --
	( 41.26, 65.99) --
	( 41.26, 65.99) --
	( 41.29, 66.00) --
	( 41.29, 66.00) --
	( 41.29, 66.00) --
	( 41.33, 66.02) --
	( 41.33, 66.02) --
	( 41.33, 66.02) --
	( 41.41, 66.03) --
	( 41.41, 66.03) --
	( 41.41, 66.03) --
	( 41.48, 66.02) --
	( 41.48, 66.02) --
	( 41.48, 66.02) --
	( 41.56, 66.04) --
	( 41.56, 66.04) --
	( 41.56, 66.04) --
	( 41.63, 66.08) --
	( 41.63, 66.08) --
	( 41.63, 66.08) --
	( 41.67, 66.08) --
	( 41.67, 66.08) --
	( 41.67, 66.08) --
	( 41.71, 66.07) --
	( 41.71, 66.07) --
	( 41.71, 66.07) --
	( 41.78, 66.08) --
	( 41.78, 66.08) --
	( 41.78, 66.08) --
	( 41.86, 66.11) --
	( 41.86, 66.11) --
	( 41.86, 66.11) --
	( 41.93, 66.12) --
	( 41.93, 66.12) --
	( 41.93, 66.12) --
	( 42.00, 66.19) --
	( 42.00, 66.19) --
	( 42.00, 66.19) --
	( 42.05, 66.16) --
	( 42.05, 66.16) --
	( 42.05, 66.16) --
	( 42.08, 66.15) --
	( 42.08, 66.15) --
	( 42.08, 66.15) --
	( 42.15, 66.24) --
	( 42.15, 66.24) --
	( 42.15, 66.24) --
	( 42.21, 66.26) --
	( 42.21, 66.26);
\definecolor{fillColor}{RGB}{163,165,0}

\path[fill=fillColor] (  4.64, 65.76) --
	(  4.64, 65.76) --
	(  4.71, 65.75) --
	(  4.71, 65.75) --
	(  4.71, 65.75) --
	(  4.79, 65.78) --
	(  4.79, 65.78) --
	(  4.79, 65.78) --
	(  4.86, 65.75) --
	(  4.86, 65.75) --
	(  4.86, 65.75) --
	(  4.94, 65.74) --
	(  4.94, 65.74) --
	(  4.94, 65.74) --
	(  5.02, 65.76) --
	(  5.02, 65.76) --
	(  5.02, 65.76) --
	(  5.09, 65.76) --
	(  5.09, 65.76) --
	(  5.09, 65.76) --
	(  5.17, 65.74) --
	(  5.17, 65.74) --
	(  5.17, 65.74) --
	(  5.25, 65.77) --
	(  5.25, 65.77) --
	(  5.25, 65.77) --
	(  5.32, 65.73) --
	(  5.32, 65.73) --
	(  5.32, 65.73) --
	(  5.40, 65.75) --
	(  5.40, 65.75) --
	(  5.40, 65.75) --
	(  5.47, 65.75) --
	(  5.47, 65.75) --
	(  5.47, 65.75) --
	(  5.55, 65.77) --
	(  5.55, 65.77) --
	(  5.55, 65.77) --
	(  5.63, 65.78) --
	(  5.63, 65.78) --
	(  5.63, 65.78) --
	(  5.70, 65.75) --
	(  5.70, 65.75) --
	(  5.70, 65.75) --
	(  5.78, 65.76) --
	(  5.78, 65.76) --
	(  5.78, 65.76) --
	(  5.86, 65.77) --
	(  5.86, 65.77) --
	(  5.86, 65.77) --
	(  5.93, 65.73) --
	(  5.93, 65.73) --
	(  5.93, 65.73) --
	(  6.01, 65.76) --
	(  6.01, 65.76) --
	(  6.01, 65.76) --
	(  6.09, 65.78) --
	(  6.09, 65.78) --
	(  6.09, 65.78) --
	(  6.16, 65.74) --
	(  6.16, 65.74) --
	(  6.16, 65.74) --
	(  6.24, 65.77) --
	(  6.24, 65.77) --
	(  6.24, 65.77) --
	(  6.31, 65.73) --
	(  6.31, 65.73) --
	(  6.31, 65.73) --
	(  6.39, 65.76) --
	(  6.39, 65.76) --
	(  6.39, 65.76) --
	(  6.47, 65.74) --
	(  6.47, 65.74) --
	(  6.47, 65.74) --
	(  6.54, 65.72) --
	(  6.54, 65.72) --
	(  6.54, 65.72) --
	(  6.62, 65.76) --
	(  6.62, 65.76) --
	(  6.62, 65.76) --
	(  6.69, 65.78) --
	(  6.69, 65.78) --
	(  6.69, 65.78) --
	(  6.77, 65.76) --
	(  6.77, 65.76) --
	(  6.77, 65.76) --
	(  6.85, 65.73) --
	(  6.85, 65.73) --
	(  6.85, 65.73) --
	(  6.92, 65.75) --
	(  6.92, 65.75) --
	(  6.92, 65.75) --
	(  7.00, 65.74) --
	(  7.00, 65.74) --
	(  7.00, 65.74) --
	(  7.08, 65.75) --
	(  7.08, 65.75) --
	(  7.08, 65.75) --
	(  7.15, 65.76) --
	(  7.15, 65.76) --
	(  7.15, 65.76) --
	(  7.23, 65.75) --
	(  7.23, 65.75) --
	(  7.23, 65.75) --
	(  7.30, 65.75) --
	(  7.30, 65.75) --
	(  7.30, 65.75) --
	(  7.38, 65.73) --
	(  7.38, 65.73) --
	(  7.38, 65.73) --
	(  7.46, 65.77) --
	(  7.46, 65.77) --
	(  7.46, 65.77) --
	(  7.53, 65.75) --
	(  7.53, 65.75) --
	(  7.53, 65.75) --
	(  7.61, 65.76) --
	(  7.61, 65.76) --
	(  7.61, 65.76) --
	(  7.68, 65.74) --
	(  7.68, 65.74) --
	(  7.68, 65.74) --
	(  7.76, 65.76) --
	(  7.76, 65.76) --
	(  7.76, 65.76) --
	(  7.84, 65.76) --
	(  7.84, 65.76) --
	(  7.84, 65.76) --
	(  7.91, 65.73) --
	(  7.91, 65.73) --
	(  7.91, 65.73) --
	(  7.99, 65.74) --
	(  7.99, 65.74) --
	(  7.99, 65.74) --
	(  8.07, 65.75) --
	(  8.07, 65.75) --
	(  8.07, 65.75) --
	(  8.14, 65.77) --
	(  8.14, 65.77) --
	(  8.14, 65.77) --
	(  8.22, 65.74) --
	(  8.22, 65.74) --
	(  8.22, 65.74) --
	(  8.29, 65.75) --
	(  8.29, 65.75) --
	(  8.29, 65.75) --
	(  8.37, 65.76) --
	(  8.37, 65.76) --
	(  8.37, 65.76) --
	(  8.45, 65.74) --
	(  8.45, 65.74) --
	(  8.45, 65.74) --
	(  8.52, 65.74) --
	(  8.52, 65.74) --
	(  8.52, 65.74) --
	(  8.60, 65.76) --
	(  8.60, 65.76) --
	(  8.60, 65.76) --
	(  8.67, 65.75) --
	(  8.67, 65.75) --
	(  8.67, 65.75) --
	(  8.75, 65.74) --
	(  8.75, 65.74) --
	(  8.75, 65.74) --
	(  8.83, 65.74) --
	(  8.83, 65.74) --
	(  8.83, 65.74) --
	(  8.90, 65.75) --
	(  8.90, 65.75) --
	(  8.90, 65.75) --
	(  8.98, 65.75) --
	(  8.98, 65.75) --
	(  8.98, 65.75) --
	(  9.05, 65.74) --
	(  9.05, 65.74) --
	(  9.05, 65.74) --
	(  9.13, 65.76) --
	(  9.13, 65.76) --
	(  9.13, 65.76) --
	(  9.21, 65.76) --
	(  9.21, 65.76) --
	(  9.21, 65.76) --
	(  9.28, 65.75) --
	(  9.28, 65.75) --
	(  9.28, 65.75) --
	(  9.36, 65.74) --
	(  9.36, 65.74) --
	(  9.36, 65.74) --
	(  9.43, 65.77) --
	(  9.43, 65.77) --
	(  9.43, 65.77) --
	(  9.51, 65.75) --
	(  9.51, 65.75) --
	(  9.51, 65.75) --
	(  9.59, 65.75) --
	(  9.59, 65.75) --
	(  9.59, 65.75) --
	(  9.66, 65.77) --
	(  9.66, 65.77) --
	(  9.66, 65.77) --
	(  9.74, 65.75) --
	(  9.74, 65.75) --
	(  9.74, 65.75) --
	(  9.82, 65.74) --
	(  9.82, 65.74) --
	(  9.82, 65.74) --
	(  9.89, 65.74) --
	(  9.89, 65.74) --
	(  9.89, 65.74) --
	(  9.97, 65.78) --
	(  9.97, 65.78) --
	(  9.97, 65.78) --
	( 10.04, 65.78) --
	( 10.04, 65.78) --
	( 10.04, 65.78) --
	( 10.12, 65.74) --
	( 10.12, 65.74) --
	( 10.12, 65.74) --
	( 10.19, 65.73) --
	( 10.19, 65.73) --
	( 10.19, 65.73) --
	( 10.27, 65.76) --
	( 10.27, 65.76) --
	( 10.27, 65.76) --
	( 10.35, 65.73) --
	( 10.35, 65.73) --
	( 10.35, 65.73) --
	( 10.42, 65.78) --
	( 10.42, 65.78) --
	( 10.42, 65.78) --
	( 10.50, 65.76) --
	( 10.50, 65.76) --
	( 10.50, 65.76) --
	( 10.58, 65.75) --
	( 10.58, 65.75) --
	( 10.58, 65.75) --
	( 10.65, 65.76) --
	( 10.65, 65.76) --
	( 10.65, 65.76) --
	( 10.73, 65.74) --
	( 10.73, 65.74) --
	( 10.73, 65.74) --
	( 10.80, 65.77) --
	( 10.80, 65.77) --
	( 10.80, 65.77) --
	( 10.88, 65.79) --
	( 10.88, 65.79) --
	( 10.88, 65.79) --
	( 10.96, 65.76) --
	( 10.96, 65.76) --
	( 10.96, 65.76) --
	( 11.03, 65.74) --
	( 11.03, 65.74) --
	( 11.03, 65.74) --
	( 11.11, 65.76) --
	( 11.11, 65.76) --
	( 11.11, 65.76) --
	( 11.18, 65.76) --
	( 11.18, 65.76) --
	( 11.18, 65.76) --
	( 11.26, 65.75) --
	( 11.26, 65.75) --
	( 11.26, 65.75) --
	( 11.33, 65.75) --
	( 11.33, 65.75) --
	( 11.33, 65.75) --
	( 11.41, 65.77) --
	( 11.41, 65.77) --
	( 11.41, 65.77) --
	( 11.49, 65.77) --
	( 11.49, 65.77) --
	( 11.49, 65.77) --
	( 11.56, 65.74) --
	( 11.56, 65.74) --
	( 11.56, 65.74) --
	( 11.64, 65.76) --
	( 11.64, 65.76) --
	( 11.64, 65.76) --
	( 11.71, 65.78) --
	( 11.71, 65.78) --
	( 11.71, 65.78) --
	( 11.79, 65.77) --
	( 11.79, 65.77) --
	( 11.79, 65.77) --
	( 11.87, 65.77) --
	( 11.87, 65.77) --
	( 11.87, 65.77) --
	( 11.94, 65.79) --
	( 11.94, 65.79) --
	( 11.94, 65.79) --
	( 12.02, 65.77) --
	( 12.02, 65.77) --
	( 12.02, 65.77) --
	( 12.09, 65.74) --
	( 12.09, 65.74) --
	( 12.09, 65.74) --
	( 12.17, 65.76) --
	( 12.17, 65.76) --
	( 12.17, 65.76) --
	( 12.25, 65.76) --
	( 12.25, 65.76) --
	( 12.25, 65.76) --
	( 12.32, 65.75) --
	( 12.32, 65.75) --
	( 12.32, 65.75) --
	( 12.40, 65.73) --
	( 12.40, 65.73) --
	( 12.40, 65.73) --
	( 12.47, 65.74) --
	( 12.47, 65.74) --
	( 12.47, 65.74) --
	( 12.55, 65.79) --
	( 12.55, 65.79) --
	( 12.55, 65.79) --
	( 12.63, 65.76) --
	( 12.63, 65.76) --
	( 12.63, 65.76) --
	( 12.70, 65.76) --
	( 12.70, 65.76) --
	( 12.70, 65.76) --
	( 12.78, 65.76) --
	( 12.78, 65.76) --
	( 12.78, 65.76) --
	( 12.85, 65.75) --
	( 12.85, 65.75) --
	( 12.85, 65.75) --
	( 12.93, 65.75) --
	( 12.93, 65.75) --
	( 12.93, 65.75) --
	( 13.00, 65.75) --
	( 13.00, 65.75) --
	( 13.00, 65.75) --
	( 13.08, 65.76) --
	( 13.08, 65.76) --
	( 13.08, 65.76) --
	( 13.16, 65.76) --
	( 13.16, 65.76) --
	( 13.16, 65.76) --
	( 13.23, 65.75) --
	( 13.23, 65.75) --
	( 13.23, 65.75) --
	( 13.31, 65.75) --
	( 13.31, 65.75) --
	( 13.31, 65.75) --
	( 13.38, 65.78) --
	( 13.38, 65.78) --
	( 13.38, 65.78) --
	( 13.46, 65.75) --
	( 13.46, 65.75) --
	( 13.46, 65.75) --
	( 13.54, 65.77) --
	( 13.54, 65.77) --
	( 13.54, 65.77) --
	( 13.61, 65.74) --
	( 13.61, 65.74) --
	( 13.61, 65.74) --
	( 13.69, 65.74) --
	( 13.69, 65.74) --
	( 13.69, 65.74) --
	( 13.76, 65.75) --
	( 13.76, 65.75) --
	( 13.76, 65.75) --
	( 13.84, 65.75) --
	( 13.84, 65.75) --
	( 13.84, 65.75) --
	( 13.92, 65.75) --
	( 13.92, 65.75) --
	( 13.92, 65.75) --
	( 13.99, 65.78) --
	( 13.99, 65.78) --
	( 13.99, 65.78) --
	( 14.07, 65.74) --
	( 14.07, 65.74) --
	( 14.07, 65.74) --
	( 14.14, 65.75) --
	( 14.14, 65.75) --
	( 14.14, 65.75) --
	( 14.22, 65.76) --
	( 14.22, 65.76) --
	( 14.22, 65.76) --
	( 14.29, 65.74) --
	( 14.29, 65.74) --
	( 14.29, 65.74) --
	( 14.37, 65.76) --
	( 14.37, 65.76) --
	( 14.37, 65.76) --
	( 14.45, 65.75) --
	( 14.45, 65.75) --
	( 14.45, 65.75) --
	( 14.52, 65.77) --
	( 14.52, 65.77) --
	( 14.52, 65.77) --
	( 14.60, 65.74) --
	( 14.60, 65.74) --
	( 14.60, 65.74) --
	( 14.67, 65.74) --
	( 14.67, 65.74) --
	( 14.67, 65.74) --
	( 14.75, 65.76) --
	( 14.75, 65.76) --
	( 14.75, 65.76) --
	( 14.82, 65.75) --
	( 14.82, 65.75) --
	( 14.82, 65.75) --
	( 14.90, 65.73) --
	( 14.90, 65.73) --
	( 14.90, 65.73) --
	( 14.98, 65.74) --
	( 14.98, 65.74) --
	( 14.98, 65.74) --
	( 15.05, 65.76) --
	( 15.05, 65.76) --
	( 15.05, 65.76) --
	( 15.13, 65.74) --
	( 15.13, 65.74) --
	( 15.13, 65.74) --
	( 15.20, 65.77) --
	( 15.20, 65.77) --
	( 15.20, 65.77) --
	( 15.28, 65.78) --
	( 15.28, 65.78) --
	( 15.28, 65.78) --
	( 15.36, 65.76) --
	( 15.36, 65.76) --
	( 15.36, 65.76) --
	( 15.43, 65.77) --
	( 15.43, 65.77) --
	( 15.43, 65.77) --
	( 15.51, 65.72) --
	( 15.51, 65.72) --
	( 15.51, 65.72) --
	( 15.58, 65.75) --
	( 15.58, 65.75) --
	( 15.58, 65.75) --
	( 15.66, 65.77) --
	( 15.66, 65.77) --
	( 15.66, 65.77) --
	( 15.74, 65.74) --
	( 15.74, 65.74) --
	( 15.74, 65.74) --
	( 15.81, 65.75) --
	( 15.81, 65.75) --
	( 15.81, 65.75) --
	( 15.89, 65.76) --
	( 15.89, 65.76) --
	( 15.89, 65.76) --
	( 15.96, 65.76) --
	( 15.96, 65.76) --
	( 15.96, 65.76) --
	( 16.04, 65.76) --
	( 16.04, 65.76) --
	( 16.04, 65.76) --
	( 16.11, 65.76) --
	( 16.11, 65.76) --
	( 16.11, 65.76) --
	( 16.19, 65.76) --
	( 16.19, 65.76) --
	( 16.19, 65.76) --
	( 16.26, 65.76) --
	( 16.26, 65.76) --
	( 16.26, 65.76) --
	( 16.34, 65.74) --
	( 16.34, 65.74) --
	( 16.34, 65.74) --
	( 16.42, 65.77) --
	( 16.42, 65.77) --
	( 16.42, 65.77) --
	( 16.49, 65.76) --
	( 16.49, 65.76) --
	( 16.49, 65.76) --
	( 16.57, 65.77) --
	( 16.57, 65.77) --
	( 16.57, 65.77) --
	( 16.64, 65.76) --
	( 16.64, 65.76) --
	( 16.64, 65.76) --
	( 16.72, 65.78) --
	( 16.72, 65.78) --
	( 16.72, 65.78) --
	( 16.79, 65.77) --
	( 16.79, 65.77) --
	( 16.79, 65.77) --
	( 16.87, 65.75) --
	( 16.87, 65.75) --
	( 16.87, 65.75) --
	( 16.95, 65.75) --
	( 16.95, 65.75) --
	( 16.95, 65.75) --
	( 17.02, 65.75) --
	( 17.02, 65.75) --
	( 17.02, 65.75) --
	( 17.10, 65.76) --
	( 17.10, 65.76) --
	( 17.10, 65.76) --
	( 17.17, 65.73) --
	( 17.17, 65.73) --
	( 17.17, 65.73) --
	( 17.25, 65.76) --
	( 17.25, 65.76) --
	( 17.25, 65.76) --
	( 17.32, 65.75) --
	( 17.32, 65.75) --
	( 17.32, 65.75) --
	( 17.40, 65.74) --
	( 17.40, 65.74) --
	( 17.40, 65.74) --
	( 17.48, 65.75) --
	( 17.48, 65.75) --
	( 17.48, 65.75) --
	( 17.55, 65.77) --
	( 17.55, 65.77) --
	( 17.55, 65.77) --
	( 17.63, 65.76) --
	( 17.63, 65.76) --
	( 17.63, 65.76) --
	( 17.70, 65.78) --
	( 17.70, 65.78) --
	( 17.70, 65.78) --
	( 17.78, 65.76) --
	( 17.78, 65.76) --
	( 17.78, 65.76) --
	( 17.85, 65.76) --
	( 17.85, 65.76) --
	( 17.85, 65.76) --
	( 17.93, 65.76) --
	( 17.93, 65.76) --
	( 17.93, 65.76) --
	( 18.01, 65.75) --
	( 18.01, 65.75) --
	( 18.01, 65.75) --
	( 18.08, 65.75) --
	( 18.08, 65.75) --
	( 18.08, 65.75) --
	( 18.16, 65.77) --
	( 18.16, 65.77) --
	( 18.16, 65.77) --
	( 18.23, 65.72) --
	( 18.23, 65.72) --
	( 18.23, 65.72) --
	( 18.31, 65.74) --
	( 18.31, 65.74) --
	( 18.31, 65.74) --
	( 18.38, 65.79) --
	( 18.38, 65.79) --
	( 18.38, 65.79) --
	( 18.46, 65.74) --
	( 18.46, 65.74) --
	( 18.46, 65.74) --
	( 18.53, 65.76) --
	( 18.53, 65.76) --
	( 18.53, 65.76) --
	( 18.61, 65.74) --
	( 18.61, 65.74) --
	( 18.61, 65.74) --
	( 18.69, 65.75) --
	( 18.69, 65.75) --
	( 18.69, 65.75) --
	( 18.76, 65.76) --
	( 18.76, 65.76) --
	( 18.76, 65.76) --
	( 18.84, 65.74) --
	( 18.84, 65.74) --
	( 18.84, 65.74) --
	( 18.91, 65.76) --
	( 18.91, 65.76) --
	( 18.91, 65.76) --
	( 18.99, 65.78) --
	( 18.99, 65.78) --
	( 18.99, 65.78) --
	( 19.06, 65.76) --
	( 19.06, 65.76) --
	( 19.06, 65.76) --
	( 19.14, 65.75) --
	( 19.14, 65.75) --
	( 19.14, 65.75) --
	( 19.22, 65.78) --
	( 19.22, 65.78) --
	( 19.22, 65.78) --
	( 19.29, 65.75) --
	( 19.29, 65.75) --
	( 19.29, 65.75) --
	( 19.37, 65.75) --
	( 19.37, 65.75) --
	( 19.37, 65.75) --
	( 19.44, 65.75) --
	( 19.44, 65.75) --
	( 19.44, 65.75) --
	( 19.52, 65.74) --
	( 19.52, 65.74) --
	( 19.52, 65.74) --
	( 19.59, 65.75) --
	( 19.59, 65.75) --
	( 19.59, 65.75) --
	( 19.67, 65.75) --
	( 19.67, 65.75) --
	( 19.67, 65.75) --
	( 19.74, 65.74) --
	( 19.74, 65.74) --
	( 19.74, 65.74) --
	( 19.82, 65.76) --
	( 19.82, 65.76) --
	( 19.82, 65.76) --
	( 19.89, 65.75) --
	( 19.89, 65.75) --
	( 19.90, 65.75) --
	( 19.97, 65.75) --
	( 19.97, 65.75) --
	( 19.97, 65.75) --
	( 20.05, 65.75) --
	( 20.05, 65.75) --
	( 20.05, 65.75) --
	( 20.12, 65.76) --
	( 20.12, 65.76) --
	( 20.12, 65.76) --
	( 20.20, 65.77) --
	( 20.20, 65.77) --
	( 20.20, 65.77) --
	( 20.27, 65.74) --
	( 20.27, 65.74) --
	( 20.27, 65.74) --
	( 20.35, 65.78) --
	( 20.35, 65.78) --
	( 20.35, 65.78) --
	( 20.42, 65.76) --
	( 20.42, 65.76) --
	( 20.42, 65.76) --
	( 20.50, 65.74) --
	( 20.50, 65.74) --
	( 20.50, 65.74) --
	( 20.57, 65.74) --
	( 20.57, 65.74) --
	( 20.57, 65.74) --
	( 20.65, 65.77) --
	( 20.65, 65.77) --
	( 20.65, 65.77) --
	( 20.72, 65.74) --
	( 20.72, 65.74) --
	( 20.73, 65.74) --
	( 20.80, 65.75) --
	( 20.80, 65.75) --
	( 20.80, 65.75) --
	( 20.88, 65.76) --
	( 20.88, 65.76) --
	( 20.88, 65.76) --
	( 20.95, 65.76) --
	( 20.95, 65.76) --
	( 20.95, 65.76) --
	( 21.03, 65.75) --
	( 21.03, 65.75) --
	( 21.03, 65.75) --
	( 21.10, 65.75) --
	( 21.10, 65.75) --
	( 21.10, 65.75) --
	( 21.18, 65.76) --
	( 21.18, 65.76) --
	( 21.18, 65.76) --
	( 21.25, 65.76) --
	( 21.25, 65.76) --
	( 21.25, 65.76) --
	( 21.33, 65.72) --
	( 21.33, 65.72) --
	( 21.33, 65.72) --
	( 21.40, 65.75) --
	( 21.40, 65.75) --
	( 21.40, 65.75) --
	( 21.48, 65.77) --
	( 21.48, 65.77) --
	( 21.48, 65.77) --
	( 21.55, 65.75) --
	( 21.56, 65.75) --
	( 21.56, 65.75) --
	( 21.63, 65.74) --
	( 21.63, 65.74) --
	( 21.63, 65.74) --
	( 21.71, 65.77) --
	( 21.71, 65.77) --
	( 21.71, 65.77) --
	( 21.78, 65.76) --
	( 21.78, 65.76) --
	( 21.78, 65.76) --
	( 21.86, 65.76) --
	( 21.86, 65.76) --
	( 21.86, 65.76) --
	( 21.93, 65.77) --
	( 21.93, 65.77) --
	( 21.93, 65.77) --
	( 22.01, 65.77) --
	( 22.01, 65.77) --
	( 22.01, 65.77) --
	( 22.08, 65.74) --
	( 22.08, 65.74) --
	( 22.08, 65.74) --
	( 22.16, 65.72) --
	( 22.16, 65.72) --
	( 22.16, 65.72) --
	( 22.23, 65.75) --
	( 22.23, 65.75) --
	( 22.23, 65.75) --
	( 22.31, 65.77) --
	( 22.31, 65.77) --
	( 22.31, 65.77) --
	( 22.38, 65.75) --
	( 22.39, 65.75) --
	( 22.39, 65.75) --
	( 22.46, 65.76) --
	( 22.46, 65.76) --
	( 22.46, 65.76) --
	( 22.54, 65.77) --
	( 22.54, 65.77) --
	( 22.54, 65.77) --
	( 22.61, 65.76) --
	( 22.61, 65.76) --
	( 22.61, 65.76) --
	( 22.69, 65.77) --
	( 22.69, 65.77) --
	( 22.69, 65.77) --
	( 22.76, 65.75) --
	( 22.76, 65.75) --
	( 22.76, 65.75) --
	( 22.84, 65.79) --
	( 22.84, 65.79) --
	( 22.84, 65.79) --
	( 22.91, 65.76) --
	( 22.91, 65.76) --
	( 22.91, 65.76) --
	( 22.99, 65.74) --
	( 22.99, 65.74) --
	( 22.99, 65.74) --
	( 23.06, 65.75) --
	( 23.06, 65.75) --
	( 23.06, 65.75) --
	( 23.14, 65.77) --
	( 23.14, 65.77) --
	( 23.14, 65.77) --
	( 23.21, 65.75) --
	( 23.22, 65.75) --
	( 23.22, 65.75) --
	( 23.29, 65.77) --
	( 23.29, 65.77) --
	( 23.29, 65.77) --
	( 23.37, 65.77) --
	( 23.37, 65.77) --
	( 23.37, 65.77) --
	( 23.44, 65.74) --
	( 23.44, 65.74) --
	( 23.44, 65.74) --
	( 23.52, 65.77) --
	( 23.52, 65.77) --
	( 23.52, 65.77) --
	( 23.59, 65.73) --
	( 23.59, 65.73) --
	( 23.59, 65.73) --
	( 23.67, 65.76) --
	( 23.67, 65.76) --
	( 23.67, 65.76) --
	( 23.74, 65.77) --
	( 23.74, 65.77) --
	( 23.74, 65.77) --
	( 23.82, 65.74) --
	( 23.82, 65.74) --
	( 23.82, 65.74) --
	( 23.89, 65.76) --
	( 23.89, 65.76) --
	( 23.89, 65.76) --
	( 23.97, 65.77) --
	( 23.97, 65.77) --
	( 23.97, 65.77) --
	( 24.04, 65.78) --
	( 24.04, 65.78) --
	( 24.04, 65.78) --
	( 24.12, 65.77) --
	( 24.12, 65.77) --
	( 24.12, 65.77) --
	( 24.19, 65.79) --
	( 24.19, 65.79) --
	( 24.19, 65.79) --
	( 24.27, 65.76) --
	( 24.27, 65.76) --
	( 24.27, 65.76) --
	( 24.27, 65.76) --
	( 24.27, 65.76) --
	( 24.27, 65.76) --
	( 24.34, 65.74) --
	( 24.34, 65.74) --
	( 24.34, 65.74) --
	( 24.42, 65.74) --
	( 24.42, 65.74) --
	( 24.42, 65.74) --
	( 24.42, 65.74) --
	( 24.42, 65.74) --
	( 24.42, 65.74) --
	( 24.50, 65.76) --
	( 24.50, 65.76) --
	( 24.50, 65.76) --
	( 24.51, 65.76) --
	( 24.51, 65.76) --
	( 24.51, 65.76) --
	( 24.55, 65.77) --
	( 24.55, 65.77) --
	( 24.55, 65.77) --
	( 24.57, 65.77) --
	( 24.57, 65.77) --
	( 24.57, 65.77) --
	( 24.63, 65.75) --
	( 24.63, 65.75) --
	( 24.63, 65.75) --
	( 24.65, 65.75) --
	( 24.65, 65.75) --
	( 24.65, 65.75) --
	( 24.72, 65.76) --
	( 24.72, 65.76) --
	( 24.72, 65.76) --
	( 24.75, 65.77) --
	( 24.75, 65.77) --
	( 24.75, 65.77) --
	( 24.80, 65.79) --
	( 24.80, 65.79) --
	( 24.80, 65.79) --
	( 24.83, 65.77) --
	( 24.83, 65.77) --
	( 24.83, 65.77) --
	( 24.87, 65.75) --
	( 24.87, 65.75) --
	( 24.87, 65.75) --
	( 24.95, 65.75) --
	( 24.95, 65.75) --
	( 24.95, 65.75) --
	( 25.02, 65.75) --
	( 25.02, 65.75) --
	( 25.02, 65.75) --
	( 25.10, 65.77) --
	( 25.10, 65.77) --
	( 25.10, 65.77) --
	( 25.12, 65.77) --
	( 25.12, 65.77) --
	( 25.12, 65.77) --
	( 25.17, 65.77) --
	( 25.17, 65.77) --
	( 25.17, 65.77) --
	( 25.25, 65.75) --
	( 25.25, 65.75) --
	( 25.25, 65.75) --
	( 25.32, 65.76) --
	( 25.32, 65.76) --
	( 25.32, 65.76) --
	( 25.40, 65.75) --
	( 25.40, 65.75) --
	( 25.40, 65.75) --
	( 25.47, 65.77) --
	( 25.47, 65.77) --
	( 25.47, 65.77) --
	( 25.55, 65.76) --
	( 25.55, 65.76) --
	( 25.55, 65.76) --
	( 25.62, 65.77) --
	( 25.62, 65.77) --
	( 25.62, 65.77) --
	( 25.70, 65.73) --
	( 25.70, 65.73) --
	( 25.70, 65.73) --
	( 25.77, 65.77) --
	( 25.77, 65.77) --
	( 25.77, 65.77) --
	( 25.80, 65.76) --
	( 25.80, 65.76) --
	( 25.80, 65.76) --
	( 25.85, 65.74) --
	( 25.85, 65.74) --
	( 25.85, 65.74) --
	( 25.92, 65.74) --
	( 25.92, 65.74) --
	( 25.92, 65.74) --
	( 26.00, 65.77) --
	( 26.00, 65.77) --
	( 26.00, 65.77) --
	( 26.07, 65.75) --
	( 26.07, 65.75) --
	( 26.07, 65.75) --
	( 26.15, 65.75) --
	( 26.15, 65.75) --
	( 26.15, 65.75) --
	( 26.22, 65.75) --
	( 26.22, 65.75) --
	( 26.22, 65.75) --
	( 26.30, 65.74) --
	( 26.30, 65.74) --
	( 26.30, 65.74) --
	( 26.38, 65.75) --
	( 26.38, 65.75) --
	( 26.38, 65.75) --
	( 26.45, 65.78) --
	( 26.45, 65.78) --
	( 26.45, 65.78) --
	( 26.53, 65.75) --
	( 26.53, 65.75) --
	( 26.53, 65.75) --
	( 26.60, 65.78) --
	( 26.60, 65.78) --
	( 26.60, 65.78) --
	( 26.68, 65.75) --
	( 26.68, 65.75) --
	( 26.68, 65.75) --
	( 26.75, 65.77) --
	( 26.75, 65.77) --
	( 26.75, 65.77) --
	( 26.83, 65.76) --
	( 26.83, 65.76) --
	( 26.83, 65.76) --
	( 26.90, 65.72) --
	( 26.90, 65.72) --
	( 26.90, 65.72) --
	( 26.98, 65.75) --
	( 26.98, 65.75) --
	( 26.98, 65.75) --
	( 27.05, 65.77) --
	( 27.05, 65.77) --
	( 27.05, 65.77) --
	( 27.06, 65.77) --
	( 27.06, 65.77) --
	( 27.06, 65.77) --
	( 27.13, 65.75) --
	( 27.13, 65.75) --
	( 27.13, 65.75) --
	( 27.20, 65.77) --
	( 27.20, 65.77) --
	( 27.20, 65.77) --
	( 27.28, 65.76) --
	( 27.28, 65.76) --
	( 27.28, 65.76) --
	( 27.35, 65.73) --
	( 27.35, 65.73) --
	( 27.35, 65.73) --
	( 27.39, 65.75) --
	( 27.39, 65.75) --
	( 27.39, 65.75) --
	( 27.43, 65.76) --
	( 27.43, 65.76) --
	( 27.43, 65.76) --
	( 27.50, 65.76) --
	( 27.50, 65.76) --
	( 27.50, 65.76) --
	( 27.58, 65.76) --
	( 27.58, 65.76) --
	( 27.58, 65.76) --
	( 27.65, 65.76) --
	( 27.65, 65.76) --
	( 27.65, 65.76) --
	( 27.66, 65.76) --
	( 27.66, 65.76) --
	( 27.66, 65.76) --
	( 27.73, 65.74) --
	( 27.73, 65.74) --
	( 27.73, 65.74) --
	( 27.80, 65.76) --
	( 27.80, 65.76) --
	( 27.80, 65.76) --
	( 27.88, 65.78) --
	( 27.88, 65.78) --
	( 27.88, 65.78) --
	( 27.95, 65.74) --
	( 27.95, 65.74) --
	( 27.95, 65.74) --
	( 28.03, 65.75) --
	( 28.03, 65.75) --
	( 28.03, 65.75) --
	( 28.03, 65.75) --
	( 28.03, 65.75) --
	( 28.03, 65.75) --
	( 28.10, 65.75) --
	( 28.10, 65.75) --
	( 28.10, 65.75) --
	( 28.18, 65.75) --
	( 28.18, 65.75) --
	( 28.18, 65.75) --
	( 28.19, 65.75) --
	( 28.19, 65.75) --
	( 28.19, 65.75) --
	( 28.25, 65.76) --
	( 28.25, 65.76) --
	( 28.25, 65.76) --
	( 28.33, 65.77) --
	( 28.33, 65.77) --
	( 28.33, 65.77) --
	( 28.39, 65.76) --
	( 28.39, 65.76) --
	( 28.39, 65.76) --
	( 28.40, 65.76) --
	( 28.40, 65.76) --
	( 28.40, 65.76) --
	( 28.48, 65.76) --
	( 28.48, 65.76) --
	( 28.48, 65.76) --
	( 28.56, 65.75) --
	( 28.56, 65.75) --
	( 28.56, 65.75) --
	( 28.63, 65.75) --
	( 28.63, 65.75) --
	( 28.63, 65.75) --
	( 28.71, 65.77) --
	( 28.71, 65.77) --
	( 28.71, 65.77) --
	( 28.78, 65.76) --
	( 28.78, 65.76) --
	( 28.78, 65.76) --
	( 28.86, 65.75) --
	( 28.86, 65.75) --
	( 28.86, 65.75) --
	( 28.93, 65.76) --
	( 28.93, 65.76) --
	( 28.93, 65.76) --
	( 29.01, 65.76) --
	( 29.01, 65.76) --
	( 29.01, 65.76) --
	( 29.08, 65.77) --
	( 29.08, 65.77) --
	( 29.08, 65.77) --
	( 29.16, 65.76) --
	( 29.16, 65.76) --
	( 29.16, 65.76) --
	( 29.20, 65.76) --
	( 29.20, 65.76) --
	( 29.20, 65.76) --
	( 29.23, 65.76) --
	( 29.23, 65.76) --
	( 29.23, 65.76) --
	( 29.31, 65.76) --
	( 29.31, 65.76) --
	( 29.31, 65.76) --
	( 29.38, 65.76) --
	( 29.38, 65.76) --
	( 29.38, 65.76) --
	( 29.40, 65.76) --
	( 29.40, 65.76) --
	( 29.40, 65.76) --
	( 29.46, 65.77) --
	( 29.46, 65.77) --
	( 29.46, 65.77) --
	( 29.53, 65.78) --
	( 29.53, 65.78) --
	( 29.53, 65.78) --
	( 29.61, 65.75) --
	( 29.61, 65.75) --
	( 29.61, 65.75) --
	( 29.68, 65.75) --
	( 29.68, 65.75) --
	( 29.68, 65.75) --
	( 29.76, 65.76) --
	( 29.76, 65.76) --
	( 29.76, 65.76) --
	( 29.83, 65.75) --
	( 29.83, 65.75) --
	( 29.83, 65.75) --
	( 29.91, 65.77) --
	( 29.91, 65.77) --
	( 29.91, 65.77) --
	( 29.93, 65.76) --
	( 29.93, 65.76) --
	( 29.93, 65.76) --
	( 29.98, 65.75) --
	( 29.98, 65.75) --
	( 29.98, 65.75) --
	( 30.06, 65.77) --
	( 30.06, 65.77) --
	( 30.06, 65.77) --
	( 30.13, 65.76) --
	( 30.13, 65.76) --
	( 30.13, 65.76) --
	( 30.21, 65.73) --
	( 30.21, 65.73) --
	( 30.21, 65.73) --
	( 30.28, 65.77) --
	( 30.28, 65.77) --
	( 30.28, 65.77) --
	( 30.36, 65.77) --
	( 30.36, 65.77) --
	( 30.36, 65.77) --
	( 30.41, 65.75) --
	( 30.41, 65.75) --
	( 30.41, 65.75) --
	( 30.43, 65.75) --
	( 30.43, 65.75) --
	( 30.43, 65.75) --
	( 30.51, 65.76) --
	( 30.51, 65.76) --
	( 30.51, 65.76) --
	( 30.58, 65.77) --
	( 30.58, 65.77) --
	( 30.58, 65.77) --
	( 30.66, 65.78) --
	( 30.66, 65.78) --
	( 30.66, 65.78) --
	( 30.73, 65.76) --
	( 30.73, 65.76) --
	( 30.73, 65.76) --
	( 30.73, 65.76) --
	( 30.73, 65.76) --
	( 30.73, 65.76) --
	( 30.81, 65.75) --
	( 30.81, 65.75) --
	( 30.81, 65.75) --
	( 30.88, 65.76) --
	( 30.88, 65.76) --
	( 30.88, 65.76) --
	( 30.94, 65.77) --
	( 30.94, 65.77) --
	( 30.94, 65.77) --
	( 30.96, 65.77) --
	( 30.96, 65.77) --
	( 30.96, 65.77) --
	( 31.03, 65.76) --
	( 31.03, 65.76) --
	( 31.03, 65.76) --
	( 31.10, 65.77) --
	( 31.10, 65.77) --
	( 31.10, 65.77) --
	( 31.14, 65.77) --
	( 31.14, 65.77) --
	( 31.14, 65.77) --
	( 31.18, 65.76) --
	( 31.18, 65.76) --
	( 31.18, 65.76) --
	( 31.26, 65.73) --
	( 31.26, 65.73) --
	( 31.26, 65.73) --
	( 31.30, 65.74) --
	( 31.30, 65.74) --
	( 31.30, 65.74) --
	( 31.33, 65.74) --
	( 31.33, 65.74) --
	( 31.33, 65.74) --
	( 31.41, 65.78) --
	( 31.41, 65.78) --
	( 31.41, 65.78) --
	( 31.42, 65.78) --
	( 31.42, 65.78) --
	( 31.42, 65.78) --
	( 31.46, 65.77) --
	( 31.46, 65.77) --
	( 31.46, 65.77) --
	( 31.48, 65.77) --
	( 31.48, 65.77) --
	( 31.48, 65.77) --
	( 31.56, 65.75) --
	( 31.56, 65.75) --
	( 31.56, 65.75) --
	( 31.58, 65.74) --
	( 31.58, 65.74) --
	( 31.58, 65.74) --
	( 31.63, 65.73) --
	( 31.63, 65.73) --
	( 31.63, 65.73) --
	( 31.70, 65.78) --
	( 31.70, 65.78) --
	( 31.70, 65.78) --
	( 31.71, 65.78) --
	( 31.71, 65.78) --
	( 31.71, 65.78) --
	( 31.74, 65.76) --
	( 31.74, 65.76) --
	( 31.74, 65.76) --
	( 31.78, 65.75) --
	( 31.78, 65.75) --
	( 31.78, 65.75) --
	( 31.85, 65.76) --
	( 31.85, 65.76) --
	( 31.85, 65.76) --
	( 31.87, 65.76) --
	( 31.87, 65.76) --
	( 31.87, 65.76) --
	( 31.93, 65.77) --
	( 31.93, 65.77) --
	( 31.93, 65.77) --
	( 31.95, 65.76) --
	( 31.95, 65.76) --
	( 31.95, 65.76) --
	( 32.00, 65.75) --
	( 32.00, 65.75) --
	( 32.00, 65.75) --
	( 32.08, 65.75) --
	( 32.08, 65.75) --
	( 32.08, 65.75) --
	( 32.11, 65.75) --
	( 32.11, 65.75) --
	( 32.11, 65.75) --
	( 32.15, 65.74) --
	( 32.15, 65.74) --
	( 32.15, 65.74) --
	( 32.23, 65.76) --
	( 32.23, 65.76) --
	( 32.23, 65.76) --
	( 32.23, 65.76) --
	( 32.23, 65.76) --
	( 32.23, 65.76) --
	( 32.27, 65.77) --
	( 32.27, 65.77) --
	( 32.27, 65.77) --
	( 32.30, 65.77) --
	( 32.30, 65.77) --
	( 32.30, 65.77) --
	( 32.38, 65.75) --
	( 32.38, 65.75) --
	( 32.38, 65.75) --
	( 32.39, 65.75) --
	( 32.39, 65.75) --
	( 32.39, 65.75) --
	( 32.43, 65.75) --
	( 32.43, 65.75) --
	( 32.43, 65.75) --
	( 32.45, 65.75) --
	( 32.45, 65.75) --
	( 32.45, 65.75) --
	( 32.47, 65.75) --
	( 32.47, 65.75) --
	( 32.47, 65.75) --
	( 32.47, 65.75) --
	( 32.47, 65.75) --
	( 32.47, 65.75) --
	( 32.53, 65.77) --
	( 32.53, 65.77) --
	( 32.53, 65.77) --
	( 32.55, 65.76) --
	( 32.55, 65.76) --
	( 32.55, 65.76) --
	( 32.59, 65.76) --
	( 32.59, 65.76) --
	( 32.59, 65.76) --
	( 32.60, 65.76) --
	( 32.60, 65.76) --
	( 32.60, 65.76) --
	( 32.67, 65.76) --
	( 32.67, 65.76) --
	( 32.67, 65.76) --
	( 32.68, 65.76) --
	( 32.68, 65.76) --
	( 32.68, 65.76) --
	( 32.71, 65.76) --
	( 32.71, 65.76) --
	( 32.71, 65.76) --
	( 32.75, 65.77) --
	( 32.75, 65.77) --
	( 32.75, 65.77) --
	( 32.83, 65.78) --
	( 32.83, 65.78) --
	( 32.83, 65.78) --
	( 32.90, 65.75) --
	( 32.90, 65.75) --
	( 32.90, 65.75) --
	( 32.96, 65.75) --
	( 32.96, 65.75) --
	( 32.96, 65.75) --
	( 32.98, 65.75) --
	( 32.98, 65.75) --
	( 32.98, 65.75) --
	( 33.00, 65.75) --
	( 33.00, 65.75) --
	( 33.00, 65.75) --
	( 33.05, 65.76) --
	( 33.05, 65.76) --
	( 33.05, 65.76) --
	( 33.08, 65.76) --
	( 33.08, 65.76) --
	( 33.08, 65.76) --
	( 33.12, 65.76) --
	( 33.12, 65.76) --
	( 33.12, 65.76) --
	( 33.13, 65.76) --
	( 33.13, 65.76) --
	( 33.13, 65.76) --
	( 33.16, 65.76) --
	( 33.16, 65.76) --
	( 33.16, 65.76) --
	( 33.20, 65.77) --
	( 33.20, 65.77) --
	( 33.20, 65.77) --
	( 33.20, 65.77) --
	( 33.20, 65.77) --
	( 33.20, 65.77) --
	( 33.28, 65.77) --
	( 33.28, 65.77) --
	( 33.28, 65.77) --
	( 33.28, 65.77) --
	( 33.28, 65.77) --
	( 33.28, 65.77) --
	( 33.35, 65.76) --
	( 33.35, 65.76) --
	( 33.35, 65.76) --
	( 33.43, 65.77) --
	( 33.43, 65.77) --
	( 33.43, 65.77) --
	( 33.44, 65.76) --
	( 33.44, 65.76) --
	( 33.44, 65.76) --
	( 33.50, 65.74) --
	( 33.50, 65.74) --
	( 33.50, 65.74) --
	( 33.58, 65.77) --
	( 33.58, 65.77) --
	( 33.58, 65.77) --
	( 33.64, 65.79) --
	( 33.64, 65.79) --
	( 33.64, 65.79) --
	( 33.65, 65.80) --
	( 33.65, 65.80) --
	( 33.65, 65.80) --
	( 33.73, 65.76) --
	( 33.73, 65.76) --
	( 33.73, 65.76) --
	( 33.80, 65.74) --
	( 33.80, 65.74) --
	( 33.80, 65.74) --
	( 33.85, 65.75) --
	( 33.85, 65.75) --
	( 33.85, 65.75) --
	( 33.88, 65.75) --
	( 33.88, 65.75) --
	( 33.88, 65.75) --
	( 33.89, 65.75) --
	( 33.89, 65.75) --
	( 33.89, 65.75) --
	( 33.95, 65.75) --
	( 33.95, 65.75) --
	( 33.95, 65.75) --
	( 34.03, 65.75) --
	( 34.03, 65.75) --
	( 34.03, 65.75) --
	( 34.05, 65.75) --
	( 34.05, 65.75) --
	( 34.05, 65.75) --
	( 34.09, 65.76) --
	( 34.09, 65.76) --
	( 34.09, 65.76) --
	( 34.10, 65.76) --
	( 34.10, 65.76) --
	( 34.10, 65.76) --
	( 34.17, 65.80) --
	( 34.17, 65.80) --
	( 34.17, 65.80) --
	( 34.21, 65.78) --
	( 34.21, 65.78) --
	( 34.21, 65.78) --
	( 34.25, 65.76) --
	( 34.25, 65.76) --
	( 34.25, 65.76) --
	( 34.25, 65.76) --
	( 34.25, 65.76) --
	( 34.25, 65.76) --
	( 34.32, 65.75) --
	( 34.32, 65.75) --
	( 34.32, 65.75) --
	( 34.40, 65.75) --
	( 34.40, 65.75) --
	( 34.40, 65.75) --
	( 34.41, 65.75) --
	( 34.41, 65.75) --
	( 34.41, 65.75) --
	( 34.47, 65.78) --
	( 34.47, 65.78) --
	( 34.47, 65.78) --
	( 34.53, 65.74) --
	( 34.53, 65.74) --
	( 34.53, 65.74) --
	( 34.55, 65.74) --
	( 34.55, 65.74) --
	( 34.55, 65.74) --
	( 34.61, 65.75) --
	( 34.61, 65.75) --
	( 34.61, 65.75) --
	( 34.62, 65.75) --
	( 34.62, 65.75) --
	( 34.62, 65.75) --
	( 34.70, 65.74) --
	( 34.70, 65.74) --
	( 34.70, 65.74) --
	( 34.70, 65.74) --
	( 34.70, 65.74) --
	( 34.70, 65.74) --
	( 34.74, 65.75) --
	( 34.74, 65.75) --
	( 34.74, 65.75) --
	( 34.77, 65.75) --
	( 34.77, 65.75) --
	( 34.77, 65.75) --
	( 34.82, 65.77) --
	( 34.82, 65.77) --
	( 34.82, 65.77) --
	( 34.85, 65.78) --
	( 34.85, 65.78) --
	( 34.85, 65.78) --
	( 34.86, 65.77) --
	( 34.86, 65.77) --
	( 34.86, 65.77) --
	( 34.87, 65.77) --
	( 34.87, 65.77) --
	( 34.87, 65.77) --
	( 34.92, 65.73) --
	( 34.92, 65.73) --
	( 34.92, 65.73) --
	( 34.98, 65.76) --
	( 34.98, 65.76) --
	( 34.98, 65.76) --
	( 35.00, 65.76) --
	( 35.00, 65.76) --
	( 35.00, 65.76) --
	( 35.06, 65.76) --
	( 35.06, 65.76) --
	( 35.06, 65.76) --
	( 35.07, 65.76) --
	( 35.07, 65.76) --
	( 35.07, 65.76) --
	( 35.15, 65.74) --
	( 35.15, 65.74) --
	( 35.15, 65.74) --
	( 35.18, 65.74) --
	( 35.18, 65.74) --
	( 35.18, 65.74) --
	( 35.22, 65.75) --
	( 35.22, 65.75) --
	( 35.22, 65.75) --
	( 35.30, 65.77) --
	( 35.30, 65.77) --
	( 35.30, 65.77) --
	( 35.34, 65.77) --
	( 35.34, 65.77) --
	( 35.34, 65.77) --
	( 35.37, 65.77) --
	( 35.37, 65.77) --
	( 35.37, 65.77) --
	( 35.38, 65.77) --
	( 35.38, 65.77) --
	( 35.38, 65.77) --
	( 35.45, 65.76) --
	( 35.45, 65.76) --
	( 35.45, 65.76) --
	( 35.50, 65.74) --
	( 35.50, 65.74) --
	( 35.50, 65.74) --
	( 35.52, 65.73) --
	( 35.52, 65.73) --
	( 35.52, 65.73) --
	( 35.58, 65.77) --
	( 35.58, 65.77) --
	( 35.58, 65.77) --
	( 35.60, 65.77) --
	( 35.60, 65.77) --
	( 35.60, 65.77) --
	( 35.67, 65.78) --
	( 35.67, 65.78) --
	( 35.67, 65.78) --
	( 35.71, 65.76) --
	( 35.71, 65.76) --
	( 35.71, 65.76) --
	( 35.74, 65.75) --
	( 35.74, 65.75) --
	( 35.74, 65.75) --
	( 35.82, 65.77) --
	( 35.82, 65.77) --
	( 35.82, 65.77) --
	( 35.82, 65.76) --
	( 35.82, 65.76) --
	( 35.82, 65.76) --
	( 35.83, 65.76) --
	( 35.83, 65.76) --
	( 35.83, 65.76) --
	( 35.87, 65.76) --
	( 35.87, 65.76) --
	( 35.87, 65.76) --
	( 35.89, 65.75) --
	( 35.89, 65.75) --
	( 35.89, 65.75) --
	( 35.97, 65.74) --
	( 35.97, 65.74) --
	( 35.97, 65.74) --
	( 36.04, 65.75) --
	( 36.04, 65.75) --
	( 36.04, 65.75) --
	( 36.11, 65.77) --
	( 36.11, 65.77) --
	( 36.11, 65.77) --
	( 36.12, 65.77) --
	( 36.12, 65.77) --
	( 36.12, 65.77) --
	( 36.15, 65.76) --
	( 36.15, 65.76) --
	( 36.15, 65.76) --
	( 36.19, 65.75) --
	( 36.19, 65.75) --
	( 36.19, 65.75) --
	( 36.23, 65.76) --
	( 36.23, 65.76) --
	( 36.23, 65.76) --
	( 36.27, 65.76) --
	( 36.27, 65.76) --
	( 36.27, 65.76) --
	( 36.34, 65.76) --
	( 36.34, 65.76) --
	( 36.34, 65.76) --
	( 36.42, 65.76) --
	( 36.42, 65.76) --
	( 36.42, 65.76) --
	( 36.43, 65.76) --
	( 36.43, 65.76) --
	( 36.43, 65.76) --
	( 36.49, 65.77) --
	( 36.49, 65.77) --
	( 36.49, 65.77) --
	( 36.57, 65.75) --
	( 36.57, 65.75) --
	( 36.57, 65.75) --
	( 36.64, 65.78) --
	( 36.64, 65.78) --
	( 36.64, 65.78) --
	( 36.71, 65.77) --
	( 36.71, 65.77) --
	( 36.71, 65.77) --
	( 36.76, 65.76) --
	( 36.76, 65.76) --
	( 36.76, 65.76) --
	( 36.79, 65.75) --
	( 36.79, 65.75) --
	( 36.79, 65.75) --
	( 36.86, 65.77) --
	( 36.86, 65.77) --
	( 36.86, 65.77) --
	( 36.92, 65.75) --
	( 36.92, 65.75) --
	( 36.92, 65.75) --
	( 36.94, 65.75) --
	( 36.94, 65.75) --
	( 36.94, 65.75) --
	( 37.01, 65.75) --
	( 37.01, 65.75) --
	( 37.01, 65.75) --
	( 37.09, 65.76) --
	( 37.09, 65.76) --
	( 37.09, 65.76) --
	( 37.16, 65.75) --
	( 37.16, 65.75) --
	( 37.16, 65.75) --
	( 37.16, 65.75) --
	( 37.16, 65.75) --
	( 37.16, 65.75) --
	( 37.24, 65.78) --
	( 37.24, 65.78) --
	( 37.24, 65.78) --
	( 37.31, 65.78) --
	( 37.31, 65.78) --
	( 37.31, 65.78) --
	( 37.32, 65.78) --
	( 37.32, 65.78) --
	( 37.32, 65.78) --
	( 37.39, 65.77) --
	( 37.39, 65.77) --
	( 37.39, 65.77) --
	( 37.46, 65.76) --
	( 37.46, 65.76) --
	( 37.46, 65.76) --
	( 37.54, 65.78) --
	( 37.54, 65.78) --
	( 37.54, 65.78) --
	( 37.61, 65.77) --
	( 37.61, 65.77) --
	( 37.61, 65.77) --
	( 37.61, 65.77) --
	( 37.61, 65.77) --
	( 37.61, 65.77) --
	( 37.65, 65.76) --
	( 37.65, 65.76) --
	( 37.65, 65.76) --
	( 37.69, 65.76) --
	( 37.69, 65.76) --
	( 37.69, 65.76) --
	( 37.76, 65.78) --
	( 37.76, 65.78) --
	( 37.76, 65.78) --
	( 37.83, 65.78) --
	( 37.83, 65.78) --
	( 37.83, 65.78) --
	( 37.91, 65.78) --
	( 37.91, 65.78) --
	( 37.91, 65.78) --
	( 37.97, 65.77) --
	( 37.97, 65.77) --
	( 37.97, 65.77) --
	( 37.98, 65.77) --
	( 37.98, 65.77) --
	( 37.98, 65.77) --
	( 38.06, 65.79) --
	( 38.06, 65.79) --
	( 38.06, 65.79) --
	( 38.13, 65.78) --
	( 38.13, 65.78) --
	( 38.13, 65.78) --
	( 38.21, 65.76) --
	( 38.21, 65.76) --
	( 38.21, 65.76) --
	( 38.28, 65.77) --
	( 38.28, 65.77) --
	( 38.28, 65.77) --
	( 38.33, 65.77) --
	( 38.33, 65.77) --
	( 38.33, 65.77) --
	( 38.36, 65.77) --
	( 38.36, 65.77) --
	( 38.36, 65.77) --
	( 38.43, 65.74) --
	( 38.43, 65.74) --
	( 38.43, 65.74) --
	( 38.51, 65.75) --
	( 38.51, 65.75) --
	( 38.51, 65.75) --
	( 38.58, 65.75) --
	( 38.58, 65.75) --
	( 38.58, 65.75) --
	( 38.65, 65.76) --
	( 38.65, 65.76) --
	( 38.65, 65.76) --
	( 38.73, 65.77) --
	( 38.73, 65.77) --
	( 38.73, 65.77) --
	( 38.80, 65.76) --
	( 38.80, 65.76) --
	( 38.80, 65.76) --
	( 38.82, 65.76) --
	( 38.82, 65.76) --
	( 38.82, 65.76) --
	( 38.88, 65.75) --
	( 38.88, 65.75) --
	( 38.88, 65.75) --
	( 38.95, 65.78) --
	( 38.95, 65.78) --
	( 38.95, 65.78) --
	( 38.99, 65.76) --
	( 38.99, 65.76) --
	( 38.99, 65.76) --
	( 39.03, 65.74) --
	( 39.03, 65.74) --
	( 39.03, 65.74) --
	( 39.10, 65.78) --
	( 39.10, 65.78) --
	( 39.10, 65.78) --
	( 39.14, 65.78) --
	( 39.14, 65.78) --
	( 39.14, 65.78) --
	( 39.18, 65.78) --
	( 39.18, 65.78) --
	( 39.18, 65.78) --
	( 39.25, 65.77) --
	( 39.25, 65.77) --
	( 39.25, 65.77) --
	( 39.33, 65.77) --
	( 39.33, 65.77) --
	( 39.33, 65.77) --
	( 39.40, 65.79) --
	( 39.40, 65.79) --
	( 39.40, 65.79) --
	( 39.47, 65.77) --
	( 39.47, 65.77) --
	( 39.47, 65.77) --
	( 39.47, 65.76) --
	( 39.47, 65.76) --
	( 39.47, 65.76) --
	( 39.51, 65.77) --
	( 39.51, 65.77) --
	( 39.51, 65.77) --
	( 39.55, 65.77) --
	( 39.55, 65.77) --
	( 39.55, 65.77) --
	( 39.62, 65.75) --
	( 39.62, 65.75) --
	( 39.62, 65.75) --
	( 39.67, 65.76) --
	( 39.67, 65.76) --
	( 39.67, 65.76) --
	( 39.70, 65.77) --
	( 39.70, 65.77) --
	( 39.70, 65.77) --
	( 39.71, 65.78) --
	( 39.71, 65.78) --
	( 39.71, 65.78) --
	( 39.77, 65.80) --
	( 39.77, 65.80) --
	( 39.77, 65.80) --
	( 39.79, 65.80) --
	( 39.79, 65.80) --
	( 39.79, 65.80) --
	( 39.83, 65.79) --
	( 39.83, 65.79) --
	( 39.83, 65.79) --
	( 39.85, 65.79) --
	( 39.85, 65.79) --
	( 39.85, 65.79) --
	( 39.87, 65.78) --
	( 39.87, 65.78) --
	( 39.87, 65.78) --
	( 39.91, 65.78) --
	( 39.91, 65.78) --
	( 39.91, 65.78) --
	( 39.92, 65.78) --
	( 39.92, 65.78) --
	( 39.92, 65.78) --
	( 40.00, 65.79) --
	( 40.00, 65.79) --
	( 40.00, 65.79) --
	( 40.03, 65.78) --
	( 40.03, 65.78) --
	( 40.03, 65.78) --
	( 40.07, 65.77) --
	( 40.07, 65.77) --
	( 40.07, 65.77) --
	( 40.14, 65.77) --
	( 40.14, 65.77) --
	( 40.14, 65.77) --
	( 40.22, 65.79) --
	( 40.22, 65.79) --
	( 40.22, 65.79) --
	( 40.29, 65.80) --
	( 40.29, 65.80) --
	( 40.29, 65.80) --
	( 40.35, 65.78) --
	( 40.35, 65.78) --
	( 40.35, 65.78) --
	( 40.37, 65.78) --
	( 40.37, 65.78) --
	( 40.37, 65.78) --
	( 40.44, 65.76) --
	( 40.44, 65.76) --
	( 40.44, 65.76) --
	( 40.52, 65.80) --
	( 40.52, 65.80) --
	( 40.52, 65.80) --
	( 40.52, 65.80) --
	( 40.52, 65.80) --
	( 40.52, 65.80) --
	( 40.59, 65.81) --
	( 40.59, 65.81) --
	( 40.59, 65.81) --
	( 40.67, 65.79) --
	( 40.67, 65.79) --
	( 40.67, 65.79) --
	( 40.74, 65.81) --
	( 40.74, 65.81) --
	( 40.74, 65.81) --
	( 40.81, 65.83) --
	( 40.81, 65.83) --
	( 40.81, 65.83) --
	( 40.89, 65.80) --
	( 40.89, 65.80) --
	( 40.89, 65.80) --
	( 40.92, 65.81) --
	( 40.92, 65.81) --
	( 40.92, 65.81) --
	( 40.96, 65.82) --
	( 40.96, 65.82) --
	( 40.96, 65.82) --
	( 41.04, 65.83) --
	( 41.04, 65.83) --
	( 41.04, 65.83) --
	( 41.11, 65.82) --
	( 41.11, 65.82) --
	( 41.11, 65.82) --
	( 41.19, 65.84) --
	( 41.19, 65.84) --
	( 41.19, 65.84) --
	( 41.26, 65.84) --
	( 41.26, 65.84) --
	( 41.26, 65.84) --
	( 41.29, 65.85) --
	( 41.29, 65.85) --
	( 41.29, 65.85) --
	( 41.33, 65.87) --
	( 41.33, 65.87) --
	( 41.33, 65.87) --
	( 41.41, 65.87) --
	( 41.41, 65.87) --
	( 41.41, 65.87) --
	( 41.48, 65.86) --
	( 41.48, 65.86) --
	( 41.48, 65.86) --
	( 41.56, 65.88) --
	( 41.56, 65.88) --
	( 41.56, 65.88) --
	( 41.63, 65.92) --
	( 41.63, 65.92) --
	( 41.63, 65.92) --
	( 41.67, 65.92) --
	( 41.67, 65.92) --
	( 41.67, 65.92) --
	( 41.71, 65.91) --
	( 41.71, 65.91) --
	( 41.71, 65.91) --
	( 41.78, 65.92) --
	( 41.78, 65.92) --
	( 41.78, 65.92) --
	( 41.86, 65.95) --
	( 41.86, 65.95) --
	( 41.86, 65.95) --
	( 41.93, 65.95) --
	( 41.93, 65.95) --
	( 41.93, 65.95) --
	( 42.00, 66.02) --
	( 42.00, 66.02) --
	( 42.00, 66.02) --
	( 42.05, 66.00) --
	( 42.05, 66.00) --
	( 42.05, 66.00) --
	( 42.08, 65.98) --
	( 42.08, 65.98) --
	( 42.08, 65.98) --
	( 42.15, 66.07) --
	( 42.15, 66.07) --
	( 42.15, 66.07) --
	( 42.21, 66.09) --
	( 42.21, 66.09) --
	( 42.21, 66.09) --
	( 42.23, 66.09) --
	( 42.23, 66.09) --
	( 42.23, 66.09) --
	( 42.30, 66.10) --
	( 42.30, 66.10) --
	( 42.30, 66.10) --
	( 42.38, 66.15) --
	( 42.38, 66.15) --
	( 42.38, 66.15) --
	( 42.45, 66.20) --
	( 42.45, 66.20) --
	( 42.45, 66.20) --
	( 42.52, 66.20) --
	( 42.52, 66.20) --
	( 42.52, 66.20) --
	( 42.53, 66.20) --
	( 42.53, 66.20) --
	( 42.53, 66.20) --
	( 42.60, 66.22) --
	( 42.60, 66.22) --
	( 42.60, 66.22) --
	( 42.67, 66.31) --
	( 42.67, 66.31) --
	( 42.67, 66.31) --
	( 42.75, 66.37) --
	( 42.75, 66.37) --
	( 42.75, 66.37) --
	( 42.82, 66.41) --
	( 42.82, 66.41) --
	( 42.82, 66.41) --
	( 42.82, 66.41) --
	( 42.82, 66.41) --
	( 42.82, 66.41) --
	( 42.90, 66.48) --
	( 42.90, 66.48) --
	( 42.90, 66.48) --
	( 42.97, 66.53) --
	( 42.97, 66.53) --
	( 42.97, 66.53) --
	( 43.01, 66.61) --
	( 43.01, 66.61) --
	( 43.01, 66.61) --
	( 43.04, 66.67) --
	( 43.04, 66.67) --
	( 43.04, 66.67) --
	( 43.12, 66.71) --
	( 43.12, 66.71) --
	( 43.12, 66.71) --
	( 43.19, 66.72) --
	( 43.19, 66.72) --
	( 43.19, 66.72) --
	( 43.27, 66.89) --
	( 43.27, 66.89) --
	( 43.27, 66.89) --
	( 43.30, 66.93) --
	( 43.30, 66.93) --
	( 43.30, 66.93) --
	( 43.34, 66.98) --
	( 43.34, 66.98) --
	( 43.34, 66.98) --
	( 43.42, 67.08) --
	( 43.42, 67.08) --
	( 43.42, 67.08) --
	( 43.49, 67.15) --
	( 43.49, 67.15) --
	( 43.49, 67.15) --
	( 43.49, 67.15) --
	( 43.49, 67.15) --
	( 43.49, 67.15) --
	( 43.56, 67.34) --
	( 43.56, 67.34) --
	( 43.56, 67.34) --
	( 43.59, 67.38) --
	( 43.59, 67.38) --
	( 43.59, 67.38) --
	( 43.64, 67.46) --
	( 43.64, 67.46) --
	( 43.64, 67.46) --
	( 43.71, 67.64) --
	( 43.71, 67.64) --
	( 43.71, 67.64) --
	( 43.79, 67.81) --
	( 43.79, 67.81) --
	( 43.79, 67.81) --
	( 43.86, 67.90) --
	( 43.86, 67.90) --
	( 43.86, 67.90) --
	( 43.94, 68.07) --
	( 43.94, 68.07) --
	( 43.94, 68.07) --
	( 43.97, 68.14) --
	( 43.97, 68.14) --
	( 43.97, 68.14) --
	( 44.01, 68.22) --
	( 44.01, 68.22) --
	( 44.01, 68.22) --
	( 44.08, 68.61) --
	( 44.08, 68.61) --
	( 44.08, 68.61) --
	( 44.16, 68.74) --
	( 44.16, 68.74) --
	( 44.16, 68.74) --
	( 44.16, 68.75) --
	( 44.16, 68.75) --
	( 44.16, 68.75) --
	( 44.23, 68.97) --
	( 44.23, 68.97) --
	( 44.23, 68.97) --
	( 44.31, 69.22) --
	( 44.31, 69.22) --
	( 44.31, 69.22) --
	( 44.38, 69.51) --
	( 44.38, 69.51) --
	( 44.38, 69.51) --
	( 44.45, 69.90) --
	( 44.45, 69.90) --
	( 44.45, 69.90) --
	( 44.46, 69.94) --
	( 44.46, 69.94) --
	( 44.46, 69.94) --
	( 44.53, 70.34) --
	( 44.53, 70.34) --
	( 44.53, 70.34) --
	( 44.60, 70.65) --
	( 44.60, 70.65) --
	( 44.60, 70.65) --
	( 44.64, 70.76) --
	( 44.64, 70.76) --
	( 44.64, 70.76) --
	( 44.68, 70.88) --
	( 44.68, 70.88) --
	( 44.68, 70.88) --
	( 44.74, 71.16) --
	( 44.74, 71.16) --
	( 44.74, 71.16) --
	( 44.75, 71.23) --
	( 44.75, 71.23) --
	( 44.75, 71.23) --
	( 44.83, 71.82) --
	( 44.83, 71.82) --
	( 44.83, 71.82) --
	( 44.90, 72.27) --
	( 44.90, 72.27) --
	( 44.90, 72.27) --
	( 44.93, 72.40) --
	( 44.93, 72.40) --
	( 44.93, 72.40) --
	( 44.98, 72.64) --
	( 44.98, 72.64) --
	( 44.98, 72.64) --
	( 45.02, 73.21) --
	( 45.02, 73.21) --
	( 45.02, 73.21) --
	( 45.05, 73.50) --
	( 45.05, 73.50) --
	( 45.05, 73.50) --
	( 45.12, 73.59) --
	( 45.12, 73.59) --
	( 45.12, 73.59) --
	( 45.20, 74.18) --
	( 45.20, 74.18) --
	( 45.20, 74.18) --
	( 45.27, 74.83) --
	( 45.27, 74.83) --
	( 45.27, 74.83) --
	( 45.31, 75.22) --
	( 45.31, 75.22) --
	( 45.31, 75.22) --
	( 45.35, 75.54) --
	( 45.35, 75.54) --
	( 45.35, 75.54) --
	( 45.41, 75.96) --
	( 45.41, 75.96) --
	( 45.41, 75.96) --
	( 45.42, 76.04) --
	( 45.42, 76.04) --
	( 45.42, 76.04) --
	( 45.49, 76.82) --
	( 45.49, 76.82) --
	( 45.49, 76.82) --
	( 45.50, 76.85) --
	( 45.50, 76.85) --
	( 45.50, 76.85) --
	( 45.57, 77.05) --
	( 45.57, 77.05) --
	( 45.57, 77.05) --
	( 45.60, 77.38) --
	( 45.60, 77.38) --
	( 45.60, 77.38) --
	( 45.64, 77.82) --
	( 45.64, 77.82) --
	( 45.64, 77.82) --
	( 45.70, 78.40) --
	( 45.70, 78.40) --
	( 45.70, 78.40) --
	( 45.72, 78.64) --
	( 45.72, 78.64) --
	( 45.72, 78.64) --
	( 45.79, 79.50) --
	( 45.79, 79.50) --
	( 45.79, 79.50) --
	( 45.79, 79.50) --
	( 45.79, 79.50) --
	( 45.79, 79.50) --
	( 45.87, 79.65) --
	( 45.87, 79.65) --
	( 45.87, 79.65) --
	( 45.94, 80.15) --
	( 45.94, 80.15) --
	( 45.94, 80.15) --
	( 45.98, 80.50) --
	( 45.98, 80.50) --
	( 45.98, 80.50) --
	( 46.01, 80.74) --
	( 46.01, 80.74) --
	( 46.01, 80.74) --
	( 46.08, 81.24) --
	( 46.08, 81.24) --
	( 46.08, 81.24) --
	( 46.09, 81.30) --
	( 46.09, 81.30) --
	( 46.09, 81.30) --
	( 46.16, 81.22) --
	( 46.16, 81.22) --
	( 46.16, 81.22) --
	( 46.17, 81.40) --
	( 46.17, 81.40) --
	( 46.17, 81.40) --
	( 46.24, 82.22) --
	( 46.24, 82.22) --
	( 46.24, 82.22) --
	( 46.31, 81.99) --
	( 46.31, 81.99) --
	( 46.31, 81.99) --
	( 46.38, 82.16) --
	( 46.38, 82.16) --
	( 46.38, 82.16) --
	( 46.46, 82.13) --
	( 46.46, 82.13) --
	( 46.46, 82.13) --
	( 46.46, 82.15) --
	( 46.46, 82.15) --
	( 46.46, 82.15) --
	( 46.53, 82.54) --
	( 46.53, 82.54) --
	( 46.53, 82.54) --
	( 46.61, 82.41) --
	( 46.61, 82.41) --
	( 46.61, 82.41) --
	( 46.68, 82.06) --
	( 46.68, 82.06) --
	( 46.68, 82.06) --
	( 46.75, 82.16) --
	( 46.75, 82.16) --
	( 46.75, 82.16) --
	( 46.75, 82.16) --
	( 46.75, 82.16) --
	( 46.75, 82.16) --
	( 46.83, 81.51) --
	( 46.83, 81.51) --
	( 46.83, 81.51) --
	( 46.85, 81.53) --
	( 46.85, 81.53) --
	( 46.85, 81.53) --
	( 46.90, 81.59) --
	( 46.90, 81.59) --
	( 46.90, 81.59) --
	( 46.94, 81.32) --
	( 46.94, 81.32) --
	( 46.94, 81.32) --
	( 46.98, 81.08) --
	( 46.98, 81.08) --
	( 46.98, 81.08) --
	( 47.05, 80.50) --
	( 47.05, 80.50) --
	( 47.05, 80.50) --
	( 47.13, 79.87) --
	( 47.13, 79.87) --
	( 47.13, 79.87) --
	( 47.13, 79.83) --
	( 47.13, 79.83) --
	( 47.13, 79.83) --
	( 47.20, 79.50) --
	( 47.20, 79.50) --
	( 47.20, 79.50) --
	( 47.23, 79.21) --
	( 47.23, 79.21) --
	( 47.23, 79.21) --
	( 47.27, 78.77) --
	( 47.27, 78.77) --
	( 47.27, 78.77) --
	( 47.32, 78.62) --
	( 47.32, 78.62) --
	( 47.32, 78.62) --
	( 47.35, 78.56) --
	( 47.35, 78.56) --
	( 47.35, 78.56) --
	( 47.42, 78.58) --
	( 47.42, 78.58) --
	( 47.42, 78.58) --
	( 47.50, 77.41) --
	( 47.50, 77.41) --
	( 47.50, 77.41) --
	( 47.52, 77.30) --
	( 47.52, 77.30) --
	( 47.52, 77.30) --
	( 47.57, 77.02) --
	( 47.57, 77.02) --
	( 47.57, 77.02) --
	( 47.61, 76.63) --
	( 47.61, 76.63) --
	( 47.61, 76.63) --
	( 47.64, 76.34) --
	( 47.64, 76.34) --
	( 47.64, 76.34) --
	( 47.72, 76.19) --
	( 47.72, 76.19) --
	( 47.72, 76.19) --
	( 47.79, 75.74) --
	( 47.79, 75.74) --
	( 47.79, 75.74) --
	( 47.80, 75.71) --
	( 47.80, 75.71) --
	( 47.80, 75.71) --
	( 47.86, 75.58) --
	( 47.86, 75.58) --
	( 47.86, 75.58) --
	( 47.90, 75.23) --
	( 47.90, 75.23) --
	( 47.90, 75.23) --
	( 47.94, 74.84) --
	( 47.94, 74.84) --
	( 47.94, 74.84) --
	( 48.00, 74.40) --
	( 48.00, 74.40) --
	( 48.00, 74.40) --
	( 48.01, 74.26) --
	( 48.01, 74.26) --
	( 48.01, 74.26) --
	( 48.09, 74.08) --
	( 48.09, 74.08) --
	( 48.09, 74.08) --
	( 48.16, 73.97) --
	( 48.16, 73.97) --
	( 48.16, 73.97) --
	( 48.19, 73.84) --
	( 48.19, 73.84) --
	( 48.19, 73.84) --
	( 48.23, 73.62) --
	( 48.23, 73.62) --
	( 48.23, 73.62) --
	( 48.28, 73.26) --
	( 48.28, 73.26) --
	( 48.28, 73.26) --
	( 48.31, 73.07) --
	( 48.31, 73.07) --
	( 48.31, 73.07) --
	( 48.38, 72.89) --
	( 48.38, 72.89) --
	( 48.38, 72.89) --
	( 48.46, 72.70) --
	( 48.46, 72.70) --
	( 48.46, 72.70) --
	( 48.47, 72.71) --
	( 48.47, 72.71) --
	( 48.47, 72.71) --
	( 48.53, 72.76) --
	( 48.53, 72.76) --
	( 48.53, 72.76) --
	( 48.60, 72.60) --
	( 48.60, 72.60) --
	( 48.60, 72.60) --
	( 48.67, 72.16) --
	( 48.67, 72.16) --
	( 48.67, 72.16) --
	( 48.68, 72.07) --
	( 48.68, 72.07) --
	( 48.68, 72.07) --
	( 48.75, 71.87) --
	( 48.75, 71.87) --
	( 48.75, 71.87) --
	( 48.76, 71.84) --
	( 48.76, 71.84) --
	( 48.76, 71.84) --
	( 48.83, 71.66) --
	( 48.83, 71.66) --
	( 48.83, 71.66) --
	( 48.90, 71.49) --
	( 48.90, 71.49) --
	( 48.90, 71.49) --
	( 48.95, 71.46) --
	( 48.95, 71.46) --
	( 48.95, 71.46) --
	( 48.97, 71.44) --
	( 48.97, 71.44) --
	( 48.97, 71.44) --
	( 49.05, 71.30) --
	( 49.05, 71.30) --
	( 49.05, 71.29) --
	( 49.05, 71.29) --
	( 49.05, 71.29) --
	( 49.05, 71.29) --
	( 49.12, 71.10) --
	( 49.12, 71.10) --
	( 49.12, 71.10) --
	( 49.20, 70.77) --
	( 49.20, 70.77) --
	( 49.20, 70.77) --
	( 49.27, 70.83) --
	( 49.27, 70.83) --
	( 49.27, 70.83) --
	( 49.34, 70.61) --
	( 49.34, 70.61) --
	( 49.34, 70.61) --
	( 49.34, 70.58) --
	( 49.34, 70.58) --
	( 49.34, 70.58) --
	( 49.42, 70.32) --
	( 49.42, 70.32) --
	( 49.42, 70.32) --
	( 49.49, 70.30) --
	( 49.49, 70.30) --
	( 49.49, 70.30) --
	( 49.53, 70.18) --
	( 49.53, 70.18) --
	( 49.53, 70.18) --
	( 49.57, 70.07) --
	( 49.57, 70.07) --
	( 49.57, 70.07) --
	( 49.64, 69.93) --
	( 49.64, 69.93) --
	( 49.64, 69.93) --
	( 49.71, 69.78) --
	( 49.71, 69.78) --
	( 49.71, 69.78) --
	( 49.72, 69.75) --
	( 49.72, 69.75) --
	( 49.72, 69.75) --
	( 49.79, 69.49) --
	( 49.79, 69.49) --
	( 49.79, 69.49) --
	( 49.86, 69.37) --
	( 49.86, 69.37) --
	( 49.86, 69.37) --
	( 49.91, 69.35) --
	( 49.91, 69.35) --
	( 49.91, 69.35) --
	( 49.94, 69.34) --
	( 49.94, 69.34) --
	( 49.94, 69.34) --
	( 50.01, 69.16) --
	( 50.01, 69.16) --
	( 50.01, 69.16) --
	( 50.08, 68.90) --
	( 50.08, 68.90) --
	( 50.08, 68.90) --
	( 50.10, 68.87) --
	( 50.10, 68.87) --
	( 50.10, 68.86) --
	( 50.16, 68.77) --
	( 50.16, 68.77) --
	( 50.16, 68.77) --
	( 50.23, 68.69) --
	( 50.23, 68.69) --
	( 50.23, 68.69) --
	( 50.30, 68.56) --
	( 50.30, 68.56) --
	( 50.30, 68.56) --
	( 50.38, 68.47) --
	( 50.38, 68.47) --
	( 50.38, 68.47) --
	( 50.45, 68.28) --
	( 50.45, 68.28) --
	( 50.45, 68.28) --
	( 50.49, 68.27) --
	( 50.49, 68.27) --
	( 50.49, 68.27) --
	( 50.53, 68.26) --
	( 50.53, 68.26) --
	( 50.53, 68.26) --
	( 50.60, 68.10) --
	( 50.60, 68.10) --
	( 50.60, 68.10) --
	( 50.67, 68.08) --
	( 50.67, 68.08) --
	( 50.67, 68.08) --
	( 50.75, 68.01) --
	( 50.75, 68.01) --
	( 50.75, 68.01) --
	( 50.82, 67.89) --
	( 50.82, 67.89) --
	( 50.82, 67.89) --
	( 50.87, 67.87) --
	( 50.87, 67.87) --
	( 50.87, 67.87) --
	( 50.90, 67.85) --
	( 50.90, 67.85) --
	( 50.90, 67.85) --
	( 50.97, 67.76) --
	( 50.97, 67.76) --
	( 50.97, 67.76) --
	( 50.97, 67.76) --
	( 50.97, 67.76) --
	( 50.97, 67.76) --
	( 51.04, 67.75) --
	( 51.04, 67.75) --
	( 51.04, 67.75) --
	( 51.12, 67.63) --
	( 51.12, 67.63) --
	( 51.12, 67.63) --
	( 51.19, 67.60) --
	( 51.19, 67.60) --
	( 51.19, 67.60) --
	( 51.27, 67.50) --
	( 51.27, 67.50) --
	( 51.27, 67.50) --
	( 51.34, 67.48) --
	( 51.34, 67.48) --
	( 51.34, 67.48) --
	( 51.35, 67.48) --
	( 51.35, 67.48) --
	( 51.35, 67.48) --
	( 51.41, 67.47) --
	( 51.41, 67.47) --
	( 51.41, 67.47) --
	( 51.49, 67.42) --
	( 51.49, 67.42) --
	( 51.49, 67.42) --
	( 51.56, 67.31) --
	( 51.56, 67.31) --
	( 51.56, 67.31) --
	( 51.63, 67.33) --
	( 51.63, 67.33) --
	( 51.63, 67.33) --
	( 51.71, 67.25) --
	( 51.71, 67.25) --
	( 51.71, 67.25) --
	( 51.73, 67.24) --
	( 51.73, 67.24) --
	( 51.73, 67.24) --
	( 51.78, 67.22) --
	( 51.78, 67.22) --
	( 51.78, 67.22) --
	( 51.86, 67.16) --
	( 51.86, 67.16) --
	( 51.86, 67.16) --
	( 51.93, 67.13) --
	( 51.93, 67.13) --
	( 51.93, 67.13) --
	( 52.00, 67.04) --
	( 52.00, 67.04) --
	( 52.00, 67.04) --
	( 52.08, 67.01) --
	( 52.08, 67.01) --
	( 52.08, 67.01) --
	( 52.15, 67.01) --
	( 52.15, 67.01) --
	( 52.15, 67.01) --
	( 52.21, 66.93) --
	( 52.21, 66.93) --
	( 52.21, 66.93) --
	( 52.22, 66.91) --
	( 52.22, 66.91) --
	( 52.22, 66.91) --
	( 52.30, 66.87) --
	( 52.30, 66.87) --
	( 52.30, 66.87) --
	( 52.37, 66.85) --
	( 52.37, 66.85) --
	( 52.37, 66.85) --
	( 52.45, 66.76) --
	( 52.45, 66.76) --
	( 52.45, 66.76) --
	( 52.52, 66.76) --
	( 52.52, 66.76) --
	( 52.52, 66.76) --
	( 52.59, 66.70) --
	( 52.59, 66.70) --
	( 52.59, 66.70) --
	( 52.67, 66.71) --
	( 52.67, 66.71) --
	( 52.67, 66.71) --
	( 52.69, 66.70) --
	( 52.69, 66.70) --
	( 52.69, 66.70) --
	( 52.74, 66.67) --
	( 52.74, 66.67) --
	( 52.74, 66.67) --
	( 52.81, 66.59) --
	( 52.81, 66.59) --
	( 52.81, 66.59) --
	( 52.89, 66.56) --
	( 52.89, 66.56) --
	( 52.89, 66.56) --
	( 52.96, 66.53) --
	( 52.96, 66.53) --
	( 52.96, 66.53) --
	( 53.04, 66.47) --
	( 53.04, 66.47) --
	( 53.04, 66.47) --
	( 53.11, 66.49) --
	( 53.11, 66.49) --
	( 53.11, 66.49) --
	( 53.17, 66.47) --
	( 53.17, 66.47) --
	( 53.17, 66.47) --
	( 53.18, 66.47) --
	( 53.18, 66.47) --
	( 53.18, 66.47) --
	( 53.26, 66.39) --
	( 53.26, 66.39) --
	( 53.26, 66.39) --
	( 53.33, 66.39) --
	( 53.33, 66.39) --
	( 53.33, 66.39) --
	( 53.41, 66.39) --
	( 53.41, 66.39) --
	( 53.41, 66.39) --
	( 53.48, 66.34) --
	( 53.48, 66.34) --
	( 53.48, 66.34) --
	( 53.55, 66.30) --
	( 53.55, 66.30) --
	( 53.55, 66.30) --
	( 53.62, 66.31) --
	( 53.62, 66.31) --
	( 53.62, 66.31) --
	( 53.70, 66.26) --
	( 53.70, 66.26) --
	( 53.70, 66.26) --
	( 53.75, 66.24) --
	( 53.75, 66.24) --
	( 53.75, 66.24) --
	( 53.77, 66.23) --
	( 53.77, 66.23) --
	( 53.77, 66.23) --
	( 53.85, 66.20) --
	( 53.85, 66.20) --
	( 53.85, 66.20) --
	( 53.92, 66.22) --
	( 53.92, 66.22) --
	( 53.92, 66.22) --
	( 53.99, 66.19) --
	( 53.99, 66.19) --
	( 53.99, 66.19) --
	( 54.07, 66.14) --
	( 54.07, 66.14) --
	( 54.07, 66.14) --
	( 54.14, 66.15) --
	( 54.14, 66.15) --
	( 54.14, 66.15) --
	( 54.21, 66.16) --
	( 54.21, 66.16) --
	( 54.21, 66.16) --
	( 54.29, 66.11) --
	( 54.29, 66.11) --
	( 54.29, 66.11) --
	( 54.36, 66.12) --
	( 54.36, 66.12) --
	( 54.36, 66.12) --
	( 54.44, 66.09) --
	( 54.44, 66.09) --
	( 54.44, 66.09) --
	( 54.51, 66.08) --
	( 54.51, 66.08) --
	( 54.51, 66.08) --
	( 54.51, 66.08) --
	( 54.51, 66.08) --
	( 54.51, 66.08) --
	( 54.58, 66.08) --
	( 54.58, 66.08) --
	( 54.58, 66.08) --
	( 54.66, 66.04) --
	( 54.66, 66.04) --
	( 54.66, 66.04) --
	( 54.73, 66.03) --
	( 54.73, 66.03) --
	( 54.73, 66.03) --
	( 54.80, 66.02) --
	( 54.80, 66.02) --
	( 54.80, 66.02) --
	( 54.88, 65.99) --
	( 54.88, 65.99) --
	( 54.88, 65.99) --
	( 54.95, 65.99) --
	( 54.95, 65.99) --
	( 54.95, 65.99) --
	( 55.02, 66.00) --
	( 55.02, 66.00) --
	( 55.02, 66.00) --
	( 55.10, 65.96) --
	( 55.10, 65.96) --
	( 55.10, 65.96) --
	( 55.17, 65.98) --
	( 55.17, 65.98) --
	( 55.17, 65.98) --
	( 55.25, 65.96) --
	( 55.25, 65.96) --
	( 55.25, 65.96) --
	( 55.28, 65.96) --
	( 55.28, 65.96) --
	( 55.28, 65.96) --
	( 55.32, 65.96) --
	( 55.32, 65.96) --
	( 55.32, 65.96) --
	( 55.39, 65.92) --
	( 55.39, 65.92) --
	( 55.39, 65.92) --
	( 55.47, 65.90) --
	( 55.47, 65.90) --
	( 55.47, 65.90) --
	( 55.54, 65.96) --
	( 55.54, 65.96) --
	( 55.54, 65.96) --
	( 55.57, 65.95) --
	( 55.57, 65.95) --
	( 55.57, 65.95) --
	( 55.61, 65.92) --
	( 55.61, 65.92) --
	( 55.61, 65.92) --
	( 55.69, 65.90) --
	( 55.69, 65.90) --
	( 55.69, 65.90) --
	( 55.76, 65.90) --
	( 55.76, 65.90) --
	( 55.76, 65.90) --
	( 55.83, 65.93) --
	( 55.83, 65.93) --
	( 55.83, 65.93) --
	( 55.91, 65.89) --
	( 55.91, 65.89) --
	( 55.91, 65.89) --
	( 55.98, 65.87) --
	( 55.98, 65.87) --
	( 55.98, 65.87) --
	( 56.06, 65.90) --
	( 56.06, 65.90) --
	( 56.06, 65.90) --
	( 56.13, 65.86) --
	( 56.13, 65.86) --
	( 56.13, 65.86) --
	( 56.14, 65.86) --
	( 56.14, 65.86) --
	( 56.14, 65.86) --
	( 56.20, 65.87) --
	( 56.20, 65.87) --
	( 56.20, 65.87) --
	( 56.27, 65.85) --
	( 56.27, 65.85) --
	( 56.27, 65.85) --
	( 56.35, 65.85) --
	( 56.35, 65.85) --
	( 56.35, 65.85) --
	( 56.42, 65.87) --
	( 56.42, 65.87) --
	( 56.42, 65.87) --
	( 56.43, 65.87) --
	( 56.43, 65.87) --
	( 56.43, 65.87) --
	( 56.50, 65.86) --
	( 56.50, 65.86) --
	( 56.50, 65.86) --
	( 56.57, 65.86) --
	( 56.57, 65.86) --
	( 56.57, 65.86) --
	( 56.64, 65.89) --
	( 56.64, 65.89) --
	( 56.64, 65.89) --
	( 56.72, 65.84) --
	( 56.72, 65.84) --
	( 56.72, 65.84) --
	( 56.79, 65.83) --
	( 56.79, 65.83) --
	( 56.79, 65.83) --
	( 56.86, 65.83) --
	( 56.86, 65.83) --
	( 56.86, 65.83) --
	( 56.91, 65.84) --
	( 56.91, 65.84) --
	( 56.91, 65.84) --
	( 56.94, 65.84) --
	( 56.94, 65.84) --
	( 56.94, 65.84) --
	( 57.01, 65.85) --
	( 57.01, 65.85) --
	( 57.01, 65.85) --
	( 57.08, 65.82) --
	( 57.08, 65.82) --
	( 57.08, 65.82) --
	( 57.16, 65.82) --
	( 57.16, 65.82) --
	( 57.16, 65.82) --
	( 57.20, 65.83) --
	( 57.20, 65.83) --
	( 57.20, 65.83) --
	( 57.23, 65.83) --
	( 57.23, 65.83) --
	( 57.23, 65.83) --
	( 57.30, 65.80) --
	( 57.30, 65.80) --
	( 57.30, 65.80) --
	( 57.38, 65.81) --
	( 57.38, 65.81) --
	( 57.38, 65.81) --
	( 57.45, 65.83) --
	( 57.45, 65.83) --
	( 57.45, 65.83) --
	( 57.52, 65.81) --
	( 57.52, 65.81) --
	( 57.52, 65.81) --
	( 57.60, 65.82) --
	( 57.60, 65.82) --
	( 57.60, 65.82) --
	( 57.67, 65.81) --
	( 57.67, 65.81) --
	( 57.67, 65.81) --
	( 57.74, 65.82) --
	( 57.74, 65.82) --
	( 57.74, 65.82) --
	( 57.82, 65.83) --
	( 57.82, 65.83) --
	( 57.82, 65.83) --
	( 57.87, 65.79) --
	( 57.87, 65.79) --
	( 57.87, 65.79) --
	( 57.89, 65.77) --
	( 57.89, 65.77) --
	( 57.89, 65.77) --
	( 57.97, 65.81) --
	( 57.97, 65.81) --
	( 57.97, 65.81) --
	( 58.04, 65.83) --
	( 58.04, 65.83) --
	( 58.04, 65.83) --
	( 58.11, 65.80) --
	( 58.11, 65.80) --
	( 58.11, 65.80) --
	( 58.18, 65.80) --
	( 58.18, 65.80) --
	( 58.18, 65.80) --
	( 58.26, 65.82) --
	( 58.26, 65.82) --
	( 58.26, 65.82) --
	( 58.33, 65.78) --
	( 58.33, 65.78) --
	( 58.33, 65.78) --
	( 58.35, 65.78) --
	( 58.35, 65.78) --
	( 58.35, 65.78) --
	( 58.41, 65.80) --
	( 58.41, 65.80) --
	( 58.41, 65.80) --
	( 58.48, 65.79) --
	( 58.48, 65.79) --
	( 58.48, 65.79) --
	( 58.55, 65.80) --
	( 58.55, 65.80) --
	( 58.55, 65.80) --
	( 58.63, 65.79) --
	( 58.63, 65.79) --
	( 58.63, 65.79) --
	( 58.70, 65.77) --
	( 58.70, 65.77) --
	( 58.70, 65.77) --
	( 58.77, 65.79) --
	( 58.77, 65.79) --
	( 58.77, 65.79) --
	( 58.85, 65.79) --
	( 58.85, 65.79) --
	( 58.85, 65.79) --
	( 58.92, 65.78) --
	( 58.92, 65.78) --
	( 58.92, 65.78) --
	( 58.92, 65.78) --
	( 58.92, 65.78) --
	( 58.92, 65.78) --
	( 58.99, 65.79) --
	( 58.99, 65.79) --
	( 58.99, 65.79) --
	( 59.07, 65.79) --
	( 59.07, 65.79) --
	( 59.07, 65.79) --
	( 59.14, 65.77) --
	( 59.14, 65.77) --
	( 59.14, 65.77) --
	( 59.21, 65.77) --
	( 59.21, 65.77) --
	( 59.21, 65.77) --
	( 59.29, 65.79) --
	( 59.29, 65.79) --
	( 59.29, 65.79) --
	( 59.36, 65.80) --
	( 59.36, 65.80) --
	( 59.36, 65.80) --
	( 59.43, 65.77) --
	( 59.43, 65.77) --
	( 59.43, 65.77) --
	( 59.50, 65.78) --
	( 59.50, 65.78) --
	( 59.50, 65.78) --
	( 59.51, 65.78) --
	( 59.51, 65.78) --
	( 59.51, 65.78) --
	( 59.58, 65.78) --
	( 59.58, 65.78) --
	( 59.58, 65.78) --
	( 59.65, 65.78) --
	( 59.65, 65.78) --
	( 59.65, 65.78) --
	( 59.73, 65.78) --
	( 59.73, 65.78) --
	( 59.73, 65.78) --
	( 59.80, 65.77) --
	( 59.80, 65.77) --
	( 59.80, 65.77) --
	( 59.87, 65.80) --
	( 59.87, 65.80) --
	( 59.87, 65.80) --
	( 59.95, 65.77) --
	( 59.95, 65.77) --
	( 59.95, 65.77) --
	( 60.02, 65.77) --
	( 60.02, 65.77) --
	( 60.02, 65.77) --
	( 60.09, 65.78) --
	( 60.09, 65.78) --
	( 60.09, 65.78) --
	( 60.17, 65.78) --
	( 60.17, 65.78) --
	( 60.17, 65.78) --
	( 60.24, 65.78) --
	( 60.24, 65.78) --
	( 60.24, 65.78) --
	( 60.31, 65.77) --
	( 60.31, 65.77) --
	( 60.31, 65.77) --
	( 60.39, 65.79) --
	( 60.39, 65.79) --
	( 60.39, 65.79) --
	( 60.45, 65.79) --
	( 60.45, 65.79) --
	( 60.45, 65.79) --
	( 60.46, 65.79) --
	( 60.46, 65.79) --
	( 60.46, 65.79) --
	( 60.53, 65.77) --
	( 60.53, 65.77) --
	( 60.53, 65.77) --
	( 60.61, 65.77) --
	( 60.61, 65.77) --
	( 60.61, 65.77) --
	( 60.68, 65.77) --
	( 60.68, 65.77) --
	( 60.68, 65.77) --
	( 60.75, 65.75) --
	( 60.75, 65.75) --
	( 60.75, 65.75) --
	( 60.83, 65.77) --
	( 60.83, 65.77) --
	( 60.83, 65.77) --
	( 60.90, 65.76) --
	( 60.90, 65.76) --
	( 60.90, 65.76) --
	( 60.97, 65.76) --
	( 60.97, 65.76) --
	( 60.97, 65.76) --
	( 61.04, 65.77) --
	( 61.04, 65.77) --
	( 61.04, 65.77) --
	( 61.12, 65.76) --
	( 61.12, 65.76) --
	( 61.12, 65.76) --
	( 61.19, 65.76) --
	( 61.19, 65.76) --
	( 61.19, 65.76) --
	( 61.26, 65.78) --
	( 61.26, 65.78) --
	( 61.26, 65.78) --
	( 61.34, 65.76) --
	( 61.34, 65.76) --
	( 61.34, 65.76) --
	( 61.41, 65.78) --
	( 61.41, 65.78) --
	( 61.41, 65.78) --
	( 61.41, 65.78) --
	( 61.41, 65.78) --
	( 61.41, 65.78) --
	( 61.48, 65.77) --
	( 61.48, 65.77) --
	( 61.48, 65.77) --
	( 61.56, 65.76) --
	( 61.56, 65.76) --
	( 61.56, 65.76) --
	( 61.63, 65.77) --
	( 61.63, 65.77) --
	( 61.63, 65.77) --
	( 61.70, 65.76) --
	( 61.70, 65.76) --
	( 61.70, 65.76) --
	( 61.78, 65.78) --
	( 61.78, 65.78) --
	( 61.78, 65.78) --
	( 61.85, 65.78) --
	( 61.85, 65.78) --
	( 61.85, 65.78) --
	( 61.92, 65.75) --
	( 61.92, 65.75) --
	( 61.92, 65.75) --
	( 62.00, 65.77) --
	( 62.00, 65.77) --
	( 62.00, 65.77) --
	( 62.07, 65.79) --
	( 62.07, 65.79) --
	( 62.07, 65.79) --
	( 62.14, 65.77) --
	( 62.14, 65.77) --
	( 62.14, 65.77) --
	( 62.22, 65.78) --
	( 62.22, 65.78) --
	( 62.22, 65.78) --
	( 62.29, 65.78) --
	( 62.29, 65.78) --
	( 62.29, 65.78) --
	( 62.36, 65.80) --
	( 62.36, 65.80) --
	( 62.36, 65.80) --
	( 62.44, 65.76) --
	( 62.44, 65.76) --
	( 62.44, 65.76) --
	( 62.51, 65.76) --
	( 62.51, 65.76) --
	( 62.51, 65.76) --
	( 62.58, 65.77) --
	( 62.58, 65.77) --
	( 62.58, 65.77) --
	( 62.65, 65.76) --
	( 62.65, 65.76) --
	( 62.65, 65.76) --
	( 62.73, 65.76) --
	( 62.73, 65.76) --
	( 62.73, 65.76) --
	( 62.80, 65.78) --
	( 62.80, 65.78) --
	( 62.80, 65.78) --
	( 62.87, 65.79) --
	( 62.87, 65.79) --
	( 62.87, 65.79) --
	( 62.95, 65.78) --
	( 62.95, 65.78) --
	( 62.95, 65.78) --
	( 63.02, 65.75) --
	( 63.02, 65.75) --
	( 63.02, 65.75) --
	( 63.09, 65.77) --
	( 63.09, 65.77) --
	( 63.09, 65.77) --
	( 63.17, 65.75) --
	( 63.17, 65.75) --
	( 63.17, 65.75) --
	( 63.23, 65.75) --
	( 63.23, 65.75) --
	( 63.23, 65.75) --
	( 63.24, 65.75) --
	( 63.24, 65.75) --
	( 63.24, 65.75) --
	( 63.31, 65.76) --
	( 63.31, 65.76) --
	( 63.31, 65.76) --
	( 63.39, 65.78) --
	( 63.39, 65.78) --
	( 63.39, 65.78) --
	( 63.46, 65.77) --
	( 63.46, 65.77) --
	( 63.46, 65.77) --
	( 63.53, 65.74) --
	( 63.53, 65.74) --
	( 63.53, 65.74) --
	( 63.61, 65.76) --
	( 63.61, 65.76) --
	( 63.61, 65.76) --
	( 63.68, 65.79) --
	( 63.68, 65.79) --
	( 63.68, 65.79) --
	( 63.75, 65.75) --
	( 63.75, 65.75) --
	( 63.75, 65.75) --
	( 63.83, 65.76) --
	( 63.83, 65.76) --
	( 63.83, 65.76) --
	( 63.90, 65.76) --
	( 63.90, 65.76) --
	( 63.90, 65.76) --
	( 63.97, 65.76) --
	( 63.97, 65.76) --
	( 63.97, 65.76) --
	( 64.04, 65.77) --
	( 64.04, 65.77) --
	( 64.04, 65.77) --
	( 64.12, 65.76) --
	( 64.12, 65.76) --
	( 64.12, 65.76) --
	( 64.19, 65.77) --
	( 64.19, 65.77) --
	( 64.19, 65.77) --
	( 64.26, 65.79) --
	( 64.26, 65.79) --
	( 64.26, 65.79) --
	( 64.34, 65.77) --
	( 64.34, 65.77) --
	( 64.34, 65.77) --
	( 64.41, 65.75) --
	( 64.41, 65.75) --
	( 64.41, 65.75) --
	( 64.48, 65.79) --
	( 64.48, 65.79) --
	( 64.48, 65.79) --
	( 64.56, 65.75) --
	( 64.56, 65.75) --
	( 64.56, 65.75) --
	( 64.57, 65.75) --
	( 64.57, 65.75) --
	( 64.57, 65.75) --
	( 64.63, 65.76) --
	( 64.63, 65.76) --
	( 64.63, 65.76) --
	( 64.70, 65.78) --
	( 64.70, 65.78) --
	( 64.70, 65.78) --
	( 64.77, 65.76) --
	( 64.77, 65.76) --
	( 64.77, 65.76) --
	( 64.85, 65.76) --
	( 64.85, 65.76) --
	( 64.85, 65.76) --
	( 64.92, 65.75) --
	( 64.92, 65.75) --
	( 64.92, 65.75) --
	( 64.99, 65.77) --
	( 64.99, 65.77) --
	( 64.99, 65.77) --
	( 65.07, 65.77) --
	( 65.07, 65.77) --
	( 65.07, 65.77) --
	( 65.14, 65.78) --
	( 65.14, 65.78) --
	( 65.14, 65.78) --
	( 65.21, 65.78) --
	( 65.21, 65.78) --
	( 65.21, 65.78) --
	( 65.29, 65.77) --
	( 65.29, 65.77) --
	( 65.29, 65.77) --
	( 65.36, 65.75) --
	( 65.36, 65.75) --
	( 65.36, 65.75) --
	( 65.43, 65.76) --
	( 65.43, 65.76) --
	( 65.43, 65.76) --
	( 65.50, 65.75) --
	( 65.50, 65.75) --
	( 65.50, 65.75) --
	( 65.58, 65.75) --
	( 65.58, 65.75) --
	( 65.58, 65.75) --
	( 65.65, 65.77) --
	( 65.65, 65.77) --
	( 65.65, 65.77) --
	( 65.72, 65.74) --
	( 65.72, 65.74) --
	( 65.72, 65.74) --
	( 65.80, 65.75) --
	( 65.80, 65.75) --
	( 65.80, 65.75) --
	( 65.87, 65.78) --
	( 65.87, 65.78) --
	( 65.87, 65.78) --
	( 65.94, 65.75) --
	( 65.94, 65.75) --
	( 65.94, 65.75) --
	( 66.02, 65.75) --
	( 66.02, 65.75) --
	( 66.02, 65.75) --
	( 66.09, 65.76) --
	( 66.09, 65.76) --
	( 66.09, 65.76) --
	( 66.16, 65.73) --
	( 66.16, 65.73) --
	( 66.16, 65.73) --
	( 66.23, 65.76) --
	( 66.23, 65.76) --
	( 66.23, 65.76) --
	( 66.31, 65.73) --
	( 66.31, 65.73) --
	( 66.31, 65.73) --
	( 66.38, 65.77) --
	( 66.38, 65.77) --
	( 66.38, 65.77) --
	( 66.40, 65.77) --
	( 66.40, 65.77) --
	( 66.40, 65.77) --
	( 66.45, 65.77) --
	( 66.45, 65.77) --
	( 66.45, 65.77) --
	( 66.53, 65.75) --
	( 66.53, 65.75) --
	( 66.53, 65.75) --
	( 66.60, 65.74) --
	( 66.60, 65.74) --
	( 66.60, 65.74) --
	( 66.67, 65.78) --
	( 66.67, 65.78) --
	( 66.67, 65.78) --
	( 66.74, 65.75) --
	( 66.74, 65.75) --
	( 66.74, 65.75) --
	( 66.82, 65.75) --
	( 66.82, 65.75) --
	( 66.82, 65.75) --
	( 66.89, 65.75) --
	( 66.89, 65.75) --
	( 66.89, 65.75) --
	( 66.96, 65.73) --
	( 66.96, 65.73) --
	( 66.96, 65.73) --
	( 67.04, 65.76) --
	( 67.04, 65.76) --
	( 67.04, 65.76) --
	( 67.11, 65.73) --
	( 67.11, 65.73) --
	( 67.11, 65.73) --
	( 67.18, 65.75) --
	( 67.18, 65.75) --
	( 67.18, 65.75) --
	( 67.25, 65.76) --
	( 67.25, 65.76) --
	( 67.25, 65.76) --
	( 67.33, 65.72) --
	( 67.33, 65.72) --
	( 67.33, 65.72) --
	( 67.40, 65.77) --
	( 67.40, 65.77) --
	( 67.40, 65.77) --
	( 67.47, 65.78) --
	( 67.47, 65.78) --
	( 67.47, 65.78) --
	( 67.55, 65.74) --
	( 67.55, 65.74) --
	( 67.55, 65.74) --
	( 67.62, 65.76) --
	( 67.62, 65.76) --
	( 67.62, 65.76) --
	( 67.69, 65.78) --
	( 67.69, 65.78) --
	( 67.69, 65.78) --
	( 67.76, 65.76) --
	( 67.76, 65.76) --
	( 67.76, 65.76) --
	( 67.84, 65.76) --
	( 67.84, 65.76) --
	( 67.84, 65.76) --
	( 67.91, 65.75) --
	( 67.91, 65.75) --
	( 67.91, 65.75) --
	( 67.98, 65.79) --
	( 67.98, 65.79) --
	( 67.98, 65.79) --
	( 68.05, 65.75) --
	( 68.05, 65.75) --
	( 68.05, 65.75) --
	( 68.13, 65.76) --
	( 68.13, 65.76) --
	( 68.13, 65.76) --
	( 68.20, 65.76) --
	( 68.20, 65.76) --
	( 68.20, 65.76) --
	( 68.22, 65.76) --
	( 68.22, 65.76) --
	( 68.22, 65.76) --
	( 68.27, 65.78) --
	( 68.27, 65.78) --
	( 68.27, 65.78) --
	( 68.35, 65.78) --
	( 68.35, 65.78) --
	( 68.35, 65.78) --
	( 68.42, 65.75) --
	( 68.42, 65.75) --
	( 68.42, 65.75) --
	( 68.49, 65.78) --
	( 68.49, 65.78) --
	( 68.49, 65.78) --
	( 68.56, 65.76) --
	( 68.56, 65.76) --
	( 68.56, 65.76) --
	( 68.64, 65.75) --
	( 68.64, 65.75) --
	( 68.64, 65.75) --
	( 68.71, 65.72) --
	( 68.71, 65.72) --
	( 68.71, 65.72) --
	( 68.78, 65.77) --
	( 68.78, 65.77) --
	( 68.78, 65.77) --
	( 68.86, 65.78) --
	( 68.86, 65.78) --
	( 68.86, 65.78) --
	( 68.93, 65.77) --
	( 68.93, 65.77) --
	( 68.93, 65.77) --
	( 69.00, 65.78) --
	( 69.00, 65.78) --
	( 69.00, 65.78) --
	( 69.07, 65.78) --
	( 69.07, 65.78) --
	( 69.07, 65.78) --
	( 69.15, 65.73) --
	( 69.15, 65.73) --
	( 69.15, 65.73) --
	( 69.22, 65.75) --
	( 69.22, 65.75) --
	( 69.22, 65.75) --
	( 69.27, 65.77) --
	( 69.27, 65.77) --
	( 69.27, 65.77) --
	( 69.29, 65.78) --
	( 69.29, 65.78) --
	( 69.29, 65.78) --
	( 69.37, 65.78) --
	( 69.37, 65.78) --
	( 69.37, 65.78) --
	( 69.44, 65.74) --
	( 69.44, 65.74) --
	( 69.44, 65.74) --
	( 69.51, 65.74) --
	( 69.51, 65.74) --
	( 69.51, 65.74) --
	( 69.58, 65.76) --
	( 69.58, 65.76) --
	( 69.58, 65.76) --
	( 69.66, 65.75) --
	( 69.66, 65.75) --
	( 69.66, 65.75) --
	( 69.73, 65.76) --
	( 69.73, 65.76) --
	( 69.73, 65.76) --
	( 69.80, 65.77) --
	( 69.80, 65.77) --
	( 69.80, 65.77) --
	( 69.87, 65.79) --
	( 69.87, 65.79) --
	( 69.87, 65.79) --
	( 69.95, 65.75) --
	( 69.95, 65.75) --
	( 69.95, 65.75) --
	( 70.02, 65.76) --
	( 70.02, 65.76) --
	( 70.02, 65.76) --
	( 70.09, 65.76) --
	( 70.09, 65.76) --
	( 70.09, 65.76) --
	( 70.17, 65.77) --
	( 70.17, 65.77) --
	( 70.17, 65.77) --
	( 70.24, 65.76) --
	( 70.24, 65.76) --
	( 70.24, 65.76) --
	( 70.31, 65.75) --
	( 70.31, 65.75) --
	( 70.31, 65.75) --
	( 70.38, 65.78) --
	( 70.38, 65.78) --
	( 70.38, 65.78) --
	( 70.42, 65.77) --
	( 70.42, 65.77) --
	( 70.42, 65.77) --
	( 70.46, 65.76) --
	( 70.46, 65.76) --
	( 70.46, 65.76) --
	( 70.53, 65.73) --
	( 70.53, 65.73) --
	( 70.53, 65.73) --
	( 70.60, 65.76) --
	( 70.60, 65.76) --
	( 70.60, 65.76) --
	( 70.67, 65.77) --
	( 70.67, 65.77) --
	( 70.67, 65.77) --
	( 70.75, 65.77) --
	( 70.75, 65.77) --
	( 70.75, 65.77) --
	( 70.82, 65.75) --
	( 70.82, 65.75) --
	( 70.82, 65.75) --
	( 70.89, 65.75) --
	( 70.89, 65.75) --
	( 70.89, 65.75) --
	( 70.96, 65.76) --
	( 70.96, 65.76) --
	( 70.96, 65.76) --
	( 71.04, 65.77) --
	( 71.04, 65.77) --
	( 71.04, 65.77) --
	( 71.11, 65.75) --
	( 71.11, 65.75) --
	( 71.11, 65.75) --
	( 71.18, 65.77) --
	( 71.18, 65.77) --
	( 71.18, 65.77) --
	( 71.19, 65.77) --
	( 71.19, 65.77) --
	( 71.19, 65.77) --
	( 71.26, 65.76) --
	( 71.26, 65.76) --
	( 71.26, 65.76) --
	( 71.33, 65.74) --
	( 71.33, 65.74) --
	( 71.33, 65.74) --
	( 71.40, 65.77) --
	( 71.40, 65.77) --
	( 71.40, 65.77) --
	( 71.47, 65.77) --
	( 71.47, 65.77) --
	( 71.47, 65.77) --
	( 71.54, 65.74) --
	( 71.54, 65.74) --
	( 71.54, 65.74) --
	( 71.62, 65.75) --
	( 71.62, 65.75) --
	( 71.62, 65.75) --
	( 71.69, 65.75) --
	( 71.69, 65.75) --
	( 71.69, 65.75) --
	( 71.76, 65.74) --
	( 71.76, 65.74) --
	( 71.76, 65.74) --
	( 71.83, 65.77) --
	( 71.83, 65.77) --
	( 71.83, 65.77) --
	( 71.91, 65.74) --
	( 71.91, 65.74) --
	( 71.91, 65.74) --
	( 71.95, 65.77) --
	( 71.95, 65.77) --
	( 71.95, 65.77) --
	( 71.98, 65.78) --
	( 71.98, 65.78) --
	( 71.98, 65.78) --
	( 72.05, 65.77) --
	( 72.05, 65.77) --
	( 72.05, 65.77) --
	( 72.13, 65.75) --
	( 72.13, 65.75) --
	( 72.13, 65.75) --
	( 72.20, 65.76) --
	( 72.20, 65.76) --
	( 72.20, 65.76) --
	( 72.27, 65.79) --
	( 72.27, 65.79) --
	( 72.27, 65.79) --
	( 72.34, 65.75) --
	( 72.34, 65.75) --
	( 72.34, 65.75) --
	( 72.42, 65.76) --
	( 72.42, 65.76) --
	( 72.42, 65.76) --
	( 72.49, 65.75) --
	( 72.49, 65.75) --
	( 72.49, 65.75) --
	( 72.56, 65.76) --
	( 72.56, 65.76) --
	( 72.56, 65.76) --
	( 72.63, 65.75) --
	( 72.63, 65.75) --
	( 72.63, 65.75) --
	( 72.71, 65.75) --
	( 72.71, 65.75) --
	( 72.71, 65.75) --
	( 72.78, 65.77) --
	( 72.78, 65.77) --
	( 72.78, 65.77) --
	( 72.85, 65.78) --
	( 72.85, 65.78) --
	( 72.85, 65.78) --
	( 72.92, 65.76) --
	( 72.92, 65.76) --
	( 72.92, 65.76) --
	( 73.00, 65.77) --
	( 73.00, 65.77) --
	( 73.00, 65.77) --
	( 73.07, 65.79) --
	( 73.07, 65.79) --
	( 73.07, 65.79) --
	( 73.14, 65.76) --
	( 73.14, 65.76) --
	( 73.14, 65.76) --
	( 73.20, 65.76) --
	( 73.20, 65.76) --
	( 73.20, 65.76) --
	( 73.21, 65.76) --
	( 73.21, 65.76) --
	( 73.21, 65.76) --
	( 73.28, 65.76) --
	( 73.28, 65.76) --
	( 73.28, 65.76) --
	( 73.36, 65.77) --
	( 73.36, 65.77) --
	( 73.36, 65.77) --
	( 73.43, 65.78) --
	( 73.43, 65.78) --
	( 73.43, 65.78) --
	( 73.50, 65.77) --
	( 73.50, 65.77) --
	( 73.50, 65.77) --
	( 73.58, 65.80) --
	( 73.58, 65.80) --
	( 73.58, 65.80) --
	( 73.65, 65.76) --
	( 73.65, 65.76) --
	( 73.65, 65.76) --
	( 73.72, 65.74) --
	( 73.72, 65.74) --
	( 73.72, 65.74) --
	( 73.79, 65.76) --
	( 73.79, 65.76) --
	( 73.79, 65.76) --
	( 73.86, 65.76) --
	( 73.86, 65.76) --
	( 73.86, 65.76) --
	( 73.94, 65.74) --
	( 73.94, 65.74) --
	( 73.94, 65.74) --
	( 73.97, 65.75) --
	( 73.97, 65.75) --
	( 73.97, 65.75) --
	( 74.01, 65.76) --
	( 74.01, 65.76) --
	( 74.01, 65.76) --
	( 74.08, 65.74) --
	( 74.08, 65.74) --
	( 74.08, 65.74) --
	( 74.16, 65.76) --
	( 74.16, 65.76) --
	( 74.16, 65.76) --
	( 74.23, 65.78) --
	( 74.23, 65.78) --
	( 74.23, 65.78) --
	( 74.30, 65.74) --
	( 74.30, 65.74) --
	( 74.30, 65.74) --
	( 74.37, 65.78) --
	( 74.37, 65.78) --
	( 74.37, 65.78) --
	( 74.44, 65.80) --
	( 74.44, 65.80) --
	( 74.44, 65.80) --
	( 74.52, 65.73) --
	( 74.52, 65.73) --
	( 74.52, 65.73) --
	( 74.59, 65.74) --
	( 74.59, 65.74) --
	( 74.59, 65.74) --
	( 74.66, 65.78) --
	( 74.66, 65.78) --
	( 74.66, 65.78) --
	( 74.73, 65.75) --
	( 74.73, 65.75) --
	( 74.73, 65.75) --
	( 74.81, 65.76) --
	( 74.81, 65.76) --
	( 74.81, 65.76) --
	( 74.88, 65.74) --
	( 74.88, 65.74) --
	( 74.88, 65.74) --
	( 74.92, 65.76) --
	( 74.92, 65.76) --
	( 74.92, 65.76) --
	( 74.95, 65.78) --
	( 74.95, 65.78) --
	( 74.95, 65.78) --
	( 75.02, 65.77) --
	( 75.02, 65.77) --
	( 75.02, 65.77) --
	( 75.10, 65.75) --
	( 75.10, 65.75) --
	( 75.10, 65.75) --
	( 75.17, 65.73) --
	( 75.17, 65.73) --
	( 75.17, 65.73) --
	( 75.24, 65.76) --
	( 75.24, 65.76) --
	( 75.24, 65.76) --
	( 75.31, 65.75) --
	( 75.31, 65.75) --
	( 75.31, 65.75) --
	( 75.39, 65.74) --
	( 75.39, 65.74) --
	( 75.39, 65.74) --
	( 75.46, 65.76) --
	( 75.46, 65.76) --
	( 75.46, 65.76) --
	( 75.53, 65.73) --
	( 75.53, 65.73) --
	( 75.53, 65.73) --
	( 75.60, 65.76) --
	( 75.60, 65.76) --
	( 75.60, 65.76) --
	( 75.67, 65.72) --
	( 75.67, 65.72) --
	( 75.67, 65.72) --
	( 75.75, 65.73) --
	( 75.75, 65.73) --
	( 75.75, 65.73) --
	( 75.79, 65.75) --
	( 75.79, 65.75) --
	( 75.79, 65.75) --
	( 75.82, 65.77) --
	( 75.82, 65.77) --
	( 75.82, 65.77) --
	( 75.89, 65.76) --
	( 75.89, 65.76) --
	( 75.89, 65.76) --
	( 75.96, 65.74) --
	( 75.96, 65.74) --
	( 75.96, 65.74) --
	( 76.04, 65.76) --
	( 76.04, 65.76) --
	( 76.04, 65.76) --
	( 76.11, 65.76) --
	( 76.11, 65.76) --
	( 76.11, 65.76) --
	( 76.18, 65.75) --
	( 76.18, 65.75) --
	( 76.18, 65.75) --
	( 76.25, 65.76) --
	( 76.25, 65.76) --
	( 76.25, 65.76) --
	( 76.33, 65.77) --
	( 76.33, 65.77) --
	( 76.33, 65.77) --
	( 76.40, 65.76) --
	( 76.40, 65.76) --
	( 76.40, 65.76) --
	( 76.47, 65.75) --
	( 76.47, 65.75) --
	( 76.47, 65.75) --
	( 76.54, 65.76) --
	( 76.54, 65.76) --
	( 76.54, 65.76) --
	( 76.61, 65.77) --
	( 76.61, 65.77) --
	( 76.61, 65.77) --
	( 76.69, 65.73) --
	( 76.69, 65.73) --
	( 76.69, 65.73) --
	( 76.75, 65.74) --
	( 76.75, 65.74) --
	( 76.75, 65.74) --
	( 76.76, 65.74) --
	( 76.76, 65.74) --
	( 76.76, 65.74) --
	( 76.83, 65.75) --
	( 76.83, 65.75) --
	( 76.83, 65.75) --
	( 76.90, 65.74) --
	( 76.90, 65.74) --
	( 76.90, 65.74) --
	( 76.98, 65.75) --
	( 76.98, 65.75) --
	( 76.98, 65.75) --
	( 77.05, 65.76) --
	( 77.05, 65.76) --
	( 77.05, 65.76) --
	( 77.12, 65.76) --
	( 77.12, 65.76) --
	( 77.12, 65.76) --
	( 77.19, 65.76) --
	( 77.19, 65.76) --
	( 77.19, 65.76) --
	( 77.27, 65.75) --
	( 77.27, 65.75) --
	( 77.27, 65.75) --
	( 77.34, 65.74) --
	( 77.34, 65.74) --
	( 77.34, 65.74) --
	( 77.41, 65.76) --
	( 77.41, 65.76) --
	( 77.41, 65.76) --
	( 77.48, 65.74) --
	( 77.48, 65.74) --
	( 77.48, 65.74) --
	( 77.55, 65.75) --
	( 77.55, 65.75) --
	( 77.55, 65.75) --
	( 77.63, 65.76) --
	( 77.63, 65.76) --
	( 77.63, 65.76) --
	( 77.70, 65.75) --
	( 77.70, 65.75) --
	( 77.70, 65.75) --
	( 77.70, 65.75) --
	( 77.70, 65.75) --
	( 77.70, 65.75) --
	( 77.77, 65.74) --
	( 77.77, 65.74) --
	( 77.77, 65.74) --
	( 77.84, 65.77) --
	( 77.84, 65.77) --
	( 77.84, 65.77) --
	( 77.91, 65.75) --
	( 77.91, 65.75) --
	( 77.91, 65.75) --
	( 77.99, 65.77) --
	( 77.99, 65.77) --
	( 77.99, 65.77) --
	( 78.06, 65.74) --
	( 78.06, 65.74) --
	( 78.06, 65.74) --
	( 78.13, 65.75) --
	( 78.13, 65.75) --
	( 78.13, 65.75) --
	( 78.20, 65.75) --
	( 78.20, 65.75) --
	( 78.20, 65.75) --
	( 78.28, 65.74) --
	( 78.28, 65.74) --
	( 78.28, 65.74) --
	( 78.35, 65.76) --
	( 78.35, 65.76) --
	( 78.35, 65.76) --
	( 78.42, 65.75) --
	( 78.42, 65.75) --
	( 78.42, 65.75) --
	( 78.49, 65.74) --
	( 78.49, 65.74) --
	( 78.49, 65.74) --
	( 78.56, 65.75) --
	( 78.56, 65.75) --
	( 78.56, 65.75) --
	( 78.64, 65.75) --
	( 78.64, 65.75) --
	( 78.64, 65.75) --
	( 78.71, 65.74) --
	( 78.71, 65.74) --
	( 78.71, 65.74) --
	( 78.76, 65.75) --
	( 78.76, 65.75) --
	( 78.76, 65.75) --
	( 78.78, 65.76) --
	( 78.78, 65.76) --
	( 78.78, 65.76) --
	( 78.85, 65.74) --
	( 78.85, 65.74) --
	( 78.85, 65.74) --
	( 78.93, 65.76) --
	( 78.93, 65.76) --
	( 78.93, 65.76) --
	( 79.00, 65.76) --
	( 79.00, 65.76) --
	( 79.00, 65.76) --
	( 79.07, 65.74) --
	( 79.07, 65.74) --
	( 79.07, 65.74) --
	( 79.14, 65.75) --
	( 79.14, 65.75) --
	( 79.14, 65.75) --
	( 79.21, 65.78) --
	( 79.21, 65.78) --
	( 79.21, 65.78) --
	( 79.29, 65.75) --
	( 79.29, 65.75) --
	( 79.29, 65.75) --
	( 79.36, 65.74) --
	( 79.36, 65.74) --
	( 79.36, 65.74) --
	( 79.43, 65.78) --
	( 79.43, 65.78) --
	( 79.43, 65.78) --
	( 79.50, 65.76) --
	( 79.50, 65.76) --
	( 79.50, 65.76) --
	( 79.57, 65.74) --
	( 79.57, 65.74) --
	( 79.57, 65.74) --
	( 79.62, 65.74) --
	( 79.62, 65.74) --
	( 79.62, 65.74) --
	( 79.65, 65.74) --
	( 79.65, 65.74) --
	( 79.65, 65.74) --
	( 79.72, 65.75) --
	( 79.72, 65.75) --
	( 79.72, 65.75) --
	( 79.79, 65.77) --
	( 79.79, 65.77) --
	( 79.79, 65.77) --
	( 79.86, 65.74) --
	( 79.86, 65.74) --
	( 79.86, 65.74) --
	( 79.93, 65.76) --
	( 79.93, 65.76) --
	( 79.93, 65.76) --
	( 80.01, 65.78) --
	( 80.01, 65.78) --
	( 80.01, 65.78) --
	( 80.08, 65.76) --
	( 80.08, 65.76) --
	( 80.08, 65.76) --
	( 80.15, 65.76) --
	( 80.15, 65.76) --
	( 80.15, 65.76) --
	( 80.22, 65.76) --
	( 80.22, 65.76) --
	( 80.22, 65.76) --
	( 80.29, 65.76) --
	( 80.29, 65.76) --
	( 80.29, 65.76) --
	( 80.37, 65.75) --
	( 80.37, 65.75) --
	( 80.37, 65.75) --
	( 80.44, 65.74) --
	( 80.44, 65.74) --
	( 80.44, 65.74) --
	( 80.48, 65.76) --
	( 80.48, 65.76) --
	( 80.48, 65.76) --
	( 80.51, 65.77) --
	( 80.51, 65.77) --
	( 80.51, 65.77) --
	( 80.58, 65.78) --
	( 80.58, 65.78) --
	( 80.58, 65.78) --
	( 80.65, 65.73) --
	( 80.65, 65.73) --
	( 80.65, 65.73) --
	( 80.73, 65.75) --
	( 80.73, 65.75) --
	( 80.73, 65.75) --
	( 80.80, 65.76) --
	( 80.80, 65.76) --
	( 80.80, 65.76) --
	( 80.87, 65.73) --
	( 80.87, 65.73) --
	( 80.87, 65.73) --
	( 80.94, 65.73) --
	( 80.94, 65.73) --
	( 80.94, 65.73) --
	( 81.01, 65.74) --
	( 81.01, 65.74) --
	( 81.01, 65.74) --
	( 81.09, 65.76) --
	( 81.09, 65.76) --
	( 81.09, 65.76) --
	( 81.16, 65.75) --
	( 81.16, 65.75) --
	( 81.16, 65.75) --
	( 81.23, 65.74) --
	( 81.23, 65.74) --
	( 81.23, 65.74) --
	( 81.30, 65.72) --
	( 81.30, 65.72) --
	( 81.30, 65.72) --
	( 81.37, 65.75) --
	( 81.37, 65.75) --
	( 81.37, 65.75) --
	( 81.44, 65.75) --
	( 81.44, 65.75) --
	( 81.44, 65.75) --
	( 81.45, 65.75) --
	( 81.45, 65.75) --
	( 81.45, 65.75) --
	( 81.52, 65.75) --
	( 81.52, 65.75) --
	( 81.52, 65.75) --
	( 81.59, 65.77) --
	( 81.59, 65.77) --
	( 81.59, 65.77) --
	( 81.66, 65.76) --
	( 81.66, 65.76) --
	( 81.66, 65.76) --
	( 81.73, 65.76) --
	( 81.73, 65.76) --
	( 81.73, 65.76) --
	( 81.81, 65.75) --
	( 81.81, 65.75) --
	( 81.81, 65.75) --
	( 81.88, 65.74) --
	( 81.88, 65.74) --
	( 81.88, 65.74) --
	( 81.95, 65.75) --
	( 81.95, 65.75) --
	( 81.95, 65.75) --
	( 82.02, 65.77) --
	( 82.02, 65.77) --
	( 82.02, 65.77) --
	( 82.09, 65.76) --
	( 82.09, 65.76) --
	( 82.09, 65.76) --
	( 82.16, 65.78) --
	( 82.16, 65.78) --
	( 82.16, 65.78) --
	( 82.24, 65.75) --
	( 82.24, 65.75) --
	( 82.24, 65.75) --
	( 82.31, 65.72) --
	( 82.31, 65.72) --
	( 82.31, 65.72) --
	( 82.38, 65.77) --
	( 82.38, 65.77) --
	( 82.38, 65.77) --
	( 82.45, 65.73) --
	( 82.45, 65.73) --
	( 82.45, 65.73) --
	( 82.52, 65.76) --
	( 82.52, 65.76) --
	( 82.52, 65.76) --
	( 82.59, 65.73) --
	( 82.59, 65.73) --
	( 82.59, 65.73) --
	( 82.60, 65.73) --
	( 82.60, 65.73) --
	( 82.60, 65.73) --
	( 82.67, 65.76) --
	( 82.67, 65.76) --
	( 82.67, 65.76) --
	( 82.74, 65.76) --
	( 82.74, 65.76) --
	( 82.74, 65.76) --
	( 82.81, 65.73) --
	( 82.81, 65.73) --
	( 82.81, 65.73) --
	( 82.88, 65.77) --
	( 82.88, 65.77) --
	( 82.88, 65.77) --
	( 82.96, 65.76) --
	( 82.96, 65.76) --
	( 82.96, 65.76) --
	( 83.03, 65.74) --
	( 83.03, 65.74) --
	( 83.03, 65.74) --
	( 83.10, 65.74) --
	( 83.10, 65.74) --
	( 83.10, 65.74) --
	( 83.17, 65.74) --
	( 83.17, 65.74) --
	( 83.17, 65.74) --
	( 83.24, 65.75) --
	( 83.24, 65.75) --
	( 83.24, 65.75) --
	( 83.31, 65.77) --
	( 83.31, 65.77) --
	( 83.31, 65.77) --
	( 83.39, 65.74) --
	( 83.39, 65.74) --
	( 83.39, 65.74) --
	( 83.46, 65.76) --
	( 83.46, 65.76) --
	( 83.46, 65.76) --
	( 83.53, 65.76) --
	( 83.53, 65.76) --
	( 83.53, 65.76) --
	( 83.60, 65.74) --
	( 83.60, 65.74) --
	( 83.60, 65.74) --
	( 83.67, 65.76) --
	( 83.67, 65.76) --
	( 83.67, 65.76) --
	( 83.75, 65.77) --
	( 83.75, 65.77) --
	( 83.75, 65.77) --
	( 83.82, 65.75) --
	( 83.82, 65.75) --
	( 83.82, 65.75) --
	( 83.89, 65.74) --
	( 83.89, 65.74) --
	( 83.89, 65.74) --
	( 83.96, 65.76) --
	( 83.96, 65.76) --
	( 83.96, 65.76) --
	( 84.03, 65.78) --
	( 84.03, 65.78) --
	( 84.03, 65.78) --
	( 84.10, 65.74) --
	( 84.10, 65.74) --
	( 84.10, 65.74) --
	( 84.12, 65.74) --
	( 84.12, 65.74) --
	( 84.12, 65.74) --
	( 84.18, 65.75) --
	( 84.18, 65.75) --
	( 84.18, 65.75) --
	( 84.25, 65.74) --
	( 84.25, 65.74) --
	( 84.25, 65.74) --
	( 84.32, 65.75) --
	( 84.32, 65.75) --
	( 84.32, 65.75) --
	( 84.39, 65.74) --
	( 84.39, 65.74) --
	( 84.39, 65.74) --
	( 84.46, 65.75) --
	( 84.46, 65.75) --
	( 84.46, 65.75) --
	( 84.54, 65.76) --
	( 84.54, 65.76) --
	( 84.54, 65.76) --
	( 84.61, 65.74) --
	( 84.61, 65.74) --
	( 84.61, 65.74) --
	( 84.68, 65.76) --
	( 84.68, 65.76) --
	( 84.68, 65.76) --
	( 84.75, 65.77) --
	( 84.75, 65.77) --
	( 84.75, 65.77) --
	( 84.82, 65.76) --
	( 84.82, 65.76) --
	( 84.82, 65.76) --
	( 84.89, 65.74) --
	( 84.89, 65.74) --
	( 84.89, 65.74) --
	( 84.96, 65.73) --
	( 84.96, 65.73) --
	( 84.96, 65.73) --
	( 85.04, 65.74) --
	( 85.04, 65.74) --
	( 85.04, 65.74) --
	( 85.11, 65.76) --
	( 85.11, 65.76) --
	( 85.11, 65.76) --
	( 85.18, 65.75) --
	( 85.18, 65.75) --
	( 85.18, 65.75) --
	( 85.25, 65.76) --
	( 85.25, 65.76) --
	( 85.25, 65.76) --
	( 85.32, 65.76) --
	( 85.32, 65.76) --
	( 85.32, 65.76) --
	( 85.40, 65.73) --
	( 85.40, 65.73) --
	( 85.40, 65.73) --
	( 85.47, 65.77) --
	( 85.47, 65.77) --
	( 85.47, 65.77) --
	( 85.47, 65.77) --
	( 85.47, 65.77) --
	( 85.47, 65.77) --
	( 85.54, 65.76) --
	( 85.54, 65.76) --
	( 85.54, 65.76) --
	( 85.61, 65.75) --
	( 85.61, 65.75) --
	( 85.61, 65.75) --
	( 85.68, 65.76) --
	( 85.68, 65.76) --
	( 85.68, 65.76) --
	( 85.75, 65.74) --
	( 85.75, 65.74) --
	( 85.75, 65.74) --
	( 85.83, 65.77) --
	( 85.83, 65.77) --
	( 85.83, 65.77) --
	( 85.90, 65.77) --
	( 85.90, 65.77) --
	( 85.90, 65.77) --
	( 85.97, 65.74) --
	( 85.97, 65.74) --
	( 85.97, 65.74) --
	( 86.04, 65.76) --
	( 86.04, 65.76) --
	( 86.04, 65.76) --
	( 86.11, 65.78) --
	( 86.11, 65.78) --
	( 86.11, 65.78) --
	( 86.18, 65.75) --
	( 86.18, 65.75) --
	( 86.18, 65.75) --
	( 86.25, 65.76) --
	( 86.25, 65.76) --
	( 86.25, 65.76) --
	( 86.33, 65.74) --
	( 86.33, 65.74) --
	( 86.33, 65.74) --
	( 86.40, 65.73) --
	( 86.40, 65.73) --
	( 86.40, 65.73) --
	( 86.47, 65.76) --
	( 86.47, 65.76) --
	( 86.47, 65.76) --
	( 86.54, 65.73) --
	( 86.54, 65.73) --
	( 86.54, 65.73) --
	( 86.61, 65.76) --
	( 86.61, 65.76) --
	( 86.61, 65.76) --
	( 86.68, 65.76) --
	( 86.68, 65.76) --
	( 86.68, 65.76) --
	( 86.76, 65.74) --
	( 86.76, 65.74) --
	( 86.76, 65.74) --
	( 86.83, 65.76) --
	( 86.83, 65.76) --
	( 86.83, 65.76) --
	( 86.90, 65.77) --
	( 86.90, 65.77) --
	( 86.90, 65.77) --
	( 86.90, 65.77) --
	( 86.90, 65.77) --
	( 86.90, 65.77) --
	( 86.97, 65.73) --
	( 86.97, 65.73) --
	( 86.97, 65.73) --
	( 87.04, 65.76) --
	( 87.04, 65.76) --
	( 87.04, 65.76) --
	( 87.11, 65.75) --
	( 87.11, 65.75) --
	( 87.11, 65.75) --
	( 87.19, 65.76) --
	( 87.19, 65.76) --
	( 87.19, 65.76) --
	( 87.26, 65.76) --
	( 87.26, 65.76) --
	( 87.26, 65.76) --
	( 87.33, 65.74) --
	( 87.33, 65.74) --
	( 87.33, 65.74) --
	( 87.40, 65.78) --
	( 87.40, 65.78) --
	( 87.40, 65.78) --
	( 87.47, 65.76) --
	( 87.47, 65.76) --
	( 87.47, 65.76) --
	( 87.54, 65.73) --
	( 87.54, 65.73) --
	( 87.54, 65.73) --
	( 87.62, 65.74) --
	( 87.62, 65.74) --
	( 87.62, 65.74) --
	( 87.69, 65.76) --
	( 87.69, 65.76) --
	( 87.69, 65.76) --
	( 87.76, 65.72) --
	( 87.76, 65.72) --
	( 87.76, 65.72) --
	( 87.83, 65.75) --
	( 87.83, 65.75) --
	( 87.83, 65.75) --
	( 87.90, 65.75) --
	( 87.90, 65.75) --
	( 87.90, 65.75) --
	( 87.97, 65.76) --
	( 87.97, 65.76) --
	( 87.97, 65.76) --
	( 88.04, 65.76) --
	( 88.04, 65.76) --
	( 88.04, 65.76) --
	( 88.12, 65.75) --
	( 88.12, 65.75) --
	( 88.12, 65.75) --
	( 88.19, 65.74) --
	( 88.19, 65.74) --
	( 88.19, 65.74) --
	( 88.26, 65.76) --
	( 88.26, 65.76) --
	( 88.26, 65.76) --
	( 88.33, 65.76) --
	( 88.33, 65.76) --
	( 88.33, 65.76) --
	( 88.40, 65.77) --
	( 88.40, 65.77) --
	( 88.40, 65.77) --
	( 88.47, 65.78) --
	( 88.47, 65.78) --
	( 88.47, 65.78) --
	( 88.54, 65.74) --
	( 88.54, 65.74) --
	( 88.54, 65.74) --
	( 88.62, 65.74) --
	( 88.62, 65.74) --
	( 88.62, 65.74) --
	( 88.69, 65.76) --
	( 88.69, 65.76) --
	( 88.69, 65.76) --
	( 88.76, 65.75) --
	( 88.76, 65.75) --
	( 88.76, 65.75) --
	( 88.83, 65.75) --
	( 88.83, 65.75) --
	( 88.83, 65.75) --
	( 88.90, 65.72) --
	( 88.90, 65.72) --
	( 88.90, 65.72) --
	( 88.92, 65.72) --
	( 88.92, 65.72) --
	( 88.92, 65.72) --
	( 88.97, 65.76) --
	( 88.97, 65.76) --
	( 88.97, 65.76) --
	( 89.05, 65.77) --
	( 89.05, 65.77) --
	( 89.05, 65.77) --
	( 89.12, 65.73) --
	( 89.12, 65.73) --
	( 89.12, 65.73) --
	( 89.19, 65.75) --
	( 89.19, 65.75) --
	( 89.19, 65.75) --
	( 89.26, 65.77) --
	( 89.26, 65.77) --
	( 89.26, 65.77) --
	( 89.33, 65.76) --
	( 89.33, 65.76) --
	( 89.33, 65.76) --
	( 89.40, 65.76) --
	( 89.40, 65.76) --
	( 89.40, 65.76) --
	( 89.47, 65.76) --
	( 89.47, 65.76) --
	( 89.47, 65.76) --
	( 89.55, 65.76) --
	( 89.55, 65.76) --
	( 89.55, 65.76) --
	( 89.62, 65.75) --
	( 89.62, 65.75) --
	( 89.62, 65.75) --
	( 89.69, 65.74) --
	( 89.69, 65.74) --
	( 89.69, 65.74) --
	( 89.76, 65.75) --
	( 89.76, 65.75) --
	( 89.76, 65.75) --
	( 89.83, 65.76) --
	( 89.83, 65.76) --
	( 89.83, 65.76) --
	( 89.90, 65.76) --
	( 89.90, 65.76) --
	( 89.90, 65.76) --
	( 89.97, 65.76) --
	( 89.97, 65.76) --
	( 89.97, 65.76) --
	( 90.04, 65.77) --
	( 90.04, 65.77) --
	( 90.04, 65.77) --
	( 90.12, 65.75) --
	( 90.12, 65.75) --
	( 90.12, 65.75) --
	( 90.19, 65.74) --
	( 90.19, 65.74) --
	( 90.19, 65.74) --
	( 90.26, 65.74) --
	( 90.26, 65.74) --
	( 90.26, 65.74) --
	( 90.33, 65.75) --
	( 90.33, 65.75) --
	( 90.33, 65.75) --
	( 90.40, 65.75) --
	( 90.40, 65.75) --
	( 90.40, 65.75) --
	( 90.47, 65.73) --
	( 90.47, 65.73) --
	( 90.47, 65.73) --
	( 90.54, 65.76) --
	( 90.54, 65.76) --
	( 90.54, 65.76) --
	( 90.62, 65.74) --
	( 90.62, 65.74) --
	( 90.62, 65.74) --
	( 90.64, 65.74) --
	( 90.64, 65.74) --
	( 90.64, 65.74) --
	( 90.69, 65.74) --
	( 90.69, 65.74) --
	( 90.69, 65.74) --
	( 90.76, 65.74) --
	( 90.76, 65.74) --
	( 90.76, 65.74) --
	( 90.83, 65.76) --
	( 90.83, 65.76) --
	( 90.83, 65.76) --
	( 90.90, 65.77) --
	( 90.90, 65.77) --
	( 90.90, 65.77) --
	( 90.97, 65.75) --
	( 90.97, 65.75) --
	( 90.97, 65.75) --
	( 91.04, 65.76) --
	( 91.04, 65.76) --
	( 91.04, 65.76) --
	( 91.12, 65.78) --
	( 91.12, 65.78) --
	( 91.12, 65.78) --
	( 91.19, 65.74) --
	( 91.19, 65.74) --
	( 91.19, 65.74) --
	( 91.26, 65.74) --
	( 91.26, 65.74) --
	( 91.26, 65.74) --
	( 91.33, 65.76) --
	( 91.33, 65.76) --
	( 91.33, 65.76) --
	( 91.40, 65.75) --
	( 91.40, 65.75) --
	( 91.40, 65.75) --
	( 91.47, 65.77) --
	( 91.47, 65.77) --
	( 91.47, 65.77) --
	( 91.54, 65.75) --
	( 91.54, 65.75) --
	( 91.54, 65.75) --
	( 91.61, 65.79) --
	( 91.61, 65.79) --
	( 91.61, 65.79) --
	( 91.69, 65.74) --
	( 91.69, 65.74) --
	( 91.69, 65.74) --
	( 91.76, 65.75) --
	( 91.76, 65.75) --
	( 91.76, 65.75) --
	( 91.83, 65.75) --
	( 91.83, 65.75) --
	( 91.83, 65.75) --
	( 91.90, 65.76) --
	( 91.90, 65.76) --
	( 91.90, 65.76) --
	( 91.97, 65.76) --
	( 91.97, 65.76) --
	( 91.97, 65.76) --
	( 92.04, 65.75) --
	( 92.04, 65.75) --
	( 92.04, 65.75) --
	( 92.11, 65.74) --
	( 92.11, 65.74) --
	( 92.11, 65.74) --
	( 92.18, 65.78) --
	( 92.18, 65.78) --
	( 92.18, 65.78) --
	( 92.26, 65.77) --
	( 92.26, 65.77) --
	( 92.26, 65.77) --
	( 92.33, 65.75) --
	( 92.33, 65.75) --
	( 92.33, 65.75) --
	( 92.40, 65.76) --
	( 92.40, 65.76) --
	( 92.40, 65.76) --
	( 92.47, 65.75) --
	( 92.47, 65.75) --
	( 92.47, 65.75) --
	( 92.54, 65.77) --
	( 92.54, 65.77) --
	( 92.54, 65.77) --
	( 92.61, 65.74) --
	( 92.61, 65.74) --
	( 92.61, 65.74) --
	( 92.65, 65.74) --
	( 92.65, 65.74) --
	( 92.65, 65.74) --
	( 92.68, 65.74) --
	( 92.68, 65.74) --
	( 92.68, 65.74) --
	( 92.75, 65.77) --
	( 92.75, 65.77) --
	( 92.75, 65.77) --
	( 92.83, 65.73) --
	( 92.83, 65.73) --
	( 92.83, 65.73) --
	( 92.90, 65.74) --
	( 92.90, 65.74) --
	( 92.90, 65.74) --
	( 92.97, 65.76) --
	( 92.97, 65.76) --
	( 92.97, 65.76) --
	( 93.04, 65.73) --
	( 93.04, 65.73) --
	( 93.04, 65.73) --
	( 93.11, 65.75) --
	( 93.11, 65.75) --
	( 93.11, 65.75) --
	( 93.18, 65.76) --
	( 93.18, 65.76) --
	( 93.18, 65.76) --
	( 93.25, 65.73) --
	( 93.25, 65.73) --
	( 93.25, 65.73) --
	( 93.32, 65.78) --
	( 93.32, 65.78) --
	( 93.32, 65.78) --
	( 93.39, 65.74) --
	( 93.39, 65.74) --
	( 93.39, 65.74) --
	( 93.46, 65.75) --
	( 93.46, 65.75) --
	( 93.46, 65.75) --
	( 93.54, 65.74) --
	( 93.54, 65.74) --
	( 93.54, 65.74) --
	( 93.61, 65.72) --
	( 93.61, 65.72) --
	( 93.61, 65.72) --
	( 93.68, 65.75) --
	( 93.68, 65.75) --
	( 93.68, 65.75) --
	( 93.75, 65.77) --
	( 93.75, 65.77) --
	( 93.75, 65.77) --
	( 93.82, 65.75) --
	( 93.82, 65.75) --
	( 93.82, 65.75) --
	( 93.89, 65.77) --
	( 93.89, 65.77) --
	( 93.89, 65.77) --
	( 93.96, 65.76) --
	( 93.96, 65.76) --
	( 93.96, 65.76) --
	( 94.03, 65.75) --
	( 94.03, 65.75) --
	( 94.03, 65.75) --
	( 94.10, 65.74) --
	( 94.10, 65.74) --
	( 94.10, 65.74) --
	( 94.18, 65.73) --
	( 94.18, 65.73) --
	( 94.18, 65.73) --
	( 94.25, 65.75) --
	( 94.25, 65.75) --
	( 94.25, 65.75) --
	( 94.32, 65.74) --
	( 94.32, 65.74) --
	( 94.32, 65.74) --
	( 94.39, 65.75) --
	( 94.39, 65.75) --
	( 94.39, 65.75) --
	( 94.46, 65.76) --
	( 94.46, 65.76) --
	( 94.46, 65.76) --
	( 94.53, 65.76) --
	( 94.53, 65.76) --
	( 94.53, 65.76) --
	( 94.60, 65.75) --
	( 94.60, 65.75) --
	( 94.60, 65.75) --
	( 94.67, 65.75) --
	( 94.67, 65.75) --
	( 94.67, 65.75) --
	( 94.75, 65.73) --
	( 94.75, 65.73) --
	( 94.75, 65.73) --
	( 94.82, 65.76) --
	( 94.82, 65.76) --
	( 94.82, 65.76) --
	( 94.89, 65.75) --
	( 94.89, 65.75) --
	( 94.89, 65.75) --
	( 94.95, 65.74) --
	( 94.95, 65.74) --
	( 94.95, 65.74) --
	( 94.96, 65.74) --
	( 94.96, 65.74) --
	( 94.96, 65.74) --
	( 95.03, 65.78) --
	( 95.03, 65.78) --
	( 95.03, 65.78) --
	( 95.10, 65.74) --
	( 95.10, 65.74) --
	( 95.10, 65.74) --
	( 95.17, 65.74) --
	( 95.17, 65.74) --
	( 95.17, 65.74) --
	( 95.24, 65.73) --
	( 95.24, 65.73) --
	( 95.24, 65.73) --
	( 95.31, 65.76) --
	( 95.31, 65.76) --
	( 95.31, 65.76) --
	( 95.38, 65.72) --
	( 95.38, 65.72) --
	( 95.38, 65.72) --
	( 95.45, 65.76) --
	( 95.45, 65.76) --
	( 95.45, 65.76) --
	( 95.53, 65.74) --
	( 95.53, 65.74) --
	( 95.53, 65.74) --
	( 95.60, 65.75) --
	( 95.60, 65.75) --
	( 95.60, 65.75) --
	( 95.67, 65.77) --
	( 95.67, 65.77) --
	( 95.67, 65.77) --
	( 95.74, 65.73) --
	( 95.74, 65.73) --
	( 95.74, 65.73) --
	( 95.81, 65.77) --
	( 95.81, 65.77) --
	( 95.81, 65.77) --
	( 95.88, 65.76) --
	( 95.88, 65.76) --
	( 95.88, 65.76) --
	( 95.95, 65.74) --
	( 95.95, 65.74) --
	( 95.95, 65.74) --
	( 96.02, 65.76) --
	( 96.02, 65.76) --
	( 96.02, 65.76) --
	( 96.09, 65.77) --
	( 96.09, 65.77) --
	( 96.09, 65.77) --
	( 96.16, 65.74) --
	( 96.16, 65.74) --
	( 96.16, 65.74) --
	( 96.24, 65.76) --
	( 96.24, 65.76) --
	( 96.24, 65.76) --
	( 96.31, 65.74) --
	( 96.31, 65.74) --
	( 96.31, 65.74) --
	( 96.38, 65.74) --
	( 96.38, 65.74) --
	( 96.38, 65.74) --
	( 96.45, 65.76) --
	( 96.45, 65.76) --
	( 96.45, 65.76) --
	( 96.52, 65.75) --
	( 96.52, 65.75) --
	( 96.52, 65.75) --
	( 96.59, 65.75) --
	( 96.59, 65.75) --
	( 96.59, 65.75) --
	( 96.66, 65.78) --
	( 96.66, 65.78) --
	( 96.66, 65.78) --
	( 96.73, 65.76) --
	( 96.73, 65.76) --
	( 96.73, 65.76) --
	( 96.77, 65.75) --
	( 96.77, 65.75) --
	( 96.77, 65.75) --
	( 96.80, 65.75) --
	( 96.80, 65.75) --
	( 96.80, 65.75) --
	( 96.87, 65.75) --
	( 96.87, 65.75) --
	( 96.87, 65.75) --
	( 96.94, 65.75) --
	( 96.94, 65.75) --
	( 96.94, 65.75) --
	( 97.02, 65.77) --
	( 97.02, 65.77) --
	( 97.02, 65.77) --
	( 97.09, 65.75) --
	( 97.09, 65.75) --
	( 97.09, 65.75) --
	( 97.16, 65.77) --
	( 97.16, 65.77) --
	( 97.16, 65.77) --
	( 97.23, 65.76) --
	( 97.23, 65.76) --
	( 97.23, 65.76) --
	( 97.30, 65.74) --
	( 97.30, 65.74) --
	( 97.30, 65.74) --
	( 97.37, 65.76) --
	( 97.37, 65.76) --
	( 97.37, 65.76) --
	( 97.44, 65.76) --
	( 97.44, 65.76) --
	( 97.44, 65.76) --
	( 97.51, 65.76) --
	( 97.51, 65.76) --
	( 97.51, 65.76) --
	( 97.58, 65.75) --
	( 97.58, 65.75) --
	( 97.58, 65.75) --
	( 97.65, 65.73) --
	( 97.65, 65.73) --
	( 97.65, 65.73) --
	( 97.72, 65.75) --
	( 97.72, 65.75) --
	( 97.72, 65.75) --
	( 97.80, 65.76) --
	( 97.80, 65.76) --
	( 97.80, 65.76) --
	( 97.87, 65.75) --
	( 97.87, 65.75) --
	( 97.87, 65.75) --
	( 97.94, 65.75) --
	( 97.94, 65.75) --
	( 97.94, 65.75) --
	( 98.01, 65.75) --
	( 98.01, 65.75) --
	( 98.01, 65.75) --
	( 98.08, 65.73) --
	( 98.08, 65.73) --
	( 98.08, 65.73) --
	( 98.15, 65.75) --
	( 98.15, 65.75) --
	( 98.15, 65.75) --
	( 98.22, 65.76) --
	( 98.22, 65.76) --
	( 98.22, 65.76) --
	( 98.29, 65.76) --
	( 98.29, 65.76) --
	( 98.29, 65.76) --
	( 98.36, 65.75) --
	( 98.36, 65.75) --
	( 98.36, 65.75) --
	( 98.43, 65.73) --
	( 98.43, 65.73) --
	( 98.43, 65.73) --
	( 98.50, 65.74) --
	( 98.50, 65.74) --
	( 98.50, 65.74) --
	( 98.57, 65.76) --
	( 98.57, 65.76) --
	( 98.57, 65.76) --
	( 98.65, 65.75) --
	( 98.65, 65.75) --
	( 98.65, 65.75) --
	( 98.72, 65.76) --
	( 98.72, 65.76) --
	( 98.72, 65.76) --
	( 98.79, 65.75) --
	( 98.79, 65.75) --
	( 98.79, 65.75) --
	( 98.86, 65.74) --
	( 98.86, 65.74) --
	( 98.86, 65.74) --
	( 98.93, 65.76) --
	( 98.93, 65.76) --
	( 98.93, 65.76) --
	( 99.00, 65.74) --
	( 99.00, 65.74) --
	( 99.00, 65.74) --
	( 99.07, 65.74) --
	( 99.07, 65.74) --
	( 99.07, 65.74) --
	( 99.07, 65.74) --
	( 99.07, 65.74) --
	( 99.07, 65.74) --
	( 99.14, 65.74) --
	( 99.14, 65.74) --
	( 99.14, 65.74) --
	( 99.21, 65.76) --
	( 99.21, 65.76) --
	( 99.21, 65.76) --
	( 99.28, 65.74) --
	( 99.28, 65.74) --
	( 99.28, 65.74) --
	( 99.35, 65.77) --
	( 99.35, 65.77) --
	( 99.35, 65.77) --
	( 99.42, 65.73) --
	( 99.42, 65.73) --
	( 99.42, 65.73) --
	( 99.49, 65.75) --
	( 99.49, 65.75) --
	( 99.49, 65.75) --
	( 99.56, 65.77) --
	( 99.56, 65.77) --
	( 99.56, 65.77) --
	( 99.64, 65.75) --
	( 99.64, 65.75) --
	( 99.64, 65.75) --
	( 99.71, 65.76) --
	( 99.71, 65.76) --
	( 99.71, 65.76) --
	( 99.78, 65.75) --
	( 99.78, 65.75) --
	( 99.78, 65.75) --
	( 99.85, 65.72) --
	( 99.85, 65.72) --
	( 99.85, 65.72) --
	( 99.92, 65.74) --
	( 99.92, 65.74) --
	( 99.92, 65.74) --
	( 99.99, 65.76) --
	( 99.99, 65.76) --
	( 99.99, 65.76) --
	(100.06, 65.75) --
	(100.06, 65.75) --
	(100.06, 65.75) --
	(100.13, 65.76) --
	(100.13, 65.76) --
	(100.13, 65.76) --
	(100.20, 65.73) --
	(100.20, 65.73) --
	(100.20, 65.73) --
	(100.27, 65.74) --
	(100.27, 65.74) --
	(100.27, 65.74) --
	(100.34, 65.75) --
	(100.34, 65.75) --
	(100.34, 65.75) --
	(100.41, 65.73) --
	(100.41, 65.73) --
	(100.41, 65.73) --
	(100.48, 65.75) --
	(100.48, 65.75) --
	(100.48, 65.75) --
	(100.55, 65.78) --
	(100.55, 65.78) --
	(100.55, 65.78) --
	(100.62, 65.77) --
	(100.62, 65.77) --
	(100.62, 65.77) --
	(100.70, 65.74) --
	(100.70, 65.74) --
	(100.70, 65.74) --
	(100.77, 65.76) --
	(100.77, 65.76) --
	(100.77, 65.76) --
	(100.84, 65.74) --
	(100.84, 65.74) --
	(100.84, 65.74) --
	(100.91, 65.74) --
	(100.91, 65.74) --
	(100.91, 65.74) --
	(100.98, 65.74) --
	(100.98, 65.74) --
	(100.98, 65.74) --
	(101.05, 65.78) --
	(101.05, 65.78) --
	(101.05, 65.78) --
	(101.12, 65.76) --
	(101.12, 65.76) --
	(101.12, 65.76) --
	(101.19, 65.76) --
	(101.19, 65.76) --
	(101.19, 65.76) --
	(101.26, 65.72) --
	(101.26, 65.72) --
	(101.26, 65.72) --
	(101.33, 65.76) --
	(101.33, 65.76) --
	(101.33, 65.76) --
	(101.40, 65.74) --
	(101.40, 65.74) --
	(101.40, 65.74) --
	(101.47, 65.72) --
	(101.47, 65.72) --
	(101.47, 65.72) --
	(101.54, 65.76) --
	(101.54, 65.76) --
	(101.54, 65.76) --
	(101.61, 65.74) --
	(101.61, 65.74) --
	(101.61, 65.74) --
	(101.66, 65.75) --
	(101.66, 65.75) --
	(101.66, 65.75) --
	(101.68, 65.76) --
	(101.68, 65.76) --
	(101.68, 65.76) --
	(101.75, 65.74) --
	(101.75, 65.74) --
	(101.75, 65.74) --
	(101.82, 65.78) --
	(101.82, 65.78) --
	(101.82, 65.78) --
	(101.90, 65.76) --
	(101.90, 65.76) --
	(101.90, 65.76) --
	(101.97, 65.75) --
	(101.97, 65.75) --
	(101.97, 65.75) --
	(102.04, 65.75) --
	(102.04, 65.75) --
	(102.04, 65.75) --
	(102.11, 65.75) --
	(102.11, 65.75) --
	(102.11, 65.75) --
	(102.18, 65.73) --
	(102.18, 65.73) --
	(102.18, 65.73) --
	(102.25, 65.75) --
	(102.25, 65.75) --
	(102.25, 65.75) --
	(102.32, 65.77) --
	(102.32, 65.77) --
	(102.32, 65.77) --
	(102.39, 65.76) --
	(102.39, 65.76) --
	(102.39, 65.76) --
	(102.46, 65.73) --
	(102.46, 65.73) --
	(102.46, 65.73) --
	(102.53, 65.73) --
	(102.53, 65.73) --
	(102.53, 65.73) --
	(102.60, 65.77) --
	(102.60, 65.77) --
	(102.60, 65.77) --
	(102.67, 65.76) --
	(102.67, 65.76) --
	(102.67, 65.76) --
	(102.74, 65.74) --
	(102.74, 65.74) --
	(102.74, 65.74) --
	(102.81, 65.72) --
	(102.81, 65.72) --
	(102.81, 65.72) --
	(102.88, 65.78) --
	(102.88, 65.78) --
	(102.88, 65.78) --
	(102.95, 65.75) --
	(102.95, 65.75) --
	(102.95, 65.75) --
	(103.02, 65.76) --
	(103.02, 65.76) --
	(103.02, 65.76) --
	(103.09, 65.74) --
	(103.09, 65.74) --
	(103.09, 65.74) --
	(103.16, 65.76) --
	(103.16, 65.76) --
	(103.16, 65.76) --
	(103.23, 65.75) --
	(103.23, 65.75) --
	(103.23, 65.75) --
	(103.31, 65.72) --
	(103.31, 65.72) --
	(103.31, 65.72) --
	(103.37, 65.76) --
	(103.37, 65.76) --
	(103.37, 65.76) --
	(103.45, 65.75) --
	(103.45, 65.75) --
	(103.45, 65.75) --
	(103.52, 65.76) --
	(103.52, 65.76) --
	(103.52, 65.76) --
	(103.59, 65.75) --
	(103.59, 65.75) --
	(103.59, 65.75) --
	(103.66, 65.76) --
	(103.66, 65.76) --
	(103.66, 65.76) --
	(103.73, 65.75) --
	(103.73, 65.75) --
	(103.73, 65.75) --
	(103.80, 65.74) --
	(103.80, 65.74) --
	(103.80, 65.74) --
	(103.87, 65.74) --
	(103.87, 65.74) --
	(103.87, 65.74) --
	(103.94, 65.74) --
	(103.94, 65.74) --
	(103.94, 65.74) --
	(104.01, 65.75) --
	(104.01, 65.75) --
	(104.01, 65.75) --
	(104.08, 65.74) --
	(104.08, 65.74) --
	(104.08, 65.74) --
	(104.15, 65.77) --
	(104.15, 65.77) --
	(104.15, 65.77) --
	(104.22, 65.78) --
	(104.22, 65.78) --
	(104.22, 65.78) --
	(104.29, 65.75) --
	(104.29, 65.75) --
	(104.29, 65.75) --
	(104.36, 65.74) --
	(104.36, 65.74) --
	(104.36, 65.74) --
	(104.43, 65.76) --
	(104.43, 65.76) --
	(104.43, 65.76) --
	(104.44, 65.76) --
	(104.44, 65.76) --
	(104.44, 65.76) --
	(104.50, 65.75) --
	(104.50, 65.75) --
	(104.50, 65.75) --
	(104.57, 65.76) --
	(104.57, 65.76) --
	(104.57, 65.76) --
	(104.64, 65.76) --
	(104.64, 65.76) --
	(104.64, 65.76) --
	(104.71, 65.77) --
	(104.71, 65.77) --
	(104.71, 65.77) --
	(104.78, 65.75) --
	(104.78, 65.75) --
	(104.78, 65.75) --
	(104.85, 65.77) --
	(104.85, 65.77) --
	(104.85, 65.77) --
	(104.92, 65.77) --
	(104.92, 65.77) --
	(104.92, 65.77) --
	(104.99, 65.77) --
	(104.99, 65.77) --
	(104.99, 65.77) --
	(105.06, 65.75) --
	(105.06, 65.75) --
	(105.06, 65.75) --
	(105.14, 65.75) --
	(105.14, 65.75) --
	(105.14, 65.75) --
	(105.20, 65.76) --
	(105.20, 65.76) --
	(105.20, 65.76) --
	(105.28, 65.75) --
	(105.28, 65.75) --
	(105.28, 65.75) --
	(105.34, 65.75) --
	(105.34, 65.75) --
	(105.34, 65.75) --
	(105.42, 65.71) --
	(105.42, 65.71) --
	(105.42, 65.71) --
	(105.49, 65.76) --
	(105.49, 65.76) --
	(105.49, 65.76) --
	(105.56, 65.75) --
	(105.56, 65.75) --
	(105.56, 65.75) --
	(105.63, 65.76) --
	(105.63, 65.76) --
	(105.63, 65.76) --
	(105.70, 65.75) --
	(105.70, 65.75) --
	(105.70, 65.75) --
	(105.77, 65.77) --
	(105.77, 65.77) --
	(105.77, 65.77) --
	(105.84, 65.75) --
	(105.84, 65.75) --
	(105.84, 65.75) --
	(105.91, 65.74) --
	(105.91, 65.74) --
	(105.91, 65.74) --
	(105.98, 65.76) --
	(105.98, 65.76) --
	(105.98, 65.76) --
	(106.05, 65.76) --
	(106.05, 65.76) --
	(106.05, 65.76) --
	(106.12, 65.75) --
	(106.12, 65.75) --
	(106.12, 65.75) --
	(106.19, 65.72) --
	(106.19, 65.72) --
	(106.19, 65.72) --
	(106.26, 65.75) --
	(106.26, 65.75) --
	(106.26, 65.75) --
	(106.33, 65.76) --
	(106.33, 65.76) --
	(106.33, 65.76) --
	(106.40, 65.73) --
	(106.40, 65.73) --
	(106.40, 65.73) --
	(106.47, 65.75) --
	(106.47, 65.75) --
	(106.47, 65.75) --
	(106.54, 65.75) --
	(106.54, 65.75) --
	(106.54, 65.75) --
	(106.61, 65.74) --
	(106.61, 65.74) --
	(106.61, 65.74) --
	(106.68, 65.74) --
	(106.68, 65.74) --
	(106.68, 65.74) --
	(106.75, 65.76) --
	(106.75, 65.76) --
	(106.75, 65.76) --
	(106.82, 65.75) --
	(106.82, 65.75) --
	(106.82, 65.75) --
	(106.89, 65.75) --
	(106.89, 65.75) --
	(106.89, 65.75) --
	(106.96, 65.73) --
	(106.96, 65.73) --
	(106.96, 65.73) --
	(107.03, 65.74) --
	(107.03, 65.74) --
	(107.03, 65.74) --
	(107.10, 65.74) --
	(107.10, 65.74) --
	(107.10, 65.74) --
	(107.17, 65.74) --
	(107.17, 65.74) --
	(107.17, 65.74) --
	(107.24, 65.74) --
	(107.24, 65.74) --
	(107.24, 65.74) --
	(107.31, 65.78) --
	(107.31, 65.78) --
	(107.31, 65.78) --
	(107.38, 65.76) --
	(107.38, 65.76) --
	(107.38, 65.76) --
	(107.41, 65.75) --
	(107.41, 65.75) --
	(107.41, 65.75) --
	(107.45, 65.74) --
	(107.45, 65.74) --
	(107.45, 65.74) --
	(107.52, 65.73) --
	(107.52, 65.73) --
	(107.52, 65.73) --
	(107.59, 65.75) --
	(107.59, 65.75) --
	(107.59, 65.75) --
	(107.66, 65.76) --
	(107.66, 65.76) --
	(107.66, 65.76) --
	(107.73, 65.73) --
	(107.73, 65.73) --
	(107.73, 65.73) --
	(107.80, 65.74) --
	(107.80, 65.74) --
	(107.80, 65.74) --
	(107.87, 65.75) --
	(107.87, 65.75) --
	(107.87, 65.75) --
	(107.94, 65.73) --
	(107.94, 65.73) --
	(107.94, 65.73) --
	(108.01, 65.75) --
	(108.01, 65.75) --
	(108.01, 65.75) --
	(108.08, 65.77) --
	(108.08, 65.77) --
	(108.08, 65.77) --
	(108.15, 65.76) --
	(108.15, 65.76) --
	(108.15, 65.76) --
	(108.22, 65.75) --
	(108.22, 65.75) --
	(108.22, 65.75) --
	(108.29, 65.76) --
	(108.29, 65.76) --
	(108.29, 65.76) --
	(108.36, 65.76) --
	(108.36, 65.76) --
	(108.36, 65.76) --
	(108.43, 65.76) --
	(108.43, 65.76) --
	(108.43, 65.76) --
	(108.50, 65.73) --
	(108.50, 65.73) --
	(108.50, 65.73) --
	(108.57, 65.76) --
	(108.57, 65.76) --
	(108.57, 65.76) --
	(108.64, 65.75) --
	(108.64, 65.75) --
	(108.64, 65.75) --
	(108.71, 65.76) --
	(108.71, 65.76) --
	(108.71, 65.76) --
	(108.78, 65.74) --
	(108.78, 65.74) --
	(108.78, 65.74) --
	(108.85, 65.76) --
	(108.85, 65.76) --
	(108.85, 65.76) --
	(108.92, 65.73) --
	(108.92, 65.73) --
	(108.92, 65.73) --
	(108.99, 65.76) --
	(108.99, 65.76) --
	(108.99, 65.76) --
	(109.06, 65.75) --
	(109.06, 65.75) --
	(109.06, 65.75) --
	(109.13, 65.74) --
	(109.13, 65.74) --
	(109.13, 65.74) --
	(109.20, 65.76) --
	(109.20, 65.76) --
	(109.20, 65.76) --
	(109.27, 65.76) --
	(109.27, 65.76) --
	(109.27, 65.76) --
	(109.34, 65.75) --
	(109.34, 65.75) --
	(109.34, 65.75) --
	(109.41, 65.76) --
	(109.41, 65.76) --
	(109.41, 65.76) --
	(109.48, 65.74) --
	(109.48, 65.74) --
	(109.48, 65.74) --
	(109.55, 65.76) --
	(109.55, 65.76) --
	(109.55, 65.76) --
	(109.62, 65.76) --
	(109.62, 65.76) --
	(109.62, 65.76) --
	(109.69, 65.73) --
	(109.69, 65.73) --
	(109.69, 65.73) --
	(109.76, 65.75) --
	(109.76, 65.75) --
	(109.76, 65.75) --
	(109.83, 65.75) --
	(109.83, 65.75) --
	(109.83, 65.75) --
	(109.90, 65.75) --
	(109.90, 65.75) --
	(109.90, 65.75) --
	(109.90, 65.75) --
	(109.90, 65.75) --
	(109.90, 65.75) --
	(109.97, 65.77) --
	(109.97, 65.77) --
	(109.97, 65.77) --
	(110.04, 65.74) --
	(110.04, 65.74) --
	(110.04, 65.74) --
	(110.11, 65.74) --
	(110.11, 65.74) --
	(110.11, 65.74) --
	(110.18, 65.76) --
	(110.18, 65.76) --
	(110.18, 65.76) --
	(110.25, 65.74) --
	(110.25, 65.74) --
	(110.25, 65.74) --
	(110.32, 65.75) --
	(110.32, 65.75) --
	(110.32, 65.75) --
	(110.39, 65.76) --
	(110.39, 65.76) --
	(110.39, 65.76) --
	(110.46, 65.75) --
	(110.46, 65.75) --
	(110.46, 65.75) --
	(110.53, 65.73) --
	(110.53, 65.73) --
	(110.53, 65.73) --
	(110.60, 65.73) --
	(110.60, 65.73) --
	(110.60, 65.73) --
	(110.67, 65.76) --
	(110.67, 65.76) --
	(110.67, 65.76) --
	(110.74, 65.76) --
	(110.74, 65.76) --
	(110.74, 65.76) --
	(110.81, 65.76) --
	(110.81, 65.76) --
	(110.81, 65.76) --
	(110.88, 65.75) --
	(110.88, 65.75) --
	(110.88, 65.75) --
	(110.95, 65.76) --
	(110.95, 65.76) --
	(110.95, 65.76) --
	(111.02, 65.73) --
	(111.02, 65.73) --
	(111.02, 65.73) --
	(111.09, 65.76) --
	(111.09, 65.76) --
	(111.09, 65.76) --
	(111.16, 65.76) --
	(111.16, 65.76) --
	(111.16, 65.76) --
	(111.23, 65.74) --
	(111.23, 65.74) --
	(111.23, 65.74) --
	(111.30, 65.76) --
	(111.30, 65.76) --
	(111.30, 65.76) --
	(111.37, 65.73) --
	(111.37, 65.73) --
	(111.37, 65.73) --
	(111.44, 65.77) --
	(111.44, 65.77) --
	(111.44, 65.77) --
	(111.51, 65.76) --
	(111.51, 65.76) --
	(111.51, 65.76) --
	(111.58, 65.75) --
	(111.58, 65.75) --
	(111.58, 65.75) --
	(111.65, 65.76) --
	(111.65, 65.76) --
	(111.65, 65.76) --
	(111.72, 65.77) --
	(111.72, 65.77) --
	(111.72, 65.77) --
	(111.72, 65.77) --
	(111.72, 65.77) --
	(111.72, 65.77) --
	(111.79, 65.74) --
	(111.79, 65.74) --
	(111.79, 65.74) --
	(111.86, 65.74) --
	(111.86, 65.74) --
	(111.86, 65.74) --
	(111.93, 65.79) --
	(111.93, 65.79) --
	(111.93, 65.79) --
	(112.00, 65.75) --
	(112.00, 65.75) --
	(112.00, 65.75) --
	(112.07, 65.76) --
	(112.07, 65.76) --
	(112.07, 65.76) --
	(112.14, 65.75) --
	(112.14, 65.75) --
	(112.14, 65.75) --
	(112.21, 65.77) --
	(112.21, 65.77) --
	(112.21, 65.77) --
	(112.28, 65.75) --
	(112.28, 65.75) --
	(112.28, 65.75) --
	(112.35, 65.73) --
	(112.35, 65.73) --
	(112.35, 65.73) --
	(112.42, 65.76) --
	(112.42, 65.76) --
	(112.42, 65.76) --
	(112.49, 65.77) --
	(112.49, 65.77) --
	(112.49, 65.77) --
	(112.56, 65.75) --
	(112.56, 65.75) --
	(112.56, 65.75) --
	(112.63, 65.76) --
	(112.63, 65.76) --
	(112.63, 65.76) --
	(112.70, 65.75) --
	(112.70, 65.75) --
	(112.70, 65.75) --
	(112.77, 65.73) --
	(112.77, 65.73) --
	(112.77, 65.73) --
	(112.84, 65.74) --
	(112.84, 65.74) --
	(112.84, 65.74) --
	(112.91, 65.74) --
	(112.91, 65.74) --
	(112.91, 65.74) --
	(112.98, 65.76) --
	(112.98, 65.76) --
	(112.98, 65.76) --
	(113.05, 65.76) --
	(113.05, 65.76) --
	(113.05, 65.76) --
	(113.12, 65.74) --
	(113.12, 65.74) --
	(113.12, 65.74) --
	(113.19, 65.77) --
	(113.19, 65.77) --
	(113.19, 65.77) --
	(113.26, 65.75) --
	(113.26, 65.75) --
	(113.26, 65.75) --
	(113.33, 65.74) --
	(113.33, 65.74) --
	(113.33, 65.74) --
	(113.39, 65.73) --
	(113.39, 65.73) --
	(113.39, 65.73) --
	(113.47, 65.76) --
	(113.47, 65.76) --
	(113.47, 65.76) --
	(113.54, 65.75) --
	(113.54, 65.75) --
	(113.54, 65.75) --
	(113.60, 65.76) --
	(113.60, 65.76) --
	(113.60, 65.76) --
	(113.67, 65.75) --
	(113.67, 65.75) --
	(113.67, 65.75) --
	(113.74, 65.75) --
	(113.74, 65.75) --
	(113.74, 65.75) --
	(113.81, 65.78) --
	(113.81, 65.78) --
	(113.81, 65.78) --
	(113.88, 65.74) --
	(113.88, 65.74) --
	(113.88, 65.74) --
	(113.95, 65.76) --
	(113.95, 65.76) --
	(113.95, 65.76) --
	(114.02, 65.78) --
	(114.02, 65.78) --
	(114.02, 65.78) --
	(114.09, 65.75) --
	(114.09, 65.75) --
	(114.09, 65.75) --
	(114.16, 65.75) --
	(114.16, 65.75) --
	(114.16, 65.75) --
	(114.23, 65.74) --
	(114.23, 65.74) --
	(114.23, 65.74) --
	(114.30, 65.75) --
	(114.30, 65.75) --
	(114.30, 65.75) --
	(114.37, 65.76) --
	(114.37, 65.76) --
	(114.37, 65.76) --
	(114.44, 65.73) --
	(114.44, 65.73) --
	(114.44, 65.73) --
	(114.51, 65.75) --
	(114.51, 65.75) --
	(114.51, 65.75) --
	(114.58, 65.74) --
	(114.58, 65.74) --
	(114.58, 65.74) --
	(114.65, 65.74) --
	(114.65, 65.74) --
	(114.65, 65.74) --
	(114.72, 65.75) --
	(114.72, 65.75) --
	(114.72, 65.75) --
	(114.79, 65.75) --
	(114.79, 65.75) --
	(114.79, 65.75) --
	(114.86, 65.74) --
	(114.86, 65.74) --
	(114.86, 65.74) --
	(114.93, 65.75) --
	(114.93, 65.75) --
	(114.93, 65.75) --
	(115.00, 65.74) --
	(115.00, 65.74) --
	(115.00, 65.74) --
	(115.07, 65.74) --
	(115.07, 65.74) --
	(115.07, 65.74) --
	(115.14, 65.75) --
	(115.14, 65.75) --
	(115.14, 65.75) --
	(115.21, 65.74) --
	(115.21, 65.74) --
	(115.21, 65.74) --
	(115.28, 65.76) --
	(115.28, 65.76) --
	(115.28, 65.76) --
	(115.35, 65.78) --
	(115.35, 65.78) --
	(115.35, 65.78) --
	(115.41, 65.73) --
	(115.41, 65.73) --
	(115.41, 65.73) --
	(115.48, 65.76) --
	(115.48, 65.76) --
	(115.48, 65.76) --
	(115.55, 65.76) --
	(115.55, 65.76) --
	(115.55, 65.76) --
	(115.62, 65.75) --
	(115.62, 65.75) --
	(115.62, 65.75) --
	(115.69, 65.74) --
	(115.69, 65.74) --
	(115.69, 65.74) --
	(115.76, 65.74) --
	(115.76, 65.74) --
	(115.76, 65.74) --
	(115.83, 65.76) --
	(115.83, 65.76) --
	(115.83, 65.76) --
	(115.90, 65.76) --
	(115.90, 65.76) --
	(115.90, 65.76) --
	(115.97, 65.74) --
	(115.97, 65.74) --
	(115.97, 65.74) --
	(116.04, 65.74) --
	(116.04, 65.74) --
	(116.04, 65.74) --
	(116.11, 65.75) --
	(116.11, 65.75) --
	(116.11, 65.75) --
	(116.18, 65.74) --
	(116.18, 65.74) --
	(116.18, 65.74) --
	(116.25, 65.76) --
	(116.25, 65.76) --
	(116.25, 65.76) --
	(116.32, 65.76) --
	(116.32, 65.76) --
	(116.32, 65.76) --
	(116.39, 65.74) --
	(116.39, 65.74) --
	(116.39, 65.74) --
	(116.46, 65.76) --
	(116.46, 65.76) --
	(116.46, 65.76) --
	(116.53, 65.77) --
	(116.53, 65.77) --
	(116.53, 65.77) --
	(116.60, 65.76) --
	(116.60, 65.76) --
	(116.60, 65.76) --
	(116.67, 65.75) --
	(116.67, 65.75) --
	(116.67, 65.75) --
	(116.74, 65.74) --
	(116.74, 65.74) --
	(116.74, 65.74) --
	(116.81, 65.73) --
	(116.81, 65.73) --
	(116.81, 65.73) --
	(116.87, 65.76) --
	(116.87, 65.76) --
	(116.87, 65.76) --
	(116.94, 65.74) --
	(116.94, 65.74) --
	(116.94, 65.74) --
	(117.01, 65.75) --
	(117.01, 65.75) --
	(117.01, 65.75) --
	(117.08, 65.76) --
	(117.08, 65.76) --
	(117.08, 65.76) --
	(117.15, 65.75) --
	(117.15, 65.75) --
	(117.15, 65.75) --
	(117.22, 65.76) --
	(117.22, 65.76) --
	(117.22, 65.76) --
	(117.29, 65.75) --
	(117.29, 65.75) --
	(117.29, 65.75) --
	(117.36, 65.78) --
	(117.36, 65.78) --
	(117.36, 65.78) --
	(117.43, 65.77) --
	(117.43, 65.77) --
	(117.43, 65.77) --
	(117.50, 65.75) --
	(117.50, 65.75) --
	(117.50, 65.75) --
	(117.57, 65.75) --
	(117.57, 65.75) --
	(117.57, 65.75) --
	(117.64, 65.76) --
	(117.64, 65.76) --
	(117.64, 65.76) --
	(117.71, 65.74) --
	(117.71, 65.74) --
	(117.71, 65.74) --
	(117.78, 65.75) --
	(117.78, 65.75) --
	(117.78, 65.75) --
	(117.85, 65.77) --
	(117.85, 65.77) --
	(117.85, 65.77) --
	(117.91, 65.75) --
	(117.91, 65.75) --
	(117.91, 65.75) --
	(117.98, 65.75) --
	(117.98, 65.75) --
	(117.98, 65.75) --
	(118.05, 65.72) --
	(118.05, 65.72) --
	(118.05, 65.72) --
	(118.12, 65.76) --
	(118.12, 65.76) --
	(118.12, 65.76) --
	(118.19, 65.77) --
	(118.19, 65.77) --
	(118.19, 65.77) --
	(118.26, 65.72) --
	(118.26, 65.72) --
	(118.26, 65.72) --
	(118.33, 65.76) --
	(118.33, 65.76) --
	(118.33, 65.76) --
	(118.40, 65.76) --
	(118.40, 65.76) --
	(118.40, 65.76) --
	(118.47, 65.74) --
	(118.47, 65.74) --
	(118.47, 65.74) --
	(118.54, 65.75) --
	(118.54, 65.75) --
	(118.54, 65.75) --
	(118.61, 65.75) --
	(118.61, 65.75) --
	(118.61, 65.75) --
	(118.68, 65.77) --
	(118.68, 65.77) --
	(118.68, 65.77) --
	(118.75, 65.75) --
	(118.75, 65.75) --
	(118.75, 65.75) --
	(118.82, 65.73) --
	(118.82, 65.73) --
	(118.82, 65.73) --
	(118.88, 65.75) --
	(118.88, 65.75) --
	(118.88, 65.75) --
	(118.95, 65.74) --
	(118.95, 65.74) --
	(118.95, 65.74) --
	(119.02, 65.73) --
	(119.02, 65.73) --
	(119.02, 65.73) --
	(119.09, 65.73) --
	(119.09, 65.73) --
	(119.09, 65.73) --
	(119.16, 65.77) --
	(119.16, 65.77) --
	(119.16, 65.77) --
	(119.23, 65.73) --
	(119.23, 65.73) --
	(119.23, 65.73) --
	(119.30, 65.75) --
	(119.30, 65.75) --
	(119.30, 65.75) --
	(119.37, 65.74) --
	(119.37, 65.74) --
	(119.37, 65.74) --
	(119.44, 65.74) --
	(119.44, 65.74) --
	(119.44, 65.74) --
	(119.51, 65.76) --
	(119.51, 65.76) --
	(119.51, 65.76) --
	(119.58, 65.73) --
	(119.58, 65.73) --
	(119.58, 65.73) --
	(119.65, 65.76) --
	(119.65, 65.76) --
	(119.65, 65.76) --
	(119.72, 65.76) --
	(119.72, 65.76) --
	(119.72, 65.76) --
	(119.79, 65.73) --
	(119.79, 65.73) --
	(119.79, 65.73) --
	(119.86, 65.75) --
	(119.86, 65.75) --
	(119.86, 65.75) --
	(119.92, 65.75) --
	(119.92, 65.75) --
	(119.92, 65.75) --
	(119.99, 65.74) --
	(119.99, 65.74) --
	(119.99, 65.74) --
	(120.06, 65.73) --
	(120.06, 65.73) --
	(120.06, 65.73) --
	(120.13, 65.73) --
	(120.13, 65.73) --
	(120.13, 65.73) --
	(120.20, 65.76) --
	(120.20, 65.76) --
	(120.20, 65.76) --
	(120.27, 65.76) --
	(120.27, 65.76) --
	(120.27, 65.76) --
	(120.34, 65.74) --
	(120.34, 65.74) --
	(120.34, 65.74) --
	(120.41, 65.75) --
	(120.41, 65.75) --
	(120.41, 65.75) --
	(120.48, 65.77) --
	(120.48, 65.77) --
	(120.48, 65.77) --
	(120.55, 65.74) --
	(120.55, 65.74) --
	(120.55, 65.74) --
	(120.62, 65.74) --
	(120.62, 65.74) --
	(120.62, 65.74) --
	(120.69, 65.77) --
	(120.69, 65.77) --
	(120.69, 65.77) --
	(120.75, 65.74) --
	(120.75, 65.74) --
	(120.75, 65.74) --
	(120.82, 65.76) --
	(120.82, 65.76) --
	(120.82, 65.76) --
	(120.89, 65.75) --
	(120.89, 65.75) --
	(120.89, 65.75) --
	(120.96, 65.75) --
	(120.96, 65.75) --
	(120.96, 65.75) --
	(121.03, 65.75) --
	(121.03, 65.75) --
	(121.03, 65.75) --
	(121.10, 65.74) --
	(121.10, 65.74) --
	(121.10, 65.74) --
	(121.17, 65.75) --
	(121.17, 65.75) --
	(121.17, 65.75) --
	(121.24, 65.77) --
	(121.24, 65.77) --
	(121.24, 65.77) --
	(121.31, 65.75) --
	(121.31, 65.75) --
	(121.31, 65.75) --
	(121.38, 65.76) --
	(121.38, 65.76) --
	(121.38, 65.76) --
	(121.44, 65.79) --
	(121.44, 65.79) --
	(121.44, 65.79) --
	(121.51, 65.74) --
	(121.51, 65.74) --
	(121.51, 65.74) --
	(121.58, 65.77) --
	(121.58, 65.77) --
	(121.58, 65.77) --
	(121.65, 65.75) --
	(121.65, 65.75) --
	(121.65, 65.75) --
	(121.72, 65.75) --
	(121.72, 65.75) --
	(121.72, 65.75) --
	(121.79, 65.75) --
	(121.79, 65.75) --
	(121.79, 65.75) --
	(121.86, 65.74) --
	(121.86, 65.74) --
	(121.86, 65.74) --
	(121.93, 65.76) --
	(121.93, 65.76) --
	(121.93, 65.76) --
	(122.00, 65.75) --
	(122.00, 65.75) --
	(122.00, 65.75) --
	(122.07, 65.75) --
	(122.07, 65.75) --
	(122.07, 65.75) --
	(122.14, 65.74) --
	(122.14, 65.74) --
	(122.14, 65.74) --
	(122.20, 65.76) --
	(122.21, 65.76) --
	(122.21, 65.76) --
	(122.27, 65.72) --
	(122.27, 65.72) --
	(122.27, 65.72) --
	(122.34, 65.72) --
	(122.34, 65.72) --
	(122.34, 65.72) --
	(122.41, 65.75) --
	(122.41, 65.75) --
	(122.41, 65.75) --
	(122.48, 65.77) --
	(122.48, 65.77) --
	(122.48, 65.77) --
	(122.55, 65.75) --
	(122.55, 65.75) --
	(122.55, 65.75) --
	(122.62, 65.75) --
	(122.62, 65.75) --
	(122.62, 65.75) --
	(122.69, 65.75) --
	(122.69, 65.75) --
	(122.69, 65.75) --
	(122.76, 65.76) --
	(122.76, 65.76) --
	(122.76, 65.76) --
	(122.83, 65.73) --
	(122.83, 65.73) --
	(122.83, 65.73) --
	(122.89, 65.72) --
	(122.89, 65.72) --
	(122.89, 65.72) --
	(122.96, 65.76) --
	(122.96, 65.76) --
	(122.96, 65.76) --
	(123.03, 65.75) --
	(123.03, 65.75) --
	(123.03, 65.75) --
	(123.10, 65.76) --
	(123.10, 65.76) --
	(123.10, 65.76) --
	(123.17, 65.72) --
	(123.17, 65.72) --
	(123.17, 65.72) --
	(123.24, 65.76) --
	(123.24, 65.76) --
	(123.24, 65.76) --
	(123.31, 65.78) --
	(123.31, 65.78) --
	(123.31, 65.78) --
	(123.38, 65.75) --
	(123.38, 65.75) --
	(123.38, 65.75) --
	(123.45, 65.73) --
	(123.45, 65.73) --
	(123.45, 65.73) --
	(123.51, 65.76) --
	(123.51, 65.76) --
	(123.51, 65.76) --
	(123.58, 65.74) --
	(123.58, 65.74) --
	(123.58, 65.74) --
	(123.65, 65.73) --
	(123.65, 65.73) --
	(123.65, 65.73) --
	(123.72, 65.76) --
	(123.72, 65.76) --
	(123.72, 65.76) --
	(123.79, 65.76) --
	(123.79, 65.76) --
	(123.79, 65.76) --
	(123.86, 65.76) --
	(123.86, 65.76) --
	(123.86, 65.76) --
	(123.93, 65.75) --
	(123.93, 65.75) --
	(123.93, 65.75) --
	(124.00, 65.78) --
	(124.00, 65.78) --
	(124.00, 65.78) --
	(124.07, 65.75) --
	(124.07, 65.75) --
	(124.07, 65.75) --
	(124.13, 65.75) --
	(124.13, 65.75) --
	(124.13, 65.75) --
	(124.20, 65.76) --
	(124.20, 65.76) --
	(124.20, 65.76) --
	(124.27, 65.76) --
	(124.27, 65.76) --
	(124.27, 65.76) --
	(124.34, 65.73) --
	(124.34, 65.73) --
	(124.34, 65.73) --
	(124.41, 65.75) --
	(124.41, 65.75) --
	(124.41, 65.75) --
	(124.48, 65.77) --
	(124.48, 65.77) --
	(124.48, 65.77) --
	(124.55, 65.76) --
	(124.55, 65.76) --
	(124.55, 65.76) --
	(124.62, 65.74) --
	(124.62, 65.74) --
	(124.62, 65.74) --
	(124.69, 65.74) --
	(124.69, 65.74) --
	(124.69, 65.74) --
	(124.75, 65.76) --
	(124.75, 65.76) --
	(124.75, 65.76) --
	(124.82, 65.76) --
	(124.82, 65.76) --
	(124.82, 65.76) --
	(124.89, 65.76) --
	(124.89, 65.76) --
	(124.89, 65.76) --
	(124.96, 65.75) --
	(124.96, 65.75) --
	(124.96, 65.75) --
	(125.03, 65.75) --
	(125.03, 65.75) --
	(125.03, 65.75) --
	(125.10, 65.73) --
	(125.10, 65.73) --
	(125.10, 65.73) --
	(125.17, 65.73) --
	(125.17, 65.73) --
	(125.17, 65.73) --
	(125.24, 65.75) --
	(125.24, 65.75) --
	(125.24, 65.75) --
	(125.31, 65.75) --
	(125.31, 65.75) --
	(125.31, 65.75) --
	(125.37, 65.74) --
	(125.37, 65.74) --
	(125.37, 65.74) --
	(125.44, 65.74) --
	(125.44, 65.74) --
	(125.44, 65.74) --
	(125.51, 65.75) --
	(125.51, 65.75) --
	(125.51, 65.75) --
	(125.58, 65.73) --
	(125.58, 65.73) --
	(125.58, 65.73) --
	(125.65, 65.74) --
	(125.65, 65.74) --
	(125.65, 65.74) --
	(125.72, 65.74) --
	(125.72, 65.74) --
	(125.72, 65.74) --
	(125.79, 65.77) --
	(125.79, 65.77) --
	(125.79, 65.77) --
	(125.86, 65.75) --
	(125.86, 65.75) --
	(125.86, 65.75) --
	(125.92, 65.74) --
	(125.92, 65.74) --
	(125.92, 65.74) --
	(125.99, 65.76) --
	(125.99, 65.76) --
	(125.99, 65.76) --
	(126.06, 65.74) --
	(126.06, 65.74) --
	(126.06, 65.74) --
	(126.13, 65.76) --
	(126.13, 65.76) --
	(126.13, 65.76) --
	(126.20, 65.74) --
	(126.20, 65.74) --
	(126.20, 65.74) --
	(126.27, 65.75) --
	(126.27, 65.75) --
	(126.27, 65.75) --
	(126.34, 65.76) --
	(126.34, 65.76) --
	(126.34, 65.76) --
	(126.41, 65.72) --
	(126.41, 65.72) --
	(126.41, 65.72) --
	(126.47, 65.75) --
	(126.47, 65.75) --
	(126.47, 65.75) --
	(126.54, 65.76) --
	(126.54, 65.76) --
	(126.54, 65.76) --
	(126.61, 65.74) --
	(126.61, 65.74) --
	(126.61, 65.74) --
	(126.68, 65.74) --
	(126.68, 65.74) --
	(126.68, 65.74) --
	(126.75, 65.75) --
	(126.75, 65.75) --
	(126.75, 65.75) --
	(126.82, 65.74) --
	(126.82, 65.74) --
	(126.82, 65.74) --
	(126.89, 65.76) --
	(126.89, 65.76) --
	(126.89, 65.76) --
	(126.95, 65.74) --
	(126.95, 65.74) --
	(126.95, 65.74) --
	(127.02, 65.76) --
	(127.02, 65.76) --
	(127.02, 65.76) --
	(127.09, 65.76) --
	(127.09, 65.76) --
	(127.09, 65.76) --
	(127.16, 65.76) --
	(127.16, 65.76) --
	(127.16, 65.76) --
	(127.23, 65.76) --
	(127.23, 65.76) --
	(127.23, 65.76) --
	(127.30, 65.77) --
	(127.30, 65.77) --
	(127.30, 65.77) --
	(127.37, 65.73) --
	(127.37, 65.73) --
	(127.37, 65.73) --
	(127.44, 65.73) --
	(127.44, 65.73) --
	(127.44, 65.73) --
	(127.50, 65.75) --
	(127.50, 65.75) --
	(127.50, 65.75) --
	(127.57, 65.74) --
	(127.57, 65.74) --
	(127.57, 65.74) --
	(127.64, 65.75) --
	(127.64, 65.75) --
	(127.64, 65.75) --
	(127.71, 65.74) --
	(127.71, 65.74) --
	(127.71, 65.74) --
	(127.78, 65.77) --
	(127.78, 65.77) --
	(127.78, 65.77) --
	(127.85, 65.74) --
	(127.85, 65.74) --
	(127.85, 65.74) --
	(127.91, 65.73) --
	(127.91, 65.73) --
	(127.91, 65.73) --
	(127.98, 65.74) --
	(127.98, 65.74) --
	(127.98, 65.74) --
	(128.05, 65.77) --
	(128.05, 65.77) --
	(128.05, 65.77) --
	(128.12, 65.75) --
	(128.12, 65.75) --
	(128.12, 65.75) --
	(128.19, 65.74) --
	(128.19, 65.74) --
	(128.19, 65.74) --
	(128.26, 65.73) --
	(128.26, 65.73) --
	(128.26, 65.73) --
	(128.33, 65.75) --
	(128.33, 65.75) --
	(128.33, 65.75) --
	(128.40, 65.76) --
	(128.40, 65.76) --
	(128.40, 65.76) --
	(128.46, 65.76) --
	(128.46, 65.76) --
	(128.46, 65.76) --
	(128.53, 65.74) --
	(128.53, 65.74) --
	(128.53, 65.74) --
	(128.60, 65.77) --
	(128.60, 65.77) --
	(128.60, 65.77) --
	(128.67, 65.75) --
	(128.67, 65.75) --
	(128.67, 65.75) --
	(128.74, 65.75) --
	(128.74, 65.75) --
	(128.74, 65.75) --
	(128.81, 65.77) --
	(128.81, 65.77) --
	(128.81, 65.77) --
	(128.88, 65.73) --
	(128.88, 65.73) --
	(128.88, 65.73) --
	(128.94, 65.73) --
	(128.94, 65.73) --
	(128.94, 65.73) --
	(129.01, 65.74) --
	(129.01, 65.74) --
	(129.01, 65.74) --
	(129.08, 65.76) --
	(129.08, 65.76) --
	(129.08, 65.76) --
	(129.15, 65.76) --
	(129.15, 65.76) --
	(129.15, 65.76) --
	(129.22, 65.73) --
	(129.22, 65.73) --
	(129.22, 65.73) --
	(129.29, 65.77) --
	(129.29, 65.77) --
	(129.29, 65.77) --
	(129.35, 65.74) --
	(129.35, 65.74) --
	(129.35, 65.74) --
	(129.42, 65.74) --
	(129.42, 65.74) --
	(129.42, 65.74) --
	(129.49, 65.74) --
	(129.49, 65.74) --
	(129.49, 65.74) --
	(129.56, 65.76) --
	(129.56, 65.76) --
	(129.56, 65.76) --
	(129.63, 65.76) --
	(129.63, 65.76) --
	(129.63, 65.76) --
	(129.70, 65.76) --
	(129.70, 65.76) --
	(129.70, 65.76) --
	(129.77, 65.73) --
	(129.77, 65.73) --
	(129.77, 65.73) --
	(129.83, 65.77) --
	(129.83, 65.77) --
	(129.83, 65.77) --
	(129.90, 65.78) --
	(129.90, 65.78) --
	(129.90, 65.78) --
	(129.97, 65.74) --
	(129.97, 65.74) --
	(129.97, 65.74) --
	(130.04, 65.75) --
	(130.04, 65.75) --
	(130.04, 65.75) --
	(130.11, 65.76) --
	(130.11, 65.76) --
	(130.11, 65.76) --
	(130.18, 65.72) --
	(130.18, 65.72) --
	(130.18, 65.72) --
	(130.24, 65.74) --
	(130.24, 65.74) --
	(130.24, 65.74) --
	(130.31, 65.77) --
	(130.31, 65.77) --
	(130.31, 65.77) --
	(130.38, 65.74) --
	(130.38, 65.74) --
	(130.38, 65.74) --
	(130.45, 65.76) --
	(130.45, 65.76) --
	(130.45, 65.76) --
	(130.52, 65.75) --
	(130.52, 65.75) --
	(130.52, 65.75) --
	(130.59, 65.76) --
	(130.59, 65.76) --
	(130.59, 65.76) --
	(130.65, 65.76) --
	(130.65, 65.76) --
	(130.65, 65.76) --
	(130.72, 65.71) --
	(130.72, 65.71) --
	(130.72, 65.71) --
	(130.79, 65.76) --
	(130.79, 65.76) --
	(130.79, 65.76) --
	(130.86, 65.76) --
	(130.86, 65.76) --
	(130.86, 65.76) --
	(130.93, 65.76) --
	(130.93, 65.76) --
	(130.93, 65.76) --
	(131.00, 65.73) --
	(131.00, 65.73) --
	(131.00, 65.73) --
	(131.06, 65.76) --
	(131.06, 65.76) --
	(131.06, 65.76) --
	(131.13, 65.75) --
	(131.13, 65.75) --
	(131.13, 65.75) --
	(131.20, 65.75) --
	(131.20, 65.75) --
	(131.20, 65.75) --
	(131.27, 65.74) --
	(131.27, 65.74) --
	(131.27, 65.74) --
	(131.34, 65.74) --
	(131.34, 65.74) --
	(131.34, 65.74) --
	(131.41, 65.73) --
	(131.41, 65.73) --
	(131.41, 65.73) --
	(131.47, 65.73) --
	(131.47, 65.73) --
	(131.47, 65.73) --
	(131.54, 65.74) --
	(131.54, 65.74) --
	(131.54, 65.74) --
	(131.61, 65.76) --
	(131.61, 65.76) --
	(131.61, 65.76) --
	(131.68, 65.73) --
	(131.68, 65.73) --
	(131.68, 65.73) --
	(131.75, 65.76) --
	(131.75, 65.76) --
	(131.75, 65.76) --
	(131.82, 65.76) --
	(131.82, 65.76) --
	(131.82, 65.76) --
	(131.88, 65.75) --
	(131.88, 65.75) --
	(131.88, 65.75) --
	(131.95, 65.74) --
	(131.95, 65.74) --
	(131.95, 65.74) --
	(132.02, 65.72) --
	(132.02, 65.72) --
	(132.02, 65.72) --
	(132.09, 65.76) --
	(132.09, 65.76) --
	(132.09, 65.76) --
	(132.16, 65.77) --
	(132.16, 65.77) --
	(132.16, 65.77) --
	(132.22, 65.75) --
	(132.22, 65.75) --
	(132.22, 65.75) --
	(132.29, 65.76) --
	(132.29, 65.76) --
	(132.29, 65.76) --
	(132.36, 65.75) --
	(132.36, 65.75) --
	(132.36, 65.75) --
	(132.43, 65.74) --
	(132.43, 65.74) --
	(132.43, 65.74) --
	(132.50, 65.75) --
	(132.50, 65.75) --
	(132.50, 65.75) --
	(132.57, 65.75) --
	(132.57, 65.75) --
	(132.57, 65.75) --
	(132.63, 65.75) --
	(132.63, 65.75) --
	(132.63, 65.75) --
	(132.70, 65.73) --
	(132.70, 65.73) --
	(132.70, 65.73) --
	(132.77, 65.73) --
	(132.77, 65.73) --
	(132.77, 65.73) --
	(132.84, 65.76) --
	(132.84, 65.76) --
	(132.84, 65.76) --
	(132.91, 65.77) --
	(132.91, 65.77) --
	(132.91, 65.77) --
	(132.98, 65.74) --
	(132.98, 65.74) --
	(132.98, 65.74) --
	(133.04, 65.76) --
	(133.04, 65.76) --
	(133.04, 65.76) --
	(133.11, 65.75) --
	(133.11, 65.75) --
	(133.11, 65.75) --
	(133.18, 65.73) --
	(133.18, 65.73) --
	(133.18, 65.73) --
	(133.25, 65.76) --
	(133.25, 65.76) --
	(133.25, 65.76) --
	(133.32, 65.74) --
	(133.32, 65.74) --
	(133.32, 65.74) --
	(133.38, 65.76) --
	(133.38, 65.76) --
	(133.38, 65.76) --
	(133.45, 65.77) --
	(133.45, 65.77) --
	(133.45, 65.77) --
	(133.52, 65.72) --
	(133.52, 65.72) --
	(133.52, 65.72) --
	(133.59, 65.76) --
	(133.59, 65.76) --
	(133.59, 65.76) --
	(133.66, 65.75) --
	(133.66, 65.75) --
	(133.66, 65.75) --
	(133.72, 65.75) --
	(133.72, 65.75) --
	(133.72, 65.75) --
	(133.79, 65.74) --
	(133.79, 65.74) --
	(133.79, 65.74) --
	(133.86, 65.76) --
	(133.86, 65.76) --
	(133.86, 65.76) --
	(133.93, 65.74) --
	(133.93, 65.74) --
	(133.93, 65.74) --
	(134.00, 65.76) --
	(134.00, 65.76) --
	(134.00, 65.76) --
	(134.07, 65.76) --
	(134.07, 65.76) --
	(134.07, 65.76) --
	(134.13, 65.76) --
	(134.13, 65.76) --
	(134.13, 65.76) --
	(134.20, 65.76) --
	(134.20, 65.76) --
	(134.20, 65.76) --
	(134.27, 65.73) --
	(134.27, 65.73) --
	(134.27, 65.73) --
	(134.34, 65.75) --
	(134.34, 65.75) --
	(134.34, 65.75) --
	(134.41, 65.77) --
	(134.41, 65.77) --
	(134.41, 65.77) --
	(134.47, 65.73) --
	(134.47, 65.73) --
	(134.47, 65.73) --
	(134.54, 65.74) --
	(134.54, 65.74) --
	(134.54, 65.74) --
	(134.61, 65.76) --
	(134.61, 65.76) --
	(134.61, 65.76) --
	(134.68, 65.74) --
	(134.68, 65.74) --
	(134.68, 65.74) --
	(134.75, 65.75) --
	(134.75, 65.75) --
	(134.75, 65.75) --
	(134.81, 65.72) --
	(134.81, 65.72) --
	(134.81, 65.72) --
	(134.88, 65.76) --
	(134.88, 65.76) --
	(134.88, 65.76) --
	(134.95, 65.75) --
	(134.95, 65.75) --
	(134.95, 65.75) --
	(135.02, 65.73) --
	(135.02, 65.73) --
	(135.02, 65.73) --
	(135.09, 65.76) --
	(135.09, 65.76) --
	(135.09, 65.76) --
	(135.15, 65.78) --
	(135.15, 65.78) --
	(135.15, 65.78) --
	(135.22, 65.72) --
	(135.22, 65.72) --
	(135.22, 65.72) --
	(135.29, 65.74) --
	(135.29, 65.74) --
	(135.29, 65.74) --
	(135.36, 65.74) --
	(135.36, 65.74) --
	(135.36, 65.74) --
	(135.43, 65.73) --
	(135.43, 65.73) --
	(135.43, 65.73) --
	(135.49, 65.76) --
	(135.49, 65.76) --
	(135.49, 65.76) --
	(135.56, 65.74) --
	(135.56, 65.74) --
	(135.56, 65.74) --
	(135.63, 65.75) --
	(135.63, 65.75) --
	(135.63, 65.75) --
	(135.70, 65.74) --
	(135.70, 65.74) --
	(135.70, 65.74) --
	(135.77, 65.74) --
	(135.77, 65.74) --
	(135.77, 65.74) --
	(135.83, 65.75) --
	(135.83, 65.75) --
	(135.83, 65.75) --
	(135.90, 65.76) --
	(135.90, 65.76) --
	(135.90, 65.76) --
	(135.97, 65.72) --
	(135.97, 65.72) --
	(135.97, 65.72) --
	(136.04, 65.73) --
	(136.04, 65.73) --
	(136.04, 65.73) --
	(136.11, 65.75) --
	(136.11, 65.75) --
	(136.11, 65.75) --
	(136.17, 65.74) --
	(136.17, 65.74) --
	(136.17, 65.74) --
	(136.24, 65.75) --
	(136.24, 65.75) --
	(136.24, 65.75) --
	(136.31, 65.73) --
	(136.31, 65.73) --
	(136.31, 65.73) --
	(136.38, 65.76) --
	(136.38, 65.76) --
	(136.38, 65.76) --
	(136.44, 65.74) --
	(136.44, 65.74) --
	(136.44, 65.74) --
	(136.51, 65.73) --
	(136.51, 65.73) --
	(136.51, 65.73) --
	(136.58, 65.77) --
	(136.58, 65.77) --
	(136.58, 65.77) --
	(136.65, 65.75) --
	(136.65, 65.75) --
	(136.65, 65.75) --
	(136.72, 65.73) --
	(136.72, 65.73) --
	(136.72, 65.73) --
	(136.78, 65.74) --
	(136.78, 65.74) --
	(136.78, 65.74) --
	(136.85, 65.76) --
	(136.85, 65.76) --
	(136.85, 65.76) --
	(136.92, 65.75) --
	(136.92, 65.75) --
	(136.92, 65.75) --
	(136.99, 65.76) --
	(136.99, 65.76) --
	(136.99, 65.76) --
	(137.06, 65.72) --
	(137.06, 65.72) --
	(137.06, 65.72) --
	(137.12, 65.77) --
	(137.12, 65.77) --
	(137.12, 65.77) --
	(137.19, 65.74) --
	(137.19, 65.74) --
	(137.19, 65.74) --
	(137.26, 65.74) --
	(137.26, 65.74) --
	(137.26, 65.74) --
	(137.33, 65.74) --
	(137.33, 65.74) --
	(137.33, 65.74) --
	(137.40, 65.76) --
	(137.40, 65.76) --
	(137.40, 65.76) --
	(137.46, 65.73) --
	(137.46, 65.73) --
	(137.46, 65.73) --
	(137.53, 65.74) --
	(137.53, 65.74) --
	(137.53, 65.74) --
	(137.60, 65.74) --
	(137.60, 65.74) --
	(137.60, 65.74) --
	(137.67, 65.76) --
	(137.67, 65.76) --
	(137.67, 65.76) --
	(137.73, 65.75) --
	(137.73, 65.75) --
	(137.73, 65.75) --
	(137.80, 65.75) --
	(137.80, 65.75) --
	(137.80, 65.75) --
	(137.87, 65.77) --
	(137.87, 65.77) --
	(137.87, 65.77) --
	(137.94, 65.77) --
	(137.94, 65.77) --
	(137.94, 65.77) --
	(138.00, 65.74) --
	(138.00, 65.74) --
	(138.00, 65.74) --
	(138.07, 65.72) --
	(138.07, 65.72) --
	(138.07, 65.72) --
	(138.14, 65.75) --
	(138.14, 65.75) --
	(138.14, 65.75) --
	(138.21, 65.74) --
	(138.21, 65.74) --
	(138.21, 65.74) --
	(138.28, 65.76) --
	(138.28, 65.76) --
	(138.28, 65.76) --
	(138.34, 65.74) --
	(138.34, 65.74) --
	(138.34, 65.74) --
	(138.41, 65.75) --
	(138.41, 65.75) --
	(138.41, 65.75) --
	(138.48, 65.76) --
	(138.48, 65.76) --
	(138.48, 65.76) --
	(138.55, 65.74) --
	(138.55, 65.74) --
	(138.55, 65.74) --
	(138.61, 65.75) --
	(138.61, 65.75) --
	(138.61, 65.75) --
	(138.68, 65.74) --
	(138.68, 65.74) --
	(138.68, 65.74) --
	(138.75, 65.74) --
	(138.75, 65.74) --
	(138.75, 65.74) --
	(138.82, 65.75) --
	(138.82, 65.75) --
	(138.82, 65.75) --
	(138.89, 65.78) --
	(138.89, 65.78) --
	(138.89, 65.78) --
	(138.95, 65.73) --
	(138.95, 65.73) --
	(138.95, 65.73) --
	(139.02, 65.75) --
	(139.02, 65.75) --
	(139.02, 65.75) --
	(139.09, 65.73) --
	(139.09, 65.73) --
	(139.09, 65.73) --
	(139.16, 65.77) --
	(139.16, 65.77) --
	(139.16, 65.77) --
	(139.22, 65.75) --
	(139.22, 65.75) --
	(139.22, 65.75) --
	(139.29, 65.73) --
	(139.29, 65.73) --
	(139.29, 65.73) --
	(139.36, 65.75) --
	(139.36, 65.75) --
	(139.36, 65.75) --
	(139.43, 65.78) --
	(139.43, 65.78) --
	(139.43, 65.78) --
	(139.49, 65.74) --
	(139.49, 65.74) --
	(139.49, 65.74) --
	(139.56, 65.74) --
	(139.56, 65.74) --
	(139.56, 65.74) --
	(139.63, 65.76) --
	(139.63, 65.76) --
	(139.63, 65.76) --
	(139.70, 65.75) --
	(139.70, 65.75) --
	(139.70, 65.75) --
	(139.76, 65.76) --
	(139.76, 65.76) --
	(139.76, 65.76) --
	(139.83, 65.76) --
	(139.83, 65.76) --
	(139.83, 65.76) --
	(139.90, 65.75) --
	(139.90, 65.75) --
	(139.90, 65.75) --
	(139.97, 65.77) --
	(139.97, 65.77) --
	(139.97, 65.77) --
	(140.04, 65.74) --
	(140.04, 65.74) --
	(140.04, 65.74) --
	(140.10, 65.73) --
	(140.10, 65.73) --
	(140.10, 65.73) --
	(140.17, 65.76) --
	(140.17, 65.76) --
	(140.17, 65.76) --
	(140.24, 65.74) --
	(140.24, 65.74) --
	(140.24, 65.74) --
	(140.31, 65.75) --
	(140.31, 65.75) --
	(140.31, 65.75) --
	(140.37, 65.74) --
	(140.37, 65.74) --
	(140.37, 65.74) --
	(140.44, 65.73) --
	(140.44, 65.73) --
	(140.44, 65.73) --
	(140.51, 65.76) --
	(140.51, 65.76) --
	(140.51, 65.76) --
	(140.58, 65.74) --
	(140.58, 65.74) --
	(140.58, 65.74) --
	(140.64, 65.77) --
	(140.64, 65.77) --
	(140.64, 65.77) --
	(140.71, 65.77) --
	(140.71, 65.77) --
	(140.71, 65.77) --
	(140.78, 65.75) --
	(140.78, 65.75) --
	(140.78, 65.75) --
	(140.85, 65.75) --
	(140.85, 65.75) --
	(140.85, 65.75) --
	(140.91, 65.78) --
	(140.91, 65.78) --
	(140.91, 65.78) --
	(140.98, 65.74) --
	(140.98, 65.74) --
	(140.98, 65.74) --
	(141.05, 65.74) --
	(141.05, 65.74) --
	(141.05, 65.74) --
	(141.12, 65.74) --
	(141.12, 65.74) --
	(141.12, 65.74) --
	(141.18, 65.77) --
	(141.18, 65.77) --
	(141.18, 65.77) --
	(141.25, 65.74) --
	(141.25, 65.74) --
	(141.25, 65.74) --
	(141.32, 65.72) --
	(141.32, 65.72) --
	(141.32, 65.72) --
	(141.39, 65.76) --
	(141.39, 65.76) --
	(141.39, 65.76) --
	(141.45, 65.76) --
	(141.45, 65.76) --
	(141.45, 65.76) --
	(141.52, 65.74) --
	(141.52, 65.74) --
	(141.52, 65.74) --
	(141.59, 65.74) --
	(141.59, 65.74) --
	(141.59, 65.74) --
	(141.66, 65.75) --
	(141.66, 65.75) --
	(141.66, 65.75) --
	(141.72, 65.73) --
	(141.72, 65.73) --
	(141.72, 65.73) --
	(141.79, 65.73) --
	(141.79, 65.73) --
	(141.79, 65.73) --
	(141.86, 65.75) --
	(141.86, 65.75) --
	(141.86, 65.75) --
	(141.92, 65.73) --
	(141.92, 65.73) --
	(141.92, 65.73) --
	(141.99, 65.73) --
	(141.99, 65.73) --
	(141.99, 65.73) --
	(142.06, 65.74) --
	(142.06, 65.74) --
	(142.06, 65.74) --
	(142.13, 65.75) --
	(142.13, 65.75) --
	(142.13, 65.75) --
	(142.19, 65.76) --
	(142.19, 65.76) --
	(142.19, 65.76) --
	(142.26, 65.71) --
	(142.26, 65.71) --
	(142.26, 65.71) --
	(142.33, 65.73) --
	(142.33, 65.73) --
	(142.33, 65.73) --
	(142.40, 65.76) --
	(142.40, 65.76) --
	(142.40, 65.76) --
	(142.46, 65.73) --
	(142.46, 65.73) --
	(142.46, 65.73) --
	(142.53, 65.75) --
	(142.53, 65.75) --
	(142.53, 65.75) --
	(142.60, 65.74) --
	(142.60, 65.74) --
	(142.60, 65.74) --
	(142.67, 65.74) --
	(142.67, 65.74) --
	(142.67, 65.74) --
	(142.73, 65.74) --
	(142.73, 65.74) --
	(142.73, 65.74) --
	(142.80, 65.74) --
	(142.80, 65.74) --
	(142.80, 65.74) --
	(142.87, 65.73) --
	(142.87, 65.73) --
	(142.87, 65.73) --
	(142.94, 65.75) --
	(142.94, 65.75) --
	(142.94, 65.75) --
	(143.00, 65.73) --
	(143.00, 65.73) --
	(143.00, 65.73) --
	(143.07, 65.74) --
	(143.07, 65.74) --
	(143.07, 65.74) --
	(143.14, 65.78) --
	(143.14, 65.78) --
	(143.14, 65.78) --
	(143.21, 65.74) --
	(143.21, 65.74) --
	(143.21, 65.74) --
	(143.27, 65.75) --
	(143.27, 65.75) --
	(143.27, 65.75) --
	(143.34, 65.74) --
	(143.34, 65.74) --
	(143.34, 65.74) --
	(143.41, 65.73) --
	(143.41, 65.73) --
	(143.41, 65.73) --
	(143.48, 65.75) --
	(143.48, 65.75) --
	(143.48, 65.75) --
	(143.54, 65.74) --
	(143.54, 65.74) --
	(143.54, 65.74) --
	(143.61, 65.76) --
	(143.61, 65.76) --
	(143.61, 65.76) --
	(143.68, 65.78) --
	(143.68, 65.78) --
	(143.68, 65.78) --
	(143.74, 65.73) --
	(143.74, 65.73) --
	(143.74, 65.73) --
	(143.81, 65.76) --
	(143.81, 65.76) --
	(143.81, 65.76) --
	(143.88, 65.77) --
	(143.88, 65.77) --
	(143.88, 65.77) --
	(143.95, 65.74) --
	(143.95, 65.74) --
	(143.95, 65.74) --
	(144.01, 65.74) --
	(144.01, 65.74) --
	(144.01, 65.74) --
	(144.08, 65.73) --
	(144.08, 65.73) --
	(144.08, 65.73) --
	(144.15, 65.78) --
	(144.15, 65.78) --
	(144.15, 65.78) --
	(144.21, 65.75) --
	(144.21, 65.75) --
	(144.21, 65.75) --
	(144.28, 65.74) --
	(144.28, 65.74) --
	(144.28, 65.74) --
	(144.35, 65.78) --
	(144.35, 65.78) --
	(144.35, 65.78) --
	(144.42, 65.75) --
	(144.42, 65.75) --
	(144.42, 65.75) --
	(144.48, 65.73) --
	(144.48, 65.73) --
	(144.48, 65.73) --
	(144.55, 65.75) --
	(144.55, 65.75) --
	(144.55, 65.75) --
	(144.62, 65.75) --
	(144.62, 65.75) --
	(144.62, 65.75) --
	(144.69, 65.71) --
	(144.69, 65.71) --
	(144.69, 65.71) --
	(144.75, 65.73) --
	(144.75, 65.73) --
	(144.75, 65.73) --
	(144.82, 65.72) --
	(144.82, 65.72) --
	(144.82, 65.72) --
	(144.89, 65.75) --
	(144.89, 65.75) --
	(144.89, 65.75) --
	(144.95, 65.78) --
	(144.95, 65.78) --
	(144.95, 65.78) --
	(145.02, 65.73) --
	(145.02, 65.73) --
	(145.02, 65.73) --
	(145.09, 65.76) --
	(145.09, 65.76) --
	(145.09, 65.76) --
	(145.16, 65.76) --
	(145.16, 65.76) --
	(145.16, 65.76) --
	(145.22, 65.74) --
	(145.22, 65.74) --
	(145.22, 65.74) --
	(145.29, 65.76) --
	(145.29, 65.76) --
	(145.29, 65.76) --
	(145.36, 65.76) --
	(145.36, 65.76) --
	(145.36, 65.76) --
	(145.42, 65.73) --
	(145.42, 65.73) --
	(145.42, 65.73) --
	(145.49, 65.75) --
	(145.49, 65.75) --
	(145.49, 65.75) --
	(145.56, 65.73) --
	(145.56, 65.73) --
	(145.56, 65.73) --
	(145.63, 65.75) --
	(145.63, 65.75) --
	(145.63, 65.75) --
	(145.69, 65.75) --
	(145.69, 65.75) --
	(145.69, 65.75) --
	(145.76, 65.74) --
	(145.76, 65.74) --
	(145.76, 65.74) --
	(145.83, 65.77) --
	(145.83, 65.77) --
	(145.83, 65.77) --
	(145.89, 65.76) --
	(145.89, 65.76) --
	(145.89, 65.76) --
	(145.96, 65.75) --
	(145.96, 65.75) --
	(145.96, 65.75) --
	(146.03, 65.73) --
	(146.03, 65.73) --
	(146.03, 65.73) --
	(146.10, 65.74) --
	(146.10, 65.74) --
	(146.10, 65.74) --
	(146.16, 65.75) --
	(146.16, 65.75) --
	(146.16, 65.75) --
	(146.23, 65.75) --
	(146.23, 65.75) --
	(146.23, 65.75) --
	(146.30, 65.76) --
	(146.30, 65.76) --
	(146.30, 65.76) --
	(146.36, 65.74) --
	(146.36, 65.74) --
	(146.36, 65.74) --
	(146.43, 65.76) --
	(146.43, 65.76) --
	(146.43, 65.76) --
	(146.50, 65.74) --
	(146.50, 65.74) --
	(146.50, 65.74) --
	(146.56, 65.73) --
	(146.56, 65.73) --
	(146.56, 65.73) --
	(146.63, 65.76) --
	(146.63, 65.76) --
	(146.63, 65.76) --
	(146.70, 65.74) --
	(146.70, 65.74) --
	(146.70, 65.74) --
	(146.77, 65.73) --
	(146.77, 65.73) --
	(146.77, 65.73) --
	(146.83, 65.77) --
	(146.83, 65.77) --
	(146.83, 65.77) --
	(146.90, 65.77) --
	(146.90, 65.77) --
	(146.90, 65.77) --
	(146.97, 65.74) --
	(146.97, 65.74) --
	(146.97, 65.74) --
	(147.03, 65.73) --
	(147.03, 65.73) --
	(147.03, 65.73) --
	(147.10, 65.74) --
	(147.10, 65.74) --
	(147.10, 65.74) --
	(147.17, 65.75) --
	(147.17, 65.75) --
	(147.17, 65.75) --
	(147.23, 65.73) --
	(147.23, 65.73) --
	(147.23, 65.73) --
	(147.30, 65.75) --
	(147.30, 65.75) --
	(147.30, 65.75) --
	(147.37, 65.78) --
	(147.37, 65.78) --
	(147.37, 65.78) --
	(147.44, 65.74) --
	(147.44, 65.74) --
	(147.44, 65.74) --
	(147.50, 65.73) --
	(147.50, 65.73) --
	(147.50, 65.73) --
	(147.57, 65.75) --
	(147.57, 65.75) --
	(147.57, 65.75) --
	(147.64, 65.75) --
	(147.64, 65.75) --
	(147.64, 65.75) --
	(147.70, 65.76) --
	(147.70, 65.76) --
	(147.70, 65.76) --
	(147.77, 65.73) --
	(147.77, 65.73) --
	(147.77, 65.73) --
	(147.84, 65.75) --
	(147.84, 65.75) --
	(147.84, 65.75) --
	(147.90, 65.76) --
	(147.90, 65.76) --
	(147.90, 65.76) --
	(147.97, 65.72) --
	(147.97, 65.72) --
	(147.97, 65.72) --
	(148.04, 65.73) --
	(148.04, 65.73) --
	(148.04, 65.73) --
	(148.10, 65.76) --
	(148.10, 65.76) --
	(148.10, 65.76) --
	(148.17, 65.73) --
	(148.17, 65.73) --
	(148.17, 65.73) --
	(148.24, 65.74) --
	(148.24, 65.74) --
	(148.24, 65.74) --
	(148.31, 65.75) --
	(148.31, 65.75) --
	(148.31, 65.75) --
	(148.37, 65.74) --
	(148.37, 65.74) --
	(148.37, 65.74) --
	(148.44, 65.77) --
	(148.44, 65.77) --
	(148.44, 65.77) --
	(148.51, 65.75) --
	(148.51, 65.75) --
	(148.51, 65.75) --
	(148.57, 65.76) --
	(148.57, 65.76) --
	(148.57, 65.76) --
	(148.64, 65.77) --
	(148.64, 65.77) --
	(148.64, 65.77) --
	(148.71, 65.73) --
	(148.71, 65.73) --
	(148.71, 65.73) --
	(148.77, 65.74) --
	(148.77, 65.74) --
	(148.77, 65.74) --
	(148.84, 65.80) --
	(148.84, 65.80) --
	(148.84, 65.80) --
	(148.91, 65.73) --
	(148.91, 65.73) --
	(148.91, 65.73) --
	(148.98, 65.76) --
	(148.98, 65.76) --
	(148.98, 65.76) --
	(149.04, 65.75) --
	(149.04, 65.75) --
	(149.04, 65.75) --
	(149.11, 65.75) --
	(149.11, 65.75) --
	(149.11, 65.75) --
	(149.18, 65.76) --
	(149.18, 65.76) --
	(149.18, 65.76) --
	(149.24, 65.73) --
	(149.24, 65.73) --
	(149.24, 65.73) --
	(149.31, 65.73) --
	(149.31, 65.73) --
	(149.31, 65.73) --
	(149.38, 65.75) --
	(149.38, 65.75) --
	(149.38, 65.75) --
	(149.44, 65.74) --
	(149.44, 65.74) --
	(149.44, 65.74) --
	(149.51, 65.73) --
	(149.51, 65.73) --
	(149.51, 65.73) --
	(149.58, 65.74) --
	(149.58, 65.74) --
	(149.58, 65.74) --
	(149.64, 65.73) --
	(149.64, 65.73) --
	(149.64, 65.73) --
	(149.71, 65.73) --
	(149.71, 65.73) --
	(149.71, 65.73) --
	(149.78, 65.74) --
	(149.78, 65.74) --
	(149.78, 65.74) --
	(149.84, 65.76) --
	(149.84, 65.76) --
	(149.84, 65.76) --
	(149.91, 65.78) --
	(149.91, 65.78) --
	(149.91, 65.78) --
	(149.98, 65.74) --
	(149.98, 65.74) --
	(149.98, 65.74) --
	(150.04, 65.74) --
	(150.04, 65.74) --
	(150.04, 65.74) --
	(150.11, 65.75) --
	(150.11, 65.75) --
	(150.11, 65.75) --
	(150.18, 65.74) --
	(150.18, 65.74) --
	(150.18, 65.74) --
	(150.24, 65.75) --
	(150.24, 65.75) --
	(150.24, 65.75) --
	(150.31, 65.77) --
	(150.31, 65.77) --
	(150.31, 65.77) --
	(150.38, 65.75) --
	(150.38, 65.75) --
	(150.38, 65.75) --
	(150.44, 65.76) --
	(150.44, 65.76) --
	(150.44, 65.76) --
	(150.51, 65.73) --
	(150.51, 65.73) --
	(150.51, 65.73) --
	(150.58, 65.75) --
	(150.58, 65.75) --
	(150.58, 65.75) --
	(150.64, 65.75) --
	(150.64, 65.75) --
	(150.64, 65.75) --
	(150.71, 65.72) --
	(150.71, 65.72) --
	(150.71, 65.72) --
	(150.78, 65.74) --
	(150.78, 65.74) --
	(150.78, 65.74) --
	(150.84, 65.76) --
	(150.84, 65.76) --
	(150.84, 65.76) --
	(150.91, 65.74) --
	(150.91, 65.74) --
	(150.91, 65.74) --
	(150.98, 65.74) --
	(150.98, 65.74) --
	(150.98, 65.74) --
	(151.04, 65.77) --
	(151.04, 65.77) --
	(151.04, 65.77) --
	(151.11, 65.73) --
	(151.11, 65.73) --
	(151.11, 65.73) --
	(151.18, 65.74) --
	(151.18, 65.74) --
	(151.18, 65.74) --
	(151.24, 65.73) --
	(151.24, 65.73) --
	(151.24, 65.73) --
	(151.31, 65.76) --
	(151.31, 65.76) --
	(151.31, 65.76) --
	(151.38, 65.74) --
	(151.38, 65.74) --
	(151.38, 65.74) --
	(151.44, 65.72) --
	(151.44, 65.72) --
	(151.44, 65.72) --
	(151.51, 65.77) --
	(151.51, 65.77) --
	(151.51, 65.77) --
	(151.58, 65.76) --
	(151.58, 65.76) --
	(151.58, 65.76) --
	(151.64, 65.74) --
	(151.64, 65.74) --
	(151.64, 65.71) --
	(151.64, 65.71) --
	(151.64, 65.71) --
	(151.58, 65.71) --
	(151.58, 65.71) --
	(151.58, 65.71) --
	(151.51, 65.71) --
	(151.51, 65.71) --
	(151.51, 65.71) --
	(151.44, 65.71) --
	(151.44, 65.71) --
	(151.44, 65.71) --
	(151.38, 65.71) --
	(151.38, 65.71) --
	(151.38, 65.71) --
	(151.31, 65.71) --
	(151.31, 65.71) --
	(151.31, 65.71) --
	(151.24, 65.71) --
	(151.24, 65.71) --
	(151.24, 65.71) --
	(151.18, 65.71) --
	(151.18, 65.71) --
	(151.18, 65.71) --
	(151.11, 65.71) --
	(151.11, 65.71) --
	(151.11, 65.71) --
	(151.04, 65.71) --
	(151.04, 65.71) --
	(151.04, 65.71) --
	(150.98, 65.71) --
	(150.98, 65.71) --
	(150.98, 65.71) --
	(150.91, 65.71) --
	(150.91, 65.71) --
	(150.91, 65.71) --
	(150.84, 65.71) --
	(150.84, 65.71) --
	(150.84, 65.71) --
	(150.78, 65.71) --
	(150.78, 65.71) --
	(150.78, 65.71) --
	(150.71, 65.71) --
	(150.71, 65.71) --
	(150.71, 65.71) --
	(150.64, 65.71) --
	(150.64, 65.71) --
	(150.64, 65.71) --
	(150.58, 65.71) --
	(150.58, 65.71) --
	(150.58, 65.71) --
	(150.51, 65.71) --
	(150.51, 65.71) --
	(150.51, 65.71) --
	(150.44, 65.71) --
	(150.44, 65.71) --
	(150.44, 65.71) --
	(150.38, 65.71) --
	(150.38, 65.71) --
	(150.38, 65.71) --
	(150.31, 65.71) --
	(150.31, 65.71) --
	(150.31, 65.71) --
	(150.24, 65.71) --
	(150.24, 65.71) --
	(150.24, 65.71) --
	(150.18, 65.71) --
	(150.18, 65.71) --
	(150.18, 65.71) --
	(150.11, 65.71) --
	(150.11, 65.71) --
	(150.11, 65.71) --
	(150.04, 65.71) --
	(150.04, 65.71) --
	(150.04, 65.71) --
	(149.98, 65.71) --
	(149.98, 65.71) --
	(149.98, 65.71) --
	(149.91, 65.71) --
	(149.91, 65.71) --
	(149.91, 65.71) --
	(149.84, 65.71) --
	(149.84, 65.71) --
	(149.84, 65.71) --
	(149.78, 65.71) --
	(149.78, 65.71) --
	(149.78, 65.71) --
	(149.71, 65.71) --
	(149.71, 65.71) --
	(149.71, 65.71) --
	(149.64, 65.71) --
	(149.64, 65.71) --
	(149.64, 65.71) --
	(149.58, 65.71) --
	(149.58, 65.71) --
	(149.58, 65.71) --
	(149.51, 65.71) --
	(149.51, 65.71) --
	(149.51, 65.71) --
	(149.44, 65.71) --
	(149.44, 65.71) --
	(149.44, 65.71) --
	(149.38, 65.71) --
	(149.38, 65.71) --
	(149.38, 65.71) --
	(149.31, 65.71) --
	(149.31, 65.71) --
	(149.31, 65.71) --
	(149.24, 65.71) --
	(149.24, 65.71) --
	(149.24, 65.71) --
	(149.18, 65.71) --
	(149.18, 65.71) --
	(149.18, 65.71) --
	(149.11, 65.71) --
	(149.11, 65.71) --
	(149.11, 65.71) --
	(149.04, 65.71) --
	(149.04, 65.71) --
	(149.04, 65.71) --
	(148.98, 65.71) --
	(148.98, 65.71) --
	(148.98, 65.71) --
	(148.91, 65.71) --
	(148.91, 65.71) --
	(148.91, 65.71) --
	(148.84, 65.71) --
	(148.84, 65.71) --
	(148.84, 65.71) --
	(148.77, 65.71) --
	(148.77, 65.71) --
	(148.77, 65.71) --
	(148.71, 65.71) --
	(148.71, 65.71) --
	(148.71, 65.71) --
	(148.64, 65.71) --
	(148.64, 65.71) --
	(148.64, 65.71) --
	(148.57, 65.71) --
	(148.57, 65.71) --
	(148.57, 65.71) --
	(148.51, 65.71) --
	(148.51, 65.71) --
	(148.51, 65.71) --
	(148.44, 65.71) --
	(148.44, 65.71) --
	(148.44, 65.71) --
	(148.37, 65.71) --
	(148.37, 65.71) --
	(148.37, 65.71) --
	(148.31, 65.71) --
	(148.31, 65.71) --
	(148.31, 65.71) --
	(148.24, 65.71) --
	(148.24, 65.71) --
	(148.24, 65.71) --
	(148.17, 65.71) --
	(148.17, 65.71) --
	(148.17, 65.71) --
	(148.10, 65.71) --
	(148.10, 65.71) --
	(148.10, 65.71) --
	(148.04, 65.71) --
	(148.04, 65.71) --
	(148.04, 65.71) --
	(147.97, 65.71) --
	(147.97, 65.71) --
	(147.97, 65.71) --
	(147.90, 65.71) --
	(147.90, 65.71) --
	(147.90, 65.71) --
	(147.84, 65.71) --
	(147.84, 65.71) --
	(147.84, 65.71) --
	(147.77, 65.71) --
	(147.77, 65.71) --
	(147.77, 65.71) --
	(147.70, 65.71) --
	(147.70, 65.71) --
	(147.70, 65.71) --
	(147.64, 65.71) --
	(147.64, 65.71) --
	(147.64, 65.71) --
	(147.57, 65.71) --
	(147.57, 65.71) --
	(147.57, 65.71) --
	(147.50, 65.71) --
	(147.50, 65.71) --
	(147.50, 65.71) --
	(147.44, 65.71) --
	(147.44, 65.71) --
	(147.44, 65.71) --
	(147.37, 65.71) --
	(147.37, 65.71) --
	(147.37, 65.71) --
	(147.30, 65.71) --
	(147.30, 65.71) --
	(147.30, 65.71) --
	(147.23, 65.71) --
	(147.23, 65.71) --
	(147.23, 65.71) --
	(147.17, 65.71) --
	(147.17, 65.71) --
	(147.17, 65.71) --
	(147.10, 65.71) --
	(147.10, 65.71) --
	(147.10, 65.71) --
	(147.03, 65.71) --
	(147.03, 65.71) --
	(147.03, 65.71) --
	(146.97, 65.71) --
	(146.97, 65.71) --
	(146.97, 65.71) --
	(146.90, 65.71) --
	(146.90, 65.71) --
	(146.90, 65.71) --
	(146.83, 65.71) --
	(146.83, 65.71) --
	(146.83, 65.71) --
	(146.77, 65.71) --
	(146.77, 65.71) --
	(146.77, 65.71) --
	(146.70, 65.71) --
	(146.70, 65.71) --
	(146.70, 65.71) --
	(146.63, 65.71) --
	(146.63, 65.71) --
	(146.63, 65.71) --
	(146.56, 65.71) --
	(146.56, 65.71) --
	(146.56, 65.71) --
	(146.50, 65.71) --
	(146.50, 65.71) --
	(146.50, 65.71) --
	(146.43, 65.71) --
	(146.43, 65.71) --
	(146.43, 65.71) --
	(146.36, 65.71) --
	(146.36, 65.71) --
	(146.36, 65.71) --
	(146.30, 65.71) --
	(146.30, 65.71) --
	(146.30, 65.71) --
	(146.23, 65.71) --
	(146.23, 65.71) --
	(146.23, 65.71) --
	(146.16, 65.71) --
	(146.16, 65.71) --
	(146.16, 65.71) --
	(146.10, 65.71) --
	(146.10, 65.71) --
	(146.10, 65.71) --
	(146.03, 65.71) --
	(146.03, 65.71) --
	(146.03, 65.71) --
	(145.96, 65.71) --
	(145.96, 65.71) --
	(145.96, 65.71) --
	(145.89, 65.71) --
	(145.89, 65.71) --
	(145.89, 65.71) --
	(145.83, 65.71) --
	(145.83, 65.71) --
	(145.83, 65.71) --
	(145.76, 65.71) --
	(145.76, 65.71) --
	(145.76, 65.71) --
	(145.69, 65.71) --
	(145.69, 65.71) --
	(145.69, 65.71) --
	(145.63, 65.71) --
	(145.63, 65.71) --
	(145.63, 65.71) --
	(145.56, 65.71) --
	(145.56, 65.71) --
	(145.56, 65.71) --
	(145.49, 65.71) --
	(145.49, 65.71) --
	(145.49, 65.71) --
	(145.42, 65.71) --
	(145.42, 65.71) --
	(145.42, 65.71) --
	(145.36, 65.71) --
	(145.36, 65.71) --
	(145.36, 65.71) --
	(145.29, 65.71) --
	(145.29, 65.71) --
	(145.29, 65.71) --
	(145.22, 65.71) --
	(145.22, 65.71) --
	(145.22, 65.71) --
	(145.16, 65.71) --
	(145.16, 65.71) --
	(145.16, 65.71) --
	(145.09, 65.71) --
	(145.09, 65.71) --
	(145.09, 65.71) --
	(145.02, 65.71) --
	(145.02, 65.71) --
	(145.02, 65.71) --
	(144.95, 65.71) --
	(144.95, 65.71) --
	(144.95, 65.71) --
	(144.89, 65.71) --
	(144.89, 65.71) --
	(144.89, 65.71) --
	(144.82, 65.71) --
	(144.82, 65.71) --
	(144.82, 65.71) --
	(144.75, 65.71) --
	(144.75, 65.71) --
	(144.75, 65.71) --
	(144.69, 65.71) --
	(144.69, 65.71) --
	(144.69, 65.71) --
	(144.62, 65.71) --
	(144.62, 65.71) --
	(144.62, 65.71) --
	(144.55, 65.71) --
	(144.55, 65.71) --
	(144.55, 65.71) --
	(144.48, 65.71) --
	(144.48, 65.71) --
	(144.48, 65.71) --
	(144.42, 65.71) --
	(144.42, 65.71) --
	(144.42, 65.71) --
	(144.35, 65.71) --
	(144.35, 65.71) --
	(144.35, 65.71) --
	(144.28, 65.71) --
	(144.28, 65.71) --
	(144.28, 65.71) --
	(144.21, 65.71) --
	(144.21, 65.71) --
	(144.21, 65.71) --
	(144.15, 65.71) --
	(144.15, 65.71) --
	(144.15, 65.71) --
	(144.08, 65.71) --
	(144.08, 65.71) --
	(144.08, 65.71) --
	(144.01, 65.71) --
	(144.01, 65.71) --
	(144.01, 65.71) --
	(143.95, 65.71) --
	(143.95, 65.71) --
	(143.95, 65.71) --
	(143.88, 65.71) --
	(143.88, 65.71) --
	(143.88, 65.71) --
	(143.81, 65.71) --
	(143.81, 65.71) --
	(143.81, 65.71) --
	(143.74, 65.71) --
	(143.74, 65.71) --
	(143.74, 65.71) --
	(143.68, 65.71) --
	(143.68, 65.71) --
	(143.68, 65.71) --
	(143.61, 65.71) --
	(143.61, 65.71) --
	(143.61, 65.71) --
	(143.54, 65.71) --
	(143.54, 65.71) --
	(143.54, 65.71) --
	(143.48, 65.71) --
	(143.48, 65.71) --
	(143.48, 65.71) --
	(143.41, 65.71) --
	(143.41, 65.71) --
	(143.41, 65.71) --
	(143.34, 65.71) --
	(143.34, 65.71) --
	(143.34, 65.71) --
	(143.27, 65.71) --
	(143.27, 65.71) --
	(143.27, 65.71) --
	(143.21, 65.71) --
	(143.21, 65.71) --
	(143.21, 65.71) --
	(143.14, 65.71) --
	(143.14, 65.71) --
	(143.14, 65.71) --
	(143.07, 65.71) --
	(143.07, 65.71) --
	(143.07, 65.71) --
	(143.00, 65.71) --
	(143.00, 65.71) --
	(143.00, 65.71) --
	(142.94, 65.71) --
	(142.94, 65.71) --
	(142.94, 65.71) --
	(142.87, 65.71) --
	(142.87, 65.71) --
	(142.87, 65.71) --
	(142.80, 65.71) --
	(142.80, 65.71) --
	(142.80, 65.71) --
	(142.73, 65.71) --
	(142.73, 65.71) --
	(142.73, 65.71) --
	(142.67, 65.71) --
	(142.67, 65.71) --
	(142.67, 65.71) --
	(142.60, 65.71) --
	(142.60, 65.71) --
	(142.60, 65.71) --
	(142.53, 65.71) --
	(142.53, 65.71) --
	(142.53, 65.71) --
	(142.46, 65.71) --
	(142.46, 65.71) --
	(142.46, 65.71) --
	(142.40, 65.71) --
	(142.40, 65.71) --
	(142.40, 65.71) --
	(142.33, 65.71) --
	(142.33, 65.71) --
	(142.33, 65.71) --
	(142.26, 65.71) --
	(142.26, 65.71) --
	(142.26, 65.71) --
	(142.19, 65.71) --
	(142.19, 65.71) --
	(142.19, 65.71) --
	(142.13, 65.71) --
	(142.13, 65.71) --
	(142.13, 65.71) --
	(142.06, 65.71) --
	(142.06, 65.71) --
	(142.06, 65.71) --
	(141.99, 65.71) --
	(141.99, 65.71) --
	(141.99, 65.71) --
	(141.92, 65.71) --
	(141.92, 65.71) --
	(141.92, 65.71) --
	(141.86, 65.71) --
	(141.86, 65.71) --
	(141.86, 65.71) --
	(141.79, 65.71) --
	(141.79, 65.71) --
	(141.79, 65.71) --
	(141.72, 65.71) --
	(141.72, 65.71) --
	(141.72, 65.71) --
	(141.66, 65.71) --
	(141.66, 65.71) --
	(141.66, 65.71) --
	(141.59, 65.71) --
	(141.59, 65.71) --
	(141.59, 65.71) --
	(141.52, 65.71) --
	(141.52, 65.71) --
	(141.52, 65.71) --
	(141.45, 65.71) --
	(141.45, 65.71) --
	(141.45, 65.71) --
	(141.39, 65.71) --
	(141.39, 65.71) --
	(141.39, 65.71) --
	(141.32, 65.71) --
	(141.32, 65.71) --
	(141.32, 65.71) --
	(141.25, 65.71) --
	(141.25, 65.71) --
	(141.25, 65.71) --
	(141.18, 65.71) --
	(141.18, 65.71) --
	(141.18, 65.71) --
	(141.12, 65.71) --
	(141.12, 65.71) --
	(141.12, 65.71) --
	(141.05, 65.71) --
	(141.05, 65.71) --
	(141.05, 65.71) --
	(140.98, 65.71) --
	(140.98, 65.71) --
	(140.98, 65.71) --
	(140.91, 65.71) --
	(140.91, 65.71) --
	(140.91, 65.71) --
	(140.85, 65.71) --
	(140.85, 65.71) --
	(140.85, 65.71) --
	(140.78, 65.71) --
	(140.78, 65.71) --
	(140.78, 65.71) --
	(140.71, 65.71) --
	(140.71, 65.71) --
	(140.71, 65.71) --
	(140.64, 65.71) --
	(140.64, 65.71) --
	(140.64, 65.71) --
	(140.58, 65.71) --
	(140.58, 65.71) --
	(140.58, 65.71) --
	(140.51, 65.71) --
	(140.51, 65.71) --
	(140.51, 65.71) --
	(140.44, 65.71) --
	(140.44, 65.71) --
	(140.44, 65.71) --
	(140.37, 65.71) --
	(140.37, 65.71) --
	(140.37, 65.71) --
	(140.31, 65.71) --
	(140.31, 65.71) --
	(140.31, 65.71) --
	(140.24, 65.71) --
	(140.24, 65.71) --
	(140.24, 65.71) --
	(140.17, 65.71) --
	(140.17, 65.71) --
	(140.17, 65.71) --
	(140.10, 65.71) --
	(140.10, 65.71) --
	(140.10, 65.71) --
	(140.04, 65.71) --
	(140.04, 65.71) --
	(140.04, 65.71) --
	(139.97, 65.71) --
	(139.97, 65.71) --
	(139.97, 65.71) --
	(139.90, 65.71) --
	(139.90, 65.71) --
	(139.90, 65.71) --
	(139.83, 65.71) --
	(139.83, 65.71) --
	(139.83, 65.71) --
	(139.76, 65.71) --
	(139.76, 65.71) --
	(139.76, 65.71) --
	(139.70, 65.71) --
	(139.70, 65.71) --
	(139.70, 65.71) --
	(139.63, 65.71) --
	(139.63, 65.71) --
	(139.63, 65.71) --
	(139.56, 65.71) --
	(139.56, 65.71) --
	(139.56, 65.71) --
	(139.49, 65.71) --
	(139.49, 65.71) --
	(139.49, 65.71) --
	(139.43, 65.71) --
	(139.43, 65.71) --
	(139.43, 65.71) --
	(139.36, 65.71) --
	(139.36, 65.71) --
	(139.36, 65.71) --
	(139.29, 65.71) --
	(139.29, 65.71) --
	(139.29, 65.71) --
	(139.22, 65.71) --
	(139.22, 65.71) --
	(139.22, 65.71) --
	(139.16, 65.71) --
	(139.16, 65.71) --
	(139.16, 65.71) --
	(139.09, 65.71) --
	(139.09, 65.71) --
	(139.09, 65.71) --
	(139.02, 65.71) --
	(139.02, 65.71) --
	(139.02, 65.71) --
	(138.95, 65.71) --
	(138.95, 65.71) --
	(138.95, 65.71) --
	(138.89, 65.71) --
	(138.89, 65.71) --
	(138.89, 65.71) --
	(138.82, 65.71) --
	(138.82, 65.71) --
	(138.82, 65.71) --
	(138.75, 65.71) --
	(138.75, 65.71) --
	(138.75, 65.71) --
	(138.68, 65.71) --
	(138.68, 65.71) --
	(138.68, 65.71) --
	(138.61, 65.71) --
	(138.61, 65.71) --
	(138.61, 65.71) --
	(138.55, 65.71) --
	(138.55, 65.71) --
	(138.55, 65.71) --
	(138.48, 65.71) --
	(138.48, 65.71) --
	(138.48, 65.71) --
	(138.41, 65.71) --
	(138.41, 65.71) --
	(138.41, 65.71) --
	(138.34, 65.71) --
	(138.34, 65.71) --
	(138.34, 65.71) --
	(138.28, 65.71) --
	(138.28, 65.71) --
	(138.28, 65.71) --
	(138.21, 65.71) --
	(138.21, 65.71) --
	(138.21, 65.71) --
	(138.14, 65.71) --
	(138.14, 65.71) --
	(138.14, 65.71) --
	(138.07, 65.71) --
	(138.07, 65.71) --
	(138.07, 65.71) --
	(138.00, 65.71) --
	(138.00, 65.71) --
	(138.00, 65.71) --
	(137.94, 65.71) --
	(137.94, 65.71) --
	(137.94, 65.71) --
	(137.87, 65.71) --
	(137.87, 65.71) --
	(137.87, 65.71) --
	(137.80, 65.71) --
	(137.80, 65.71) --
	(137.80, 65.71) --
	(137.73, 65.71) --
	(137.73, 65.71) --
	(137.73, 65.71) --
	(137.67, 65.71) --
	(137.67, 65.71) --
	(137.67, 65.71) --
	(137.60, 65.71) --
	(137.60, 65.71) --
	(137.60, 65.71) --
	(137.53, 65.71) --
	(137.53, 65.71) --
	(137.53, 65.71) --
	(137.46, 65.71) --
	(137.46, 65.71) --
	(137.46, 65.71) --
	(137.40, 65.71) --
	(137.40, 65.71) --
	(137.40, 65.71) --
	(137.33, 65.71) --
	(137.33, 65.71) --
	(137.33, 65.71) --
	(137.26, 65.71) --
	(137.26, 65.71) --
	(137.26, 65.71) --
	(137.19, 65.71) --
	(137.19, 65.71) --
	(137.19, 65.71) --
	(137.12, 65.71) --
	(137.12, 65.71) --
	(137.12, 65.71) --
	(137.06, 65.71) --
	(137.06, 65.71) --
	(137.06, 65.71) --
	(136.99, 65.71) --
	(136.99, 65.71) --
	(136.99, 65.71) --
	(136.92, 65.71) --
	(136.92, 65.71) --
	(136.92, 65.71) --
	(136.85, 65.71) --
	(136.85, 65.71) --
	(136.85, 65.71) --
	(136.78, 65.71) --
	(136.78, 65.71) --
	(136.78, 65.71) --
	(136.72, 65.71) --
	(136.72, 65.71) --
	(136.72, 65.71) --
	(136.65, 65.71) --
	(136.65, 65.71) --
	(136.65, 65.71) --
	(136.58, 65.71) --
	(136.58, 65.71) --
	(136.58, 65.71) --
	(136.51, 65.71) --
	(136.51, 65.71) --
	(136.51, 65.71) --
	(136.44, 65.71) --
	(136.44, 65.71) --
	(136.44, 65.71) --
	(136.38, 65.71) --
	(136.38, 65.71) --
	(136.38, 65.71) --
	(136.31, 65.71) --
	(136.31, 65.71) --
	(136.31, 65.71) --
	(136.24, 65.71) --
	(136.24, 65.71) --
	(136.24, 65.71) --
	(136.17, 65.71) --
	(136.17, 65.71) --
	(136.17, 65.71) --
	(136.11, 65.71) --
	(136.11, 65.71) --
	(136.11, 65.71) --
	(136.04, 65.71) --
	(136.04, 65.71) --
	(136.04, 65.71) --
	(135.97, 65.71) --
	(135.97, 65.71) --
	(135.97, 65.71) --
	(135.90, 65.71) --
	(135.90, 65.71) --
	(135.90, 65.71) --
	(135.83, 65.71) --
	(135.83, 65.71) --
	(135.83, 65.71) --
	(135.77, 65.71) --
	(135.77, 65.71) --
	(135.77, 65.71) --
	(135.70, 65.71) --
	(135.70, 65.71) --
	(135.70, 65.71) --
	(135.63, 65.71) --
	(135.63, 65.71) --
	(135.63, 65.71) --
	(135.56, 65.71) --
	(135.56, 65.71) --
	(135.56, 65.71) --
	(135.49, 65.71) --
	(135.49, 65.71) --
	(135.49, 65.71) --
	(135.43, 65.71) --
	(135.43, 65.71) --
	(135.43, 65.71) --
	(135.36, 65.71) --
	(135.36, 65.71) --
	(135.36, 65.71) --
	(135.29, 65.71) --
	(135.29, 65.71) --
	(135.29, 65.71) --
	(135.22, 65.71) --
	(135.22, 65.71) --
	(135.22, 65.71) --
	(135.15, 65.71) --
	(135.15, 65.71) --
	(135.15, 65.71) --
	(135.09, 65.71) --
	(135.09, 65.71) --
	(135.09, 65.71) --
	(135.02, 65.71) --
	(135.02, 65.71) --
	(135.02, 65.71) --
	(134.95, 65.71) --
	(134.95, 65.71) --
	(134.95, 65.71) --
	(134.88, 65.71) --
	(134.88, 65.71) --
	(134.88, 65.71) --
	(134.81, 65.71) --
	(134.81, 65.71) --
	(134.81, 65.71) --
	(134.75, 65.71) --
	(134.75, 65.71) --
	(134.75, 65.71) --
	(134.68, 65.71) --
	(134.68, 65.71) --
	(134.68, 65.71) --
	(134.61, 65.71) --
	(134.61, 65.71) --
	(134.61, 65.71) --
	(134.54, 65.71) --
	(134.54, 65.71) --
	(134.54, 65.71) --
	(134.47, 65.71) --
	(134.47, 65.71) --
	(134.47, 65.71) --
	(134.41, 65.71) --
	(134.41, 65.71) --
	(134.41, 65.71) --
	(134.34, 65.71) --
	(134.34, 65.71) --
	(134.34, 65.71) --
	(134.27, 65.71) --
	(134.27, 65.71) --
	(134.27, 65.71) --
	(134.20, 65.71) --
	(134.20, 65.71) --
	(134.20, 65.71) --
	(134.13, 65.71) --
	(134.13, 65.71) --
	(134.13, 65.71) --
	(134.07, 65.71) --
	(134.07, 65.71) --
	(134.07, 65.71) --
	(134.00, 65.71) --
	(134.00, 65.71) --
	(134.00, 65.71) --
	(133.93, 65.71) --
	(133.93, 65.71) --
	(133.93, 65.71) --
	(133.86, 65.71) --
	(133.86, 65.71) --
	(133.86, 65.71) --
	(133.79, 65.71) --
	(133.79, 65.71) --
	(133.79, 65.71) --
	(133.72, 65.71) --
	(133.72, 65.71) --
	(133.72, 65.71) --
	(133.66, 65.71) --
	(133.66, 65.71) --
	(133.66, 65.71) --
	(133.59, 65.71) --
	(133.59, 65.71) --
	(133.59, 65.71) --
	(133.52, 65.71) --
	(133.52, 65.71) --
	(133.52, 65.71) --
	(133.45, 65.71) --
	(133.45, 65.71) --
	(133.45, 65.71) --
	(133.38, 65.71) --
	(133.38, 65.71) --
	(133.38, 65.71) --
	(133.32, 65.71) --
	(133.32, 65.71) --
	(133.32, 65.71) --
	(133.25, 65.71) --
	(133.25, 65.71) --
	(133.25, 65.71) --
	(133.18, 65.71) --
	(133.18, 65.71) --
	(133.18, 65.71) --
	(133.11, 65.71) --
	(133.11, 65.71) --
	(133.11, 65.71) --
	(133.04, 65.71) --
	(133.04, 65.71) --
	(133.04, 65.71) --
	(132.98, 65.71) --
	(132.98, 65.71) --
	(132.98, 65.71) --
	(132.91, 65.71) --
	(132.91, 65.71) --
	(132.91, 65.71) --
	(132.84, 65.71) --
	(132.84, 65.71) --
	(132.84, 65.71) --
	(132.77, 65.71) --
	(132.77, 65.71) --
	(132.77, 65.71) --
	(132.70, 65.71) --
	(132.70, 65.71) --
	(132.70, 65.71) --
	(132.63, 65.71) --
	(132.63, 65.71) --
	(132.63, 65.71) --
	(132.57, 65.71) --
	(132.57, 65.71) --
	(132.57, 65.71) --
	(132.50, 65.71) --
	(132.50, 65.71) --
	(132.50, 65.71) --
	(132.43, 65.71) --
	(132.43, 65.71) --
	(132.43, 65.71) --
	(132.36, 65.71) --
	(132.36, 65.71) --
	(132.36, 65.71) --
	(132.29, 65.71) --
	(132.29, 65.71) --
	(132.29, 65.71) --
	(132.22, 65.71) --
	(132.22, 65.71) --
	(132.22, 65.71) --
	(132.16, 65.71) --
	(132.16, 65.71) --
	(132.16, 65.71) --
	(132.09, 65.71) --
	(132.09, 65.71) --
	(132.09, 65.71) --
	(132.02, 65.71) --
	(132.02, 65.71) --
	(132.02, 65.71) --
	(131.95, 65.71) --
	(131.95, 65.71) --
	(131.95, 65.71) --
	(131.88, 65.71) --
	(131.88, 65.71) --
	(131.88, 65.71) --
	(131.82, 65.71) --
	(131.82, 65.71) --
	(131.82, 65.71) --
	(131.75, 65.71) --
	(131.75, 65.71) --
	(131.75, 65.71) --
	(131.68, 65.71) --
	(131.68, 65.71) --
	(131.68, 65.71) --
	(131.61, 65.71) --
	(131.61, 65.71) --
	(131.61, 65.71) --
	(131.54, 65.71) --
	(131.54, 65.71) --
	(131.54, 65.71) --
	(131.47, 65.71) --
	(131.47, 65.71) --
	(131.47, 65.71) --
	(131.41, 65.71) --
	(131.41, 65.71) --
	(131.41, 65.71) --
	(131.34, 65.71) --
	(131.34, 65.71) --
	(131.34, 65.71) --
	(131.27, 65.71) --
	(131.27, 65.71) --
	(131.27, 65.71) --
	(131.20, 65.71) --
	(131.20, 65.71) --
	(131.20, 65.71) --
	(131.13, 65.71) --
	(131.13, 65.71) --
	(131.13, 65.71) --
	(131.06, 65.71) --
	(131.06, 65.71) --
	(131.06, 65.71) --
	(131.00, 65.71) --
	(131.00, 65.71) --
	(131.00, 65.71) --
	(130.93, 65.71) --
	(130.93, 65.71) --
	(130.93, 65.71) --
	(130.86, 65.71) --
	(130.86, 65.71) --
	(130.86, 65.71) --
	(130.79, 65.71) --
	(130.79, 65.71) --
	(130.79, 65.71) --
	(130.72, 65.71) --
	(130.72, 65.71) --
	(130.72, 65.71) --
	(130.65, 65.71) --
	(130.65, 65.71) --
	(130.65, 65.71) --
	(130.59, 65.71) --
	(130.59, 65.71) --
	(130.59, 65.71) --
	(130.52, 65.71) --
	(130.52, 65.71) --
	(130.52, 65.71) --
	(130.45, 65.71) --
	(130.45, 65.71) --
	(130.45, 65.71) --
	(130.38, 65.71) --
	(130.38, 65.71) --
	(130.38, 65.71) --
	(130.31, 65.71) --
	(130.31, 65.71) --
	(130.31, 65.71) --
	(130.24, 65.71) --
	(130.24, 65.71) --
	(130.24, 65.71) --
	(130.18, 65.71) --
	(130.18, 65.71) --
	(130.18, 65.71) --
	(130.11, 65.71) --
	(130.11, 65.71) --
	(130.11, 65.71) --
	(130.04, 65.71) --
	(130.04, 65.71) --
	(130.04, 65.71) --
	(129.97, 65.71) --
	(129.97, 65.71) --
	(129.97, 65.71) --
	(129.90, 65.71) --
	(129.90, 65.71) --
	(129.90, 65.71) --
	(129.83, 65.71) --
	(129.83, 65.71) --
	(129.83, 65.71) --
	(129.77, 65.71) --
	(129.77, 65.71) --
	(129.77, 65.71) --
	(129.70, 65.71) --
	(129.70, 65.71) --
	(129.70, 65.71) --
	(129.63, 65.71) --
	(129.63, 65.71) --
	(129.63, 65.71) --
	(129.56, 65.71) --
	(129.56, 65.71) --
	(129.56, 65.71) --
	(129.49, 65.71) --
	(129.49, 65.71) --
	(129.49, 65.71) --
	(129.42, 65.71) --
	(129.42, 65.71) --
	(129.42, 65.71) --
	(129.35, 65.71) --
	(129.35, 65.71) --
	(129.35, 65.71) --
	(129.29, 65.71) --
	(129.29, 65.71) --
	(129.29, 65.71) --
	(129.22, 65.71) --
	(129.22, 65.71) --
	(129.22, 65.71) --
	(129.15, 65.71) --
	(129.15, 65.71) --
	(129.15, 65.71) --
	(129.08, 65.71) --
	(129.08, 65.71) --
	(129.08, 65.71) --
	(129.01, 65.71) --
	(129.01, 65.71) --
	(129.01, 65.71) --
	(128.94, 65.71) --
	(128.94, 65.71) --
	(128.94, 65.71) --
	(128.88, 65.71) --
	(128.88, 65.71) --
	(128.88, 65.71) --
	(128.81, 65.71) --
	(128.81, 65.71) --
	(128.81, 65.71) --
	(128.74, 65.71) --
	(128.74, 65.71) --
	(128.74, 65.71) --
	(128.67, 65.71) --
	(128.67, 65.71) --
	(128.67, 65.71) --
	(128.60, 65.71) --
	(128.60, 65.71) --
	(128.60, 65.71) --
	(128.53, 65.71) --
	(128.53, 65.71) --
	(128.53, 65.71) --
	(128.46, 65.71) --
	(128.46, 65.71) --
	(128.46, 65.71) --
	(128.40, 65.71) --
	(128.40, 65.71) --
	(128.40, 65.71) --
	(128.33, 65.71) --
	(128.33, 65.71) --
	(128.33, 65.71) --
	(128.26, 65.71) --
	(128.26, 65.71) --
	(128.26, 65.71) --
	(128.19, 65.71) --
	(128.19, 65.71) --
	(128.19, 65.71) --
	(128.12, 65.71) --
	(128.12, 65.71) --
	(128.12, 65.71) --
	(128.05, 65.71) --
	(128.05, 65.71) --
	(128.05, 65.71) --
	(127.98, 65.71) --
	(127.98, 65.71) --
	(127.98, 65.71) --
	(127.91, 65.71) --
	(127.91, 65.71) --
	(127.91, 65.71) --
	(127.85, 65.71) --
	(127.85, 65.71) --
	(127.85, 65.71) --
	(127.78, 65.71) --
	(127.78, 65.71) --
	(127.78, 65.71) --
	(127.71, 65.71) --
	(127.71, 65.71) --
	(127.71, 65.71) --
	(127.64, 65.71) --
	(127.64, 65.71) --
	(127.64, 65.71) --
	(127.57, 65.71) --
	(127.57, 65.71) --
	(127.57, 65.71) --
	(127.50, 65.71) --
	(127.50, 65.71) --
	(127.50, 65.71) --
	(127.44, 65.71) --
	(127.44, 65.71) --
	(127.44, 65.71) --
	(127.37, 65.71) --
	(127.37, 65.71) --
	(127.37, 65.71) --
	(127.30, 65.71) --
	(127.30, 65.71) --
	(127.30, 65.71) --
	(127.23, 65.71) --
	(127.23, 65.71) --
	(127.23, 65.71) --
	(127.16, 65.71) --
	(127.16, 65.71) --
	(127.16, 65.71) --
	(127.09, 65.71) --
	(127.09, 65.71) --
	(127.09, 65.71) --
	(127.02, 65.71) --
	(127.02, 65.71) --
	(127.02, 65.71) --
	(126.95, 65.71) --
	(126.95, 65.71) --
	(126.95, 65.71) --
	(126.89, 65.71) --
	(126.89, 65.71) --
	(126.89, 65.71) --
	(126.82, 65.71) --
	(126.82, 65.71) --
	(126.82, 65.71) --
	(126.75, 65.71) --
	(126.75, 65.71) --
	(126.75, 65.71) --
	(126.68, 65.71) --
	(126.68, 65.71) --
	(126.68, 65.71) --
	(126.61, 65.71) --
	(126.61, 65.71) --
	(126.61, 65.71) --
	(126.54, 65.71) --
	(126.54, 65.71) --
	(126.54, 65.71) --
	(126.47, 65.71) --
	(126.47, 65.71) --
	(126.47, 65.71) --
	(126.41, 65.71) --
	(126.41, 65.71) --
	(126.41, 65.71) --
	(126.34, 65.71) --
	(126.34, 65.71) --
	(126.34, 65.71) --
	(126.27, 65.71) --
	(126.27, 65.71) --
	(126.27, 65.71) --
	(126.20, 65.71) --
	(126.20, 65.71) --
	(126.20, 65.71) --
	(126.13, 65.71) --
	(126.13, 65.71) --
	(126.13, 65.71) --
	(126.06, 65.71) --
	(126.06, 65.71) --
	(126.06, 65.71) --
	(125.99, 65.71) --
	(125.99, 65.71) --
	(125.99, 65.71) --
	(125.92, 65.71) --
	(125.92, 65.71) --
	(125.92, 65.71) --
	(125.86, 65.71) --
	(125.86, 65.71) --
	(125.86, 65.71) --
	(125.79, 65.71) --
	(125.79, 65.71) --
	(125.79, 65.71) --
	(125.72, 65.71) --
	(125.72, 65.71) --
	(125.72, 65.71) --
	(125.65, 65.71) --
	(125.65, 65.71) --
	(125.65, 65.71) --
	(125.58, 65.71) --
	(125.58, 65.71) --
	(125.58, 65.71) --
	(125.51, 65.71) --
	(125.51, 65.71) --
	(125.51, 65.71) --
	(125.44, 65.71) --
	(125.44, 65.71) --
	(125.44, 65.71) --
	(125.37, 65.71) --
	(125.37, 65.71) --
	(125.37, 65.71) --
	(125.31, 65.71) --
	(125.31, 65.71) --
	(125.31, 65.71) --
	(125.24, 65.71) --
	(125.24, 65.71) --
	(125.24, 65.71) --
	(125.17, 65.71) --
	(125.17, 65.71) --
	(125.17, 65.71) --
	(125.10, 65.71) --
	(125.10, 65.71) --
	(125.10, 65.71) --
	(125.03, 65.71) --
	(125.03, 65.71) --
	(125.03, 65.71) --
	(124.96, 65.71) --
	(124.96, 65.71) --
	(124.96, 65.71) --
	(124.89, 65.71) --
	(124.89, 65.71) --
	(124.89, 65.71) --
	(124.82, 65.71) --
	(124.82, 65.71) --
	(124.82, 65.71) --
	(124.75, 65.71) --
	(124.75, 65.71) --
	(124.75, 65.71) --
	(124.69, 65.71) --
	(124.69, 65.71) --
	(124.69, 65.71) --
	(124.62, 65.71) --
	(124.62, 65.71) --
	(124.62, 65.71) --
	(124.55, 65.71) --
	(124.55, 65.71) --
	(124.55, 65.71) --
	(124.48, 65.71) --
	(124.48, 65.71) --
	(124.48, 65.71) --
	(124.41, 65.71) --
	(124.41, 65.71) --
	(124.41, 65.71) --
	(124.34, 65.71) --
	(124.34, 65.71) --
	(124.34, 65.71) --
	(124.27, 65.71) --
	(124.27, 65.71) --
	(124.27, 65.71) --
	(124.20, 65.71) --
	(124.20, 65.71) --
	(124.20, 65.71) --
	(124.13, 65.71) --
	(124.13, 65.71) --
	(124.13, 65.71) --
	(124.07, 65.71) --
	(124.07, 65.71) --
	(124.07, 65.71) --
	(124.00, 65.71) --
	(124.00, 65.71) --
	(124.00, 65.71) --
	(123.93, 65.71) --
	(123.93, 65.71) --
	(123.93, 65.71) --
	(123.86, 65.71) --
	(123.86, 65.71) --
	(123.86, 65.71) --
	(123.79, 65.71) --
	(123.79, 65.71) --
	(123.79, 65.71) --
	(123.72, 65.71) --
	(123.72, 65.71) --
	(123.72, 65.71) --
	(123.65, 65.71) --
	(123.65, 65.71) --
	(123.65, 65.71) --
	(123.58, 65.71) --
	(123.58, 65.71) --
	(123.58, 65.71) --
	(123.51, 65.71) --
	(123.51, 65.71) --
	(123.51, 65.71) --
	(123.45, 65.71) --
	(123.45, 65.71) --
	(123.45, 65.71) --
	(123.38, 65.71) --
	(123.38, 65.71) --
	(123.38, 65.71) --
	(123.31, 65.71) --
	(123.31, 65.71) --
	(123.31, 65.71) --
	(123.24, 65.71) --
	(123.24, 65.71) --
	(123.24, 65.71) --
	(123.17, 65.71) --
	(123.17, 65.71) --
	(123.17, 65.71) --
	(123.10, 65.71) --
	(123.10, 65.71) --
	(123.10, 65.71) --
	(123.03, 65.71) --
	(123.03, 65.71) --
	(123.03, 65.71) --
	(122.96, 65.71) --
	(122.96, 65.71) --
	(122.96, 65.71) --
	(122.89, 65.71) --
	(122.89, 65.71) --
	(122.89, 65.71) --
	(122.83, 65.71) --
	(122.83, 65.71) --
	(122.83, 65.71) --
	(122.76, 65.71) --
	(122.76, 65.71) --
	(122.76, 65.71) --
	(122.69, 65.71) --
	(122.69, 65.71) --
	(122.69, 65.71) --
	(122.62, 65.71) --
	(122.62, 65.71) --
	(122.62, 65.71) --
	(122.55, 65.71) --
	(122.55, 65.71) --
	(122.55, 65.71) --
	(122.48, 65.71) --
	(122.48, 65.71) --
	(122.48, 65.71) --
	(122.41, 65.71) --
	(122.41, 65.71) --
	(122.41, 65.71) --
	(122.34, 65.71) --
	(122.34, 65.71) --
	(122.34, 65.71) --
	(122.27, 65.71) --
	(122.27, 65.71) --
	(122.27, 65.71) --
	(122.21, 65.71) --
	(122.21, 65.71) --
	(122.20, 65.71) --
	(122.14, 65.71) --
	(122.14, 65.71) --
	(122.14, 65.71) --
	(122.07, 65.71) --
	(122.07, 65.71) --
	(122.07, 65.71) --
	(122.00, 65.71) --
	(122.00, 65.71) --
	(122.00, 65.71) --
	(121.93, 65.71) --
	(121.93, 65.71) --
	(121.93, 65.71) --
	(121.86, 65.71) --
	(121.86, 65.71) --
	(121.86, 65.71) --
	(121.79, 65.71) --
	(121.79, 65.71) --
	(121.79, 65.71) --
	(121.72, 65.71) --
	(121.72, 65.71) --
	(121.72, 65.71) --
	(121.65, 65.71) --
	(121.65, 65.71) --
	(121.65, 65.71) --
	(121.58, 65.71) --
	(121.58, 65.71) --
	(121.58, 65.71) --
	(121.51, 65.71) --
	(121.51, 65.71) --
	(121.51, 65.71) --
	(121.44, 65.71) --
	(121.44, 65.71) --
	(121.44, 65.71) --
	(121.38, 65.71) --
	(121.38, 65.71) --
	(121.38, 65.71) --
	(121.31, 65.71) --
	(121.31, 65.71) --
	(121.31, 65.71) --
	(121.24, 65.71) --
	(121.24, 65.71) --
	(121.24, 65.71) --
	(121.17, 65.71) --
	(121.17, 65.71) --
	(121.17, 65.71) --
	(121.10, 65.71) --
	(121.10, 65.71) --
	(121.10, 65.71) --
	(121.03, 65.71) --
	(121.03, 65.71) --
	(121.03, 65.71) --
	(120.96, 65.71) --
	(120.96, 65.71) --
	(120.96, 65.71) --
	(120.89, 65.71) --
	(120.89, 65.71) --
	(120.89, 65.71) --
	(120.82, 65.71) --
	(120.82, 65.71) --
	(120.82, 65.71) --
	(120.75, 65.71) --
	(120.75, 65.71) --
	(120.75, 65.71) --
	(120.69, 65.71) --
	(120.69, 65.71) --
	(120.69, 65.71) --
	(120.62, 65.71) --
	(120.62, 65.71) --
	(120.62, 65.71) --
	(120.55, 65.71) --
	(120.55, 65.71) --
	(120.55, 65.71) --
	(120.48, 65.71) --
	(120.48, 65.71) --
	(120.48, 65.71) --
	(120.41, 65.71) --
	(120.41, 65.71) --
	(120.41, 65.71) --
	(120.34, 65.71) --
	(120.34, 65.71) --
	(120.34, 65.71) --
	(120.27, 65.71) --
	(120.27, 65.71) --
	(120.27, 65.71) --
	(120.20, 65.71) --
	(120.20, 65.71) --
	(120.20, 65.71) --
	(120.13, 65.71) --
	(120.13, 65.71) --
	(120.13, 65.71) --
	(120.06, 65.71) --
	(120.06, 65.71) --
	(120.06, 65.71) --
	(119.99, 65.71) --
	(119.99, 65.71) --
	(119.99, 65.71) --
	(119.92, 65.71) --
	(119.92, 65.71) --
	(119.92, 65.71) --
	(119.86, 65.71) --
	(119.86, 65.71) --
	(119.86, 65.71) --
	(119.79, 65.71) --
	(119.79, 65.71) --
	(119.79, 65.71) --
	(119.72, 65.71) --
	(119.72, 65.71) --
	(119.72, 65.71) --
	(119.65, 65.71) --
	(119.65, 65.71) --
	(119.65, 65.71) --
	(119.58, 65.71) --
	(119.58, 65.71) --
	(119.58, 65.71) --
	(119.51, 65.71) --
	(119.51, 65.71) --
	(119.51, 65.71) --
	(119.44, 65.71) --
	(119.44, 65.71) --
	(119.44, 65.71) --
	(119.37, 65.71) --
	(119.37, 65.71) --
	(119.37, 65.71) --
	(119.30, 65.71) --
	(119.30, 65.71) --
	(119.30, 65.71) --
	(119.23, 65.71) --
	(119.23, 65.71) --
	(119.23, 65.71) --
	(119.16, 65.71) --
	(119.16, 65.71) --
	(119.16, 65.71) --
	(119.09, 65.71) --
	(119.09, 65.71) --
	(119.09, 65.71) --
	(119.02, 65.71) --
	(119.02, 65.71) --
	(119.02, 65.71) --
	(118.95, 65.71) --
	(118.95, 65.71) --
	(118.95, 65.71) --
	(118.88, 65.71) --
	(118.88, 65.71) --
	(118.88, 65.71) --
	(118.82, 65.71) --
	(118.82, 65.71) --
	(118.82, 65.71) --
	(118.75, 65.71) --
	(118.75, 65.71) --
	(118.75, 65.71) --
	(118.68, 65.71) --
	(118.68, 65.71) --
	(118.68, 65.71) --
	(118.61, 65.71) --
	(118.61, 65.71) --
	(118.61, 65.71) --
	(118.54, 65.71) --
	(118.54, 65.71) --
	(118.54, 65.71) --
	(118.47, 65.71) --
	(118.47, 65.71) --
	(118.47, 65.71) --
	(118.40, 65.71) --
	(118.40, 65.71) --
	(118.40, 65.71) --
	(118.33, 65.71) --
	(118.33, 65.71) --
	(118.33, 65.71) --
	(118.26, 65.71) --
	(118.26, 65.71) --
	(118.26, 65.71) --
	(118.19, 65.71) --
	(118.19, 65.71) --
	(118.19, 65.71) --
	(118.12, 65.71) --
	(118.12, 65.71) --
	(118.12, 65.71) --
	(118.05, 65.71) --
	(118.05, 65.71) --
	(118.05, 65.71) --
	(117.98, 65.71) --
	(117.98, 65.71) --
	(117.98, 65.71) --
	(117.91, 65.71) --
	(117.91, 65.71) --
	(117.91, 65.71) --
	(117.85, 65.71) --
	(117.85, 65.71) --
	(117.85, 65.71) --
	(117.78, 65.71) --
	(117.78, 65.71) --
	(117.78, 65.71) --
	(117.71, 65.71) --
	(117.71, 65.71) --
	(117.71, 65.71) --
	(117.64, 65.71) --
	(117.64, 65.71) --
	(117.64, 65.71) --
	(117.57, 65.71) --
	(117.57, 65.71) --
	(117.57, 65.71) --
	(117.50, 65.71) --
	(117.50, 65.71) --
	(117.50, 65.71) --
	(117.43, 65.71) --
	(117.43, 65.71) --
	(117.43, 65.71) --
	(117.36, 65.71) --
	(117.36, 65.71) --
	(117.36, 65.71) --
	(117.29, 65.71) --
	(117.29, 65.71) --
	(117.29, 65.71) --
	(117.22, 65.71) --
	(117.22, 65.71) --
	(117.22, 65.71) --
	(117.15, 65.71) --
	(117.15, 65.71) --
	(117.15, 65.71) --
	(117.08, 65.71) --
	(117.08, 65.71) --
	(117.08, 65.71) --
	(117.01, 65.71) --
	(117.01, 65.71) --
	(117.01, 65.71) --
	(116.94, 65.71) --
	(116.94, 65.71) --
	(116.94, 65.71) --
	(116.87, 65.71) --
	(116.87, 65.71) --
	(116.87, 65.71) --
	(116.81, 65.71) --
	(116.81, 65.71) --
	(116.81, 65.71) --
	(116.74, 65.71) --
	(116.74, 65.71) --
	(116.74, 65.71) --
	(116.67, 65.71) --
	(116.67, 65.71) --
	(116.67, 65.71) --
	(116.60, 65.71) --
	(116.60, 65.71) --
	(116.60, 65.71) --
	(116.53, 65.71) --
	(116.53, 65.71) --
	(116.53, 65.71) --
	(116.46, 65.71) --
	(116.46, 65.71) --
	(116.46, 65.71) --
	(116.39, 65.71) --
	(116.39, 65.71) --
	(116.39, 65.71) --
	(116.32, 65.71) --
	(116.32, 65.71) --
	(116.32, 65.71) --
	(116.25, 65.71) --
	(116.25, 65.71) --
	(116.25, 65.71) --
	(116.18, 65.71) --
	(116.18, 65.71) --
	(116.18, 65.71) --
	(116.11, 65.71) --
	(116.11, 65.71) --
	(116.11, 65.71) --
	(116.04, 65.71) --
	(116.04, 65.71) --
	(116.04, 65.71) --
	(115.97, 65.71) --
	(115.97, 65.71) --
	(115.97, 65.71) --
	(115.90, 65.71) --
	(115.90, 65.71) --
	(115.90, 65.71) --
	(115.83, 65.71) --
	(115.83, 65.71) --
	(115.83, 65.71) --
	(115.76, 65.71) --
	(115.76, 65.71) --
	(115.76, 65.71) --
	(115.69, 65.71) --
	(115.69, 65.71) --
	(115.69, 65.71) --
	(115.62, 65.71) --
	(115.62, 65.71) --
	(115.62, 65.71) --
	(115.55, 65.71) --
	(115.55, 65.71) --
	(115.55, 65.71) --
	(115.48, 65.71) --
	(115.48, 65.71) --
	(115.48, 65.71) --
	(115.41, 65.71) --
	(115.41, 65.71) --
	(115.41, 65.71) --
	(115.35, 65.71) --
	(115.35, 65.71) --
	(115.35, 65.71) --
	(115.28, 65.71) --
	(115.28, 65.71) --
	(115.28, 65.71) --
	(115.21, 65.71) --
	(115.21, 65.71) --
	(115.21, 65.71) --
	(115.14, 65.71) --
	(115.14, 65.71) --
	(115.14, 65.71) --
	(115.07, 65.71) --
	(115.07, 65.71) --
	(115.07, 65.71) --
	(115.00, 65.71) --
	(115.00, 65.71) --
	(115.00, 65.71) --
	(114.93, 65.71) --
	(114.93, 65.71) --
	(114.93, 65.71) --
	(114.86, 65.71) --
	(114.86, 65.71) --
	(114.86, 65.71) --
	(114.79, 65.71) --
	(114.79, 65.71) --
	(114.79, 65.71) --
	(114.72, 65.71) --
	(114.72, 65.71) --
	(114.72, 65.71) --
	(114.65, 65.71) --
	(114.65, 65.71) --
	(114.65, 65.71) --
	(114.58, 65.71) --
	(114.58, 65.71) --
	(114.58, 65.71) --
	(114.51, 65.71) --
	(114.51, 65.71) --
	(114.51, 65.71) --
	(114.44, 65.71) --
	(114.44, 65.71) --
	(114.44, 65.71) --
	(114.37, 65.71) --
	(114.37, 65.71) --
	(114.37, 65.71) --
	(114.30, 65.71) --
	(114.30, 65.71) --
	(114.30, 65.71) --
	(114.23, 65.71) --
	(114.23, 65.71) --
	(114.23, 65.71) --
	(114.16, 65.71) --
	(114.16, 65.71) --
	(114.16, 65.71) --
	(114.09, 65.71) --
	(114.09, 65.71) --
	(114.09, 65.71) --
	(114.02, 65.71) --
	(114.02, 65.71) --
	(114.02, 65.71) --
	(113.95, 65.71) --
	(113.95, 65.71) --
	(113.95, 65.71) --
	(113.88, 65.71) --
	(113.88, 65.71) --
	(113.88, 65.71) --
	(113.81, 65.71) --
	(113.81, 65.71) --
	(113.81, 65.71) --
	(113.74, 65.71) --
	(113.74, 65.71) --
	(113.74, 65.71) --
	(113.67, 65.71) --
	(113.67, 65.71) --
	(113.67, 65.71) --
	(113.60, 65.71) --
	(113.60, 65.71) --
	(113.60, 65.71) --
	(113.54, 65.71) --
	(113.54, 65.71) --
	(113.54, 65.71) --
	(113.47, 65.71) --
	(113.47, 65.71) --
	(113.47, 65.71) --
	(113.39, 65.71) --
	(113.39, 65.71) --
	(113.39, 65.71) --
	(113.33, 65.71) --
	(113.33, 65.71) --
	(113.33, 65.71) --
	(113.26, 65.71) --
	(113.26, 65.71) --
	(113.26, 65.71) --
	(113.19, 65.71) --
	(113.19, 65.71) --
	(113.19, 65.71) --
	(113.12, 65.71) --
	(113.12, 65.71) --
	(113.12, 65.71) --
	(113.05, 65.71) --
	(113.05, 65.71) --
	(113.05, 65.71) --
	(112.98, 65.71) --
	(112.98, 65.71) --
	(112.98, 65.71) --
	(112.91, 65.71) --
	(112.91, 65.71) --
	(112.91, 65.71) --
	(112.84, 65.71) --
	(112.84, 65.71) --
	(112.84, 65.71) --
	(112.77, 65.71) --
	(112.77, 65.71) --
	(112.77, 65.71) --
	(112.70, 65.71) --
	(112.70, 65.71) --
	(112.70, 65.71) --
	(112.63, 65.71) --
	(112.63, 65.71) --
	(112.63, 65.71) --
	(112.56, 65.71) --
	(112.56, 65.71) --
	(112.56, 65.71) --
	(112.49, 65.71) --
	(112.49, 65.71) --
	(112.49, 65.71) --
	(112.42, 65.71) --
	(112.42, 65.71) --
	(112.42, 65.71) --
	(112.35, 65.71) --
	(112.35, 65.71) --
	(112.35, 65.71) --
	(112.28, 65.71) --
	(112.28, 65.71) --
	(112.28, 65.71) --
	(112.21, 65.71) --
	(112.21, 65.71) --
	(112.21, 65.71) --
	(112.14, 65.71) --
	(112.14, 65.71) --
	(112.14, 65.71) --
	(112.07, 65.71) --
	(112.07, 65.71) --
	(112.07, 65.71) --
	(112.00, 65.71) --
	(112.00, 65.71) --
	(112.00, 65.71) --
	(111.93, 65.71) --
	(111.93, 65.71) --
	(111.93, 65.71) --
	(111.86, 65.71) --
	(111.86, 65.71) --
	(111.86, 65.71) --
	(111.79, 65.71) --
	(111.79, 65.71) --
	(111.79, 65.71) --
	(111.72, 65.71) --
	(111.72, 65.71) --
	(111.72, 65.71) --
	(111.72, 65.71) --
	(111.72, 65.71) --
	(111.72, 65.71) --
	(111.65, 65.71) --
	(111.65, 65.71) --
	(111.65, 65.71) --
	(111.58, 65.71) --
	(111.58, 65.71) --
	(111.58, 65.71) --
	(111.51, 65.71) --
	(111.51, 65.71) --
	(111.51, 65.71) --
	(111.44, 65.71) --
	(111.44, 65.71) --
	(111.44, 65.71) --
	(111.37, 65.71) --
	(111.37, 65.71) --
	(111.37, 65.71) --
	(111.30, 65.71) --
	(111.30, 65.71) --
	(111.30, 65.71) --
	(111.23, 65.71) --
	(111.23, 65.71) --
	(111.23, 65.71) --
	(111.16, 65.71) --
	(111.16, 65.71) --
	(111.16, 65.71) --
	(111.09, 65.71) --
	(111.09, 65.71) --
	(111.09, 65.71) --
	(111.02, 65.71) --
	(111.02, 65.71) --
	(111.02, 65.71) --
	(110.95, 65.71) --
	(110.95, 65.71) --
	(110.95, 65.71) --
	(110.88, 65.71) --
	(110.88, 65.71) --
	(110.88, 65.71) --
	(110.81, 65.71) --
	(110.81, 65.71) --
	(110.81, 65.71) --
	(110.74, 65.71) --
	(110.74, 65.71) --
	(110.74, 65.71) --
	(110.67, 65.71) --
	(110.67, 65.71) --
	(110.67, 65.71) --
	(110.60, 65.71) --
	(110.60, 65.71) --
	(110.60, 65.71) --
	(110.53, 65.71) --
	(110.53, 65.71) --
	(110.53, 65.71) --
	(110.46, 65.71) --
	(110.46, 65.71) --
	(110.46, 65.71) --
	(110.39, 65.71) --
	(110.39, 65.71) --
	(110.39, 65.71) --
	(110.32, 65.71) --
	(110.32, 65.71) --
	(110.32, 65.71) --
	(110.25, 65.71) --
	(110.25, 65.71) --
	(110.25, 65.71) --
	(110.18, 65.71) --
	(110.18, 65.71) --
	(110.18, 65.71) --
	(110.11, 65.71) --
	(110.11, 65.71) --
	(110.11, 65.71) --
	(110.04, 65.71) --
	(110.04, 65.71) --
	(110.04, 65.71) --
	(109.97, 65.71) --
	(109.97, 65.71) --
	(109.97, 65.71) --
	(109.90, 65.71) --
	(109.90, 65.71) --
	(109.90, 65.71) --
	(109.90, 65.71) --
	(109.90, 65.71) --
	(109.90, 65.71) --
	(109.83, 65.71) --
	(109.83, 65.71) --
	(109.83, 65.71) --
	(109.76, 65.71) --
	(109.76, 65.71) --
	(109.76, 65.71) --
	(109.69, 65.71) --
	(109.69, 65.71) --
	(109.69, 65.71) --
	(109.62, 65.71) --
	(109.62, 65.71) --
	(109.62, 65.71) --
	(109.55, 65.71) --
	(109.55, 65.71) --
	(109.55, 65.71) --
	(109.48, 65.71) --
	(109.48, 65.71) --
	(109.48, 65.71) --
	(109.41, 65.71) --
	(109.41, 65.71) --
	(109.41, 65.71) --
	(109.34, 65.71) --
	(109.34, 65.71) --
	(109.34, 65.71) --
	(109.27, 65.71) --
	(109.27, 65.71) --
	(109.27, 65.71) --
	(109.20, 65.71) --
	(109.20, 65.71) --
	(109.20, 65.71) --
	(109.13, 65.71) --
	(109.13, 65.71) --
	(109.13, 65.71) --
	(109.06, 65.71) --
	(109.06, 65.71) --
	(109.06, 65.71) --
	(108.99, 65.71) --
	(108.99, 65.71) --
	(108.99, 65.71) --
	(108.92, 65.71) --
	(108.92, 65.71) --
	(108.92, 65.71) --
	(108.85, 65.71) --
	(108.85, 65.71) --
	(108.85, 65.71) --
	(108.78, 65.71) --
	(108.78, 65.71) --
	(108.78, 65.71) --
	(108.71, 65.71) --
	(108.71, 65.71) --
	(108.71, 65.71) --
	(108.64, 65.71) --
	(108.64, 65.71) --
	(108.64, 65.71) --
	(108.57, 65.71) --
	(108.57, 65.71) --
	(108.57, 65.71) --
	(108.50, 65.71) --
	(108.50, 65.71) --
	(108.50, 65.71) --
	(108.43, 65.71) --
	(108.43, 65.71) --
	(108.43, 65.71) --
	(108.36, 65.71) --
	(108.36, 65.71) --
	(108.36, 65.71) --
	(108.29, 65.71) --
	(108.29, 65.71) --
	(108.29, 65.71) --
	(108.22, 65.71) --
	(108.22, 65.71) --
	(108.22, 65.71) --
	(108.15, 65.71) --
	(108.15, 65.71) --
	(108.15, 65.71) --
	(108.08, 65.71) --
	(108.08, 65.71) --
	(108.08, 65.71) --
	(108.01, 65.71) --
	(108.01, 65.71) --
	(108.01, 65.71) --
	(107.94, 65.71) --
	(107.94, 65.71) --
	(107.94, 65.71) --
	(107.87, 65.71) --
	(107.87, 65.71) --
	(107.87, 65.71) --
	(107.80, 65.71) --
	(107.80, 65.71) --
	(107.80, 65.71) --
	(107.73, 65.71) --
	(107.73, 65.71) --
	(107.73, 65.71) --
	(107.66, 65.71) --
	(107.66, 65.71) --
	(107.66, 65.71) --
	(107.59, 65.71) --
	(107.59, 65.71) --
	(107.59, 65.71) --
	(107.52, 65.71) --
	(107.52, 65.71) --
	(107.52, 65.71) --
	(107.45, 65.71) --
	(107.45, 65.71) --
	(107.45, 65.71) --
	(107.41, 65.71) --
	(107.41, 65.71) --
	(107.41, 65.71) --
	(107.38, 65.71) --
	(107.38, 65.71) --
	(107.38, 65.71) --
	(107.31, 65.71) --
	(107.31, 65.71) --
	(107.31, 65.71) --
	(107.24, 65.71) --
	(107.24, 65.71) --
	(107.24, 65.71) --
	(107.17, 65.71) --
	(107.17, 65.71) --
	(107.17, 65.71) --
	(107.10, 65.71) --
	(107.10, 65.71) --
	(107.10, 65.71) --
	(107.03, 65.71) --
	(107.03, 65.71) --
	(107.03, 65.71) --
	(106.96, 65.71) --
	(106.96, 65.71) --
	(106.96, 65.71) --
	(106.89, 65.71) --
	(106.89, 65.71) --
	(106.89, 65.71) --
	(106.82, 65.71) --
	(106.82, 65.71) --
	(106.82, 65.71) --
	(106.75, 65.71) --
	(106.75, 65.71) --
	(106.75, 65.71) --
	(106.68, 65.71) --
	(106.68, 65.71) --
	(106.68, 65.71) --
	(106.61, 65.71) --
	(106.61, 65.71) --
	(106.61, 65.71) --
	(106.54, 65.71) --
	(106.54, 65.71) --
	(106.54, 65.71) --
	(106.47, 65.71) --
	(106.47, 65.71) --
	(106.47, 65.71) --
	(106.40, 65.71) --
	(106.40, 65.71) --
	(106.40, 65.71) --
	(106.33, 65.71) --
	(106.33, 65.71) --
	(106.33, 65.71) --
	(106.26, 65.71) --
	(106.26, 65.71) --
	(106.26, 65.71) --
	(106.19, 65.71) --
	(106.19, 65.71) --
	(106.19, 65.71) --
	(106.12, 65.71) --
	(106.12, 65.71) --
	(106.12, 65.71) --
	(106.05, 65.71) --
	(106.05, 65.71) --
	(106.05, 65.71) --
	(105.98, 65.71) --
	(105.98, 65.71) --
	(105.98, 65.71) --
	(105.91, 65.71) --
	(105.91, 65.71) --
	(105.91, 65.71) --
	(105.84, 65.71) --
	(105.84, 65.71) --
	(105.84, 65.71) --
	(105.77, 65.71) --
	(105.77, 65.71) --
	(105.77, 65.71) --
	(105.70, 65.71) --
	(105.70, 65.71) --
	(105.70, 65.71) --
	(105.63, 65.71) --
	(105.63, 65.71) --
	(105.63, 65.71) --
	(105.56, 65.71) --
	(105.56, 65.71) --
	(105.56, 65.71) --
	(105.49, 65.71) --
	(105.49, 65.71) --
	(105.49, 65.71) --
	(105.42, 65.71) --
	(105.42, 65.71) --
	(105.42, 65.71) --
	(105.34, 65.71) --
	(105.34, 65.71) --
	(105.34, 65.71) --
	(105.28, 65.71) --
	(105.28, 65.71) --
	(105.28, 65.71) --
	(105.20, 65.71) --
	(105.20, 65.71) --
	(105.20, 65.71) --
	(105.14, 65.71) --
	(105.14, 65.71) --
	(105.14, 65.71) --
	(105.06, 65.71) --
	(105.06, 65.71) --
	(105.06, 65.71) --
	(104.99, 65.71) --
	(104.99, 65.71) --
	(104.99, 65.71) --
	(104.92, 65.71) --
	(104.92, 65.71) --
	(104.92, 65.71) --
	(104.85, 65.71) --
	(104.85, 65.71) --
	(104.85, 65.71) --
	(104.78, 65.71) --
	(104.78, 65.71) --
	(104.78, 65.71) --
	(104.71, 65.71) --
	(104.71, 65.71) --
	(104.71, 65.71) --
	(104.64, 65.71) --
	(104.64, 65.71) --
	(104.64, 65.71) --
	(104.57, 65.71) --
	(104.57, 65.71) --
	(104.57, 65.71) --
	(104.50, 65.71) --
	(104.50, 65.71) --
	(104.50, 65.71) --
	(104.44, 65.71) --
	(104.44, 65.71) --
	(104.44, 65.71) --
	(104.43, 65.71) --
	(104.43, 65.71) --
	(104.43, 65.71) --
	(104.36, 65.71) --
	(104.36, 65.71) --
	(104.36, 65.71) --
	(104.29, 65.71) --
	(104.29, 65.71) --
	(104.29, 65.71) --
	(104.22, 65.71) --
	(104.22, 65.71) --
	(104.22, 65.71) --
	(104.15, 65.71) --
	(104.15, 65.71) --
	(104.15, 65.71) --
	(104.08, 65.71) --
	(104.08, 65.71) --
	(104.08, 65.71) --
	(104.01, 65.71) --
	(104.01, 65.71) --
	(104.01, 65.71) --
	(103.94, 65.71) --
	(103.94, 65.71) --
	(103.94, 65.71) --
	(103.87, 65.71) --
	(103.87, 65.71) --
	(103.87, 65.71) --
	(103.80, 65.71) --
	(103.80, 65.71) --
	(103.80, 65.71) --
	(103.73, 65.71) --
	(103.73, 65.71) --
	(103.73, 65.71) --
	(103.66, 65.71) --
	(103.66, 65.71) --
	(103.66, 65.71) --
	(103.59, 65.71) --
	(103.59, 65.71) --
	(103.59, 65.71) --
	(103.52, 65.71) --
	(103.52, 65.71) --
	(103.52, 65.71) --
	(103.45, 65.71) --
	(103.45, 65.71) --
	(103.45, 65.71) --
	(103.37, 65.71) --
	(103.37, 65.71) --
	(103.37, 65.71) --
	(103.31, 65.71) --
	(103.31, 65.71) --
	(103.31, 65.71) --
	(103.23, 65.71) --
	(103.23, 65.71) --
	(103.23, 65.71) --
	(103.16, 65.71) --
	(103.16, 65.71) --
	(103.16, 65.71) --
	(103.09, 65.71) --
	(103.09, 65.71) --
	(103.09, 65.71) --
	(103.02, 65.71) --
	(103.02, 65.71) --
	(103.02, 65.71) --
	(102.95, 65.71) --
	(102.95, 65.71) --
	(102.95, 65.71) --
	(102.88, 65.71) --
	(102.88, 65.71) --
	(102.88, 65.71) --
	(102.81, 65.71) --
	(102.81, 65.71) --
	(102.81, 65.71) --
	(102.74, 65.71) --
	(102.74, 65.71) --
	(102.74, 65.71) --
	(102.67, 65.71) --
	(102.67, 65.71) --
	(102.67, 65.71) --
	(102.60, 65.71) --
	(102.60, 65.71) --
	(102.60, 65.71) --
	(102.53, 65.71) --
	(102.53, 65.71) --
	(102.53, 65.71) --
	(102.46, 65.71) --
	(102.46, 65.71) --
	(102.46, 65.71) --
	(102.39, 65.71) --
	(102.39, 65.71) --
	(102.39, 65.71) --
	(102.32, 65.71) --
	(102.32, 65.71) --
	(102.32, 65.71) --
	(102.25, 65.71) --
	(102.25, 65.71) --
	(102.25, 65.71) --
	(102.18, 65.71) --
	(102.18, 65.71) --
	(102.18, 65.71) --
	(102.11, 65.71) --
	(102.11, 65.71) --
	(102.11, 65.71) --
	(102.04, 65.71) --
	(102.04, 65.71) --
	(102.04, 65.71) --
	(101.97, 65.71) --
	(101.97, 65.71) --
	(101.97, 65.71) --
	(101.90, 65.71) --
	(101.90, 65.71) --
	(101.90, 65.71) --
	(101.82, 65.71) --
	(101.82, 65.71) --
	(101.82, 65.71) --
	(101.75, 65.71) --
	(101.75, 65.71) --
	(101.75, 65.71) --
	(101.68, 65.71) --
	(101.68, 65.71) --
	(101.68, 65.71) --
	(101.66, 65.71) --
	(101.66, 65.71) --
	(101.66, 65.71) --
	(101.61, 65.71) --
	(101.61, 65.71) --
	(101.61, 65.71) --
	(101.54, 65.71) --
	(101.54, 65.71) --
	(101.54, 65.71) --
	(101.47, 65.71) --
	(101.47, 65.71) --
	(101.47, 65.71) --
	(101.40, 65.71) --
	(101.40, 65.71) --
	(101.40, 65.71) --
	(101.33, 65.71) --
	(101.33, 65.71) --
	(101.33, 65.71) --
	(101.26, 65.71) --
	(101.26, 65.71) --
	(101.26, 65.71) --
	(101.19, 65.71) --
	(101.19, 65.71) --
	(101.19, 65.71) --
	(101.12, 65.71) --
	(101.12, 65.71) --
	(101.12, 65.71) --
	(101.05, 65.71) --
	(101.05, 65.71) --
	(101.05, 65.71) --
	(100.98, 65.71) --
	(100.98, 65.71) --
	(100.98, 65.71) --
	(100.91, 65.71) --
	(100.91, 65.71) --
	(100.91, 65.71) --
	(100.84, 65.71) --
	(100.84, 65.71) --
	(100.84, 65.71) --
	(100.77, 65.71) --
	(100.77, 65.71) --
	(100.77, 65.71) --
	(100.70, 65.71) --
	(100.70, 65.71) --
	(100.70, 65.71) --
	(100.62, 65.71) --
	(100.62, 65.71) --
	(100.62, 65.71) --
	(100.55, 65.71) --
	(100.55, 65.71) --
	(100.55, 65.71) --
	(100.48, 65.71) --
	(100.48, 65.71) --
	(100.48, 65.71) --
	(100.41, 65.71) --
	(100.41, 65.71) --
	(100.41, 65.71) --
	(100.34, 65.71) --
	(100.34, 65.71) --
	(100.34, 65.71) --
	(100.27, 65.71) --
	(100.27, 65.71) --
	(100.27, 65.71) --
	(100.20, 65.71) --
	(100.20, 65.71) --
	(100.20, 65.71) --
	(100.13, 65.71) --
	(100.13, 65.71) --
	(100.13, 65.71) --
	(100.06, 65.71) --
	(100.06, 65.71) --
	(100.06, 65.71) --
	( 99.99, 65.71) --
	( 99.99, 65.71) --
	( 99.99, 65.71) --
	( 99.92, 65.71) --
	( 99.92, 65.71) --
	( 99.92, 65.71) --
	( 99.85, 65.71) --
	( 99.85, 65.71) --
	( 99.85, 65.71) --
	( 99.78, 65.71) --
	( 99.78, 65.71) --
	( 99.78, 65.71) --
	( 99.71, 65.71) --
	( 99.71, 65.71) --
	( 99.71, 65.71) --
	( 99.64, 65.71) --
	( 99.64, 65.71) --
	( 99.64, 65.71) --
	( 99.56, 65.71) --
	( 99.56, 65.71) --
	( 99.56, 65.71) --
	( 99.49, 65.71) --
	( 99.49, 65.71) --
	( 99.49, 65.71) --
	( 99.42, 65.71) --
	( 99.42, 65.71) --
	( 99.42, 65.71) --
	( 99.35, 65.71) --
	( 99.35, 65.71) --
	( 99.35, 65.71) --
	( 99.28, 65.71) --
	( 99.28, 65.71) --
	( 99.28, 65.71) --
	( 99.21, 65.71) --
	( 99.21, 65.71) --
	( 99.21, 65.71) --
	( 99.14, 65.71) --
	( 99.14, 65.71) --
	( 99.14, 65.71) --
	( 99.07, 65.71) --
	( 99.07, 65.71) --
	( 99.07, 65.71) --
	( 99.07, 65.71) --
	( 99.07, 65.71) --
	( 99.07, 65.71) --
	( 99.00, 65.71) --
	( 99.00, 65.71) --
	( 99.00, 65.71) --
	( 98.93, 65.71) --
	( 98.93, 65.71) --
	( 98.93, 65.71) --
	( 98.86, 65.71) --
	( 98.86, 65.71) --
	( 98.86, 65.71) --
	( 98.79, 65.71) --
	( 98.79, 65.71) --
	( 98.79, 65.71) --
	( 98.72, 65.71) --
	( 98.72, 65.71) --
	( 98.72, 65.71) --
	( 98.65, 65.71) --
	( 98.65, 65.71) --
	( 98.65, 65.71) --
	( 98.57, 65.71) --
	( 98.57, 65.71) --
	( 98.57, 65.71) --
	( 98.50, 65.71) --
	( 98.50, 65.71) --
	( 98.50, 65.71) --
	( 98.43, 65.71) --
	( 98.43, 65.71) --
	( 98.43, 65.71) --
	( 98.36, 65.71) --
	( 98.36, 65.71) --
	( 98.36, 65.71) --
	( 98.29, 65.71) --
	( 98.29, 65.71) --
	( 98.29, 65.71) --
	( 98.22, 65.71) --
	( 98.22, 65.71) --
	( 98.22, 65.71) --
	( 98.15, 65.71) --
	( 98.15, 65.71) --
	( 98.15, 65.71) --
	( 98.08, 65.71) --
	( 98.08, 65.71) --
	( 98.08, 65.71) --
	( 98.01, 65.71) --
	( 98.01, 65.71) --
	( 98.01, 65.71) --
	( 97.94, 65.71) --
	( 97.94, 65.71) --
	( 97.94, 65.71) --
	( 97.87, 65.71) --
	( 97.87, 65.71) --
	( 97.87, 65.71) --
	( 97.80, 65.71) --
	( 97.80, 65.71) --
	( 97.80, 65.71) --
	( 97.72, 65.71) --
	( 97.72, 65.71) --
	( 97.72, 65.71) --
	( 97.65, 65.71) --
	( 97.65, 65.71) --
	( 97.65, 65.71) --
	( 97.58, 65.71) --
	( 97.58, 65.71) --
	( 97.58, 65.71) --
	( 97.51, 65.71) --
	( 97.51, 65.71) --
	( 97.51, 65.71) --
	( 97.44, 65.71) --
	( 97.44, 65.71) --
	( 97.44, 65.71) --
	( 97.37, 65.71) --
	( 97.37, 65.71) --
	( 97.37, 65.71) --
	( 97.30, 65.71) --
	( 97.30, 65.71) --
	( 97.30, 65.71) --
	( 97.23, 65.71) --
	( 97.23, 65.71) --
	( 97.23, 65.71) --
	( 97.16, 65.71) --
	( 97.16, 65.71) --
	( 97.16, 65.71) --
	( 97.09, 65.71) --
	( 97.09, 65.71) --
	( 97.09, 65.71) --
	( 97.02, 65.71) --
	( 97.02, 65.71) --
	( 97.02, 65.71) --
	( 96.94, 65.71) --
	( 96.94, 65.71) --
	( 96.94, 65.71) --
	( 96.87, 65.71) --
	( 96.87, 65.71) --
	( 96.87, 65.71) --
	( 96.80, 65.71) --
	( 96.80, 65.71) --
	( 96.80, 65.71) --
	( 96.77, 65.71) --
	( 96.77, 65.71) --
	( 96.77, 65.71) --
	( 96.73, 65.71) --
	( 96.73, 65.71) --
	( 96.73, 65.71) --
	( 96.66, 65.71) --
	( 96.66, 65.71) --
	( 96.66, 65.71) --
	( 96.59, 65.71) --
	( 96.59, 65.71) --
	( 96.59, 65.71) --
	( 96.52, 65.71) --
	( 96.52, 65.71) --
	( 96.52, 65.71) --
	( 96.45, 65.71) --
	( 96.45, 65.71) --
	( 96.45, 65.71) --
	( 96.38, 65.71) --
	( 96.38, 65.71) --
	( 96.38, 65.71) --
	( 96.31, 65.71) --
	( 96.31, 65.71) --
	( 96.31, 65.71) --
	( 96.24, 65.71) --
	( 96.24, 65.71) --
	( 96.24, 65.71) --
	( 96.16, 65.71) --
	( 96.16, 65.71) --
	( 96.16, 65.71) --
	( 96.09, 65.71) --
	( 96.09, 65.71) --
	( 96.09, 65.71) --
	( 96.02, 65.71) --
	( 96.02, 65.71) --
	( 96.02, 65.71) --
	( 95.95, 65.71) --
	( 95.95, 65.71) --
	( 95.95, 65.71) --
	( 95.88, 65.71) --
	( 95.88, 65.71) --
	( 95.88, 65.71) --
	( 95.81, 65.71) --
	( 95.81, 65.71) --
	( 95.81, 65.71) --
	( 95.74, 65.71) --
	( 95.74, 65.71) --
	( 95.74, 65.71) --
	( 95.67, 65.71) --
	( 95.67, 65.71) --
	( 95.67, 65.71) --
	( 95.60, 65.71) --
	( 95.60, 65.71) --
	( 95.60, 65.71) --
	( 95.53, 65.71) --
	( 95.53, 65.71) --
	( 95.53, 65.71) --
	( 95.45, 65.71) --
	( 95.45, 65.71) --
	( 95.45, 65.71) --
	( 95.38, 65.71) --
	( 95.38, 65.71) --
	( 95.38, 65.71) --
	( 95.31, 65.71) --
	( 95.31, 65.71) --
	( 95.31, 65.71) --
	( 95.24, 65.71) --
	( 95.24, 65.71) --
	( 95.24, 65.71) --
	( 95.17, 65.71) --
	( 95.17, 65.71) --
	( 95.17, 65.71) --
	( 95.10, 65.71) --
	( 95.10, 65.71) --
	( 95.10, 65.71) --
	( 95.03, 65.71) --
	( 95.03, 65.71) --
	( 95.03, 65.71) --
	( 94.96, 65.71) --
	( 94.96, 65.71) --
	( 94.96, 65.71) --
	( 94.95, 65.71) --
	( 94.95, 65.71) --
	( 94.95, 65.71) --
	( 94.89, 65.71) --
	( 94.89, 65.71) --
	( 94.89, 65.71) --
	( 94.82, 65.71) --
	( 94.82, 65.71) --
	( 94.82, 65.71) --
	( 94.75, 65.71) --
	( 94.75, 65.71) --
	( 94.75, 65.71) --
	( 94.67, 65.71) --
	( 94.67, 65.71) --
	( 94.67, 65.71) --
	( 94.60, 65.71) --
	( 94.60, 65.71) --
	( 94.60, 65.71) --
	( 94.53, 65.71) --
	( 94.53, 65.71) --
	( 94.53, 65.71) --
	( 94.46, 65.71) --
	( 94.46, 65.71) --
	( 94.46, 65.71) --
	( 94.39, 65.71) --
	( 94.39, 65.71) --
	( 94.39, 65.71) --
	( 94.32, 65.71) --
	( 94.32, 65.71) --
	( 94.32, 65.71) --
	( 94.25, 65.71) --
	( 94.25, 65.71) --
	( 94.25, 65.71) --
	( 94.18, 65.71) --
	( 94.18, 65.71) --
	( 94.18, 65.71) --
	( 94.10, 65.71) --
	( 94.10, 65.71) --
	( 94.10, 65.71) --
	( 94.03, 65.71) --
	( 94.03, 65.71) --
	( 94.03, 65.71) --
	( 93.96, 65.71) --
	( 93.96, 65.71) --
	( 93.96, 65.71) --
	( 93.89, 65.71) --
	( 93.89, 65.71) --
	( 93.89, 65.71) --
	( 93.82, 65.71) --
	( 93.82, 65.71) --
	( 93.82, 65.71) --
	( 93.75, 65.71) --
	( 93.75, 65.71) --
	( 93.75, 65.71) --
	( 93.68, 65.71) --
	( 93.68, 65.71) --
	( 93.68, 65.71) --
	( 93.61, 65.71) --
	( 93.61, 65.71) --
	( 93.61, 65.71) --
	( 93.54, 65.71) --
	( 93.54, 65.71) --
	( 93.54, 65.71) --
	( 93.46, 65.71) --
	( 93.46, 65.71) --
	( 93.46, 65.71) --
	( 93.39, 65.71) --
	( 93.39, 65.71) --
	( 93.39, 65.71) --
	( 93.32, 65.71) --
	( 93.32, 65.71) --
	( 93.32, 65.71) --
	( 93.25, 65.71) --
	( 93.25, 65.71) --
	( 93.25, 65.71) --
	( 93.18, 65.71) --
	( 93.18, 65.71) --
	( 93.18, 65.71) --
	( 93.11, 65.71) --
	( 93.11, 65.71) --
	( 93.11, 65.71) --
	( 93.04, 65.71) --
	( 93.04, 65.71) --
	( 93.04, 65.71) --
	( 92.97, 65.71) --
	( 92.97, 65.71) --
	( 92.97, 65.71) --
	( 92.90, 65.71) --
	( 92.90, 65.71) --
	( 92.90, 65.71) --
	( 92.83, 65.71) --
	( 92.83, 65.71) --
	( 92.83, 65.71) --
	( 92.75, 65.71) --
	( 92.75, 65.71) --
	( 92.75, 65.71) --
	( 92.68, 65.71) --
	( 92.68, 65.71) --
	( 92.68, 65.71) --
	( 92.65, 65.71) --
	( 92.65, 65.71) --
	( 92.65, 65.71) --
	( 92.61, 65.71) --
	( 92.61, 65.71) --
	( 92.61, 65.71) --
	( 92.54, 65.71) --
	( 92.54, 65.71) --
	( 92.54, 65.71) --
	( 92.47, 65.71) --
	( 92.47, 65.71) --
	( 92.47, 65.71) --
	( 92.40, 65.71) --
	( 92.40, 65.71) --
	( 92.40, 65.71) --
	( 92.33, 65.71) --
	( 92.33, 65.71) --
	( 92.33, 65.71) --
	( 92.26, 65.71) --
	( 92.26, 65.71) --
	( 92.26, 65.71) --
	( 92.18, 65.71) --
	( 92.18, 65.71) --
	( 92.18, 65.71) --
	( 92.11, 65.71) --
	( 92.11, 65.71) --
	( 92.11, 65.71) --
	( 92.04, 65.71) --
	( 92.04, 65.71) --
	( 92.04, 65.71) --
	( 91.97, 65.71) --
	( 91.97, 65.71) --
	( 91.97, 65.71) --
	( 91.90, 65.71) --
	( 91.90, 65.71) --
	( 91.90, 65.71) --
	( 91.83, 65.71) --
	( 91.83, 65.71) --
	( 91.83, 65.71) --
	( 91.76, 65.71) --
	( 91.76, 65.71) --
	( 91.76, 65.71) --
	( 91.69, 65.71) --
	( 91.69, 65.71) --
	( 91.69, 65.71) --
	( 91.61, 65.71) --
	( 91.61, 65.71) --
	( 91.61, 65.71) --
	( 91.54, 65.71) --
	( 91.54, 65.71) --
	( 91.54, 65.71) --
	( 91.47, 65.71) --
	( 91.47, 65.71) --
	( 91.47, 65.71) --
	( 91.40, 65.71) --
	( 91.40, 65.71) --
	( 91.40, 65.71) --
	( 91.33, 65.71) --
	( 91.33, 65.71) --
	( 91.33, 65.71) --
	( 91.26, 65.71) --
	( 91.26, 65.71) --
	( 91.26, 65.71) --
	( 91.19, 65.71) --
	( 91.19, 65.71) --
	( 91.19, 65.71) --
	( 91.12, 65.71) --
	( 91.12, 65.71) --
	( 91.12, 65.71) --
	( 91.04, 65.71) --
	( 91.04, 65.71) --
	( 91.04, 65.71) --
	( 90.97, 65.71) --
	( 90.97, 65.71) --
	( 90.97, 65.71) --
	( 90.90, 65.71) --
	( 90.90, 65.71) --
	( 90.90, 65.71) --
	( 90.83, 65.71) --
	( 90.83, 65.71) --
	( 90.83, 65.71) --
	( 90.76, 65.71) --
	( 90.76, 65.71) --
	( 90.76, 65.71) --
	( 90.69, 65.71) --
	( 90.69, 65.71) --
	( 90.69, 65.71) --
	( 90.64, 65.71) --
	( 90.64, 65.71) --
	( 90.64, 65.71) --
	( 90.62, 65.71) --
	( 90.62, 65.71) --
	( 90.62, 65.71) --
	( 90.54, 65.71) --
	( 90.54, 65.71) --
	( 90.54, 65.71) --
	( 90.47, 65.71) --
	( 90.47, 65.71) --
	( 90.47, 65.71) --
	( 90.40, 65.71) --
	( 90.40, 65.71) --
	( 90.40, 65.71) --
	( 90.33, 65.71) --
	( 90.33, 65.71) --
	( 90.33, 65.71) --
	( 90.26, 65.71) --
	( 90.26, 65.71) --
	( 90.26, 65.71) --
	( 90.19, 65.71) --
	( 90.19, 65.71) --
	( 90.19, 65.71) --
	( 90.12, 65.71) --
	( 90.12, 65.71) --
	( 90.12, 65.71) --
	( 90.04, 65.71) --
	( 90.04, 65.71) --
	( 90.04, 65.71) --
	( 89.97, 65.71) --
	( 89.97, 65.71) --
	( 89.97, 65.71) --
	( 89.90, 65.71) --
	( 89.90, 65.71) --
	( 89.90, 65.71) --
	( 89.83, 65.71) --
	( 89.83, 65.71) --
	( 89.83, 65.71) --
	( 89.76, 65.71) --
	( 89.76, 65.71) --
	( 89.76, 65.71) --
	( 89.69, 65.71) --
	( 89.69, 65.71) --
	( 89.69, 65.71) --
	( 89.62, 65.71) --
	( 89.62, 65.71) --
	( 89.62, 65.71) --
	( 89.55, 65.71) --
	( 89.55, 65.71) --
	( 89.55, 65.71) --
	( 89.47, 65.71) --
	( 89.47, 65.71) --
	( 89.47, 65.71) --
	( 89.40, 65.71) --
	( 89.40, 65.71) --
	( 89.40, 65.71) --
	( 89.33, 65.71) --
	( 89.33, 65.71) --
	( 89.33, 65.71) --
	( 89.26, 65.71) --
	( 89.26, 65.71) --
	( 89.26, 65.71) --
	( 89.19, 65.71) --
	( 89.19, 65.71) --
	( 89.19, 65.71) --
	( 89.12, 65.71) --
	( 89.12, 65.71) --
	( 89.12, 65.71) --
	( 89.05, 65.71) --
	( 89.05, 65.71) --
	( 89.05, 65.71) --
	( 88.97, 65.71) --
	( 88.97, 65.71) --
	( 88.97, 65.71) --
	( 88.92, 65.71) --
	( 88.92, 65.71) --
	( 88.92, 65.71) --
	( 88.90, 65.71) --
	( 88.90, 65.71) --
	( 88.90, 65.71) --
	( 88.83, 65.71) --
	( 88.83, 65.71) --
	( 88.83, 65.71) --
	( 88.76, 65.71) --
	( 88.76, 65.71) --
	( 88.76, 65.71) --
	( 88.69, 65.71) --
	( 88.69, 65.71) --
	( 88.69, 65.71) --
	( 88.62, 65.71) --
	( 88.62, 65.71) --
	( 88.62, 65.71) --
	( 88.54, 65.71) --
	( 88.54, 65.71) --
	( 88.54, 65.71) --
	( 88.47, 65.71) --
	( 88.47, 65.71) --
	( 88.47, 65.71) --
	( 88.40, 65.71) --
	( 88.40, 65.71) --
	( 88.40, 65.71) --
	( 88.33, 65.71) --
	( 88.33, 65.71) --
	( 88.33, 65.71) --
	( 88.26, 65.71) --
	( 88.26, 65.71) --
	( 88.26, 65.71) --
	( 88.19, 65.71) --
	( 88.19, 65.71) --
	( 88.19, 65.71) --
	( 88.12, 65.71) --
	( 88.12, 65.71) --
	( 88.12, 65.71) --
	( 88.04, 65.71) --
	( 88.04, 65.71) --
	( 88.04, 65.71) --
	( 87.97, 65.71) --
	( 87.97, 65.71) --
	( 87.97, 65.71) --
	( 87.90, 65.71) --
	( 87.90, 65.71) --
	( 87.90, 65.71) --
	( 87.83, 65.71) --
	( 87.83, 65.71) --
	( 87.83, 65.71) --
	( 87.76, 65.71) --
	( 87.76, 65.71) --
	( 87.76, 65.71) --
	( 87.69, 65.71) --
	( 87.69, 65.71) --
	( 87.69, 65.71) --
	( 87.62, 65.71) --
	( 87.62, 65.71) --
	( 87.62, 65.71) --
	( 87.54, 65.71) --
	( 87.54, 65.71) --
	( 87.54, 65.71) --
	( 87.47, 65.71) --
	( 87.47, 65.71) --
	( 87.47, 65.71) --
	( 87.40, 65.71) --
	( 87.40, 65.71) --
	( 87.40, 65.71) --
	( 87.33, 65.71) --
	( 87.33, 65.71) --
	( 87.33, 65.71) --
	( 87.26, 65.71) --
	( 87.26, 65.71) --
	( 87.26, 65.71) --
	( 87.19, 65.71) --
	( 87.19, 65.71) --
	( 87.19, 65.71) --
	( 87.11, 65.71) --
	( 87.11, 65.71) --
	( 87.11, 65.71) --
	( 87.04, 65.71) --
	( 87.04, 65.71) --
	( 87.04, 65.71) --
	( 86.97, 65.71) --
	( 86.97, 65.71) --
	( 86.97, 65.71) --
	( 86.90, 65.71) --
	( 86.90, 65.71) --
	( 86.90, 65.71) --
	( 86.90, 65.71) --
	( 86.90, 65.71) --
	( 86.90, 65.71) --
	( 86.83, 65.71) --
	( 86.83, 65.71) --
	( 86.83, 65.71) --
	( 86.76, 65.71) --
	( 86.76, 65.71) --
	( 86.76, 65.71) --
	( 86.68, 65.71) --
	( 86.68, 65.71) --
	( 86.68, 65.71) --
	( 86.61, 65.71) --
	( 86.61, 65.71) --
	( 86.61, 65.71) --
	( 86.54, 65.71) --
	( 86.54, 65.71) --
	( 86.54, 65.71) --
	( 86.47, 65.71) --
	( 86.47, 65.71) --
	( 86.47, 65.71) --
	( 86.40, 65.71) --
	( 86.40, 65.71) --
	( 86.40, 65.71) --
	( 86.33, 65.71) --
	( 86.33, 65.71) --
	( 86.33, 65.71) --
	( 86.25, 65.71) --
	( 86.25, 65.71) --
	( 86.25, 65.71) --
	( 86.18, 65.71) --
	( 86.18, 65.71) --
	( 86.18, 65.71) --
	( 86.11, 65.71) --
	( 86.11, 65.71) --
	( 86.11, 65.71) --
	( 86.04, 65.71) --
	( 86.04, 65.71) --
	( 86.04, 65.71) --
	( 85.97, 65.71) --
	( 85.97, 65.71) --
	( 85.97, 65.71) --
	( 85.90, 65.71) --
	( 85.90, 65.71) --
	( 85.90, 65.71) --
	( 85.83, 65.71) --
	( 85.83, 65.71) --
	( 85.83, 65.71) --
	( 85.75, 65.71) --
	( 85.75, 65.71) --
	( 85.75, 65.71) --
	( 85.68, 65.71) --
	( 85.68, 65.71) --
	( 85.68, 65.71) --
	( 85.61, 65.71) --
	( 85.61, 65.71) --
	( 85.61, 65.71) --
	( 85.54, 65.71) --
	( 85.54, 65.71) --
	( 85.54, 65.71) --
	( 85.47, 65.71) --
	( 85.47, 65.71) --
	( 85.47, 65.71) --
	( 85.47, 65.71) --
	( 85.47, 65.71) --
	( 85.47, 65.71) --
	( 85.40, 65.71) --
	( 85.40, 65.71) --
	( 85.40, 65.71) --
	( 85.32, 65.71) --
	( 85.32, 65.71) --
	( 85.32, 65.71) --
	( 85.25, 65.71) --
	( 85.25, 65.71) --
	( 85.25, 65.71) --
	( 85.18, 65.71) --
	( 85.18, 65.71) --
	( 85.18, 65.71) --
	( 85.11, 65.71) --
	( 85.11, 65.71) --
	( 85.11, 65.71) --
	( 85.04, 65.71) --
	( 85.04, 65.71) --
	( 85.04, 65.71) --
	( 84.96, 65.71) --
	( 84.96, 65.71) --
	( 84.96, 65.71) --
	( 84.89, 65.71) --
	( 84.89, 65.71) --
	( 84.89, 65.71) --
	( 84.82, 65.71) --
	( 84.82, 65.71) --
	( 84.82, 65.71) --
	( 84.75, 65.71) --
	( 84.75, 65.71) --
	( 84.75, 65.71) --
	( 84.68, 65.71) --
	( 84.68, 65.71) --
	( 84.68, 65.71) --
	( 84.61, 65.71) --
	( 84.61, 65.71) --
	( 84.61, 65.71) --
	( 84.54, 65.71) --
	( 84.54, 65.71) --
	( 84.54, 65.71) --
	( 84.46, 65.71) --
	( 84.46, 65.71) --
	( 84.46, 65.71) --
	( 84.39, 65.71) --
	( 84.39, 65.71) --
	( 84.39, 65.71) --
	( 84.32, 65.71) --
	( 84.32, 65.71) --
	( 84.32, 65.71) --
	( 84.25, 65.71) --
	( 84.25, 65.71) --
	( 84.25, 65.71) --
	( 84.18, 65.71) --
	( 84.18, 65.71) --
	( 84.18, 65.71) --
	( 84.12, 65.71) --
	( 84.12, 65.71) --
	( 84.12, 65.71) --
	( 84.10, 65.71) --
	( 84.10, 65.71) --
	( 84.10, 65.71) --
	( 84.03, 65.71) --
	( 84.03, 65.71) --
	( 84.03, 65.71) --
	( 83.96, 65.71) --
	( 83.96, 65.71) --
	( 83.96, 65.71) --
	( 83.89, 65.71) --
	( 83.89, 65.71) --
	( 83.89, 65.71) --
	( 83.82, 65.71) --
	( 83.82, 65.71) --
	( 83.82, 65.71) --
	( 83.75, 65.71) --
	( 83.75, 65.71) --
	( 83.75, 65.71) --
	( 83.67, 65.71) --
	( 83.67, 65.71) --
	( 83.67, 65.71) --
	( 83.60, 65.71) --
	( 83.60, 65.71) --
	( 83.60, 65.71) --
	( 83.53, 65.71) --
	( 83.53, 65.71) --
	( 83.53, 65.71) --
	( 83.46, 65.71) --
	( 83.46, 65.71) --
	( 83.46, 65.71) --
	( 83.39, 65.71) --
	( 83.39, 65.71) --
	( 83.39, 65.71) --
	( 83.31, 65.71) --
	( 83.31, 65.71) --
	( 83.31, 65.71) --
	( 83.24, 65.71) --
	( 83.24, 65.71) --
	( 83.24, 65.71) --
	( 83.17, 65.71) --
	( 83.17, 65.71) --
	( 83.17, 65.71) --
	( 83.10, 65.71) --
	( 83.10, 65.71) --
	( 83.10, 65.71) --
	( 83.03, 65.71) --
	( 83.03, 65.71) --
	( 83.03, 65.71) --
	( 82.96, 65.71) --
	( 82.96, 65.71) --
	( 82.96, 65.71) --
	( 82.88, 65.71) --
	( 82.88, 65.71) --
	( 82.88, 65.71) --
	( 82.81, 65.71) --
	( 82.81, 65.71) --
	( 82.81, 65.71) --
	( 82.74, 65.71) --
	( 82.74, 65.71) --
	( 82.74, 65.71) --
	( 82.67, 65.71) --
	( 82.67, 65.71) --
	( 82.67, 65.71) --
	( 82.60, 65.71) --
	( 82.60, 65.71) --
	( 82.60, 65.71) --
	( 82.59, 65.71) --
	( 82.59, 65.71) --
	( 82.59, 65.71) --
	( 82.52, 65.71) --
	( 82.52, 65.71) --
	( 82.52, 65.71) --
	( 82.45, 65.71) --
	( 82.45, 65.71) --
	( 82.45, 65.71) --
	( 82.38, 65.71) --
	( 82.38, 65.71) --
	( 82.38, 65.71) --
	( 82.31, 65.71) --
	( 82.31, 65.71) --
	( 82.31, 65.71) --
	( 82.24, 65.71) --
	( 82.24, 65.71) --
	( 82.24, 65.71) --
	( 82.16, 65.71) --
	( 82.16, 65.71) --
	( 82.16, 65.71) --
	( 82.09, 65.71) --
	( 82.09, 65.71) --
	( 82.09, 65.71) --
	( 82.02, 65.71) --
	( 82.02, 65.71) --
	( 82.02, 65.71) --
	( 81.95, 65.71) --
	( 81.95, 65.71) --
	( 81.95, 65.71) --
	( 81.88, 65.71) --
	( 81.88, 65.71) --
	( 81.88, 65.71) --
	( 81.81, 65.71) --
	( 81.81, 65.71) --
	( 81.81, 65.71) --
	( 81.73, 65.71) --
	( 81.73, 65.71) --
	( 81.73, 65.71) --
	( 81.66, 65.71) --
	( 81.66, 65.71) --
	( 81.66, 65.71) --
	( 81.59, 65.71) --
	( 81.59, 65.71) --
	( 81.59, 65.71) --
	( 81.52, 65.71) --
	( 81.52, 65.71) --
	( 81.52, 65.71) --
	( 81.45, 65.71) --
	( 81.45, 65.71) --
	( 81.45, 65.71) --
	( 81.44, 65.71) --
	( 81.44, 65.71) --
	( 81.44, 65.71) --
	( 81.37, 65.71) --
	( 81.37, 65.71) --
	( 81.37, 65.71) --
	( 81.30, 65.71) --
	( 81.30, 65.71) --
	( 81.30, 65.71) --
	( 81.23, 65.71) --
	( 81.23, 65.71) --
	( 81.23, 65.71) --
	( 81.16, 65.71) --
	( 81.16, 65.71) --
	( 81.16, 65.71) --
	( 81.09, 65.71) --
	( 81.09, 65.71) --
	( 81.09, 65.71) --
	( 81.01, 65.71) --
	( 81.01, 65.71) --
	( 81.01, 65.71) --
	( 80.94, 65.71) --
	( 80.94, 65.71) --
	( 80.94, 65.71) --
	( 80.87, 65.71) --
	( 80.87, 65.71) --
	( 80.87, 65.71) --
	( 80.80, 65.71) --
	( 80.80, 65.71) --
	( 80.80, 65.71) --
	( 80.73, 65.71) --
	( 80.73, 65.71) --
	( 80.73, 65.71) --
	( 80.65, 65.71) --
	( 80.65, 65.71) --
	( 80.65, 65.71) --
	( 80.58, 65.71) --
	( 80.58, 65.71) --
	( 80.58, 65.71) --
	( 80.51, 65.71) --
	( 80.51, 65.71) --
	( 80.51, 65.71) --
	( 80.48, 65.71) --
	( 80.48, 65.71) --
	( 80.48, 65.71) --
	( 80.44, 65.71) --
	( 80.44, 65.71) --
	( 80.44, 65.71) --
	( 80.37, 65.71) --
	( 80.37, 65.71) --
	( 80.37, 65.71) --
	( 80.29, 65.71) --
	( 80.29, 65.71) --
	( 80.29, 65.71) --
	( 80.22, 65.71) --
	( 80.22, 65.71) --
	( 80.22, 65.71) --
	( 80.15, 65.71) --
	( 80.15, 65.71) --
	( 80.15, 65.71) --
	( 80.08, 65.71) --
	( 80.08, 65.71) --
	( 80.08, 65.71) --
	( 80.01, 65.71) --
	( 80.01, 65.71) --
	( 80.01, 65.71) --
	( 79.93, 65.71) --
	( 79.93, 65.71) --
	( 79.93, 65.71) --
	( 79.86, 65.71) --
	( 79.86, 65.71) --
	( 79.86, 65.71) --
	( 79.79, 65.71) --
	( 79.79, 65.71) --
	( 79.79, 65.71) --
	( 79.72, 65.71) --
	( 79.72, 65.71) --
	( 79.72, 65.71) --
	( 79.65, 65.71) --
	( 79.65, 65.71) --
	( 79.65, 65.71) --
	( 79.62, 65.71) --
	( 79.62, 65.71) --
	( 79.62, 65.71) --
	( 79.57, 65.71) --
	( 79.57, 65.71) --
	( 79.57, 65.71) --
	( 79.50, 65.71) --
	( 79.50, 65.71) --
	( 79.50, 65.71) --
	( 79.43, 65.71) --
	( 79.43, 65.71) --
	( 79.43, 65.71) --
	( 79.36, 65.71) --
	( 79.36, 65.71) --
	( 79.36, 65.71) --
	( 79.29, 65.71) --
	( 79.29, 65.71) --
	( 79.29, 65.71) --
	( 79.21, 65.71) --
	( 79.21, 65.71) --
	( 79.21, 65.71) --
	( 79.14, 65.71) --
	( 79.14, 65.71) --
	( 79.14, 65.71) --
	( 79.07, 65.71) --
	( 79.07, 65.71) --
	( 79.07, 65.71) --
	( 79.00, 65.71) --
	( 79.00, 65.71) --
	( 79.00, 65.71) --
	( 78.93, 65.71) --
	( 78.93, 65.71) --
	( 78.93, 65.71) --
	( 78.85, 65.71) --
	( 78.85, 65.71) --
	( 78.85, 65.71) --
	( 78.78, 65.71) --
	( 78.78, 65.71) --
	( 78.78, 65.71) --
	( 78.76, 65.71) --
	( 78.76, 65.71) --
	( 78.76, 65.71) --
	( 78.71, 65.71) --
	( 78.71, 65.71) --
	( 78.71, 65.71) --
	( 78.64, 65.71) --
	( 78.64, 65.71) --
	( 78.64, 65.71) --
	( 78.56, 65.71) --
	( 78.56, 65.71) --
	( 78.56, 65.71) --
	( 78.49, 65.71) --
	( 78.49, 65.71) --
	( 78.49, 65.71) --
	( 78.42, 65.71) --
	( 78.42, 65.71) --
	( 78.42, 65.71) --
	( 78.35, 65.71) --
	( 78.35, 65.71) --
	( 78.35, 65.71) --
	( 78.28, 65.71) --
	( 78.28, 65.71) --
	( 78.28, 65.71) --
	( 78.20, 65.71) --
	( 78.20, 65.71) --
	( 78.20, 65.71) --
	( 78.13, 65.71) --
	( 78.13, 65.71) --
	( 78.13, 65.71) --
	( 78.06, 65.71) --
	( 78.06, 65.71) --
	( 78.06, 65.71) --
	( 77.99, 65.71) --
	( 77.99, 65.71) --
	( 77.99, 65.71) --
	( 77.91, 65.71) --
	( 77.91, 65.71) --
	( 77.91, 65.71) --
	( 77.84, 65.71) --
	( 77.84, 65.71) --
	( 77.84, 65.71) --
	( 77.77, 65.71) --
	( 77.77, 65.71) --
	( 77.77, 65.71) --
	( 77.70, 65.71) --
	( 77.70, 65.71) --
	( 77.70, 65.71) --
	( 77.70, 65.71) --
	( 77.70, 65.71) --
	( 77.70, 65.71) --
	( 77.63, 65.71) --
	( 77.63, 65.71) --
	( 77.63, 65.71) --
	( 77.55, 65.71) --
	( 77.55, 65.71) --
	( 77.55, 65.71) --
	( 77.48, 65.71) --
	( 77.48, 65.71) --
	( 77.48, 65.71) --
	( 77.41, 65.71) --
	( 77.41, 65.71) --
	( 77.41, 65.71) --
	( 77.34, 65.71) --
	( 77.34, 65.71) --
	( 77.34, 65.71) --
	( 77.27, 65.71) --
	( 77.27, 65.71) --
	( 77.27, 65.71) --
	( 77.19, 65.71) --
	( 77.19, 65.71) --
	( 77.19, 65.71) --
	( 77.12, 65.71) --
	( 77.12, 65.71) --
	( 77.12, 65.71) --
	( 77.05, 65.71) --
	( 77.05, 65.71) --
	( 77.05, 65.71) --
	( 76.98, 65.71) --
	( 76.98, 65.71) --
	( 76.98, 65.71) --
	( 76.90, 65.71) --
	( 76.90, 65.71) --
	( 76.90, 65.71) --
	( 76.83, 65.71) --
	( 76.83, 65.71) --
	( 76.83, 65.71) --
	( 76.76, 65.71) --
	( 76.76, 65.71) --
	( 76.76, 65.71) --
	( 76.75, 65.71) --
	( 76.75, 65.71) --
	( 76.75, 65.71) --
	( 76.69, 65.71) --
	( 76.69, 65.71) --
	( 76.69, 65.71) --
	( 76.61, 65.71) --
	( 76.61, 65.71) --
	( 76.61, 65.71) --
	( 76.54, 65.71) --
	( 76.54, 65.71) --
	( 76.54, 65.71) --
	( 76.47, 65.71) --
	( 76.47, 65.71) --
	( 76.47, 65.71) --
	( 76.40, 65.71) --
	( 76.40, 65.71) --
	( 76.40, 65.71) --
	( 76.33, 65.71) --
	( 76.33, 65.71) --
	( 76.33, 65.71) --
	( 76.25, 65.71) --
	( 76.25, 65.71) --
	( 76.25, 65.71) --
	( 76.18, 65.71) --
	( 76.18, 65.71) --
	( 76.18, 65.71) --
	( 76.11, 65.71) --
	( 76.11, 65.71) --
	( 76.11, 65.71) --
	( 76.04, 65.71) --
	( 76.04, 65.71) --
	( 76.04, 65.71) --
	( 75.96, 65.71) --
	( 75.96, 65.71) --
	( 75.96, 65.71) --
	( 75.89, 65.71) --
	( 75.89, 65.71) --
	( 75.89, 65.71) --
	( 75.82, 65.71) --
	( 75.82, 65.71) --
	( 75.82, 65.71) --
	( 75.79, 65.71) --
	( 75.79, 65.71) --
	( 75.79, 65.71) --
	( 75.75, 65.71) --
	( 75.75, 65.71) --
	( 75.75, 65.71) --
	( 75.67, 65.71) --
	( 75.67, 65.71) --
	( 75.67, 65.71) --
	( 75.60, 65.71) --
	( 75.60, 65.71) --
	( 75.60, 65.71) --
	( 75.53, 65.71) --
	( 75.53, 65.71) --
	( 75.53, 65.71) --
	( 75.46, 65.71) --
	( 75.46, 65.71) --
	( 75.46, 65.71) --
	( 75.39, 65.71) --
	( 75.39, 65.71) --
	( 75.39, 65.71) --
	( 75.31, 65.71) --
	( 75.31, 65.71) --
	( 75.31, 65.71) --
	( 75.24, 65.71) --
	( 75.24, 65.71) --
	( 75.24, 65.71) --
	( 75.17, 65.71) --
	( 75.17, 65.71) --
	( 75.17, 65.71) --
	( 75.10, 65.71) --
	( 75.10, 65.71) --
	( 75.10, 65.71) --
	( 75.02, 65.71) --
	( 75.02, 65.71) --
	( 75.02, 65.71) --
	( 74.95, 65.71) --
	( 74.95, 65.71) --
	( 74.95, 65.71) --
	( 74.92, 65.71) --
	( 74.92, 65.71) --
	( 74.92, 65.71) --
	( 74.88, 65.71) --
	( 74.88, 65.71) --
	( 74.88, 65.71) --
	( 74.81, 65.71) --
	( 74.81, 65.71) --
	( 74.81, 65.71) --
	( 74.73, 65.71) --
	( 74.73, 65.71) --
	( 74.73, 65.71) --
	( 74.66, 65.71) --
	( 74.66, 65.71) --
	( 74.66, 65.71) --
	( 74.59, 65.71) --
	( 74.59, 65.71) --
	( 74.59, 65.71) --
	( 74.52, 65.71) --
	( 74.52, 65.71) --
	( 74.52, 65.71) --
	( 74.44, 65.71) --
	( 74.44, 65.71) --
	( 74.44, 65.71) --
	( 74.37, 65.71) --
	( 74.37, 65.71) --
	( 74.37, 65.71) --
	( 74.30, 65.71) --
	( 74.30, 65.71) --
	( 74.30, 65.71) --
	( 74.23, 65.71) --
	( 74.23, 65.71) --
	( 74.23, 65.71) --
	( 74.16, 65.71) --
	( 74.16, 65.71) --
	( 74.16, 65.71) --
	( 74.08, 65.71) --
	( 74.08, 65.71) --
	( 74.08, 65.71) --
	( 74.01, 65.71) --
	( 74.01, 65.71) --
	( 74.01, 65.71) --
	( 73.97, 65.71) --
	( 73.97, 65.71) --
	( 73.97, 65.71) --
	( 73.94, 65.71) --
	( 73.94, 65.71) --
	( 73.94, 65.71) --
	( 73.86, 65.71) --
	( 73.86, 65.71) --
	( 73.86, 65.71) --
	( 73.79, 65.71) --
	( 73.79, 65.71) --
	( 73.79, 65.71) --
	( 73.72, 65.71) --
	( 73.72, 65.71) --
	( 73.72, 65.71) --
	( 73.65, 65.71) --
	( 73.65, 65.71) --
	( 73.65, 65.71) --
	( 73.58, 65.71) --
	( 73.58, 65.71) --
	( 73.58, 65.71) --
	( 73.50, 65.71) --
	( 73.50, 65.71) --
	( 73.50, 65.71) --
	( 73.43, 65.71) --
	( 73.43, 65.71) --
	( 73.43, 65.71) --
	( 73.36, 65.71) --
	( 73.36, 65.71) --
	( 73.36, 65.71) --
	( 73.28, 65.71) --
	( 73.28, 65.71) --
	( 73.28, 65.71) --
	( 73.21, 65.71) --
	( 73.21, 65.71) --
	( 73.21, 65.71) --
	( 73.20, 65.71) --
	( 73.20, 65.71) --
	( 73.20, 65.71) --
	( 73.14, 65.71) --
	( 73.14, 65.71) --
	( 73.14, 65.71) --
	( 73.07, 65.71) --
	( 73.07, 65.71) --
	( 73.07, 65.71) --
	( 73.00, 65.71) --
	( 73.00, 65.71) --
	( 73.00, 65.71) --
	( 72.92, 65.71) --
	( 72.92, 65.71) --
	( 72.92, 65.71) --
	( 72.85, 65.71) --
	( 72.85, 65.71) --
	( 72.85, 65.71) --
	( 72.78, 65.71) --
	( 72.78, 65.71) --
	( 72.78, 65.71) --
	( 72.71, 65.71) --
	( 72.71, 65.71) --
	( 72.71, 65.71) --
	( 72.63, 65.71) --
	( 72.63, 65.71) --
	( 72.63, 65.71) --
	( 72.56, 65.71) --
	( 72.56, 65.71) --
	( 72.56, 65.71) --
	( 72.49, 65.71) --
	( 72.49, 65.71) --
	( 72.49, 65.71) --
	( 72.42, 65.71) --
	( 72.42, 65.71) --
	( 72.42, 65.71) --
	( 72.34, 65.71) --
	( 72.34, 65.71) --
	( 72.34, 65.71) --
	( 72.27, 65.71) --
	( 72.27, 65.71) --
	( 72.27, 65.71) --
	( 72.20, 65.71) --
	( 72.20, 65.71) --
	( 72.20, 65.71) --
	( 72.13, 65.71) --
	( 72.13, 65.71) --
	( 72.13, 65.71) --
	( 72.05, 65.71) --
	( 72.05, 65.71) --
	( 72.05, 65.71) --
	( 71.98, 65.71) --
	( 71.98, 65.71) --
	( 71.98, 65.71) --
	( 71.95, 65.71) --
	( 71.95, 65.71) --
	( 71.95, 65.71) --
	( 71.91, 65.71) --
	( 71.91, 65.71) --
	( 71.91, 65.71) --
	( 71.83, 65.71) --
	( 71.83, 65.71) --
	( 71.83, 65.71) --
	( 71.76, 65.71) --
	( 71.76, 65.71) --
	( 71.76, 65.71) --
	( 71.69, 65.71) --
	( 71.69, 65.71) --
	( 71.69, 65.71) --
	( 71.62, 65.71) --
	( 71.62, 65.71) --
	( 71.62, 65.71) --
	( 71.54, 65.71) --
	( 71.54, 65.71) --
	( 71.54, 65.71) --
	( 71.47, 65.71) --
	( 71.47, 65.71) --
	( 71.47, 65.71) --
	( 71.40, 65.71) --
	( 71.40, 65.71) --
	( 71.40, 65.71) --
	( 71.33, 65.71) --
	( 71.33, 65.71) --
	( 71.33, 65.71) --
	( 71.26, 65.71) --
	( 71.26, 65.71) --
	( 71.26, 65.71) --
	( 71.19, 65.71) --
	( 71.19, 65.71) --
	( 71.19, 65.71) --
	( 71.18, 65.71) --
	( 71.18, 65.71) --
	( 71.18, 65.71) --
	( 71.11, 65.71) --
	( 71.11, 65.71) --
	( 71.11, 65.71) --
	( 71.04, 65.71) --
	( 71.04, 65.71) --
	( 71.04, 65.71) --
	( 70.96, 65.71) --
	( 70.96, 65.71) --
	( 70.96, 65.71) --
	( 70.89, 65.71) --
	( 70.89, 65.71) --
	( 70.89, 65.71) --
	( 70.82, 65.71) --
	( 70.82, 65.71) --
	( 70.82, 65.71) --
	( 70.75, 65.71) --
	( 70.75, 65.71) --
	( 70.75, 65.71) --
	( 70.67, 65.71) --
	( 70.67, 65.71) --
	( 70.67, 65.71) --
	( 70.60, 65.71) --
	( 70.60, 65.71) --
	( 70.60, 65.71) --
	( 70.53, 65.71) --
	( 70.53, 65.71) --
	( 70.53, 65.71) --
	( 70.46, 65.71) --
	( 70.46, 65.71) --
	( 70.46, 65.71) --
	( 70.42, 65.71) --
	( 70.42, 65.71) --
	( 70.42, 65.71) --
	( 70.38, 65.71) --
	( 70.38, 65.71) --
	( 70.38, 65.71) --
	( 70.31, 65.71) --
	( 70.31, 65.71) --
	( 70.31, 65.71) --
	( 70.24, 65.71) --
	( 70.24, 65.71) --
	( 70.24, 65.71) --
	( 70.17, 65.71) --
	( 70.17, 65.71) --
	( 70.17, 65.71) --
	( 70.09, 65.71) --
	( 70.09, 65.71) --
	( 70.09, 65.71) --
	( 70.02, 65.71) --
	( 70.02, 65.71) --
	( 70.02, 65.71) --
	( 69.95, 65.71) --
	( 69.95, 65.71) --
	( 69.95, 65.71) --
	( 69.87, 65.71) --
	( 69.87, 65.71) --
	( 69.87, 65.71) --
	( 69.80, 65.71) --
	( 69.80, 65.71) --
	( 69.80, 65.71) --
	( 69.73, 65.71) --
	( 69.73, 65.71) --
	( 69.73, 65.71) --
	( 69.66, 65.71) --
	( 69.66, 65.71) --
	( 69.66, 65.71) --
	( 69.58, 65.71) --
	( 69.58, 65.71) --
	( 69.58, 65.71) --
	( 69.51, 65.71) --
	( 69.51, 65.71) --
	( 69.51, 65.71) --
	( 69.44, 65.71) --
	( 69.44, 65.71) --
	( 69.44, 65.71) --
	( 69.37, 65.71) --
	( 69.37, 65.71) --
	( 69.37, 65.71) --
	( 69.29, 65.71) --
	( 69.29, 65.71) --
	( 69.29, 65.71) --
	( 69.27, 65.71) --
	( 69.27, 65.71) --
	( 69.27, 65.71) --
	( 69.22, 65.71) --
	( 69.22, 65.71) --
	( 69.22, 65.71) --
	( 69.15, 65.71) --
	( 69.15, 65.71) --
	( 69.15, 65.71) --
	( 69.07, 65.71) --
	( 69.07, 65.71) --
	( 69.07, 65.71) --
	( 69.00, 65.71) --
	( 69.00, 65.71) --
	( 69.00, 65.71) --
	( 68.93, 65.71) --
	( 68.93, 65.71) --
	( 68.93, 65.71) --
	( 68.86, 65.71) --
	( 68.86, 65.71) --
	( 68.86, 65.71) --
	( 68.78, 65.71) --
	( 68.78, 65.71) --
	( 68.78, 65.71) --
	( 68.71, 65.71) --
	( 68.71, 65.71) --
	( 68.71, 65.71) --
	( 68.64, 65.71) --
	( 68.64, 65.71) --
	( 68.64, 65.71) --
	( 68.56, 65.71) --
	( 68.56, 65.71) --
	( 68.56, 65.71) --
	( 68.49, 65.71) --
	( 68.49, 65.71) --
	( 68.49, 65.71) --
	( 68.42, 65.71) --
	( 68.42, 65.71) --
	( 68.42, 65.71) --
	( 68.35, 65.71) --
	( 68.35, 65.71) --
	( 68.35, 65.71) --
	( 68.27, 65.71) --
	( 68.27, 65.71) --
	( 68.27, 65.71) --
	( 68.22, 65.71) --
	( 68.22, 65.71) --
	( 68.22, 65.71) --
	( 68.20, 65.71) --
	( 68.20, 65.71) --
	( 68.20, 65.71) --
	( 68.13, 65.71) --
	( 68.13, 65.71) --
	( 68.13, 65.71) --
	( 68.05, 65.71) --
	( 68.05, 65.71) --
	( 68.05, 65.71) --
	( 67.98, 65.71) --
	( 67.98, 65.71) --
	( 67.98, 65.71) --
	( 67.91, 65.71) --
	( 67.91, 65.71) --
	( 67.91, 65.71) --
	( 67.84, 65.71) --
	( 67.84, 65.71) --
	( 67.84, 65.71) --
	( 67.76, 65.71) --
	( 67.76, 65.71) --
	( 67.76, 65.71) --
	( 67.69, 65.71) --
	( 67.69, 65.71) --
	( 67.69, 65.71) --
	( 67.62, 65.71) --
	( 67.62, 65.71) --
	( 67.62, 65.71) --
	( 67.55, 65.71) --
	( 67.55, 65.71) --
	( 67.55, 65.71) --
	( 67.47, 65.71) --
	( 67.47, 65.71) --
	( 67.47, 65.71) --
	( 67.40, 65.71) --
	( 67.40, 65.71) --
	( 67.40, 65.71) --
	( 67.33, 65.71) --
	( 67.33, 65.71) --
	( 67.33, 65.71) --
	( 67.25, 65.71) --
	( 67.25, 65.71) --
	( 67.25, 65.71) --
	( 67.18, 65.71) --
	( 67.18, 65.71) --
	( 67.18, 65.71) --
	( 67.11, 65.71) --
	( 67.11, 65.71) --
	( 67.11, 65.71) --
	( 67.04, 65.71) --
	( 67.04, 65.71) --
	( 67.04, 65.71) --
	( 66.96, 65.71) --
	( 66.96, 65.71) --
	( 66.96, 65.71) --
	( 66.89, 65.71) --
	( 66.89, 65.71) --
	( 66.89, 65.71) --
	( 66.82, 65.71) --
	( 66.82, 65.71) --
	( 66.82, 65.71) --
	( 66.74, 65.71) --
	( 66.74, 65.71) --
	( 66.74, 65.71) --
	( 66.67, 65.71) --
	( 66.67, 65.71) --
	( 66.67, 65.71) --
	( 66.60, 65.71) --
	( 66.60, 65.71) --
	( 66.60, 65.71) --
	( 66.53, 65.71) --
	( 66.53, 65.71) --
	( 66.53, 65.71) --
	( 66.45, 65.71) --
	( 66.45, 65.71) --
	( 66.45, 65.71) --
	( 66.40, 65.71) --
	( 66.40, 65.71) --
	( 66.40, 65.71) --
	( 66.38, 65.71) --
	( 66.38, 65.71) --
	( 66.38, 65.71) --
	( 66.31, 65.71) --
	( 66.31, 65.71) --
	( 66.31, 65.71) --
	( 66.23, 65.71) --
	( 66.23, 65.71) --
	( 66.23, 65.71) --
	( 66.16, 65.71) --
	( 66.16, 65.71) --
	( 66.16, 65.71) --
	( 66.09, 65.71) --
	( 66.09, 65.71) --
	( 66.09, 65.71) --
	( 66.02, 65.71) --
	( 66.02, 65.71) --
	( 66.02, 65.71) --
	( 65.94, 65.71) --
	( 65.94, 65.71) --
	( 65.94, 65.71) --
	( 65.87, 65.71) --
	( 65.87, 65.71) --
	( 65.87, 65.71) --
	( 65.80, 65.71) --
	( 65.80, 65.71) --
	( 65.80, 65.71) --
	( 65.72, 65.71) --
	( 65.72, 65.71) --
	( 65.72, 65.71) --
	( 65.65, 65.71) --
	( 65.65, 65.71) --
	( 65.65, 65.71) --
	( 65.58, 65.71) --
	( 65.58, 65.71) --
	( 65.58, 65.71) --
	( 65.50, 65.71) --
	( 65.50, 65.71) --
	( 65.50, 65.71) --
	( 65.43, 65.71) --
	( 65.43, 65.71) --
	( 65.43, 65.71) --
	( 65.36, 65.71) --
	( 65.36, 65.71) --
	( 65.36, 65.71) --
	( 65.29, 65.71) --
	( 65.29, 65.71) --
	( 65.29, 65.71) --
	( 65.21, 65.71) --
	( 65.21, 65.71) --
	( 65.21, 65.71) --
	( 65.14, 65.71) --
	( 65.14, 65.71) --
	( 65.14, 65.71) --
	( 65.07, 65.71) --
	( 65.07, 65.71) --
	( 65.07, 65.71) --
	( 64.99, 65.71) --
	( 64.99, 65.71) --
	( 64.99, 65.71) --
	( 64.92, 65.71) --
	( 64.92, 65.71) --
	( 64.92, 65.71) --
	( 64.85, 65.71) --
	( 64.85, 65.71) --
	( 64.85, 65.71) --
	( 64.77, 65.71) --
	( 64.77, 65.71) --
	( 64.77, 65.71) --
	( 64.70, 65.71) --
	( 64.70, 65.71) --
	( 64.70, 65.71) --
	( 64.63, 65.71) --
	( 64.63, 65.71) --
	( 64.63, 65.71) --
	( 64.57, 65.71) --
	( 64.57, 65.71) --
	( 64.57, 65.71) --
	( 64.56, 65.71) --
	( 64.56, 65.71) --
	( 64.56, 65.71) --
	( 64.48, 65.71) --
	( 64.48, 65.71) --
	( 64.48, 65.71) --
	( 64.41, 65.71) --
	( 64.41, 65.71) --
	( 64.41, 65.71) --
	( 64.34, 65.71) --
	( 64.34, 65.71) --
	( 64.34, 65.71) --
	( 64.26, 65.71) --
	( 64.26, 65.71) --
	( 64.26, 65.71) --
	( 64.19, 65.71) --
	( 64.19, 65.71) --
	( 64.19, 65.71) --
	( 64.12, 65.71) --
	( 64.12, 65.71) --
	( 64.12, 65.71) --
	( 64.04, 65.71) --
	( 64.04, 65.71) --
	( 64.04, 65.71) --
	( 63.97, 65.71) --
	( 63.97, 65.71) --
	( 63.97, 65.71) --
	( 63.90, 65.71) --
	( 63.90, 65.71) --
	( 63.90, 65.71) --
	( 63.83, 65.71) --
	( 63.83, 65.71) --
	( 63.83, 65.71) --
	( 63.75, 65.71) --
	( 63.75, 65.71) --
	( 63.75, 65.71) --
	( 63.68, 65.71) --
	( 63.68, 65.71) --
	( 63.68, 65.71) --
	( 63.61, 65.71) --
	( 63.61, 65.71) --
	( 63.61, 65.71) --
	( 63.53, 65.71) --
	( 63.53, 65.71) --
	( 63.53, 65.71) --
	( 63.46, 65.71) --
	( 63.46, 65.71) --
	( 63.46, 65.71) --
	( 63.39, 65.71) --
	( 63.39, 65.71) --
	( 63.39, 65.71) --
	( 63.31, 65.71) --
	( 63.31, 65.71) --
	( 63.31, 65.71) --
	( 63.24, 65.71) --
	( 63.24, 65.71) --
	( 63.24, 65.71) --
	( 63.23, 65.71) --
	( 63.23, 65.71) --
	( 63.23, 65.71) --
	( 63.17, 65.71) --
	( 63.17, 65.71) --
	( 63.17, 65.71) --
	( 63.09, 65.71) --
	( 63.09, 65.71) --
	( 63.09, 65.71) --
	( 63.02, 65.71) --
	( 63.02, 65.71) --
	( 63.02, 65.71) --
	( 62.95, 65.71) --
	( 62.95, 65.71) --
	( 62.95, 65.71) --
	( 62.87, 65.71) --
	( 62.87, 65.71) --
	( 62.87, 65.71) --
	( 62.80, 65.71) --
	( 62.80, 65.71) --
	( 62.80, 65.71) --
	( 62.73, 65.71) --
	( 62.73, 65.71) --
	( 62.73, 65.71) --
	( 62.65, 65.71) --
	( 62.65, 65.71) --
	( 62.65, 65.71) --
	( 62.58, 65.71) --
	( 62.58, 65.71) --
	( 62.58, 65.71) --
	( 62.51, 65.71) --
	( 62.51, 65.71) --
	( 62.51, 65.71) --
	( 62.44, 65.71) --
	( 62.44, 65.71) --
	( 62.44, 65.71) --
	( 62.36, 65.71) --
	( 62.36, 65.71) --
	( 62.36, 65.71) --
	( 62.29, 65.71) --
	( 62.29, 65.71) --
	( 62.29, 65.71) --
	( 62.22, 65.71) --
	( 62.22, 65.71) --
	( 62.22, 65.71) --
	( 62.14, 65.71) --
	( 62.14, 65.71) --
	( 62.14, 65.71) --
	( 62.07, 65.71) --
	( 62.07, 65.71) --
	( 62.07, 65.71) --
	( 62.00, 65.71) --
	( 62.00, 65.71) --
	( 62.00, 65.71) --
	( 61.92, 65.71) --
	( 61.92, 65.71) --
	( 61.92, 65.71) --
	( 61.85, 65.71) --
	( 61.85, 65.71) --
	( 61.85, 65.71) --
	( 61.78, 65.71) --
	( 61.78, 65.71) --
	( 61.78, 65.71) --
	( 61.70, 65.71) --
	( 61.70, 65.71) --
	( 61.70, 65.71) --
	( 61.63, 65.71) --
	( 61.63, 65.71) --
	( 61.63, 65.71) --
	( 61.56, 65.71) --
	( 61.56, 65.71) --
	( 61.56, 65.71) --
	( 61.48, 65.71) --
	( 61.48, 65.71) --
	( 61.48, 65.71) --
	( 61.41, 65.71) --
	( 61.41, 65.71) --
	( 61.41, 65.71) --
	( 61.41, 65.71) --
	( 61.41, 65.71) --
	( 61.41, 65.71) --
	( 61.34, 65.71) --
	( 61.34, 65.71) --
	( 61.34, 65.71) --
	( 61.26, 65.71) --
	( 61.26, 65.71) --
	( 61.26, 65.71) --
	( 61.19, 65.71) --
	( 61.19, 65.71) --
	( 61.19, 65.71) --
	( 61.12, 65.71) --
	( 61.12, 65.71) --
	( 61.12, 65.71) --
	( 61.04, 65.71) --
	( 61.04, 65.71) --
	( 61.04, 65.71) --
	( 60.97, 65.71) --
	( 60.97, 65.71) --
	( 60.97, 65.71) --
	( 60.90, 65.71) --
	( 60.90, 65.71) --
	( 60.90, 65.71) --
	( 60.83, 65.71) --
	( 60.83, 65.71) --
	( 60.83, 65.71) --
	( 60.75, 65.71) --
	( 60.75, 65.71) --
	( 60.75, 65.71) --
	( 60.68, 65.71) --
	( 60.68, 65.71) --
	( 60.68, 65.71) --
	( 60.61, 65.71) --
	( 60.61, 65.71) --
	( 60.61, 65.71) --
	( 60.53, 65.71) --
	( 60.53, 65.71) --
	( 60.53, 65.71) --
	( 60.46, 65.71) --
	( 60.46, 65.71) --
	( 60.46, 65.71) --
	( 60.45, 65.71) --
	( 60.45, 65.71) --
	( 60.45, 65.71) --
	( 60.39, 65.71) --
	( 60.39, 65.71) --
	( 60.39, 65.71) --
	( 60.31, 65.71) --
	( 60.31, 65.71) --
	( 60.31, 65.71) --
	( 60.24, 65.71) --
	( 60.24, 65.71) --
	( 60.24, 65.71) --
	( 60.17, 65.71) --
	( 60.17, 65.71) --
	( 60.17, 65.71) --
	( 60.09, 65.71) --
	( 60.09, 65.71) --
	( 60.09, 65.71) --
	( 60.02, 65.71) --
	( 60.02, 65.71) --
	( 60.02, 65.71) --
	( 59.95, 65.71) --
	( 59.95, 65.71) --
	( 59.95, 65.71) --
	( 59.87, 65.71) --
	( 59.87, 65.71) --
	( 59.87, 65.71) --
	( 59.80, 65.71) --
	( 59.80, 65.71) --
	( 59.80, 65.71) --
	( 59.73, 65.71) --
	( 59.73, 65.71) --
	( 59.73, 65.71) --
	( 59.65, 65.71) --
	( 59.65, 65.71) --
	( 59.65, 65.71) --
	( 59.58, 65.71) --
	( 59.58, 65.71) --
	( 59.58, 65.71) --
	( 59.51, 65.71) --
	( 59.51, 65.71) --
	( 59.51, 65.71) --
	( 59.50, 65.71) --
	( 59.50, 65.71) --
	( 59.50, 65.71) --
	( 59.43, 65.71) --
	( 59.43, 65.71) --
	( 59.43, 65.71) --
	( 59.36, 65.71) --
	( 59.36, 65.71) --
	( 59.36, 65.71) --
	( 59.29, 65.71) --
	( 59.29, 65.71) --
	( 59.29, 65.71) --
	( 59.21, 65.71) --
	( 59.21, 65.71) --
	( 59.21, 65.71) --
	( 59.14, 65.71) --
	( 59.14, 65.71) --
	( 59.14, 65.71) --
	( 59.07, 65.71) --
	( 59.07, 65.71) --
	( 59.07, 65.71) --
	( 58.99, 65.71) --
	( 58.99, 65.71) --
	( 58.99, 65.71) --
	( 58.92, 65.71) --
	( 58.92, 65.71) --
	( 58.92, 65.71) --
	( 58.92, 65.71) --
	( 58.92, 65.71) --
	( 58.92, 65.71) --
	( 58.85, 65.71) --
	( 58.85, 65.71) --
	( 58.85, 65.71) --
	( 58.77, 65.71) --
	( 58.77, 65.71) --
	( 58.77, 65.71) --
	( 58.70, 65.71) --
	( 58.70, 65.71) --
	( 58.70, 65.71) --
	( 58.63, 65.71) --
	( 58.63, 65.71) --
	( 58.63, 65.71) --
	( 58.55, 65.71) --
	( 58.55, 65.71) --
	( 58.55, 65.71) --
	( 58.48, 65.71) --
	( 58.48, 65.71) --
	( 58.48, 65.71) --
	( 58.41, 65.71) --
	( 58.41, 65.71) --
	( 58.41, 65.71) --
	( 58.35, 65.71) --
	( 58.35, 65.71) --
	( 58.35, 65.71) --
	( 58.33, 65.71) --
	( 58.33, 65.71) --
	( 58.33, 65.71) --
	( 58.26, 65.71) --
	( 58.26, 65.71) --
	( 58.26, 65.71) --
	( 58.18, 65.71) --
	( 58.18, 65.71) --
	( 58.18, 65.71) --
	( 58.11, 65.71) --
	( 58.11, 65.71) --
	( 58.11, 65.71) --
	( 58.04, 65.71) --
	( 58.04, 65.71) --
	( 58.04, 65.71) --
	( 57.97, 65.71) --
	( 57.97, 65.71) --
	( 57.97, 65.71) --
	( 57.89, 65.71) --
	( 57.89, 65.71) --
	( 57.89, 65.71) --
	( 57.87, 65.71) --
	( 57.87, 65.71) --
	( 57.87, 65.71) --
	( 57.82, 65.71) --
	( 57.82, 65.71) --
	( 57.82, 65.71) --
	( 57.74, 65.71) --
	( 57.74, 65.71) --
	( 57.74, 65.71) --
	( 57.67, 65.71) --
	( 57.67, 65.71) --
	( 57.67, 65.71) --
	( 57.60, 65.71) --
	( 57.60, 65.71) --
	( 57.60, 65.71) --
	( 57.52, 65.71) --
	( 57.52, 65.71) --
	( 57.52, 65.71) --
	( 57.45, 65.71) --
	( 57.45, 65.71) --
	( 57.45, 65.71) --
	( 57.38, 65.71) --
	( 57.38, 65.71) --
	( 57.38, 65.71) --
	( 57.30, 65.71) --
	( 57.30, 65.71) --
	( 57.30, 65.71) --
	( 57.23, 65.71) --
	( 57.23, 65.71) --
	( 57.23, 65.71) --
	( 57.20, 65.71) --
	( 57.20, 65.71) --
	( 57.20, 65.71) --
	( 57.16, 65.71) --
	( 57.16, 65.71) --
	( 57.16, 65.71) --
	( 57.08, 65.71) --
	( 57.08, 65.71) --
	( 57.08, 65.71) --
	( 57.01, 65.71) --
	( 57.01, 65.71) --
	( 57.01, 65.71) --
	( 56.94, 65.71) --
	( 56.94, 65.71) --
	( 56.94, 65.71) --
	( 56.91, 65.71) --
	( 56.91, 65.71) --
	( 56.91, 65.71) --
	( 56.86, 65.71) --
	( 56.86, 65.71) --
	( 56.86, 65.71) --
	( 56.79, 65.71) --
	( 56.79, 65.71) --
	( 56.79, 65.71) --
	( 56.72, 65.71) --
	( 56.72, 65.71) --
	( 56.72, 65.71) --
	( 56.64, 65.71) --
	( 56.64, 65.71) --
	( 56.64, 65.71) --
	( 56.57, 65.71) --
	( 56.57, 65.71) --
	( 56.57, 65.71) --
	( 56.50, 65.71) --
	( 56.50, 65.71) --
	( 56.50, 65.71) --
	( 56.43, 65.71) --
	( 56.43, 65.71) --
	( 56.43, 65.71) --
	( 56.42, 65.71) --
	( 56.42, 65.71) --
	( 56.42, 65.71) --
	( 56.35, 65.71) --
	( 56.35, 65.71) --
	( 56.35, 65.71) --
	( 56.27, 65.71) --
	( 56.27, 65.71) --
	( 56.27, 65.71) --
	( 56.20, 65.71) --
	( 56.20, 65.71) --
	( 56.20, 65.71) --
	( 56.14, 65.71) --
	( 56.14, 65.71) --
	( 56.14, 65.71) --
	( 56.13, 65.71) --
	( 56.13, 65.71) --
	( 56.13, 65.71) --
	( 56.06, 65.71) --
	( 56.06, 65.71) --
	( 56.06, 65.71) --
	( 55.98, 65.71) --
	( 55.98, 65.71) --
	( 55.98, 65.71) --
	( 55.91, 65.71) --
	( 55.91, 65.71) --
	( 55.91, 65.71) --
	( 55.83, 65.71) --
	( 55.83, 65.71) --
	( 55.83, 65.71) --
	( 55.76, 65.71) --
	( 55.76, 65.71) --
	( 55.76, 65.71) --
	( 55.69, 65.71) --
	( 55.69, 65.71) --
	( 55.69, 65.71) --
	( 55.61, 65.71) --
	( 55.61, 65.71) --
	( 55.61, 65.71) --
	( 55.57, 65.71) --
	( 55.57, 65.71) --
	( 55.57, 65.71) --
	( 55.54, 65.71) --
	( 55.54, 65.71) --
	( 55.54, 65.71) --
	( 55.47, 65.71) --
	( 55.47, 65.71) --
	( 55.47, 65.71) --
	( 55.39, 65.71) --
	( 55.39, 65.71) --
	( 55.39, 65.71) --
	( 55.32, 65.71) --
	( 55.32, 65.71) --
	( 55.32, 65.71) --
	( 55.28, 65.71) --
	( 55.28, 65.71) --
	( 55.28, 65.71) --
	( 55.25, 65.71) --
	( 55.25, 65.71) --
	( 55.25, 65.71) --
	( 55.17, 65.71) --
	( 55.17, 65.71) --
	( 55.17, 65.71) --
	( 55.10, 65.71) --
	( 55.10, 65.71) --
	( 55.10, 65.71) --
	( 55.02, 65.71) --
	( 55.02, 65.71) --
	( 55.02, 65.71) --
	( 54.95, 65.71) --
	( 54.95, 65.71) --
	( 54.95, 65.71) --
	( 54.88, 65.71) --
	( 54.88, 65.71) --
	( 54.88, 65.71) --
	( 54.80, 65.71) --
	( 54.80, 65.71) --
	( 54.80, 65.71) --
	( 54.73, 65.71) --
	( 54.73, 65.71) --
	( 54.73, 65.71) --
	( 54.66, 65.71) --
	( 54.66, 65.71) --
	( 54.66, 65.71) --
	( 54.58, 65.71) --
	( 54.58, 65.71) --
	( 54.58, 65.71) --
	( 54.51, 65.71) --
	( 54.51, 65.71) --
	( 54.51, 65.71) --
	( 54.51, 65.71) --
	( 54.51, 65.71) --
	( 54.51, 65.71) --
	( 54.44, 65.71) --
	( 54.44, 65.71) --
	( 54.44, 65.71) --
	( 54.36, 65.71) --
	( 54.36, 65.71) --
	( 54.36, 65.71) --
	( 54.29, 65.71) --
	( 54.29, 65.71) --
	( 54.29, 65.71) --
	( 54.21, 65.71) --
	( 54.21, 65.71) --
	( 54.21, 65.71) --
	( 54.14, 65.71) --
	( 54.14, 65.71) --
	( 54.14, 65.71) --
	( 54.07, 65.71) --
	( 54.07, 65.71) --
	( 54.07, 65.71) --
	( 53.99, 65.71) --
	( 53.99, 65.71) --
	( 53.99, 65.71) --
	( 53.92, 65.71) --
	( 53.92, 65.71) --
	( 53.92, 65.71) --
	( 53.85, 65.71) --
	( 53.85, 65.71) --
	( 53.85, 65.71) --
	( 53.77, 65.71) --
	( 53.77, 65.71) --
	( 53.77, 65.71) --
	( 53.75, 65.71) --
	( 53.75, 65.71) --
	( 53.75, 65.71) --
	( 53.70, 65.71) --
	( 53.70, 65.71) --
	( 53.70, 65.71) --
	( 53.62, 65.71) --
	( 53.62, 65.71) --
	( 53.62, 65.71) --
	( 53.55, 65.71) --
	( 53.55, 65.71) --
	( 53.55, 65.71) --
	( 53.48, 65.71) --
	( 53.48, 65.71) --
	( 53.48, 65.71) --
	( 53.41, 65.71) --
	( 53.41, 65.71) --
	( 53.41, 65.71) --
	( 53.33, 65.71) --
	( 53.33, 65.71) --
	( 53.33, 65.71) --
	( 53.26, 65.71) --
	( 53.26, 65.71) --
	( 53.26, 65.71) --
	( 53.18, 65.71) --
	( 53.18, 65.71) --
	( 53.18, 65.71) --
	( 53.17, 65.71) --
	( 53.17, 65.71) --
	( 53.17, 65.71) --
	( 53.11, 65.71) --
	( 53.11, 65.71) --
	( 53.11, 65.71) --
	( 53.04, 65.71) --
	( 53.04, 65.71) --
	( 53.04, 65.71) --
	( 52.96, 65.71) --
	( 52.96, 65.71) --
	( 52.96, 65.71) --
	( 52.89, 65.71) --
	( 52.89, 65.71) --
	( 52.89, 65.71) --
	( 52.81, 65.71) --
	( 52.81, 65.71) --
	( 52.81, 65.71) --
	( 52.74, 65.71) --
	( 52.74, 65.71) --
	( 52.74, 65.71) --
	( 52.69, 65.71) --
	( 52.69, 65.71) --
	( 52.69, 65.71) --
	( 52.67, 65.71) --
	( 52.67, 65.71) --
	( 52.67, 65.71) --
	( 52.59, 65.71) --
	( 52.59, 65.71) --
	( 52.59, 65.71) --
	( 52.52, 65.71) --
	( 52.52, 65.71) --
	( 52.52, 65.71) --
	( 52.45, 65.71) --
	( 52.45, 65.71) --
	( 52.45, 65.71) --
	( 52.37, 65.71) --
	( 52.37, 65.71) --
	( 52.37, 65.71) --
	( 52.30, 65.71) --
	( 52.30, 65.71) --
	( 52.30, 65.71) --
	( 52.22, 65.71) --
	( 52.22, 65.71) --
	( 52.22, 65.71) --
	( 52.21, 65.71) --
	( 52.21, 65.71) --
	( 52.21, 65.71) --
	( 52.15, 65.71) --
	( 52.15, 65.71) --
	( 52.15, 65.71) --
	( 52.08, 65.71) --
	( 52.08, 65.71) --
	( 52.08, 65.71) --
	( 52.00, 65.71) --
	( 52.00, 65.71) --
	( 52.00, 65.71) --
	( 51.93, 65.71) --
	( 51.93, 65.71) --
	( 51.93, 65.71) --
	( 51.86, 65.71) --
	( 51.86, 65.71) --
	( 51.86, 65.71) --
	( 51.78, 65.71) --
	( 51.78, 65.71) --
	( 51.78, 65.71) --
	( 51.73, 65.71) --
	( 51.73, 65.71) --
	( 51.73, 65.71) --
	( 51.71, 65.71) --
	( 51.71, 65.71) --
	( 51.71, 65.71) --
	( 51.63, 65.71) --
	( 51.63, 65.71) --
	( 51.63, 65.71) --
	( 51.56, 65.71) --
	( 51.56, 65.71) --
	( 51.56, 65.71) --
	( 51.49, 65.71) --
	( 51.49, 65.71) --
	( 51.49, 65.71) --
	( 51.41, 65.71) --
	( 51.41, 65.71) --
	( 51.41, 65.71) --
	( 51.35, 65.71) --
	( 51.35, 65.71) --
	( 51.35, 65.71) --
	( 51.34, 65.71) --
	( 51.34, 65.71) --
	( 51.34, 65.71) --
	( 51.27, 65.71) --
	( 51.27, 65.71) --
	( 51.27, 65.71) --
	( 51.19, 65.71) --
	( 51.19, 65.71) --
	( 51.19, 65.71) --
	( 51.12, 65.71) --
	( 51.12, 65.71) --
	( 51.12, 65.71) --
	( 51.04, 65.71) --
	( 51.04, 65.71) --
	( 51.04, 65.71) --
	( 50.97, 65.71) --
	( 50.97, 65.71) --
	( 50.97, 65.71) --
	( 50.97, 65.71) --
	( 50.97, 65.71) --
	( 50.97, 65.71) --
	( 50.90, 65.71) --
	( 50.90, 65.71) --
	( 50.90, 65.71) --
	( 50.87, 65.71) --
	( 50.87, 65.71) --
	( 50.87, 65.71) --
	( 50.82, 65.71) --
	( 50.82, 65.71) --
	( 50.82, 65.71) --
	( 50.75, 65.71) --
	( 50.75, 65.71) --
	( 50.75, 65.71) --
	( 50.67, 65.71) --
	( 50.67, 65.71) --
	( 50.67, 65.71) --
	( 50.60, 65.71) --
	( 50.60, 65.71) --
	( 50.60, 65.71) --
	( 50.53, 65.71) --
	( 50.53, 65.71) --
	( 50.53, 65.71) --
	( 50.49, 65.71) --
	( 50.49, 65.71) --
	( 50.49, 65.71) --
	( 50.45, 65.71) --
	( 50.45, 65.71) --
	( 50.45, 65.71) --
	( 50.38, 65.71) --
	( 50.38, 65.71) --
	( 50.38, 65.71) --
	( 50.30, 65.71) --
	( 50.30, 65.71) --
	( 50.30, 65.71) --
	( 50.23, 65.71) --
	( 50.23, 65.71) --
	( 50.23, 65.71) --
	( 50.16, 65.71) --
	( 50.16, 65.71) --
	( 50.16, 65.71) --
	( 50.10, 65.71) --
	( 50.10, 65.71) --
	( 50.10, 65.71) --
	( 50.08, 65.71) --
	( 50.08, 65.71) --
	( 50.08, 65.71) --
	( 50.01, 65.71) --
	( 50.01, 65.71) --
	( 50.01, 65.71) --
	( 49.94, 65.71) --
	( 49.94, 65.71) --
	( 49.94, 65.71) --
	( 49.91, 65.71) --
	( 49.91, 65.71) --
	( 49.91, 65.71) --
	( 49.86, 65.71) --
	( 49.86, 65.71) --
	( 49.86, 65.71) --
	( 49.79, 65.71) --
	( 49.79, 65.71) --
	( 49.79, 65.71) --
	( 49.72, 65.71) --
	( 49.72, 65.71) --
	( 49.72, 65.71) --
	( 49.71, 65.71) --
	( 49.71, 65.71) --
	( 49.71, 65.71) --
	( 49.64, 65.71) --
	( 49.64, 65.71) --
	( 49.64, 65.71) --
	( 49.57, 65.71) --
	( 49.57, 65.71) --
	( 49.57, 65.71) --
	( 49.53, 65.71) --
	( 49.53, 65.71) --
	( 49.53, 65.71) --
	( 49.49, 65.71) --
	( 49.49, 65.71) --
	( 49.49, 65.71) --
	( 49.42, 65.71) --
	( 49.42, 65.71) --
	( 49.42, 65.71) --
	( 49.34, 65.71) --
	( 49.34, 65.71) --
	( 49.34, 65.71) --
	( 49.34, 65.71) --
	( 49.34, 65.71) --
	( 49.34, 65.71) --
	( 49.27, 65.71) --
	( 49.27, 65.71) --
	( 49.27, 65.71) --
	( 49.20, 65.71) --
	( 49.20, 65.71) --
	( 49.20, 65.71) --
	( 49.12, 65.71) --
	( 49.12, 65.71) --
	( 49.12, 65.71) --
	( 49.05, 65.71) --
	( 49.05, 65.71) --
	( 49.05, 65.71) --
	( 49.05, 65.71) --
	( 49.05, 65.71) --
	( 49.05, 65.71) --
	( 48.97, 65.71) --
	( 48.97, 65.71) --
	( 48.97, 65.71) --
	( 48.95, 65.71) --
	( 48.95, 65.71) --
	( 48.95, 65.71) --
	( 48.90, 65.71) --
	( 48.90, 65.71) --
	( 48.90, 65.71) --
	( 48.83, 65.71) --
	( 48.83, 65.71) --
	( 48.83, 65.71) --
	( 48.76, 65.71) --
	( 48.76, 65.71) --
	( 48.76, 65.71) --
	( 48.75, 65.71) --
	( 48.75, 65.71) --
	( 48.75, 65.71) --
	( 48.68, 65.71) --
	( 48.68, 65.71) --
	( 48.68, 65.71) --
	( 48.67, 65.71) --
	( 48.67, 65.71) --
	( 48.67, 65.71) --
	( 48.60, 65.71) --
	( 48.60, 65.71) --
	( 48.60, 65.71) --
	( 48.53, 65.71) --
	( 48.53, 65.71) --
	( 48.53, 65.71) --
	( 48.47, 65.71) --
	( 48.47, 65.71) --
	( 48.47, 65.71) --
	( 48.46, 65.71) --
	( 48.46, 65.71) --
	( 48.46, 65.71) --
	( 48.38, 65.71) --
	( 48.38, 65.71) --
	( 48.38, 65.71) --
	( 48.31, 65.71) --
	( 48.31, 65.71) --
	( 48.31, 65.71) --
	( 48.28, 65.71) --
	( 48.28, 65.71) --
	( 48.28, 65.71) --
	( 48.23, 65.71) --
	( 48.23, 65.71) --
	( 48.23, 65.71) --
	( 48.19, 65.71) --
	( 48.19, 65.71) --
	( 48.19, 65.71) --
	( 48.16, 65.71) --
	( 48.16, 65.71) --
	( 48.16, 65.71) --
	( 48.09, 65.71) --
	( 48.09, 65.71) --
	( 48.09, 65.71) --
	( 48.01, 65.71) --
	( 48.01, 65.71) --
	( 48.01, 65.71) --
	( 48.00, 65.71) --
	( 48.00, 65.71) --
	( 48.00, 65.71) --
	( 47.94, 65.71) --
	( 47.94, 65.71) --
	( 47.94, 65.71) --
	( 47.90, 65.71) --
	( 47.90, 65.71) --
	( 47.90, 65.71) --
	( 47.86, 65.71) --
	( 47.86, 65.71) --
	( 47.86, 65.71) --
	( 47.80, 65.71) --
	( 47.80, 65.71) --
	( 47.80, 65.71) --
	( 47.79, 65.71) --
	( 47.79, 65.71) --
	( 47.79, 65.71) --
	( 47.72, 65.71) --
	( 47.72, 65.71) --
	( 47.72, 65.71) --
	( 47.64, 65.71) --
	( 47.64, 65.71) --
	( 47.64, 65.71) --
	( 47.61, 65.71) --
	( 47.61, 65.71) --
	( 47.61, 65.71) --
	( 47.57, 65.71) --
	( 47.57, 65.71) --
	( 47.57, 65.71) --
	( 47.52, 65.71) --
	( 47.52, 65.71) --
	( 47.52, 65.71) --
	( 47.50, 65.71) --
	( 47.50, 65.71) --
	( 47.50, 65.71) --
	( 47.42, 65.71) --
	( 47.42, 65.71) --
	( 47.42, 65.71) --
	( 47.35, 65.71) --
	( 47.35, 65.71) --
	( 47.35, 65.71) --
	( 47.32, 65.71) --
	( 47.32, 65.71) --
	( 47.32, 65.71) --
	( 47.27, 65.71) --
	( 47.27, 65.71) --
	( 47.27, 65.71) --
	( 47.23, 65.71) --
	( 47.23, 65.71) --
	( 47.23, 65.71) --
	( 47.20, 65.71) --
	( 47.20, 65.71) --
	( 47.20, 65.71) --
	( 47.13, 65.71) --
	( 47.13, 65.71) --
	( 47.13, 65.71) --
	( 47.13, 65.71) --
	( 47.13, 65.71) --
	( 47.13, 65.71) --
	( 47.05, 65.71) --
	( 47.05, 65.71) --
	( 47.05, 65.71) --
	( 46.98, 65.71) --
	( 46.98, 65.71) --
	( 46.98, 65.71) --
	( 46.94, 65.71) --
	( 46.94, 65.71) --
	( 46.94, 65.71) --
	( 46.90, 65.71) --
	( 46.90, 65.71) --
	( 46.90, 65.71) --
	( 46.85, 65.71) --
	( 46.85, 65.71) --
	( 46.85, 65.71) --
	( 46.83, 65.71) --
	( 46.83, 65.71) --
	( 46.83, 65.71) --
	( 46.75, 65.71) --
	( 46.75, 65.71) --
	( 46.75, 65.71) --
	( 46.75, 65.71) --
	( 46.75, 65.71) --
	( 46.75, 65.71) --
	( 46.68, 65.71) --
	( 46.68, 65.71) --
	( 46.68, 65.71) --
	( 46.61, 65.71) --
	( 46.61, 65.71) --
	( 46.61, 65.71) --
	( 46.53, 65.71) --
	( 46.53, 65.71) --
	( 46.53, 65.71) --
	( 46.46, 65.71) --
	( 46.46, 65.71) --
	( 46.46, 65.71) --
	( 46.46, 65.71) --
	( 46.46, 65.71) --
	( 46.46, 65.71) --
	( 46.38, 65.71) --
	( 46.38, 65.71) --
	( 46.38, 65.71) --
	( 46.31, 65.71) --
	( 46.31, 65.71) --
	( 46.31, 65.71) --
	( 46.24, 65.71) --
	( 46.24, 65.71) --
	( 46.24, 65.71) --
	( 46.17, 65.71) --
	( 46.17, 65.71) --
	( 46.17, 65.71) --
	( 46.16, 65.71) --
	( 46.16, 65.71) --
	( 46.16, 65.71) --
	( 46.09, 65.71) --
	( 46.09, 65.71) --
	( 46.09, 65.71) --
	( 46.08, 65.71) --
	( 46.08, 65.71) --
	( 46.08, 65.71) --
	( 46.01, 65.71) --
	( 46.01, 65.71) --
	( 46.01, 65.71) --
	( 45.98, 65.71) --
	( 45.98, 65.71) --
	( 45.98, 65.71) --
	( 45.94, 65.71) --
	( 45.94, 65.71) --
	( 45.94, 65.71) --
	( 45.87, 65.71) --
	( 45.87, 65.71) --
	( 45.87, 65.71) --
	( 45.79, 65.71) --
	( 45.79, 65.71) --
	( 45.79, 65.71) --
	( 45.79, 65.71) --
	( 45.79, 65.71) --
	( 45.79, 65.71) --
	( 45.72, 65.71) --
	( 45.72, 65.71) --
	( 45.72, 65.71) --
	( 45.70, 65.71) --
	( 45.70, 65.71) --
	( 45.70, 65.71) --
	( 45.64, 65.71) --
	( 45.64, 65.71) --
	( 45.64, 65.71) --
	( 45.60, 65.71) --
	( 45.60, 65.71) --
	( 45.60, 65.71) --
	( 45.57, 65.71) --
	( 45.57, 65.71) --
	( 45.57, 65.71) --
	( 45.50, 65.71) --
	( 45.50, 65.71) --
	( 45.50, 65.71) --
	( 45.49, 65.71) --
	( 45.49, 65.71) --
	( 45.49, 65.71) --
	( 45.42, 65.71) --
	( 45.42, 65.71) --
	( 45.42, 65.71) --
	( 45.41, 65.71) --
	( 45.41, 65.71) --
	( 45.41, 65.71) --
	( 45.35, 65.71) --
	( 45.35, 65.71) --
	( 45.35, 65.71) --
	( 45.31, 65.71) --
	( 45.31, 65.71) --
	( 45.31, 65.71) --
	( 45.27, 65.71) --
	( 45.27, 65.71) --
	( 45.27, 65.71) --
	( 45.20, 65.71) --
	( 45.20, 65.71) --
	( 45.20, 65.71) --
	( 45.12, 65.71) --
	( 45.12, 65.71) --
	( 45.12, 65.71) --
	( 45.05, 65.71) --
	( 45.05, 65.71) --
	( 45.05, 65.71) --
	( 45.02, 65.71) --
	( 45.02, 65.71) --
	( 45.02, 65.71) --
	( 44.98, 65.71) --
	( 44.98, 65.71) --
	( 44.98, 65.71) --
	( 44.93, 65.71) --
	( 44.93, 65.71) --
	( 44.93, 65.71) --
	( 44.90, 65.71) --
	( 44.90, 65.71) --
	( 44.90, 65.71) --
	( 44.83, 65.71) --
	( 44.83, 65.71) --
	( 44.83, 65.71) --
	( 44.75, 65.71) --
	( 44.75, 65.71) --
	( 44.75, 65.71) --
	( 44.74, 65.71) --
	( 44.74, 65.71) --
	( 44.74, 65.71) --
	( 44.68, 65.71) --
	( 44.68, 65.71) --
	( 44.68, 65.71) --
	( 44.64, 65.71) --
	( 44.64, 65.71) --
	( 44.64, 65.71) --
	( 44.60, 65.71) --
	( 44.60, 65.71) --
	( 44.60, 65.71) --
	( 44.53, 65.71) --
	( 44.53, 65.71) --
	( 44.53, 65.71) --
	( 44.46, 65.71) --
	( 44.46, 65.71) --
	( 44.46, 65.71) --
	( 44.45, 65.71) --
	( 44.45, 65.71) --
	( 44.45, 65.71) --
	( 44.38, 65.71) --
	( 44.38, 65.71) --
	( 44.38, 65.71) --
	( 44.31, 65.71) --
	( 44.31, 65.71) --
	( 44.31, 65.71) --
	( 44.23, 65.71) --
	( 44.23, 65.71) --
	( 44.23, 65.71) --
	( 44.16, 65.71) --
	( 44.16, 65.71) --
	( 44.16, 65.71) --
	( 44.16, 65.71) --
	( 44.16, 65.71) --
	( 44.16, 65.71) --
	( 44.08, 65.71) --
	( 44.08, 65.71) --
	( 44.08, 65.71) --
	( 44.01, 65.71) --
	( 44.01, 65.71) --
	( 44.01, 65.71) --
	( 43.97, 65.71) --
	( 43.97, 65.71) --
	( 43.97, 65.71) --
	( 43.94, 65.71) --
	( 43.94, 65.71) --
	( 43.94, 65.71) --
	( 43.86, 65.71) --
	( 43.86, 65.71) --
	( 43.86, 65.71) --
	( 43.79, 65.71) --
	( 43.79, 65.71) --
	( 43.79, 65.71) --
	( 43.71, 65.71) --
	( 43.71, 65.71) --
	( 43.71, 65.71) --
	( 43.64, 65.71) --
	( 43.64, 65.71) --
	( 43.64, 65.71) --
	( 43.59, 65.71) --
	( 43.59, 65.71) --
	( 43.59, 65.71) --
	( 43.56, 65.71) --
	( 43.56, 65.71) --
	( 43.56, 65.71) --
	( 43.49, 65.71) --
	( 43.49, 65.71) --
	( 43.49, 65.71) --
	( 43.49, 65.71) --
	( 43.49, 65.71) --
	( 43.49, 65.71) --
	( 43.42, 65.71) --
	( 43.42, 65.71) --
	( 43.42, 65.71) --
	( 43.34, 65.71) --
	( 43.34, 65.71) --
	( 43.34, 65.71) --
	( 43.30, 65.71) --
	( 43.30, 65.71) --
	( 43.30, 65.71) --
	( 43.27, 65.71) --
	( 43.27, 65.71) --
	( 43.27, 65.71) --
	( 43.19, 65.71) --
	( 43.19, 65.71) --
	( 43.19, 65.71) --
	( 43.12, 65.71) --
	( 43.12, 65.71) --
	( 43.12, 65.71) --
	( 43.04, 65.71) --
	( 43.04, 65.71) --
	( 43.04, 65.71) --
	( 43.01, 65.71) --
	( 43.01, 65.71) --
	( 43.01, 65.71) --
	( 42.97, 65.71) --
	( 42.97, 65.71) --
	( 42.97, 65.71) --
	( 42.90, 65.71) --
	( 42.90, 65.71) --
	( 42.90, 65.71) --
	( 42.82, 65.71) --
	( 42.82, 65.71) --
	( 42.82, 65.71) --
	( 42.82, 65.71) --
	( 42.82, 65.71) --
	( 42.82, 65.71) --
	( 42.75, 65.71) --
	( 42.75, 65.71) --
	( 42.75, 65.71) --
	( 42.67, 65.71) --
	( 42.67, 65.71) --
	( 42.67, 65.71) --
	( 42.60, 65.71) --
	( 42.60, 65.71) --
	( 42.60, 65.71) --
	( 42.53, 65.71) --
	( 42.53, 65.71) --
	( 42.53, 65.71) --
	( 42.52, 65.71) --
	( 42.52, 65.71) --
	( 42.52, 65.71) --
	( 42.45, 65.71) --
	( 42.45, 65.71) --
	( 42.45, 65.71) --
	( 42.38, 65.71) --
	( 42.38, 65.71) --
	( 42.38, 65.71) --
	( 42.30, 65.71) --
	( 42.30, 65.71) --
	( 42.30, 65.71) --
	( 42.23, 65.71) --
	( 42.23, 65.71) --
	( 42.23, 65.71) --
	( 42.21, 65.71) --
	( 42.21, 65.71) --
	( 42.21, 65.71) --
	( 42.15, 65.71) --
	( 42.15, 65.71) --
	( 42.15, 65.71) --
	( 42.08, 65.71) --
	( 42.08, 65.71) --
	( 42.08, 65.71) --
	( 42.05, 65.71) --
	( 42.05, 65.71) --
	( 42.05, 65.71) --
	( 42.00, 65.71) --
	( 42.00, 65.71) --
	( 42.00, 65.71) --
	( 41.93, 65.71) --
	( 41.93, 65.71) --
	( 41.93, 65.71) --
	( 41.86, 65.71) --
	( 41.86, 65.71) --
	( 41.86, 65.71) --
	( 41.78, 65.71) --
	( 41.78, 65.71) --
	( 41.78, 65.71) --
	( 41.71, 65.71) --
	( 41.71, 65.71) --
	( 41.71, 65.71) --
	( 41.67, 65.71) --
	( 41.67, 65.71) --
	( 41.67, 65.71) --
	( 41.63, 65.71) --
	( 41.63, 65.71) --
	( 41.63, 65.71) --
	( 41.56, 65.71) --
	( 41.56, 65.71) --
	( 41.56, 65.71) --
	( 41.48, 65.71) --
	( 41.48, 65.71) --
	( 41.48, 65.71) --
	( 41.41, 65.71) --
	( 41.41, 65.71) --
	( 41.41, 65.71) --
	( 41.33, 65.71) --
	( 41.33, 65.71) --
	( 41.33, 65.71) --
	( 41.29, 65.71) --
	( 41.29, 65.71) --
	( 41.29, 65.71) --
	( 41.26, 65.71) --
	( 41.26, 65.71) --
	( 41.26, 65.71) --
	( 41.19, 65.71) --
	( 41.19, 65.71) --
	( 41.19, 65.71) --
	( 41.11, 65.71) --
	( 41.11, 65.71) --
	( 41.11, 65.71) --
	( 41.04, 65.71) --
	( 41.04, 65.71) --
	( 41.04, 65.71) --
	( 40.96, 65.71) --
	( 40.96, 65.71) --
	( 40.96, 65.71) --
	( 40.92, 65.71) --
	( 40.92, 65.71) --
	( 40.92, 65.71) --
	( 40.89, 65.71) --
	( 40.89, 65.71) --
	( 40.89, 65.71) --
	( 40.81, 65.71) --
	( 40.81, 65.71) --
	( 40.81, 65.71) --
	( 40.74, 65.71) --
	( 40.74, 65.71) --
	( 40.74, 65.71) --
	( 40.67, 65.71) --
	( 40.67, 65.71) --
	( 40.67, 65.71) --
	( 40.59, 65.71) --
	( 40.59, 65.71) --
	( 40.59, 65.71) --
	( 40.52, 65.71) --
	( 40.52, 65.71) --
	( 40.52, 65.71) --
	( 40.52, 65.71) --
	( 40.52, 65.71) --
	( 40.52, 65.71) --
	( 40.44, 65.71) --
	( 40.44, 65.71) --
	( 40.44, 65.71) --
	( 40.37, 65.71) --
	( 40.37, 65.71) --
	( 40.37, 65.71) --
	( 40.35, 65.71) --
	( 40.35, 65.71) --
	( 40.35, 65.71) --
	( 40.29, 65.71) --
	( 40.29, 65.71) --
	( 40.29, 65.71) --
	( 40.22, 65.71) --
	( 40.22, 65.71) --
	( 40.22, 65.71) --
	( 40.14, 65.71) --
	( 40.14, 65.71) --
	( 40.14, 65.71) --
	( 40.07, 65.71) --
	( 40.07, 65.71) --
	( 40.07, 65.71) --
	( 40.03, 65.71) --
	( 40.03, 65.71) --
	( 40.03, 65.71) --
	( 40.00, 65.71) --
	( 40.00, 65.71) --
	( 40.00, 65.71) --
	( 39.92, 65.71) --
	( 39.92, 65.71) --
	( 39.92, 65.71) --
	( 39.91, 65.71) --
	( 39.91, 65.71) --
	( 39.91, 65.71) --
	( 39.87, 65.71) --
	( 39.87, 65.71) --
	( 39.87, 65.71) --
	( 39.85, 65.71) --
	( 39.85, 65.71) --
	( 39.85, 65.71) --
	( 39.83, 65.71) --
	( 39.83, 65.71) --
	( 39.83, 65.71) --
	( 39.79, 65.71) --
	( 39.79, 65.71) --
	( 39.79, 65.71) --
	( 39.77, 65.71) --
	( 39.77, 65.71) --
	( 39.77, 65.71) --
	( 39.71, 65.71) --
	( 39.71, 65.71) --
	( 39.71, 65.71) --
	( 39.70, 65.71) --
	( 39.70, 65.71) --
	( 39.70, 65.71) --
	( 39.67, 65.71) --
	( 39.67, 65.71) --
	( 39.67, 65.71) --
	( 39.62, 65.71) --
	( 39.62, 65.71) --
	( 39.62, 65.71) --
	( 39.55, 65.71) --
	( 39.55, 65.71) --
	( 39.55, 65.71) --
	( 39.51, 65.71) --
	( 39.51, 65.71) --
	( 39.51, 65.71) --
	( 39.47, 65.71) --
	( 39.47, 65.71) --
	( 39.47, 65.71) --
	( 39.47, 65.71) --
	( 39.47, 65.71) --
	( 39.47, 65.71) --
	( 39.40, 65.71) --
	( 39.40, 65.71) --
	( 39.40, 65.71) --
	( 39.33, 65.71) --
	( 39.33, 65.71) --
	( 39.33, 65.71) --
	( 39.25, 65.71) --
	( 39.25, 65.71) --
	( 39.25, 65.71) --
	( 39.18, 65.71) --
	( 39.18, 65.71) --
	( 39.18, 65.71) --
	( 39.14, 65.71) --
	( 39.14, 65.71) --
	( 39.14, 65.71) --
	( 39.10, 65.71) --
	( 39.10, 65.71) --
	( 39.10, 65.71) --
	( 39.03, 65.71) --
	( 39.03, 65.71) --
	( 39.03, 65.71) --
	( 38.99, 65.71) --
	( 38.99, 65.71) --
	( 38.99, 65.71) --
	( 38.95, 65.71) --
	( 38.95, 65.71) --
	( 38.95, 65.71) --
	( 38.88, 65.71) --
	( 38.88, 65.71) --
	( 38.88, 65.71) --
	( 38.82, 65.71) --
	( 38.82, 65.71) --
	( 38.82, 65.71) --
	( 38.80, 65.71) --
	( 38.80, 65.71) --
	( 38.80, 65.71) --
	( 38.73, 65.71) --
	( 38.73, 65.71) --
	( 38.73, 65.71) --
	( 38.65, 65.71) --
	( 38.65, 65.71) --
	( 38.65, 65.71) --
	( 38.58, 65.71) --
	( 38.58, 65.71) --
	( 38.58, 65.71) --
	( 38.51, 65.71) --
	( 38.51, 65.71) --
	( 38.51, 65.71) --
	( 38.43, 65.71) --
	( 38.43, 65.71) --
	( 38.43, 65.71) --
	( 38.36, 65.71) --
	( 38.36, 65.71) --
	( 38.36, 65.71) --
	( 38.33, 65.71) --
	( 38.33, 65.71) --
	( 38.33, 65.71) --
	( 38.28, 65.71) --
	( 38.28, 65.71) --
	( 38.28, 65.71) --
	( 38.21, 65.71) --
	( 38.21, 65.71) --
	( 38.21, 65.71) --
	( 38.13, 65.71) --
	( 38.13, 65.71) --
	( 38.13, 65.71) --
	( 38.06, 65.71) --
	( 38.06, 65.71) --
	( 38.06, 65.71) --
	( 37.98, 65.71) --
	( 37.98, 65.71) --
	( 37.98, 65.71) --
	( 37.97, 65.71) --
	( 37.97, 65.71) --
	( 37.97, 65.71) --
	( 37.91, 65.71) --
	( 37.91, 65.71) --
	( 37.91, 65.71) --
	( 37.83, 65.71) --
	( 37.83, 65.71) --
	( 37.83, 65.71) --
	( 37.76, 65.71) --
	( 37.76, 65.71) --
	( 37.76, 65.71) --
	( 37.69, 65.71) --
	( 37.69, 65.71) --
	( 37.69, 65.71) --
	( 37.65, 65.71) --
	( 37.65, 65.71) --
	( 37.65, 65.71) --
	( 37.61, 65.71) --
	( 37.61, 65.71) --
	( 37.61, 65.71) --
	( 37.61, 65.71) --
	( 37.61, 65.71) --
	( 37.61, 65.71) --
	( 37.54, 65.71) --
	( 37.54, 65.71) --
	( 37.54, 65.71) --
	( 37.46, 65.71) --
	( 37.46, 65.71) --
	( 37.46, 65.71) --
	( 37.39, 65.71) --
	( 37.39, 65.71) --
	( 37.39, 65.71) --
	( 37.32, 65.71) --
	( 37.32, 65.71) --
	( 37.32, 65.71) --
	( 37.31, 65.71) --
	( 37.31, 65.71) --
	( 37.31, 65.71) --
	( 37.24, 65.71) --
	( 37.24, 65.71) --
	( 37.24, 65.71) --
	( 37.16, 65.71) --
	( 37.16, 65.71) --
	( 37.16, 65.71) --
	( 37.16, 65.71) --
	( 37.16, 65.71) --
	( 37.16, 65.71) --
	( 37.09, 65.71) --
	( 37.09, 65.71) --
	( 37.09, 65.71) --
	( 37.01, 65.71) --
	( 37.01, 65.71) --
	( 37.01, 65.71) --
	( 36.94, 65.71) --
	( 36.94, 65.71) --
	( 36.94, 65.71) --
	( 36.92, 65.71) --
	( 36.92, 65.71) --
	( 36.92, 65.71) --
	( 36.86, 65.71) --
	( 36.86, 65.71) --
	( 36.86, 65.71) --
	( 36.79, 65.71) --
	( 36.79, 65.71) --
	( 36.79, 65.71) --
	( 36.76, 65.71) --
	( 36.76, 65.71) --
	( 36.76, 65.71) --
	( 36.71, 65.71) --
	( 36.71, 65.71) --
	( 36.71, 65.71) --
	( 36.64, 65.71) --
	( 36.64, 65.71) --
	( 36.64, 65.71) --
	( 36.57, 65.71) --
	( 36.57, 65.71) --
	( 36.57, 65.71) --
	( 36.49, 65.71) --
	( 36.49, 65.71) --
	( 36.49, 65.71) --
	( 36.43, 65.71) --
	( 36.43, 65.71) --
	( 36.43, 65.71) --
	( 36.42, 65.71) --
	( 36.42, 65.71) --
	( 36.42, 65.71) --
	( 36.34, 65.71) --
	( 36.34, 65.71) --
	( 36.34, 65.71) --
	( 36.27, 65.71) --
	( 36.27, 65.71) --
	( 36.27, 65.71) --
	( 36.23, 65.71) --
	( 36.23, 65.71) --
	( 36.23, 65.71) --
	( 36.19, 65.71) --
	( 36.19, 65.71) --
	( 36.19, 65.71) --
	( 36.15, 65.71) --
	( 36.15, 65.71) --
	( 36.15, 65.71) --
	( 36.12, 65.71) --
	( 36.12, 65.71) --
	( 36.12, 65.71) --
	( 36.11, 65.71) --
	( 36.11, 65.71) --
	( 36.11, 65.71) --
	( 36.04, 65.71) --
	( 36.04, 65.71) --
	( 36.04, 65.71) --
	( 35.97, 65.71) --
	( 35.97, 65.71) --
	( 35.97, 65.71) --
	( 35.89, 65.71) --
	( 35.89, 65.71) --
	( 35.89, 65.71) --
	( 35.87, 65.71) --
	( 35.87, 65.71) --
	( 35.87, 65.71) --
	( 35.83, 65.71) --
	( 35.83, 65.71) --
	( 35.83, 65.71) --
	( 35.82, 65.71) --
	( 35.82, 65.71) --
	( 35.82, 65.71) --
	( 35.82, 65.71) --
	( 35.82, 65.71) --
	( 35.82, 65.71) --
	( 35.74, 65.71) --
	( 35.74, 65.71) --
	( 35.74, 65.71) --
	( 35.71, 65.71) --
	( 35.71, 65.71) --
	( 35.71, 65.71) --
	( 35.67, 65.71) --
	( 35.67, 65.71) --
	( 35.67, 65.71) --
	( 35.60, 65.71) --
	( 35.60, 65.71) --
	( 35.60, 65.71) --
	( 35.58, 65.71) --
	( 35.58, 65.71) --
	( 35.58, 65.71) --
	( 35.52, 65.71) --
	( 35.52, 65.71) --
	( 35.52, 65.71) --
	( 35.50, 65.71) --
	( 35.50, 65.71) --
	( 35.50, 65.71) --
	( 35.45, 65.71) --
	( 35.45, 65.71) --
	( 35.45, 65.71) --
	( 35.38, 65.71) --
	( 35.38, 65.71) --
	( 35.38, 65.71) --
	( 35.37, 65.71) --
	( 35.37, 65.71) --
	( 35.37, 65.71) --
	( 35.34, 65.71) --
	( 35.34, 65.71) --
	( 35.34, 65.71) --
	( 35.30, 65.71) --
	( 35.30, 65.71) --
	( 35.30, 65.71) --
	( 35.22, 65.71) --
	( 35.22, 65.71) --
	( 35.22, 65.71) --
	( 35.18, 65.71) --
	( 35.18, 65.71) --
	( 35.18, 65.71) --
	( 35.15, 65.71) --
	( 35.15, 65.71) --
	( 35.15, 65.71) --
	( 35.07, 65.71) --
	( 35.07, 65.71) --
	( 35.07, 65.71) --
	( 35.06, 65.71) --
	( 35.06, 65.71) --
	( 35.06, 65.71) --
	( 35.00, 65.71) --
	( 35.00, 65.71) --
	( 35.00, 65.71) --
	( 34.98, 65.71) --
	( 34.98, 65.71) --
	( 34.98, 65.71) --
	( 34.92, 65.71) --
	( 34.92, 65.71) --
	( 34.92, 65.71) --
	( 34.87, 65.71) --
	( 34.87, 65.71) --
	( 34.87, 65.71) --
	( 34.86, 65.71) --
	( 34.86, 65.71) --
	( 34.86, 65.71) --
	( 34.85, 65.71) --
	( 34.85, 65.71) --
	( 34.85, 65.71) --
	( 34.82, 65.71) --
	( 34.82, 65.71) --
	( 34.82, 65.71) --
	( 34.77, 65.71) --
	( 34.77, 65.71) --
	( 34.77, 65.71) --
	( 34.74, 65.71) --
	( 34.74, 65.71) --
	( 34.74, 65.71) --
	( 34.70, 65.71) --
	( 34.70, 65.71) --
	( 34.70, 65.71) --
	( 34.70, 65.71) --
	( 34.70, 65.71) --
	( 34.70, 65.71) --
	( 34.62, 65.71) --
	( 34.62, 65.71) --
	( 34.62, 65.71) --
	( 34.61, 65.71) --
	( 34.61, 65.71) --
	( 34.61, 65.71) --
	( 34.55, 65.71) --
	( 34.55, 65.71) --
	( 34.55, 65.71) --
	( 34.53, 65.71) --
	( 34.53, 65.71) --
	( 34.53, 65.71) --
	( 34.47, 65.71) --
	( 34.47, 65.71) --
	( 34.47, 65.71) --
	( 34.41, 65.71) --
	( 34.41, 65.71) --
	( 34.41, 65.71) --
	( 34.40, 65.71) --
	( 34.40, 65.71) --
	( 34.40, 65.71) --
	( 34.32, 65.71) --
	( 34.32, 65.71) --
	( 34.32, 65.71) --
	( 34.25, 65.71) --
	( 34.25, 65.71) --
	( 34.25, 65.71) --
	( 34.25, 65.71) --
	( 34.25, 65.71) --
	( 34.25, 65.71) --
	( 34.21, 65.71) --
	( 34.21, 65.71) --
	( 34.21, 65.71) --
	( 34.17, 65.71) --
	( 34.17, 65.71) --
	( 34.17, 65.71) --
	( 34.10, 65.71) --
	( 34.10, 65.71) --
	( 34.10, 65.71) --
	( 34.09, 65.71) --
	( 34.09, 65.71) --
	( 34.09, 65.71) --
	( 34.05, 65.71) --
	( 34.05, 65.71) --
	( 34.05, 65.71) --
	( 34.03, 65.71) --
	( 34.03, 65.71) --
	( 34.03, 65.71) --
	( 33.95, 65.71) --
	( 33.95, 65.71) --
	( 33.95, 65.71) --
	( 33.89, 65.71) --
	( 33.89, 65.71) --
	( 33.89, 65.71) --
	( 33.88, 65.71) --
	( 33.88, 65.71) --
	( 33.88, 65.71) --
	( 33.85, 65.71) --
	( 33.85, 65.71) --
	( 33.85, 65.71) --
	( 33.80, 65.71) --
	( 33.80, 65.71) --
	( 33.80, 65.71) --
	( 33.73, 65.71) --
	( 33.73, 65.71) --
	( 33.73, 65.71) --
	( 33.65, 65.71) --
	( 33.65, 65.71) --
	( 33.65, 65.71) --
	( 33.64, 65.71) --
	( 33.64, 65.71) --
	( 33.64, 65.71) --
	( 33.58, 65.71) --
	( 33.58, 65.71) --
	( 33.58, 65.71) --
	( 33.50, 65.71) --
	( 33.50, 65.71) --
	( 33.50, 65.71) --
	( 33.44, 65.71) --
	( 33.44, 65.71) --
	( 33.44, 65.71) --
	( 33.43, 65.71) --
	( 33.43, 65.71) --
	( 33.43, 65.71) --
	( 33.35, 65.71) --
	( 33.35, 65.71) --
	( 33.35, 65.71) --
	( 33.28, 65.71) --
	( 33.28, 65.71) --
	( 33.28, 65.71) --
	( 33.28, 65.71) --
	( 33.28, 65.71) --
	( 33.28, 65.71) --
	( 33.20, 65.71) --
	( 33.20, 65.71) --
	( 33.20, 65.71) --
	( 33.20, 65.71) --
	( 33.20, 65.71) --
	( 33.20, 65.71) --
	( 33.16, 65.71) --
	( 33.16, 65.71) --
	( 33.16, 65.71) --
	( 33.13, 65.71) --
	( 33.13, 65.71) --
	( 33.13, 65.71) --
	( 33.12, 65.71) --
	( 33.12, 65.71) --
	( 33.12, 65.71) --
	( 33.08, 65.71) --
	( 33.08, 65.71) --
	( 33.08, 65.71) --
	( 33.05, 65.71) --
	( 33.05, 65.71) --
	( 33.05, 65.71) --
	( 33.00, 65.71) --
	( 33.00, 65.71) --
	( 33.00, 65.71) --
	( 32.98, 65.71) --
	( 32.98, 65.71) --
	( 32.98, 65.71) --
	( 32.96, 65.71) --
	( 32.96, 65.71) --
	( 32.96, 65.71) --
	( 32.90, 65.71) --
	( 32.90, 65.71) --
	( 32.90, 65.71) --
	( 32.83, 65.71) --
	( 32.83, 65.71) --
	( 32.83, 65.71) --
	( 32.75, 65.71) --
	( 32.75, 65.71) --
	( 32.75, 65.71) --
	( 32.71, 65.71) --
	( 32.71, 65.71) --
	( 32.71, 65.71) --
	( 32.68, 65.71) --
	( 32.68, 65.71) --
	( 32.68, 65.71) --
	( 32.67, 65.71) --
	( 32.67, 65.71) --
	( 32.67, 65.71) --
	( 32.60, 65.71) --
	( 32.60, 65.71) --
	( 32.60, 65.71) --
	( 32.59, 65.71) --
	( 32.59, 65.71) --
	( 32.59, 65.71) --
	( 32.55, 65.71) --
	( 32.55, 65.71) --
	( 32.55, 65.71) --
	( 32.53, 65.71) --
	( 32.53, 65.71) --
	( 32.53, 65.71) --
	( 32.47, 65.71) --
	( 32.47, 65.71) --
	( 32.47, 65.71) --
	( 32.47, 65.71) --
	( 32.47, 65.71) --
	( 32.47, 65.71) --
	( 32.45, 65.71) --
	( 32.45, 65.71) --
	( 32.45, 65.71) --
	( 32.43, 65.71) --
	( 32.43, 65.71) --
	( 32.43, 65.71) --
	( 32.39, 65.71) --
	( 32.39, 65.71) --
	( 32.39, 65.71) --
	( 32.38, 65.71) --
	( 32.38, 65.71) --
	( 32.38, 65.71) --
	( 32.30, 65.71) --
	( 32.30, 65.71) --
	( 32.30, 65.71) --
	( 32.27, 65.71) --
	( 32.27, 65.71) --
	( 32.27, 65.71) --
	( 32.23, 65.71) --
	( 32.23, 65.71) --
	( 32.23, 65.71) --
	( 32.23, 65.71) --
	( 32.23, 65.71) --
	( 32.23, 65.71) --
	( 32.15, 65.71) --
	( 32.15, 65.71) --
	( 32.15, 65.71) --
	( 32.11, 65.71) --
	( 32.11, 65.71) --
	( 32.11, 65.71) --
	( 32.08, 65.71) --
	( 32.08, 65.71) --
	( 32.08, 65.71) --
	( 32.00, 65.71) --
	( 32.00, 65.71) --
	( 32.00, 65.71) --
	( 31.95, 65.71) --
	( 31.95, 65.71) --
	( 31.95, 65.71) --
	( 31.93, 65.71) --
	( 31.93, 65.71) --
	( 31.93, 65.71) --
	( 31.87, 65.71) --
	( 31.87, 65.71) --
	( 31.87, 65.71) --
	( 31.85, 65.71) --
	( 31.85, 65.71) --
	( 31.85, 65.71) --
	( 31.78, 65.71) --
	( 31.78, 65.71) --
	( 31.78, 65.71) --
	( 31.74, 65.71) --
	( 31.74, 65.71) --
	( 31.74, 65.71) --
	( 31.71, 65.71) --
	( 31.71, 65.71) --
	( 31.71, 65.71) --
	( 31.70, 65.71) --
	( 31.70, 65.71) --
	( 31.70, 65.71) --
	( 31.63, 65.71) --
	( 31.63, 65.71) --
	( 31.63, 65.71) --
	( 31.58, 65.71) --
	( 31.58, 65.71) --
	( 31.58, 65.71) --
	( 31.56, 65.71) --
	( 31.56, 65.71) --
	( 31.56, 65.71) --
	( 31.48, 65.71) --
	( 31.48, 65.71) --
	( 31.48, 65.71) --
	( 31.46, 65.71) --
	( 31.46, 65.71) --
	( 31.46, 65.71) --
	( 31.42, 65.71) --
	( 31.42, 65.71) --
	( 31.42, 65.71) --
	( 31.41, 65.71) --
	( 31.41, 65.71) --
	( 31.41, 65.71) --
	( 31.33, 65.71) --
	( 31.33, 65.71) --
	( 31.33, 65.71) --
	( 31.30, 65.71) --
	( 31.30, 65.71) --
	( 31.30, 65.71) --
	( 31.26, 65.71) --
	( 31.26, 65.71) --
	( 31.26, 65.71) --
	( 31.18, 65.71) --
	( 31.18, 65.71) --
	( 31.18, 65.71) --
	( 31.14, 65.71) --
	( 31.14, 65.71) --
	( 31.14, 65.71) --
	( 31.10, 65.71) --
	( 31.10, 65.71) --
	( 31.10, 65.71) --
	( 31.03, 65.71) --
	( 31.03, 65.71) --
	( 31.03, 65.71) --
	( 30.96, 65.71) --
	( 30.96, 65.71) --
	( 30.96, 65.71) --
	( 30.94, 65.71) --
	( 30.94, 65.71) --
	( 30.94, 65.71) --
	( 30.88, 65.71) --
	( 30.88, 65.71) --
	( 30.88, 65.71) --
	( 30.81, 65.71) --
	( 30.81, 65.71) --
	( 30.81, 65.71) --
	( 30.73, 65.71) --
	( 30.73, 65.71) --
	( 30.73, 65.71) --
	( 30.73, 65.71) --
	( 30.73, 65.71) --
	( 30.73, 65.71) --
	( 30.66, 65.71) --
	( 30.66, 65.71) --
	( 30.66, 65.71) --
	( 30.58, 65.71) --
	( 30.58, 65.71) --
	( 30.58, 65.71) --
	( 30.51, 65.71) --
	( 30.51, 65.71) --
	( 30.51, 65.71) --
	( 30.43, 65.71) --
	( 30.43, 65.71) --
	( 30.43, 65.71) --
	( 30.41, 65.71) --
	( 30.41, 65.71) --
	( 30.41, 65.71) --
	( 30.36, 65.71) --
	( 30.36, 65.71) --
	( 30.36, 65.71) --
	( 30.28, 65.71) --
	( 30.28, 65.71) --
	( 30.28, 65.71) --
	( 30.21, 65.71) --
	( 30.21, 65.71) --
	( 30.21, 65.71) --
	( 30.13, 65.71) --
	( 30.13, 65.71) --
	( 30.13, 65.71) --
	( 30.06, 65.71) --
	( 30.06, 65.71) --
	( 30.06, 65.71) --
	( 29.98, 65.71) --
	( 29.98, 65.71) --
	( 29.98, 65.71) --
	( 29.93, 65.71) --
	( 29.93, 65.71) --
	( 29.93, 65.71) --
	( 29.91, 65.71) --
	( 29.91, 65.71) --
	( 29.91, 65.71) --
	( 29.83, 65.71) --
	( 29.83, 65.71) --
	( 29.83, 65.71) --
	( 29.76, 65.71) --
	( 29.76, 65.71) --
	( 29.76, 65.71) --
	( 29.68, 65.71) --
	( 29.68, 65.71) --
	( 29.68, 65.71) --
	( 29.61, 65.71) --
	( 29.61, 65.71) --
	( 29.61, 65.71) --
	( 29.53, 65.71) --
	( 29.53, 65.71) --
	( 29.53, 65.71) --
	( 29.46, 65.71) --
	( 29.46, 65.71) --
	( 29.46, 65.71) --
	( 29.40, 65.71) --
	( 29.40, 65.71) --
	( 29.40, 65.71) --
	( 29.38, 65.71) --
	( 29.38, 65.71) --
	( 29.38, 65.71) --
	( 29.31, 65.71) --
	( 29.31, 65.71) --
	( 29.31, 65.71) --
	( 29.23, 65.71) --
	( 29.23, 65.71) --
	( 29.23, 65.71) --
	( 29.20, 65.71) --
	( 29.20, 65.71) --
	( 29.20, 65.71) --
	( 29.16, 65.71) --
	( 29.16, 65.71) --
	( 29.16, 65.71) --
	( 29.08, 65.71) --
	( 29.08, 65.71) --
	( 29.08, 65.71) --
	( 29.01, 65.71) --
	( 29.01, 65.71) --
	( 29.01, 65.71) --
	( 28.93, 65.71) --
	( 28.93, 65.71) --
	( 28.93, 65.71) --
	( 28.86, 65.71) --
	( 28.86, 65.71) --
	( 28.86, 65.71) --
	( 28.78, 65.71) --
	( 28.78, 65.71) --
	( 28.78, 65.71) --
	( 28.71, 65.71) --
	( 28.71, 65.71) --
	( 28.71, 65.71) --
	( 28.63, 65.71) --
	( 28.63, 65.71) --
	( 28.63, 65.71) --
	( 28.56, 65.71) --
	( 28.56, 65.71) --
	( 28.56, 65.71) --
	( 28.48, 65.71) --
	( 28.48, 65.71) --
	( 28.48, 65.71) --
	( 28.40, 65.71) --
	( 28.40, 65.71) --
	( 28.40, 65.71) --
	( 28.39, 65.71) --
	( 28.39, 65.71) --
	( 28.39, 65.71) --
	( 28.33, 65.71) --
	( 28.33, 65.71) --
	( 28.33, 65.71) --
	( 28.25, 65.71) --
	( 28.25, 65.71) --
	( 28.25, 65.71) --
	( 28.19, 65.71) --
	( 28.19, 65.71) --
	( 28.19, 65.71) --
	( 28.18, 65.71) --
	( 28.18, 65.71) --
	( 28.18, 65.71) --
	( 28.10, 65.71) --
	( 28.10, 65.71) --
	( 28.10, 65.71) --
	( 28.03, 65.71) --
	( 28.03, 65.71) --
	( 28.03, 65.71) --
	( 28.03, 65.71) --
	( 28.03, 65.71) --
	( 28.03, 65.71) --
	( 27.95, 65.71) --
	( 27.95, 65.71) --
	( 27.95, 65.71) --
	( 27.88, 65.71) --
	( 27.88, 65.71) --
	( 27.88, 65.71) --
	( 27.80, 65.71) --
	( 27.80, 65.71) --
	( 27.80, 65.71) --
	( 27.73, 65.71) --
	( 27.73, 65.71) --
	( 27.73, 65.71) --
	( 27.66, 65.71) --
	( 27.66, 65.71) --
	( 27.66, 65.71) --
	( 27.65, 65.71) --
	( 27.65, 65.71) --
	( 27.65, 65.71) --
	( 27.58, 65.71) --
	( 27.58, 65.71) --
	( 27.58, 65.71) --
	( 27.50, 65.71) --
	( 27.50, 65.71) --
	( 27.50, 65.71) --
	( 27.43, 65.71) --
	( 27.43, 65.71) --
	( 27.43, 65.71) --
	( 27.39, 65.71) --
	( 27.39, 65.71) --
	( 27.39, 65.71) --
	( 27.35, 65.71) --
	( 27.35, 65.71) --
	( 27.35, 65.71) --
	( 27.28, 65.71) --
	( 27.28, 65.71) --
	( 27.28, 65.71) --
	( 27.20, 65.71) --
	( 27.20, 65.71) --
	( 27.20, 65.71) --
	( 27.13, 65.71) --
	( 27.13, 65.71) --
	( 27.13, 65.71) --
	( 27.06, 65.71) --
	( 27.06, 65.71) --
	( 27.06, 65.71) --
	( 27.05, 65.71) --
	( 27.05, 65.71) --
	( 27.05, 65.71) --
	( 26.98, 65.71) --
	( 26.98, 65.71) --
	( 26.98, 65.71) --
	( 26.90, 65.71) --
	( 26.90, 65.71) --
	( 26.90, 65.71) --
	( 26.83, 65.71) --
	( 26.83, 65.71) --
	( 26.83, 65.71) --
	( 26.75, 65.71) --
	( 26.75, 65.71) --
	( 26.75, 65.71) --
	( 26.68, 65.71) --
	( 26.68, 65.71) --
	( 26.68, 65.71) --
	( 26.60, 65.71) --
	( 26.60, 65.71) --
	( 26.60, 65.71) --
	( 26.53, 65.71) --
	( 26.53, 65.71) --
	( 26.53, 65.71) --
	( 26.45, 65.71) --
	( 26.45, 65.71) --
	( 26.45, 65.71) --
	( 26.38, 65.71) --
	( 26.38, 65.71) --
	( 26.38, 65.71) --
	( 26.30, 65.71) --
	( 26.30, 65.71) --
	( 26.30, 65.71) --
	( 26.22, 65.71) --
	( 26.22, 65.71) --
	( 26.22, 65.71) --
	( 26.15, 65.71) --
	( 26.15, 65.71) --
	( 26.15, 65.71) --
	( 26.07, 65.71) --
	( 26.07, 65.71) --
	( 26.07, 65.71) --
	( 26.00, 65.71) --
	( 26.00, 65.71) --
	( 26.00, 65.71) --
	( 25.92, 65.71) --
	( 25.92, 65.71) --
	( 25.92, 65.71) --
	( 25.85, 65.71) --
	( 25.85, 65.71) --
	( 25.85, 65.71) --
	( 25.80, 65.71) --
	( 25.80, 65.71) --
	( 25.80, 65.71) --
	( 25.77, 65.71) --
	( 25.77, 65.71) --
	( 25.77, 65.71) --
	( 25.70, 65.71) --
	( 25.70, 65.71) --
	( 25.70, 65.71) --
	( 25.62, 65.71) --
	( 25.62, 65.71) --
	( 25.62, 65.71) --
	( 25.55, 65.71) --
	( 25.55, 65.71) --
	( 25.55, 65.71) --
	( 25.47, 65.71) --
	( 25.47, 65.71) --
	( 25.47, 65.71) --
	( 25.40, 65.71) --
	( 25.40, 65.71) --
	( 25.40, 65.71) --
	( 25.32, 65.71) --
	( 25.32, 65.71) --
	( 25.32, 65.71) --
	( 25.25, 65.71) --
	( 25.25, 65.71) --
	( 25.25, 65.71) --
	( 25.17, 65.71) --
	( 25.17, 65.71) --
	( 25.17, 65.71) --
	( 25.12, 65.71) --
	( 25.12, 65.71) --
	( 25.12, 65.71) --
	( 25.10, 65.71) --
	( 25.10, 65.71) --
	( 25.10, 65.71) --
	( 25.02, 65.71) --
	( 25.02, 65.71) --
	( 25.02, 65.71) --
	( 24.95, 65.71) --
	( 24.95, 65.71) --
	( 24.95, 65.71) --
	( 24.87, 65.71) --
	( 24.87, 65.71) --
	( 24.87, 65.71) --
	( 24.83, 65.71) --
	( 24.83, 65.71) --
	( 24.83, 65.71) --
	( 24.80, 65.71) --
	( 24.80, 65.71) --
	( 24.80, 65.71) --
	( 24.75, 65.71) --
	( 24.75, 65.71) --
	( 24.75, 65.71) --
	( 24.72, 65.71) --
	( 24.72, 65.71) --
	( 24.72, 65.71) --
	( 24.65, 65.71) --
	( 24.65, 65.71) --
	( 24.65, 65.71) --
	( 24.63, 65.71) --
	( 24.63, 65.71) --
	( 24.63, 65.71) --
	( 24.57, 65.71) --
	( 24.57, 65.71) --
	( 24.57, 65.71) --
	( 24.55, 65.71) --
	( 24.55, 65.71) --
	( 24.55, 65.71) --
	( 24.51, 65.71) --
	( 24.51, 65.71) --
	( 24.51, 65.71) --
	( 24.50, 65.71) --
	( 24.50, 65.71) --
	( 24.50, 65.71) --
	( 24.42, 65.71) --
	( 24.42, 65.71) --
	( 24.42, 65.71) --
	( 24.42, 65.71) --
	( 24.42, 65.71) --
	( 24.42, 65.71) --
	( 24.34, 65.71) --
	( 24.34, 65.71) --
	( 24.34, 65.71) --
	( 24.27, 65.71) --
	( 24.27, 65.71) --
	( 24.27, 65.71) --
	( 24.27, 65.71) --
	( 24.27, 65.71) --
	( 24.27, 65.71) --
	( 24.19, 65.71) --
	( 24.19, 65.71) --
	( 24.19, 65.71) --
	( 24.12, 65.71) --
	( 24.12, 65.71) --
	( 24.12, 65.71) --
	( 24.04, 65.71) --
	( 24.04, 65.71) --
	( 24.04, 65.71) --
	( 23.97, 65.71) --
	( 23.97, 65.71) --
	( 23.97, 65.71) --
	( 23.89, 65.71) --
	( 23.89, 65.71) --
	( 23.89, 65.71) --
	( 23.82, 65.71) --
	( 23.82, 65.71) --
	( 23.82, 65.71) --
	( 23.74, 65.71) --
	( 23.74, 65.71) --
	( 23.74, 65.71) --
	( 23.67, 65.71) --
	( 23.67, 65.71) --
	( 23.67, 65.71) --
	( 23.59, 65.71) --
	( 23.59, 65.71) --
	( 23.59, 65.71) --
	( 23.52, 65.71) --
	( 23.52, 65.71) --
	( 23.52, 65.71) --
	( 23.44, 65.71) --
	( 23.44, 65.71) --
	( 23.44, 65.71) --
	( 23.37, 65.71) --
	( 23.37, 65.71) --
	( 23.37, 65.71) --
	( 23.29, 65.71) --
	( 23.29, 65.71) --
	( 23.29, 65.71) --
	( 23.22, 65.71) --
	( 23.22, 65.71) --
	( 23.21, 65.71) --
	( 23.14, 65.71) --
	( 23.14, 65.71) --
	( 23.14, 65.71) --
	( 23.06, 65.71) --
	( 23.06, 65.71) --
	( 23.06, 65.71) --
	( 22.99, 65.71) --
	( 22.99, 65.71) --
	( 22.99, 65.71) --
	( 22.91, 65.71) --
	( 22.91, 65.71) --
	( 22.91, 65.71) --
	( 22.84, 65.71) --
	( 22.84, 65.71) --
	( 22.84, 65.71) --
	( 22.76, 65.71) --
	( 22.76, 65.71) --
	( 22.76, 65.71) --
	( 22.69, 65.71) --
	( 22.69, 65.71) --
	( 22.69, 65.71) --
	( 22.61, 65.71) --
	( 22.61, 65.71) --
	( 22.61, 65.71) --
	( 22.54, 65.71) --
	( 22.54, 65.71) --
	( 22.54, 65.71) --
	( 22.46, 65.71) --
	( 22.46, 65.71) --
	( 22.46, 65.71) --
	( 22.39, 65.71) --
	( 22.39, 65.71) --
	( 22.38, 65.71) --
	( 22.31, 65.71) --
	( 22.31, 65.71) --
	( 22.31, 65.71) --
	( 22.23, 65.71) --
	( 22.23, 65.71) --
	( 22.23, 65.71) --
	( 22.16, 65.71) --
	( 22.16, 65.71) --
	( 22.16, 65.71) --
	( 22.08, 65.71) --
	( 22.08, 65.71) --
	( 22.08, 65.71) --
	( 22.01, 65.71) --
	( 22.01, 65.71) --
	( 22.01, 65.71) --
	( 21.93, 65.71) --
	( 21.93, 65.71) --
	( 21.93, 65.71) --
	( 21.86, 65.71) --
	( 21.86, 65.71) --
	( 21.86, 65.71) --
	( 21.78, 65.71) --
	( 21.78, 65.71) --
	( 21.78, 65.71) --
	( 21.71, 65.71) --
	( 21.71, 65.71) --
	( 21.71, 65.71) --
	( 21.63, 65.71) --
	( 21.63, 65.71) --
	( 21.63, 65.71) --
	( 21.56, 65.71) --
	( 21.56, 65.71) --
	( 21.55, 65.71) --
	( 21.48, 65.71) --
	( 21.48, 65.71) --
	( 21.48, 65.71) --
	( 21.40, 65.71) --
	( 21.40, 65.71) --
	( 21.40, 65.71) --
	( 21.33, 65.71) --
	( 21.33, 65.71) --
	( 21.33, 65.71) --
	( 21.25, 65.71) --
	( 21.25, 65.71) --
	( 21.25, 65.71) --
	( 21.18, 65.71) --
	( 21.18, 65.71) --
	( 21.18, 65.71) --
	( 21.10, 65.71) --
	( 21.10, 65.71) --
	( 21.10, 65.71) --
	( 21.03, 65.71) --
	( 21.03, 65.71) --
	( 21.03, 65.71) --
	( 20.95, 65.71) --
	( 20.95, 65.71) --
	( 20.95, 65.71) --
	( 20.88, 65.71) --
	( 20.88, 65.71) --
	( 20.88, 65.71) --
	( 20.80, 65.71) --
	( 20.80, 65.71) --
	( 20.80, 65.71) --
	( 20.73, 65.71) --
	( 20.72, 65.71) --
	( 20.72, 65.71) --
	( 20.65, 65.71) --
	( 20.65, 65.71) --
	( 20.65, 65.71) --
	( 20.57, 65.71) --
	( 20.57, 65.71) --
	( 20.57, 65.71) --
	( 20.50, 65.71) --
	( 20.50, 65.71) --
	( 20.50, 65.71) --
	( 20.42, 65.71) --
	( 20.42, 65.71) --
	( 20.42, 65.71) --
	( 20.35, 65.71) --
	( 20.35, 65.71) --
	( 20.35, 65.71) --
	( 20.27, 65.71) --
	( 20.27, 65.71) --
	( 20.27, 65.71) --
	( 20.20, 65.71) --
	( 20.20, 65.71) --
	( 20.20, 65.71) --
	( 20.12, 65.71) --
	( 20.12, 65.71) --
	( 20.12, 65.71) --
	( 20.05, 65.71) --
	( 20.05, 65.71) --
	( 20.05, 65.71) --
	( 19.97, 65.71) --
	( 19.97, 65.71) --
	( 19.97, 65.71) --
	( 19.90, 65.71) --
	( 19.89, 65.71) --
	( 19.89, 65.71) --
	( 19.82, 65.71) --
	( 19.82, 65.71) --
	( 19.82, 65.71) --
	( 19.74, 65.71) --
	( 19.74, 65.71) --
	( 19.74, 65.71) --
	( 19.67, 65.71) --
	( 19.67, 65.71) --
	( 19.67, 65.71) --
	( 19.59, 65.71) --
	( 19.59, 65.71) --
	( 19.59, 65.71) --
	( 19.52, 65.71) --
	( 19.52, 65.71) --
	( 19.52, 65.71) --
	( 19.44, 65.71) --
	( 19.44, 65.71) --
	( 19.44, 65.71) --
	( 19.37, 65.71) --
	( 19.37, 65.71) --
	( 19.37, 65.71) --
	( 19.29, 65.71) --
	( 19.29, 65.71) --
	( 19.29, 65.71) --
	( 19.22, 65.71) --
	( 19.22, 65.71) --
	( 19.22, 65.71) --
	( 19.14, 65.71) --
	( 19.14, 65.71) --
	( 19.14, 65.71) --
	( 19.06, 65.71) --
	( 19.06, 65.71) --
	( 19.06, 65.71) --
	( 18.99, 65.71) --
	( 18.99, 65.71) --
	( 18.99, 65.71) --
	( 18.91, 65.71) --
	( 18.91, 65.71) --
	( 18.91, 65.71) --
	( 18.84, 65.71) --
	( 18.84, 65.71) --
	( 18.84, 65.71) --
	( 18.76, 65.71) --
	( 18.76, 65.71) --
	( 18.76, 65.71) --
	( 18.69, 65.71) --
	( 18.69, 65.71) --
	( 18.69, 65.71) --
	( 18.61, 65.71) --
	( 18.61, 65.71) --
	( 18.61, 65.71) --
	( 18.53, 65.71) --
	( 18.53, 65.71) --
	( 18.53, 65.71) --
	( 18.46, 65.71) --
	( 18.46, 65.71) --
	( 18.46, 65.71) --
	( 18.38, 65.71) --
	( 18.38, 65.71) --
	( 18.38, 65.71) --
	( 18.31, 65.71) --
	( 18.31, 65.71) --
	( 18.31, 65.71) --
	( 18.23, 65.71) --
	( 18.23, 65.71) --
	( 18.23, 65.71) --
	( 18.16, 65.71) --
	( 18.16, 65.71) --
	( 18.16, 65.71) --
	( 18.08, 65.71) --
	( 18.08, 65.71) --
	( 18.08, 65.71) --
	( 18.01, 65.71) --
	( 18.01, 65.71) --
	( 18.01, 65.71) --
	( 17.93, 65.71) --
	( 17.93, 65.71) --
	( 17.93, 65.71) --
	( 17.85, 65.71) --
	( 17.85, 65.71) --
	( 17.85, 65.71) --
	( 17.78, 65.71) --
	( 17.78, 65.71) --
	( 17.78, 65.71) --
	( 17.70, 65.71) --
	( 17.70, 65.71) --
	( 17.70, 65.71) --
	( 17.63, 65.71) --
	( 17.63, 65.71) --
	( 17.63, 65.71) --
	( 17.55, 65.71) --
	( 17.55, 65.71) --
	( 17.55, 65.71) --
	( 17.48, 65.71) --
	( 17.48, 65.71) --
	( 17.48, 65.71) --
	( 17.40, 65.71) --
	( 17.40, 65.71) --
	( 17.40, 65.71) --
	( 17.32, 65.71) --
	( 17.32, 65.71) --
	( 17.32, 65.71) --
	( 17.25, 65.71) --
	( 17.25, 65.71) --
	( 17.25, 65.71) --
	( 17.17, 65.71) --
	( 17.17, 65.71) --
	( 17.17, 65.71) --
	( 17.10, 65.71) --
	( 17.10, 65.71) --
	( 17.10, 65.71) --
	( 17.02, 65.71) --
	( 17.02, 65.71) --
	( 17.02, 65.71) --
	( 16.95, 65.71) --
	( 16.95, 65.71) --
	( 16.95, 65.71) --
	( 16.87, 65.71) --
	( 16.87, 65.71) --
	( 16.87, 65.71) --
	( 16.79, 65.71) --
	( 16.79, 65.71) --
	( 16.79, 65.71) --
	( 16.72, 65.71) --
	( 16.72, 65.71) --
	( 16.72, 65.71) --
	( 16.64, 65.71) --
	( 16.64, 65.71) --
	( 16.64, 65.71) --
	( 16.57, 65.71) --
	( 16.57, 65.71) --
	( 16.57, 65.71) --
	( 16.49, 65.71) --
	( 16.49, 65.71) --
	( 16.49, 65.71) --
	( 16.42, 65.71) --
	( 16.42, 65.71) --
	( 16.42, 65.71) --
	( 16.34, 65.71) --
	( 16.34, 65.71) --
	( 16.34, 65.71) --
	( 16.26, 65.71) --
	( 16.26, 65.71) --
	( 16.26, 65.71) --
	( 16.19, 65.71) --
	( 16.19, 65.71) --
	( 16.19, 65.71) --
	( 16.11, 65.71) --
	( 16.11, 65.71) --
	( 16.11, 65.71) --
	( 16.04, 65.71) --
	( 16.04, 65.71) --
	( 16.04, 65.71) --
	( 15.96, 65.71) --
	( 15.96, 65.71) --
	( 15.96, 65.71) --
	( 15.89, 65.71) --
	( 15.89, 65.71) --
	( 15.89, 65.71) --
	( 15.81, 65.71) --
	( 15.81, 65.71) --
	( 15.81, 65.71) --
	( 15.74, 65.71) --
	( 15.74, 65.71) --
	( 15.74, 65.71) --
	( 15.66, 65.71) --
	( 15.66, 65.71) --
	( 15.66, 65.71) --
	( 15.58, 65.71) --
	( 15.58, 65.71) --
	( 15.58, 65.71) --
	( 15.51, 65.71) --
	( 15.51, 65.71) --
	( 15.51, 65.71) --
	( 15.43, 65.71) --
	( 15.43, 65.71) --
	( 15.43, 65.71) --
	( 15.36, 65.71) --
	( 15.36, 65.71) --
	( 15.36, 65.71) --
	( 15.28, 65.71) --
	( 15.28, 65.71) --
	( 15.28, 65.71) --
	( 15.20, 65.71) --
	( 15.20, 65.71) --
	( 15.20, 65.71) --
	( 15.13, 65.71) --
	( 15.13, 65.71) --
	( 15.13, 65.71) --
	( 15.05, 65.71) --
	( 15.05, 65.71) --
	( 15.05, 65.71) --
	( 14.98, 65.71) --
	( 14.98, 65.71) --
	( 14.98, 65.71) --
	( 14.90, 65.71) --
	( 14.90, 65.71) --
	( 14.90, 65.71) --
	( 14.82, 65.71) --
	( 14.82, 65.71) --
	( 14.82, 65.71) --
	( 14.75, 65.71) --
	( 14.75, 65.71) --
	( 14.75, 65.71) --
	( 14.67, 65.71) --
	( 14.67, 65.71) --
	( 14.67, 65.71) --
	( 14.60, 65.71) --
	( 14.60, 65.71) --
	( 14.60, 65.71) --
	( 14.52, 65.71) --
	( 14.52, 65.71) --
	( 14.52, 65.71) --
	( 14.45, 65.71) --
	( 14.45, 65.71) --
	( 14.45, 65.71) --
	( 14.37, 65.71) --
	( 14.37, 65.71) --
	( 14.37, 65.71) --
	( 14.29, 65.71) --
	( 14.29, 65.71) --
	( 14.29, 65.71) --
	( 14.22, 65.71) --
	( 14.22, 65.71) --
	( 14.22, 65.71) --
	( 14.14, 65.71) --
	( 14.14, 65.71) --
	( 14.14, 65.71) --
	( 14.07, 65.71) --
	( 14.07, 65.71) --
	( 14.07, 65.71) --
	( 13.99, 65.71) --
	( 13.99, 65.71) --
	( 13.99, 65.71) --
	( 13.92, 65.71) --
	( 13.92, 65.71) --
	( 13.92, 65.71) --
	( 13.84, 65.71) --
	( 13.84, 65.71) --
	( 13.84, 65.71) --
	( 13.76, 65.71) --
	( 13.76, 65.71) --
	( 13.76, 65.71) --
	( 13.69, 65.71) --
	( 13.69, 65.71) --
	( 13.69, 65.71) --
	( 13.61, 65.71) --
	( 13.61, 65.71) --
	( 13.61, 65.71) --
	( 13.54, 65.71) --
	( 13.54, 65.71) --
	( 13.54, 65.71) --
	( 13.46, 65.71) --
	( 13.46, 65.71) --
	( 13.46, 65.71) --
	( 13.38, 65.71) --
	( 13.38, 65.71) --
	( 13.38, 65.71) --
	( 13.31, 65.71) --
	( 13.31, 65.71) --
	( 13.31, 65.71) --
	( 13.23, 65.71) --
	( 13.23, 65.71) --
	( 13.23, 65.71) --
	( 13.16, 65.71) --
	( 13.16, 65.71) --
	( 13.16, 65.71) --
	( 13.08, 65.71) --
	( 13.08, 65.71) --
	( 13.08, 65.71) --
	( 13.00, 65.71) --
	( 13.00, 65.71) --
	( 13.00, 65.71) --
	( 12.93, 65.71) --
	( 12.93, 65.71) --
	( 12.93, 65.71) --
	( 12.85, 65.71) --
	( 12.85, 65.71) --
	( 12.85, 65.71) --
	( 12.78, 65.71) --
	( 12.78, 65.71) --
	( 12.78, 65.71) --
	( 12.70, 65.71) --
	( 12.70, 65.71) --
	( 12.70, 65.71) --
	( 12.63, 65.71) --
	( 12.63, 65.71) --
	( 12.63, 65.71) --
	( 12.55, 65.71) --
	( 12.55, 65.71) --
	( 12.55, 65.71) --
	( 12.47, 65.71) --
	( 12.47, 65.71) --
	( 12.47, 65.71) --
	( 12.40, 65.71) --
	( 12.40, 65.71) --
	( 12.40, 65.71) --
	( 12.32, 65.71) --
	( 12.32, 65.71) --
	( 12.32, 65.71) --
	( 12.25, 65.71) --
	( 12.25, 65.71) --
	( 12.25, 65.71) --
	( 12.17, 65.71) --
	( 12.17, 65.71) --
	( 12.17, 65.71) --
	( 12.09, 65.71) --
	( 12.09, 65.71) --
	( 12.09, 65.71) --
	( 12.02, 65.71) --
	( 12.02, 65.71) --
	( 12.02, 65.71) --
	( 11.94, 65.71) --
	( 11.94, 65.71) --
	( 11.94, 65.71) --
	( 11.87, 65.71) --
	( 11.87, 65.71) --
	( 11.87, 65.71) --
	( 11.79, 65.71) --
	( 11.79, 65.71) --
	( 11.79, 65.71) --
	( 11.71, 65.71) --
	( 11.71, 65.71) --
	( 11.71, 65.71) --
	( 11.64, 65.71) --
	( 11.64, 65.71) --
	( 11.64, 65.71) --
	( 11.56, 65.71) --
	( 11.56, 65.71) --
	( 11.56, 65.71) --
	( 11.49, 65.71) --
	( 11.49, 65.71) --
	( 11.49, 65.71) --
	( 11.41, 65.71) --
	( 11.41, 65.71) --
	( 11.41, 65.71) --
	( 11.33, 65.71) --
	( 11.33, 65.71) --
	( 11.33, 65.71) --
	( 11.26, 65.71) --
	( 11.26, 65.71) --
	( 11.26, 65.71) --
	( 11.18, 65.71) --
	( 11.18, 65.71) --
	( 11.18, 65.71) --
	( 11.11, 65.71) --
	( 11.11, 65.71) --
	( 11.11, 65.71) --
	( 11.03, 65.71) --
	( 11.03, 65.71) --
	( 11.03, 65.71) --
	( 10.96, 65.71) --
	( 10.96, 65.71) --
	( 10.96, 65.71) --
	( 10.88, 65.71) --
	( 10.88, 65.71) --
	( 10.88, 65.71) --
	( 10.80, 65.71) --
	( 10.80, 65.71) --
	( 10.80, 65.71) --
	( 10.73, 65.71) --
	( 10.73, 65.71) --
	( 10.73, 65.71) --
	( 10.65, 65.71) --
	( 10.65, 65.71) --
	( 10.65, 65.71) --
	( 10.58, 65.71) --
	( 10.58, 65.71) --
	( 10.58, 65.71) --
	( 10.50, 65.71) --
	( 10.50, 65.71) --
	( 10.50, 65.71) --
	( 10.42, 65.71) --
	( 10.42, 65.71) --
	( 10.42, 65.71) --
	( 10.35, 65.71) --
	( 10.35, 65.71) --
	( 10.35, 65.71) --
	( 10.27, 65.71) --
	( 10.27, 65.71) --
	( 10.27, 65.71) --
	( 10.19, 65.71) --
	( 10.19, 65.71) --
	( 10.19, 65.71) --
	( 10.12, 65.71) --
	( 10.12, 65.71) --
	( 10.12, 65.71) --
	( 10.04, 65.71) --
	( 10.04, 65.71) --
	( 10.04, 65.71) --
	(  9.97, 65.71) --
	(  9.97, 65.71) --
	(  9.97, 65.71) --
	(  9.89, 65.71) --
	(  9.89, 65.71) --
	(  9.89, 65.71) --
	(  9.82, 65.71) --
	(  9.82, 65.71) --
	(  9.82, 65.71) --
	(  9.74, 65.71) --
	(  9.74, 65.71) --
	(  9.74, 65.71) --
	(  9.66, 65.71) --
	(  9.66, 65.71) --
	(  9.66, 65.71) --
	(  9.59, 65.71) --
	(  9.59, 65.71) --
	(  9.59, 65.71) --
	(  9.51, 65.71) --
	(  9.51, 65.71) --
	(  9.51, 65.71) --
	(  9.43, 65.71) --
	(  9.43, 65.71) --
	(  9.43, 65.71) --
	(  9.36, 65.71) --
	(  9.36, 65.71) --
	(  9.36, 65.71) --
	(  9.28, 65.71) --
	(  9.28, 65.71) --
	(  9.28, 65.71) --
	(  9.21, 65.71) --
	(  9.21, 65.71) --
	(  9.21, 65.71) --
	(  9.13, 65.71) --
	(  9.13, 65.71) --
	(  9.13, 65.71) --
	(  9.05, 65.71) --
	(  9.05, 65.71) --
	(  9.05, 65.71) --
	(  8.98, 65.71) --
	(  8.98, 65.71) --
	(  8.98, 65.71) --
	(  8.90, 65.71) --
	(  8.90, 65.71) --
	(  8.90, 65.71) --
	(  8.83, 65.71) --
	(  8.83, 65.71) --
	(  8.83, 65.71) --
	(  8.75, 65.71) --
	(  8.75, 65.71) --
	(  8.75, 65.71) --
	(  8.67, 65.71) --
	(  8.67, 65.71) --
	(  8.67, 65.71) --
	(  8.60, 65.71) --
	(  8.60, 65.71) --
	(  8.60, 65.71) --
	(  8.52, 65.71) --
	(  8.52, 65.71) --
	(  8.52, 65.71) --
	(  8.45, 65.71) --
	(  8.45, 65.71) --
	(  8.45, 65.71) --
	(  8.37, 65.71) --
	(  8.37, 65.71) --
	(  8.37, 65.71) --
	(  8.29, 65.71) --
	(  8.29, 65.71) --
	(  8.29, 65.71) --
	(  8.22, 65.71) --
	(  8.22, 65.71) --
	(  8.22, 65.71) --
	(  8.14, 65.71) --
	(  8.14, 65.71) --
	(  8.14, 65.71) --
	(  8.07, 65.71) --
	(  8.07, 65.71) --
	(  8.07, 65.71) --
	(  7.99, 65.71) --
	(  7.99, 65.71) --
	(  7.99, 65.71) --
	(  7.91, 65.71) --
	(  7.91, 65.71) --
	(  7.91, 65.71) --
	(  7.84, 65.71) --
	(  7.84, 65.71) --
	(  7.84, 65.71) --
	(  7.76, 65.71) --
	(  7.76, 65.71) --
	(  7.76, 65.71) --
	(  7.68, 65.71) --
	(  7.68, 65.71) --
	(  7.68, 65.71) --
	(  7.61, 65.71) --
	(  7.61, 65.71) --
	(  7.61, 65.71) --
	(  7.53, 65.71) --
	(  7.53, 65.71) --
	(  7.53, 65.71) --
	(  7.46, 65.71) --
	(  7.46, 65.71) --
	(  7.46, 65.71) --
	(  7.38, 65.71) --
	(  7.38, 65.71) --
	(  7.38, 65.71) --
	(  7.30, 65.71) --
	(  7.30, 65.71) --
	(  7.30, 65.71) --
	(  7.23, 65.71) --
	(  7.23, 65.71) --
	(  7.23, 65.71) --
	(  7.15, 65.71) --
	(  7.15, 65.71) --
	(  7.15, 65.71) --
	(  7.08, 65.71) --
	(  7.08, 65.71) --
	(  7.08, 65.71) --
	(  7.00, 65.71) --
	(  7.00, 65.71) --
	(  7.00, 65.71) --
	(  6.92, 65.71) --
	(  6.92, 65.71) --
	(  6.92, 65.71) --
	(  6.85, 65.71) --
	(  6.85, 65.71) --
	(  6.85, 65.71) --
	(  6.77, 65.71) --
	(  6.77, 65.71) --
	(  6.77, 65.71) --
	(  6.69, 65.71) --
	(  6.69, 65.71) --
	(  6.69, 65.71) --
	(  6.62, 65.71) --
	(  6.62, 65.71) --
	(  6.62, 65.71) --
	(  6.54, 65.71) --
	(  6.54, 65.71) --
	(  6.54, 65.71) --
	(  6.47, 65.71) --
	(  6.47, 65.71) --
	(  6.47, 65.71) --
	(  6.39, 65.71) --
	(  6.39, 65.71) --
	(  6.39, 65.71) --
	(  6.31, 65.71) --
	(  6.31, 65.71) --
	(  6.31, 65.71) --
	(  6.24, 65.71) --
	(  6.24, 65.71) --
	(  6.24, 65.71) --
	(  6.16, 65.71) --
	(  6.16, 65.71) --
	(  6.16, 65.71) --
	(  6.09, 65.71) --
	(  6.09, 65.71) --
	(  6.09, 65.71) --
	(  6.01, 65.71) --
	(  6.01, 65.71) --
	(  6.01, 65.71) --
	(  5.93, 65.71) --
	(  5.93, 65.71) --
	(  5.93, 65.71) --
	(  5.86, 65.71) --
	(  5.86, 65.71) --
	(  5.86, 65.71) --
	(  5.78, 65.71) --
	(  5.78, 65.71) --
	(  5.78, 65.71) --
	(  5.70, 65.71) --
	(  5.70, 65.71) --
	(  5.70, 65.71) --
	(  5.63, 65.71) --
	(  5.63, 65.71) --
	(  5.63, 65.71) --
	(  5.55, 65.71) --
	(  5.55, 65.71) --
	(  5.55, 65.71) --
	(  5.47, 65.71) --
	(  5.47, 65.71) --
	(  5.47, 65.71) --
	(  5.40, 65.71) --
	(  5.40, 65.71) --
	(  5.40, 65.71) --
	(  5.32, 65.71) --
	(  5.32, 65.71) --
	(  5.32, 65.71) --
	(  5.25, 65.71) --
	(  5.25, 65.71) --
	(  5.25, 65.71) --
	(  5.17, 65.71) --
	(  5.17, 65.71) --
	(  5.17, 65.71) --
	(  5.09, 65.71) --
	(  5.09, 65.71) --
	(  5.09, 65.71) --
	(  5.02, 65.71) --
	(  5.02, 65.71) --
	(  5.02, 65.71) --
	(  4.94, 65.71) --
	(  4.94, 65.71) --
	(  4.94, 65.71) --
	(  4.86, 65.71) --
	(  4.86, 65.71) --
	(  4.86, 65.71) --
	(  4.79, 65.71) --
	(  4.79, 65.71) --
	(  4.79, 65.71) --
	(  4.71, 65.71) --
	(  4.71, 65.71) --
	(  4.71, 65.71) --
	(  4.64, 65.71) --
	(  4.64, 65.71) --
	(  4.64, 65.71) --
	cycle;
\definecolor{drawColor}{RGB}{163,165,0}

\path[draw=drawColor,line width= 0.6pt,line join=round] (  4.64, 65.76) --
	(  4.64, 65.76) --
	(  4.71, 65.75) --
	(  4.71, 65.75) --
	(  4.71, 65.75) --
	(  4.79, 65.78) --
	(  4.79, 65.78) --
	(  4.79, 65.78) --
	(  4.86, 65.75) --
	(  4.86, 65.75) --
	(  4.86, 65.75) --
	(  4.94, 65.74) --
	(  4.94, 65.74) --
	(  4.94, 65.74) --
	(  5.02, 65.76) --
	(  5.02, 65.76) --
	(  5.02, 65.76) --
	(  5.09, 65.76) --
	(  5.09, 65.76) --
	(  5.09, 65.76) --
	(  5.17, 65.74) --
	(  5.17, 65.74) --
	(  5.17, 65.74) --
	(  5.25, 65.77) --
	(  5.25, 65.77) --
	(  5.25, 65.77) --
	(  5.32, 65.73) --
	(  5.32, 65.73) --
	(  5.32, 65.73) --
	(  5.40, 65.75) --
	(  5.40, 65.75) --
	(  5.40, 65.75) --
	(  5.47, 65.75) --
	(  5.47, 65.75) --
	(  5.47, 65.75) --
	(  5.55, 65.77) --
	(  5.55, 65.77) --
	(  5.55, 65.77) --
	(  5.63, 65.78) --
	(  5.63, 65.78) --
	(  5.63, 65.78) --
	(  5.70, 65.75) --
	(  5.70, 65.75) --
	(  5.70, 65.75) --
	(  5.78, 65.76) --
	(  5.78, 65.76) --
	(  5.78, 65.76) --
	(  5.86, 65.77) --
	(  5.86, 65.77) --
	(  5.86, 65.77) --
	(  5.93, 65.73) --
	(  5.93, 65.73) --
	(  5.93, 65.73) --
	(  6.01, 65.76) --
	(  6.01, 65.76) --
	(  6.01, 65.76) --
	(  6.09, 65.78) --
	(  6.09, 65.78) --
	(  6.09, 65.78) --
	(  6.16, 65.74) --
	(  6.16, 65.74) --
	(  6.16, 65.74) --
	(  6.24, 65.77) --
	(  6.24, 65.77) --
	(  6.24, 65.77) --
	(  6.31, 65.73) --
	(  6.31, 65.73) --
	(  6.31, 65.73) --
	(  6.39, 65.76) --
	(  6.39, 65.76) --
	(  6.39, 65.76) --
	(  6.47, 65.74) --
	(  6.47, 65.74) --
	(  6.47, 65.74) --
	(  6.54, 65.72) --
	(  6.54, 65.72) --
	(  6.54, 65.72) --
	(  6.62, 65.76) --
	(  6.62, 65.76) --
	(  6.62, 65.76) --
	(  6.69, 65.78) --
	(  6.69, 65.78) --
	(  6.69, 65.78) --
	(  6.77, 65.76) --
	(  6.77, 65.76) --
	(  6.77, 65.76) --
	(  6.85, 65.73) --
	(  6.85, 65.73) --
	(  6.85, 65.73) --
	(  6.92, 65.75) --
	(  6.92, 65.75) --
	(  6.92, 65.75) --
	(  7.00, 65.74) --
	(  7.00, 65.74) --
	(  7.00, 65.74) --
	(  7.08, 65.75) --
	(  7.08, 65.75) --
	(  7.08, 65.75) --
	(  7.15, 65.76) --
	(  7.15, 65.76) --
	(  7.15, 65.76) --
	(  7.23, 65.75) --
	(  7.23, 65.75) --
	(  7.23, 65.75) --
	(  7.30, 65.75) --
	(  7.30, 65.75) --
	(  7.30, 65.75) --
	(  7.38, 65.73) --
	(  7.38, 65.73) --
	(  7.38, 65.73) --
	(  7.46, 65.77) --
	(  7.46, 65.77) --
	(  7.46, 65.77) --
	(  7.53, 65.75) --
	(  7.53, 65.75) --
	(  7.53, 65.75) --
	(  7.61, 65.76) --
	(  7.61, 65.76) --
	(  7.61, 65.76) --
	(  7.68, 65.74) --
	(  7.68, 65.74) --
	(  7.68, 65.74) --
	(  7.76, 65.76) --
	(  7.76, 65.76) --
	(  7.76, 65.76) --
	(  7.84, 65.76) --
	(  7.84, 65.76) --
	(  7.84, 65.76) --
	(  7.91, 65.73) --
	(  7.91, 65.73) --
	(  7.91, 65.73) --
	(  7.99, 65.74) --
	(  7.99, 65.74) --
	(  7.99, 65.74) --
	(  8.07, 65.75) --
	(  8.07, 65.75) --
	(  8.07, 65.75) --
	(  8.14, 65.77) --
	(  8.14, 65.77) --
	(  8.14, 65.77) --
	(  8.22, 65.74) --
	(  8.22, 65.74) --
	(  8.22, 65.74) --
	(  8.29, 65.75) --
	(  8.29, 65.75) --
	(  8.29, 65.75) --
	(  8.37, 65.76) --
	(  8.37, 65.76) --
	(  8.37, 65.76) --
	(  8.45, 65.74) --
	(  8.45, 65.74) --
	(  8.45, 65.74) --
	(  8.52, 65.74) --
	(  8.52, 65.74) --
	(  8.52, 65.74) --
	(  8.60, 65.76) --
	(  8.60, 65.76) --
	(  8.60, 65.76) --
	(  8.67, 65.75) --
	(  8.67, 65.75) --
	(  8.67, 65.75) --
	(  8.75, 65.74) --
	(  8.75, 65.74) --
	(  8.75, 65.74) --
	(  8.83, 65.74) --
	(  8.83, 65.74) --
	(  8.83, 65.74) --
	(  8.90, 65.75) --
	(  8.90, 65.75) --
	(  8.90, 65.75) --
	(  8.98, 65.75) --
	(  8.98, 65.75) --
	(  8.98, 65.75) --
	(  9.05, 65.74) --
	(  9.05, 65.74) --
	(  9.05, 65.74) --
	(  9.13, 65.76) --
	(  9.13, 65.76) --
	(  9.13, 65.76) --
	(  9.21, 65.76) --
	(  9.21, 65.76) --
	(  9.21, 65.76) --
	(  9.28, 65.75) --
	(  9.28, 65.75) --
	(  9.28, 65.75) --
	(  9.36, 65.74) --
	(  9.36, 65.74) --
	(  9.36, 65.74) --
	(  9.43, 65.77) --
	(  9.43, 65.77) --
	(  9.43, 65.77) --
	(  9.51, 65.75) --
	(  9.51, 65.75) --
	(  9.51, 65.75) --
	(  9.59, 65.75) --
	(  9.59, 65.75) --
	(  9.59, 65.75) --
	(  9.66, 65.77) --
	(  9.66, 65.77) --
	(  9.66, 65.77) --
	(  9.74, 65.75) --
	(  9.74, 65.75) --
	(  9.74, 65.75) --
	(  9.82, 65.74) --
	(  9.82, 65.74) --
	(  9.82, 65.74) --
	(  9.89, 65.74) --
	(  9.89, 65.74) --
	(  9.89, 65.74) --
	(  9.97, 65.78) --
	(  9.97, 65.78) --
	(  9.97, 65.78) --
	( 10.04, 65.78) --
	( 10.04, 65.78) --
	( 10.04, 65.78) --
	( 10.12, 65.74) --
	( 10.12, 65.74) --
	( 10.12, 65.74) --
	( 10.19, 65.73) --
	( 10.19, 65.73) --
	( 10.19, 65.73) --
	( 10.27, 65.76) --
	( 10.27, 65.76) --
	( 10.27, 65.76) --
	( 10.35, 65.73) --
	( 10.35, 65.73) --
	( 10.35, 65.73) --
	( 10.42, 65.78) --
	( 10.42, 65.78) --
	( 10.42, 65.78) --
	( 10.50, 65.76) --
	( 10.50, 65.76) --
	( 10.50, 65.76) --
	( 10.58, 65.75) --
	( 10.58, 65.75) --
	( 10.58, 65.75) --
	( 10.65, 65.76) --
	( 10.65, 65.76) --
	( 10.65, 65.76) --
	( 10.73, 65.74) --
	( 10.73, 65.74) --
	( 10.73, 65.74) --
	( 10.80, 65.77) --
	( 10.80, 65.77) --
	( 10.80, 65.77) --
	( 10.88, 65.79) --
	( 10.88, 65.79) --
	( 10.88, 65.79) --
	( 10.96, 65.76) --
	( 10.96, 65.76) --
	( 10.96, 65.76) --
	( 11.03, 65.74) --
	( 11.03, 65.74) --
	( 11.03, 65.74) --
	( 11.11, 65.76) --
	( 11.11, 65.76) --
	( 11.11, 65.76) --
	( 11.18, 65.76) --
	( 11.18, 65.76) --
	( 11.18, 65.76) --
	( 11.26, 65.75) --
	( 11.26, 65.75) --
	( 11.26, 65.75) --
	( 11.33, 65.75) --
	( 11.33, 65.75) --
	( 11.33, 65.75) --
	( 11.41, 65.77) --
	( 11.41, 65.77) --
	( 11.41, 65.77) --
	( 11.49, 65.77) --
	( 11.49, 65.77) --
	( 11.49, 65.77) --
	( 11.56, 65.74) --
	( 11.56, 65.74) --
	( 11.56, 65.74) --
	( 11.64, 65.76) --
	( 11.64, 65.76) --
	( 11.64, 65.76) --
	( 11.71, 65.78) --
	( 11.71, 65.78) --
	( 11.71, 65.78) --
	( 11.79, 65.77) --
	( 11.79, 65.77) --
	( 11.79, 65.77) --
	( 11.87, 65.77) --
	( 11.87, 65.77) --
	( 11.87, 65.77) --
	( 11.94, 65.79) --
	( 11.94, 65.79) --
	( 11.94, 65.79) --
	( 12.02, 65.77) --
	( 12.02, 65.77) --
	( 12.02, 65.77) --
	( 12.09, 65.74) --
	( 12.09, 65.74) --
	( 12.09, 65.74) --
	( 12.17, 65.76) --
	( 12.17, 65.76) --
	( 12.17, 65.76) --
	( 12.25, 65.76) --
	( 12.25, 65.76) --
	( 12.25, 65.76) --
	( 12.32, 65.75) --
	( 12.32, 65.75) --
	( 12.32, 65.75) --
	( 12.40, 65.73) --
	( 12.40, 65.73) --
	( 12.40, 65.73) --
	( 12.47, 65.74) --
	( 12.47, 65.74) --
	( 12.47, 65.74) --
	( 12.55, 65.79) --
	( 12.55, 65.79) --
	( 12.55, 65.79) --
	( 12.63, 65.76) --
	( 12.63, 65.76) --
	( 12.63, 65.76) --
	( 12.70, 65.76) --
	( 12.70, 65.76) --
	( 12.70, 65.76) --
	( 12.78, 65.76) --
	( 12.78, 65.76) --
	( 12.78, 65.76) --
	( 12.85, 65.75) --
	( 12.85, 65.75) --
	( 12.85, 65.75) --
	( 12.93, 65.75) --
	( 12.93, 65.75) --
	( 12.93, 65.75) --
	( 13.00, 65.75) --
	( 13.00, 65.75) --
	( 13.00, 65.75) --
	( 13.08, 65.76) --
	( 13.08, 65.76) --
	( 13.08, 65.76) --
	( 13.16, 65.76) --
	( 13.16, 65.76) --
	( 13.16, 65.76) --
	( 13.23, 65.75) --
	( 13.23, 65.75) --
	( 13.23, 65.75) --
	( 13.31, 65.75) --
	( 13.31, 65.75) --
	( 13.31, 65.75) --
	( 13.38, 65.78) --
	( 13.38, 65.78) --
	( 13.38, 65.78) --
	( 13.46, 65.75) --
	( 13.46, 65.75) --
	( 13.46, 65.75) --
	( 13.54, 65.77) --
	( 13.54, 65.77) --
	( 13.54, 65.77) --
	( 13.61, 65.74) --
	( 13.61, 65.74) --
	( 13.61, 65.74) --
	( 13.69, 65.74) --
	( 13.69, 65.74) --
	( 13.69, 65.74) --
	( 13.76, 65.75) --
	( 13.76, 65.75) --
	( 13.76, 65.75) --
	( 13.84, 65.75) --
	( 13.84, 65.75) --
	( 13.84, 65.75) --
	( 13.92, 65.75) --
	( 13.92, 65.75) --
	( 13.92, 65.75) --
	( 13.99, 65.78) --
	( 13.99, 65.78) --
	( 13.99, 65.78) --
	( 14.07, 65.74) --
	( 14.07, 65.74) --
	( 14.07, 65.74) --
	( 14.14, 65.75) --
	( 14.14, 65.75) --
	( 14.14, 65.75) --
	( 14.22, 65.76) --
	( 14.22, 65.76) --
	( 14.22, 65.76) --
	( 14.29, 65.74) --
	( 14.29, 65.74) --
	( 14.29, 65.74) --
	( 14.37, 65.76) --
	( 14.37, 65.76) --
	( 14.37, 65.76) --
	( 14.45, 65.75) --
	( 14.45, 65.75) --
	( 14.45, 65.75) --
	( 14.52, 65.77) --
	( 14.52, 65.77) --
	( 14.52, 65.77) --
	( 14.60, 65.74) --
	( 14.60, 65.74) --
	( 14.60, 65.74) --
	( 14.67, 65.74) --
	( 14.67, 65.74) --
	( 14.67, 65.74) --
	( 14.75, 65.76) --
	( 14.75, 65.76) --
	( 14.75, 65.76) --
	( 14.82, 65.75) --
	( 14.82, 65.75) --
	( 14.82, 65.75) --
	( 14.90, 65.73) --
	( 14.90, 65.73) --
	( 14.90, 65.73) --
	( 14.98, 65.74) --
	( 14.98, 65.74) --
	( 14.98, 65.74) --
	( 15.05, 65.76) --
	( 15.05, 65.76) --
	( 15.05, 65.76) --
	( 15.13, 65.74) --
	( 15.13, 65.74) --
	( 15.13, 65.74) --
	( 15.20, 65.77) --
	( 15.20, 65.77) --
	( 15.20, 65.77) --
	( 15.28, 65.78) --
	( 15.28, 65.78) --
	( 15.28, 65.78) --
	( 15.36, 65.76) --
	( 15.36, 65.76) --
	( 15.36, 65.76) --
	( 15.43, 65.77) --
	( 15.43, 65.77) --
	( 15.43, 65.77) --
	( 15.51, 65.72) --
	( 15.51, 65.72) --
	( 15.51, 65.72) --
	( 15.58, 65.75) --
	( 15.58, 65.75) --
	( 15.58, 65.75) --
	( 15.66, 65.77) --
	( 15.66, 65.77) --
	( 15.66, 65.77) --
	( 15.74, 65.74) --
	( 15.74, 65.74) --
	( 15.74, 65.74) --
	( 15.81, 65.75) --
	( 15.81, 65.75) --
	( 15.81, 65.75) --
	( 15.89, 65.76) --
	( 15.89, 65.76) --
	( 15.89, 65.76) --
	( 15.96, 65.76) --
	( 15.96, 65.76) --
	( 15.96, 65.76) --
	( 16.04, 65.76) --
	( 16.04, 65.76) --
	( 16.04, 65.76) --
	( 16.11, 65.76) --
	( 16.11, 65.76) --
	( 16.11, 65.76) --
	( 16.19, 65.76) --
	( 16.19, 65.76) --
	( 16.19, 65.76) --
	( 16.26, 65.76) --
	( 16.26, 65.76) --
	( 16.26, 65.76) --
	( 16.34, 65.74) --
	( 16.34, 65.74) --
	( 16.34, 65.74) --
	( 16.42, 65.77) --
	( 16.42, 65.77) --
	( 16.42, 65.77) --
	( 16.49, 65.76) --
	( 16.49, 65.76) --
	( 16.49, 65.76) --
	( 16.57, 65.77) --
	( 16.57, 65.77) --
	( 16.57, 65.77) --
	( 16.64, 65.76) --
	( 16.64, 65.76) --
	( 16.64, 65.76) --
	( 16.72, 65.78) --
	( 16.72, 65.78) --
	( 16.72, 65.78) --
	( 16.79, 65.77) --
	( 16.79, 65.77) --
	( 16.79, 65.77) --
	( 16.87, 65.75) --
	( 16.87, 65.75) --
	( 16.87, 65.75) --
	( 16.95, 65.75) --
	( 16.95, 65.75) --
	( 16.95, 65.75) --
	( 17.02, 65.75) --
	( 17.02, 65.75) --
	( 17.02, 65.75) --
	( 17.10, 65.76) --
	( 17.10, 65.76) --
	( 17.10, 65.76) --
	( 17.17, 65.73) --
	( 17.17, 65.73) --
	( 17.17, 65.73) --
	( 17.25, 65.76) --
	( 17.25, 65.76) --
	( 17.25, 65.76) --
	( 17.32, 65.75) --
	( 17.32, 65.75) --
	( 17.32, 65.75) --
	( 17.40, 65.74) --
	( 17.40, 65.74) --
	( 17.40, 65.74) --
	( 17.48, 65.75) --
	( 17.48, 65.75) --
	( 17.48, 65.75) --
	( 17.55, 65.77) --
	( 17.55, 65.77) --
	( 17.55, 65.77) --
	( 17.63, 65.76) --
	( 17.63, 65.76) --
	( 17.63, 65.76) --
	( 17.70, 65.78) --
	( 17.70, 65.78) --
	( 17.70, 65.78) --
	( 17.78, 65.76) --
	( 17.78, 65.76) --
	( 17.78, 65.76) --
	( 17.85, 65.76) --
	( 17.85, 65.76) --
	( 17.85, 65.76) --
	( 17.93, 65.76) --
	( 17.93, 65.76) --
	( 17.93, 65.76) --
	( 18.01, 65.75) --
	( 18.01, 65.75) --
	( 18.01, 65.75) --
	( 18.08, 65.75) --
	( 18.08, 65.75) --
	( 18.08, 65.75) --
	( 18.16, 65.77) --
	( 18.16, 65.77) --
	( 18.16, 65.77) --
	( 18.23, 65.72) --
	( 18.23, 65.72) --
	( 18.23, 65.72) --
	( 18.31, 65.74) --
	( 18.31, 65.74) --
	( 18.31, 65.74) --
	( 18.38, 65.79) --
	( 18.38, 65.79) --
	( 18.38, 65.79) --
	( 18.46, 65.74) --
	( 18.46, 65.74) --
	( 18.46, 65.74) --
	( 18.53, 65.76) --
	( 18.53, 65.76) --
	( 18.53, 65.76) --
	( 18.61, 65.74) --
	( 18.61, 65.74) --
	( 18.61, 65.74) --
	( 18.69, 65.75) --
	( 18.69, 65.75) --
	( 18.69, 65.75) --
	( 18.76, 65.76) --
	( 18.76, 65.76) --
	( 18.76, 65.76) --
	( 18.84, 65.74) --
	( 18.84, 65.74) --
	( 18.84, 65.74) --
	( 18.91, 65.76) --
	( 18.91, 65.76) --
	( 18.91, 65.76) --
	( 18.99, 65.78) --
	( 18.99, 65.78) --
	( 18.99, 65.78) --
	( 19.06, 65.76) --
	( 19.06, 65.76) --
	( 19.06, 65.76) --
	( 19.14, 65.75) --
	( 19.14, 65.75) --
	( 19.14, 65.75) --
	( 19.22, 65.78) --
	( 19.22, 65.78) --
	( 19.22, 65.78) --
	( 19.29, 65.75) --
	( 19.29, 65.75) --
	( 19.29, 65.75) --
	( 19.37, 65.75) --
	( 19.37, 65.75) --
	( 19.37, 65.75) --
	( 19.44, 65.75) --
	( 19.44, 65.75) --
	( 19.44, 65.75) --
	( 19.52, 65.74) --
	( 19.52, 65.74) --
	( 19.52, 65.74) --
	( 19.59, 65.75) --
	( 19.59, 65.75) --
	( 19.59, 65.75) --
	( 19.67, 65.75) --
	( 19.67, 65.75) --
	( 19.67, 65.75) --
	( 19.74, 65.74) --
	( 19.74, 65.74) --
	( 19.74, 65.74) --
	( 19.82, 65.76) --
	( 19.82, 65.76) --
	( 19.82, 65.76) --
	( 19.89, 65.75) --
	( 19.89, 65.75) --
	( 19.90, 65.75) --
	( 19.97, 65.75) --
	( 19.97, 65.75) --
	( 19.97, 65.75) --
	( 20.05, 65.75) --
	( 20.05, 65.75) --
	( 20.05, 65.75) --
	( 20.12, 65.76) --
	( 20.12, 65.76) --
	( 20.12, 65.76) --
	( 20.20, 65.77) --
	( 20.20, 65.77) --
	( 20.20, 65.77) --
	( 20.27, 65.74) --
	( 20.27, 65.74) --
	( 20.27, 65.74) --
	( 20.35, 65.78) --
	( 20.35, 65.78) --
	( 20.35, 65.78) --
	( 20.42, 65.76) --
	( 20.42, 65.76) --
	( 20.42, 65.76) --
	( 20.50, 65.74) --
	( 20.50, 65.74) --
	( 20.50, 65.74) --
	( 20.57, 65.74) --
	( 20.57, 65.74) --
	( 20.57, 65.74) --
	( 20.65, 65.77) --
	( 20.65, 65.77) --
	( 20.65, 65.77) --
	( 20.72, 65.74) --
	( 20.72, 65.74) --
	( 20.73, 65.74) --
	( 20.80, 65.75) --
	( 20.80, 65.75) --
	( 20.80, 65.75) --
	( 20.88, 65.76) --
	( 20.88, 65.76) --
	( 20.88, 65.76) --
	( 20.95, 65.76) --
	( 20.95, 65.76) --
	( 20.95, 65.76) --
	( 21.03, 65.75) --
	( 21.03, 65.75) --
	( 21.03, 65.75) --
	( 21.10, 65.75) --
	( 21.10, 65.75) --
	( 21.10, 65.75) --
	( 21.18, 65.76) --
	( 21.18, 65.76) --
	( 21.18, 65.76) --
	( 21.25, 65.76) --
	( 21.25, 65.76) --
	( 21.25, 65.76) --
	( 21.33, 65.72) --
	( 21.33, 65.72) --
	( 21.33, 65.72) --
	( 21.40, 65.75) --
	( 21.40, 65.75) --
	( 21.40, 65.75) --
	( 21.48, 65.77) --
	( 21.48, 65.77) --
	( 21.48, 65.77) --
	( 21.55, 65.75) --
	( 21.56, 65.75) --
	( 21.56, 65.75) --
	( 21.63, 65.74) --
	( 21.63, 65.74) --
	( 21.63, 65.74) --
	( 21.71, 65.77) --
	( 21.71, 65.77) --
	( 21.71, 65.77) --
	( 21.78, 65.76) --
	( 21.78, 65.76) --
	( 21.78, 65.76) --
	( 21.86, 65.76) --
	( 21.86, 65.76) --
	( 21.86, 65.76) --
	( 21.93, 65.77) --
	( 21.93, 65.77) --
	( 21.93, 65.77) --
	( 22.01, 65.77) --
	( 22.01, 65.77) --
	( 22.01, 65.77) --
	( 22.08, 65.74) --
	( 22.08, 65.74) --
	( 22.08, 65.74) --
	( 22.16, 65.72) --
	( 22.16, 65.72) --
	( 22.16, 65.72) --
	( 22.23, 65.75) --
	( 22.23, 65.75) --
	( 22.23, 65.75) --
	( 22.31, 65.77) --
	( 22.31, 65.77) --
	( 22.31, 65.77) --
	( 22.38, 65.75) --
	( 22.39, 65.75) --
	( 22.39, 65.75) --
	( 22.46, 65.76) --
	( 22.46, 65.76) --
	( 22.46, 65.76) --
	( 22.54, 65.77) --
	( 22.54, 65.77) --
	( 22.54, 65.77) --
	( 22.61, 65.76) --
	( 22.61, 65.76) --
	( 22.61, 65.76) --
	( 22.69, 65.77) --
	( 22.69, 65.77) --
	( 22.69, 65.77) --
	( 22.76, 65.75) --
	( 22.76, 65.75) --
	( 22.76, 65.75) --
	( 22.84, 65.79) --
	( 22.84, 65.79) --
	( 22.84, 65.79) --
	( 22.91, 65.76) --
	( 22.91, 65.76) --
	( 22.91, 65.76) --
	( 22.99, 65.74) --
	( 22.99, 65.74) --
	( 22.99, 65.74) --
	( 23.06, 65.75) --
	( 23.06, 65.75) --
	( 23.06, 65.75) --
	( 23.14, 65.77) --
	( 23.14, 65.77) --
	( 23.14, 65.77) --
	( 23.21, 65.75) --
	( 23.22, 65.75) --
	( 23.22, 65.75) --
	( 23.29, 65.77) --
	( 23.29, 65.77) --
	( 23.29, 65.77) --
	( 23.37, 65.77) --
	( 23.37, 65.77) --
	( 23.37, 65.77) --
	( 23.44, 65.74) --
	( 23.44, 65.74) --
	( 23.44, 65.74) --
	( 23.52, 65.77) --
	( 23.52, 65.77) --
	( 23.52, 65.77) --
	( 23.59, 65.73) --
	( 23.59, 65.73) --
	( 23.59, 65.73) --
	( 23.67, 65.76) --
	( 23.67, 65.76) --
	( 23.67, 65.76) --
	( 23.74, 65.77) --
	( 23.74, 65.77) --
	( 23.74, 65.77) --
	( 23.82, 65.74) --
	( 23.82, 65.74) --
	( 23.82, 65.74) --
	( 23.89, 65.76) --
	( 23.89, 65.76) --
	( 23.89, 65.76) --
	( 23.97, 65.77) --
	( 23.97, 65.77) --
	( 23.97, 65.77) --
	( 24.04, 65.78) --
	( 24.04, 65.78) --
	( 24.04, 65.78) --
	( 24.12, 65.77) --
	( 24.12, 65.77) --
	( 24.12, 65.77) --
	( 24.19, 65.79) --
	( 24.19, 65.79) --
	( 24.19, 65.79) --
	( 24.27, 65.76) --
	( 24.27, 65.76) --
	( 24.27, 65.76) --
	( 24.27, 65.76) --
	( 24.27, 65.76) --
	( 24.27, 65.76) --
	( 24.34, 65.74) --
	( 24.34, 65.74) --
	( 24.34, 65.74) --
	( 24.42, 65.74) --
	( 24.42, 65.74) --
	( 24.42, 65.74) --
	( 24.42, 65.74) --
	( 24.42, 65.74) --
	( 24.42, 65.74) --
	( 24.50, 65.76) --
	( 24.50, 65.76) --
	( 24.50, 65.76) --
	( 24.51, 65.76) --
	( 24.51, 65.76) --
	( 24.51, 65.76) --
	( 24.55, 65.77) --
	( 24.55, 65.77) --
	( 24.55, 65.77) --
	( 24.57, 65.77) --
	( 24.57, 65.77) --
	( 24.57, 65.77) --
	( 24.63, 65.75) --
	( 24.63, 65.75) --
	( 24.63, 65.75) --
	( 24.65, 65.75) --
	( 24.65, 65.75) --
	( 24.65, 65.75) --
	( 24.72, 65.76) --
	( 24.72, 65.76) --
	( 24.72, 65.76) --
	( 24.75, 65.77) --
	( 24.75, 65.77) --
	( 24.75, 65.77) --
	( 24.80, 65.79) --
	( 24.80, 65.79) --
	( 24.80, 65.79) --
	( 24.83, 65.77) --
	( 24.83, 65.77) --
	( 24.83, 65.77) --
	( 24.87, 65.75) --
	( 24.87, 65.75) --
	( 24.87, 65.75) --
	( 24.95, 65.75) --
	( 24.95, 65.75) --
	( 24.95, 65.75) --
	( 25.02, 65.75) --
	( 25.02, 65.75) --
	( 25.02, 65.75) --
	( 25.10, 65.77) --
	( 25.10, 65.77) --
	( 25.10, 65.77) --
	( 25.12, 65.77) --
	( 25.12, 65.77) --
	( 25.12, 65.77) --
	( 25.17, 65.77) --
	( 25.17, 65.77) --
	( 25.17, 65.77) --
	( 25.25, 65.75) --
	( 25.25, 65.75) --
	( 25.25, 65.75) --
	( 25.32, 65.76) --
	( 25.32, 65.76) --
	( 25.32, 65.76) --
	( 25.40, 65.75) --
	( 25.40, 65.75) --
	( 25.40, 65.75) --
	( 25.47, 65.77) --
	( 25.47, 65.77) --
	( 25.47, 65.77) --
	( 25.55, 65.76) --
	( 25.55, 65.76) --
	( 25.55, 65.76) --
	( 25.62, 65.77) --
	( 25.62, 65.77) --
	( 25.62, 65.77) --
	( 25.70, 65.73) --
	( 25.70, 65.73) --
	( 25.70, 65.73) --
	( 25.77, 65.77) --
	( 25.77, 65.77) --
	( 25.77, 65.77) --
	( 25.80, 65.76) --
	( 25.80, 65.76) --
	( 25.80, 65.76) --
	( 25.85, 65.74) --
	( 25.85, 65.74) --
	( 25.85, 65.74) --
	( 25.92, 65.74) --
	( 25.92, 65.74) --
	( 25.92, 65.74) --
	( 26.00, 65.77) --
	( 26.00, 65.77) --
	( 26.00, 65.77) --
	( 26.07, 65.75) --
	( 26.07, 65.75) --
	( 26.07, 65.75) --
	( 26.15, 65.75) --
	( 26.15, 65.75) --
	( 26.15, 65.75) --
	( 26.22, 65.75) --
	( 26.22, 65.75) --
	( 26.22, 65.75) --
	( 26.30, 65.74) --
	( 26.30, 65.74) --
	( 26.30, 65.74) --
	( 26.38, 65.75) --
	( 26.38, 65.75) --
	( 26.38, 65.75) --
	( 26.45, 65.78) --
	( 26.45, 65.78) --
	( 26.45, 65.78) --
	( 26.53, 65.75) --
	( 26.53, 65.75) --
	( 26.53, 65.75) --
	( 26.60, 65.78) --
	( 26.60, 65.78) --
	( 26.60, 65.78) --
	( 26.68, 65.75) --
	( 26.68, 65.75) --
	( 26.68, 65.75) --
	( 26.75, 65.77) --
	( 26.75, 65.77) --
	( 26.75, 65.77) --
	( 26.83, 65.76) --
	( 26.83, 65.76) --
	( 26.83, 65.76) --
	( 26.90, 65.72) --
	( 26.90, 65.72) --
	( 26.90, 65.72) --
	( 26.98, 65.75) --
	( 26.98, 65.75) --
	( 26.98, 65.75) --
	( 27.05, 65.77) --
	( 27.05, 65.77) --
	( 27.05, 65.77) --
	( 27.06, 65.77) --
	( 27.06, 65.77) --
	( 27.06, 65.77) --
	( 27.13, 65.75) --
	( 27.13, 65.75) --
	( 27.13, 65.75) --
	( 27.20, 65.77) --
	( 27.20, 65.77) --
	( 27.20, 65.77) --
	( 27.28, 65.76) --
	( 27.28, 65.76) --
	( 27.28, 65.76) --
	( 27.35, 65.73) --
	( 27.35, 65.73) --
	( 27.35, 65.73) --
	( 27.39, 65.75) --
	( 27.39, 65.75) --
	( 27.39, 65.75) --
	( 27.43, 65.76) --
	( 27.43, 65.76) --
	( 27.43, 65.76) --
	( 27.50, 65.76) --
	( 27.50, 65.76) --
	( 27.50, 65.76) --
	( 27.58, 65.76) --
	( 27.58, 65.76) --
	( 27.58, 65.76) --
	( 27.65, 65.76) --
	( 27.65, 65.76) --
	( 27.65, 65.76) --
	( 27.66, 65.76) --
	( 27.66, 65.76) --
	( 27.66, 65.76) --
	( 27.73, 65.74) --
	( 27.73, 65.74) --
	( 27.73, 65.74) --
	( 27.80, 65.76) --
	( 27.80, 65.76) --
	( 27.80, 65.76) --
	( 27.88, 65.78) --
	( 27.88, 65.78) --
	( 27.88, 65.78) --
	( 27.95, 65.74) --
	( 27.95, 65.74) --
	( 27.95, 65.74) --
	( 28.03, 65.75) --
	( 28.03, 65.75) --
	( 28.03, 65.75) --
	( 28.03, 65.75) --
	( 28.03, 65.75) --
	( 28.03, 65.75) --
	( 28.10, 65.75) --
	( 28.10, 65.75) --
	( 28.10, 65.75) --
	( 28.18, 65.75) --
	( 28.18, 65.75) --
	( 28.18, 65.75) --
	( 28.19, 65.75) --
	( 28.19, 65.75) --
	( 28.19, 65.75) --
	( 28.25, 65.76) --
	( 28.25, 65.76) --
	( 28.25, 65.76) --
	( 28.33, 65.77) --
	( 28.33, 65.77) --
	( 28.33, 65.77) --
	( 28.39, 65.76) --
	( 28.39, 65.76) --
	( 28.39, 65.76) --
	( 28.40, 65.76) --
	( 28.40, 65.76) --
	( 28.40, 65.76) --
	( 28.48, 65.76) --
	( 28.48, 65.76) --
	( 28.48, 65.76) --
	( 28.56, 65.75) --
	( 28.56, 65.75) --
	( 28.56, 65.75) --
	( 28.63, 65.75) --
	( 28.63, 65.75) --
	( 28.63, 65.75) --
	( 28.71, 65.77) --
	( 28.71, 65.77) --
	( 28.71, 65.77) --
	( 28.78, 65.76) --
	( 28.78, 65.76) --
	( 28.78, 65.76) --
	( 28.86, 65.75) --
	( 28.86, 65.75) --
	( 28.86, 65.75) --
	( 28.93, 65.76) --
	( 28.93, 65.76) --
	( 28.93, 65.76) --
	( 29.01, 65.76) --
	( 29.01, 65.76) --
	( 29.01, 65.76) --
	( 29.08, 65.77) --
	( 29.08, 65.77) --
	( 29.08, 65.77) --
	( 29.16, 65.76) --
	( 29.16, 65.76) --
	( 29.16, 65.76) --
	( 29.20, 65.76) --
	( 29.20, 65.76) --
	( 29.20, 65.76) --
	( 29.23, 65.76) --
	( 29.23, 65.76) --
	( 29.23, 65.76) --
	( 29.31, 65.76) --
	( 29.31, 65.76) --
	( 29.31, 65.76) --
	( 29.38, 65.76) --
	( 29.38, 65.76) --
	( 29.38, 65.76) --
	( 29.40, 65.76) --
	( 29.40, 65.76) --
	( 29.40, 65.76) --
	( 29.46, 65.77) --
	( 29.46, 65.77) --
	( 29.46, 65.77) --
	( 29.53, 65.78) --
	( 29.53, 65.78) --
	( 29.53, 65.78) --
	( 29.61, 65.75) --
	( 29.61, 65.75) --
	( 29.61, 65.75) --
	( 29.68, 65.75) --
	( 29.68, 65.75) --
	( 29.68, 65.75) --
	( 29.76, 65.76) --
	( 29.76, 65.76) --
	( 29.76, 65.76) --
	( 29.83, 65.75) --
	( 29.83, 65.75) --
	( 29.83, 65.75) --
	( 29.91, 65.77) --
	( 29.91, 65.77) --
	( 29.91, 65.77) --
	( 29.93, 65.76) --
	( 29.93, 65.76) --
	( 29.93, 65.76) --
	( 29.98, 65.75) --
	( 29.98, 65.75) --
	( 29.98, 65.75) --
	( 30.06, 65.77) --
	( 30.06, 65.77) --
	( 30.06, 65.77) --
	( 30.13, 65.76) --
	( 30.13, 65.76) --
	( 30.13, 65.76) --
	( 30.21, 65.73) --
	( 30.21, 65.73) --
	( 30.21, 65.73) --
	( 30.28, 65.77) --
	( 30.28, 65.77) --
	( 30.28, 65.77) --
	( 30.36, 65.77) --
	( 30.36, 65.77) --
	( 30.36, 65.77) --
	( 30.41, 65.75) --
	( 30.41, 65.75) --
	( 30.41, 65.75) --
	( 30.43, 65.75) --
	( 30.43, 65.75) --
	( 30.43, 65.75) --
	( 30.51, 65.76) --
	( 30.51, 65.76) --
	( 30.51, 65.76) --
	( 30.58, 65.77) --
	( 30.58, 65.77) --
	( 30.58, 65.77) --
	( 30.66, 65.78) --
	( 30.66, 65.78) --
	( 30.66, 65.78) --
	( 30.73, 65.76) --
	( 30.73, 65.76) --
	( 30.73, 65.76) --
	( 30.73, 65.76) --
	( 30.73, 65.76) --
	( 30.73, 65.76) --
	( 30.81, 65.75) --
	( 30.81, 65.75) --
	( 30.81, 65.75) --
	( 30.88, 65.76) --
	( 30.88, 65.76) --
	( 30.88, 65.76) --
	( 30.94, 65.77) --
	( 30.94, 65.77) --
	( 30.94, 65.77) --
	( 30.96, 65.77) --
	( 30.96, 65.77) --
	( 30.96, 65.77) --
	( 31.03, 65.76) --
	( 31.03, 65.76) --
	( 31.03, 65.76) --
	( 31.10, 65.77) --
	( 31.10, 65.77) --
	( 31.10, 65.77) --
	( 31.14, 65.77) --
	( 31.14, 65.77) --
	( 31.14, 65.77) --
	( 31.18, 65.76) --
	( 31.18, 65.76) --
	( 31.18, 65.76) --
	( 31.26, 65.73) --
	( 31.26, 65.73) --
	( 31.26, 65.73) --
	( 31.30, 65.74) --
	( 31.30, 65.74) --
	( 31.30, 65.74) --
	( 31.33, 65.74) --
	( 31.33, 65.74) --
	( 31.33, 65.74) --
	( 31.41, 65.78) --
	( 31.41, 65.78) --
	( 31.41, 65.78) --
	( 31.42, 65.78) --
	( 31.42, 65.78) --
	( 31.42, 65.78) --
	( 31.46, 65.77) --
	( 31.46, 65.77) --
	( 31.46, 65.77) --
	( 31.48, 65.77) --
	( 31.48, 65.77) --
	( 31.48, 65.77) --
	( 31.56, 65.75) --
	( 31.56, 65.75) --
	( 31.56, 65.75) --
	( 31.58, 65.74) --
	( 31.58, 65.74) --
	( 31.58, 65.74) --
	( 31.63, 65.73) --
	( 31.63, 65.73) --
	( 31.63, 65.73) --
	( 31.70, 65.78) --
	( 31.70, 65.78) --
	( 31.70, 65.78) --
	( 31.71, 65.78) --
	( 31.71, 65.78) --
	( 31.71, 65.78) --
	( 31.74, 65.76) --
	( 31.74, 65.76) --
	( 31.74, 65.76) --
	( 31.78, 65.75) --
	( 31.78, 65.75) --
	( 31.78, 65.75) --
	( 31.85, 65.76) --
	( 31.85, 65.76) --
	( 31.85, 65.76) --
	( 31.87, 65.76) --
	( 31.87, 65.76) --
	( 31.87, 65.76) --
	( 31.93, 65.77) --
	( 31.93, 65.77) --
	( 31.93, 65.77) --
	( 31.95, 65.76) --
	( 31.95, 65.76) --
	( 31.95, 65.76) --
	( 32.00, 65.75) --
	( 32.00, 65.75) --
	( 32.00, 65.75) --
	( 32.08, 65.75) --
	( 32.08, 65.75) --
	( 32.08, 65.75) --
	( 32.11, 65.75) --
	( 32.11, 65.75) --
	( 32.11, 65.75) --
	( 32.15, 65.74) --
	( 32.15, 65.74) --
	( 32.15, 65.74) --
	( 32.23, 65.76) --
	( 32.23, 65.76) --
	( 32.23, 65.76) --
	( 32.23, 65.76) --
	( 32.23, 65.76) --
	( 32.23, 65.76) --
	( 32.27, 65.77) --
	( 32.27, 65.77) --
	( 32.27, 65.77) --
	( 32.30, 65.77) --
	( 32.30, 65.77) --
	( 32.30, 65.77) --
	( 32.38, 65.75) --
	( 32.38, 65.75) --
	( 32.38, 65.75) --
	( 32.39, 65.75) --
	( 32.39, 65.75) --
	( 32.39, 65.75) --
	( 32.43, 65.75) --
	( 32.43, 65.75) --
	( 32.43, 65.75) --
	( 32.45, 65.75) --
	( 32.45, 65.75) --
	( 32.45, 65.75) --
	( 32.47, 65.75) --
	( 32.47, 65.75) --
	( 32.47, 65.75) --
	( 32.47, 65.75) --
	( 32.47, 65.75) --
	( 32.47, 65.75) --
	( 32.53, 65.77) --
	( 32.53, 65.77) --
	( 32.53, 65.77) --
	( 32.55, 65.76) --
	( 32.55, 65.76) --
	( 32.55, 65.76) --
	( 32.59, 65.76) --
	( 32.59, 65.76) --
	( 32.59, 65.76) --
	( 32.60, 65.76) --
	( 32.60, 65.76) --
	( 32.60, 65.76) --
	( 32.67, 65.76) --
	( 32.67, 65.76) --
	( 32.67, 65.76) --
	( 32.68, 65.76) --
	( 32.68, 65.76) --
	( 32.68, 65.76) --
	( 32.71, 65.76) --
	( 32.71, 65.76) --
	( 32.71, 65.76) --
	( 32.75, 65.77) --
	( 32.75, 65.77) --
	( 32.75, 65.77) --
	( 32.83, 65.78) --
	( 32.83, 65.78) --
	( 32.83, 65.78) --
	( 32.90, 65.75) --
	( 32.90, 65.75) --
	( 32.90, 65.75) --
	( 32.96, 65.75) --
	( 32.96, 65.75) --
	( 32.96, 65.75) --
	( 32.98, 65.75) --
	( 32.98, 65.75) --
	( 32.98, 65.75) --
	( 33.00, 65.75) --
	( 33.00, 65.75) --
	( 33.00, 65.75) --
	( 33.05, 65.76) --
	( 33.05, 65.76) --
	( 33.05, 65.76) --
	( 33.08, 65.76) --
	( 33.08, 65.76) --
	( 33.08, 65.76) --
	( 33.12, 65.76) --
	( 33.12, 65.76) --
	( 33.12, 65.76) --
	( 33.13, 65.76) --
	( 33.13, 65.76) --
	( 33.13, 65.76) --
	( 33.16, 65.76) --
	( 33.16, 65.76) --
	( 33.16, 65.76) --
	( 33.20, 65.77) --
	( 33.20, 65.77) --
	( 33.20, 65.77) --
	( 33.20, 65.77) --
	( 33.20, 65.77) --
	( 33.20, 65.77) --
	( 33.28, 65.77) --
	( 33.28, 65.77) --
	( 33.28, 65.77) --
	( 33.28, 65.77) --
	( 33.28, 65.77) --
	( 33.28, 65.77) --
	( 33.35, 65.76) --
	( 33.35, 65.76) --
	( 33.35, 65.76) --
	( 33.43, 65.77) --
	( 33.43, 65.77) --
	( 33.43, 65.77) --
	( 33.44, 65.76) --
	( 33.44, 65.76) --
	( 33.44, 65.76) --
	( 33.50, 65.74) --
	( 33.50, 65.74) --
	( 33.50, 65.74) --
	( 33.58, 65.77) --
	( 33.58, 65.77) --
	( 33.58, 65.77) --
	( 33.64, 65.79) --
	( 33.64, 65.79) --
	( 33.64, 65.79) --
	( 33.65, 65.80) --
	( 33.65, 65.80) --
	( 33.65, 65.80) --
	( 33.73, 65.76) --
	( 33.73, 65.76) --
	( 33.73, 65.76) --
	( 33.80, 65.74) --
	( 33.80, 65.74) --
	( 33.80, 65.74) --
	( 33.85, 65.75) --
	( 33.85, 65.75) --
	( 33.85, 65.75) --
	( 33.88, 65.75) --
	( 33.88, 65.75) --
	( 33.88, 65.75) --
	( 33.89, 65.75) --
	( 33.89, 65.75) --
	( 33.89, 65.75) --
	( 33.95, 65.75) --
	( 33.95, 65.75) --
	( 33.95, 65.75) --
	( 34.03, 65.75) --
	( 34.03, 65.75) --
	( 34.03, 65.75) --
	( 34.05, 65.75) --
	( 34.05, 65.75) --
	( 34.05, 65.75) --
	( 34.09, 65.76) --
	( 34.09, 65.76) --
	( 34.09, 65.76) --
	( 34.10, 65.76) --
	( 34.10, 65.76) --
	( 34.10, 65.76) --
	( 34.17, 65.80) --
	( 34.17, 65.80) --
	( 34.17, 65.80) --
	( 34.21, 65.78) --
	( 34.21, 65.78) --
	( 34.21, 65.78) --
	( 34.25, 65.76) --
	( 34.25, 65.76) --
	( 34.25, 65.76) --
	( 34.25, 65.76) --
	( 34.25, 65.76) --
	( 34.25, 65.76) --
	( 34.32, 65.75) --
	( 34.32, 65.75) --
	( 34.32, 65.75) --
	( 34.40, 65.75) --
	( 34.40, 65.75) --
	( 34.40, 65.75) --
	( 34.41, 65.75) --
	( 34.41, 65.75) --
	( 34.41, 65.75) --
	( 34.47, 65.78) --
	( 34.47, 65.78) --
	( 34.47, 65.78) --
	( 34.53, 65.74) --
	( 34.53, 65.74) --
	( 34.53, 65.74) --
	( 34.55, 65.74) --
	( 34.55, 65.74) --
	( 34.55, 65.74) --
	( 34.61, 65.75) --
	( 34.61, 65.75) --
	( 34.61, 65.75) --
	( 34.62, 65.75) --
	( 34.62, 65.75) --
	( 34.62, 65.75) --
	( 34.70, 65.74) --
	( 34.70, 65.74) --
	( 34.70, 65.74) --
	( 34.70, 65.74) --
	( 34.70, 65.74) --
	( 34.70, 65.74) --
	( 34.74, 65.75) --
	( 34.74, 65.75) --
	( 34.74, 65.75) --
	( 34.77, 65.75) --
	( 34.77, 65.75) --
	( 34.77, 65.75) --
	( 34.82, 65.77) --
	( 34.82, 65.77) --
	( 34.82, 65.77) --
	( 34.85, 65.78) --
	( 34.85, 65.78) --
	( 34.85, 65.78) --
	( 34.86, 65.77) --
	( 34.86, 65.77) --
	( 34.86, 65.77) --
	( 34.87, 65.77) --
	( 34.87, 65.77) --
	( 34.87, 65.77) --
	( 34.92, 65.73) --
	( 34.92, 65.73) --
	( 34.92, 65.73) --
	( 34.98, 65.76) --
	( 34.98, 65.76) --
	( 34.98, 65.76) --
	( 35.00, 65.76) --
	( 35.00, 65.76) --
	( 35.00, 65.76) --
	( 35.06, 65.76) --
	( 35.06, 65.76) --
	( 35.06, 65.76) --
	( 35.07, 65.76) --
	( 35.07, 65.76) --
	( 35.07, 65.76) --
	( 35.15, 65.74) --
	( 35.15, 65.74) --
	( 35.15, 65.74) --
	( 35.18, 65.74) --
	( 35.18, 65.74) --
	( 35.18, 65.74) --
	( 35.22, 65.75) --
	( 35.22, 65.75) --
	( 35.22, 65.75) --
	( 35.30, 65.77) --
	( 35.30, 65.77) --
	( 35.30, 65.77) --
	( 35.34, 65.77) --
	( 35.34, 65.77) --
	( 35.34, 65.77) --
	( 35.37, 65.77) --
	( 35.37, 65.77) --
	( 35.37, 65.77) --
	( 35.38, 65.77) --
	( 35.38, 65.77) --
	( 35.38, 65.77) --
	( 35.45, 65.76) --
	( 35.45, 65.76) --
	( 35.45, 65.76) --
	( 35.50, 65.74) --
	( 35.50, 65.74) --
	( 35.50, 65.74) --
	( 35.52, 65.73) --
	( 35.52, 65.73) --
	( 35.52, 65.73) --
	( 35.58, 65.77) --
	( 35.58, 65.77) --
	( 35.58, 65.77) --
	( 35.60, 65.77) --
	( 35.60, 65.77) --
	( 35.60, 65.77) --
	( 35.67, 65.78) --
	( 35.67, 65.78) --
	( 35.67, 65.78) --
	( 35.71, 65.76) --
	( 35.71, 65.76) --
	( 35.71, 65.76) --
	( 35.74, 65.75) --
	( 35.74, 65.75) --
	( 35.74, 65.75) --
	( 35.82, 65.77) --
	( 35.82, 65.77) --
	( 35.82, 65.77) --
	( 35.82, 65.76) --
	( 35.82, 65.76) --
	( 35.82, 65.76) --
	( 35.83, 65.76) --
	( 35.83, 65.76) --
	( 35.83, 65.76) --
	( 35.87, 65.76) --
	( 35.87, 65.76) --
	( 35.87, 65.76) --
	( 35.89, 65.75) --
	( 35.89, 65.75) --
	( 35.89, 65.75) --
	( 35.97, 65.74) --
	( 35.97, 65.74) --
	( 35.97, 65.74) --
	( 36.04, 65.75) --
	( 36.04, 65.75) --
	( 36.04, 65.75) --
	( 36.11, 65.77) --
	( 36.11, 65.77) --
	( 36.11, 65.77) --
	( 36.12, 65.77) --
	( 36.12, 65.77) --
	( 36.12, 65.77) --
	( 36.15, 65.76) --
	( 36.15, 65.76) --
	( 36.15, 65.76) --
	( 36.19, 65.75) --
	( 36.19, 65.75) --
	( 36.19, 65.75) --
	( 36.23, 65.76) --
	( 36.23, 65.76) --
	( 36.23, 65.76) --
	( 36.27, 65.76) --
	( 36.27, 65.76) --
	( 36.27, 65.76) --
	( 36.34, 65.76) --
	( 36.34, 65.76) --
	( 36.34, 65.76) --
	( 36.42, 65.76) --
	( 36.42, 65.76) --
	( 36.42, 65.76) --
	( 36.43, 65.76) --
	( 36.43, 65.76) --
	( 36.43, 65.76) --
	( 36.49, 65.77) --
	( 36.49, 65.77) --
	( 36.49, 65.77) --
	( 36.57, 65.75) --
	( 36.57, 65.75) --
	( 36.57, 65.75) --
	( 36.64, 65.78) --
	( 36.64, 65.78) --
	( 36.64, 65.78) --
	( 36.71, 65.77) --
	( 36.71, 65.77) --
	( 36.71, 65.77) --
	( 36.76, 65.76) --
	( 36.76, 65.76) --
	( 36.76, 65.76) --
	( 36.79, 65.75) --
	( 36.79, 65.75) --
	( 36.79, 65.75) --
	( 36.86, 65.77) --
	( 36.86, 65.77) --
	( 36.86, 65.77) --
	( 36.92, 65.75) --
	( 36.92, 65.75) --
	( 36.92, 65.75) --
	( 36.94, 65.75) --
	( 36.94, 65.75) --
	( 36.94, 65.75) --
	( 37.01, 65.75) --
	( 37.01, 65.75) --
	( 37.01, 65.75) --
	( 37.09, 65.76) --
	( 37.09, 65.76) --
	( 37.09, 65.76) --
	( 37.16, 65.75) --
	( 37.16, 65.75) --
	( 37.16, 65.75) --
	( 37.16, 65.75) --
	( 37.16, 65.75) --
	( 37.16, 65.75) --
	( 37.24, 65.78) --
	( 37.24, 65.78) --
	( 37.24, 65.78) --
	( 37.31, 65.78) --
	( 37.31, 65.78) --
	( 37.31, 65.78) --
	( 37.32, 65.78) --
	( 37.32, 65.78) --
	( 37.32, 65.78) --
	( 37.39, 65.77) --
	( 37.39, 65.77) --
	( 37.39, 65.77) --
	( 37.46, 65.76) --
	( 37.46, 65.76) --
	( 37.46, 65.76) --
	( 37.54, 65.78) --
	( 37.54, 65.78) --
	( 37.54, 65.78) --
	( 37.61, 65.77) --
	( 37.61, 65.77) --
	( 37.61, 65.77) --
	( 37.61, 65.77) --
	( 37.61, 65.77) --
	( 37.61, 65.77) --
	( 37.65, 65.76) --
	( 37.65, 65.76) --
	( 37.65, 65.76) --
	( 37.69, 65.76) --
	( 37.69, 65.76) --
	( 37.69, 65.76) --
	( 37.76, 65.78) --
	( 37.76, 65.78) --
	( 37.76, 65.78) --
	( 37.83, 65.78) --
	( 37.83, 65.78) --
	( 37.83, 65.78) --
	( 37.91, 65.78) --
	( 37.91, 65.78) --
	( 37.91, 65.78) --
	( 37.97, 65.77) --
	( 37.97, 65.77) --
	( 37.97, 65.77) --
	( 37.98, 65.77) --
	( 37.98, 65.77) --
	( 37.98, 65.77) --
	( 38.06, 65.79) --
	( 38.06, 65.79) --
	( 38.06, 65.79) --
	( 38.13, 65.78) --
	( 38.13, 65.78) --
	( 38.13, 65.78) --
	( 38.21, 65.76) --
	( 38.21, 65.76) --
	( 38.21, 65.76) --
	( 38.28, 65.77) --
	( 38.28, 65.77) --
	( 38.28, 65.77) --
	( 38.33, 65.77) --
	( 38.33, 65.77) --
	( 38.33, 65.77) --
	( 38.36, 65.77) --
	( 38.36, 65.77) --
	( 38.36, 65.77) --
	( 38.43, 65.74) --
	( 38.43, 65.74) --
	( 38.43, 65.74) --
	( 38.51, 65.75) --
	( 38.51, 65.75) --
	( 38.51, 65.75) --
	( 38.58, 65.75) --
	( 38.58, 65.75) --
	( 38.58, 65.75) --
	( 38.65, 65.76) --
	( 38.65, 65.76) --
	( 38.65, 65.76) --
	( 38.73, 65.77) --
	( 38.73, 65.77) --
	( 38.73, 65.77) --
	( 38.80, 65.76) --
	( 38.80, 65.76) --
	( 38.80, 65.76) --
	( 38.82, 65.76) --
	( 38.82, 65.76) --
	( 38.82, 65.76) --
	( 38.88, 65.75) --
	( 38.88, 65.75) --
	( 38.88, 65.75) --
	( 38.95, 65.78) --
	( 38.95, 65.78) --
	( 38.95, 65.78) --
	( 38.99, 65.76) --
	( 38.99, 65.76) --
	( 38.99, 65.76) --
	( 39.03, 65.74) --
	( 39.03, 65.74) --
	( 39.03, 65.74) --
	( 39.10, 65.78) --
	( 39.10, 65.78) --
	( 39.10, 65.78) --
	( 39.14, 65.78) --
	( 39.14, 65.78) --
	( 39.14, 65.78) --
	( 39.18, 65.78) --
	( 39.18, 65.78) --
	( 39.18, 65.78) --
	( 39.25, 65.77) --
	( 39.25, 65.77) --
	( 39.25, 65.77) --
	( 39.33, 65.77) --
	( 39.33, 65.77) --
	( 39.33, 65.77) --
	( 39.40, 65.79) --
	( 39.40, 65.79) --
	( 39.40, 65.79) --
	( 39.47, 65.77) --
	( 39.47, 65.77) --
	( 39.47, 65.77) --
	( 39.47, 65.76) --
	( 39.47, 65.76) --
	( 39.47, 65.76) --
	( 39.51, 65.77) --
	( 39.51, 65.77) --
	( 39.51, 65.77) --
	( 39.55, 65.77) --
	( 39.55, 65.77) --
	( 39.55, 65.77) --
	( 39.62, 65.75) --
	( 39.62, 65.75) --
	( 39.62, 65.75) --
	( 39.67, 65.76) --
	( 39.67, 65.76) --
	( 39.67, 65.76) --
	( 39.70, 65.77) --
	( 39.70, 65.77) --
	( 39.70, 65.77) --
	( 39.71, 65.78) --
	( 39.71, 65.78) --
	( 39.71, 65.78) --
	( 39.77, 65.80) --
	( 39.77, 65.80) --
	( 39.77, 65.80) --
	( 39.79, 65.80) --
	( 39.79, 65.80) --
	( 39.79, 65.80) --
	( 39.83, 65.79) --
	( 39.83, 65.79) --
	( 39.83, 65.79) --
	( 39.85, 65.79) --
	( 39.85, 65.79) --
	( 39.85, 65.79) --
	( 39.87, 65.78) --
	( 39.87, 65.78) --
	( 39.87, 65.78) --
	( 39.91, 65.78) --
	( 39.91, 65.78) --
	( 39.91, 65.78) --
	( 39.92, 65.78) --
	( 39.92, 65.78) --
	( 39.92, 65.78) --
	( 40.00, 65.79) --
	( 40.00, 65.79) --
	( 40.00, 65.79) --
	( 40.03, 65.78) --
	( 40.03, 65.78) --
	( 40.03, 65.78) --
	( 40.07, 65.77) --
	( 40.07, 65.77) --
	( 40.07, 65.77) --
	( 40.14, 65.77) --
	( 40.14, 65.77) --
	( 40.14, 65.77) --
	( 40.22, 65.79) --
	( 40.22, 65.79) --
	( 40.22, 65.79) --
	( 40.29, 65.80) --
	( 40.29, 65.80) --
	( 40.29, 65.80) --
	( 40.35, 65.78) --
	( 40.35, 65.78) --
	( 40.35, 65.78) --
	( 40.37, 65.78) --
	( 40.37, 65.78) --
	( 40.37, 65.78) --
	( 40.44, 65.76) --
	( 40.44, 65.76) --
	( 40.44, 65.76) --
	( 40.52, 65.80) --
	( 40.52, 65.80) --
	( 40.52, 65.80) --
	( 40.52, 65.80) --
	( 40.52, 65.80) --
	( 40.52, 65.80) --
	( 40.59, 65.81) --
	( 40.59, 65.81) --
	( 40.59, 65.81) --
	( 40.67, 65.79) --
	( 40.67, 65.79) --
	( 40.67, 65.79) --
	( 40.74, 65.81) --
	( 40.74, 65.81) --
	( 40.74, 65.81) --
	( 40.81, 65.83) --
	( 40.81, 65.83) --
	( 40.81, 65.83) --
	( 40.89, 65.80) --
	( 40.89, 65.80) --
	( 40.89, 65.80) --
	( 40.92, 65.81) --
	( 40.92, 65.81) --
	( 40.92, 65.81) --
	( 40.96, 65.82) --
	( 40.96, 65.82) --
	( 40.96, 65.82) --
	( 41.04, 65.83) --
	( 41.04, 65.83) --
	( 41.04, 65.83) --
	( 41.11, 65.82) --
	( 41.11, 65.82) --
	( 41.11, 65.82) --
	( 41.19, 65.84) --
	( 41.19, 65.84) --
	( 41.19, 65.84) --
	( 41.26, 65.84) --
	( 41.26, 65.84) --
	( 41.26, 65.84) --
	( 41.29, 65.85) --
	( 41.29, 65.85) --
	( 41.29, 65.85) --
	( 41.33, 65.87) --
	( 41.33, 65.87) --
	( 41.33, 65.87) --
	( 41.41, 65.87) --
	( 41.41, 65.87) --
	( 41.41, 65.87) --
	( 41.48, 65.86) --
	( 41.48, 65.86) --
	( 41.48, 65.86) --
	( 41.56, 65.88) --
	( 41.56, 65.88) --
	( 41.56, 65.88) --
	( 41.63, 65.92) --
	( 41.63, 65.92) --
	( 41.63, 65.92) --
	( 41.67, 65.92) --
	( 41.67, 65.92) --
	( 41.67, 65.92) --
	( 41.71, 65.91) --
	( 41.71, 65.91) --
	( 41.71, 65.91) --
	( 41.78, 65.92) --
	( 41.78, 65.92) --
	( 41.78, 65.92) --
	( 41.86, 65.95) --
	( 41.86, 65.95) --
	( 41.86, 65.95) --
	( 41.93, 65.95) --
	( 41.93, 65.95) --
	( 41.93, 65.95) --
	( 42.00, 66.02) --
	( 42.00, 66.02) --
	( 42.00, 66.02) --
	( 42.05, 66.00) --
	( 42.05, 66.00) --
	( 42.05, 66.00) --
	( 42.08, 65.98) --
	( 42.08, 65.98) --
	( 42.08, 65.98) --
	( 42.15, 66.07) --
	( 42.15, 66.07) --
	( 42.15, 66.07) --
	( 42.21, 66.09) --
	( 42.21, 66.09) --
	( 42.21, 66.09) --
	( 42.23, 66.09) --
	( 42.23, 66.09) --
	( 42.23, 66.09) --
	( 42.30, 66.10) --
	( 42.30, 66.10) --
	( 42.30, 66.10) --
	( 42.38, 66.15) --
	( 42.38, 66.15) --
	( 42.38, 66.15) --
	( 42.45, 66.20) --
	( 42.45, 66.20) --
	( 42.45, 66.20) --
	( 42.52, 66.20) --
	( 42.52, 66.20) --
	( 42.52, 66.20) --
	( 42.53, 66.20) --
	( 42.53, 66.20) --
	( 42.53, 66.20) --
	( 42.60, 66.22) --
	( 42.60, 66.22) --
	( 42.60, 66.22) --
	( 42.67, 66.31) --
	( 42.67, 66.31) --
	( 42.67, 66.31) --
	( 42.75, 66.37) --
	( 42.75, 66.37) --
	( 42.75, 66.37) --
	( 42.82, 66.41) --
	( 42.82, 66.41) --
	( 42.82, 66.41) --
	( 42.82, 66.41) --
	( 42.82, 66.41) --
	( 42.82, 66.41) --
	( 42.90, 66.48) --
	( 42.90, 66.48) --
	( 42.90, 66.48) --
	( 42.97, 66.53) --
	( 42.97, 66.53) --
	( 42.97, 66.53) --
	( 43.01, 66.61) --
	( 43.01, 66.61) --
	( 43.01, 66.61) --
	( 43.04, 66.67) --
	( 43.04, 66.67) --
	( 43.04, 66.67) --
	( 43.12, 66.71) --
	( 43.12, 66.71) --
	( 43.12, 66.71) --
	( 43.19, 66.72) --
	( 43.19, 66.72) --
	( 43.19, 66.72) --
	( 43.27, 66.89) --
	( 43.27, 66.89) --
	( 43.27, 66.89) --
	( 43.30, 66.93) --
	( 43.30, 66.93) --
	( 43.30, 66.93) --
	( 43.34, 66.98) --
	( 43.34, 66.98) --
	( 43.34, 66.98) --
	( 43.42, 67.08) --
	( 43.42, 67.08) --
	( 43.42, 67.08) --
	( 43.49, 67.15) --
	( 43.49, 67.15) --
	( 43.49, 67.15) --
	( 43.49, 67.15) --
	( 43.49, 67.15) --
	( 43.49, 67.15) --
	( 43.56, 67.34) --
	( 43.56, 67.34) --
	( 43.56, 67.34) --
	( 43.59, 67.38) --
	( 43.59, 67.38) --
	( 43.59, 67.38) --
	( 43.64, 67.46) --
	( 43.64, 67.46) --
	( 43.64, 67.46) --
	( 43.71, 67.64) --
	( 43.71, 67.64) --
	( 43.71, 67.64) --
	( 43.79, 67.81) --
	( 43.79, 67.81) --
	( 43.79, 67.81) --
	( 43.86, 67.90) --
	( 43.86, 67.90) --
	( 43.86, 67.90) --
	( 43.94, 68.07) --
	( 43.94, 68.07) --
	( 43.94, 68.07) --
	( 43.97, 68.14) --
	( 43.97, 68.14) --
	( 43.97, 68.14) --
	( 44.01, 68.22) --
	( 44.01, 68.22) --
	( 44.01, 68.22) --
	( 44.08, 68.61) --
	( 44.08, 68.61) --
	( 44.08, 68.61) --
	( 44.16, 68.74) --
	( 44.16, 68.74) --
	( 44.16, 68.74) --
	( 44.16, 68.75) --
	( 44.16, 68.75) --
	( 44.16, 68.75) --
	( 44.23, 68.97) --
	( 44.23, 68.97) --
	( 44.23, 68.97) --
	( 44.31, 69.22) --
	( 44.31, 69.22) --
	( 44.31, 69.22) --
	( 44.38, 69.51) --
	( 44.38, 69.51) --
	( 44.38, 69.51) --
	( 44.45, 69.90) --
	( 44.45, 69.90) --
	( 44.45, 69.90) --
	( 44.46, 69.94) --
	( 44.46, 69.94) --
	( 44.46, 69.94) --
	( 44.53, 70.34) --
	( 44.53, 70.34) --
	( 44.53, 70.34) --
	( 44.60, 70.65) --
	( 44.60, 70.65) --
	( 44.60, 70.65) --
	( 44.64, 70.76) --
	( 44.64, 70.76) --
	( 44.64, 70.76) --
	( 44.68, 70.88) --
	( 44.68, 70.88) --
	( 44.68, 70.88) --
	( 44.74, 71.16) --
	( 44.74, 71.16) --
	( 44.74, 71.16) --
	( 44.75, 71.23) --
	( 44.75, 71.23) --
	( 44.75, 71.23) --
	( 44.83, 71.82) --
	( 44.83, 71.82) --
	( 44.83, 71.82) --
	( 44.90, 72.27) --
	( 44.90, 72.27) --
	( 44.90, 72.27) --
	( 44.93, 72.40) --
	( 44.93, 72.40) --
	( 44.93, 72.40) --
	( 44.98, 72.64) --
	( 44.98, 72.64) --
	( 44.98, 72.64) --
	( 45.02, 73.21) --
	( 45.02, 73.21) --
	( 45.02, 73.21) --
	( 45.05, 73.50) --
	( 45.05, 73.50) --
	( 45.05, 73.50) --
	( 45.12, 73.59) --
	( 45.12, 73.59) --
	( 45.12, 73.59) --
	( 45.20, 74.18) --
	( 45.20, 74.18) --
	( 45.20, 74.18) --
	( 45.27, 74.83) --
	( 45.27, 74.83) --
	( 45.27, 74.83) --
	( 45.31, 75.22) --
	( 45.31, 75.22) --
	( 45.31, 75.22) --
	( 45.35, 75.54) --
	( 45.35, 75.54) --
	( 45.35, 75.54) --
	( 45.41, 75.96) --
	( 45.41, 75.96) --
	( 45.41, 75.96) --
	( 45.42, 76.04) --
	( 45.42, 76.04) --
	( 45.42, 76.04) --
	( 45.49, 76.82) --
	( 45.49, 76.82) --
	( 45.49, 76.82) --
	( 45.50, 76.85) --
	( 45.50, 76.85) --
	( 45.50, 76.85) --
	( 45.57, 77.05) --
	( 45.57, 77.05) --
	( 45.57, 77.05) --
	( 45.60, 77.38) --
	( 45.60, 77.38) --
	( 45.60, 77.38) --
	( 45.64, 77.82) --
	( 45.64, 77.82) --
	( 45.64, 77.82) --
	( 45.70, 78.40) --
	( 45.70, 78.40) --
	( 45.70, 78.40) --
	( 45.72, 78.64) --
	( 45.72, 78.64) --
	( 45.72, 78.64) --
	( 45.79, 79.50) --
	( 45.79, 79.50) --
	( 45.79, 79.50) --
	( 45.79, 79.50) --
	( 45.79, 79.50) --
	( 45.79, 79.50) --
	( 45.87, 79.65) --
	( 45.87, 79.65) --
	( 45.87, 79.65) --
	( 45.94, 80.15) --
	( 45.94, 80.15) --
	( 45.94, 80.15) --
	( 45.98, 80.50) --
	( 45.98, 80.50) --
	( 45.98, 80.50) --
	( 46.01, 80.74) --
	( 46.01, 80.74) --
	( 46.01, 80.74) --
	( 46.08, 81.24) --
	( 46.08, 81.24) --
	( 46.08, 81.24) --
	( 46.09, 81.30) --
	( 46.09, 81.30) --
	( 46.09, 81.30) --
	( 46.16, 81.22) --
	( 46.16, 81.22) --
	( 46.16, 81.22) --
	( 46.17, 81.40) --
	( 46.17, 81.40) --
	( 46.17, 81.40) --
	( 46.24, 82.22) --
	( 46.24, 82.22) --
	( 46.24, 82.22) --
	( 46.31, 81.99) --
	( 46.31, 81.99) --
	( 46.31, 81.99) --
	( 46.38, 82.16) --
	( 46.38, 82.16) --
	( 46.38, 82.16) --
	( 46.46, 82.13) --
	( 46.46, 82.13) --
	( 46.46, 82.13) --
	( 46.46, 82.15) --
	( 46.46, 82.15) --
	( 46.46, 82.15) --
	( 46.53, 82.54) --
	( 46.53, 82.54) --
	( 46.53, 82.54) --
	( 46.61, 82.41) --
	( 46.61, 82.41) --
	( 46.61, 82.41) --
	( 46.68, 82.06) --
	( 46.68, 82.06) --
	( 46.68, 82.06) --
	( 46.75, 82.16) --
	( 46.75, 82.16) --
	( 46.75, 82.16) --
	( 46.75, 82.16) --
	( 46.75, 82.16) --
	( 46.75, 82.16) --
	( 46.83, 81.51) --
	( 46.83, 81.51) --
	( 46.83, 81.51) --
	( 46.85, 81.53) --
	( 46.85, 81.53) --
	( 46.85, 81.53) --
	( 46.90, 81.59) --
	( 46.90, 81.59) --
	( 46.90, 81.59) --
	( 46.94, 81.32) --
	( 46.94, 81.32) --
	( 46.94, 81.32) --
	( 46.98, 81.08) --
	( 46.98, 81.08) --
	( 46.98, 81.08) --
	( 47.05, 80.50) --
	( 47.05, 80.50) --
	( 47.05, 80.50) --
	( 47.13, 79.87) --
	( 47.13, 79.87) --
	( 47.13, 79.87) --
	( 47.13, 79.83) --
	( 47.13, 79.83) --
	( 47.13, 79.83) --
	( 47.20, 79.50) --
	( 47.20, 79.50) --
	( 47.20, 79.50) --
	( 47.23, 79.21) --
	( 47.23, 79.21) --
	( 47.23, 79.21) --
	( 47.27, 78.77) --
	( 47.27, 78.77) --
	( 47.27, 78.77) --
	( 47.32, 78.62) --
	( 47.32, 78.62) --
	( 47.32, 78.62) --
	( 47.35, 78.56) --
	( 47.35, 78.56) --
	( 47.35, 78.56) --
	( 47.42, 78.58) --
	( 47.42, 78.58) --
	( 47.42, 78.58) --
	( 47.50, 77.41) --
	( 47.50, 77.41) --
	( 47.50, 77.41) --
	( 47.52, 77.30) --
	( 47.52, 77.30) --
	( 47.52, 77.30) --
	( 47.57, 77.02) --
	( 47.57, 77.02) --
	( 47.57, 77.02) --
	( 47.61, 76.63) --
	( 47.61, 76.63) --
	( 47.61, 76.63) --
	( 47.64, 76.34) --
	( 47.64, 76.34) --
	( 47.64, 76.34) --
	( 47.72, 76.19) --
	( 47.72, 76.19) --
	( 47.72, 76.19) --
	( 47.79, 75.74) --
	( 47.79, 75.74) --
	( 47.79, 75.74) --
	( 47.80, 75.71) --
	( 47.80, 75.71) --
	( 47.80, 75.71) --
	( 47.86, 75.58) --
	( 47.86, 75.58) --
	( 47.86, 75.58) --
	( 47.90, 75.23) --
	( 47.90, 75.23) --
	( 47.90, 75.23) --
	( 47.94, 74.84) --
	( 47.94, 74.84) --
	( 47.94, 74.84) --
	( 48.00, 74.40) --
	( 48.00, 74.40) --
	( 48.00, 74.40) --
	( 48.01, 74.26) --
	( 48.01, 74.26) --
	( 48.01, 74.26) --
	( 48.09, 74.08) --
	( 48.09, 74.08) --
	( 48.09, 74.08) --
	( 48.16, 73.97) --
	( 48.16, 73.97) --
	( 48.16, 73.97) --
	( 48.19, 73.84) --
	( 48.19, 73.84) --
	( 48.19, 73.84) --
	( 48.23, 73.62) --
	( 48.23, 73.62) --
	( 48.23, 73.62) --
	( 48.28, 73.26) --
	( 48.28, 73.26) --
	( 48.28, 73.26) --
	( 48.31, 73.07) --
	( 48.31, 73.07) --
	( 48.31, 73.07) --
	( 48.38, 72.89) --
	( 48.38, 72.89) --
	( 48.38, 72.89) --
	( 48.46, 72.70) --
	( 48.46, 72.70) --
	( 48.46, 72.70) --
	( 48.47, 72.71) --
	( 48.47, 72.71) --
	( 48.47, 72.71) --
	( 48.53, 72.76) --
	( 48.53, 72.76) --
	( 48.53, 72.76) --
	( 48.60, 72.60) --
	( 48.60, 72.60) --
	( 48.60, 72.60) --
	( 48.67, 72.16) --
	( 48.67, 72.16) --
	( 48.67, 72.16) --
	( 48.68, 72.07) --
	( 48.68, 72.07) --
	( 48.68, 72.07) --
	( 48.75, 71.87) --
	( 48.75, 71.87) --
	( 48.75, 71.87) --
	( 48.76, 71.84) --
	( 48.76, 71.84) --
	( 48.76, 71.84) --
	( 48.83, 71.66) --
	( 48.83, 71.66) --
	( 48.83, 71.66) --
	( 48.90, 71.49) --
	( 48.90, 71.49) --
	( 48.90, 71.49) --
	( 48.95, 71.46) --
	( 48.95, 71.46) --
	( 48.95, 71.46) --
	( 48.97, 71.44) --
	( 48.97, 71.44) --
	( 48.97, 71.44) --
	( 49.05, 71.30) --
	( 49.05, 71.30) --
	( 49.05, 71.29) --
	( 49.05, 71.29) --
	( 49.05, 71.29) --
	( 49.05, 71.29) --
	( 49.12, 71.10) --
	( 49.12, 71.10) --
	( 49.12, 71.10) --
	( 49.20, 70.77) --
	( 49.20, 70.77) --
	( 49.20, 70.77) --
	( 49.27, 70.83) --
	( 49.27, 70.83) --
	( 49.27, 70.83) --
	( 49.34, 70.61) --
	( 49.34, 70.61) --
	( 49.34, 70.61) --
	( 49.34, 70.58) --
	( 49.34, 70.58) --
	( 49.34, 70.58) --
	( 49.42, 70.32) --
	( 49.42, 70.32) --
	( 49.42, 70.32) --
	( 49.49, 70.30) --
	( 49.49, 70.30) --
	( 49.49, 70.30) --
	( 49.53, 70.18) --
	( 49.53, 70.18) --
	( 49.53, 70.18) --
	( 49.57, 70.07) --
	( 49.57, 70.07) --
	( 49.57, 70.07) --
	( 49.64, 69.93) --
	( 49.64, 69.93) --
	( 49.64, 69.93) --
	( 49.71, 69.78) --
	( 49.71, 69.78) --
	( 49.71, 69.78) --
	( 49.72, 69.75) --
	( 49.72, 69.75) --
	( 49.72, 69.75) --
	( 49.79, 69.49) --
	( 49.79, 69.49) --
	( 49.79, 69.49) --
	( 49.86, 69.37) --
	( 49.86, 69.37) --
	( 49.86, 69.37) --
	( 49.91, 69.35) --
	( 49.91, 69.35) --
	( 49.91, 69.35) --
	( 49.94, 69.34) --
	( 49.94, 69.34) --
	( 49.94, 69.34) --
	( 50.01, 69.16) --
	( 50.01, 69.16) --
	( 50.01, 69.16) --
	( 50.08, 68.90) --
	( 50.08, 68.90) --
	( 50.08, 68.90) --
	( 50.10, 68.87) --
	( 50.10, 68.87) --
	( 50.10, 68.86) --
	( 50.16, 68.77) --
	( 50.16, 68.77) --
	( 50.16, 68.77) --
	( 50.23, 68.69) --
	( 50.23, 68.69) --
	( 50.23, 68.69) --
	( 50.30, 68.56) --
	( 50.30, 68.56) --
	( 50.30, 68.56) --
	( 50.38, 68.47) --
	( 50.38, 68.47) --
	( 50.38, 68.47) --
	( 50.45, 68.28) --
	( 50.45, 68.28) --
	( 50.45, 68.28) --
	( 50.49, 68.27) --
	( 50.49, 68.27) --
	( 50.49, 68.27) --
	( 50.53, 68.26) --
	( 50.53, 68.26) --
	( 50.53, 68.26) --
	( 50.60, 68.10) --
	( 50.60, 68.10) --
	( 50.60, 68.10) --
	( 50.67, 68.08) --
	( 50.67, 68.08) --
	( 50.67, 68.08) --
	( 50.75, 68.01) --
	( 50.75, 68.01) --
	( 50.75, 68.01) --
	( 50.82, 67.89) --
	( 50.82, 67.89) --
	( 50.82, 67.89) --
	( 50.87, 67.87) --
	( 50.87, 67.87) --
	( 50.87, 67.87) --
	( 50.90, 67.85) --
	( 50.90, 67.85) --
	( 50.90, 67.85) --
	( 50.97, 67.76) --
	( 50.97, 67.76) --
	( 50.97, 67.76) --
	( 50.97, 67.76) --
	( 50.97, 67.76) --
	( 50.97, 67.76) --
	( 51.04, 67.75) --
	( 51.04, 67.75) --
	( 51.04, 67.75) --
	( 51.12, 67.63) --
	( 51.12, 67.63) --
	( 51.12, 67.63) --
	( 51.19, 67.60) --
	( 51.19, 67.60) --
	( 51.19, 67.60) --
	( 51.27, 67.50) --
	( 51.27, 67.50) --
	( 51.27, 67.50) --
	( 51.34, 67.48) --
	( 51.34, 67.48) --
	( 51.34, 67.48) --
	( 51.35, 67.48) --
	( 51.35, 67.48) --
	( 51.35, 67.48) --
	( 51.41, 67.47) --
	( 51.41, 67.47) --
	( 51.41, 67.47) --
	( 51.49, 67.42) --
	( 51.49, 67.42) --
	( 51.49, 67.42) --
	( 51.56, 67.31) --
	( 51.56, 67.31) --
	( 51.56, 67.31) --
	( 51.63, 67.33) --
	( 51.63, 67.33) --
	( 51.63, 67.33) --
	( 51.71, 67.25) --
	( 51.71, 67.25) --
	( 51.71, 67.25) --
	( 51.73, 67.24) --
	( 51.73, 67.24) --
	( 51.73, 67.24) --
	( 51.78, 67.22) --
	( 51.78, 67.22) --
	( 51.78, 67.22) --
	( 51.86, 67.16) --
	( 51.86, 67.16) --
	( 51.86, 67.16) --
	( 51.93, 67.13) --
	( 51.93, 67.13) --
	( 51.93, 67.13) --
	( 52.00, 67.04) --
	( 52.00, 67.04) --
	( 52.00, 67.04) --
	( 52.08, 67.01) --
	( 52.08, 67.01) --
	( 52.08, 67.01) --
	( 52.15, 67.01) --
	( 52.15, 67.01) --
	( 52.15, 67.01) --
	( 52.21, 66.93) --
	( 52.21, 66.93) --
	( 52.21, 66.93) --
	( 52.22, 66.91) --
	( 52.22, 66.91) --
	( 52.22, 66.91) --
	( 52.30, 66.87) --
	( 52.30, 66.87) --
	( 52.30, 66.87) --
	( 52.37, 66.85) --
	( 52.37, 66.85) --
	( 52.37, 66.85) --
	( 52.45, 66.76) --
	( 52.45, 66.76) --
	( 52.45, 66.76) --
	( 52.52, 66.76) --
	( 52.52, 66.76) --
	( 52.52, 66.76) --
	( 52.59, 66.70) --
	( 52.59, 66.70) --
	( 52.59, 66.70) --
	( 52.67, 66.71) --
	( 52.67, 66.71) --
	( 52.67, 66.71) --
	( 52.69, 66.70) --
	( 52.69, 66.70) --
	( 52.69, 66.70) --
	( 52.74, 66.67) --
	( 52.74, 66.67) --
	( 52.74, 66.67) --
	( 52.81, 66.59) --
	( 52.81, 66.59) --
	( 52.81, 66.59) --
	( 52.89, 66.56) --
	( 52.89, 66.56) --
	( 52.89, 66.56) --
	( 52.96, 66.53) --
	( 52.96, 66.53) --
	( 52.96, 66.53) --
	( 53.04, 66.47) --
	( 53.04, 66.47) --
	( 53.04, 66.47) --
	( 53.11, 66.49) --
	( 53.11, 66.49) --
	( 53.11, 66.49) --
	( 53.17, 66.47) --
	( 53.17, 66.47) --
	( 53.17, 66.47) --
	( 53.18, 66.47) --
	( 53.18, 66.47) --
	( 53.18, 66.47) --
	( 53.26, 66.39) --
	( 53.26, 66.39) --
	( 53.26, 66.39) --
	( 53.33, 66.39) --
	( 53.33, 66.39) --
	( 53.33, 66.39) --
	( 53.41, 66.39) --
	( 53.41, 66.39) --
	( 53.41, 66.39) --
	( 53.48, 66.34) --
	( 53.48, 66.34) --
	( 53.48, 66.34) --
	( 53.55, 66.30) --
	( 53.55, 66.30) --
	( 53.55, 66.30) --
	( 53.62, 66.31) --
	( 53.62, 66.31) --
	( 53.62, 66.31) --
	( 53.70, 66.26) --
	( 53.70, 66.26) --
	( 53.70, 66.26) --
	( 53.75, 66.24) --
	( 53.75, 66.24) --
	( 53.75, 66.24) --
	( 53.77, 66.23) --
	( 53.77, 66.23) --
	( 53.77, 66.23) --
	( 53.85, 66.20) --
	( 53.85, 66.20) --
	( 53.85, 66.20) --
	( 53.92, 66.22) --
	( 53.92, 66.22) --
	( 53.92, 66.22) --
	( 53.99, 66.19) --
	( 53.99, 66.19) --
	( 53.99, 66.19) --
	( 54.07, 66.14) --
	( 54.07, 66.14) --
	( 54.07, 66.14) --
	( 54.14, 66.15) --
	( 54.14, 66.15) --
	( 54.14, 66.15) --
	( 54.21, 66.16) --
	( 54.21, 66.16) --
	( 54.21, 66.16) --
	( 54.29, 66.11) --
	( 54.29, 66.11) --
	( 54.29, 66.11) --
	( 54.36, 66.12) --
	( 54.36, 66.12) --
	( 54.36, 66.12) --
	( 54.44, 66.09) --
	( 54.44, 66.09) --
	( 54.44, 66.09) --
	( 54.51, 66.08) --
	( 54.51, 66.08) --
	( 54.51, 66.08) --
	( 54.51, 66.08) --
	( 54.51, 66.08) --
	( 54.51, 66.08) --
	( 54.58, 66.08) --
	( 54.58, 66.08) --
	( 54.58, 66.08) --
	( 54.66, 66.04) --
	( 54.66, 66.04) --
	( 54.66, 66.04) --
	( 54.73, 66.03) --
	( 54.73, 66.03) --
	( 54.73, 66.03) --
	( 54.80, 66.02) --
	( 54.80, 66.02) --
	( 54.80, 66.02) --
	( 54.88, 65.99) --
	( 54.88, 65.99) --
	( 54.88, 65.99) --
	( 54.95, 65.99) --
	( 54.95, 65.99) --
	( 54.95, 65.99) --
	( 55.02, 66.00) --
	( 55.02, 66.00) --
	( 55.02, 66.00) --
	( 55.10, 65.96) --
	( 55.10, 65.96) --
	( 55.10, 65.96) --
	( 55.17, 65.98) --
	( 55.17, 65.98) --
	( 55.17, 65.98) --
	( 55.25, 65.96) --
	( 55.25, 65.96) --
	( 55.25, 65.96) --
	( 55.28, 65.96) --
	( 55.28, 65.96) --
	( 55.28, 65.96) --
	( 55.32, 65.96) --
	( 55.32, 65.96) --
	( 55.32, 65.96) --
	( 55.39, 65.92) --
	( 55.39, 65.92) --
	( 55.39, 65.92) --
	( 55.47, 65.90) --
	( 55.47, 65.90) --
	( 55.47, 65.90) --
	( 55.54, 65.96) --
	( 55.54, 65.96) --
	( 55.54, 65.96) --
	( 55.57, 65.95) --
	( 55.57, 65.95) --
	( 55.57, 65.95) --
	( 55.61, 65.92) --
	( 55.61, 65.92) --
	( 55.61, 65.92) --
	( 55.69, 65.90) --
	( 55.69, 65.90) --
	( 55.69, 65.90) --
	( 55.76, 65.90) --
	( 55.76, 65.90) --
	( 55.76, 65.90) --
	( 55.83, 65.93) --
	( 55.83, 65.93) --
	( 55.83, 65.93) --
	( 55.91, 65.89) --
	( 55.91, 65.89) --
	( 55.91, 65.89) --
	( 55.98, 65.87) --
	( 55.98, 65.87) --
	( 55.98, 65.87) --
	( 56.06, 65.90) --
	( 56.06, 65.90) --
	( 56.06, 65.90) --
	( 56.13, 65.86) --
	( 56.13, 65.86) --
	( 56.13, 65.86) --
	( 56.14, 65.86) --
	( 56.14, 65.86) --
	( 56.14, 65.86) --
	( 56.20, 65.87) --
	( 56.20, 65.87) --
	( 56.20, 65.87) --
	( 56.27, 65.85) --
	( 56.27, 65.85) --
	( 56.27, 65.85) --
	( 56.35, 65.85) --
	( 56.35, 65.85) --
	( 56.35, 65.85) --
	( 56.42, 65.87) --
	( 56.42, 65.87) --
	( 56.42, 65.87) --
	( 56.43, 65.87) --
	( 56.43, 65.87) --
	( 56.43, 65.87) --
	( 56.50, 65.86) --
	( 56.50, 65.86) --
	( 56.50, 65.86) --
	( 56.57, 65.86) --
	( 56.57, 65.86) --
	( 56.57, 65.86) --
	( 56.64, 65.89) --
	( 56.64, 65.89) --
	( 56.64, 65.89) --
	( 56.72, 65.84) --
	( 56.72, 65.84) --
	( 56.72, 65.84) --
	( 56.79, 65.83) --
	( 56.79, 65.83) --
	( 56.79, 65.83) --
	( 56.86, 65.83) --
	( 56.86, 65.83) --
	( 56.86, 65.83) --
	( 56.91, 65.84) --
	( 56.91, 65.84) --
	( 56.91, 65.84) --
	( 56.94, 65.84) --
	( 56.94, 65.84) --
	( 56.94, 65.84) --
	( 57.01, 65.85) --
	( 57.01, 65.85) --
	( 57.01, 65.85) --
	( 57.08, 65.82) --
	( 57.08, 65.82) --
	( 57.08, 65.82) --
	( 57.16, 65.82) --
	( 57.16, 65.82) --
	( 57.16, 65.82) --
	( 57.20, 65.83) --
	( 57.20, 65.83) --
	( 57.20, 65.83) --
	( 57.23, 65.83) --
	( 57.23, 65.83) --
	( 57.23, 65.83) --
	( 57.30, 65.80) --
	( 57.30, 65.80) --
	( 57.30, 65.80) --
	( 57.38, 65.81) --
	( 57.38, 65.81) --
	( 57.38, 65.81) --
	( 57.45, 65.83) --
	( 57.45, 65.83) --
	( 57.45, 65.83) --
	( 57.52, 65.81) --
	( 57.52, 65.81) --
	( 57.52, 65.81) --
	( 57.60, 65.82) --
	( 57.60, 65.82) --
	( 57.60, 65.82) --
	( 57.67, 65.81) --
	( 57.67, 65.81) --
	( 57.67, 65.81) --
	( 57.74, 65.82) --
	( 57.74, 65.82) --
	( 57.74, 65.82) --
	( 57.82, 65.83) --
	( 57.82, 65.83) --
	( 57.82, 65.83) --
	( 57.87, 65.79) --
	( 57.87, 65.79) --
	( 57.87, 65.79) --
	( 57.89, 65.77) --
	( 57.89, 65.77) --
	( 57.89, 65.77) --
	( 57.97, 65.81) --
	( 57.97, 65.81) --
	( 57.97, 65.81) --
	( 58.04, 65.83) --
	( 58.04, 65.83) --
	( 58.04, 65.83) --
	( 58.11, 65.80) --
	( 58.11, 65.80) --
	( 58.11, 65.80) --
	( 58.18, 65.80) --
	( 58.18, 65.80) --
	( 58.18, 65.80) --
	( 58.26, 65.82) --
	( 58.26, 65.82) --
	( 58.26, 65.82) --
	( 58.33, 65.78) --
	( 58.33, 65.78) --
	( 58.33, 65.78) --
	( 58.35, 65.78) --
	( 58.35, 65.78) --
	( 58.35, 65.78) --
	( 58.41, 65.80) --
	( 58.41, 65.80) --
	( 58.41, 65.80) --
	( 58.48, 65.79) --
	( 58.48, 65.79) --
	( 58.48, 65.79) --
	( 58.55, 65.80) --
	( 58.55, 65.80) --
	( 58.55, 65.80) --
	( 58.63, 65.79) --
	( 58.63, 65.79) --
	( 58.63, 65.79) --
	( 58.70, 65.77) --
	( 58.70, 65.77) --
	( 58.70, 65.77) --
	( 58.77, 65.79) --
	( 58.77, 65.79) --
	( 58.77, 65.79) --
	( 58.85, 65.79) --
	( 58.85, 65.79) --
	( 58.85, 65.79) --
	( 58.92, 65.78) --
	( 58.92, 65.78) --
	( 58.92, 65.78) --
	( 58.92, 65.78) --
	( 58.92, 65.78) --
	( 58.92, 65.78) --
	( 58.99, 65.79) --
	( 58.99, 65.79) --
	( 58.99, 65.79) --
	( 59.07, 65.79) --
	( 59.07, 65.79) --
	( 59.07, 65.79) --
	( 59.14, 65.77) --
	( 59.14, 65.77) --
	( 59.14, 65.77) --
	( 59.21, 65.77) --
	( 59.21, 65.77) --
	( 59.21, 65.77) --
	( 59.29, 65.79) --
	( 59.29, 65.79) --
	( 59.29, 65.79) --
	( 59.36, 65.80) --
	( 59.36, 65.80) --
	( 59.36, 65.80) --
	( 59.43, 65.77) --
	( 59.43, 65.77) --
	( 59.43, 65.77) --
	( 59.50, 65.78) --
	( 59.50, 65.78) --
	( 59.50, 65.78) --
	( 59.51, 65.78) --
	( 59.51, 65.78) --
	( 59.51, 65.78) --
	( 59.58, 65.78) --
	( 59.58, 65.78) --
	( 59.58, 65.78) --
	( 59.65, 65.78) --
	( 59.65, 65.78) --
	( 59.65, 65.78) --
	( 59.73, 65.78) --
	( 59.73, 65.78) --
	( 59.73, 65.78) --
	( 59.80, 65.77) --
	( 59.80, 65.77) --
	( 59.80, 65.77) --
	( 59.87, 65.80) --
	( 59.87, 65.80) --
	( 59.87, 65.80) --
	( 59.95, 65.77) --
	( 59.95, 65.77) --
	( 59.95, 65.77) --
	( 60.02, 65.77) --
	( 60.02, 65.77) --
	( 60.02, 65.77) --
	( 60.09, 65.78) --
	( 60.09, 65.78) --
	( 60.09, 65.78) --
	( 60.17, 65.78) --
	( 60.17, 65.78) --
	( 60.17, 65.78) --
	( 60.24, 65.78) --
	( 60.24, 65.78) --
	( 60.24, 65.78) --
	( 60.31, 65.77) --
	( 60.31, 65.77) --
	( 60.31, 65.77) --
	( 60.39, 65.79) --
	( 60.39, 65.79) --
	( 60.39, 65.79) --
	( 60.45, 65.79) --
	( 60.45, 65.79) --
	( 60.45, 65.79) --
	( 60.46, 65.79) --
	( 60.46, 65.79) --
	( 60.46, 65.79) --
	( 60.53, 65.77) --
	( 60.53, 65.77) --
	( 60.53, 65.77) --
	( 60.61, 65.77) --
	( 60.61, 65.77) --
	( 60.61, 65.77) --
	( 60.68, 65.77) --
	( 60.68, 65.77) --
	( 60.68, 65.77) --
	( 60.75, 65.75) --
	( 60.75, 65.75) --
	( 60.75, 65.75) --
	( 60.83, 65.77) --
	( 60.83, 65.77) --
	( 60.83, 65.77) --
	( 60.90, 65.76) --
	( 60.90, 65.76) --
	( 60.90, 65.76) --
	( 60.97, 65.76) --
	( 60.97, 65.76) --
	( 60.97, 65.76) --
	( 61.04, 65.77) --
	( 61.04, 65.77) --
	( 61.04, 65.77) --
	( 61.12, 65.76) --
	( 61.12, 65.76) --
	( 61.12, 65.76) --
	( 61.19, 65.76) --
	( 61.19, 65.76) --
	( 61.19, 65.76) --
	( 61.26, 65.78) --
	( 61.26, 65.78) --
	( 61.26, 65.78) --
	( 61.34, 65.76) --
	( 61.34, 65.76) --
	( 61.34, 65.76) --
	( 61.41, 65.78) --
	( 61.41, 65.78) --
	( 61.41, 65.78) --
	( 61.41, 65.78) --
	( 61.41, 65.78) --
	( 61.41, 65.78) --
	( 61.48, 65.77) --
	( 61.48, 65.77) --
	( 61.48, 65.77) --
	( 61.56, 65.76) --
	( 61.56, 65.76) --
	( 61.56, 65.76) --
	( 61.63, 65.77) --
	( 61.63, 65.77) --
	( 61.63, 65.77) --
	( 61.70, 65.76) --
	( 61.70, 65.76) --
	( 61.70, 65.76) --
	( 61.78, 65.78) --
	( 61.78, 65.78) --
	( 61.78, 65.78) --
	( 61.85, 65.78) --
	( 61.85, 65.78) --
	( 61.85, 65.78) --
	( 61.92, 65.75) --
	( 61.92, 65.75) --
	( 61.92, 65.75) --
	( 62.00, 65.77) --
	( 62.00, 65.77) --
	( 62.00, 65.77) --
	( 62.07, 65.79) --
	( 62.07, 65.79) --
	( 62.07, 65.79) --
	( 62.14, 65.77) --
	( 62.14, 65.77) --
	( 62.14, 65.77) --
	( 62.22, 65.78) --
	( 62.22, 65.78) --
	( 62.22, 65.78) --
	( 62.29, 65.78) --
	( 62.29, 65.78) --
	( 62.29, 65.78) --
	( 62.36, 65.80) --
	( 62.36, 65.80) --
	( 62.36, 65.80) --
	( 62.44, 65.76) --
	( 62.44, 65.76) --
	( 62.44, 65.76) --
	( 62.51, 65.76) --
	( 62.51, 65.76) --
	( 62.51, 65.76) --
	( 62.58, 65.77) --
	( 62.58, 65.77) --
	( 62.58, 65.77) --
	( 62.65, 65.76) --
	( 62.65, 65.76) --
	( 62.65, 65.76) --
	( 62.73, 65.76) --
	( 62.73, 65.76) --
	( 62.73, 65.76) --
	( 62.80, 65.78) --
	( 62.80, 65.78) --
	( 62.80, 65.78) --
	( 62.87, 65.79) --
	( 62.87, 65.79) --
	( 62.87, 65.79) --
	( 62.95, 65.78) --
	( 62.95, 65.78) --
	( 62.95, 65.78) --
	( 63.02, 65.75) --
	( 63.02, 65.75) --
	( 63.02, 65.75) --
	( 63.09, 65.77) --
	( 63.09, 65.77) --
	( 63.09, 65.77) --
	( 63.17, 65.75) --
	( 63.17, 65.75) --
	( 63.17, 65.75) --
	( 63.23, 65.75) --
	( 63.23, 65.75) --
	( 63.23, 65.75) --
	( 63.24, 65.75) --
	( 63.24, 65.75) --
	( 63.24, 65.75) --
	( 63.31, 65.76) --
	( 63.31, 65.76) --
	( 63.31, 65.76) --
	( 63.39, 65.78) --
	( 63.39, 65.78) --
	( 63.39, 65.78) --
	( 63.46, 65.77) --
	( 63.46, 65.77) --
	( 63.46, 65.77) --
	( 63.53, 65.74) --
	( 63.53, 65.74) --
	( 63.53, 65.74) --
	( 63.61, 65.76) --
	( 63.61, 65.76) --
	( 63.61, 65.76) --
	( 63.68, 65.79) --
	( 63.68, 65.79) --
	( 63.68, 65.79) --
	( 63.75, 65.75) --
	( 63.75, 65.75) --
	( 63.75, 65.75) --
	( 63.83, 65.76) --
	( 63.83, 65.76) --
	( 63.83, 65.76) --
	( 63.90, 65.76) --
	( 63.90, 65.76) --
	( 63.90, 65.76) --
	( 63.97, 65.76) --
	( 63.97, 65.76) --
	( 63.97, 65.76) --
	( 64.04, 65.77) --
	( 64.04, 65.77) --
	( 64.04, 65.77) --
	( 64.12, 65.76) --
	( 64.12, 65.76) --
	( 64.12, 65.76) --
	( 64.19, 65.77) --
	( 64.19, 65.77) --
	( 64.19, 65.77) --
	( 64.26, 65.79) --
	( 64.26, 65.79) --
	( 64.26, 65.79) --
	( 64.34, 65.77) --
	( 64.34, 65.77) --
	( 64.34, 65.77) --
	( 64.41, 65.75) --
	( 64.41, 65.75) --
	( 64.41, 65.75) --
	( 64.48, 65.79) --
	( 64.48, 65.79) --
	( 64.48, 65.79) --
	( 64.56, 65.75) --
	( 64.56, 65.75) --
	( 64.56, 65.75) --
	( 64.57, 65.75) --
	( 64.57, 65.75) --
	( 64.57, 65.75) --
	( 64.63, 65.76) --
	( 64.63, 65.76) --
	( 64.63, 65.76) --
	( 64.70, 65.78) --
	( 64.70, 65.78) --
	( 64.70, 65.78) --
	( 64.77, 65.76) --
	( 64.77, 65.76) --
	( 64.77, 65.76) --
	( 64.85, 65.76) --
	( 64.85, 65.76) --
	( 64.85, 65.76) --
	( 64.92, 65.75) --
	( 64.92, 65.75) --
	( 64.92, 65.75) --
	( 64.99, 65.77) --
	( 64.99, 65.77) --
	( 64.99, 65.77) --
	( 65.07, 65.77) --
	( 65.07, 65.77) --
	( 65.07, 65.77) --
	( 65.14, 65.78) --
	( 65.14, 65.78) --
	( 65.14, 65.78) --
	( 65.21, 65.78) --
	( 65.21, 65.78) --
	( 65.21, 65.78) --
	( 65.29, 65.77) --
	( 65.29, 65.77) --
	( 65.29, 65.77) --
	( 65.36, 65.75) --
	( 65.36, 65.75) --
	( 65.36, 65.75) --
	( 65.43, 65.76) --
	( 65.43, 65.76) --
	( 65.43, 65.76) --
	( 65.50, 65.75) --
	( 65.50, 65.75) --
	( 65.50, 65.75) --
	( 65.58, 65.75) --
	( 65.58, 65.75) --
	( 65.58, 65.75) --
	( 65.65, 65.77) --
	( 65.65, 65.77) --
	( 65.65, 65.77) --
	( 65.72, 65.74) --
	( 65.72, 65.74) --
	( 65.72, 65.74) --
	( 65.80, 65.75) --
	( 65.80, 65.75) --
	( 65.80, 65.75) --
	( 65.87, 65.78) --
	( 65.87, 65.78) --
	( 65.87, 65.78) --
	( 65.94, 65.75) --
	( 65.94, 65.75) --
	( 65.94, 65.75) --
	( 66.02, 65.75) --
	( 66.02, 65.75) --
	( 66.02, 65.75) --
	( 66.09, 65.76) --
	( 66.09, 65.76) --
	( 66.09, 65.76) --
	( 66.16, 65.73) --
	( 66.16, 65.73) --
	( 66.16, 65.73) --
	( 66.23, 65.76) --
	( 66.23, 65.76) --
	( 66.23, 65.76) --
	( 66.31, 65.73) --
	( 66.31, 65.73) --
	( 66.31, 65.73) --
	( 66.38, 65.77) --
	( 66.38, 65.77) --
	( 66.38, 65.77) --
	( 66.40, 65.77) --
	( 66.40, 65.77) --
	( 66.40, 65.77) --
	( 66.45, 65.77) --
	( 66.45, 65.77) --
	( 66.45, 65.77) --
	( 66.53, 65.75) --
	( 66.53, 65.75) --
	( 66.53, 65.75) --
	( 66.60, 65.74) --
	( 66.60, 65.74) --
	( 66.60, 65.74) --
	( 66.67, 65.78) --
	( 66.67, 65.78) --
	( 66.67, 65.78) --
	( 66.74, 65.75) --
	( 66.74, 65.75) --
	( 66.74, 65.75) --
	( 66.82, 65.75) --
	( 66.82, 65.75) --
	( 66.82, 65.75) --
	( 66.89, 65.75) --
	( 66.89, 65.75) --
	( 66.89, 65.75) --
	( 66.96, 65.73) --
	( 66.96, 65.73) --
	( 66.96, 65.73) --
	( 67.04, 65.76) --
	( 67.04, 65.76) --
	( 67.04, 65.76) --
	( 67.11, 65.73) --
	( 67.11, 65.73) --
	( 67.11, 65.73) --
	( 67.18, 65.75) --
	( 67.18, 65.75) --
	( 67.18, 65.75) --
	( 67.25, 65.76) --
	( 67.25, 65.76) --
	( 67.25, 65.76) --
	( 67.33, 65.72) --
	( 67.33, 65.72) --
	( 67.33, 65.72) --
	( 67.40, 65.77) --
	( 67.40, 65.77) --
	( 67.40, 65.77) --
	( 67.47, 65.78) --
	( 67.47, 65.78) --
	( 67.47, 65.78) --
	( 67.55, 65.74) --
	( 67.55, 65.74) --
	( 67.55, 65.74) --
	( 67.62, 65.76) --
	( 67.62, 65.76) --
	( 67.62, 65.76) --
	( 67.69, 65.78) --
	( 67.69, 65.78) --
	( 67.69, 65.78) --
	( 67.76, 65.76) --
	( 67.76, 65.76) --
	( 67.76, 65.76) --
	( 67.84, 65.76) --
	( 67.84, 65.76) --
	( 67.84, 65.76) --
	( 67.91, 65.75) --
	( 67.91, 65.75) --
	( 67.91, 65.75) --
	( 67.98, 65.79) --
	( 67.98, 65.79) --
	( 67.98, 65.79) --
	( 68.05, 65.75) --
	( 68.05, 65.75) --
	( 68.05, 65.75) --
	( 68.13, 65.76) --
	( 68.13, 65.76) --
	( 68.13, 65.76) --
	( 68.20, 65.76) --
	( 68.20, 65.76) --
	( 68.20, 65.76) --
	( 68.22, 65.76) --
	( 68.22, 65.76) --
	( 68.22, 65.76) --
	( 68.27, 65.78) --
	( 68.27, 65.78) --
	( 68.27, 65.78) --
	( 68.35, 65.78) --
	( 68.35, 65.78) --
	( 68.35, 65.78) --
	( 68.42, 65.75) --
	( 68.42, 65.75) --
	( 68.42, 65.75) --
	( 68.49, 65.78) --
	( 68.49, 65.78) --
	( 68.49, 65.78) --
	( 68.56, 65.76) --
	( 68.56, 65.76) --
	( 68.56, 65.76) --
	( 68.64, 65.75) --
	( 68.64, 65.75) --
	( 68.64, 65.75) --
	( 68.71, 65.72) --
	( 68.71, 65.72) --
	( 68.71, 65.72) --
	( 68.78, 65.77) --
	( 68.78, 65.77) --
	( 68.78, 65.77) --
	( 68.86, 65.78) --
	( 68.86, 65.78) --
	( 68.86, 65.78) --
	( 68.93, 65.77) --
	( 68.93, 65.77) --
	( 68.93, 65.77) --
	( 69.00, 65.78) --
	( 69.00, 65.78) --
	( 69.00, 65.78) --
	( 69.07, 65.78) --
	( 69.07, 65.78) --
	( 69.07, 65.78) --
	( 69.15, 65.73) --
	( 69.15, 65.73) --
	( 69.15, 65.73) --
	( 69.22, 65.75) --
	( 69.22, 65.75) --
	( 69.22, 65.75) --
	( 69.27, 65.77) --
	( 69.27, 65.77) --
	( 69.27, 65.77) --
	( 69.29, 65.78) --
	( 69.29, 65.78) --
	( 69.29, 65.78) --
	( 69.37, 65.78) --
	( 69.37, 65.78) --
	( 69.37, 65.78) --
	( 69.44, 65.74) --
	( 69.44, 65.74) --
	( 69.44, 65.74) --
	( 69.51, 65.74) --
	( 69.51, 65.74) --
	( 69.51, 65.74) --
	( 69.58, 65.76) --
	( 69.58, 65.76) --
	( 69.58, 65.76) --
	( 69.66, 65.75) --
	( 69.66, 65.75) --
	( 69.66, 65.75) --
	( 69.73, 65.76) --
	( 69.73, 65.76) --
	( 69.73, 65.76) --
	( 69.80, 65.77) --
	( 69.80, 65.77) --
	( 69.80, 65.77) --
	( 69.87, 65.79) --
	( 69.87, 65.79) --
	( 69.87, 65.79) --
	( 69.95, 65.75) --
	( 69.95, 65.75) --
	( 69.95, 65.75) --
	( 70.02, 65.76) --
	( 70.02, 65.76) --
	( 70.02, 65.76) --
	( 70.09, 65.76) --
	( 70.09, 65.76) --
	( 70.09, 65.76) --
	( 70.17, 65.77) --
	( 70.17, 65.77) --
	( 70.17, 65.77) --
	( 70.24, 65.76) --
	( 70.24, 65.76) --
	( 70.24, 65.76) --
	( 70.31, 65.75) --
	( 70.31, 65.75) --
	( 70.31, 65.75) --
	( 70.38, 65.78) --
	( 70.38, 65.78) --
	( 70.38, 65.78) --
	( 70.42, 65.77) --
	( 70.42, 65.77) --
	( 70.42, 65.77) --
	( 70.46, 65.76) --
	( 70.46, 65.76) --
	( 70.46, 65.76) --
	( 70.53, 65.73) --
	( 70.53, 65.73) --
	( 70.53, 65.73) --
	( 70.60, 65.76) --
	( 70.60, 65.76) --
	( 70.60, 65.76) --
	( 70.67, 65.77) --
	( 70.67, 65.77) --
	( 70.67, 65.77) --
	( 70.75, 65.77) --
	( 70.75, 65.77) --
	( 70.75, 65.77) --
	( 70.82, 65.75) --
	( 70.82, 65.75) --
	( 70.82, 65.75) --
	( 70.89, 65.75) --
	( 70.89, 65.75) --
	( 70.89, 65.75) --
	( 70.96, 65.76) --
	( 70.96, 65.76) --
	( 70.96, 65.76) --
	( 71.04, 65.77) --
	( 71.04, 65.77) --
	( 71.04, 65.77) --
	( 71.11, 65.75) --
	( 71.11, 65.75) --
	( 71.11, 65.75) --
	( 71.18, 65.77) --
	( 71.18, 65.77) --
	( 71.18, 65.77) --
	( 71.19, 65.77) --
	( 71.19, 65.77) --
	( 71.19, 65.77) --
	( 71.26, 65.76) --
	( 71.26, 65.76) --
	( 71.26, 65.76) --
	( 71.33, 65.74) --
	( 71.33, 65.74) --
	( 71.33, 65.74) --
	( 71.40, 65.77) --
	( 71.40, 65.77) --
	( 71.40, 65.77) --
	( 71.47, 65.77) --
	( 71.47, 65.77) --
	( 71.47, 65.77) --
	( 71.54, 65.74) --
	( 71.54, 65.74) --
	( 71.54, 65.74) --
	( 71.62, 65.75) --
	( 71.62, 65.75) --
	( 71.62, 65.75) --
	( 71.69, 65.75) --
	( 71.69, 65.75) --
	( 71.69, 65.75) --
	( 71.76, 65.74) --
	( 71.76, 65.74) --
	( 71.76, 65.74) --
	( 71.83, 65.77) --
	( 71.83, 65.77) --
	( 71.83, 65.77) --
	( 71.91, 65.74) --
	( 71.91, 65.74) --
	( 71.91, 65.74) --
	( 71.95, 65.77) --
	( 71.95, 65.77) --
	( 71.95, 65.77) --
	( 71.98, 65.78) --
	( 71.98, 65.78) --
	( 71.98, 65.78) --
	( 72.05, 65.77) --
	( 72.05, 65.77) --
	( 72.05, 65.77) --
	( 72.13, 65.75) --
	( 72.13, 65.75) --
	( 72.13, 65.75) --
	( 72.20, 65.76) --
	( 72.20, 65.76) --
	( 72.20, 65.76) --
	( 72.27, 65.79) --
	( 72.27, 65.79) --
	( 72.27, 65.79) --
	( 72.34, 65.75) --
	( 72.34, 65.75) --
	( 72.34, 65.75) --
	( 72.42, 65.76) --
	( 72.42, 65.76) --
	( 72.42, 65.76) --
	( 72.49, 65.75) --
	( 72.49, 65.75) --
	( 72.49, 65.75) --
	( 72.56, 65.76) --
	( 72.56, 65.76) --
	( 72.56, 65.76) --
	( 72.63, 65.75) --
	( 72.63, 65.75) --
	( 72.63, 65.75) --
	( 72.71, 65.75) --
	( 72.71, 65.75) --
	( 72.71, 65.75) --
	( 72.78, 65.77) --
	( 72.78, 65.77) --
	( 72.78, 65.77) --
	( 72.85, 65.78) --
	( 72.85, 65.78) --
	( 72.85, 65.78) --
	( 72.92, 65.76) --
	( 72.92, 65.76) --
	( 72.92, 65.76) --
	( 73.00, 65.77) --
	( 73.00, 65.77) --
	( 73.00, 65.77) --
	( 73.07, 65.79) --
	( 73.07, 65.79) --
	( 73.07, 65.79) --
	( 73.14, 65.76) --
	( 73.14, 65.76) --
	( 73.14, 65.76) --
	( 73.20, 65.76) --
	( 73.20, 65.76) --
	( 73.20, 65.76) --
	( 73.21, 65.76) --
	( 73.21, 65.76) --
	( 73.21, 65.76) --
	( 73.28, 65.76) --
	( 73.28, 65.76) --
	( 73.28, 65.76) --
	( 73.36, 65.77) --
	( 73.36, 65.77) --
	( 73.36, 65.77) --
	( 73.43, 65.78) --
	( 73.43, 65.78) --
	( 73.43, 65.78) --
	( 73.50, 65.77) --
	( 73.50, 65.77) --
	( 73.50, 65.77) --
	( 73.58, 65.80) --
	( 73.58, 65.80) --
	( 73.58, 65.80) --
	( 73.65, 65.76) --
	( 73.65, 65.76) --
	( 73.65, 65.76) --
	( 73.72, 65.74) --
	( 73.72, 65.74) --
	( 73.72, 65.74) --
	( 73.79, 65.76) --
	( 73.79, 65.76) --
	( 73.79, 65.76) --
	( 73.86, 65.76) --
	( 73.86, 65.76) --
	( 73.86, 65.76) --
	( 73.94, 65.74) --
	( 73.94, 65.74) --
	( 73.94, 65.74) --
	( 73.97, 65.75) --
	( 73.97, 65.75) --
	( 73.97, 65.75) --
	( 74.01, 65.76) --
	( 74.01, 65.76) --
	( 74.01, 65.76) --
	( 74.08, 65.74) --
	( 74.08, 65.74) --
	( 74.08, 65.74) --
	( 74.16, 65.76) --
	( 74.16, 65.76) --
	( 74.16, 65.76) --
	( 74.23, 65.78) --
	( 74.23, 65.78) --
	( 74.23, 65.78) --
	( 74.30, 65.74) --
	( 74.30, 65.74) --
	( 74.30, 65.74) --
	( 74.37, 65.78) --
	( 74.37, 65.78) --
	( 74.37, 65.78) --
	( 74.44, 65.80) --
	( 74.44, 65.80) --
	( 74.44, 65.80) --
	( 74.52, 65.73) --
	( 74.52, 65.73) --
	( 74.52, 65.73) --
	( 74.59, 65.74) --
	( 74.59, 65.74) --
	( 74.59, 65.74) --
	( 74.66, 65.78) --
	( 74.66, 65.78) --
	( 74.66, 65.78) --
	( 74.73, 65.75) --
	( 74.73, 65.75) --
	( 74.73, 65.75) --
	( 74.81, 65.76) --
	( 74.81, 65.76) --
	( 74.81, 65.76) --
	( 74.88, 65.74) --
	( 74.88, 65.74) --
	( 74.88, 65.74) --
	( 74.92, 65.76) --
	( 74.92, 65.76) --
	( 74.92, 65.76) --
	( 74.95, 65.78) --
	( 74.95, 65.78) --
	( 74.95, 65.78) --
	( 75.02, 65.77) --
	( 75.02, 65.77) --
	( 75.02, 65.77) --
	( 75.10, 65.75) --
	( 75.10, 65.75) --
	( 75.10, 65.75) --
	( 75.17, 65.73) --
	( 75.17, 65.73) --
	( 75.17, 65.73) --
	( 75.24, 65.76) --
	( 75.24, 65.76) --
	( 75.24, 65.76) --
	( 75.31, 65.75) --
	( 75.31, 65.75) --
	( 75.31, 65.75) --
	( 75.39, 65.74) --
	( 75.39, 65.74) --
	( 75.39, 65.74) --
	( 75.46, 65.76) --
	( 75.46, 65.76) --
	( 75.46, 65.76) --
	( 75.53, 65.73) --
	( 75.53, 65.73) --
	( 75.53, 65.73) --
	( 75.60, 65.76) --
	( 75.60, 65.76) --
	( 75.60, 65.76) --
	( 75.67, 65.72) --
	( 75.67, 65.72) --
	( 75.67, 65.72) --
	( 75.75, 65.73) --
	( 75.75, 65.73) --
	( 75.75, 65.73) --
	( 75.79, 65.75) --
	( 75.79, 65.75) --
	( 75.79, 65.75) --
	( 75.82, 65.77) --
	( 75.82, 65.77) --
	( 75.82, 65.77) --
	( 75.89, 65.76) --
	( 75.89, 65.76) --
	( 75.89, 65.76) --
	( 75.96, 65.74) --
	( 75.96, 65.74) --
	( 75.96, 65.74) --
	( 76.04, 65.76) --
	( 76.04, 65.76) --
	( 76.04, 65.76) --
	( 76.11, 65.76) --
	( 76.11, 65.76) --
	( 76.11, 65.76) --
	( 76.18, 65.75) --
	( 76.18, 65.75) --
	( 76.18, 65.75) --
	( 76.25, 65.76) --
	( 76.25, 65.76) --
	( 76.25, 65.76) --
	( 76.33, 65.77) --
	( 76.33, 65.77) --
	( 76.33, 65.77) --
	( 76.40, 65.76) --
	( 76.40, 65.76) --
	( 76.40, 65.76) --
	( 76.47, 65.75) --
	( 76.47, 65.75) --
	( 76.47, 65.75) --
	( 76.54, 65.76) --
	( 76.54, 65.76) --
	( 76.54, 65.76) --
	( 76.61, 65.77) --
	( 76.61, 65.77) --
	( 76.61, 65.77) --
	( 76.69, 65.73) --
	( 76.69, 65.73) --
	( 76.69, 65.73) --
	( 76.75, 65.74) --
	( 76.75, 65.74) --
	( 76.75, 65.74) --
	( 76.76, 65.74) --
	( 76.76, 65.74) --
	( 76.76, 65.74) --
	( 76.83, 65.75) --
	( 76.83, 65.75) --
	( 76.83, 65.75) --
	( 76.90, 65.74) --
	( 76.90, 65.74) --
	( 76.90, 65.74) --
	( 76.98, 65.75) --
	( 76.98, 65.75) --
	( 76.98, 65.75) --
	( 77.05, 65.76) --
	( 77.05, 65.76) --
	( 77.05, 65.76) --
	( 77.12, 65.76) --
	( 77.12, 65.76) --
	( 77.12, 65.76) --
	( 77.19, 65.76) --
	( 77.19, 65.76) --
	( 77.19, 65.76) --
	( 77.27, 65.75) --
	( 77.27, 65.75) --
	( 77.27, 65.75) --
	( 77.34, 65.74) --
	( 77.34, 65.74) --
	( 77.34, 65.74) --
	( 77.41, 65.76) --
	( 77.41, 65.76) --
	( 77.41, 65.76) --
	( 77.48, 65.74) --
	( 77.48, 65.74) --
	( 77.48, 65.74) --
	( 77.55, 65.75) --
	( 77.55, 65.75) --
	( 77.55, 65.75) --
	( 77.63, 65.76) --
	( 77.63, 65.76) --
	( 77.63, 65.76) --
	( 77.70, 65.75) --
	( 77.70, 65.75) --
	( 77.70, 65.75) --
	( 77.70, 65.75) --
	( 77.70, 65.75) --
	( 77.70, 65.75) --
	( 77.77, 65.74) --
	( 77.77, 65.74) --
	( 77.77, 65.74) --
	( 77.84, 65.77) --
	( 77.84, 65.77) --
	( 77.84, 65.77) --
	( 77.91, 65.75) --
	( 77.91, 65.75) --
	( 77.91, 65.75) --
	( 77.99, 65.77) --
	( 77.99, 65.77) --
	( 77.99, 65.77) --
	( 78.06, 65.74) --
	( 78.06, 65.74) --
	( 78.06, 65.74) --
	( 78.13, 65.75) --
	( 78.13, 65.75) --
	( 78.13, 65.75) --
	( 78.20, 65.75) --
	( 78.20, 65.75) --
	( 78.20, 65.75) --
	( 78.28, 65.74) --
	( 78.28, 65.74) --
	( 78.28, 65.74) --
	( 78.35, 65.76) --
	( 78.35, 65.76) --
	( 78.35, 65.76) --
	( 78.42, 65.75) --
	( 78.42, 65.75) --
	( 78.42, 65.75) --
	( 78.49, 65.74) --
	( 78.49, 65.74) --
	( 78.49, 65.74) --
	( 78.56, 65.75) --
	( 78.56, 65.75) --
	( 78.56, 65.75) --
	( 78.64, 65.75) --
	( 78.64, 65.75) --
	( 78.64, 65.75) --
	( 78.71, 65.74) --
	( 78.71, 65.74) --
	( 78.71, 65.74) --
	( 78.76, 65.75) --
	( 78.76, 65.75) --
	( 78.76, 65.75) --
	( 78.78, 65.76) --
	( 78.78, 65.76) --
	( 78.78, 65.76) --
	( 78.85, 65.74) --
	( 78.85, 65.74) --
	( 78.85, 65.74) --
	( 78.93, 65.76) --
	( 78.93, 65.76) --
	( 78.93, 65.76) --
	( 79.00, 65.76) --
	( 79.00, 65.76) --
	( 79.00, 65.76) --
	( 79.07, 65.74) --
	( 79.07, 65.74) --
	( 79.07, 65.74) --
	( 79.14, 65.75) --
	( 79.14, 65.75) --
	( 79.14, 65.75) --
	( 79.21, 65.78) --
	( 79.21, 65.78) --
	( 79.21, 65.78) --
	( 79.29, 65.75) --
	( 79.29, 65.75) --
	( 79.29, 65.75) --
	( 79.36, 65.74) --
	( 79.36, 65.74) --
	( 79.36, 65.74) --
	( 79.43, 65.78) --
	( 79.43, 65.78) --
	( 79.43, 65.78) --
	( 79.50, 65.76) --
	( 79.50, 65.76) --
	( 79.50, 65.76) --
	( 79.57, 65.74) --
	( 79.57, 65.74) --
	( 79.57, 65.74) --
	( 79.62, 65.74) --
	( 79.62, 65.74) --
	( 79.62, 65.74) --
	( 79.65, 65.74) --
	( 79.65, 65.74) --
	( 79.65, 65.74) --
	( 79.72, 65.75) --
	( 79.72, 65.75) --
	( 79.72, 65.75) --
	( 79.79, 65.77) --
	( 79.79, 65.77) --
	( 79.79, 65.77) --
	( 79.86, 65.74) --
	( 79.86, 65.74) --
	( 79.86, 65.74) --
	( 79.93, 65.76) --
	( 79.93, 65.76) --
	( 79.93, 65.76) --
	( 80.01, 65.78) --
	( 80.01, 65.78) --
	( 80.01, 65.78) --
	( 80.08, 65.76) --
	( 80.08, 65.76) --
	( 80.08, 65.76) --
	( 80.15, 65.76) --
	( 80.15, 65.76) --
	( 80.15, 65.76) --
	( 80.22, 65.76) --
	( 80.22, 65.76) --
	( 80.22, 65.76) --
	( 80.29, 65.76) --
	( 80.29, 65.76) --
	( 80.29, 65.76) --
	( 80.37, 65.75) --
	( 80.37, 65.75) --
	( 80.37, 65.75) --
	( 80.44, 65.74) --
	( 80.44, 65.74) --
	( 80.44, 65.74) --
	( 80.48, 65.76) --
	( 80.48, 65.76) --
	( 80.48, 65.76) --
	( 80.51, 65.77) --
	( 80.51, 65.77) --
	( 80.51, 65.77) --
	( 80.58, 65.78) --
	( 80.58, 65.78) --
	( 80.58, 65.78) --
	( 80.65, 65.73) --
	( 80.65, 65.73) --
	( 80.65, 65.73) --
	( 80.73, 65.75) --
	( 80.73, 65.75) --
	( 80.73, 65.75) --
	( 80.80, 65.76) --
	( 80.80, 65.76) --
	( 80.80, 65.76) --
	( 80.87, 65.73) --
	( 80.87, 65.73) --
	( 80.87, 65.73) --
	( 80.94, 65.73) --
	( 80.94, 65.73) --
	( 80.94, 65.73) --
	( 81.01, 65.74) --
	( 81.01, 65.74) --
	( 81.01, 65.74) --
	( 81.09, 65.76) --
	( 81.09, 65.76) --
	( 81.09, 65.76) --
	( 81.16, 65.75) --
	( 81.16, 65.75) --
	( 81.16, 65.75) --
	( 81.23, 65.74) --
	( 81.23, 65.74) --
	( 81.23, 65.74) --
	( 81.30, 65.72) --
	( 81.30, 65.72) --
	( 81.30, 65.72) --
	( 81.37, 65.75) --
	( 81.37, 65.75) --
	( 81.37, 65.75) --
	( 81.44, 65.75) --
	( 81.44, 65.75) --
	( 81.44, 65.75) --
	( 81.45, 65.75) --
	( 81.45, 65.75) --
	( 81.45, 65.75) --
	( 81.52, 65.75) --
	( 81.52, 65.75) --
	( 81.52, 65.75) --
	( 81.59, 65.77) --
	( 81.59, 65.77) --
	( 81.59, 65.77) --
	( 81.66, 65.76) --
	( 81.66, 65.76) --
	( 81.66, 65.76) --
	( 81.73, 65.76) --
	( 81.73, 65.76) --
	( 81.73, 65.76) --
	( 81.81, 65.75) --
	( 81.81, 65.75) --
	( 81.81, 65.75) --
	( 81.88, 65.74) --
	( 81.88, 65.74) --
	( 81.88, 65.74) --
	( 81.95, 65.75) --
	( 81.95, 65.75) --
	( 81.95, 65.75) --
	( 82.02, 65.77) --
	( 82.02, 65.77) --
	( 82.02, 65.77) --
	( 82.09, 65.76) --
	( 82.09, 65.76) --
	( 82.09, 65.76) --
	( 82.16, 65.78) --
	( 82.16, 65.78) --
	( 82.16, 65.78) --
	( 82.24, 65.75) --
	( 82.24, 65.75) --
	( 82.24, 65.75) --
	( 82.31, 65.72) --
	( 82.31, 65.72) --
	( 82.31, 65.72) --
	( 82.38, 65.77) --
	( 82.38, 65.77) --
	( 82.38, 65.77) --
	( 82.45, 65.73) --
	( 82.45, 65.73) --
	( 82.45, 65.73) --
	( 82.52, 65.76) --
	( 82.52, 65.76) --
	( 82.52, 65.76) --
	( 82.59, 65.73) --
	( 82.59, 65.73) --
	( 82.59, 65.73) --
	( 82.60, 65.73) --
	( 82.60, 65.73) --
	( 82.60, 65.73) --
	( 82.67, 65.76) --
	( 82.67, 65.76) --
	( 82.67, 65.76) --
	( 82.74, 65.76) --
	( 82.74, 65.76) --
	( 82.74, 65.76) --
	( 82.81, 65.73) --
	( 82.81, 65.73) --
	( 82.81, 65.73) --
	( 82.88, 65.77) --
	( 82.88, 65.77) --
	( 82.88, 65.77) --
	( 82.96, 65.76) --
	( 82.96, 65.76) --
	( 82.96, 65.76) --
	( 83.03, 65.74) --
	( 83.03, 65.74) --
	( 83.03, 65.74) --
	( 83.10, 65.74) --
	( 83.10, 65.74) --
	( 83.10, 65.74) --
	( 83.17, 65.74) --
	( 83.17, 65.74) --
	( 83.17, 65.74) --
	( 83.24, 65.75) --
	( 83.24, 65.75) --
	( 83.24, 65.75) --
	( 83.31, 65.77) --
	( 83.31, 65.77) --
	( 83.31, 65.77) --
	( 83.39, 65.74) --
	( 83.39, 65.74) --
	( 83.39, 65.74) --
	( 83.46, 65.76) --
	( 83.46, 65.76) --
	( 83.46, 65.76) --
	( 83.53, 65.76) --
	( 83.53, 65.76) --
	( 83.53, 65.76) --
	( 83.60, 65.74) --
	( 83.60, 65.74) --
	( 83.60, 65.74) --
	( 83.67, 65.76) --
	( 83.67, 65.76) --
	( 83.67, 65.76) --
	( 83.75, 65.77) --
	( 83.75, 65.77) --
	( 83.75, 65.77) --
	( 83.82, 65.75) --
	( 83.82, 65.75) --
	( 83.82, 65.75) --
	( 83.89, 65.74) --
	( 83.89, 65.74) --
	( 83.89, 65.74) --
	( 83.96, 65.76) --
	( 83.96, 65.76) --
	( 83.96, 65.76) --
	( 84.03, 65.78) --
	( 84.03, 65.78) --
	( 84.03, 65.78) --
	( 84.10, 65.74) --
	( 84.10, 65.74) --
	( 84.10, 65.74) --
	( 84.12, 65.74) --
	( 84.12, 65.74) --
	( 84.12, 65.74) --
	( 84.18, 65.75) --
	( 84.18, 65.75) --
	( 84.18, 65.75) --
	( 84.25, 65.74) --
	( 84.25, 65.74) --
	( 84.25, 65.74) --
	( 84.32, 65.75) --
	( 84.32, 65.75) --
	( 84.32, 65.75) --
	( 84.39, 65.74) --
	( 84.39, 65.74) --
	( 84.39, 65.74) --
	( 84.46, 65.75) --
	( 84.46, 65.75) --
	( 84.46, 65.75) --
	( 84.54, 65.76) --
	( 84.54, 65.76) --
	( 84.54, 65.76) --
	( 84.61, 65.74) --
	( 84.61, 65.74) --
	( 84.61, 65.74) --
	( 84.68, 65.76) --
	( 84.68, 65.76) --
	( 84.68, 65.76) --
	( 84.75, 65.77) --
	( 84.75, 65.77) --
	( 84.75, 65.77) --
	( 84.82, 65.76) --
	( 84.82, 65.76) --
	( 84.82, 65.76) --
	( 84.89, 65.74) --
	( 84.89, 65.74) --
	( 84.89, 65.74) --
	( 84.96, 65.73) --
	( 84.96, 65.73) --
	( 84.96, 65.73) --
	( 85.04, 65.74) --
	( 85.04, 65.74) --
	( 85.04, 65.74) --
	( 85.11, 65.76) --
	( 85.11, 65.76) --
	( 85.11, 65.76) --
	( 85.18, 65.75) --
	( 85.18, 65.75) --
	( 85.18, 65.75) --
	( 85.25, 65.76) --
	( 85.25, 65.76) --
	( 85.25, 65.76) --
	( 85.32, 65.76) --
	( 85.32, 65.76) --
	( 85.32, 65.76) --
	( 85.40, 65.73) --
	( 85.40, 65.73) --
	( 85.40, 65.73) --
	( 85.47, 65.77) --
	( 85.47, 65.77) --
	( 85.47, 65.77) --
	( 85.47, 65.77) --
	( 85.47, 65.77) --
	( 85.47, 65.77) --
	( 85.54, 65.76) --
	( 85.54, 65.76) --
	( 85.54, 65.76) --
	( 85.61, 65.75) --
	( 85.61, 65.75) --
	( 85.61, 65.75) --
	( 85.68, 65.76) --
	( 85.68, 65.76) --
	( 85.68, 65.76) --
	( 85.75, 65.74) --
	( 85.75, 65.74) --
	( 85.75, 65.74) --
	( 85.83, 65.77) --
	( 85.83, 65.77) --
	( 85.83, 65.77) --
	( 85.90, 65.77) --
	( 85.90, 65.77) --
	( 85.90, 65.77) --
	( 85.97, 65.74) --
	( 85.97, 65.74) --
	( 85.97, 65.74) --
	( 86.04, 65.76) --
	( 86.04, 65.76) --
	( 86.04, 65.76) --
	( 86.11, 65.78) --
	( 86.11, 65.78) --
	( 86.11, 65.78) --
	( 86.18, 65.75) --
	( 86.18, 65.75) --
	( 86.18, 65.75) --
	( 86.25, 65.76) --
	( 86.25, 65.76) --
	( 86.25, 65.76) --
	( 86.33, 65.74) --
	( 86.33, 65.74) --
	( 86.33, 65.74) --
	( 86.40, 65.73) --
	( 86.40, 65.73) --
	( 86.40, 65.73) --
	( 86.47, 65.76) --
	( 86.47, 65.76) --
	( 86.47, 65.76) --
	( 86.54, 65.73) --
	( 86.54, 65.73) --
	( 86.54, 65.73) --
	( 86.61, 65.76) --
	( 86.61, 65.76) --
	( 86.61, 65.76) --
	( 86.68, 65.76) --
	( 86.68, 65.76) --
	( 86.68, 65.76) --
	( 86.76, 65.74) --
	( 86.76, 65.74) --
	( 86.76, 65.74) --
	( 86.83, 65.76) --
	( 86.83, 65.76) --
	( 86.83, 65.76) --
	( 86.90, 65.77) --
	( 86.90, 65.77) --
	( 86.90, 65.77) --
	( 86.90, 65.77) --
	( 86.90, 65.77) --
	( 86.90, 65.77) --
	( 86.97, 65.73) --
	( 86.97, 65.73) --
	( 86.97, 65.73) --
	( 87.04, 65.76) --
	( 87.04, 65.76) --
	( 87.04, 65.76) --
	( 87.11, 65.75) --
	( 87.11, 65.75) --
	( 87.11, 65.75) --
	( 87.19, 65.76) --
	( 87.19, 65.76) --
	( 87.19, 65.76) --
	( 87.26, 65.76) --
	( 87.26, 65.76) --
	( 87.26, 65.76) --
	( 87.33, 65.74) --
	( 87.33, 65.74) --
	( 87.33, 65.74) --
	( 87.40, 65.78) --
	( 87.40, 65.78) --
	( 87.40, 65.78) --
	( 87.47, 65.76) --
	( 87.47, 65.76) --
	( 87.47, 65.76) --
	( 87.54, 65.73) --
	( 87.54, 65.73) --
	( 87.54, 65.73) --
	( 87.62, 65.74) --
	( 87.62, 65.74) --
	( 87.62, 65.74) --
	( 87.69, 65.76) --
	( 87.69, 65.76) --
	( 87.69, 65.76) --
	( 87.76, 65.72) --
	( 87.76, 65.72) --
	( 87.76, 65.72) --
	( 87.83, 65.75) --
	( 87.83, 65.75) --
	( 87.83, 65.75) --
	( 87.90, 65.75) --
	( 87.90, 65.75) --
	( 87.90, 65.75) --
	( 87.97, 65.76) --
	( 87.97, 65.76) --
	( 87.97, 65.76) --
	( 88.04, 65.76) --
	( 88.04, 65.76) --
	( 88.04, 65.76) --
	( 88.12, 65.75) --
	( 88.12, 65.75) --
	( 88.12, 65.75) --
	( 88.19, 65.74) --
	( 88.19, 65.74) --
	( 88.19, 65.74) --
	( 88.26, 65.76) --
	( 88.26, 65.76) --
	( 88.26, 65.76) --
	( 88.33, 65.76) --
	( 88.33, 65.76) --
	( 88.33, 65.76) --
	( 88.40, 65.77) --
	( 88.40, 65.77) --
	( 88.40, 65.77) --
	( 88.47, 65.78) --
	( 88.47, 65.78) --
	( 88.47, 65.78) --
	( 88.54, 65.74) --
	( 88.54, 65.74) --
	( 88.54, 65.74) --
	( 88.62, 65.74) --
	( 88.62, 65.74) --
	( 88.62, 65.74) --
	( 88.69, 65.76) --
	( 88.69, 65.76) --
	( 88.69, 65.76) --
	( 88.76, 65.75) --
	( 88.76, 65.75) --
	( 88.76, 65.75) --
	( 88.83, 65.75) --
	( 88.83, 65.75) --
	( 88.83, 65.75) --
	( 88.90, 65.72) --
	( 88.90, 65.72) --
	( 88.90, 65.72) --
	( 88.92, 65.72) --
	( 88.92, 65.72) --
	( 88.92, 65.72) --
	( 88.97, 65.76) --
	( 88.97, 65.76) --
	( 88.97, 65.76) --
	( 89.05, 65.77) --
	( 89.05, 65.77) --
	( 89.05, 65.77) --
	( 89.12, 65.73) --
	( 89.12, 65.73) --
	( 89.12, 65.73) --
	( 89.19, 65.75) --
	( 89.19, 65.75) --
	( 89.19, 65.75) --
	( 89.26, 65.77) --
	( 89.26, 65.77) --
	( 89.26, 65.77) --
	( 89.33, 65.76) --
	( 89.33, 65.76) --
	( 89.33, 65.76) --
	( 89.40, 65.76) --
	( 89.40, 65.76) --
	( 89.40, 65.76) --
	( 89.47, 65.76) --
	( 89.47, 65.76) --
	( 89.47, 65.76) --
	( 89.55, 65.76) --
	( 89.55, 65.76) --
	( 89.55, 65.76) --
	( 89.62, 65.75) --
	( 89.62, 65.75) --
	( 89.62, 65.75) --
	( 89.69, 65.74) --
	( 89.69, 65.74) --
	( 89.69, 65.74) --
	( 89.76, 65.75) --
	( 89.76, 65.75) --
	( 89.76, 65.75) --
	( 89.83, 65.76) --
	( 89.83, 65.76) --
	( 89.83, 65.76) --
	( 89.90, 65.76) --
	( 89.90, 65.76) --
	( 89.90, 65.76) --
	( 89.97, 65.76) --
	( 89.97, 65.76) --
	( 89.97, 65.76) --
	( 90.04, 65.77) --
	( 90.04, 65.77) --
	( 90.04, 65.77) --
	( 90.12, 65.75) --
	( 90.12, 65.75) --
	( 90.12, 65.75) --
	( 90.19, 65.74) --
	( 90.19, 65.74) --
	( 90.19, 65.74) --
	( 90.26, 65.74) --
	( 90.26, 65.74) --
	( 90.26, 65.74) --
	( 90.33, 65.75) --
	( 90.33, 65.75) --
	( 90.33, 65.75) --
	( 90.40, 65.75) --
	( 90.40, 65.75) --
	( 90.40, 65.75) --
	( 90.47, 65.73) --
	( 90.47, 65.73) --
	( 90.47, 65.73) --
	( 90.54, 65.76) --
	( 90.54, 65.76) --
	( 90.54, 65.76) --
	( 90.62, 65.74) --
	( 90.62, 65.74) --
	( 90.62, 65.74) --
	( 90.64, 65.74) --
	( 90.64, 65.74) --
	( 90.64, 65.74) --
	( 90.69, 65.74) --
	( 90.69, 65.74) --
	( 90.69, 65.74) --
	( 90.76, 65.74) --
	( 90.76, 65.74) --
	( 90.76, 65.74) --
	( 90.83, 65.76) --
	( 90.83, 65.76) --
	( 90.83, 65.76) --
	( 90.90, 65.77) --
	( 90.90, 65.77) --
	( 90.90, 65.77) --
	( 90.97, 65.75) --
	( 90.97, 65.75) --
	( 90.97, 65.75) --
	( 91.04, 65.76) --
	( 91.04, 65.76) --
	( 91.04, 65.76) --
	( 91.12, 65.78) --
	( 91.12, 65.78) --
	( 91.12, 65.78) --
	( 91.19, 65.74) --
	( 91.19, 65.74) --
	( 91.19, 65.74) --
	( 91.26, 65.74) --
	( 91.26, 65.74) --
	( 91.26, 65.74) --
	( 91.33, 65.76) --
	( 91.33, 65.76) --
	( 91.33, 65.76) --
	( 91.40, 65.75) --
	( 91.40, 65.75) --
	( 91.40, 65.75) --
	( 91.47, 65.77) --
	( 91.47, 65.77) --
	( 91.47, 65.77) --
	( 91.54, 65.75) --
	( 91.54, 65.75) --
	( 91.54, 65.75) --
	( 91.61, 65.79) --
	( 91.61, 65.79) --
	( 91.61, 65.79) --
	( 91.69, 65.74) --
	( 91.69, 65.74) --
	( 91.69, 65.74) --
	( 91.76, 65.75) --
	( 91.76, 65.75) --
	( 91.76, 65.75) --
	( 91.83, 65.75) --
	( 91.83, 65.75) --
	( 91.83, 65.75) --
	( 91.90, 65.76) --
	( 91.90, 65.76) --
	( 91.90, 65.76) --
	( 91.97, 65.76) --
	( 91.97, 65.76) --
	( 91.97, 65.76) --
	( 92.04, 65.75) --
	( 92.04, 65.75) --
	( 92.04, 65.75) --
	( 92.11, 65.74) --
	( 92.11, 65.74) --
	( 92.11, 65.74) --
	( 92.18, 65.78) --
	( 92.18, 65.78) --
	( 92.18, 65.78) --
	( 92.26, 65.77) --
	( 92.26, 65.77) --
	( 92.26, 65.77) --
	( 92.33, 65.75) --
	( 92.33, 65.75) --
	( 92.33, 65.75) --
	( 92.40, 65.76) --
	( 92.40, 65.76) --
	( 92.40, 65.76) --
	( 92.47, 65.75) --
	( 92.47, 65.75) --
	( 92.47, 65.75) --
	( 92.54, 65.77) --
	( 92.54, 65.77) --
	( 92.54, 65.77) --
	( 92.61, 65.74) --
	( 92.61, 65.74) --
	( 92.61, 65.74) --
	( 92.65, 65.74) --
	( 92.65, 65.74) --
	( 92.65, 65.74) --
	( 92.68, 65.74) --
	( 92.68, 65.74) --
	( 92.68, 65.74) --
	( 92.75, 65.77) --
	( 92.75, 65.77) --
	( 92.75, 65.77) --
	( 92.83, 65.73) --
	( 92.83, 65.73) --
	( 92.83, 65.73) --
	( 92.90, 65.74) --
	( 92.90, 65.74) --
	( 92.90, 65.74) --
	( 92.97, 65.76) --
	( 92.97, 65.76) --
	( 92.97, 65.76) --
	( 93.04, 65.73) --
	( 93.04, 65.73) --
	( 93.04, 65.73) --
	( 93.11, 65.75) --
	( 93.11, 65.75) --
	( 93.11, 65.75) --
	( 93.18, 65.76) --
	( 93.18, 65.76) --
	( 93.18, 65.76) --
	( 93.25, 65.73) --
	( 93.25, 65.73) --
	( 93.25, 65.73) --
	( 93.32, 65.78) --
	( 93.32, 65.78) --
	( 93.32, 65.78) --
	( 93.39, 65.74) --
	( 93.39, 65.74) --
	( 93.39, 65.74) --
	( 93.46, 65.75) --
	( 93.46, 65.75) --
	( 93.46, 65.75) --
	( 93.54, 65.74) --
	( 93.54, 65.74) --
	( 93.54, 65.74) --
	( 93.61, 65.72) --
	( 93.61, 65.72) --
	( 93.61, 65.72) --
	( 93.68, 65.75) --
	( 93.68, 65.75) --
	( 93.68, 65.75) --
	( 93.75, 65.77) --
	( 93.75, 65.77) --
	( 93.75, 65.77) --
	( 93.82, 65.75) --
	( 93.82, 65.75) --
	( 93.82, 65.75) --
	( 93.89, 65.77) --
	( 93.89, 65.77) --
	( 93.89, 65.77) --
	( 93.96, 65.76) --
	( 93.96, 65.76) --
	( 93.96, 65.76) --
	( 94.03, 65.75) --
	( 94.03, 65.75) --
	( 94.03, 65.75) --
	( 94.10, 65.74) --
	( 94.10, 65.74) --
	( 94.10, 65.74) --
	( 94.18, 65.73) --
	( 94.18, 65.73) --
	( 94.18, 65.73) --
	( 94.25, 65.75) --
	( 94.25, 65.75) --
	( 94.25, 65.75) --
	( 94.32, 65.74) --
	( 94.32, 65.74) --
	( 94.32, 65.74) --
	( 94.39, 65.75) --
	( 94.39, 65.75) --
	( 94.39, 65.75) --
	( 94.46, 65.76) --
	( 94.46, 65.76) --
	( 94.46, 65.76) --
	( 94.53, 65.76) --
	( 94.53, 65.76) --
	( 94.53, 65.76) --
	( 94.60, 65.75) --
	( 94.60, 65.75) --
	( 94.60, 65.75) --
	( 94.67, 65.75) --
	( 94.67, 65.75) --
	( 94.67, 65.75) --
	( 94.75, 65.73) --
	( 94.75, 65.73) --
	( 94.75, 65.73) --
	( 94.82, 65.76) --
	( 94.82, 65.76) --
	( 94.82, 65.76) --
	( 94.89, 65.75) --
	( 94.89, 65.75) --
	( 94.89, 65.75) --
	( 94.95, 65.74) --
	( 94.95, 65.74) --
	( 94.95, 65.74) --
	( 94.96, 65.74) --
	( 94.96, 65.74) --
	( 94.96, 65.74) --
	( 95.03, 65.78) --
	( 95.03, 65.78) --
	( 95.03, 65.78) --
	( 95.10, 65.74) --
	( 95.10, 65.74) --
	( 95.10, 65.74) --
	( 95.17, 65.74) --
	( 95.17, 65.74) --
	( 95.17, 65.74) --
	( 95.24, 65.73) --
	( 95.24, 65.73) --
	( 95.24, 65.73) --
	( 95.31, 65.76) --
	( 95.31, 65.76) --
	( 95.31, 65.76) --
	( 95.38, 65.72) --
	( 95.38, 65.72) --
	( 95.38, 65.72) --
	( 95.45, 65.76) --
	( 95.45, 65.76) --
	( 95.45, 65.76) --
	( 95.53, 65.74) --
	( 95.53, 65.74) --
	( 95.53, 65.74) --
	( 95.60, 65.75) --
	( 95.60, 65.75) --
	( 95.60, 65.75) --
	( 95.67, 65.77) --
	( 95.67, 65.77) --
	( 95.67, 65.77) --
	( 95.74, 65.73) --
	( 95.74, 65.73) --
	( 95.74, 65.73) --
	( 95.81, 65.77) --
	( 95.81, 65.77) --
	( 95.81, 65.77) --
	( 95.88, 65.76) --
	( 95.88, 65.76) --
	( 95.88, 65.76) --
	( 95.95, 65.74) --
	( 95.95, 65.74) --
	( 95.95, 65.74) --
	( 96.02, 65.76) --
	( 96.02, 65.76) --
	( 96.02, 65.76) --
	( 96.09, 65.77) --
	( 96.09, 65.77) --
	( 96.09, 65.77) --
	( 96.16, 65.74) --
	( 96.16, 65.74) --
	( 96.16, 65.74) --
	( 96.24, 65.76) --
	( 96.24, 65.76) --
	( 96.24, 65.76) --
	( 96.31, 65.74) --
	( 96.31, 65.74) --
	( 96.31, 65.74) --
	( 96.38, 65.74) --
	( 96.38, 65.74) --
	( 96.38, 65.74) --
	( 96.45, 65.76) --
	( 96.45, 65.76) --
	( 96.45, 65.76) --
	( 96.52, 65.75) --
	( 96.52, 65.75) --
	( 96.52, 65.75) --
	( 96.59, 65.75) --
	( 96.59, 65.75) --
	( 96.59, 65.75) --
	( 96.66, 65.78) --
	( 96.66, 65.78) --
	( 96.66, 65.78) --
	( 96.73, 65.76) --
	( 96.73, 65.76) --
	( 96.73, 65.76) --
	( 96.77, 65.75) --
	( 96.77, 65.75) --
	( 96.77, 65.75) --
	( 96.80, 65.75) --
	( 96.80, 65.75) --
	( 96.80, 65.75) --
	( 96.87, 65.75) --
	( 96.87, 65.75) --
	( 96.87, 65.75) --
	( 96.94, 65.75) --
	( 96.94, 65.75) --
	( 96.94, 65.75) --
	( 97.02, 65.77) --
	( 97.02, 65.77) --
	( 97.02, 65.77) --
	( 97.09, 65.75) --
	( 97.09, 65.75) --
	( 97.09, 65.75) --
	( 97.16, 65.77) --
	( 97.16, 65.77) --
	( 97.16, 65.77) --
	( 97.23, 65.76) --
	( 97.23, 65.76) --
	( 97.23, 65.76) --
	( 97.30, 65.74) --
	( 97.30, 65.74) --
	( 97.30, 65.74) --
	( 97.37, 65.76) --
	( 97.37, 65.76) --
	( 97.37, 65.76) --
	( 97.44, 65.76) --
	( 97.44, 65.76) --
	( 97.44, 65.76) --
	( 97.51, 65.76) --
	( 97.51, 65.76) --
	( 97.51, 65.76) --
	( 97.58, 65.75) --
	( 97.58, 65.75) --
	( 97.58, 65.75) --
	( 97.65, 65.73) --
	( 97.65, 65.73) --
	( 97.65, 65.73) --
	( 97.72, 65.75) --
	( 97.72, 65.75) --
	( 97.72, 65.75) --
	( 97.80, 65.76) --
	( 97.80, 65.76) --
	( 97.80, 65.76) --
	( 97.87, 65.75) --
	( 97.87, 65.75) --
	( 97.87, 65.75) --
	( 97.94, 65.75) --
	( 97.94, 65.75) --
	( 97.94, 65.75) --
	( 98.01, 65.75) --
	( 98.01, 65.75) --
	( 98.01, 65.75) --
	( 98.08, 65.73) --
	( 98.08, 65.73) --
	( 98.08, 65.73) --
	( 98.15, 65.75) --
	( 98.15, 65.75) --
	( 98.15, 65.75) --
	( 98.22, 65.76) --
	( 98.22, 65.76) --
	( 98.22, 65.76) --
	( 98.29, 65.76) --
	( 98.29, 65.76) --
	( 98.29, 65.76) --
	( 98.36, 65.75) --
	( 98.36, 65.75) --
	( 98.36, 65.75) --
	( 98.43, 65.73) --
	( 98.43, 65.73) --
	( 98.43, 65.73) --
	( 98.50, 65.74) --
	( 98.50, 65.74) --
	( 98.50, 65.74) --
	( 98.57, 65.76) --
	( 98.57, 65.76) --
	( 98.57, 65.76) --
	( 98.65, 65.75) --
	( 98.65, 65.75) --
	( 98.65, 65.75) --
	( 98.72, 65.76) --
	( 98.72, 65.76) --
	( 98.72, 65.76) --
	( 98.79, 65.75) --
	( 98.79, 65.75) --
	( 98.79, 65.75) --
	( 98.86, 65.74) --
	( 98.86, 65.74) --
	( 98.86, 65.74) --
	( 98.93, 65.76) --
	( 98.93, 65.76) --
	( 98.93, 65.76) --
	( 99.00, 65.74) --
	( 99.00, 65.74) --
	( 99.00, 65.74) --
	( 99.07, 65.74) --
	( 99.07, 65.74) --
	( 99.07, 65.74) --
	( 99.07, 65.74) --
	( 99.07, 65.74) --
	( 99.07, 65.74) --
	( 99.14, 65.74) --
	( 99.14, 65.74) --
	( 99.14, 65.74) --
	( 99.21, 65.76) --
	( 99.21, 65.76) --
	( 99.21, 65.76) --
	( 99.28, 65.74) --
	( 99.28, 65.74) --
	( 99.28, 65.74) --
	( 99.35, 65.77) --
	( 99.35, 65.77) --
	( 99.35, 65.77) --
	( 99.42, 65.73) --
	( 99.42, 65.73) --
	( 99.42, 65.73) --
	( 99.49, 65.75) --
	( 99.49, 65.75) --
	( 99.49, 65.75) --
	( 99.56, 65.77) --
	( 99.56, 65.77) --
	( 99.56, 65.77) --
	( 99.64, 65.75) --
	( 99.64, 65.75) --
	( 99.64, 65.75) --
	( 99.71, 65.76) --
	( 99.71, 65.76) --
	( 99.71, 65.76) --
	( 99.78, 65.75) --
	( 99.78, 65.75) --
	( 99.78, 65.75) --
	( 99.85, 65.72) --
	( 99.85, 65.72) --
	( 99.85, 65.72) --
	( 99.92, 65.74) --
	( 99.92, 65.74) --
	( 99.92, 65.74) --
	( 99.99, 65.76) --
	( 99.99, 65.76) --
	( 99.99, 65.76) --
	(100.06, 65.75) --
	(100.06, 65.75) --
	(100.06, 65.75) --
	(100.13, 65.76) --
	(100.13, 65.76) --
	(100.13, 65.76) --
	(100.20, 65.73) --
	(100.20, 65.73) --
	(100.20, 65.73) --
	(100.27, 65.74) --
	(100.27, 65.74) --
	(100.27, 65.74) --
	(100.34, 65.75) --
	(100.34, 65.75) --
	(100.34, 65.75) --
	(100.41, 65.73) --
	(100.41, 65.73) --
	(100.41, 65.73) --
	(100.48, 65.75) --
	(100.48, 65.75) --
	(100.48, 65.75) --
	(100.55, 65.78) --
	(100.55, 65.78) --
	(100.55, 65.78) --
	(100.62, 65.77) --
	(100.62, 65.77) --
	(100.62, 65.77) --
	(100.70, 65.74) --
	(100.70, 65.74) --
	(100.70, 65.74) --
	(100.77, 65.76) --
	(100.77, 65.76) --
	(100.77, 65.76) --
	(100.84, 65.74) --
	(100.84, 65.74) --
	(100.84, 65.74) --
	(100.91, 65.74) --
	(100.91, 65.74) --
	(100.91, 65.74) --
	(100.98, 65.74) --
	(100.98, 65.74) --
	(100.98, 65.74) --
	(101.05, 65.78) --
	(101.05, 65.78) --
	(101.05, 65.78) --
	(101.12, 65.76) --
	(101.12, 65.76) --
	(101.12, 65.76) --
	(101.19, 65.76) --
	(101.19, 65.76) --
	(101.19, 65.76) --
	(101.26, 65.72) --
	(101.26, 65.72) --
	(101.26, 65.72) --
	(101.33, 65.76) --
	(101.33, 65.76) --
	(101.33, 65.76) --
	(101.40, 65.74) --
	(101.40, 65.74) --
	(101.40, 65.74) --
	(101.47, 65.72) --
	(101.47, 65.72) --
	(101.47, 65.72) --
	(101.54, 65.76) --
	(101.54, 65.76) --
	(101.54, 65.76) --
	(101.61, 65.74) --
	(101.61, 65.74) --
	(101.61, 65.74) --
	(101.66, 65.75) --
	(101.66, 65.75) --
	(101.66, 65.75) --
	(101.68, 65.76) --
	(101.68, 65.76) --
	(101.68, 65.76) --
	(101.75, 65.74) --
	(101.75, 65.74) --
	(101.75, 65.74) --
	(101.82, 65.78) --
	(101.82, 65.78) --
	(101.82, 65.78) --
	(101.90, 65.76) --
	(101.90, 65.76) --
	(101.90, 65.76) --
	(101.97, 65.75) --
	(101.97, 65.75) --
	(101.97, 65.75) --
	(102.04, 65.75) --
	(102.04, 65.75) --
	(102.04, 65.75) --
	(102.11, 65.75) --
	(102.11, 65.75) --
	(102.11, 65.75) --
	(102.18, 65.73) --
	(102.18, 65.73) --
	(102.18, 65.73) --
	(102.25, 65.75) --
	(102.25, 65.75) --
	(102.25, 65.75) --
	(102.32, 65.77) --
	(102.32, 65.77) --
	(102.32, 65.77) --
	(102.39, 65.76) --
	(102.39, 65.76) --
	(102.39, 65.76) --
	(102.46, 65.73) --
	(102.46, 65.73) --
	(102.46, 65.73) --
	(102.53, 65.73) --
	(102.53, 65.73) --
	(102.53, 65.73) --
	(102.60, 65.77) --
	(102.60, 65.77) --
	(102.60, 65.77) --
	(102.67, 65.76) --
	(102.67, 65.76) --
	(102.67, 65.76) --
	(102.74, 65.74) --
	(102.74, 65.74) --
	(102.74, 65.74) --
	(102.81, 65.72) --
	(102.81, 65.72) --
	(102.81, 65.72) --
	(102.88, 65.78) --
	(102.88, 65.78) --
	(102.88, 65.78) --
	(102.95, 65.75) --
	(102.95, 65.75) --
	(102.95, 65.75) --
	(103.02, 65.76) --
	(103.02, 65.76) --
	(103.02, 65.76) --
	(103.09, 65.74) --
	(103.09, 65.74) --
	(103.09, 65.74) --
	(103.16, 65.76) --
	(103.16, 65.76) --
	(103.16, 65.76) --
	(103.23, 65.75) --
	(103.23, 65.75) --
	(103.23, 65.75) --
	(103.31, 65.72) --
	(103.31, 65.72) --
	(103.31, 65.72) --
	(103.37, 65.76) --
	(103.37, 65.76) --
	(103.37, 65.76) --
	(103.45, 65.75) --
	(103.45, 65.75) --
	(103.45, 65.75) --
	(103.52, 65.76) --
	(103.52, 65.76) --
	(103.52, 65.76) --
	(103.59, 65.75) --
	(103.59, 65.75) --
	(103.59, 65.75) --
	(103.66, 65.76) --
	(103.66, 65.76) --
	(103.66, 65.76) --
	(103.73, 65.75) --
	(103.73, 65.75) --
	(103.73, 65.75) --
	(103.80, 65.74) --
	(103.80, 65.74) --
	(103.80, 65.74) --
	(103.87, 65.74) --
	(103.87, 65.74) --
	(103.87, 65.74) --
	(103.94, 65.74) --
	(103.94, 65.74) --
	(103.94, 65.74) --
	(104.01, 65.75) --
	(104.01, 65.75) --
	(104.01, 65.75) --
	(104.08, 65.74) --
	(104.08, 65.74) --
	(104.08, 65.74) --
	(104.15, 65.77) --
	(104.15, 65.77) --
	(104.15, 65.77) --
	(104.22, 65.78) --
	(104.22, 65.78) --
	(104.22, 65.78) --
	(104.29, 65.75) --
	(104.29, 65.75) --
	(104.29, 65.75) --
	(104.36, 65.74) --
	(104.36, 65.74) --
	(104.36, 65.74) --
	(104.43, 65.76) --
	(104.43, 65.76) --
	(104.43, 65.76) --
	(104.44, 65.76) --
	(104.44, 65.76) --
	(104.44, 65.76) --
	(104.50, 65.75) --
	(104.50, 65.75) --
	(104.50, 65.75) --
	(104.57, 65.76) --
	(104.57, 65.76) --
	(104.57, 65.76) --
	(104.64, 65.76) --
	(104.64, 65.76) --
	(104.64, 65.76) --
	(104.71, 65.77) --
	(104.71, 65.77) --
	(104.71, 65.77) --
	(104.78, 65.75) --
	(104.78, 65.75) --
	(104.78, 65.75) --
	(104.85, 65.77) --
	(104.85, 65.77) --
	(104.85, 65.77) --
	(104.92, 65.77) --
	(104.92, 65.77) --
	(104.92, 65.77) --
	(104.99, 65.77) --
	(104.99, 65.77) --
	(104.99, 65.77) --
	(105.06, 65.75) --
	(105.06, 65.75) --
	(105.06, 65.75) --
	(105.14, 65.75) --
	(105.14, 65.75) --
	(105.14, 65.75) --
	(105.20, 65.76) --
	(105.20, 65.76) --
	(105.20, 65.76) --
	(105.28, 65.75) --
	(105.28, 65.75) --
	(105.28, 65.75) --
	(105.34, 65.75) --
	(105.34, 65.75) --
	(105.34, 65.75) --
	(105.42, 65.71) --
	(105.42, 65.71) --
	(105.42, 65.71) --
	(105.49, 65.76) --
	(105.49, 65.76) --
	(105.49, 65.76) --
	(105.56, 65.75) --
	(105.56, 65.75) --
	(105.56, 65.75) --
	(105.63, 65.76) --
	(105.63, 65.76) --
	(105.63, 65.76) --
	(105.70, 65.75) --
	(105.70, 65.75) --
	(105.70, 65.75) --
	(105.77, 65.77) --
	(105.77, 65.77) --
	(105.77, 65.77) --
	(105.84, 65.75) --
	(105.84, 65.75) --
	(105.84, 65.75) --
	(105.91, 65.74) --
	(105.91, 65.74) --
	(105.91, 65.74) --
	(105.98, 65.76) --
	(105.98, 65.76) --
	(105.98, 65.76) --
	(106.05, 65.76) --
	(106.05, 65.76) --
	(106.05, 65.76) --
	(106.12, 65.75) --
	(106.12, 65.75) --
	(106.12, 65.75) --
	(106.19, 65.72) --
	(106.19, 65.72) --
	(106.19, 65.72) --
	(106.26, 65.75) --
	(106.26, 65.75) --
	(106.26, 65.75) --
	(106.33, 65.76) --
	(106.33, 65.76) --
	(106.33, 65.76) --
	(106.40, 65.73) --
	(106.40, 65.73) --
	(106.40, 65.73) --
	(106.47, 65.75) --
	(106.47, 65.75) --
	(106.47, 65.75) --
	(106.54, 65.75) --
	(106.54, 65.75) --
	(106.54, 65.75) --
	(106.61, 65.74) --
	(106.61, 65.74) --
	(106.61, 65.74) --
	(106.68, 65.74) --
	(106.68, 65.74) --
	(106.68, 65.74) --
	(106.75, 65.76) --
	(106.75, 65.76) --
	(106.75, 65.76) --
	(106.82, 65.75) --
	(106.82, 65.75) --
	(106.82, 65.75) --
	(106.89, 65.75) --
	(106.89, 65.75) --
	(106.89, 65.75) --
	(106.96, 65.73) --
	(106.96, 65.73) --
	(106.96, 65.73) --
	(107.03, 65.74) --
	(107.03, 65.74) --
	(107.03, 65.74) --
	(107.10, 65.74) --
	(107.10, 65.74) --
	(107.10, 65.74) --
	(107.17, 65.74) --
	(107.17, 65.74) --
	(107.17, 65.74) --
	(107.24, 65.74) --
	(107.24, 65.74) --
	(107.24, 65.74) --
	(107.31, 65.78) --
	(107.31, 65.78) --
	(107.31, 65.78) --
	(107.38, 65.76) --
	(107.38, 65.76) --
	(107.38, 65.76) --
	(107.41, 65.75) --
	(107.41, 65.75) --
	(107.41, 65.75) --
	(107.45, 65.74) --
	(107.45, 65.74) --
	(107.45, 65.74) --
	(107.52, 65.73) --
	(107.52, 65.73) --
	(107.52, 65.73) --
	(107.59, 65.75) --
	(107.59, 65.75) --
	(107.59, 65.75) --
	(107.66, 65.76) --
	(107.66, 65.76) --
	(107.66, 65.76) --
	(107.73, 65.73) --
	(107.73, 65.73) --
	(107.73, 65.73) --
	(107.80, 65.74) --
	(107.80, 65.74) --
	(107.80, 65.74) --
	(107.87, 65.75) --
	(107.87, 65.75) --
	(107.87, 65.75) --
	(107.94, 65.73) --
	(107.94, 65.73) --
	(107.94, 65.73) --
	(108.01, 65.75) --
	(108.01, 65.75) --
	(108.01, 65.75) --
	(108.08, 65.77) --
	(108.08, 65.77) --
	(108.08, 65.77) --
	(108.15, 65.76) --
	(108.15, 65.76) --
	(108.15, 65.76) --
	(108.22, 65.75) --
	(108.22, 65.75) --
	(108.22, 65.75) --
	(108.29, 65.76) --
	(108.29, 65.76) --
	(108.29, 65.76) --
	(108.36, 65.76) --
	(108.36, 65.76) --
	(108.36, 65.76) --
	(108.43, 65.76) --
	(108.43, 65.76) --
	(108.43, 65.76) --
	(108.50, 65.73) --
	(108.50, 65.73) --
	(108.50, 65.73) --
	(108.57, 65.76) --
	(108.57, 65.76) --
	(108.57, 65.76) --
	(108.64, 65.75) --
	(108.64, 65.75) --
	(108.64, 65.75) --
	(108.71, 65.76) --
	(108.71, 65.76) --
	(108.71, 65.76) --
	(108.78, 65.74) --
	(108.78, 65.74) --
	(108.78, 65.74) --
	(108.85, 65.76) --
	(108.85, 65.76) --
	(108.85, 65.76) --
	(108.92, 65.73) --
	(108.92, 65.73) --
	(108.92, 65.73) --
	(108.99, 65.76) --
	(108.99, 65.76) --
	(108.99, 65.76) --
	(109.06, 65.75) --
	(109.06, 65.75) --
	(109.06, 65.75) --
	(109.13, 65.74) --
	(109.13, 65.74) --
	(109.13, 65.74) --
	(109.20, 65.76) --
	(109.20, 65.76) --
	(109.20, 65.76) --
	(109.27, 65.76) --
	(109.27, 65.76) --
	(109.27, 65.76) --
	(109.34, 65.75) --
	(109.34, 65.75) --
	(109.34, 65.75) --
	(109.41, 65.76) --
	(109.41, 65.76) --
	(109.41, 65.76) --
	(109.48, 65.74) --
	(109.48, 65.74) --
	(109.48, 65.74) --
	(109.55, 65.76) --
	(109.55, 65.76) --
	(109.55, 65.76) --
	(109.62, 65.76) --
	(109.62, 65.76) --
	(109.62, 65.76) --
	(109.69, 65.73) --
	(109.69, 65.73) --
	(109.69, 65.73) --
	(109.76, 65.75) --
	(109.76, 65.75) --
	(109.76, 65.75) --
	(109.83, 65.75) --
	(109.83, 65.75) --
	(109.83, 65.75) --
	(109.90, 65.75) --
	(109.90, 65.75) --
	(109.90, 65.75) --
	(109.90, 65.75) --
	(109.90, 65.75) --
	(109.90, 65.75) --
	(109.97, 65.77) --
	(109.97, 65.77) --
	(109.97, 65.77) --
	(110.04, 65.74) --
	(110.04, 65.74) --
	(110.04, 65.74) --
	(110.11, 65.74) --
	(110.11, 65.74) --
	(110.11, 65.74) --
	(110.18, 65.76) --
	(110.18, 65.76) --
	(110.18, 65.76) --
	(110.25, 65.74) --
	(110.25, 65.74) --
	(110.25, 65.74) --
	(110.32, 65.75) --
	(110.32, 65.75) --
	(110.32, 65.75) --
	(110.39, 65.76) --
	(110.39, 65.76) --
	(110.39, 65.76) --
	(110.46, 65.75) --
	(110.46, 65.75) --
	(110.46, 65.75) --
	(110.53, 65.73) --
	(110.53, 65.73) --
	(110.53, 65.73) --
	(110.60, 65.73) --
	(110.60, 65.73) --
	(110.60, 65.73) --
	(110.67, 65.76) --
	(110.67, 65.76) --
	(110.67, 65.76) --
	(110.74, 65.76) --
	(110.74, 65.76) --
	(110.74, 65.76) --
	(110.81, 65.76) --
	(110.81, 65.76) --
	(110.81, 65.76) --
	(110.88, 65.75) --
	(110.88, 65.75) --
	(110.88, 65.75) --
	(110.95, 65.76) --
	(110.95, 65.76) --
	(110.95, 65.76) --
	(111.02, 65.73) --
	(111.02, 65.73) --
	(111.02, 65.73) --
	(111.09, 65.76) --
	(111.09, 65.76) --
	(111.09, 65.76) --
	(111.16, 65.76) --
	(111.16, 65.76) --
	(111.16, 65.76) --
	(111.23, 65.74) --
	(111.23, 65.74) --
	(111.23, 65.74) --
	(111.30, 65.76) --
	(111.30, 65.76) --
	(111.30, 65.76) --
	(111.37, 65.73) --
	(111.37, 65.73) --
	(111.37, 65.73) --
	(111.44, 65.77) --
	(111.44, 65.77) --
	(111.44, 65.77) --
	(111.51, 65.76) --
	(111.51, 65.76) --
	(111.51, 65.76) --
	(111.58, 65.75) --
	(111.58, 65.75) --
	(111.58, 65.75) --
	(111.65, 65.76) --
	(111.65, 65.76) --
	(111.65, 65.76) --
	(111.72, 65.77) --
	(111.72, 65.77) --
	(111.72, 65.77) --
	(111.72, 65.77) --
	(111.72, 65.77) --
	(111.72, 65.77) --
	(111.79, 65.74) --
	(111.79, 65.74) --
	(111.79, 65.74) --
	(111.86, 65.74) --
	(111.86, 65.74) --
	(111.86, 65.74) --
	(111.93, 65.79) --
	(111.93, 65.79) --
	(111.93, 65.79) --
	(112.00, 65.75) --
	(112.00, 65.75) --
	(112.00, 65.75) --
	(112.07, 65.76) --
	(112.07, 65.76) --
	(112.07, 65.76) --
	(112.14, 65.75) --
	(112.14, 65.75) --
	(112.14, 65.75) --
	(112.21, 65.77) --
	(112.21, 65.77) --
	(112.21, 65.77) --
	(112.28, 65.75) --
	(112.28, 65.75) --
	(112.28, 65.75) --
	(112.35, 65.73) --
	(112.35, 65.73) --
	(112.35, 65.73) --
	(112.42, 65.76) --
	(112.42, 65.76) --
	(112.42, 65.76) --
	(112.49, 65.77) --
	(112.49, 65.77) --
	(112.49, 65.77) --
	(112.56, 65.75) --
	(112.56, 65.75) --
	(112.56, 65.75) --
	(112.63, 65.76) --
	(112.63, 65.76) --
	(112.63, 65.76) --
	(112.70, 65.75) --
	(112.70, 65.75) --
	(112.70, 65.75) --
	(112.77, 65.73) --
	(112.77, 65.73) --
	(112.77, 65.73) --
	(112.84, 65.74) --
	(112.84, 65.74) --
	(112.84, 65.74) --
	(112.91, 65.74) --
	(112.91, 65.74) --
	(112.91, 65.74) --
	(112.98, 65.76) --
	(112.98, 65.76) --
	(112.98, 65.76) --
	(113.05, 65.76) --
	(113.05, 65.76) --
	(113.05, 65.76) --
	(113.12, 65.74) --
	(113.12, 65.74) --
	(113.12, 65.74) --
	(113.19, 65.77) --
	(113.19, 65.77) --
	(113.19, 65.77) --
	(113.26, 65.75) --
	(113.26, 65.75) --
	(113.26, 65.75) --
	(113.33, 65.74) --
	(113.33, 65.74) --
	(113.33, 65.74) --
	(113.39, 65.73) --
	(113.39, 65.73) --
	(113.39, 65.73) --
	(113.47, 65.76) --
	(113.47, 65.76) --
	(113.47, 65.76) --
	(113.54, 65.75) --
	(113.54, 65.75) --
	(113.54, 65.75) --
	(113.60, 65.76) --
	(113.60, 65.76) --
	(113.60, 65.76) --
	(113.67, 65.75) --
	(113.67, 65.75) --
	(113.67, 65.75) --
	(113.74, 65.75) --
	(113.74, 65.75) --
	(113.74, 65.75) --
	(113.81, 65.78) --
	(113.81, 65.78) --
	(113.81, 65.78) --
	(113.88, 65.74) --
	(113.88, 65.74) --
	(113.88, 65.74) --
	(113.95, 65.76) --
	(113.95, 65.76) --
	(113.95, 65.76) --
	(114.02, 65.78) --
	(114.02, 65.78) --
	(114.02, 65.78) --
	(114.09, 65.75) --
	(114.09, 65.75) --
	(114.09, 65.75) --
	(114.16, 65.75) --
	(114.16, 65.75) --
	(114.16, 65.75) --
	(114.23, 65.74) --
	(114.23, 65.74) --
	(114.23, 65.74) --
	(114.30, 65.75) --
	(114.30, 65.75) --
	(114.30, 65.75) --
	(114.37, 65.76) --
	(114.37, 65.76) --
	(114.37, 65.76) --
	(114.44, 65.73) --
	(114.44, 65.73) --
	(114.44, 65.73) --
	(114.51, 65.75) --
	(114.51, 65.75) --
	(114.51, 65.75) --
	(114.58, 65.74) --
	(114.58, 65.74) --
	(114.58, 65.74) --
	(114.65, 65.74) --
	(114.65, 65.74) --
	(114.65, 65.74) --
	(114.72, 65.75) --
	(114.72, 65.75) --
	(114.72, 65.75) --
	(114.79, 65.75) --
	(114.79, 65.75) --
	(114.79, 65.75) --
	(114.86, 65.74) --
	(114.86, 65.74) --
	(114.86, 65.74) --
	(114.93, 65.75) --
	(114.93, 65.75) --
	(114.93, 65.75) --
	(115.00, 65.74) --
	(115.00, 65.74) --
	(115.00, 65.74) --
	(115.07, 65.74) --
	(115.07, 65.74) --
	(115.07, 65.74) --
	(115.14, 65.75) --
	(115.14, 65.75) --
	(115.14, 65.75) --
	(115.21, 65.74) --
	(115.21, 65.74) --
	(115.21, 65.74) --
	(115.28, 65.76) --
	(115.28, 65.76) --
	(115.28, 65.76) --
	(115.35, 65.78) --
	(115.35, 65.78) --
	(115.35, 65.78) --
	(115.41, 65.73) --
	(115.41, 65.73) --
	(115.41, 65.73) --
	(115.48, 65.76) --
	(115.48, 65.76) --
	(115.48, 65.76) --
	(115.55, 65.76) --
	(115.55, 65.76) --
	(115.55, 65.76) --
	(115.62, 65.75) --
	(115.62, 65.75) --
	(115.62, 65.75) --
	(115.69, 65.74) --
	(115.69, 65.74) --
	(115.69, 65.74) --
	(115.76, 65.74) --
	(115.76, 65.74) --
	(115.76, 65.74) --
	(115.83, 65.76) --
	(115.83, 65.76) --
	(115.83, 65.76) --
	(115.90, 65.76) --
	(115.90, 65.76) --
	(115.90, 65.76) --
	(115.97, 65.74) --
	(115.97, 65.74) --
	(115.97, 65.74) --
	(116.04, 65.74) --
	(116.04, 65.74) --
	(116.04, 65.74) --
	(116.11, 65.75) --
	(116.11, 65.75) --
	(116.11, 65.75) --
	(116.18, 65.74) --
	(116.18, 65.74) --
	(116.18, 65.74) --
	(116.25, 65.76) --
	(116.25, 65.76) --
	(116.25, 65.76) --
	(116.32, 65.76) --
	(116.32, 65.76) --
	(116.32, 65.76) --
	(116.39, 65.74) --
	(116.39, 65.74) --
	(116.39, 65.74) --
	(116.46, 65.76) --
	(116.46, 65.76) --
	(116.46, 65.76) --
	(116.53, 65.77) --
	(116.53, 65.77) --
	(116.53, 65.77) --
	(116.60, 65.76) --
	(116.60, 65.76) --
	(116.60, 65.76) --
	(116.67, 65.75) --
	(116.67, 65.75) --
	(116.67, 65.75) --
	(116.74, 65.74) --
	(116.74, 65.74) --
	(116.74, 65.74) --
	(116.81, 65.73) --
	(116.81, 65.73) --
	(116.81, 65.73) --
	(116.87, 65.76) --
	(116.87, 65.76) --
	(116.87, 65.76) --
	(116.94, 65.74) --
	(116.94, 65.74) --
	(116.94, 65.74) --
	(117.01, 65.75) --
	(117.01, 65.75) --
	(117.01, 65.75) --
	(117.08, 65.76) --
	(117.08, 65.76) --
	(117.08, 65.76) --
	(117.15, 65.75) --
	(117.15, 65.75) --
	(117.15, 65.75) --
	(117.22, 65.76) --
	(117.22, 65.76) --
	(117.22, 65.76) --
	(117.29, 65.75) --
	(117.29, 65.75) --
	(117.29, 65.75) --
	(117.36, 65.78) --
	(117.36, 65.78) --
	(117.36, 65.78) --
	(117.43, 65.77) --
	(117.43, 65.77) --
	(117.43, 65.77) --
	(117.50, 65.75) --
	(117.50, 65.75) --
	(117.50, 65.75) --
	(117.57, 65.75) --
	(117.57, 65.75) --
	(117.57, 65.75) --
	(117.64, 65.76) --
	(117.64, 65.76) --
	(117.64, 65.76) --
	(117.71, 65.74) --
	(117.71, 65.74) --
	(117.71, 65.74) --
	(117.78, 65.75) --
	(117.78, 65.75) --
	(117.78, 65.75) --
	(117.85, 65.77) --
	(117.85, 65.77) --
	(117.85, 65.77) --
	(117.91, 65.75) --
	(117.91, 65.75) --
	(117.91, 65.75) --
	(117.98, 65.75) --
	(117.98, 65.75) --
	(117.98, 65.75) --
	(118.05, 65.72) --
	(118.05, 65.72) --
	(118.05, 65.72) --
	(118.12, 65.76) --
	(118.12, 65.76) --
	(118.12, 65.76) --
	(118.19, 65.77) --
	(118.19, 65.77) --
	(118.19, 65.77) --
	(118.26, 65.72) --
	(118.26, 65.72) --
	(118.26, 65.72) --
	(118.33, 65.76) --
	(118.33, 65.76) --
	(118.33, 65.76) --
	(118.40, 65.76) --
	(118.40, 65.76) --
	(118.40, 65.76) --
	(118.47, 65.74) --
	(118.47, 65.74) --
	(118.47, 65.74) --
	(118.54, 65.75) --
	(118.54, 65.75) --
	(118.54, 65.75) --
	(118.61, 65.75) --
	(118.61, 65.75) --
	(118.61, 65.75) --
	(118.68, 65.77) --
	(118.68, 65.77) --
	(118.68, 65.77) --
	(118.75, 65.75) --
	(118.75, 65.75) --
	(118.75, 65.75) --
	(118.82, 65.73) --
	(118.82, 65.73) --
	(118.82, 65.73) --
	(118.88, 65.75) --
	(118.88, 65.75) --
	(118.88, 65.75) --
	(118.95, 65.74) --
	(118.95, 65.74) --
	(118.95, 65.74) --
	(119.02, 65.73) --
	(119.02, 65.73) --
	(119.02, 65.73) --
	(119.09, 65.73) --
	(119.09, 65.73) --
	(119.09, 65.73) --
	(119.16, 65.77) --
	(119.16, 65.77) --
	(119.16, 65.77) --
	(119.23, 65.73) --
	(119.23, 65.73) --
	(119.23, 65.73) --
	(119.30, 65.75) --
	(119.30, 65.75) --
	(119.30, 65.75) --
	(119.37, 65.74) --
	(119.37, 65.74) --
	(119.37, 65.74) --
	(119.44, 65.74) --
	(119.44, 65.74) --
	(119.44, 65.74) --
	(119.51, 65.76) --
	(119.51, 65.76) --
	(119.51, 65.76) --
	(119.58, 65.73) --
	(119.58, 65.73) --
	(119.58, 65.73) --
	(119.65, 65.76) --
	(119.65, 65.76) --
	(119.65, 65.76) --
	(119.72, 65.76) --
	(119.72, 65.76) --
	(119.72, 65.76) --
	(119.79, 65.73) --
	(119.79, 65.73) --
	(119.79, 65.73) --
	(119.86, 65.75) --
	(119.86, 65.75) --
	(119.86, 65.75) --
	(119.92, 65.75) --
	(119.92, 65.75) --
	(119.92, 65.75) --
	(119.99, 65.74) --
	(119.99, 65.74) --
	(119.99, 65.74) --
	(120.06, 65.73) --
	(120.06, 65.73) --
	(120.06, 65.73) --
	(120.13, 65.73) --
	(120.13, 65.73) --
	(120.13, 65.73) --
	(120.20, 65.76) --
	(120.20, 65.76) --
	(120.20, 65.76) --
	(120.27, 65.76) --
	(120.27, 65.76) --
	(120.27, 65.76) --
	(120.34, 65.74) --
	(120.34, 65.74) --
	(120.34, 65.74) --
	(120.41, 65.75) --
	(120.41, 65.75) --
	(120.41, 65.75) --
	(120.48, 65.77) --
	(120.48, 65.77) --
	(120.48, 65.77) --
	(120.55, 65.74) --
	(120.55, 65.74) --
	(120.55, 65.74) --
	(120.62, 65.74) --
	(120.62, 65.74) --
	(120.62, 65.74) --
	(120.69, 65.77) --
	(120.69, 65.77) --
	(120.69, 65.77) --
	(120.75, 65.74) --
	(120.75, 65.74) --
	(120.75, 65.74) --
	(120.82, 65.76) --
	(120.82, 65.76) --
	(120.82, 65.76) --
	(120.89, 65.75) --
	(120.89, 65.75) --
	(120.89, 65.75) --
	(120.96, 65.75) --
	(120.96, 65.75) --
	(120.96, 65.75) --
	(121.03, 65.75) --
	(121.03, 65.75) --
	(121.03, 65.75) --
	(121.10, 65.74) --
	(121.10, 65.74) --
	(121.10, 65.74) --
	(121.17, 65.75) --
	(121.17, 65.75) --
	(121.17, 65.75) --
	(121.24, 65.77) --
	(121.24, 65.77) --
	(121.24, 65.77) --
	(121.31, 65.75) --
	(121.31, 65.75) --
	(121.31, 65.75) --
	(121.38, 65.76) --
	(121.38, 65.76) --
	(121.38, 65.76) --
	(121.44, 65.79) --
	(121.44, 65.79) --
	(121.44, 65.79) --
	(121.51, 65.74) --
	(121.51, 65.74) --
	(121.51, 65.74) --
	(121.58, 65.77) --
	(121.58, 65.77) --
	(121.58, 65.77) --
	(121.65, 65.75) --
	(121.65, 65.75) --
	(121.65, 65.75) --
	(121.72, 65.75) --
	(121.72, 65.75) --
	(121.72, 65.75) --
	(121.79, 65.75) --
	(121.79, 65.75) --
	(121.79, 65.75) --
	(121.86, 65.74) --
	(121.86, 65.74) --
	(121.86, 65.74) --
	(121.93, 65.76) --
	(121.93, 65.76) --
	(121.93, 65.76) --
	(122.00, 65.75) --
	(122.00, 65.75) --
	(122.00, 65.75) --
	(122.07, 65.75) --
	(122.07, 65.75) --
	(122.07, 65.75) --
	(122.14, 65.74) --
	(122.14, 65.74) --
	(122.14, 65.74) --
	(122.20, 65.76) --
	(122.21, 65.76) --
	(122.21, 65.76) --
	(122.27, 65.72) --
	(122.27, 65.72) --
	(122.27, 65.72) --
	(122.34, 65.72) --
	(122.34, 65.72) --
	(122.34, 65.72) --
	(122.41, 65.75) --
	(122.41, 65.75) --
	(122.41, 65.75) --
	(122.48, 65.77) --
	(122.48, 65.77) --
	(122.48, 65.77) --
	(122.55, 65.75) --
	(122.55, 65.75) --
	(122.55, 65.75) --
	(122.62, 65.75) --
	(122.62, 65.75) --
	(122.62, 65.75) --
	(122.69, 65.75) --
	(122.69, 65.75) --
	(122.69, 65.75) --
	(122.76, 65.76) --
	(122.76, 65.76) --
	(122.76, 65.76) --
	(122.83, 65.73) --
	(122.83, 65.73) --
	(122.83, 65.73) --
	(122.89, 65.72) --
	(122.89, 65.72) --
	(122.89, 65.72) --
	(122.96, 65.76) --
	(122.96, 65.76) --
	(122.96, 65.76) --
	(123.03, 65.75) --
	(123.03, 65.75) --
	(123.03, 65.75) --
	(123.10, 65.76) --
	(123.10, 65.76) --
	(123.10, 65.76) --
	(123.17, 65.72) --
	(123.17, 65.72) --
	(123.17, 65.72) --
	(123.24, 65.76) --
	(123.24, 65.76) --
	(123.24, 65.76) --
	(123.31, 65.78) --
	(123.31, 65.78) --
	(123.31, 65.78) --
	(123.38, 65.75) --
	(123.38, 65.75) --
	(123.38, 65.75) --
	(123.45, 65.73) --
	(123.45, 65.73) --
	(123.45, 65.73) --
	(123.51, 65.76) --
	(123.51, 65.76) --
	(123.51, 65.76) --
	(123.58, 65.74) --
	(123.58, 65.74) --
	(123.58, 65.74) --
	(123.65, 65.73) --
	(123.65, 65.73) --
	(123.65, 65.73) --
	(123.72, 65.76) --
	(123.72, 65.76) --
	(123.72, 65.76) --
	(123.79, 65.76) --
	(123.79, 65.76) --
	(123.79, 65.76) --
	(123.86, 65.76) --
	(123.86, 65.76) --
	(123.86, 65.76) --
	(123.93, 65.75) --
	(123.93, 65.75) --
	(123.93, 65.75) --
	(124.00, 65.78) --
	(124.00, 65.78) --
	(124.00, 65.78) --
	(124.07, 65.75) --
	(124.07, 65.75) --
	(124.07, 65.75) --
	(124.13, 65.75) --
	(124.13, 65.75) --
	(124.13, 65.75) --
	(124.20, 65.76) --
	(124.20, 65.76) --
	(124.20, 65.76) --
	(124.27, 65.76) --
	(124.27, 65.76) --
	(124.27, 65.76) --
	(124.34, 65.73) --
	(124.34, 65.73) --
	(124.34, 65.73) --
	(124.41, 65.75) --
	(124.41, 65.75) --
	(124.41, 65.75) --
	(124.48, 65.77) --
	(124.48, 65.77) --
	(124.48, 65.77) --
	(124.55, 65.76) --
	(124.55, 65.76) --
	(124.55, 65.76) --
	(124.62, 65.74) --
	(124.62, 65.74) --
	(124.62, 65.74) --
	(124.69, 65.74) --
	(124.69, 65.74) --
	(124.69, 65.74) --
	(124.75, 65.76) --
	(124.75, 65.76) --
	(124.75, 65.76) --
	(124.82, 65.76) --
	(124.82, 65.76) --
	(124.82, 65.76) --
	(124.89, 65.76) --
	(124.89, 65.76) --
	(124.89, 65.76) --
	(124.96, 65.75) --
	(124.96, 65.75) --
	(124.96, 65.75) --
	(125.03, 65.75) --
	(125.03, 65.75) --
	(125.03, 65.75) --
	(125.10, 65.73) --
	(125.10, 65.73) --
	(125.10, 65.73) --
	(125.17, 65.73) --
	(125.17, 65.73) --
	(125.17, 65.73) --
	(125.24, 65.75) --
	(125.24, 65.75) --
	(125.24, 65.75) --
	(125.31, 65.75) --
	(125.31, 65.75) --
	(125.31, 65.75) --
	(125.37, 65.74) --
	(125.37, 65.74) --
	(125.37, 65.74) --
	(125.44, 65.74) --
	(125.44, 65.74) --
	(125.44, 65.74) --
	(125.51, 65.75) --
	(125.51, 65.75) --
	(125.51, 65.75) --
	(125.58, 65.73) --
	(125.58, 65.73) --
	(125.58, 65.73) --
	(125.65, 65.74) --
	(125.65, 65.74) --
	(125.65, 65.74) --
	(125.72, 65.74) --
	(125.72, 65.74) --
	(125.72, 65.74) --
	(125.79, 65.77) --
	(125.79, 65.77) --
	(125.79, 65.77) --
	(125.86, 65.75) --
	(125.86, 65.75) --
	(125.86, 65.75) --
	(125.92, 65.74) --
	(125.92, 65.74) --
	(125.92, 65.74) --
	(125.99, 65.76) --
	(125.99, 65.76) --
	(125.99, 65.76) --
	(126.06, 65.74) --
	(126.06, 65.74) --
	(126.06, 65.74) --
	(126.13, 65.76) --
	(126.13, 65.76) --
	(126.13, 65.76) --
	(126.20, 65.74) --
	(126.20, 65.74) --
	(126.20, 65.74) --
	(126.27, 65.75) --
	(126.27, 65.75) --
	(126.27, 65.75) --
	(126.34, 65.76) --
	(126.34, 65.76) --
	(126.34, 65.76) --
	(126.41, 65.72) --
	(126.41, 65.72) --
	(126.41, 65.72) --
	(126.47, 65.75) --
	(126.47, 65.75) --
	(126.47, 65.75) --
	(126.54, 65.76) --
	(126.54, 65.76) --
	(126.54, 65.76) --
	(126.61, 65.74) --
	(126.61, 65.74) --
	(126.61, 65.74) --
	(126.68, 65.74) --
	(126.68, 65.74) --
	(126.68, 65.74) --
	(126.75, 65.75) --
	(126.75, 65.75) --
	(126.75, 65.75) --
	(126.82, 65.74) --
	(126.82, 65.74) --
	(126.82, 65.74) --
	(126.89, 65.76) --
	(126.89, 65.76) --
	(126.89, 65.76) --
	(126.95, 65.74) --
	(126.95, 65.74) --
	(126.95, 65.74) --
	(127.02, 65.76) --
	(127.02, 65.76) --
	(127.02, 65.76) --
	(127.09, 65.76) --
	(127.09, 65.76) --
	(127.09, 65.76) --
	(127.16, 65.76) --
	(127.16, 65.76) --
	(127.16, 65.76) --
	(127.23, 65.76) --
	(127.23, 65.76) --
	(127.23, 65.76) --
	(127.30, 65.77) --
	(127.30, 65.77) --
	(127.30, 65.77) --
	(127.37, 65.73) --
	(127.37, 65.73) --
	(127.37, 65.73) --
	(127.44, 65.73) --
	(127.44, 65.73) --
	(127.44, 65.73) --
	(127.50, 65.75) --
	(127.50, 65.75) --
	(127.50, 65.75) --
	(127.57, 65.74) --
	(127.57, 65.74) --
	(127.57, 65.74) --
	(127.64, 65.75) --
	(127.64, 65.75) --
	(127.64, 65.75) --
	(127.71, 65.74) --
	(127.71, 65.74) --
	(127.71, 65.74) --
	(127.78, 65.77) --
	(127.78, 65.77) --
	(127.78, 65.77) --
	(127.85, 65.74) --
	(127.85, 65.74) --
	(127.85, 65.74) --
	(127.91, 65.73) --
	(127.91, 65.73) --
	(127.91, 65.73) --
	(127.98, 65.74) --
	(127.98, 65.74) --
	(127.98, 65.74) --
	(128.05, 65.77) --
	(128.05, 65.77) --
	(128.05, 65.77) --
	(128.12, 65.75) --
	(128.12, 65.75) --
	(128.12, 65.75) --
	(128.19, 65.74) --
	(128.19, 65.74) --
	(128.19, 65.74) --
	(128.26, 65.73) --
	(128.26, 65.73) --
	(128.26, 65.73) --
	(128.33, 65.75) --
	(128.33, 65.75) --
	(128.33, 65.75) --
	(128.40, 65.76) --
	(128.40, 65.76) --
	(128.40, 65.76) --
	(128.46, 65.76) --
	(128.46, 65.76) --
	(128.46, 65.76) --
	(128.53, 65.74) --
	(128.53, 65.74) --
	(128.53, 65.74) --
	(128.60, 65.77) --
	(128.60, 65.77) --
	(128.60, 65.77) --
	(128.67, 65.75) --
	(128.67, 65.75) --
	(128.67, 65.75) --
	(128.74, 65.75) --
	(128.74, 65.75) --
	(128.74, 65.75) --
	(128.81, 65.77) --
	(128.81, 65.77) --
	(128.81, 65.77) --
	(128.88, 65.73) --
	(128.88, 65.73) --
	(128.88, 65.73) --
	(128.94, 65.73) --
	(128.94, 65.73) --
	(128.94, 65.73) --
	(129.01, 65.74) --
	(129.01, 65.74) --
	(129.01, 65.74) --
	(129.08, 65.76) --
	(129.08, 65.76) --
	(129.08, 65.76) --
	(129.15, 65.76) --
	(129.15, 65.76) --
	(129.15, 65.76) --
	(129.22, 65.73) --
	(129.22, 65.73) --
	(129.22, 65.73) --
	(129.29, 65.77) --
	(129.29, 65.77) --
	(129.29, 65.77) --
	(129.35, 65.74) --
	(129.35, 65.74) --
	(129.35, 65.74) --
	(129.42, 65.74) --
	(129.42, 65.74) --
	(129.42, 65.74) --
	(129.49, 65.74) --
	(129.49, 65.74) --
	(129.49, 65.74) --
	(129.56, 65.76) --
	(129.56, 65.76) --
	(129.56, 65.76) --
	(129.63, 65.76) --
	(129.63, 65.76) --
	(129.63, 65.76) --
	(129.70, 65.76) --
	(129.70, 65.76) --
	(129.70, 65.76) --
	(129.77, 65.73) --
	(129.77, 65.73) --
	(129.77, 65.73) --
	(129.83, 65.77) --
	(129.83, 65.77) --
	(129.83, 65.77) --
	(129.90, 65.78) --
	(129.90, 65.78) --
	(129.90, 65.78) --
	(129.97, 65.74) --
	(129.97, 65.74) --
	(129.97, 65.74) --
	(130.04, 65.75) --
	(130.04, 65.75) --
	(130.04, 65.75) --
	(130.11, 65.76) --
	(130.11, 65.76) --
	(130.11, 65.76) --
	(130.18, 65.72) --
	(130.18, 65.72) --
	(130.18, 65.72) --
	(130.24, 65.74) --
	(130.24, 65.74) --
	(130.24, 65.74) --
	(130.31, 65.77) --
	(130.31, 65.77) --
	(130.31, 65.77) --
	(130.38, 65.74) --
	(130.38, 65.74) --
	(130.38, 65.74) --
	(130.45, 65.76) --
	(130.45, 65.76) --
	(130.45, 65.76) --
	(130.52, 65.75) --
	(130.52, 65.75) --
	(130.52, 65.75) --
	(130.59, 65.76) --
	(130.59, 65.76) --
	(130.59, 65.76) --
	(130.65, 65.76) --
	(130.65, 65.76) --
	(130.65, 65.76) --
	(130.72, 65.71) --
	(130.72, 65.71) --
	(130.72, 65.71) --
	(130.79, 65.76) --
	(130.79, 65.76) --
	(130.79, 65.76) --
	(130.86, 65.76) --
	(130.86, 65.76) --
	(130.86, 65.76) --
	(130.93, 65.76) --
	(130.93, 65.76) --
	(130.93, 65.76) --
	(131.00, 65.73) --
	(131.00, 65.73) --
	(131.00, 65.73) --
	(131.06, 65.76) --
	(131.06, 65.76) --
	(131.06, 65.76) --
	(131.13, 65.75) --
	(131.13, 65.75) --
	(131.13, 65.75) --
	(131.20, 65.75) --
	(131.20, 65.75) --
	(131.20, 65.75) --
	(131.27, 65.74) --
	(131.27, 65.74) --
	(131.27, 65.74) --
	(131.34, 65.74) --
	(131.34, 65.74) --
	(131.34, 65.74) --
	(131.41, 65.73) --
	(131.41, 65.73) --
	(131.41, 65.73) --
	(131.47, 65.73) --
	(131.47, 65.73) --
	(131.47, 65.73) --
	(131.54, 65.74) --
	(131.54, 65.74) --
	(131.54, 65.74) --
	(131.61, 65.76) --
	(131.61, 65.76) --
	(131.61, 65.76) --
	(131.68, 65.73) --
	(131.68, 65.73) --
	(131.68, 65.73) --
	(131.75, 65.76) --
	(131.75, 65.76) --
	(131.75, 65.76) --
	(131.82, 65.76) --
	(131.82, 65.76) --
	(131.82, 65.76) --
	(131.88, 65.75) --
	(131.88, 65.75) --
	(131.88, 65.75) --
	(131.95, 65.74) --
	(131.95, 65.74) --
	(131.95, 65.74) --
	(132.02, 65.72) --
	(132.02, 65.72) --
	(132.02, 65.72) --
	(132.09, 65.76) --
	(132.09, 65.76) --
	(132.09, 65.76) --
	(132.16, 65.77) --
	(132.16, 65.77) --
	(132.16, 65.77) --
	(132.22, 65.75) --
	(132.22, 65.75) --
	(132.22, 65.75) --
	(132.29, 65.76) --
	(132.29, 65.76) --
	(132.29, 65.76) --
	(132.36, 65.75) --
	(132.36, 65.75) --
	(132.36, 65.75) --
	(132.43, 65.74) --
	(132.43, 65.74) --
	(132.43, 65.74) --
	(132.50, 65.75) --
	(132.50, 65.75) --
	(132.50, 65.75) --
	(132.57, 65.75) --
	(132.57, 65.75) --
	(132.57, 65.75) --
	(132.63, 65.75) --
	(132.63, 65.75) --
	(132.63, 65.75) --
	(132.70, 65.73) --
	(132.70, 65.73) --
	(132.70, 65.73) --
	(132.77, 65.73) --
	(132.77, 65.73) --
	(132.77, 65.73) --
	(132.84, 65.76) --
	(132.84, 65.76) --
	(132.84, 65.76) --
	(132.91, 65.77) --
	(132.91, 65.77) --
	(132.91, 65.77) --
	(132.98, 65.74) --
	(132.98, 65.74) --
	(132.98, 65.74) --
	(133.04, 65.76) --
	(133.04, 65.76) --
	(133.04, 65.76) --
	(133.11, 65.75) --
	(133.11, 65.75) --
	(133.11, 65.75) --
	(133.18, 65.73) --
	(133.18, 65.73) --
	(133.18, 65.73) --
	(133.25, 65.76) --
	(133.25, 65.76) --
	(133.25, 65.76) --
	(133.32, 65.74) --
	(133.32, 65.74) --
	(133.32, 65.74) --
	(133.38, 65.76) --
	(133.38, 65.76) --
	(133.38, 65.76) --
	(133.45, 65.77) --
	(133.45, 65.77) --
	(133.45, 65.77) --
	(133.52, 65.72) --
	(133.52, 65.72) --
	(133.52, 65.72) --
	(133.59, 65.76) --
	(133.59, 65.76) --
	(133.59, 65.76) --
	(133.66, 65.75) --
	(133.66, 65.75) --
	(133.66, 65.75) --
	(133.72, 65.75) --
	(133.72, 65.75) --
	(133.72, 65.75) --
	(133.79, 65.74) --
	(133.79, 65.74) --
	(133.79, 65.74) --
	(133.86, 65.76) --
	(133.86, 65.76) --
	(133.86, 65.76) --
	(133.93, 65.74) --
	(133.93, 65.74) --
	(133.93, 65.74) --
	(134.00, 65.76) --
	(134.00, 65.76) --
	(134.00, 65.76) --
	(134.07, 65.76) --
	(134.07, 65.76) --
	(134.07, 65.76) --
	(134.13, 65.76) --
	(134.13, 65.76) --
	(134.13, 65.76) --
	(134.20, 65.76) --
	(134.20, 65.76) --
	(134.20, 65.76) --
	(134.27, 65.73) --
	(134.27, 65.73) --
	(134.27, 65.73) --
	(134.34, 65.75) --
	(134.34, 65.75) --
	(134.34, 65.75) --
	(134.41, 65.77) --
	(134.41, 65.77) --
	(134.41, 65.77) --
	(134.47, 65.73) --
	(134.47, 65.73) --
	(134.47, 65.73) --
	(134.54, 65.74) --
	(134.54, 65.74) --
	(134.54, 65.74) --
	(134.61, 65.76) --
	(134.61, 65.76) --
	(134.61, 65.76) --
	(134.68, 65.74) --
	(134.68, 65.74) --
	(134.68, 65.74) --
	(134.75, 65.75) --
	(134.75, 65.75) --
	(134.75, 65.75) --
	(134.81, 65.72) --
	(134.81, 65.72) --
	(134.81, 65.72) --
	(134.88, 65.76) --
	(134.88, 65.76) --
	(134.88, 65.76) --
	(134.95, 65.75) --
	(134.95, 65.75) --
	(134.95, 65.75) --
	(135.02, 65.73) --
	(135.02, 65.73) --
	(135.02, 65.73) --
	(135.09, 65.76) --
	(135.09, 65.76) --
	(135.09, 65.76) --
	(135.15, 65.78) --
	(135.15, 65.78) --
	(135.15, 65.78) --
	(135.22, 65.72) --
	(135.22, 65.72) --
	(135.22, 65.72) --
	(135.29, 65.74) --
	(135.29, 65.74) --
	(135.29, 65.74) --
	(135.36, 65.74) --
	(135.36, 65.74) --
	(135.36, 65.74) --
	(135.43, 65.73) --
	(135.43, 65.73) --
	(135.43, 65.73) --
	(135.49, 65.76) --
	(135.49, 65.76) --
	(135.49, 65.76) --
	(135.56, 65.74) --
	(135.56, 65.74) --
	(135.56, 65.74) --
	(135.63, 65.75) --
	(135.63, 65.75) --
	(135.63, 65.75) --
	(135.70, 65.74) --
	(135.70, 65.74) --
	(135.70, 65.74) --
	(135.77, 65.74) --
	(135.77, 65.74) --
	(135.77, 65.74) --
	(135.83, 65.75) --
	(135.83, 65.75) --
	(135.83, 65.75) --
	(135.90, 65.76) --
	(135.90, 65.76) --
	(135.90, 65.76) --
	(135.97, 65.72) --
	(135.97, 65.72) --
	(135.97, 65.72) --
	(136.04, 65.73) --
	(136.04, 65.73) --
	(136.04, 65.73) --
	(136.11, 65.75) --
	(136.11, 65.75) --
	(136.11, 65.75) --
	(136.17, 65.74) --
	(136.17, 65.74) --
	(136.17, 65.74) --
	(136.24, 65.75) --
	(136.24, 65.75) --
	(136.24, 65.75) --
	(136.31, 65.73) --
	(136.31, 65.73) --
	(136.31, 65.73) --
	(136.38, 65.76) --
	(136.38, 65.76) --
	(136.38, 65.76) --
	(136.44, 65.74) --
	(136.44, 65.74) --
	(136.44, 65.74) --
	(136.51, 65.73) --
	(136.51, 65.73) --
	(136.51, 65.73) --
	(136.58, 65.77) --
	(136.58, 65.77) --
	(136.58, 65.77) --
	(136.65, 65.75) --
	(136.65, 65.75) --
	(136.65, 65.75) --
	(136.72, 65.73) --
	(136.72, 65.73) --
	(136.72, 65.73) --
	(136.78, 65.74) --
	(136.78, 65.74) --
	(136.78, 65.74) --
	(136.85, 65.76) --
	(136.85, 65.76) --
	(136.85, 65.76) --
	(136.92, 65.75) --
	(136.92, 65.75) --
	(136.92, 65.75) --
	(136.99, 65.76) --
	(136.99, 65.76) --
	(136.99, 65.76) --
	(137.06, 65.72) --
	(137.06, 65.72) --
	(137.06, 65.72) --
	(137.12, 65.77) --
	(137.12, 65.77) --
	(137.12, 65.77) --
	(137.19, 65.74) --
	(137.19, 65.74) --
	(137.19, 65.74) --
	(137.26, 65.74) --
	(137.26, 65.74) --
	(137.26, 65.74) --
	(137.33, 65.74) --
	(137.33, 65.74) --
	(137.33, 65.74) --
	(137.40, 65.76) --
	(137.40, 65.76) --
	(137.40, 65.76) --
	(137.46, 65.73) --
	(137.46, 65.73) --
	(137.46, 65.73) --
	(137.53, 65.74) --
	(137.53, 65.74) --
	(137.53, 65.74) --
	(137.60, 65.74) --
	(137.60, 65.74) --
	(137.60, 65.74) --
	(137.67, 65.76) --
	(137.67, 65.76) --
	(137.67, 65.76) --
	(137.73, 65.75) --
	(137.73, 65.75) --
	(137.73, 65.75) --
	(137.80, 65.75) --
	(137.80, 65.75) --
	(137.80, 65.75) --
	(137.87, 65.77) --
	(137.87, 65.77) --
	(137.87, 65.77) --
	(137.94, 65.77) --
	(137.94, 65.77) --
	(137.94, 65.77) --
	(138.00, 65.74) --
	(138.00, 65.74) --
	(138.00, 65.74) --
	(138.07, 65.72) --
	(138.07, 65.72) --
	(138.07, 65.72) --
	(138.14, 65.75) --
	(138.14, 65.75) --
	(138.14, 65.75) --
	(138.21, 65.74) --
	(138.21, 65.74) --
	(138.21, 65.74) --
	(138.28, 65.76) --
	(138.28, 65.76) --
	(138.28, 65.76) --
	(138.34, 65.74) --
	(138.34, 65.74) --
	(138.34, 65.74) --
	(138.41, 65.75) --
	(138.41, 65.75) --
	(138.41, 65.75) --
	(138.48, 65.76) --
	(138.48, 65.76) --
	(138.48, 65.76) --
	(138.55, 65.74) --
	(138.55, 65.74) --
	(138.55, 65.74) --
	(138.61, 65.75) --
	(138.61, 65.75) --
	(138.61, 65.75) --
	(138.68, 65.74) --
	(138.68, 65.74) --
	(138.68, 65.74) --
	(138.75, 65.74) --
	(138.75, 65.74) --
	(138.75, 65.74) --
	(138.82, 65.75) --
	(138.82, 65.75) --
	(138.82, 65.75) --
	(138.89, 65.78) --
	(138.89, 65.78) --
	(138.89, 65.78) --
	(138.95, 65.73) --
	(138.95, 65.73) --
	(138.95, 65.73) --
	(139.02, 65.75) --
	(139.02, 65.75) --
	(139.02, 65.75) --
	(139.09, 65.73) --
	(139.09, 65.73) --
	(139.09, 65.73) --
	(139.16, 65.77) --
	(139.16, 65.77) --
	(139.16, 65.77) --
	(139.22, 65.75) --
	(139.22, 65.75) --
	(139.22, 65.75) --
	(139.29, 65.73) --
	(139.29, 65.73) --
	(139.29, 65.73) --
	(139.36, 65.75) --
	(139.36, 65.75) --
	(139.36, 65.75) --
	(139.43, 65.78) --
	(139.43, 65.78) --
	(139.43, 65.78) --
	(139.49, 65.74) --
	(139.49, 65.74) --
	(139.49, 65.74) --
	(139.56, 65.74) --
	(139.56, 65.74) --
	(139.56, 65.74) --
	(139.63, 65.76) --
	(139.63, 65.76) --
	(139.63, 65.76) --
	(139.70, 65.75) --
	(139.70, 65.75) --
	(139.70, 65.75) --
	(139.76, 65.76) --
	(139.76, 65.76) --
	(139.76, 65.76) --
	(139.83, 65.76) --
	(139.83, 65.76) --
	(139.83, 65.76) --
	(139.90, 65.75) --
	(139.90, 65.75) --
	(139.90, 65.75) --
	(139.97, 65.77) --
	(139.97, 65.77) --
	(139.97, 65.77) --
	(140.04, 65.74) --
	(140.04, 65.74) --
	(140.04, 65.74) --
	(140.10, 65.73) --
	(140.10, 65.73) --
	(140.10, 65.73) --
	(140.17, 65.76) --
	(140.17, 65.76) --
	(140.17, 65.76) --
	(140.24, 65.74) --
	(140.24, 65.74) --
	(140.24, 65.74) --
	(140.31, 65.75) --
	(140.31, 65.75) --
	(140.31, 65.75) --
	(140.37, 65.74) --
	(140.37, 65.74) --
	(140.37, 65.74) --
	(140.44, 65.73) --
	(140.44, 65.73) --
	(140.44, 65.73) --
	(140.51, 65.76) --
	(140.51, 65.76) --
	(140.51, 65.76) --
	(140.58, 65.74) --
	(140.58, 65.74) --
	(140.58, 65.74) --
	(140.64, 65.77) --
	(140.64, 65.77) --
	(140.64, 65.77) --
	(140.71, 65.77) --
	(140.71, 65.77) --
	(140.71, 65.77) --
	(140.78, 65.75) --
	(140.78, 65.75) --
	(140.78, 65.75) --
	(140.85, 65.75) --
	(140.85, 65.75) --
	(140.85, 65.75) --
	(140.91, 65.78) --
	(140.91, 65.78) --
	(140.91, 65.78) --
	(140.98, 65.74) --
	(140.98, 65.74) --
	(140.98, 65.74) --
	(141.05, 65.74) --
	(141.05, 65.74) --
	(141.05, 65.74) --
	(141.12, 65.74) --
	(141.12, 65.74) --
	(141.12, 65.74) --
	(141.18, 65.77) --
	(141.18, 65.77) --
	(141.18, 65.77) --
	(141.25, 65.74) --
	(141.25, 65.74) --
	(141.25, 65.74) --
	(141.32, 65.72) --
	(141.32, 65.72) --
	(141.32, 65.72) --
	(141.39, 65.76) --
	(141.39, 65.76) --
	(141.39, 65.76) --
	(141.45, 65.76) --
	(141.45, 65.76) --
	(141.45, 65.76) --
	(141.52, 65.74) --
	(141.52, 65.74) --
	(141.52, 65.74) --
	(141.59, 65.74) --
	(141.59, 65.74) --
	(141.59, 65.74) --
	(141.66, 65.75) --
	(141.66, 65.75) --
	(141.66, 65.75) --
	(141.72, 65.73) --
	(141.72, 65.73) --
	(141.72, 65.73) --
	(141.79, 65.73) --
	(141.79, 65.73) --
	(141.79, 65.73) --
	(141.86, 65.75) --
	(141.86, 65.75) --
	(141.86, 65.75) --
	(141.92, 65.73) --
	(141.92, 65.73) --
	(141.92, 65.73) --
	(141.99, 65.73) --
	(141.99, 65.73) --
	(141.99, 65.73) --
	(142.06, 65.74) --
	(142.06, 65.74) --
	(142.06, 65.74) --
	(142.13, 65.75) --
	(142.13, 65.75) --
	(142.13, 65.75) --
	(142.19, 65.76) --
	(142.19, 65.76) --
	(142.19, 65.76) --
	(142.26, 65.71) --
	(142.26, 65.71) --
	(142.26, 65.71) --
	(142.33, 65.73) --
	(142.33, 65.73) --
	(142.33, 65.73) --
	(142.40, 65.76) --
	(142.40, 65.76) --
	(142.40, 65.76) --
	(142.46, 65.73) --
	(142.46, 65.73) --
	(142.46, 65.73) --
	(142.53, 65.75) --
	(142.53, 65.75) --
	(142.53, 65.75) --
	(142.60, 65.74) --
	(142.60, 65.74) --
	(142.60, 65.74) --
	(142.67, 65.74) --
	(142.67, 65.74) --
	(142.67, 65.74) --
	(142.73, 65.74) --
	(142.73, 65.74) --
	(142.73, 65.74) --
	(142.80, 65.74) --
	(142.80, 65.74) --
	(142.80, 65.74) --
	(142.87, 65.73) --
	(142.87, 65.73) --
	(142.87, 65.73) --
	(142.94, 65.75) --
	(142.94, 65.75) --
	(142.94, 65.75) --
	(143.00, 65.73) --
	(143.00, 65.73) --
	(143.00, 65.73) --
	(143.07, 65.74) --
	(143.07, 65.74) --
	(143.07, 65.74) --
	(143.14, 65.78) --
	(143.14, 65.78) --
	(143.14, 65.78) --
	(143.21, 65.74) --
	(143.21, 65.74) --
	(143.21, 65.74) --
	(143.27, 65.75) --
	(143.27, 65.75) --
	(143.27, 65.75) --
	(143.34, 65.74) --
	(143.34, 65.74) --
	(143.34, 65.74) --
	(143.41, 65.73) --
	(143.41, 65.73) --
	(143.41, 65.73) --
	(143.48, 65.75) --
	(143.48, 65.75) --
	(143.48, 65.75) --
	(143.54, 65.74) --
	(143.54, 65.74) --
	(143.54, 65.74) --
	(143.61, 65.76) --
	(143.61, 65.76) --
	(143.61, 65.76) --
	(143.68, 65.78) --
	(143.68, 65.78) --
	(143.68, 65.78) --
	(143.74, 65.73) --
	(143.74, 65.73) --
	(143.74, 65.73) --
	(143.81, 65.76) --
	(143.81, 65.76) --
	(143.81, 65.76) --
	(143.88, 65.77) --
	(143.88, 65.77) --
	(143.88, 65.77) --
	(143.95, 65.74) --
	(143.95, 65.74) --
	(143.95, 65.74) --
	(144.01, 65.74) --
	(144.01, 65.74) --
	(144.01, 65.74) --
	(144.08, 65.73) --
	(144.08, 65.73) --
	(144.08, 65.73) --
	(144.15, 65.78) --
	(144.15, 65.78) --
	(144.15, 65.78) --
	(144.21, 65.75) --
	(144.21, 65.75) --
	(144.21, 65.75) --
	(144.28, 65.74) --
	(144.28, 65.74) --
	(144.28, 65.74) --
	(144.35, 65.78) --
	(144.35, 65.78) --
	(144.35, 65.78) --
	(144.42, 65.75) --
	(144.42, 65.75) --
	(144.42, 65.75) --
	(144.48, 65.73) --
	(144.48, 65.73) --
	(144.48, 65.73) --
	(144.55, 65.75) --
	(144.55, 65.75) --
	(144.55, 65.75) --
	(144.62, 65.75) --
	(144.62, 65.75) --
	(144.62, 65.75) --
	(144.69, 65.71) --
	(144.69, 65.71) --
	(144.69, 65.71) --
	(144.75, 65.73) --
	(144.75, 65.73) --
	(144.75, 65.73) --
	(144.82, 65.72) --
	(144.82, 65.72) --
	(144.82, 65.72) --
	(144.89, 65.75) --
	(144.89, 65.75) --
	(144.89, 65.75) --
	(144.95, 65.78) --
	(144.95, 65.78) --
	(144.95, 65.78) --
	(145.02, 65.73) --
	(145.02, 65.73) --
	(145.02, 65.73) --
	(145.09, 65.76) --
	(145.09, 65.76) --
	(145.09, 65.76) --
	(145.16, 65.76) --
	(145.16, 65.76) --
	(145.16, 65.76) --
	(145.22, 65.74) --
	(145.22, 65.74) --
	(145.22, 65.74) --
	(145.29, 65.76) --
	(145.29, 65.76) --
	(145.29, 65.76) --
	(145.36, 65.76) --
	(145.36, 65.76) --
	(145.36, 65.76) --
	(145.42, 65.73) --
	(145.42, 65.73) --
	(145.42, 65.73) --
	(145.49, 65.75) --
	(145.49, 65.75) --
	(145.49, 65.75) --
	(145.56, 65.73) --
	(145.56, 65.73) --
	(145.56, 65.73) --
	(145.63, 65.75) --
	(145.63, 65.75) --
	(145.63, 65.75) --
	(145.69, 65.75) --
	(145.69, 65.75) --
	(145.69, 65.75) --
	(145.76, 65.74) --
	(145.76, 65.74) --
	(145.76, 65.74) --
	(145.83, 65.77) --
	(145.83, 65.77) --
	(145.83, 65.77) --
	(145.89, 65.76) --
	(145.89, 65.76) --
	(145.89, 65.76) --
	(145.96, 65.75) --
	(145.96, 65.75) --
	(145.96, 65.75) --
	(146.03, 65.73) --
	(146.03, 65.73) --
	(146.03, 65.73) --
	(146.10, 65.74) --
	(146.10, 65.74) --
	(146.10, 65.74) --
	(146.16, 65.75) --
	(146.16, 65.75) --
	(146.16, 65.75) --
	(146.23, 65.75) --
	(146.23, 65.75) --
	(146.23, 65.75) --
	(146.30, 65.76) --
	(146.30, 65.76) --
	(146.30, 65.76) --
	(146.36, 65.74) --
	(146.36, 65.74) --
	(146.36, 65.74) --
	(146.43, 65.76) --
	(146.43, 65.76) --
	(146.43, 65.76) --
	(146.50, 65.74) --
	(146.50, 65.74) --
	(146.50, 65.74) --
	(146.56, 65.73) --
	(146.56, 65.73) --
	(146.56, 65.73) --
	(146.63, 65.76) --
	(146.63, 65.76) --
	(146.63, 65.76) --
	(146.70, 65.74) --
	(146.70, 65.74) --
	(146.70, 65.74) --
	(146.77, 65.73) --
	(146.77, 65.73) --
	(146.77, 65.73) --
	(146.83, 65.77) --
	(146.83, 65.77) --
	(146.83, 65.77) --
	(146.90, 65.77) --
	(146.90, 65.77) --
	(146.90, 65.77) --
	(146.97, 65.74) --
	(146.97, 65.74) --
	(146.97, 65.74) --
	(147.03, 65.73) --
	(147.03, 65.73) --
	(147.03, 65.73) --
	(147.10, 65.74) --
	(147.10, 65.74) --
	(147.10, 65.74) --
	(147.17, 65.75) --
	(147.17, 65.75) --
	(147.17, 65.75) --
	(147.23, 65.73) --
	(147.23, 65.73) --
	(147.23, 65.73) --
	(147.30, 65.75) --
	(147.30, 65.75) --
	(147.30, 65.75) --
	(147.37, 65.78) --
	(147.37, 65.78) --
	(147.37, 65.78) --
	(147.44, 65.74) --
	(147.44, 65.74) --
	(147.44, 65.74) --
	(147.50, 65.73) --
	(147.50, 65.73) --
	(147.50, 65.73) --
	(147.57, 65.75) --
	(147.57, 65.75) --
	(147.57, 65.75) --
	(147.64, 65.75) --
	(147.64, 65.75) --
	(147.64, 65.75) --
	(147.70, 65.76) --
	(147.70, 65.76) --
	(147.70, 65.76) --
	(147.77, 65.73) --
	(147.77, 65.73) --
	(147.77, 65.73) --
	(147.84, 65.75) --
	(147.84, 65.75) --
	(147.84, 65.75) --
	(147.90, 65.76) --
	(147.90, 65.76) --
	(147.90, 65.76) --
	(147.97, 65.72) --
	(147.97, 65.72) --
	(147.97, 65.72) --
	(148.04, 65.73) --
	(148.04, 65.73) --
	(148.04, 65.73) --
	(148.10, 65.76) --
	(148.10, 65.76) --
	(148.10, 65.76) --
	(148.17, 65.73) --
	(148.17, 65.73) --
	(148.17, 65.73) --
	(148.24, 65.74) --
	(148.24, 65.74) --
	(148.24, 65.74) --
	(148.31, 65.75) --
	(148.31, 65.75) --
	(148.31, 65.75) --
	(148.37, 65.74) --
	(148.37, 65.74) --
	(148.37, 65.74) --
	(148.44, 65.77) --
	(148.44, 65.77) --
	(148.44, 65.77) --
	(148.51, 65.75) --
	(148.51, 65.75) --
	(148.51, 65.75) --
	(148.57, 65.76) --
	(148.57, 65.76) --
	(148.57, 65.76) --
	(148.64, 65.77) --
	(148.64, 65.77) --
	(148.64, 65.77) --
	(148.71, 65.73) --
	(148.71, 65.73) --
	(148.71, 65.73) --
	(148.77, 65.74) --
	(148.77, 65.74) --
	(148.77, 65.74) --
	(148.84, 65.80) --
	(148.84, 65.80) --
	(148.84, 65.80) --
	(148.91, 65.73) --
	(148.91, 65.73) --
	(148.91, 65.73) --
	(148.98, 65.76) --
	(148.98, 65.76) --
	(148.98, 65.76) --
	(149.04, 65.75) --
	(149.04, 65.75) --
	(149.04, 65.75) --
	(149.11, 65.75) --
	(149.11, 65.75) --
	(149.11, 65.75) --
	(149.18, 65.76) --
	(149.18, 65.76) --
	(149.18, 65.76) --
	(149.24, 65.73) --
	(149.24, 65.73) --
	(149.24, 65.73) --
	(149.31, 65.73) --
	(149.31, 65.73) --
	(149.31, 65.73) --
	(149.38, 65.75) --
	(149.38, 65.75) --
	(149.38, 65.75) --
	(149.44, 65.74) --
	(149.44, 65.74) --
	(149.44, 65.74) --
	(149.51, 65.73) --
	(149.51, 65.73) --
	(149.51, 65.73) --
	(149.58, 65.74) --
	(149.58, 65.74) --
	(149.58, 65.74) --
	(149.64, 65.73) --
	(149.64, 65.73) --
	(149.64, 65.73) --
	(149.71, 65.73) --
	(149.71, 65.73) --
	(149.71, 65.73) --
	(149.78, 65.74) --
	(149.78, 65.74) --
	(149.78, 65.74) --
	(149.84, 65.76) --
	(149.84, 65.76) --
	(149.84, 65.76) --
	(149.91, 65.78) --
	(149.91, 65.78) --
	(149.91, 65.78) --
	(149.98, 65.74) --
	(149.98, 65.74) --
	(149.98, 65.74) --
	(150.04, 65.74) --
	(150.04, 65.74) --
	(150.04, 65.74) --
	(150.11, 65.75) --
	(150.11, 65.75) --
	(150.11, 65.75) --
	(150.18, 65.74) --
	(150.18, 65.74) --
	(150.18, 65.74) --
	(150.24, 65.75) --
	(150.24, 65.75) --
	(150.24, 65.75) --
	(150.31, 65.77) --
	(150.31, 65.77) --
	(150.31, 65.77) --
	(150.38, 65.75) --
	(150.38, 65.75) --
	(150.38, 65.75) --
	(150.44, 65.76) --
	(150.44, 65.76) --
	(150.44, 65.76) --
	(150.51, 65.73) --
	(150.51, 65.73) --
	(150.51, 65.73) --
	(150.58, 65.75) --
	(150.58, 65.75) --
	(150.58, 65.75) --
	(150.64, 65.75) --
	(150.64, 65.75) --
	(150.64, 65.75) --
	(150.71, 65.72) --
	(150.71, 65.72) --
	(150.71, 65.72) --
	(150.78, 65.74) --
	(150.78, 65.74) --
	(150.78, 65.74) --
	(150.84, 65.76) --
	(150.84, 65.76) --
	(150.84, 65.76) --
	(150.91, 65.74) --
	(150.91, 65.74) --
	(150.91, 65.74) --
	(150.98, 65.74) --
	(150.98, 65.74) --
	(150.98, 65.74) --
	(151.04, 65.77) --
	(151.04, 65.77) --
	(151.04, 65.77) --
	(151.11, 65.73) --
	(151.11, 65.73) --
	(151.11, 65.73) --
	(151.18, 65.74) --
	(151.18, 65.74) --
	(151.18, 65.74) --
	(151.24, 65.73) --
	(151.24, 65.73) --
	(151.24, 65.73) --
	(151.31, 65.76) --
	(151.31, 65.76) --
	(151.31, 65.76) --
	(151.38, 65.74) --
	(151.38, 65.74) --
	(151.38, 65.74) --
	(151.44, 65.72) --
	(151.44, 65.72) --
	(151.44, 65.72) --
	(151.51, 65.77) --
	(151.51, 65.77) --
	(151.51, 65.77) --
	(151.58, 65.76) --
	(151.58, 65.76) --
	(151.58, 65.76) --
	(151.64, 65.74) --
	(151.64, 65.74);
\definecolor{drawColor}{RGB}{248,118,109}

\node[text=drawColor,anchor=base west,inner sep=0pt, outer sep=0pt, scale=  0.57] at ( 38.75, 93.95) {Black light CFL};
\definecolor{drawColor}{RGB}{163,165,0}

\node[text=drawColor,anchor=base west,inner sep=0pt, outer sep=0pt, scale=  0.57] at ( 54.90, 79.83) {Blue LED};
\definecolor{drawColor}{gray}{0.20}

\path[draw=drawColor,line width= 0.5pt,line join=round,line cap=round] ( 16.95, 65.71) rectangle (141.29,101.01);
\end{scope}
\begin{scope}
\path[clip] ( 16.95, 25.90) rectangle (141.29, 61.21);
\definecolor{fillColor}{RGB}{255,255,255}

\path[fill=fillColor] ( 16.95, 25.90) rectangle (141.29, 61.21);
\definecolor{fillColor}{RGB}{0,176,246}

\path[fill=fillColor] ( 24.42, 25.94) --
	( 24.42, 25.94) --
	( 24.50, 25.94) --
	( 24.50, 25.94) --
	( 24.50, 25.94) --
	( 24.51, 25.94) --
	( 24.51, 25.94) --
	( 24.51, 25.94) --
	( 24.55, 25.94) --
	( 24.55, 25.94) --
	( 24.55, 25.94) --
	( 24.57, 25.94) --
	( 24.57, 25.94) --
	( 24.57, 25.94) --
	( 24.63, 25.94) --
	( 24.63, 25.94) --
	( 24.63, 25.94) --
	( 24.65, 25.94) --
	( 24.65, 25.94) --
	( 24.65, 25.94) --
	( 24.72, 25.94) --
	( 24.72, 25.94) --
	( 24.72, 25.94) --
	( 24.75, 25.94) --
	( 24.75, 25.94) --
	( 24.75, 25.94) --
	( 24.80, 25.94) --
	( 24.80, 25.94) --
	( 24.80, 25.94) --
	( 24.83, 25.94) --
	( 24.83, 25.94) --
	( 24.83, 25.94) --
	( 24.87, 25.94) --
	( 24.87, 25.94) --
	( 24.87, 25.94) --
	( 24.95, 25.94) --
	( 24.95, 25.94) --
	( 24.95, 25.94) --
	( 25.02, 25.94) --
	( 25.02, 25.94) --
	( 25.02, 25.94) --
	( 25.10, 25.94) --
	( 25.10, 25.94) --
	( 25.10, 25.94) --
	( 25.12, 25.94) --
	( 25.12, 25.94) --
	( 25.12, 25.94) --
	( 25.17, 25.94) --
	( 25.17, 25.94) --
	( 25.17, 25.94) --
	( 25.25, 25.94) --
	( 25.25, 25.94) --
	( 25.25, 25.94) --
	( 25.32, 25.94) --
	( 25.32, 25.94) --
	( 25.32, 25.94) --
	( 25.40, 25.94) --
	( 25.40, 25.94) --
	( 25.40, 25.94) --
	( 25.47, 25.94) --
	( 25.47, 25.94) --
	( 25.47, 25.94) --
	( 25.55, 25.94) --
	( 25.55, 25.94) --
	( 25.55, 25.94) --
	( 25.62, 25.94) --
	( 25.62, 25.94) --
	( 25.62, 25.94) --
	( 25.70, 25.94) --
	( 25.70, 25.94) --
	( 25.70, 25.94) --
	( 25.77, 25.94) --
	( 25.77, 25.94) --
	( 25.77, 25.94) --
	( 25.80, 25.94) --
	( 25.80, 25.94) --
	( 25.80, 25.94) --
	( 25.85, 25.94) --
	( 25.85, 25.94) --
	( 25.85, 25.94) --
	( 25.92, 25.94) --
	( 25.92, 25.94) --
	( 25.92, 25.94) --
	( 26.00, 25.94) --
	( 26.00, 25.94) --
	( 26.00, 25.94) --
	( 26.07, 25.94) --
	( 26.07, 25.94) --
	( 26.07, 25.94) --
	( 26.15, 25.94) --
	( 26.15, 25.94) --
	( 26.15, 25.94) --
	( 26.22, 25.94) --
	( 26.22, 25.94) --
	( 26.22, 25.94) --
	( 26.30, 25.94) --
	( 26.30, 25.94) --
	( 26.30, 25.94) --
	( 26.38, 25.94) --
	( 26.38, 25.94) --
	( 26.38, 25.94) --
	( 26.45, 25.94) --
	( 26.45, 25.94) --
	( 26.45, 25.94) --
	( 26.53, 25.94) --
	( 26.53, 25.94) --
	( 26.53, 25.94) --
	( 26.60, 25.94) --
	( 26.60, 25.94) --
	( 26.60, 25.94) --
	( 26.68, 25.94) --
	( 26.68, 25.94) --
	( 26.68, 25.94) --
	( 26.75, 25.94) --
	( 26.75, 25.94) --
	( 26.75, 25.94) --
	( 26.83, 25.94) --
	( 26.83, 25.94) --
	( 26.83, 25.94) --
	( 26.90, 25.94) --
	( 26.90, 25.94) --
	( 26.90, 25.94) --
	( 26.98, 25.94) --
	( 26.98, 25.94) --
	( 26.98, 25.94) --
	( 27.05, 25.94) --
	( 27.05, 25.94) --
	( 27.05, 25.94) --
	( 27.06, 25.94) --
	( 27.06, 25.94) --
	( 27.06, 25.94) --
	( 27.13, 25.94) --
	( 27.13, 25.94) --
	( 27.13, 25.94) --
	( 27.20, 25.94) --
	( 27.20, 25.94) --
	( 27.20, 25.94) --
	( 27.28, 25.94) --
	( 27.28, 25.94) --
	( 27.28, 25.94) --
	( 27.35, 25.94) --
	( 27.35, 25.94) --
	( 27.35, 25.94) --
	( 27.39, 25.94) --
	( 27.39, 25.94) --
	( 27.39, 25.94) --
	( 27.43, 25.94) --
	( 27.43, 25.94) --
	( 27.43, 25.94) --
	( 27.50, 25.94) --
	( 27.50, 25.94) --
	( 27.50, 25.94) --
	( 27.58, 25.94) --
	( 27.58, 25.94) --
	( 27.58, 25.94) --
	( 27.65, 25.95) --
	( 27.65, 25.95) --
	( 27.65, 25.95) --
	( 27.66, 25.95) --
	( 27.66, 25.95) --
	( 27.66, 25.95) --
	( 27.73, 25.95) --
	( 27.73, 25.95) --
	( 27.73, 25.95) --
	( 27.80, 25.95) --
	( 27.80, 25.95) --
	( 27.80, 25.95) --
	( 27.88, 25.95) --
	( 27.88, 25.95) --
	( 27.88, 25.95) --
	( 27.95, 25.95) --
	( 27.95, 25.95) --
	( 27.95, 25.95) --
	( 28.03, 25.95) --
	( 28.03, 25.95) --
	( 28.03, 25.95) --
	( 28.03, 25.95) --
	( 28.03, 25.95) --
	( 28.03, 25.95) --
	( 28.10, 25.95) --
	( 28.10, 25.95) --
	( 28.10, 25.95) --
	( 28.18, 25.96) --
	( 28.18, 25.96) --
	( 28.18, 25.96) --
	( 28.19, 25.96) --
	( 28.19, 25.96) --
	( 28.19, 25.96) --
	( 28.25, 25.96) --
	( 28.25, 25.96) --
	( 28.25, 25.96) --
	( 28.33, 25.96) --
	( 28.33, 25.96) --
	( 28.33, 25.96) --
	( 28.39, 25.96) --
	( 28.39, 25.96) --
	( 28.39, 25.96) --
	( 28.40, 25.96) --
	( 28.40, 25.96) --
	( 28.40, 25.96) --
	( 28.48, 25.96) --
	( 28.48, 25.96) --
	( 28.48, 25.96) --
	( 28.56, 25.96) --
	( 28.56, 25.96) --
	( 28.56, 25.96) --
	( 28.63, 25.96) --
	( 28.63, 25.96) --
	( 28.63, 25.96) --
	( 28.71, 25.97) --
	( 28.71, 25.97) --
	( 28.71, 25.97) --
	( 28.78, 25.97) --
	( 28.78, 25.97) --
	( 28.78, 25.97) --
	( 28.86, 25.97) --
	( 28.86, 25.97) --
	( 28.86, 25.97) --
	( 28.93, 25.97) --
	( 28.93, 25.97) --
	( 28.93, 25.97) --
	( 29.01, 25.97) --
	( 29.01, 25.97) --
	( 29.01, 25.97) --
	( 29.08, 25.97) --
	( 29.08, 25.97) --
	( 29.08, 25.97) --
	( 29.16, 25.97) --
	( 29.16, 25.97) --
	( 29.16, 25.97) --
	( 29.20, 25.97) --
	( 29.20, 25.97) --
	( 29.20, 25.97) --
	( 29.23, 25.98) --
	( 29.23, 25.98) --
	( 29.23, 25.98) --
	( 29.31, 25.98) --
	( 29.31, 25.98) --
	( 29.31, 25.98) --
	( 29.38, 25.98) --
	( 29.38, 25.98) --
	( 29.38, 25.98) --
	( 29.40, 25.98) --
	( 29.40, 25.98) --
	( 29.40, 25.98) --
	( 29.46, 25.98) --
	( 29.46, 25.98) --
	( 29.46, 25.98) --
	( 29.53, 25.98) --
	( 29.53, 25.98) --
	( 29.53, 25.98) --
	( 29.61, 25.98) --
	( 29.61, 25.98) --
	( 29.61, 25.98) --
	( 29.68, 25.98) --
	( 29.68, 25.98) --
	( 29.68, 25.98) --
	( 29.76, 25.98) --
	( 29.76, 25.98) --
	( 29.76, 25.98) --
	( 29.83, 25.98) --
	( 29.83, 25.98) --
	( 29.83, 25.98) --
	( 29.91, 25.98) --
	( 29.91, 25.98) --
	( 29.91, 25.98) --
	( 29.93, 25.98) --
	( 29.93, 25.98) --
	( 29.93, 25.98) --
	( 29.98, 25.98) --
	( 29.98, 25.98) --
	( 29.98, 25.98) --
	( 30.06, 25.97) --
	( 30.06, 25.97) --
	( 30.06, 25.97) --
	( 30.13, 25.97) --
	( 30.13, 25.97) --
	( 30.13, 25.97) --
	( 30.21, 25.97) --
	( 30.21, 25.97) --
	( 30.21, 25.97) --
	( 30.28, 25.97) --
	( 30.28, 25.97) --
	( 30.28, 25.97) --
	( 30.36, 25.97) --
	( 30.36, 25.97) --
	( 30.36, 25.97) --
	( 30.41, 25.97) --
	( 30.41, 25.97) --
	( 30.41, 25.97) --
	( 30.43, 25.97) --
	( 30.43, 25.97) --
	( 30.43, 25.97) --
	( 30.51, 25.97) --
	( 30.51, 25.97) --
	( 30.51, 25.97) --
	( 30.58, 25.97) --
	( 30.58, 25.97) --
	( 30.58, 25.97) --
	( 30.66, 25.97) --
	( 30.66, 25.97) --
	( 30.66, 25.97) --
	( 30.73, 25.97) --
	( 30.73, 25.97) --
	( 30.73, 25.97) --
	( 30.73, 25.97) --
	( 30.73, 25.97) --
	( 30.73, 25.97) --
	( 30.81, 25.97) --
	( 30.81, 25.97) --
	( 30.81, 25.97) --
	( 30.88, 25.97) --
	( 30.88, 25.97) --
	( 30.88, 25.97) --
	( 30.94, 25.97) --
	( 30.94, 25.97) --
	( 30.94, 25.97) --
	( 30.96, 25.97) --
	( 30.96, 25.97) --
	( 30.96, 25.97) --
	( 31.03, 25.97) --
	( 31.03, 25.97) --
	( 31.03, 25.97) --
	( 31.10, 25.97) --
	( 31.10, 25.97) --
	( 31.10, 25.97) --
	( 31.14, 25.97) --
	( 31.14, 25.97) --
	( 31.14, 25.97) --
	( 31.18, 25.97) --
	( 31.18, 25.97) --
	( 31.18, 25.97) --
	( 31.26, 25.97) --
	( 31.26, 25.97) --
	( 31.26, 25.97) --
	( 31.30, 25.97) --
	( 31.30, 25.97) --
	( 31.30, 25.97) --
	( 31.33, 25.97) --
	( 31.33, 25.97) --
	( 31.33, 25.97) --
	( 31.41, 25.97) --
	( 31.41, 25.97) --
	( 31.41, 25.97) --
	( 31.42, 25.97) --
	( 31.42, 25.97) --
	( 31.42, 25.97) --
	( 31.46, 25.97) --
	( 31.46, 25.97) --
	( 31.46, 25.97) --
	( 31.48, 25.97) --
	( 31.48, 25.97) --
	( 31.48, 25.97) --
	( 31.56, 25.97) --
	( 31.56, 25.97) --
	( 31.56, 25.97) --
	( 31.58, 25.97) --
	( 31.58, 25.97) --
	( 31.58, 25.97) --
	( 31.63, 25.96) --
	( 31.63, 25.96) --
	( 31.63, 25.96) --
	( 31.70, 25.96) --
	( 31.70, 25.96) --
	( 31.70, 25.96) --
	( 31.71, 25.96) --
	( 31.71, 25.96) --
	( 31.71, 25.96) --
	( 31.74, 25.96) --
	( 31.74, 25.96) --
	( 31.74, 25.96) --
	( 31.78, 25.96) --
	( 31.78, 25.96) --
	( 31.78, 25.96) --
	( 31.85, 25.96) --
	( 31.85, 25.96) --
	( 31.85, 25.96) --
	( 31.87, 25.96) --
	( 31.87, 25.96) --
	( 31.87, 25.96) --
	( 31.93, 25.96) --
	( 31.93, 25.96) --
	( 31.93, 25.96) --
	( 31.95, 25.96) --
	( 31.95, 25.96) --
	( 31.95, 25.96) --
	( 32.00, 25.96) --
	( 32.00, 25.96) --
	( 32.00, 25.96) --
	( 32.08, 25.96) --
	( 32.08, 25.96) --
	( 32.08, 25.96) --
	( 32.11, 25.96) --
	( 32.11, 25.96) --
	( 32.11, 25.96) --
	( 32.15, 25.96) --
	( 32.15, 25.96) --
	( 32.15, 25.96) --
	( 32.23, 25.96) --
	( 32.23, 25.96) --
	( 32.23, 25.96) --
	( 32.23, 25.96) --
	( 32.23, 25.96) --
	( 32.23, 25.96) --
	( 32.27, 25.96) --
	( 32.27, 25.96) --
	( 32.27, 25.96) --
	( 32.30, 25.96) --
	( 32.30, 25.96) --
	( 32.30, 25.96) --
	( 32.38, 25.96) --
	( 32.38, 25.96) --
	( 32.38, 25.96) --
	( 32.39, 25.96) --
	( 32.39, 25.96) --
	( 32.39, 25.96) --
	( 32.43, 25.96) --
	( 32.43, 25.96) --
	( 32.43, 25.96) --
	( 32.45, 25.96) --
	( 32.45, 25.96) --
	( 32.45, 25.96) --
	( 32.47, 25.96) --
	( 32.47, 25.96) --
	( 32.47, 25.96) --
	( 32.47, 25.96) --
	( 32.47, 25.96) --
	( 32.47, 25.96) --
	( 32.53, 25.96) --
	( 32.53, 25.96) --
	( 32.53, 25.96) --
	( 32.55, 25.96) --
	( 32.55, 25.96) --
	( 32.55, 25.96) --
	( 32.59, 25.96) --
	( 32.59, 25.96) --
	( 32.59, 25.96) --
	( 32.60, 25.96) --
	( 32.60, 25.96) --
	( 32.60, 25.96) --
	( 32.67, 25.96) --
	( 32.67, 25.96) --
	( 32.67, 25.96) --
	( 32.68, 25.96) --
	( 32.68, 25.96) --
	( 32.68, 25.96) --
	( 32.71, 25.96) --
	( 32.71, 25.96) --
	( 32.71, 25.96) --
	( 32.75, 25.96) --
	( 32.75, 25.96) --
	( 32.75, 25.96) --
	( 32.83, 25.96) --
	( 32.83, 25.96) --
	( 32.83, 25.96) --
	( 32.90, 25.96) --
	( 32.90, 25.96) --
	( 32.90, 25.96) --
	( 32.96, 25.96) --
	( 32.96, 25.96) --
	( 32.96, 25.96) --
	( 32.98, 25.96) --
	( 32.98, 25.96) --
	( 32.98, 25.96) --
	( 33.00, 25.96) --
	( 33.00, 25.96) --
	( 33.00, 25.96) --
	( 33.05, 25.95) --
	( 33.05, 25.95) --
	( 33.05, 25.95) --
	( 33.08, 25.95) --
	( 33.08, 25.95) --
	( 33.08, 25.95) --
	( 33.12, 25.95) --
	( 33.12, 25.95) --
	( 33.12, 25.95) --
	( 33.13, 25.95) --
	( 33.13, 25.95) --
	( 33.13, 25.95) --
	( 33.16, 25.95) --
	( 33.16, 25.95) --
	( 33.16, 25.95) --
	( 33.20, 25.95) --
	( 33.20, 25.95) --
	( 33.20, 25.95) --
	( 33.20, 25.95) --
	( 33.20, 25.95) --
	( 33.20, 25.95) --
	( 33.28, 25.95) --
	( 33.28, 25.95) --
	( 33.28, 25.95) --
	( 33.28, 25.95) --
	( 33.28, 25.95) --
	( 33.28, 25.95) --
	( 33.35, 25.95) --
	( 33.35, 25.95) --
	( 33.35, 25.95) --
	( 33.43, 25.95) --
	( 33.43, 25.95) --
	( 33.43, 25.95) --
	( 33.44, 25.95) --
	( 33.44, 25.95) --
	( 33.44, 25.95) --
	( 33.50, 25.95) --
	( 33.50, 25.95) --
	( 33.50, 25.95) --
	( 33.58, 25.95) --
	( 33.58, 25.95) --
	( 33.58, 25.95) --
	( 33.64, 25.95) --
	( 33.64, 25.95) --
	( 33.64, 25.95) --
	( 33.65, 25.95) --
	( 33.65, 25.95) --
	( 33.65, 25.95) --
	( 33.73, 25.95) --
	( 33.73, 25.95) --
	( 33.73, 25.95) --
	( 33.80, 25.95) --
	( 33.80, 25.95) --
	( 33.80, 25.95) --
	( 33.85, 25.95) --
	( 33.85, 25.95) --
	( 33.85, 25.95) --
	( 33.88, 25.95) --
	( 33.88, 25.95) --
	( 33.88, 25.95) --
	( 33.89, 25.95) --
	( 33.89, 25.95) --
	( 33.89, 25.95) --
	( 33.95, 25.95) --
	( 33.95, 25.95) --
	( 33.95, 25.95) --
	( 34.03, 25.95) --
	( 34.03, 25.95) --
	( 34.03, 25.95) --
	( 34.05, 25.95) --
	( 34.05, 25.95) --
	( 34.05, 25.95) --
	( 34.09, 25.95) --
	( 34.09, 25.95) --
	( 34.09, 25.95) --
	( 34.10, 25.95) --
	( 34.10, 25.95) --
	( 34.10, 25.95) --
	( 34.17, 25.95) --
	( 34.17, 25.95) --
	( 34.17, 25.95) --
	( 34.21, 25.95) --
	( 34.21, 25.95) --
	( 34.21, 25.95) --
	( 34.25, 25.94) --
	( 34.25, 25.94) --
	( 34.25, 25.94) --
	( 34.25, 25.94) --
	( 34.25, 25.94) --
	( 34.25, 25.94) --
	( 34.32, 25.94) --
	( 34.32, 25.94) --
	( 34.32, 25.94) --
	( 34.40, 25.94) --
	( 34.40, 25.94) --
	( 34.40, 25.94) --
	( 34.41, 25.94) --
	( 34.41, 25.94) --
	( 34.41, 25.94) --
	( 34.47, 25.94) --
	( 34.47, 25.94) --
	( 34.47, 25.94) --
	( 34.53, 25.94) --
	( 34.53, 25.94) --
	( 34.53, 25.94) --
	( 34.55, 25.94) --
	( 34.55, 25.94) --
	( 34.55, 25.94) --
	( 34.61, 25.94) --
	( 34.61, 25.94) --
	( 34.61, 25.94) --
	( 34.62, 25.94) --
	( 34.62, 25.94) --
	( 34.62, 25.94) --
	( 34.70, 25.94) --
	( 34.70, 25.94) --
	( 34.70, 25.94) --
	( 34.70, 25.94) --
	( 34.70, 25.94) --
	( 34.70, 25.94) --
	( 34.74, 25.94) --
	( 34.74, 25.94) --
	( 34.74, 25.94) --
	( 34.77, 25.94) --
	( 34.77, 25.94) --
	( 34.77, 25.94) --
	( 34.82, 25.94) --
	( 34.82, 25.94) --
	( 34.82, 25.94) --
	( 34.85, 25.94) --
	( 34.85, 25.94) --
	( 34.85, 25.94) --
	( 34.86, 25.94) --
	( 34.86, 25.94) --
	( 34.86, 25.94) --
	( 34.87, 25.94) --
	( 34.87, 25.94) --
	( 34.87, 25.94) --
	( 34.92, 25.94) --
	( 34.92, 25.94) --
	( 34.92, 25.94) --
	( 34.98, 25.94) --
	( 34.98, 25.94) --
	( 34.98, 25.94) --
	( 35.00, 25.95) --
	( 35.00, 25.95) --
	( 35.00, 25.95) --
	( 35.06, 25.95) --
	( 35.06, 25.95) --
	( 35.06, 25.95) --
	( 35.07, 25.95) --
	( 35.07, 25.95) --
	( 35.07, 25.95) --
	( 35.15, 25.95) --
	( 35.15, 25.95) --
	( 35.15, 25.95) --
	( 35.18, 25.95) --
	( 35.18, 25.95) --
	( 35.18, 25.95) --
	( 35.22, 25.95) --
	( 35.22, 25.95) --
	( 35.22, 25.95) --
	( 35.30, 25.96) --
	( 35.30, 25.96) --
	( 35.30, 25.96) --
	( 35.34, 25.96) --
	( 35.34, 25.96) --
	( 35.34, 25.96) --
	( 35.37, 25.96) --
	( 35.37, 25.96) --
	( 35.37, 25.96) --
	( 35.38, 25.96) --
	( 35.38, 25.96) --
	( 35.38, 25.96) --
	( 35.45, 25.96) --
	( 35.45, 25.96) --
	( 35.45, 25.96) --
	( 35.50, 25.97) --
	( 35.50, 25.97) --
	( 35.50, 25.97) --
	( 35.52, 25.97) --
	( 35.52, 25.97) --
	( 35.52, 25.97) --
	( 35.58, 25.97) --
	( 35.58, 25.97) --
	( 35.58, 25.97) --
	( 35.60, 25.97) --
	( 35.60, 25.97) --
	( 35.60, 25.97) --
	( 35.67, 25.97) --
	( 35.67, 25.97) --
	( 35.67, 25.97) --
	( 35.71, 25.97) --
	( 35.71, 25.97) --
	( 35.71, 25.97) --
	( 35.74, 25.98) --
	( 35.74, 25.98) --
	( 35.74, 25.98) --
	( 35.82, 25.98) --
	( 35.82, 25.98) --
	( 35.82, 25.98) --
	( 35.82, 25.98) --
	( 35.82, 25.98) --
	( 35.82, 25.98) --
	( 35.83, 25.98) --
	( 35.83, 25.98) --
	( 35.83, 25.98) --
	( 35.87, 25.98) --
	( 35.87, 25.98) --
	( 35.87, 25.98) --
	( 35.89, 25.98) --
	( 35.89, 25.98) --
	( 35.89, 25.98) --
	( 35.97, 25.98) --
	( 35.97, 25.98) --
	( 35.97, 25.98) --
	( 36.04, 25.98) --
	( 36.04, 25.98) --
	( 36.04, 25.98) --
	( 36.11, 25.98) --
	( 36.11, 25.98) --
	( 36.11, 25.98) --
	( 36.12, 25.98) --
	( 36.12, 25.98) --
	( 36.12, 25.98) --
	( 36.15, 25.98) --
	( 36.15, 25.98) --
	( 36.15, 25.98) --
	( 36.19, 25.97) --
	( 36.19, 25.97) --
	( 36.19, 25.97) --
	( 36.23, 25.97) --
	( 36.23, 25.97) --
	( 36.23, 25.97) --
	( 36.27, 25.97) --
	( 36.27, 25.97) --
	( 36.27, 25.97) --
	( 36.34, 25.97) --
	( 36.34, 25.97) --
	( 36.34, 25.97) --
	( 36.42, 25.97) --
	( 36.42, 25.97) --
	( 36.42, 25.97) --
	( 36.43, 25.97) --
	( 36.43, 25.97) --
	( 36.43, 25.97) --
	( 36.49, 25.97) --
	( 36.49, 25.97) --
	( 36.49, 25.97) --
	( 36.57, 25.97) --
	( 36.57, 25.97) --
	( 36.57, 25.97) --
	( 36.64, 25.97) --
	( 36.64, 25.97) --
	( 36.64, 25.97) --
	( 36.71, 25.97) --
	( 36.71, 25.97) --
	( 36.71, 25.97) --
	( 36.76, 25.97) --
	( 36.76, 25.97) --
	( 36.76, 25.97) --
	( 36.79, 25.97) --
	( 36.79, 25.97) --
	( 36.79, 25.97) --
	( 36.86, 25.97) --
	( 36.86, 25.97) --
	( 36.86, 25.97) --
	( 36.92, 25.97) --
	( 36.92, 25.97) --
	( 36.92, 25.97) --
	( 36.94, 25.97) --
	( 36.94, 25.97) --
	( 36.94, 25.97) --
	( 37.01, 25.97) --
	( 37.01, 25.97) --
	( 37.01, 25.97) --
	( 37.09, 25.97) --
	( 37.09, 25.97) --
	( 37.09, 25.97) --
	( 37.16, 25.96) --
	( 37.16, 25.96) --
	( 37.16, 25.96) --
	( 37.16, 25.96) --
	( 37.16, 25.96) --
	( 37.16, 25.96) --
	( 37.24, 25.96) --
	( 37.24, 25.96) --
	( 37.24, 25.96) --
	( 37.31, 25.96) --
	( 37.31, 25.96) --
	( 37.31, 25.96) --
	( 37.32, 25.96) --
	( 37.32, 25.96) --
	( 37.32, 25.96) --
	( 37.39, 25.96) --
	( 37.39, 25.96) --
	( 37.39, 25.96) --
	( 37.46, 25.96) --
	( 37.46, 25.96) --
	( 37.46, 25.96) --
	( 37.54, 25.96) --
	( 37.54, 25.96) --
	( 37.54, 25.96) --
	( 37.61, 25.96) --
	( 37.61, 25.96) --
	( 37.61, 25.96) --
	( 37.61, 25.96) --
	( 37.61, 25.96) --
	( 37.61, 25.96) --
	( 37.65, 25.96) --
	( 37.65, 25.96) --
	( 37.65, 25.96) --
	( 37.69, 25.96) --
	( 37.69, 25.96) --
	( 37.69, 25.96) --
	( 37.76, 25.97) --
	( 37.76, 25.97) --
	( 37.76, 25.97) --
	( 37.83, 25.98) --
	( 37.83, 25.98) --
	( 37.83, 25.98) --
	( 37.91, 25.99) --
	( 37.91, 25.99) --
	( 37.91, 25.99) --
	( 37.97, 25.99) --
	( 37.97, 25.99) --
	( 37.97, 25.99) --
	( 37.98, 25.99) --
	( 37.98, 25.99) --
	( 37.98, 25.99) --
	( 38.06, 26.00) --
	( 38.06, 26.00) --
	( 38.06, 26.00) --
	( 38.13, 26.01) --
	( 38.13, 26.01) --
	( 38.13, 26.01) --
	( 38.21, 26.02) --
	( 38.21, 26.02) --
	( 38.21, 26.02) --
	( 38.28, 26.02) --
	( 38.28, 26.02) --
	( 38.28, 26.02) --
	( 38.33, 26.03) --
	( 38.33, 26.03) --
	( 38.33, 26.03) --
	( 38.36, 26.03) --
	( 38.36, 26.03) --
	( 38.36, 26.03) --
	( 38.43, 26.04) --
	( 38.43, 26.04) --
	( 38.43, 26.04) --
	( 38.51, 26.05) --
	( 38.51, 26.05) --
	( 38.51, 26.05) --
	( 38.58, 26.05) --
	( 38.58, 26.05) --
	( 38.58, 26.05) --
	( 38.65, 26.06) --
	( 38.65, 26.06) --
	( 38.65, 26.06) --
	( 38.73, 26.07) --
	( 38.73, 26.07) --
	( 38.73, 26.07) --
	( 38.80, 26.08) --
	( 38.80, 26.08) --
	( 38.80, 26.08) --
	( 38.82, 26.08) --
	( 38.82, 26.08) --
	( 38.82, 26.08) --
	( 38.88, 26.08) --
	( 38.88, 26.08) --
	( 38.88, 26.08) --
	( 38.95, 26.09) --
	( 38.95, 26.09) --
	( 38.95, 26.09) --
	( 38.99, 26.10) --
	( 38.99, 26.10) --
	( 38.99, 26.10) --
	( 39.03, 26.10) --
	( 39.03, 26.10) --
	( 39.03, 26.10) --
	( 39.10, 26.12) --
	( 39.10, 26.12) --
	( 39.10, 26.12) --
	( 39.14, 26.13) --
	( 39.14, 26.13) --
	( 39.14, 26.13) --
	( 39.18, 26.13) --
	( 39.18, 26.13) --
	( 39.18, 26.13) --
	( 39.25, 26.15) --
	( 39.25, 26.15) --
	( 39.25, 26.15) --
	( 39.33, 26.16) --
	( 39.33, 26.16) --
	( 39.33, 26.16) --
	( 39.40, 26.18) --
	( 39.40, 26.18) --
	( 39.40, 26.18) --
	( 39.47, 26.19) --
	( 39.47, 26.19) --
	( 39.47, 26.19) --
	( 39.47, 26.19) --
	( 39.47, 26.19) --
	( 39.47, 26.19) --
	( 39.51, 26.20) --
	( 39.51, 26.20) --
	( 39.51, 26.20) --
	( 39.55, 26.21) --
	( 39.55, 26.21) --
	( 39.55, 26.21) --
	( 39.62, 26.22) --
	( 39.62, 26.22) --
	( 39.62, 26.22) --
	( 39.67, 26.23) --
	( 39.67, 26.23) --
	( 39.67, 26.23) --
	( 39.70, 26.24) --
	( 39.70, 26.24) --
	( 39.70, 26.24) --
	( 39.71, 26.24) --
	( 39.71, 26.24) --
	( 39.71, 26.24) --
	( 39.77, 26.25) --
	( 39.77, 26.25) --
	( 39.77, 26.25) --
	( 39.79, 26.26) --
	( 39.79, 26.26) --
	( 39.79, 26.26) --
	( 39.83, 26.27) --
	( 39.83, 26.27) --
	( 39.83, 26.27) --
	( 39.85, 26.27) --
	( 39.85, 26.27) --
	( 39.85, 26.27) --
	( 39.87, 26.27) --
	( 39.87, 26.27) --
	( 39.87, 26.27) --
	( 39.91, 26.28) --
	( 39.91, 26.28) --
	( 39.91, 26.28) --
	( 39.92, 26.28) --
	( 39.92, 26.28) --
	( 39.92, 26.28) --
	( 40.00, 26.30) --
	( 40.00, 26.30) --
	( 40.00, 26.30) --
	( 40.03, 26.31) --
	( 40.03, 26.31) --
	( 40.03, 26.31) --
	( 40.07, 26.31) --
	( 40.07, 26.31) --
	( 40.07, 26.31) --
	( 40.14, 26.33) --
	( 40.14, 26.33) --
	( 40.14, 26.33) --
	( 40.22, 26.34) --
	( 40.22, 26.34) --
	( 40.22, 26.34) --
	( 40.29, 26.36) --
	( 40.29, 26.36) --
	( 40.29, 26.36) --
	( 40.35, 26.37) --
	( 40.35, 26.37) --
	( 40.35, 26.37) --
	( 40.37, 26.37) --
	( 40.37, 26.37) --
	( 40.37, 26.37) --
	( 40.44, 26.39) --
	( 40.44, 26.39) --
	( 40.44, 26.39) --
	( 40.52, 26.40) --
	( 40.52, 26.40) --
	( 40.52, 26.40) --
	( 40.52, 26.41) --
	( 40.52, 26.41) --
	( 40.52, 26.41) --
	( 40.59, 26.44) --
	( 40.59, 26.44) --
	( 40.59, 26.44) --
	( 40.67, 26.48) --
	( 40.67, 26.48) --
	( 40.67, 26.48) --
	( 40.74, 26.52) --
	( 40.74, 26.52) --
	( 40.74, 26.52) --
	( 40.81, 26.55) --
	( 40.81, 26.55) --
	( 40.81, 26.55) --
	( 40.89, 26.59) --
	( 40.89, 26.59) --
	( 40.89, 26.59) --
	( 40.92, 26.61) --
	( 40.92, 26.61) --
	( 40.92, 26.61) --
	( 40.96, 26.63) --
	( 40.96, 26.63) --
	( 40.96, 26.63) --
	( 41.04, 26.67) --
	( 41.04, 26.67) --
	( 41.04, 26.67) --
	( 41.11, 26.70) --
	( 41.11, 26.70) --
	( 41.11, 26.70) --
	( 41.19, 26.74) --
	( 41.19, 26.74) --
	( 41.19, 26.74) --
	( 41.26, 26.78) --
	( 41.26, 26.78) --
	( 41.26, 26.78) --
	( 41.29, 26.79) --
	( 41.29, 26.79) --
	( 41.29, 26.79) --
	( 41.33, 26.85) --
	( 41.33, 26.85) --
	( 41.33, 26.85) --
	( 41.41, 26.94) --
	( 41.41, 26.94) --
	( 41.41, 26.94) --
	( 41.48, 27.03) --
	( 41.48, 27.03) --
	( 41.48, 27.03) --
	( 41.56, 27.12) --
	( 41.56, 27.12) --
	( 41.56, 27.12) --
	( 41.63, 27.21) --
	( 41.63, 27.21) --
	( 41.63, 27.21) --
	( 41.67, 27.26) --
	( 41.67, 27.26) --
	( 41.67, 27.26) --
	( 41.71, 27.31) --
	( 41.71, 27.31) --
	( 41.71, 27.31) --
	( 41.78, 27.42) --
	( 41.78, 27.42) --
	( 41.78, 27.42) --
	( 41.86, 27.53) --
	( 41.86, 27.53) --
	( 41.86, 27.53) --
	( 41.93, 27.64) --
	( 41.93, 27.64) --
	( 41.93, 27.64) --
	( 42.00, 27.75) --
	( 42.00, 27.75) --
	( 42.00, 27.75) --
	( 42.05, 27.82) --
	( 42.05, 27.82) --
	( 42.05, 27.82) --
	( 42.08, 27.86) --
	( 42.08, 27.86) --
	( 42.08, 27.86) --
	( 42.15, 27.97) --
	( 42.15, 27.97) --
	( 42.15, 27.97) --
	( 42.21, 28.07) --
	( 42.21, 28.07) --
	( 42.21, 28.07) --
	( 42.23, 28.09) --
	( 42.23, 28.09) --
	( 42.23, 28.09) --
	( 42.30, 28.20) --
	( 42.30, 28.20) --
	( 42.30, 28.20) --
	( 42.38, 28.32) --
	( 42.38, 28.32) --
	( 42.38, 28.32) --
	( 42.45, 28.43) --
	( 42.45, 28.43) --
	( 42.45, 28.43) --
	( 42.52, 28.55) --
	( 42.52, 28.55) --
	( 42.52, 28.55) --
	( 42.53, 28.56) --
	( 42.53, 28.56) --
	( 42.53, 28.56) --
	( 42.60, 28.67) --
	( 42.60, 28.67) --
	( 42.60, 28.67) --
	( 42.67, 28.80) --
	( 42.67, 28.80) --
	( 42.67, 28.80) --
	( 42.75, 28.94) --
	( 42.75, 28.94) --
	( 42.75, 28.94) --
	( 42.82, 29.06) --
	( 42.82, 29.06) --
	( 42.82, 29.06) --
	( 42.82, 29.07) --
	( 42.82, 29.07) --
	( 42.82, 29.07) --
	( 42.90, 29.29) --
	( 42.90, 29.29) --
	( 42.90, 29.29) --
	( 42.97, 29.52) --
	( 42.97, 29.52) --
	( 42.97, 29.52) --
	( 43.01, 29.65) --
	( 43.01, 29.65) --
	( 43.01, 29.65) --
	( 43.04, 29.71) --
	( 43.04, 29.71) --
	( 43.04, 29.71) --
	( 43.12, 29.87) --
	( 43.12, 29.87) --
	( 43.12, 29.87) --
	( 43.19, 30.02) --
	( 43.19, 30.02) --
	( 43.19, 30.02) --
	( 43.27, 30.18) --
	( 43.27, 30.18) --
	( 43.27, 30.18) --
	( 43.30, 30.25) --
	( 43.30, 30.25) --
	( 43.30, 30.25) --
	( 43.34, 30.36) --
	( 43.34, 30.36) --
	( 43.34, 30.36) --
	( 43.42, 30.56) --
	( 43.42, 30.56) --
	( 43.42, 30.56) --
	( 43.49, 30.77) --
	( 43.49, 30.77) --
	( 43.49, 30.77) --
	( 43.49, 30.77) --
	( 43.49, 30.77) --
	( 43.49, 30.77) --
	( 43.56, 31.30) --
	( 43.56, 31.30) --
	( 43.56, 31.30) --
	( 43.59, 31.47) --
	( 43.59, 31.47) --
	( 43.59, 31.47) --
	( 43.64, 31.55) --
	( 43.64, 31.55) --
	( 43.64, 31.56) --
	( 43.71, 31.68) --
	( 43.71, 31.68) --
	( 43.71, 31.68) --
	( 43.79, 31.80) --
	( 43.79, 31.80) --
	( 43.79, 31.80) --
	( 43.86, 31.93) --
	( 43.86, 31.93) --
	( 43.86, 31.93) --
	( 43.94, 32.05) --
	( 43.94, 32.05) --
	( 43.94, 32.05) --
	( 43.97, 32.11) --
	( 43.97, 32.69) --
	( 43.97, 32.69) --
	( 44.01, 32.81) --
	( 44.01, 32.81) --
	( 44.01, 32.81) --
	( 44.08, 33.04) --
	( 44.08, 33.04) --
	( 44.08, 33.04) --
	( 44.16, 33.26) --
	( 44.16, 33.26) --
	( 44.16, 33.26) --
	( 44.16, 33.27) --
	( 44.16, 33.27) --
	( 44.16, 33.27) --
	( 44.23, 33.41) --
	( 44.23, 33.41) --
	( 44.23, 33.41) --
	( 44.31, 33.55) --
	( 44.31, 33.55) --
	( 44.31, 33.55) --
	( 44.38, 33.69) --
	( 44.38, 33.69) --
	( 44.38, 33.69) --
	( 44.45, 33.82) --
	( 44.45, 34.42) --
	( 44.45, 34.42) --
	( 44.46, 34.44) --
	( 44.46, 34.44) --
	( 44.46, 34.44) --
	( 44.53, 34.69) --
	( 44.53, 34.69) --
	( 44.53, 34.69) --
	( 44.60, 34.93) --
	( 44.60, 34.93) --
	( 44.60, 34.93) --
	( 44.64, 35.06) --
	( 44.64, 35.06) --
	( 44.64, 35.06) --
	( 44.68, 35.27) --
	( 44.68, 35.27) --
	( 44.68, 35.27) --
	( 44.74, 35.60) --
	( 44.74, 36.14) --
	( 44.74, 36.14) --
	( 44.75, 36.19) --
	( 44.75, 36.19) --
	( 44.75, 36.19) --
	( 44.83, 36.40) --
	( 44.83, 36.40) --
	( 44.83, 36.40) --
	( 44.90, 36.61) --
	( 44.90, 36.61) --
	( 44.90, 36.61) --
	( 44.93, 36.69) --
	( 44.93, 36.69) --
	( 44.93, 36.69) --
	( 44.98, 36.97) --
	( 44.98, 36.97) --
	( 44.98, 36.97) --
	( 45.02, 37.27) --
	( 45.02, 37.83) --
	( 45.02, 37.83) --
	( 45.05, 37.87) --
	( 45.05, 37.87) --
	( 45.05, 37.87) --
	( 45.12, 37.99) --
	( 45.12, 37.99) --
	( 45.12, 37.99) --
	( 45.20, 38.11) --
	( 45.20, 38.11) --
	( 45.20, 38.11) --
	( 45.27, 38.23) --
	( 45.27, 38.23) --
	( 45.27, 38.23) --
	( 45.31, 38.30) --
	( 45.31, 38.84) --
	( 45.31, 38.84) --
	( 45.35, 39.02) --
	( 45.35, 39.02) --
	( 45.35, 39.02) --
	( 45.41, 39.36) --
	( 45.41, 39.83) --
	( 45.41, 39.83) --
	( 45.42, 39.89) --
	( 45.42, 39.89) --
	( 45.42, 39.89) --
	( 45.49, 40.27) --
	( 45.49, 40.27) --
	( 45.49, 40.27) --
	( 45.50, 40.31) --
	( 45.50, 40.31) --
	( 45.50, 40.31) --
	( 45.57, 40.58) --
	( 45.57, 40.58) --
	( 45.57, 40.58) --
	( 45.60, 40.70) --
	( 45.60, 40.70) --
	( 45.60, 40.70) --
	( 45.64, 40.96) --
	( 45.64, 40.96) --
	( 45.64, 40.96) --
	( 45.70, 41.27) --
	( 45.70, 41.27) --
	( 45.70, 41.27) --
	( 45.72, 41.37) --
	( 45.72, 41.37) --
	( 45.72, 41.37) --
	( 45.79, 41.71) --
	( 45.79, 41.71) --
	( 45.79, 41.71) --
	( 45.79, 41.71) --
	( 45.79, 41.71) --
	( 45.79, 41.71) --
	( 45.87, 41.87) --
	( 45.87, 41.87) --
	( 45.87, 41.87) --
	( 45.94, 42.03) --
	( 45.94, 42.03) --
	( 45.94, 42.03) --
	( 45.98, 42.12) --
	( 45.98, 42.12) --
	( 45.98, 42.12) --
	( 46.01, 42.23) --
	( 46.01, 42.23) --
	( 46.01, 42.23) --
	( 46.08, 42.49) --
	( 46.08, 42.86) --
	( 46.08, 42.86) --
	( 46.09, 42.90) --
	( 46.09, 42.90) --
	( 46.09, 42.90) --
	( 46.16, 43.29) --
	( 46.16, 43.29) --
	( 46.16, 43.29) --
	( 46.17, 43.36) --
	( 46.17, 43.36) --
	( 46.17, 43.36) --
	( 46.24, 43.40) --
	( 46.24, 43.40) --
	( 46.24, 43.40) --
	( 46.31, 43.45) --
	( 46.31, 43.45) --
	( 46.31, 43.45) --
	( 46.38, 43.50) --
	( 46.38, 43.50) --
	( 46.38, 43.50) --
	( 46.46, 43.55) --
	( 46.46, 43.55) --
	( 46.46, 43.55) --
	( 46.46, 43.55) --
	( 46.46, 43.55) --
	( 46.46, 43.55) --
	( 46.53, 43.50) --
	( 46.53, 43.50) --
	( 46.53, 43.50) --
	( 46.61, 43.44) --
	( 46.61, 43.44) --
	( 46.61, 43.44) --
	( 46.68, 43.38) --
	( 46.68, 43.38) --
	( 46.68, 43.38) --
	( 46.75, 43.32) --
	( 46.75, 43.32) --
	( 46.75, 43.32) --
	( 46.75, 43.31) --
	( 46.75, 43.31) --
	( 46.75, 43.31) --
	( 46.83, 43.15) --
	( 46.83, 43.15) --
	( 46.83, 43.15) --
	( 46.85, 43.11) --
	( 46.85, 43.11) --
	( 46.85, 43.11) --
	( 46.90, 42.96) --
	( 46.90, 42.96) --
	( 46.90, 42.96) --
	( 46.94, 42.86) --
	( 46.94, 42.86) --
	( 46.94, 42.86) --
	( 46.98, 42.81) --
	( 46.98, 42.81) --
	( 46.98, 42.81) --
	( 47.05, 42.72) --
	( 47.05, 42.72) --
	( 47.05, 42.72) --
	( 47.13, 42.63) --
	( 47.13, 42.63) --
	( 47.13, 42.63) --
	( 47.13, 42.62) --
	( 47.13, 42.35) --
	( 47.13, 42.35) --
	( 47.20, 42.14) --
	( 47.20, 42.14) --
	( 47.20, 42.14) --
	( 47.23, 42.04) --
	( 47.23, 41.05) --
	( 47.23, 41.05) --
	( 47.27, 41.31) --
	( 47.27, 41.31) --
	( 47.27, 41.31) --
	( 47.32, 41.61) --
	( 47.32, 41.61) --
	( 47.32, 41.61) --
	( 47.35, 41.49) --
	( 47.35, 41.49) --
	( 47.35, 41.49) --
	( 47.42, 41.08) --
	( 47.42, 41.08) --
	( 47.42, 41.08) --
	( 47.50, 40.67) --
	( 47.50, 40.67) --
	( 47.50, 40.67) --
	( 47.52, 40.55) --
	( 47.52, 40.55) --
	( 47.52, 40.55) --
	( 47.57, 40.33) --
	( 47.57, 40.33) --
	( 47.57, 40.33) --
	( 47.61, 40.16) --
	( 47.61, 39.67) --
	( 47.61, 39.67) --
	( 47.64, 39.62) --
	( 47.64, 39.62) --
	( 47.64, 39.62) --
	( 47.72, 39.49) --
	( 47.72, 39.49) --
	( 47.72, 39.49) --
	( 47.79, 39.37) --
	( 47.79, 39.37) --
	( 47.79, 39.37) --
	( 47.80, 39.34) --
	( 47.80, 38.82) --
	( 47.80, 38.82) --
	( 47.86, 38.54) --
	( 47.86, 38.54) --
	( 47.86, 38.54) --
	( 47.90, 38.37) --
	( 47.90, 38.37) --
	( 47.90, 38.37) --
	( 47.94, 38.18) --
	( 47.94, 38.18) --
	( 47.94, 38.18) --
	( 48.00, 37.91) --
	( 48.00, 37.91) --
	( 48.00, 37.91) --
	( 48.01, 37.87) --
	( 48.01, 37.87) --
	( 48.01, 37.87) --
	( 48.09, 37.69) --
	( 48.09, 37.69) --
	( 48.09, 37.69) --
	( 48.16, 37.51) --
	( 48.16, 37.51) --
	( 48.16, 37.51) --
	( 48.19, 37.44) --
	( 48.19, 37.44) --
	( 48.19, 37.44) --
	( 48.23, 37.18) --
	( 48.23, 37.18) --
	( 48.23, 37.18) --
	( 48.28, 36.90) --
	( 48.28, 36.28) --
	( 48.28, 36.28) --
	( 48.31, 36.20) --
	( 48.31, 36.20) --
	( 48.31, 36.20) --
	( 48.38, 35.99) --
	( 48.38, 35.99) --
	( 48.38, 35.99) --
	( 48.46, 35.77) --
	( 48.46, 35.77) --
	( 48.46, 35.77) --
	( 48.47, 35.72) --
	( 48.47, 35.72) --
	( 48.47, 35.72) --
	( 48.53, 35.51) --
	( 48.53, 35.51) --
	( 48.53, 35.51) --
	( 48.60, 35.24) --
	( 48.60, 35.24) --
	( 48.60, 35.24) --
	( 48.67, 35.02) --
	( 48.67, 35.02) --
	( 48.67, 35.02) --
	( 48.68, 34.93) --
	( 48.68, 34.93) --
	( 48.68, 34.93) --
	( 48.75, 34.39) --
	( 48.75, 34.39) --
	( 48.75, 34.39) --
	( 48.76, 34.32) --
	( 48.76, 34.32) --
	( 48.76, 34.32) --
	( 48.83, 34.13) --
	( 48.83, 34.13) --
	( 48.83, 34.13) --
	( 48.90, 33.91) --
	( 48.90, 33.91) --
	( 48.90, 33.91) --
	( 48.95, 33.76) --
	( 48.95, 33.76) --
	( 48.95, 33.76) --
	( 48.97, 33.63) --
	( 48.97, 33.63) --
	( 48.97, 33.63) --
	( 49.05, 33.15) --
	( 49.05, 33.15) --
	( 49.05, 33.15) --
	( 49.05, 33.14) --
	( 49.05, 33.14) --
	( 49.05, 33.14) --
	( 49.12, 33.02) --
	( 49.12, 33.02) --
	( 49.12, 33.02) --
	( 49.20, 32.91) --
	( 49.20, 32.91) --
	( 49.20, 32.91) --
	( 49.27, 32.79) --
	( 49.27, 32.79) --
	( 49.27, 32.79) --
	( 49.34, 32.69) --
	( 49.34, 32.69) --
	( 49.34, 32.69) --
	( 49.34, 32.67) --
	( 49.34, 32.67) --
	( 49.34, 32.67) --
	( 49.42, 32.43) --
	( 49.42, 32.43) --
	( 49.42, 32.43) --
	( 49.49, 32.19) --
	( 49.49, 32.19) --
	( 49.49, 32.19) --
	( 49.53, 32.07) --
	( 49.53, 32.07) --
	( 49.53, 32.07) --
	( 49.57, 31.94) --
	( 49.57, 31.94) --
	( 49.57, 31.94) --
	( 49.64, 31.70) --
	( 49.64, 31.70) --
	( 49.64, 31.70) --
	( 49.71, 31.45) --
	( 49.71, 31.45) --
	( 49.71, 31.45) --
	( 49.72, 31.43) --
	( 49.72, 31.43) --
	( 49.72, 31.43) --
	( 49.79, 31.27) --
	( 49.79, 31.27) --
	( 49.79, 31.27) --
	( 49.86, 31.09) --
	( 49.86, 31.09) --
	( 49.86, 31.09) --
	( 49.91, 30.96) --
	( 49.91, 30.96) --
	( 49.91, 30.96) --
	( 49.94, 30.90) --
	( 49.94, 30.90) --
	( 49.94, 30.90) --
	( 50.01, 30.72) --
	( 50.01, 30.72) --
	( 50.01, 30.72) --
	( 50.08, 30.53) --
	( 50.08, 30.53) --
	( 50.08, 30.53) --
	( 50.10, 30.48) --
	( 50.10, 30.48) --
	( 50.10, 30.48) --
	( 50.16, 30.41) --
	( 50.16, 30.41) --
	( 50.16, 30.41) --
	( 50.23, 30.32) --
	( 50.23, 30.32) --
	( 50.23, 30.32) --
	( 50.30, 30.24) --
	( 50.30, 30.24) --
	( 50.30, 30.24) --
	( 50.38, 30.15) --
	( 50.38, 30.15) --
	( 50.38, 30.15) --
	( 50.45, 30.06) --
	( 50.45, 30.06) --
	( 50.45, 30.06) --
	( 50.49, 30.01) --
	( 50.49, 30.01) --
	( 50.49, 30.01) --
	( 50.53, 29.97) --
	( 50.53, 29.97) --
	( 50.53, 29.97) --
	( 50.60, 29.89) --
	( 50.60, 29.89) --
	( 50.60, 29.89) --
	( 50.67, 29.81) --
	( 50.67, 29.81) --
	( 50.67, 29.81) --
	( 50.75, 29.72) --
	( 50.75, 29.72) --
	( 50.75, 29.72) --
	( 50.82, 29.64) --
	( 50.82, 29.64) --
	( 50.82, 29.64) --
	( 50.87, 29.59) --
	( 50.87, 29.59) --
	( 50.87, 29.59) --
	( 50.90, 29.51) --
	( 50.90, 29.51) --
	( 50.90, 29.51) --
	( 50.97, 29.30) --
	( 50.97, 29.30) --
	( 50.97, 29.30) --
	( 50.97, 29.29) --
	( 50.97, 29.29) --
	( 50.97, 29.29) --
	( 51.04, 29.21) --
	( 51.04, 29.21) --
	( 51.04, 29.21) --
	( 51.12, 29.14) --
	( 51.12, 29.14) --
	( 51.12, 29.14) --
	( 51.19, 29.06) --
	( 51.19, 29.06) --
	( 51.19, 29.06) --
	( 51.27, 28.98) --
	( 51.27, 28.98) --
	( 51.27, 28.98) --
	( 51.34, 28.90) --
	( 51.34, 28.90) --
	( 51.34, 28.90) --
	( 51.35, 28.89) --
	( 51.35, 28.89) --
	( 51.35, 28.89) --
	( 51.41, 28.84) --
	( 51.41, 28.84) --
	( 51.41, 28.84) --
	( 51.49, 28.79) --
	( 51.49, 28.79) --
	( 51.49, 28.79) --
	( 51.56, 28.74) --
	( 51.56, 28.74) --
	( 51.56, 28.74) --
	( 51.63, 28.69) --
	( 51.63, 28.69) --
	( 51.63, 28.69) --
	( 51.71, 28.63) --
	( 51.71, 28.63) --
	( 51.71, 28.63) --
	( 51.73, 28.62) --
	( 51.73, 28.62) --
	( 51.73, 28.62) --
	( 51.78, 28.60) --
	( 51.78, 28.60) --
	( 51.78, 28.60) --
	( 51.86, 28.56) --
	( 51.86, 28.56) --
	( 51.86, 28.56) --
	( 51.93, 28.53) --
	( 51.93, 28.53) --
	( 51.93, 28.53) --
	( 52.00, 28.50) --
	( 52.00, 28.50) --
	( 52.00, 28.50) --
	( 52.08, 28.46) --
	( 52.08, 28.46) --
	( 52.08, 28.46) --
	( 52.15, 28.43) --
	( 52.15, 28.43) --
	( 52.15, 28.43) --
	( 52.21, 28.40) --
	( 52.21, 28.40) --
	( 52.21, 28.40) --
	( 52.22, 28.40) --
	( 52.22, 28.40) --
	( 52.22, 28.40) --
	( 52.30, 28.38) --
	( 52.30, 28.38) --
	( 52.30, 28.38) --
	( 52.37, 28.35) --
	( 52.37, 28.35) --
	( 52.37, 28.35) --
	( 52.45, 28.33) --
	( 52.45, 28.33) --
	( 52.45, 28.33) --
	( 52.52, 28.30) --
	( 52.52, 28.30) --
	( 52.52, 28.30) --
	( 52.59, 28.28) --
	( 52.59, 28.28) --
	( 52.59, 28.28) --
	( 52.67, 28.26) --
	( 52.67, 28.26) --
	( 52.67, 28.26) --
	( 52.69, 28.25) --
	( 52.69, 28.25) --
	( 52.69, 28.25) --
	( 52.74, 28.24) --
	( 52.74, 28.24) --
	( 52.74, 28.24) --
	( 52.81, 28.23) --
	( 52.81, 28.23) --
	( 52.81, 28.23) --
	( 52.89, 28.22) --
	( 52.89, 28.22) --
	( 52.89, 28.22) --
	( 52.96, 28.22) --
	( 52.96, 28.22) --
	( 52.96, 28.22) --
	( 53.04, 28.21) --
	( 53.04, 28.21) --
	( 53.04, 28.21) --
	( 53.11, 28.20) --
	( 53.11, 28.20) --
	( 53.11, 28.20) --
	( 53.17, 28.19) --
	( 53.17, 28.19) --
	( 53.17, 28.19) --
	( 53.18, 28.19) --
	( 53.18, 28.19) --
	( 53.18, 28.19) --
	( 53.26, 28.20) --
	( 53.26, 28.20) --
	( 53.26, 28.20) --
	( 53.33, 28.22) --
	( 53.33, 28.22) --
	( 53.33, 28.22) --
	( 53.41, 28.23) --
	( 53.41, 28.23) --
	( 53.41, 28.23) --
	( 53.48, 28.24) --
	( 53.48, 28.24) --
	( 53.48, 28.24) --
	( 53.55, 28.25) --
	( 53.55, 28.25) --
	( 53.55, 28.25) --
	( 53.62, 28.27) --
	( 53.62, 28.27) --
	( 53.62, 28.27) --
	( 53.70, 28.28) --
	( 53.70, 28.28) --
	( 53.70, 28.28) --
	( 53.75, 28.29) --
	( 53.75, 28.29) --
	( 53.75, 28.29) --
	( 53.77, 28.30) --
	( 53.77, 28.30) --
	( 53.77, 28.30) --
	( 53.85, 28.34) --
	( 53.85, 28.34) --
	( 53.85, 28.34) --
	( 53.92, 28.37) --
	( 53.92, 28.37) --
	( 53.92, 28.37) --
	( 53.99, 28.41) --
	( 53.99, 28.41) --
	( 53.99, 28.41) --
	( 54.07, 28.44) --
	( 54.07, 28.44) --
	( 54.07, 28.44) --
	( 54.14, 28.48) --
	( 54.14, 28.48) --
	( 54.14, 28.48) --
	( 54.21, 28.51) --
	( 54.21, 28.51) --
	( 54.21, 28.51) --
	( 54.29, 28.55) --
	( 54.29, 28.55) --
	( 54.29, 28.55) --
	( 54.36, 28.58) --
	( 54.36, 28.58) --
	( 54.36, 28.58) --
	( 54.44, 28.62) --
	( 54.44, 28.62) --
	( 54.44, 28.62) --
	( 54.51, 28.65) --
	( 54.51, 28.65) --
	( 54.51, 28.65) --
	( 54.51, 28.66) --
	( 54.51, 28.66) --
	( 54.51, 28.66) --
	( 54.58, 28.69) --
	( 54.58, 28.69) --
	( 54.58, 28.69) --
	( 54.66, 28.74) --
	( 54.66, 28.74) --
	( 54.66, 28.74) --
	( 54.73, 28.78) --
	( 54.73, 28.78) --
	( 54.73, 28.78) --
	( 54.80, 28.82) --
	( 54.80, 28.82) --
	( 54.80, 28.82) --
	( 54.88, 28.86) --
	( 54.88, 28.86) --
	( 54.88, 28.86) --
	( 54.95, 28.90) --
	( 54.95, 28.90) --
	( 54.95, 28.90) --
	( 55.02, 28.94) --
	( 55.02, 28.94) --
	( 55.02, 28.94) --
	( 55.10, 28.98) --
	( 55.10, 28.98) --
	( 55.10, 28.98) --
	( 55.17, 29.02) --
	( 55.17, 29.02) --
	( 55.17, 29.02) --
	( 55.25, 29.06) --
	( 55.25, 29.06) --
	( 55.25, 29.06) --
	( 55.28, 29.08) --
	( 55.28, 29.08) --
	( 55.28, 29.08) --
	( 55.32, 29.14) --
	( 55.32, 29.14) --
	( 55.32, 29.14) --
	( 55.39, 29.26) --
	( 55.39, 29.26) --
	( 55.39, 29.26) --
	( 55.47, 29.37) --
	( 55.47, 29.37) --
	( 55.47, 29.37) --
	( 55.54, 29.49) --
	( 55.54, 29.49) --
	( 55.54, 29.49) --
	( 55.57, 29.53) --
	( 55.57, 29.53) --
	( 55.57, 29.53) --
	( 55.61, 29.56) --
	( 55.61, 29.56) --
	( 55.61, 29.56) --
	( 55.69, 29.62) --
	( 55.69, 29.62) --
	( 55.69, 29.62) --
	( 55.76, 29.67) --
	( 55.76, 29.67) --
	( 55.76, 29.67) --
	( 55.83, 29.73) --
	( 55.83, 29.73) --
	( 55.83, 29.73) --
	( 55.91, 29.78) --
	( 55.91, 29.78) --
	( 55.91, 29.78) --
	( 55.98, 29.84) --
	( 55.98, 29.84) --
	( 55.98, 29.84) --
	( 56.06, 29.89) --
	( 56.06, 29.89) --
	( 56.06, 29.89) --
	( 56.13, 29.95) --
	( 56.13, 29.95) --
	( 56.13, 29.95) --
	( 56.14, 29.96) --
	( 56.14, 29.96) --
	( 56.14, 29.96) --
	( 56.20, 30.03) --
	( 56.20, 30.03) --
	( 56.20, 30.03) --
	( 56.27, 30.13) --
	( 56.27, 30.13) --
	( 56.27, 30.13) --
	( 56.35, 30.22) --
	( 56.35, 30.22) --
	( 56.35, 30.22) --
	( 56.42, 30.31) --
	( 56.42, 30.31) --
	( 56.42, 30.31) --
	( 56.43, 30.32) --
	( 56.43, 30.32) --
	( 56.43, 30.32) --
	( 56.50, 30.39) --
	( 56.50, 30.39) --
	( 56.50, 30.39) --
	( 56.57, 30.45) --
	( 56.57, 30.45) --
	( 56.57, 30.45) --
	( 56.64, 30.52) --
	( 56.64, 30.52) --
	( 56.64, 30.52) --
	( 56.72, 30.59) --
	( 56.72, 30.59) --
	( 56.72, 30.59) --
	( 56.79, 30.66) --
	( 56.79, 30.66) --
	( 56.79, 30.66) --
	( 56.86, 30.73) --
	( 56.86, 30.73) --
	( 56.86, 30.73) --
	( 56.91, 30.77) --
	( 56.91, 30.77) --
	( 56.91, 30.77) --
	( 56.94, 30.81) --
	( 56.94, 30.81) --
	( 56.94, 30.81) --
	( 57.01, 30.93) --
	( 57.01, 30.93) --
	( 57.01, 30.93) --
	( 57.08, 31.04) --
	( 57.08, 31.04) --
	( 57.08, 31.04) --
	( 57.16, 31.16) --
	( 57.16, 31.16) --
	( 57.16, 31.16) --
	( 57.20, 31.22) --
	( 57.20, 31.22) --
	( 57.20, 31.22) --
	( 57.23, 31.24) --
	( 57.23, 31.24) --
	( 57.23, 31.24) --
	( 57.30, 31.30) --
	( 57.30, 31.30) --
	( 57.30, 31.30) --
	( 57.38, 31.36) --
	( 57.38, 31.36) --
	( 57.38, 31.36) --
	( 57.45, 31.42) --
	( 57.45, 31.42) --
	( 57.45, 31.42) --
	( 57.52, 31.47) --
	( 57.52, 31.47) --
	( 57.52, 31.47) --
	( 57.60, 31.53) --
	( 57.60, 31.53) --
	( 57.60, 31.53) --
	( 57.67, 31.59) --
	( 57.67, 31.59) --
	( 57.67, 31.59) --
	( 57.74, 31.64) --
	( 57.74, 31.64) --
	( 57.74, 31.64) --
	( 57.82, 31.70) --
	( 57.82, 31.70) --
	( 57.82, 31.70) --
	( 57.87, 31.74) --
	( 57.87, 31.74) --
	( 57.87, 31.74) --
	( 57.89, 31.76) --
	( 57.89, 31.76) --
	( 57.89, 31.76) --
	( 57.97, 31.82) --
	( 57.97, 31.82) --
	( 57.97, 31.82) --
	( 58.04, 31.89) --
	( 58.04, 31.89) --
	( 58.04, 31.89) --
	( 58.11, 31.95) --
	( 58.11, 31.95) --
	( 58.11, 31.95) --
	( 58.18, 32.01) --
	( 58.18, 32.01) --
	( 58.18, 32.01) --
	( 58.26, 32.07) --
	( 58.26, 32.07) --
	( 58.26, 32.07) --
	( 58.33, 32.14) --
	( 58.33, 32.14) --
	( 58.33, 32.14) --
	( 58.35, 32.15) --
	( 58.35, 32.15) --
	( 58.35, 32.15) --
	( 58.41, 32.20) --
	( 58.41, 32.20) --
	( 58.41, 32.20) --
	( 58.48, 32.26) --
	( 58.48, 32.26) --
	( 58.48, 32.26) --
	( 58.55, 32.32) --
	( 58.55, 32.32) --
	( 58.55, 32.32) --
	( 58.63, 32.37) --
	( 58.63, 32.37) --
	( 58.63, 32.37) --
	( 58.70, 32.43) --
	( 58.70, 32.43) --
	( 58.70, 32.43) --
	( 58.77, 32.49) --
	( 58.77, 32.49) --
	( 58.77, 32.49) --
	( 58.85, 32.55) --
	( 58.85, 32.55) --
	( 58.85, 32.55) --
	( 58.92, 32.61) --
	( 58.92, 32.61) --
	( 58.92, 32.61) --
	( 58.92, 32.61) --
	( 58.92, 32.61) --
	( 58.92, 32.61) --
	( 58.99, 32.66) --
	( 58.99, 32.66) --
	( 58.99, 32.66) --
	( 59.07, 32.72) --
	( 59.07, 32.72) --
	( 59.07, 32.72) --
	( 59.14, 32.77) --
	( 59.14, 32.77) --
	( 59.14, 32.77) --
	( 59.21, 32.82) --
	( 59.21, 32.82) --
	( 59.21, 32.82) --
	( 59.29, 32.87) --
	( 59.29, 32.87) --
	( 59.29, 32.87) --
	( 59.36, 32.92) --
	( 59.36, 32.92) --
	( 59.36, 32.92) --
	( 59.43, 32.98) --
	( 59.43, 32.98) --
	( 59.43, 32.98) --
	( 59.50, 33.02) --
	( 59.50, 33.02) --
	( 59.50, 33.02) --
	( 59.51, 33.03) --
	( 59.51, 33.03) --
	( 59.51, 33.03) --
	( 59.58, 33.07) --
	( 59.58, 33.07) --
	( 59.58, 33.07) --
	( 59.65, 33.12) --
	( 59.65, 33.12) --
	( 59.65, 33.12) --
	( 59.73, 33.16) --
	( 59.73, 33.16) --
	( 59.73, 33.16) --
	( 59.80, 33.21) --
	( 59.80, 33.21) --
	( 59.80, 33.21) --
	( 59.87, 33.25) --
	( 59.87, 33.25) --
	( 59.87, 33.25) --
	( 59.95, 33.29) --
	( 59.95, 33.29) --
	( 59.95, 33.29) --
	( 60.02, 33.34) --
	( 60.02, 33.34) --
	( 60.02, 33.34) --
	( 60.09, 33.38) --
	( 60.09, 33.38) --
	( 60.09, 33.38) --
	( 60.17, 33.43) --
	( 60.17, 33.43) --
	( 60.17, 33.43) --
	( 60.24, 33.47) --
	( 60.24, 33.47) --
	( 60.24, 33.47) --
	( 60.31, 33.52) --
	( 60.31, 33.52) --
	( 60.31, 33.52) --
	( 60.39, 33.56) --
	( 60.39, 33.56) --
	( 60.39, 33.56) --
	( 60.45, 33.60) --
	( 60.45, 33.60) --
	( 60.45, 33.60) --
	( 60.46, 33.60) --
	( 60.46, 33.60) --
	( 60.46, 33.60) --
	( 60.53, 33.64) --
	( 60.53, 33.64) --
	( 60.53, 33.64) --
	( 60.61, 33.67) --
	( 60.61, 33.67) --
	( 60.61, 33.67) --
	( 60.68, 33.70) --
	( 60.68, 33.70) --
	( 60.68, 33.70) --
	( 60.75, 33.73) --
	( 60.75, 33.73) --
	( 60.75, 33.73) --
	( 60.83, 33.76) --
	( 60.83, 33.76) --
	( 60.83, 33.76) --
	( 60.90, 33.79) --
	( 60.90, 33.79) --
	( 60.90, 33.79) --
	( 60.97, 33.82) --
	( 60.97, 33.82) --
	( 60.97, 33.82) --
	( 61.04, 33.85) --
	( 61.04, 33.85) --
	( 61.04, 33.85) --
	( 61.12, 33.89) --
	( 61.12, 33.89) --
	( 61.12, 33.89) --
	( 61.19, 33.92) --
	( 61.19, 33.92) --
	( 61.19, 33.92) --
	( 61.26, 33.95) --
	( 61.26, 33.95) --
	( 61.26, 33.95) --
	( 61.34, 33.98) --
	( 61.34, 33.98) --
	( 61.34, 33.98) --
	( 61.41, 34.01) --
	( 61.41, 34.01) --
	( 61.41, 34.01) --
	( 61.41, 34.01) --
	( 61.41, 34.01) --
	( 61.41, 34.01) --
	( 61.48, 34.03) --
	( 61.48, 34.03) --
	( 61.48, 34.03) --
	( 61.56, 34.04) --
	( 61.56, 34.04) --
	( 61.56, 34.04) --
	( 61.63, 34.06) --
	( 61.63, 34.06) --
	( 61.63, 34.06) --
	( 61.70, 34.08) --
	( 61.70, 34.08) --
	( 61.70, 34.08) --
	( 61.78, 34.10) --
	( 61.78, 34.10) --
	( 61.78, 34.10) --
	( 61.85, 34.11) --
	( 61.85, 34.11) --
	( 61.85, 34.11) --
	( 61.92, 34.13) --
	( 61.92, 34.13) --
	( 61.92, 34.13) --
	( 62.00, 34.15) --
	( 62.00, 34.15) --
	( 62.00, 34.15) --
	( 62.07, 34.16) --
	( 62.07, 34.16) --
	( 62.07, 34.16) --
	( 62.14, 34.18) --
	( 62.14, 34.18) --
	( 62.14, 34.18) --
	( 62.22, 34.20) --
	( 62.22, 34.20) --
	( 62.22, 34.20) --
	( 62.29, 34.22) --
	( 62.29, 34.22) --
	( 62.29, 34.22) --
	( 62.36, 34.23) --
	( 62.36, 34.23) --
	( 62.36, 34.23) --
	( 62.44, 34.25) --
	( 62.44, 34.25) --
	( 62.44, 34.25) --
	( 62.51, 34.27) --
	( 62.51, 34.27) --
	( 62.51, 34.27) --
	( 62.58, 34.28) --
	( 62.58, 34.28) --
	( 62.58, 34.28) --
	( 62.65, 34.30) --
	( 62.65, 34.30) --
	( 62.65, 34.30) --
	( 62.73, 34.32) --
	( 62.73, 34.32) --
	( 62.73, 34.32) --
	( 62.80, 34.34) --
	( 62.80, 34.34) --
	( 62.80, 34.34) --
	( 62.87, 34.35) --
	( 62.87, 34.35) --
	( 62.87, 34.35) --
	( 62.95, 34.37) --
	( 62.95, 34.37) --
	( 62.95, 34.37) --
	( 63.02, 34.39) --
	( 63.02, 34.39) --
	( 63.02, 34.39) --
	( 63.09, 34.40) --
	( 63.09, 34.40) --
	( 63.09, 34.40) --
	( 63.17, 34.42) --
	( 63.17, 34.42) --
	( 63.17, 34.42) --
	( 63.23, 34.44) --
	( 63.23, 34.44) --
	( 63.23, 34.44) --
	( 63.24, 34.44) --
	( 63.24, 34.44) --
	( 63.24, 34.44) --
	( 63.31, 34.44) --
	( 63.31, 34.44) --
	( 63.31, 34.44) --
	( 63.39, 34.45) --
	( 63.39, 34.45) --
	( 63.39, 34.45) --
	( 63.46, 34.46) --
	( 63.46, 34.46) --
	( 63.46, 34.46) --
	( 63.53, 34.47) --
	( 63.53, 34.47) --
	( 63.53, 34.47) --
	( 63.61, 34.47) --
	( 63.61, 34.47) --
	( 63.61, 34.47) --
	( 63.68, 34.48) --
	( 63.68, 34.48) --
	( 63.68, 34.48) --
	( 63.75, 34.49) --
	( 63.75, 34.49) --
	( 63.75, 34.49) --
	( 63.83, 34.50) --
	( 63.83, 34.50) --
	( 63.83, 34.50) --
	( 63.90, 34.50) --
	( 63.90, 34.50) --
	( 63.90, 34.50) --
	( 63.97, 34.51) --
	( 63.97, 34.51) --
	( 63.97, 34.51) --
	( 64.04, 34.52) --
	( 64.04, 34.52) --
	( 64.04, 34.52) --
	( 64.12, 34.53) --
	( 64.12, 34.53) --
	( 64.12, 34.53) --
	( 64.19, 34.53) --
	( 64.19, 34.53) --
	( 64.19, 34.53) --
	( 64.26, 34.54) --
	( 64.26, 34.54) --
	( 64.26, 34.54) --
	( 64.34, 34.55) --
	( 64.34, 34.55) --
	( 64.34, 34.55) --
	( 64.41, 34.56) --
	( 64.41, 34.56) --
	( 64.41, 34.56) --
	( 64.48, 34.56) --
	( 64.48, 34.56) --
	( 64.48, 34.56) --
	( 64.56, 34.57) --
	( 64.56, 34.57) --
	( 64.56, 34.57) --
	( 64.57, 34.57) --
	( 64.57, 34.57) --
	( 64.57, 34.57) --
	( 64.63, 34.57) --
	( 64.63, 34.57) --
	( 64.63, 34.57) --
	( 64.70, 34.56) --
	( 64.70, 34.56) --
	( 64.70, 34.56) --
	( 64.77, 34.56) --
	( 64.77, 34.56) --
	( 64.77, 34.56) --
	( 64.85, 34.56) --
	( 64.85, 34.56) --
	( 64.85, 34.56) --
	( 64.92, 34.55) --
	( 64.92, 34.55) --
	( 64.92, 34.55) --
	( 64.99, 34.55) --
	( 64.99, 34.55) --
	( 64.99, 34.55) --
	( 65.07, 34.54) --
	( 65.07, 34.54) --
	( 65.07, 34.54) --
	( 65.14, 34.54) --
	( 65.14, 34.54) --
	( 65.14, 34.54) --
	( 65.21, 34.53) --
	( 65.21, 34.53) --
	( 65.21, 34.53) --
	( 65.29, 34.53) --
	( 65.29, 34.53) --
	( 65.29, 34.53) --
	( 65.36, 34.52) --
	( 65.36, 34.52) --
	( 65.36, 34.52) --
	( 65.43, 34.52) --
	( 65.43, 34.52) --
	( 65.43, 34.52) --
	( 65.50, 34.51) --
	( 65.50, 34.51) --
	( 65.50, 34.51) --
	( 65.58, 34.51) --
	( 65.58, 34.51) --
	( 65.58, 34.51) --
	( 65.65, 34.50) --
	( 65.65, 34.50) --
	( 65.65, 34.50) --
	( 65.72, 34.50) --
	( 65.72, 34.50) --
	( 65.72, 34.50) --
	( 65.80, 34.49) --
	( 65.80, 34.49) --
	( 65.80, 34.49) --
	( 65.87, 34.49) --
	( 65.87, 34.49) --
	( 65.87, 34.49) --
	( 65.94, 34.49) --
	( 65.94, 34.49) --
	( 65.94, 34.49) --
	( 66.02, 34.48) --
	( 66.02, 34.48) --
	( 66.02, 34.48) --
	( 66.09, 34.48) --
	( 66.09, 34.48) --
	( 66.09, 34.48) --
	( 66.16, 34.47) --
	( 66.16, 34.47) --
	( 66.16, 34.47) --
	( 66.23, 34.47) --
	( 66.23, 34.47) --
	( 66.23, 34.47) --
	( 66.31, 34.46) --
	( 66.31, 34.46) --
	( 66.31, 34.46) --
	( 66.38, 34.46) --
	( 66.38, 34.46) --
	( 66.38, 34.46) --
	( 66.40, 34.46) --
	( 66.40, 34.46) --
	( 66.40, 34.46) --
	( 66.45, 34.45) --
	( 66.45, 34.45) --
	( 66.45, 34.45) --
	( 66.53, 34.44) --
	( 66.53, 34.44) --
	( 66.53, 34.44) --
	( 66.60, 34.43) --
	( 66.60, 34.43) --
	( 66.60, 34.43) --
	( 66.67, 34.42) --
	( 66.67, 34.42) --
	( 66.67, 34.42) --
	( 66.74, 34.40) --
	( 66.74, 34.40) --
	( 66.74, 34.40) --
	( 66.82, 34.39) --
	( 66.82, 34.39) --
	( 66.82, 34.39) --
	( 66.89, 34.38) --
	( 66.89, 34.38) --
	( 66.89, 34.38) --
	( 66.96, 34.37) --
	( 66.96, 34.37) --
	( 66.96, 34.37) --
	( 67.04, 34.36) --
	( 67.04, 34.36) --
	( 67.04, 34.36) --
	( 67.11, 34.35) --
	( 67.11, 34.35) --
	( 67.11, 34.35) --
	( 67.18, 34.34) --
	( 67.18, 34.34) --
	( 67.18, 34.34) --
	( 67.25, 34.33) --
	( 67.25, 34.33) --
	( 67.25, 34.33) --
	( 67.33, 34.32) --
	( 67.33, 34.32) --
	( 67.33, 34.32) --
	( 67.40, 34.31) --
	( 67.40, 34.31) --
	( 67.40, 34.31) --
	( 67.47, 34.30) --
	( 67.47, 34.30) --
	( 67.47, 34.30) --
	( 67.55, 34.28) --
	( 67.55, 34.28) --
	( 67.55, 34.28) --
	( 67.62, 34.27) --
	( 67.62, 34.27) --
	( 67.62, 34.27) --
	( 67.69, 34.26) --
	( 67.69, 34.26) --
	( 67.69, 34.26) --
	( 67.76, 34.25) --
	( 67.76, 34.25) --
	( 67.76, 34.25) --
	( 67.84, 34.24) --
	( 67.84, 34.24) --
	( 67.84, 34.24) --
	( 67.91, 34.23) --
	( 67.91, 34.23) --
	( 67.91, 34.23) --
	( 67.98, 34.22) --
	( 67.98, 34.22) --
	( 67.98, 34.22) --
	( 68.05, 34.21) --
	( 68.05, 34.21) --
	( 68.05, 34.21) --
	( 68.13, 34.20) --
	( 68.13, 34.20) --
	( 68.13, 34.20) --
	( 68.20, 34.19) --
	( 68.20, 34.19) --
	( 68.20, 34.19) --
	( 68.22, 34.18) --
	( 68.22, 34.18) --
	( 68.22, 34.18) --
	( 68.27, 34.17) --
	( 68.27, 34.17) --
	( 68.27, 34.17) --
	( 68.35, 34.14) --
	( 68.35, 34.14) --
	( 68.35, 34.14) --
	( 68.42, 34.12) --
	( 68.42, 34.12) --
	( 68.42, 34.12) --
	( 68.49, 34.10) --
	( 68.49, 34.10) --
	( 68.49, 34.10) --
	( 68.56, 34.08) --
	( 68.56, 34.08) --
	( 68.56, 34.08) --
	( 68.64, 34.05) --
	( 68.64, 34.05) --
	( 68.64, 34.05) --
	( 68.71, 34.03) --
	( 68.71, 34.03) --
	( 68.71, 34.03) --
	( 68.78, 34.01) --
	( 68.78, 34.01) --
	( 68.78, 34.01) --
	( 68.86, 33.98) --
	( 68.86, 33.98) --
	( 68.86, 33.98) --
	( 68.93, 33.96) --
	( 68.93, 33.96) --
	( 68.93, 33.96) --
	( 69.00, 33.94) --
	( 69.00, 33.94) --
	( 69.00, 33.94) --
	( 69.07, 33.92) --
	( 69.07, 33.92) --
	( 69.07, 33.92) --
	( 69.15, 33.89) --
	( 69.15, 33.89) --
	( 69.15, 33.89) --
	( 69.22, 33.87) --
	( 69.22, 33.87) --
	( 69.22, 33.87) --
	( 69.27, 33.85) --
	( 69.27, 33.85) --
	( 69.27, 33.85) --
	( 69.29, 33.85) --
	( 69.29, 33.85) --
	( 69.29, 33.85) --
	( 69.37, 33.83) --
	( 69.37, 33.83) --
	( 69.37, 33.83) --
	( 69.44, 33.81) --
	( 69.44, 33.81) --
	( 69.44, 33.81) --
	( 69.51, 33.79) --
	( 69.51, 33.79) --
	( 69.51, 33.79) --
	( 69.58, 33.77) --
	( 69.58, 33.77) --
	( 69.58, 33.77) --
	( 69.66, 33.74) --
	( 69.66, 33.74) --
	( 69.66, 33.74) --
	( 69.73, 33.72) --
	( 69.73, 33.72) --
	( 69.73, 33.72) --
	( 69.80, 33.70) --
	( 69.80, 33.70) --
	( 69.80, 33.70) --
	( 69.87, 33.68) --
	( 69.87, 33.68) --
	( 69.87, 33.68) --
	( 69.95, 33.66) --
	( 69.95, 33.66) --
	( 69.95, 33.66) --
	( 70.02, 33.64) --
	( 70.02, 33.64) --
	( 70.02, 33.64) --
	( 70.09, 33.62) --
	( 70.09, 33.62) --
	( 70.09, 33.62) --
	( 70.17, 33.60) --
	( 70.17, 33.60) --
	( 70.17, 33.60) --
	( 70.24, 33.58) --
	( 70.24, 33.58) --
	( 70.24, 33.58) --
	( 70.31, 33.56) --
	( 70.31, 33.56) --
	( 70.31, 33.56) --
	( 70.38, 33.54) --
	( 70.38, 33.54) --
	( 70.38, 33.54) --
	( 70.42, 33.53) --
	( 70.42, 33.53) --
	( 70.42, 33.53) --
	( 70.46, 33.51) --
	( 70.46, 33.51) --
	( 70.46, 33.51) --
	( 70.53, 33.47) --
	( 70.53, 33.47) --
	( 70.53, 33.47) --
	( 70.60, 33.43) --
	( 70.60, 33.43) --
	( 70.60, 33.43) --
	( 70.67, 33.40) --
	( 70.67, 33.40) --
	( 70.67, 33.40) --
	( 70.75, 33.36) --
	( 70.75, 33.36) --
	( 70.75, 33.36) --
	( 70.82, 33.32) --
	( 70.82, 33.32) --
	( 70.82, 33.32) --
	( 70.89, 33.29) --
	( 70.89, 33.29) --
	( 70.89, 33.29) --
	( 70.96, 33.25) --
	( 70.96, 33.25) --
	( 70.96, 33.25) --
	( 71.04, 33.21) --
	( 71.04, 33.21) --
	( 71.04, 33.21) --
	( 71.11, 33.18) --
	( 71.11, 33.18) --
	( 71.11, 33.18) --
	( 71.18, 33.14) --
	( 71.18, 33.14) --
	( 71.18, 33.14) --
	( 71.19, 33.14) --
	( 71.19, 33.14) --
	( 71.19, 33.14) --
	( 71.26, 33.11) --
	( 71.26, 33.11) --
	( 71.26, 33.11) --
	( 71.33, 33.08) --
	( 71.33, 33.08) --
	( 71.33, 33.08) --
	( 71.40, 33.06) --
	( 71.40, 33.06) --
	( 71.40, 33.06) --
	( 71.47, 33.03) --
	( 71.47, 33.03) --
	( 71.47, 33.03) --
	( 71.54, 33.00) --
	( 71.54, 33.00) --
	( 71.54, 33.00) --
	( 71.62, 32.97) --
	( 71.62, 32.97) --
	( 71.62, 32.97) --
	( 71.69, 32.95) --
	( 71.69, 32.95) --
	( 71.69, 32.95) --
	( 71.76, 32.92) --
	( 71.76, 32.92) --
	( 71.76, 32.92) --
	( 71.83, 32.89) --
	( 71.83, 32.89) --
	( 71.83, 32.89) --
	( 71.91, 32.86) --
	( 71.91, 32.86) --
	( 71.91, 32.86) --
	( 71.95, 32.85) --
	( 71.95, 32.85) --
	( 71.95, 32.85) --
	( 71.98, 32.83) --
	( 71.98, 32.83) --
	( 71.98, 32.83) --
	( 72.05, 32.80) --
	( 72.05, 32.80) --
	( 72.05, 32.80) --
	( 72.13, 32.77) --
	( 72.13, 32.77) --
	( 72.13, 32.77) --
	( 72.20, 32.74) --
	( 72.20, 32.74) --
	( 72.20, 32.74) --
	( 72.27, 32.70) --
	( 72.27, 32.70) --
	( 72.27, 32.70) --
	( 72.34, 32.67) --
	( 72.34, 32.67) --
	( 72.34, 32.67) --
	( 72.42, 32.64) --
	( 72.42, 32.64) --
	( 72.42, 32.64) --
	( 72.49, 32.60) --
	( 72.49, 32.60) --
	( 72.49, 32.60) --
	( 72.56, 32.57) --
	( 72.56, 32.57) --
	( 72.56, 32.57) --
	( 72.63, 32.54) --
	( 72.63, 32.54) --
	( 72.63, 32.54) --
	( 72.71, 32.51) --
	( 72.71, 32.51) --
	( 72.71, 32.51) --
	( 72.78, 32.47) --
	( 72.78, 32.47) --
	( 72.78, 32.47) --
	( 72.85, 32.44) --
	( 72.85, 32.44) --
	( 72.85, 32.44) --
	( 72.92, 32.41) --
	( 72.92, 32.41) --
	( 72.92, 32.41) --
	( 73.00, 32.38) --
	( 73.00, 32.38) --
	( 73.00, 32.38) --
	( 73.07, 32.34) --
	( 73.07, 32.34) --
	( 73.07, 32.34) --
	( 73.14, 32.31) --
	( 73.14, 32.31) --
	( 73.14, 32.31) --
	( 73.20, 32.28) --
	( 73.20, 32.28) --
	( 73.20, 32.28) --
	( 73.21, 32.28) --
	( 73.21, 32.28) --
	( 73.21, 32.28) --
	( 73.28, 32.24) --
	( 73.28, 32.24) --
	( 73.28, 32.24) --
	( 73.36, 32.20) --
	( 73.36, 32.20) --
	( 73.36, 32.20) --
	( 73.43, 32.16) --
	( 73.43, 32.16) --
	( 73.43, 32.16) --
	( 73.50, 32.12) --
	( 73.50, 32.12) --
	( 73.50, 32.12) --
	( 73.58, 32.08) --
	( 73.58, 32.08) --
	( 73.58, 32.08) --
	( 73.65, 32.05) --
	( 73.65, 32.05) --
	( 73.65, 32.05) --
	( 73.72, 32.01) --
	( 73.72, 32.01) --
	( 73.72, 32.01) --
	( 73.79, 31.97) --
	( 73.79, 31.97) --
	( 73.79, 31.97) --
	( 73.86, 31.93) --
	( 73.86, 31.93) --
	( 73.86, 31.93) --
	( 73.94, 31.89) --
	( 73.94, 31.89) --
	( 73.94, 31.89) --
	( 73.97, 31.88) --
	( 73.97, 31.88) --
	( 73.97, 31.88) --
	( 74.01, 31.86) --
	( 74.01, 31.86) --
	( 74.01, 31.86) --
	( 74.08, 31.82) --
	( 74.08, 31.82) --
	( 74.08, 31.82) --
	( 74.16, 31.79) --
	( 74.16, 31.79) --
	( 74.16, 31.79) --
	( 74.23, 31.76) --
	( 74.23, 31.76) --
	( 74.23, 31.76) --
	( 74.30, 31.73) --
	( 74.30, 31.73) --
	( 74.30, 31.73) --
	( 74.37, 31.70) --
	( 74.37, 31.70) --
	( 74.37, 31.70) --
	( 74.44, 31.66) --
	( 74.44, 31.66) --
	( 74.44, 31.66) --
	( 74.52, 31.63) --
	( 74.52, 31.63) --
	( 74.52, 31.63) --
	( 74.59, 31.60) --
	( 74.59, 31.60) --
	( 74.59, 31.60) --
	( 74.66, 31.57) --
	( 74.66, 31.57) --
	( 74.66, 31.57) --
	( 74.73, 31.53) --
	( 74.73, 31.53) --
	( 74.73, 31.53) --
	( 74.81, 31.50) --
	( 74.81, 31.50) --
	( 74.81, 31.50) --
	( 74.88, 31.47) --
	( 74.88, 31.47) --
	( 74.88, 31.47) --
	( 74.92, 31.45) --
	( 74.92, 31.45) --
	( 74.92, 31.45) --
	( 74.95, 31.44) --
	( 74.95, 31.44) --
	( 74.95, 31.44) --
	( 75.02, 31.42) --
	( 75.02, 31.42) --
	( 75.02, 31.42) --
	( 75.10, 31.39) --
	( 75.10, 31.39) --
	( 75.10, 31.39) --
	( 75.17, 31.37) --
	( 75.17, 31.37) --
	( 75.17, 31.37) --
	( 75.24, 31.34) --
	( 75.24, 31.34) --
	( 75.24, 31.34) --
	( 75.31, 31.32) --
	( 75.31, 31.32) --
	( 75.31, 31.32) --
	( 75.39, 31.29) --
	( 75.39, 31.29) --
	( 75.39, 31.29) --
	( 75.46, 31.27) --
	( 75.46, 31.27) --
	( 75.46, 31.27) --
	( 75.53, 31.25) --
	( 75.53, 31.25) --
	( 75.53, 31.25) --
	( 75.60, 31.22) --
	( 75.60, 31.22) --
	( 75.60, 31.22) --
	( 75.67, 31.20) --
	( 75.67, 31.20) --
	( 75.67, 31.20) --
	( 75.75, 31.17) --
	( 75.75, 31.17) --
	( 75.75, 31.17) --
	( 75.79, 31.16) --
	( 75.79, 31.16) --
	( 75.79, 31.16) --
	( 75.82, 31.14) --
	( 75.82, 31.14) --
	( 75.82, 31.14) --
	( 75.89, 31.11) --
	( 75.89, 31.11) --
	( 75.89, 31.11) --
	( 75.96, 31.07) --
	( 75.96, 31.07) --
	( 75.96, 31.07) --
	( 76.04, 31.03) --
	( 76.04, 31.03) --
	( 76.04, 31.03) --
	( 76.11, 31.00) --
	( 76.11, 31.00) --
	( 76.11, 31.00) --
	( 76.18, 30.96) --
	( 76.18, 30.96) --
	( 76.18, 30.96) --
	( 76.25, 30.92) --
	( 76.25, 30.92) --
	( 76.25, 30.92) --
	( 76.33, 30.89) --
	( 76.33, 30.89) --
	( 76.33, 30.89) --
	( 76.40, 30.85) --
	( 76.40, 30.85) --
	( 76.40, 30.85) --
	( 76.47, 30.81) --
	( 76.47, 30.81) --
	( 76.47, 30.81) --
	( 76.54, 30.78) --
	( 76.54, 30.78) --
	( 76.54, 30.78) --
	( 76.61, 30.74) --
	( 76.61, 30.74) --
	( 76.61, 30.74) --
	( 76.69, 30.70) --
	( 76.69, 30.70) --
	( 76.69, 30.70) --
	( 76.75, 30.67) --
	( 76.75, 30.67) --
	( 76.75, 30.67) --
	( 76.76, 30.67) --
	( 76.76, 30.67) --
	( 76.76, 30.67) --
	( 76.83, 30.64) --
	( 76.83, 30.64) --
	( 76.83, 30.64) --
	( 76.90, 30.61) --
	( 76.90, 30.61) --
	( 76.90, 30.61) --
	( 76.98, 30.58) --
	( 76.98, 30.58) --
	( 76.98, 30.58) --
	( 77.05, 30.56) --
	( 77.05, 30.56) --
	( 77.05, 30.56) --
	( 77.12, 30.53) --
	( 77.12, 30.53) --
	( 77.12, 30.53) --
	( 77.19, 30.50) --
	( 77.19, 30.50) --
	( 77.19, 30.50) --
	( 77.27, 30.47) --
	( 77.27, 30.47) --
	( 77.27, 30.47) --
	( 77.34, 30.45) --
	( 77.34, 30.45) --
	( 77.34, 30.45) --
	( 77.41, 30.42) --
	( 77.41, 30.42) --
	( 77.41, 30.42) --
	( 77.48, 30.39) --
	( 77.48, 30.39) --
	( 77.48, 30.39) --
	( 77.55, 30.36) --
	( 77.55, 30.36) --
	( 77.55, 30.36) --
	( 77.63, 30.34) --
	( 77.63, 30.34) --
	( 77.63, 30.34) --
	( 77.70, 30.31) --
	( 77.70, 30.31) --
	( 77.70, 30.31) --
	( 77.70, 30.30) --
	( 77.70, 30.30) --
	( 77.70, 30.30) --
	( 77.77, 30.28) --
	( 77.77, 30.28) --
	( 77.77, 30.28) --
	( 77.84, 30.24) --
	( 77.84, 30.24) --
	( 77.84, 30.24) --
	( 77.91, 30.21) --
	( 77.91, 30.21) --
	( 77.91, 30.21) --
	( 77.99, 30.18) --
	( 77.99, 30.18) --
	( 77.99, 30.18) --
	( 78.06, 30.15) --
	( 78.06, 30.15) --
	( 78.06, 30.15) --
	( 78.13, 30.12) --
	( 78.13, 30.12) --
	( 78.13, 30.12) --
	( 78.20, 30.08) --
	( 78.20, 30.08) --
	( 78.20, 30.08) --
	( 78.28, 30.05) --
	( 78.28, 30.05) --
	( 78.28, 30.05) --
	( 78.35, 30.02) --
	( 78.35, 30.02) --
	( 78.35, 30.02) --
	( 78.42, 29.99) --
	( 78.42, 29.99) --
	( 78.42, 29.99) --
	( 78.49, 29.96) --
	( 78.49, 29.96) --
	( 78.49, 29.96) --
	( 78.56, 29.93) --
	( 78.56, 29.93) --
	( 78.56, 29.93) --
	( 78.64, 29.89) --
	( 78.64, 29.89) --
	( 78.64, 29.89) --
	( 78.71, 29.86) --
	( 78.71, 29.86) --
	( 78.71, 29.86) --
	( 78.76, 29.84) --
	( 78.76, 29.84) --
	( 78.76, 29.84) --
	( 78.78, 29.83) --
	( 78.78, 29.83) --
	( 78.78, 29.83) --
	( 78.85, 29.79) --
	( 78.85, 29.79) --
	( 78.85, 29.79) --
	( 78.93, 29.76) --
	( 78.93, 29.76) --
	( 78.93, 29.76) --
	( 79.00, 29.73) --
	( 79.00, 29.73) --
	( 79.00, 29.73) --
	( 79.07, 29.69) --
	( 79.07, 29.69) --
	( 79.07, 29.69) --
	( 79.14, 29.66) --
	( 79.14, 29.66) --
	( 79.14, 29.66) --
	( 79.21, 29.62) --
	( 79.21, 29.62) --
	( 79.21, 29.62) --
	( 79.29, 29.59) --
	( 79.29, 29.59) --
	( 79.29, 29.59) --
	( 79.36, 29.56) --
	( 79.36, 29.56) --
	( 79.36, 29.56) --
	( 79.43, 29.52) --
	( 79.43, 29.52) --
	( 79.43, 29.52) --
	( 79.50, 29.49) --
	( 79.50, 29.49) --
	( 79.50, 29.49) --
	( 79.57, 29.45) --
	( 79.57, 29.45) --
	( 79.57, 29.45) --
	( 79.62, 29.43) --
	( 79.62, 29.43) --
	( 79.62, 29.43) --
	( 79.65, 29.42) --
	( 79.65, 29.42) --
	( 79.65, 29.42) --
	( 79.72, 29.40) --
	( 79.72, 29.40) --
	( 79.72, 29.40) --
	( 79.79, 29.37) --
	( 79.79, 29.37) --
	( 79.79, 29.37) --
	( 79.86, 29.35) --
	( 79.86, 29.35) --
	( 79.86, 29.35) --
	( 79.93, 29.32) --
	( 79.93, 29.32) --
	( 79.93, 29.32) --
	( 80.01, 29.29) --
	( 80.01, 29.29) --
	( 80.01, 29.29) --
	( 80.08, 29.27) --
	( 80.08, 29.27) --
	( 80.08, 29.27) --
	( 80.15, 29.24) --
	( 80.15, 29.24) --
	( 80.15, 29.24) --
	( 80.22, 29.22) --
	( 80.22, 29.22) --
	( 80.22, 29.22) --
	( 80.29, 29.19) --
	( 80.29, 29.19) --
	( 80.29, 29.19) --
	( 80.37, 29.16) --
	( 80.37, 29.16) --
	( 80.37, 29.16) --
	( 80.44, 29.14) --
	( 80.44, 29.14) --
	( 80.44, 29.14) --
	( 80.48, 29.12) --
	( 80.48, 29.12) --
	( 80.48, 29.12) --
	( 80.51, 29.11) --
	( 80.51, 29.11) --
	( 80.51, 29.11) --
	( 80.58, 29.08) --
	( 80.58, 29.08) --
	( 80.58, 29.08) --
	( 80.65, 29.05) --
	( 80.65, 29.05) --
	( 80.65, 29.05) --
	( 80.73, 29.02) --
	( 80.73, 29.02) --
	( 80.73, 29.02) --
	( 80.80, 28.99) --
	( 80.80, 28.99) --
	( 80.80, 28.99) --
	( 80.87, 28.96) --
	( 80.87, 28.96) --
	( 80.87, 28.96) --
	( 80.94, 28.93) --
	( 80.94, 28.93) --
	( 80.94, 28.93) --
	( 81.01, 28.90) --
	( 81.01, 28.90) --
	( 81.01, 28.90) --
	( 81.09, 28.87) --
	( 81.09, 28.87) --
	( 81.09, 28.87) --
	( 81.16, 28.83) --
	( 81.16, 28.83) --
	( 81.16, 28.83) --
	( 81.23, 28.80) --
	( 81.23, 28.80) --
	( 81.23, 28.80) --
	( 81.30, 28.77) --
	( 81.30, 28.77) --
	( 81.30, 28.77) --
	( 81.37, 28.74) --
	( 81.37, 28.74) --
	( 81.37, 28.74) --
	( 81.44, 28.71) --
	( 81.44, 28.71) --
	( 81.44, 28.71) --
	( 81.45, 28.71) --
	( 81.45, 28.71) --
	( 81.45, 28.71) --
	( 81.52, 28.69) --
	( 81.52, 28.69) --
	( 81.52, 28.69) --
	( 81.59, 28.67) --
	( 81.59, 28.67) --
	( 81.59, 28.67) --
	( 81.66, 28.65) --
	( 81.66, 28.65) --
	( 81.66, 28.65) --
	( 81.73, 28.63) --
	( 81.73, 28.63) --
	( 81.73, 28.63) --
	( 81.81, 28.60) --
	( 81.81, 28.60) --
	( 81.81, 28.60) --
	( 81.88, 28.58) --
	( 81.88, 28.58) --
	( 81.88, 28.58) --
	( 81.95, 28.56) --
	( 81.95, 28.56) --
	( 81.95, 28.56) --
	( 82.02, 28.54) --
	( 82.02, 28.54) --
	( 82.02, 28.54) --
	( 82.09, 28.52) --
	( 82.09, 28.52) --
	( 82.09, 28.52) --
	( 82.16, 28.49) --
	( 82.16, 28.49) --
	( 82.16, 28.49) --
	( 82.24, 28.47) --
	( 82.24, 28.47) --
	( 82.24, 28.47) --
	( 82.31, 28.45) --
	( 82.31, 28.45) --
	( 82.31, 28.45) --
	( 82.38, 28.43) --
	( 82.38, 28.43) --
	( 82.38, 28.43) --
	( 82.45, 28.41) --
	( 82.45, 28.41) --
	( 82.45, 28.41) --
	( 82.52, 28.39) --
	( 82.52, 28.39) --
	( 82.52, 28.39) --
	( 82.59, 28.36) --
	( 82.59, 28.36) --
	( 82.59, 28.36) --
	( 82.60, 28.36) --
	( 82.60, 28.36) --
	( 82.60, 28.36) --
	( 82.67, 28.35) --
	( 82.67, 28.35) --
	( 82.67, 28.35) --
	( 82.74, 28.33) --
	( 82.74, 28.33) --
	( 82.74, 28.33) --
	( 82.81, 28.31) --
	( 82.81, 28.31) --
	( 82.81, 28.31) --
	( 82.88, 28.30) --
	( 82.88, 28.30) --
	( 82.88, 28.30) --
	( 82.96, 28.28) --
	( 82.96, 28.28) --
	( 82.96, 28.28) --
	( 83.03, 28.27) --
	( 83.03, 28.27) --
	( 83.03, 28.27) --
	( 83.10, 28.25) --
	( 83.10, 28.25) --
	( 83.10, 28.25) --
	( 83.17, 28.23) --
	( 83.17, 28.23) --
	( 83.17, 28.23) --
	( 83.24, 28.22) --
	( 83.24, 28.22) --
	( 83.24, 28.22) --
	( 83.31, 28.20) --
	( 83.31, 28.20) --
	( 83.31, 28.20) --
	( 83.39, 28.18) --
	( 83.39, 28.18) --
	( 83.39, 28.18) --
	( 83.46, 28.17) --
	( 83.46, 28.17) --
	( 83.46, 28.17) --
	( 83.53, 28.15) --
	( 83.53, 28.15) --
	( 83.53, 28.15) --
	( 83.60, 28.13) --
	( 83.60, 28.13) --
	( 83.60, 28.13) --
	( 83.67, 28.12) --
	( 83.67, 28.12) --
	( 83.67, 28.12) --
	( 83.75, 28.10) --
	( 83.75, 28.10) --
	( 83.75, 28.10) --
	( 83.82, 28.09) --
	( 83.82, 28.09) --
	( 83.82, 28.09) --
	( 83.89, 28.07) --
	( 83.89, 28.07) --
	( 83.89, 28.07) --
	( 83.96, 28.05) --
	( 83.96, 28.05) --
	( 83.96, 28.05) --
	( 84.03, 28.04) --
	( 84.03, 28.04) --
	( 84.03, 28.04) --
	( 84.10, 28.02) --
	( 84.10, 28.02) --
	( 84.10, 28.02) --
	( 84.12, 28.02) --
	( 84.12, 28.02) --
	( 84.12, 28.02) --
	( 84.18, 28.00) --
	( 84.18, 28.00) --
	( 84.18, 28.00) --
	( 84.25, 27.99) --
	( 84.25, 27.99) --
	( 84.25, 27.99) --
	( 84.32, 27.97) --
	( 84.32, 27.97) --
	( 84.32, 27.97) --
	( 84.39, 27.95) --
	( 84.39, 27.95) --
	( 84.39, 27.95) --
	( 84.46, 27.93) --
	( 84.46, 27.93) --
	( 84.46, 27.93) --
	( 84.54, 27.91) --
	( 84.54, 27.91) --
	( 84.54, 27.91) --
	( 84.61, 27.90) --
	( 84.61, 27.90) --
	( 84.61, 27.90) --
	( 84.68, 27.88) --
	( 84.68, 27.88) --
	( 84.68, 27.88) --
	( 84.75, 27.86) --
	( 84.75, 27.86) --
	( 84.75, 27.86) --
	( 84.82, 27.84) --
	( 84.82, 27.84) --
	( 84.82, 27.84) --
	( 84.89, 27.83) --
	( 84.89, 27.83) --
	( 84.89, 27.83) --
	( 84.96, 27.81) --
	( 84.96, 27.81) --
	( 84.96, 27.81) --
	( 85.04, 27.79) --
	( 85.04, 27.79) --
	( 85.04, 27.79) --
	( 85.11, 27.77) --
	( 85.11, 27.77) --
	( 85.11, 27.77) --
	( 85.18, 27.76) --
	( 85.18, 27.76) --
	( 85.18, 27.76) --
	( 85.25, 27.74) --
	( 85.25, 27.74) --
	( 85.25, 27.74) --
	( 85.32, 27.72) --
	( 85.32, 27.72) --
	( 85.32, 27.72) --
	( 85.40, 27.70) --
	( 85.40, 27.70) --
	( 85.40, 27.70) --
	( 85.47, 27.69) --
	( 85.47, 27.69) --
	( 85.47, 27.69) --
	( 85.47, 27.69) --
	( 85.47, 27.69) --
	( 85.47, 27.69) --
	( 85.54, 27.67) --
	( 85.54, 27.67) --
	( 85.54, 27.67) --
	( 85.61, 27.65) --
	( 85.61, 27.65) --
	( 85.61, 27.65) --
	( 85.68, 27.63) --
	( 85.68, 27.63) --
	( 85.68, 27.63) --
	( 85.75, 27.61) --
	( 85.75, 27.61) --
	( 85.75, 27.61) --
	( 85.83, 27.59) --
	( 85.83, 27.59) --
	( 85.83, 27.59) --
	( 85.90, 27.57) --
	( 85.90, 27.57) --
	( 85.90, 27.57) --
	( 85.97, 27.55) --
	( 85.97, 27.55) --
	( 85.97, 27.55) --
	( 86.04, 27.53) --
	( 86.04, 27.53) --
	( 86.04, 27.53) --
	( 86.11, 27.51) --
	( 86.11, 27.51) --
	( 86.11, 27.51) --
	( 86.18, 27.49) --
	( 86.18, 27.49) --
	( 86.18, 27.49) --
	( 86.25, 27.47) --
	( 86.25, 27.47) --
	( 86.25, 27.47) --
	( 86.33, 27.45) --
	( 86.33, 27.45) --
	( 86.33, 27.45) --
	( 86.40, 27.43) --
	( 86.40, 27.43) --
	( 86.40, 27.43) --
	( 86.47, 27.42) --
	( 86.47, 27.42) --
	( 86.47, 27.42) --
	( 86.54, 27.40) --
	( 86.54, 27.40) --
	( 86.54, 27.40) --
	( 86.61, 27.38) --
	( 86.61, 27.38) --
	( 86.61, 27.38) --
	( 86.68, 27.36) --
	( 86.68, 27.36) --
	( 86.68, 27.36) --
	( 86.76, 27.34) --
	( 86.76, 27.34) --
	( 86.76, 27.34) --
	( 86.83, 27.32) --
	( 86.83, 27.32) --
	( 86.83, 27.32) --
	( 86.90, 27.30) --
	( 86.90, 27.30) --
	( 86.90, 27.30) --
	( 86.90, 27.30) --
	( 86.90, 27.30) --
	( 86.90, 27.30) --
	( 86.97, 27.29) --
	( 86.97, 27.29) --
	( 86.97, 27.29) --
	( 87.04, 27.27) --
	( 87.04, 27.27) --
	( 87.04, 27.27) --
	( 87.11, 27.26) --
	( 87.11, 27.26) --
	( 87.11, 27.26) --
	( 87.19, 27.25) --
	( 87.19, 27.25) --
	( 87.19, 27.25) --
	( 87.26, 27.24) --
	( 87.26, 27.24) --
	( 87.26, 27.24) --
	( 87.33, 27.22) --
	( 87.33, 27.22) --
	( 87.33, 27.22) --
	( 87.40, 27.21) --
	( 87.40, 27.21) --
	( 87.40, 27.21) --
	( 87.47, 27.20) --
	( 87.47, 27.20) --
	( 87.47, 27.20) --
	( 87.54, 27.19) --
	( 87.54, 27.19) --
	( 87.54, 27.19) --
	( 87.62, 27.17) --
	( 87.62, 27.17) --
	( 87.62, 27.17) --
	( 87.69, 27.16) --
	( 87.69, 27.16) --
	( 87.69, 27.16) --
	( 87.76, 27.15) --
	( 87.76, 27.15) --
	( 87.76, 27.15) --
	( 87.83, 27.14) --
	( 87.83, 27.14) --
	( 87.83, 27.14) --
	( 87.90, 27.12) --
	( 87.90, 27.12) --
	( 87.90, 27.12) --
	( 87.97, 27.11) --
	( 87.97, 27.11) --
	( 87.97, 27.11) --
	( 88.04, 27.10) --
	( 88.04, 27.10) --
	( 88.04, 27.10) --
	( 88.12, 27.09) --
	( 88.12, 27.09) --
	( 88.12, 27.09) --
	( 88.19, 27.08) --
	( 88.19, 27.08) --
	( 88.19, 27.08) --
	( 88.26, 27.06) --
	( 88.26, 27.06) --
	( 88.26, 27.06) --
	( 88.33, 27.05) --
	( 88.33, 27.05) --
	( 88.33, 27.05) --
	( 88.40, 27.04) --
	( 88.40, 27.04) --
	( 88.40, 27.04) --
	( 88.47, 27.03) --
	( 88.47, 27.03) --
	( 88.47, 27.03) --
	( 88.54, 27.01) --
	( 88.54, 27.01) --
	( 88.54, 27.01) --
	( 88.62, 27.00) --
	( 88.62, 27.00) --
	( 88.62, 27.00) --
	( 88.69, 26.99) --
	( 88.69, 26.99) --
	( 88.69, 26.99) --
	( 88.76, 26.98) --
	( 88.76, 26.98) --
	( 88.76, 26.98) --
	( 88.83, 26.96) --
	( 88.83, 26.96) --
	( 88.83, 26.96) --
	( 88.90, 26.95) --
	( 88.90, 26.95) --
	( 88.90, 26.95) --
	( 88.92, 26.95) --
	( 88.92, 26.95) --
	( 88.92, 26.95) --
	( 88.97, 26.94) --
	( 88.97, 26.94) --
	( 88.97, 26.94) --
	( 89.05, 26.93) --
	( 89.05, 26.93) --
	( 89.05, 26.93) --
	( 89.12, 26.93) --
	( 89.12, 26.93) --
	( 89.12, 26.93) --
	( 89.19, 26.92) --
	( 89.19, 26.92) --
	( 89.19, 26.92) --
	( 89.26, 26.91) --
	( 89.26, 26.91) --
	( 89.26, 26.91) --
	( 89.33, 26.90) --
	( 89.33, 26.90) --
	( 89.33, 26.90) --
	( 89.40, 26.89) --
	( 89.40, 26.89) --
	( 89.40, 26.89) --
	( 89.47, 26.89) --
	( 89.47, 26.89) --
	( 89.47, 26.89) --
	( 89.55, 26.88) --
	( 89.55, 26.88) --
	( 89.55, 26.88) --
	( 89.62, 26.87) --
	( 89.62, 26.87) --
	( 89.62, 26.87) --
	( 89.69, 26.86) --
	( 89.69, 26.86) --
	( 89.69, 26.86) --
	( 89.76, 26.85) --
	( 89.76, 26.85) --
	( 89.76, 26.85) --
	( 89.83, 26.85) --
	( 89.83, 26.85) --
	( 89.83, 26.85) --
	( 89.90, 26.84) --
	( 89.90, 26.84) --
	( 89.90, 26.84) --
	( 89.97, 26.83) --
	( 89.97, 26.83) --
	( 89.97, 26.83) --
	( 90.04, 26.82) --
	( 90.04, 26.82) --
	( 90.04, 26.82) --
	( 90.12, 26.81) --
	( 90.12, 26.81) --
	( 90.12, 26.81) --
	( 90.19, 26.81) --
	( 90.19, 26.81) --
	( 90.19, 26.81) --
	( 90.26, 26.80) --
	( 90.26, 26.80) --
	( 90.26, 26.80) --
	( 90.33, 26.79) --
	( 90.33, 26.79) --
	( 90.33, 26.79) --
	( 90.40, 26.78) --
	( 90.40, 26.78) --
	( 90.40, 26.78) --
	( 90.47, 26.77) --
	( 90.47, 26.77) --
	( 90.47, 26.77) --
	( 90.54, 26.77) --
	( 90.54, 26.77) --
	( 90.54, 26.77) --
	( 90.62, 26.76) --
	( 90.62, 26.76) --
	( 90.62, 26.76) --
	( 90.64, 26.75) --
	( 90.64, 26.75) --
	( 90.64, 26.75) --
	( 90.69, 26.75) --
	( 90.69, 26.75) --
	( 90.69, 26.75) --
	( 90.76, 26.74) --
	( 90.76, 26.74) --
	( 90.76, 26.74) --
	( 90.83, 26.74) --
	( 90.83, 26.74) --
	( 90.83, 26.74) --
	( 90.90, 26.73) --
	( 90.90, 26.73) --
	( 90.90, 26.73) --
	( 90.97, 26.72) --
	( 90.97, 26.72) --
	( 90.97, 26.72) --
	( 91.04, 26.72) --
	( 91.04, 26.72) --
	( 91.04, 26.72) --
	( 91.12, 26.71) --
	( 91.12, 26.71) --
	( 91.12, 26.71) --
	( 91.19, 26.70) --
	( 91.19, 26.70) --
	( 91.19, 26.70) --
	( 91.26, 26.70) --
	( 91.26, 26.70) --
	( 91.26, 26.70) --
	( 91.33, 26.69) --
	( 91.33, 26.69) --
	( 91.33, 26.69) --
	( 91.40, 26.68) --
	( 91.40, 26.68) --
	( 91.40, 26.68) --
	( 91.47, 26.67) --
	( 91.47, 26.67) --
	( 91.47, 26.67) --
	( 91.54, 26.67) --
	( 91.54, 26.67) --
	( 91.54, 26.67) --
	( 91.61, 26.66) --
	( 91.61, 26.66) --
	( 91.61, 26.66) --
	( 91.69, 26.65) --
	( 91.69, 26.65) --
	( 91.69, 26.65) --
	( 91.76, 26.65) --
	( 91.76, 26.65) --
	( 91.76, 26.65) --
	( 91.83, 26.64) --
	( 91.83, 26.64) --
	( 91.83, 26.64) --
	( 91.90, 26.63) --
	( 91.90, 26.63) --
	( 91.90, 26.63) --
	( 91.97, 26.63) --
	( 91.97, 26.63) --
	( 91.97, 26.63) --
	( 92.04, 26.62) --
	( 92.04, 26.62) --
	( 92.04, 26.62) --
	( 92.11, 26.61) --
	( 92.11, 26.61) --
	( 92.11, 26.61) --
	( 92.18, 26.61) --
	( 92.18, 26.61) --
	( 92.18, 26.61) --
	( 92.26, 26.60) --
	( 92.26, 26.60) --
	( 92.26, 26.60) --
	( 92.33, 26.59) --
	( 92.33, 26.59) --
	( 92.33, 26.59) --
	( 92.40, 26.59) --
	( 92.40, 26.59) --
	( 92.40, 26.59) --
	( 92.47, 26.58) --
	( 92.47, 26.58) --
	( 92.47, 26.58) --
	( 92.54, 26.57) --
	( 92.54, 26.57) --
	( 92.54, 26.57) --
	( 92.61, 26.56) --
	( 92.61, 26.56) --
	( 92.61, 26.56) --
	( 92.65, 26.56) --
	( 92.65, 26.56) --
	( 92.65, 26.56) --
	( 92.68, 26.56) --
	( 92.68, 26.56) --
	( 92.68, 26.56) --
	( 92.75, 26.55) --
	( 92.75, 26.55) --
	( 92.75, 26.55) --
	( 92.83, 26.55) --
	( 92.83, 26.55) --
	( 92.83, 26.55) --
	( 92.90, 26.55) --
	( 92.90, 26.55) --
	( 92.90, 26.55) --
	( 92.97, 26.54) --
	( 92.97, 26.54) --
	( 92.97, 26.54) --
	( 93.04, 26.54) --
	( 93.04, 26.54) --
	( 93.04, 26.54) --
	( 93.11, 26.53) --
	( 93.11, 26.53) --
	( 93.11, 26.53) --
	( 93.18, 26.53) --
	( 93.18, 26.53) --
	( 93.18, 26.53) --
	( 93.25, 26.53) --
	( 93.25, 26.53) --
	( 93.25, 26.53) --
	( 93.32, 26.52) --
	( 93.32, 26.52) --
	( 93.32, 26.52) --
	( 93.39, 26.52) --
	( 93.39, 26.52) --
	( 93.39, 26.52) --
	( 93.46, 26.51) --
	( 93.46, 26.51) --
	( 93.46, 26.51) --
	( 93.54, 26.51) --
	( 93.54, 26.51) --
	( 93.54, 26.51) --
	( 93.61, 26.50) --
	( 93.61, 26.50) --
	( 93.61, 26.50) --
	( 93.68, 26.50) --
	( 93.68, 26.50) --
	( 93.68, 26.50) --
	( 93.75, 26.50) --
	( 93.75, 26.50) --
	( 93.75, 26.50) --
	( 93.82, 26.49) --
	( 93.82, 26.49) --
	( 93.82, 26.49) --
	( 93.89, 26.49) --
	( 93.89, 26.49) --
	( 93.89, 26.49) --
	( 93.96, 26.48) --
	( 93.96, 26.48) --
	( 93.96, 26.48) --
	( 94.03, 26.48) --
	( 94.03, 26.48) --
	( 94.03, 26.48) --
	( 94.10, 26.48) --
	( 94.10, 26.48) --
	( 94.10, 26.48) --
	( 94.18, 26.47) --
	( 94.18, 26.47) --
	( 94.18, 26.47) --
	( 94.25, 26.47) --
	( 94.25, 26.47) --
	( 94.25, 26.47) --
	( 94.32, 26.46) --
	( 94.32, 26.46) --
	( 94.32, 26.46) --
	( 94.39, 26.46) --
	( 94.39, 26.46) --
	( 94.39, 26.46) --
	( 94.46, 26.45) --
	( 94.46, 26.45) --
	( 94.46, 26.45) --
	( 94.53, 26.45) --
	( 94.53, 26.45) --
	( 94.53, 26.45) --
	( 94.60, 26.45) --
	( 94.60, 26.45) --
	( 94.60, 26.45) --
	( 94.67, 26.44) --
	( 94.67, 26.44) --
	( 94.67, 26.44) --
	( 94.75, 26.44) --
	( 94.75, 26.44) --
	( 94.75, 26.44) --
	( 94.82, 26.43) --
	( 94.82, 26.43) --
	( 94.82, 26.43) --
	( 94.89, 26.43) --
	( 94.89, 26.43) --
	( 94.89, 26.43) --
	( 94.95, 26.42) --
	( 94.95, 26.42) --
	( 94.95, 26.42) --
	( 94.96, 26.42) --
	( 94.96, 26.42) --
	( 94.96, 26.42) --
	( 95.03, 26.42) --
	( 95.03, 26.42) --
	( 95.03, 26.42) --
	( 95.10, 26.41) --
	( 95.10, 26.41) --
	( 95.10, 26.41) --
	( 95.17, 26.41) --
	( 95.17, 26.41) --
	( 95.17, 26.41) --
	( 95.24, 26.40) --
	( 95.24, 26.40) --
	( 95.24, 26.40) --
	( 95.31, 26.39) --
	( 95.31, 26.39) --
	( 95.31, 26.39) --
	( 95.38, 26.39) --
	( 95.38, 26.39) --
	( 95.38, 26.39) --
	( 95.45, 26.38) --
	( 95.45, 26.38) --
	( 95.45, 26.38) --
	( 95.53, 26.38) --
	( 95.53, 26.38) --
	( 95.53, 26.38) --
	( 95.60, 26.37) --
	( 95.60, 26.37) --
	( 95.60, 26.37) --
	( 95.67, 26.36) --
	( 95.67, 26.36) --
	( 95.67, 26.36) --
	( 95.74, 26.36) --
	( 95.74, 26.36) --
	( 95.74, 26.36) --
	( 95.81, 26.35) --
	( 95.81, 26.35) --
	( 95.81, 26.35) --
	( 95.88, 26.35) --
	( 95.88, 26.35) --
	( 95.88, 26.35) --
	( 95.95, 26.34) --
	( 95.95, 26.34) --
	( 95.95, 26.34) --
	( 96.02, 26.33) --
	( 96.02, 26.33) --
	( 96.02, 26.33) --
	( 96.09, 26.33) --
	( 96.09, 26.33) --
	( 96.09, 26.33) --
	( 96.16, 26.32) --
	( 96.16, 26.32) --
	( 96.16, 26.32) --
	( 96.24, 26.32) --
	( 96.24, 26.32) --
	( 96.24, 26.32) --
	( 96.31, 26.31) --
	( 96.31, 26.31) --
	( 96.31, 26.31) --
	( 96.38, 26.30) --
	( 96.38, 26.30) --
	( 96.38, 26.30) --
	( 96.45, 26.30) --
	( 96.45, 26.30) --
	( 96.45, 26.30) --
	( 96.52, 26.29) --
	( 96.52, 26.29) --
	( 96.52, 26.29) --
	( 96.59, 26.29) --
	( 96.59, 26.29) --
	( 96.59, 26.29) --
	( 96.66, 26.28) --
	( 96.66, 26.28) --
	( 96.66, 26.28) --
	( 96.73, 26.27) --
	( 96.73, 26.27) --
	( 96.73, 26.27) --
	( 96.77, 26.27) --
	( 96.77, 26.27) --
	( 96.77, 26.27) --
	( 96.80, 26.27) --
	( 96.80, 26.27) --
	( 96.80, 26.27) --
	( 96.87, 26.27) --
	( 96.87, 26.27) --
	( 96.87, 26.27) --
	( 96.94, 26.27) --
	( 96.94, 26.27) --
	( 96.94, 26.27) --
	( 97.02, 26.26) --
	( 97.02, 26.26) --
	( 97.02, 26.26) --
	( 97.09, 26.26) --
	( 97.09, 26.26) --
	( 97.09, 26.26) --
	( 97.16, 26.26) --
	( 97.16, 26.26) --
	( 97.16, 26.26) --
	( 97.23, 26.26) --
	( 97.23, 26.26) --
	( 97.23, 26.26) --
	( 97.30, 26.26) --
	( 97.30, 26.26) --
	( 97.30, 26.26) --
	( 97.37, 26.25) --
	( 97.37, 26.25) --
	( 97.37, 26.25) --
	( 97.44, 26.25) --
	( 97.44, 26.25) --
	( 97.44, 26.25) --
	( 97.51, 26.25) --
	( 97.51, 26.25) --
	( 97.51, 26.25) --
	( 97.58, 26.25) --
	( 97.58, 26.25) --
	( 97.58, 26.25) --
	( 97.65, 26.25) --
	( 97.65, 26.25) --
	( 97.65, 26.25) --
	( 97.72, 26.25) --
	( 97.72, 26.25) --
	( 97.72, 26.25) --
	( 97.80, 26.24) --
	( 97.80, 26.24) --
	( 97.80, 26.24) --
	( 97.87, 26.24) --
	( 97.87, 26.24) --
	( 97.87, 26.24) --
	( 97.94, 26.24) --
	( 97.94, 26.24) --
	( 97.94, 26.24) --
	( 98.01, 26.24) --
	( 98.01, 26.24) --
	( 98.01, 26.24) --
	( 98.08, 26.24) --
	( 98.08, 26.24) --
	( 98.08, 26.24) --
	( 98.15, 26.24) --
	( 98.15, 26.24) --
	( 98.15, 26.24) --
	( 98.22, 26.23) --
	( 98.22, 26.23) --
	( 98.22, 26.23) --
	( 98.29, 26.23) --
	( 98.29, 26.23) --
	( 98.29, 26.23) --
	( 98.36, 26.23) --
	( 98.36, 26.23) --
	( 98.36, 26.23) --
	( 98.43, 26.23) --
	( 98.43, 26.23) --
	( 98.43, 26.23) --
	( 98.50, 26.23) --
	( 98.50, 26.23) --
	( 98.50, 26.23) --
	( 98.57, 26.22) --
	( 98.57, 26.22) --
	( 98.57, 26.22) --
	( 98.65, 26.22) --
	( 98.65, 26.22) --
	( 98.65, 26.22) --
	( 98.72, 26.22) --
	( 98.72, 26.22) --
	( 98.72, 26.22) --
	( 98.79, 26.22) --
	( 98.79, 26.22) --
	( 98.79, 26.22) --
	( 98.86, 26.22) --
	( 98.86, 26.22) --
	( 98.86, 26.22) --
	( 98.93, 26.22) --
	( 98.93, 26.22) --
	( 98.93, 26.22) --
	( 99.00, 26.21) --
	( 99.00, 26.21) --
	( 99.00, 26.21) --
	( 99.07, 26.21) --
	( 99.07, 26.21) --
	( 99.07, 26.21) --
	( 99.07, 26.21) --
	( 99.07, 26.21) --
	( 99.07, 26.21) --
	( 99.14, 26.21) --
	( 99.14, 26.21) --
	( 99.14, 26.21) --
	( 99.21, 26.21) --
	( 99.21, 26.21) --
	( 99.21, 26.21) --
	( 99.28, 26.20) --
	( 99.28, 26.20) --
	( 99.28, 26.20) --
	( 99.35, 26.20) --
	( 99.35, 26.20) --
	( 99.35, 26.20) --
	( 99.42, 26.20) --
	( 99.42, 26.20) --
	( 99.42, 26.20) --
	( 99.49, 26.20) --
	( 99.49, 26.20) --
	( 99.49, 26.20) --
	( 99.56, 26.19) --
	( 99.56, 26.19) --
	( 99.56, 26.19) --
	( 99.64, 26.19) --
	( 99.64, 26.19) --
	( 99.64, 26.19) --
	( 99.71, 26.19) --
	( 99.71, 26.19) --
	( 99.71, 26.19) --
	( 99.78, 26.19) --
	( 99.78, 26.19) --
	( 99.78, 26.19) --
	( 99.85, 26.18) --
	( 99.85, 26.18) --
	( 99.85, 26.18) --
	( 99.92, 26.18) --
	( 99.92, 26.18) --
	( 99.92, 26.18) --
	( 99.99, 26.18) --
	( 99.99, 26.18) --
	( 99.99, 26.18) --
	(100.06, 26.17) --
	(100.06, 26.17) --
	(100.06, 26.17) --
	(100.13, 26.17) --
	(100.13, 26.17) --
	(100.13, 26.17) --
	(100.20, 26.17) --
	(100.20, 26.17) --
	(100.20, 26.17) --
	(100.27, 26.17) --
	(100.27, 26.17) --
	(100.27, 26.17) --
	(100.34, 26.16) --
	(100.34, 26.16) --
	(100.34, 26.16) --
	(100.41, 26.16) --
	(100.41, 26.16) --
	(100.41, 26.16) --
	(100.48, 26.16) --
	(100.48, 26.16) --
	(100.48, 26.16) --
	(100.55, 26.16) --
	(100.55, 26.16) --
	(100.55, 26.16) --
	(100.62, 26.15) --
	(100.62, 26.15) --
	(100.62, 26.15) --
	(100.70, 26.15) --
	(100.70, 26.15) --
	(100.70, 26.15) --
	(100.77, 26.15) --
	(100.77, 26.15) --
	(100.77, 26.15) --
	(100.84, 26.15) --
	(100.84, 26.15) --
	(100.84, 26.15) --
	(100.91, 26.14) --
	(100.91, 26.14) --
	(100.91, 26.14) --
	(100.98, 26.14) --
	(100.98, 26.14) --
	(100.98, 26.14) --
	(101.05, 26.14) --
	(101.05, 26.14) --
	(101.05, 26.14) --
	(101.12, 26.13) --
	(101.12, 26.13) --
	(101.12, 26.13) --
	(101.19, 26.13) --
	(101.19, 26.13) --
	(101.19, 26.13) --
	(101.26, 26.13) --
	(101.26, 26.13) --
	(101.26, 26.13) --
	(101.33, 26.13) --
	(101.33, 26.13) --
	(101.33, 26.13) --
	(101.40, 26.12) --
	(101.40, 26.12) --
	(101.40, 26.12) --
	(101.47, 26.12) --
	(101.47, 26.12) --
	(101.47, 26.12) --
	(101.54, 26.12) --
	(101.54, 26.12) --
	(101.54, 26.12) --
	(101.61, 26.12) --
	(101.61, 26.12) --
	(101.61, 26.12) --
	(101.66, 26.11) --
	(101.66, 26.11) --
	(101.66, 26.11) --
	(101.68, 26.11) --
	(101.68, 26.11) --
	(101.68, 26.11) --
	(101.75, 26.11) --
	(101.75, 26.11) --
	(101.75, 26.11) --
	(101.82, 26.11) --
	(101.82, 26.11) --
	(101.82, 26.11) --
	(101.90, 26.11) --
	(101.90, 26.11) --
	(101.90, 26.11) --
	(101.97, 26.11) --
	(101.97, 26.11) --
	(101.97, 26.11) --
	(102.04, 26.11) --
	(102.04, 26.11) --
	(102.04, 26.11) --
	(102.11, 26.11) --
	(102.11, 26.11) --
	(102.11, 26.11) --
	(102.18, 26.11) --
	(102.18, 26.11) --
	(102.18, 26.11) --
	(102.25, 26.11) --
	(102.25, 26.11) --
	(102.25, 26.11) --
	(102.32, 26.11) --
	(102.32, 26.11) --
	(102.32, 26.11) --
	(102.39, 26.10) --
	(102.39, 26.10) --
	(102.39, 26.10) --
	(102.46, 26.10) --
	(102.46, 26.10) --
	(102.46, 26.10) --
	(102.53, 26.10) --
	(102.53, 26.10) --
	(102.53, 26.10) --
	(102.60, 26.10) --
	(102.60, 26.10) --
	(102.60, 26.10) --
	(102.67, 26.10) --
	(102.67, 26.10) --
	(102.67, 26.10) --
	(102.74, 26.10) --
	(102.74, 26.10) --
	(102.74, 26.10) --
	(102.81, 26.10) --
	(102.81, 26.10) --
	(102.81, 26.10) --
	(102.88, 26.10) --
	(102.88, 26.10) --
	(102.88, 26.10) --
	(102.95, 26.10) --
	(102.95, 26.10) --
	(102.95, 26.10) --
	(103.02, 26.10) --
	(103.02, 26.10) --
	(103.02, 26.10) --
	(103.09, 26.09) --
	(103.09, 26.09) --
	(103.09, 26.09) --
	(103.16, 26.09) --
	(103.16, 26.09) --
	(103.16, 26.09) --
	(103.23, 26.09) --
	(103.23, 26.09) --
	(103.23, 26.09) --
	(103.31, 26.09) --
	(103.31, 26.09) --
	(103.31, 26.09) --
	(103.37, 26.09) --
	(103.37, 26.09) --
	(103.37, 26.09) --
	(103.45, 26.09) --
	(103.45, 26.09) --
	(103.45, 26.09) --
	(103.52, 26.09) --
	(103.52, 26.09) --
	(103.52, 26.09) --
	(103.59, 26.09) --
	(103.59, 26.09) --
	(103.59, 26.09) --
	(103.66, 26.09) --
	(103.66, 26.09) --
	(103.66, 26.09) --
	(103.73, 26.09) --
	(103.73, 26.09) --
	(103.73, 26.09) --
	(103.80, 26.08) --
	(103.80, 26.08) --
	(103.80, 26.08) --
	(103.87, 26.08) --
	(103.87, 26.08) --
	(103.87, 26.08) --
	(103.94, 26.08) --
	(103.94, 26.08) --
	(103.94, 26.08) --
	(104.01, 26.08) --
	(104.01, 26.08) --
	(104.01, 26.08) --
	(104.08, 26.08) --
	(104.08, 26.08) --
	(104.08, 26.08) --
	(104.15, 26.08) --
	(104.15, 26.08) --
	(104.15, 26.08) --
	(104.22, 26.08) --
	(104.22, 26.08) --
	(104.22, 26.08) --
	(104.29, 26.08) --
	(104.29, 26.08) --
	(104.29, 26.08) --
	(104.36, 26.08) --
	(104.36, 26.08) --
	(104.36, 26.08) --
	(104.43, 26.08) --
	(104.43, 26.08) --
	(104.43, 26.08) --
	(104.44, 26.08) --
	(104.44, 26.08) --
	(104.44, 26.08) --
	(104.50, 26.08) --
	(104.50, 26.08) --
	(104.50, 26.08) --
	(104.57, 26.07) --
	(104.57, 26.07) --
	(104.57, 26.07) --
	(104.64, 26.07) --
	(104.64, 26.07) --
	(104.64, 26.07) --
	(104.71, 26.07) --
	(104.71, 26.07) --
	(104.71, 26.07) --
	(104.78, 26.07) --
	(104.78, 26.07) --
	(104.78, 26.07) --
	(104.85, 26.07) --
	(104.85, 26.07) --
	(104.85, 26.07) --
	(104.92, 26.07) --
	(104.92, 26.07) --
	(104.92, 26.07) --
	(104.99, 26.07) --
	(104.99, 26.07) --
	(104.99, 26.07) --
	(105.06, 26.07) --
	(105.06, 26.07) --
	(105.06, 26.07) --
	(105.14, 26.07) --
	(105.14, 26.07) --
	(105.14, 26.07) --
	(105.20, 26.07) --
	(105.20, 26.07) --
	(105.20, 26.07) --
	(105.28, 26.07) --
	(105.28, 26.07) --
	(105.28, 26.07) --
	(105.34, 26.07) --
	(105.34, 26.07) --
	(105.34, 26.07) --
	(105.42, 26.07) --
	(105.42, 26.07) --
	(105.42, 26.07) --
	(105.49, 26.07) --
	(105.49, 26.07) --
	(105.49, 26.07) --
	(105.56, 26.07) --
	(105.56, 26.07) --
	(105.56, 26.07) --
	(105.63, 26.07) --
	(105.63, 26.07) --
	(105.63, 26.07) --
	(105.70, 26.07) --
	(105.70, 26.07) --
	(105.70, 26.07) --
	(105.77, 26.07) --
	(105.77, 26.07) --
	(105.77, 26.07) --
	(105.84, 26.07) --
	(105.84, 26.07) --
	(105.84, 26.07) --
	(105.91, 26.07) --
	(105.91, 26.07) --
	(105.91, 26.07) --
	(105.98, 26.07) --
	(105.98, 26.07) --
	(105.98, 26.07) --
	(106.05, 26.07) --
	(106.05, 26.07) --
	(106.05, 26.07) --
	(106.12, 26.06) --
	(106.12, 26.06) --
	(106.12, 26.06) --
	(106.19, 26.06) --
	(106.19, 26.06) --
	(106.19, 26.06) --
	(106.26, 26.06) --
	(106.26, 26.06) --
	(106.26, 26.06) --
	(106.33, 26.06) --
	(106.33, 26.06) --
	(106.33, 26.06) --
	(106.40, 26.06) --
	(106.40, 26.06) --
	(106.40, 26.06) --
	(106.47, 26.06) --
	(106.47, 26.06) --
	(106.47, 26.06) --
	(106.54, 26.06) --
	(106.54, 26.06) --
	(106.54, 26.06) --
	(106.61, 26.06) --
	(106.61, 26.06) --
	(106.61, 26.06) --
	(106.68, 26.06) --
	(106.68, 26.06) --
	(106.68, 26.06) --
	(106.75, 26.06) --
	(106.75, 26.06) --
	(106.75, 26.06) --
	(106.82, 26.06) --
	(106.82, 26.06) --
	(106.82, 26.06) --
	(106.89, 26.06) --
	(106.89, 26.06) --
	(106.89, 26.06) --
	(106.96, 26.06) --
	(106.96, 26.06) --
	(106.96, 26.06) --
	(107.03, 26.06) --
	(107.03, 26.06) --
	(107.03, 26.06) --
	(107.10, 26.06) --
	(107.10, 26.06) --
	(107.10, 26.06) --
	(107.17, 26.06) --
	(107.17, 26.06) --
	(107.17, 26.06) --
	(107.24, 26.06) --
	(107.24, 26.06) --
	(107.24, 26.06) --
	(107.31, 26.06) --
	(107.31, 26.06) --
	(107.31, 26.06) --
	(107.38, 26.06) --
	(107.38, 26.06) --
	(107.38, 26.06) --
	(107.41, 26.06) --
	(107.41, 26.06) --
	(107.41, 26.06) --
	(107.45, 26.06) --
	(107.45, 26.06) --
	(107.45, 26.06) --
	(107.52, 26.06) --
	(107.52, 26.06) --
	(107.52, 26.06) --
	(107.59, 26.06) --
	(107.59, 26.06) --
	(107.59, 26.06) --
	(107.66, 26.06) --
	(107.66, 26.06) --
	(107.66, 26.06) --
	(107.73, 26.06) --
	(107.73, 26.06) --
	(107.73, 26.06) --
	(107.80, 26.06) --
	(107.80, 26.06) --
	(107.80, 26.06) --
	(107.87, 26.06) --
	(107.87, 26.06) --
	(107.87, 26.06) --
	(107.94, 26.06) --
	(107.94, 26.06) --
	(107.94, 26.06) --
	(108.01, 26.06) --
	(108.01, 26.06) --
	(108.01, 26.06) --
	(108.08, 26.06) --
	(108.08, 26.06) --
	(108.08, 26.06) --
	(108.15, 26.06) --
	(108.15, 26.06) --
	(108.15, 26.06) --
	(108.22, 26.06) --
	(108.22, 26.06) --
	(108.22, 26.06) --
	(108.29, 26.06) --
	(108.29, 26.06) --
	(108.29, 26.06) --
	(108.36, 26.06) --
	(108.36, 26.06) --
	(108.36, 26.06) --
	(108.43, 26.06) --
	(108.43, 26.06) --
	(108.43, 26.06) --
	(108.50, 26.06) --
	(108.50, 26.06) --
	(108.50, 26.06) --
	(108.57, 26.06) --
	(108.57, 26.06) --
	(108.57, 26.06) --
	(108.64, 26.06) --
	(108.64, 26.06) --
	(108.64, 26.06) --
	(108.71, 26.06) --
	(108.71, 26.06) --
	(108.71, 26.06) --
	(108.78, 26.06) --
	(108.78, 26.06) --
	(108.78, 26.06) --
	(108.85, 26.06) --
	(108.85, 26.06) --
	(108.85, 26.06) --
	(108.92, 26.06) --
	(108.92, 26.06) --
	(108.92, 26.06) --
	(108.99, 26.06) --
	(108.99, 26.06) --
	(108.99, 26.06) --
	(109.06, 26.06) --
	(109.06, 26.06) --
	(109.06, 26.06) --
	(109.13, 26.06) --
	(109.13, 26.06) --
	(109.13, 26.06) --
	(109.20, 26.06) --
	(109.20, 26.06) --
	(109.20, 26.06) --
	(109.27, 26.06) --
	(109.27, 26.06) --
	(109.27, 26.06) --
	(109.34, 26.06) --
	(109.34, 26.06) --
	(109.34, 26.06) --
	(109.41, 26.06) --
	(109.41, 26.06) --
	(109.41, 26.06) --
	(109.48, 26.06) --
	(109.48, 26.06) --
	(109.48, 26.06) --
	(109.55, 26.06) --
	(109.55, 26.06) --
	(109.55, 26.06) --
	(109.62, 26.06) --
	(109.62, 26.06) --
	(109.62, 26.06) --
	(109.69, 26.06) --
	(109.69, 26.06) --
	(109.69, 26.06) --
	(109.76, 26.06) --
	(109.76, 26.06) --
	(109.76, 26.06) --
	(109.83, 26.06) --
	(109.83, 26.06) --
	(109.83, 26.06) --
	(109.90, 26.06) --
	(109.90, 26.06) --
	(109.90, 26.06) --
	(109.90, 26.06) --
	(109.90, 26.06) --
	(109.90, 26.06) --
	(109.97, 26.06) --
	(109.97, 26.06) --
	(109.97, 26.06) --
	(110.04, 26.06) --
	(110.04, 26.06) --
	(110.04, 26.06) --
	(110.11, 26.06) --
	(110.11, 26.06) --
	(110.11, 26.06) --
	(110.18, 26.06) --
	(110.18, 26.06) --
	(110.18, 26.06) --
	(110.25, 26.06) --
	(110.25, 26.06) --
	(110.25, 26.06) --
	(110.32, 26.06) --
	(110.32, 26.06) --
	(110.32, 26.06) --
	(110.39, 26.06) --
	(110.39, 26.06) --
	(110.39, 26.06) --
	(110.46, 26.06) --
	(110.46, 26.06) --
	(110.46, 26.06) --
	(110.53, 26.06) --
	(110.53, 26.06) --
	(110.53, 26.06) --
	(110.60, 26.06) --
	(110.60, 26.06) --
	(110.60, 26.06) --
	(110.67, 26.06) --
	(110.67, 26.06) --
	(110.67, 26.06) --
	(110.74, 26.07) --
	(110.74, 26.07) --
	(110.74, 26.07) --
	(110.81, 26.07) --
	(110.81, 26.07) --
	(110.81, 26.07) --
	(110.88, 26.07) --
	(110.88, 26.07) --
	(110.88, 26.07) --
	(110.95, 26.07) --
	(110.95, 26.07) --
	(110.95, 26.07) --
	(111.02, 26.07) --
	(111.02, 26.07) --
	(111.02, 26.07) --
	(111.09, 26.07) --
	(111.09, 26.07) --
	(111.09, 26.07) --
	(111.16, 26.07) --
	(111.16, 26.07) --
	(111.16, 26.07) --
	(111.23, 26.07) --
	(111.23, 26.07) --
	(111.23, 26.07) --
	(111.30, 26.07) --
	(111.30, 26.07) --
	(111.30, 26.07) --
	(111.37, 26.07) --
	(111.37, 26.07) --
	(111.37, 26.07) --
	(111.44, 26.07) --
	(111.44, 26.07) --
	(111.44, 26.07) --
	(111.51, 26.07) --
	(111.51, 26.07) --
	(111.51, 26.07) --
	(111.58, 26.07) --
	(111.58, 26.07) --
	(111.58, 26.07) --
	(111.65, 26.08) --
	(111.65, 26.08) --
	(111.65, 26.08) --
	(111.72, 26.08) --
	(111.72, 26.08) --
	(111.72, 26.08) --
	(111.72, 26.08) --
	(111.72, 26.08) --
	(111.72, 25.90) --
	(111.72, 25.90) --
	(111.72, 25.90) --
	(111.72, 25.90) --
	(111.72, 25.90) --
	(111.72, 25.90) --
	(111.65, 25.90) --
	(111.65, 25.90) --
	(111.65, 25.90) --
	(111.58, 25.90) --
	(111.58, 25.90) --
	(111.58, 25.90) --
	(111.51, 25.90) --
	(111.51, 25.90) --
	(111.51, 25.90) --
	(111.44, 25.90) --
	(111.44, 25.90) --
	(111.44, 25.90) --
	(111.37, 25.90) --
	(111.37, 25.90) --
	(111.37, 25.90) --
	(111.30, 25.90) --
	(111.30, 25.90) --
	(111.30, 25.90) --
	(111.23, 25.90) --
	(111.23, 25.90) --
	(111.23, 25.90) --
	(111.16, 25.90) --
	(111.16, 25.90) --
	(111.16, 25.90) --
	(111.09, 25.90) --
	(111.09, 25.90) --
	(111.09, 25.90) --
	(111.02, 25.90) --
	(111.02, 25.90) --
	(111.02, 25.90) --
	(110.95, 25.90) --
	(110.95, 25.90) --
	(110.95, 25.90) --
	(110.88, 25.90) --
	(110.88, 25.90) --
	(110.88, 25.90) --
	(110.81, 25.90) --
	(110.81, 25.90) --
	(110.81, 25.90) --
	(110.74, 25.90) --
	(110.74, 25.90) --
	(110.74, 25.90) --
	(110.67, 25.90) --
	(110.67, 25.90) --
	(110.67, 25.90) --
	(110.60, 25.90) --
	(110.60, 25.90) --
	(110.60, 25.90) --
	(110.53, 25.90) --
	(110.53, 25.90) --
	(110.53, 25.90) --
	(110.46, 25.90) --
	(110.46, 25.90) --
	(110.46, 25.90) --
	(110.39, 25.90) --
	(110.39, 25.90) --
	(110.39, 25.90) --
	(110.32, 25.90) --
	(110.32, 25.90) --
	(110.32, 25.90) --
	(110.25, 25.90) --
	(110.25, 25.90) --
	(110.25, 25.90) --
	(110.18, 25.90) --
	(110.18, 25.90) --
	(110.18, 25.90) --
	(110.11, 25.90) --
	(110.11, 25.90) --
	(110.11, 25.90) --
	(110.04, 25.90) --
	(110.04, 25.90) --
	(110.04, 25.90) --
	(109.97, 25.90) --
	(109.97, 25.90) --
	(109.97, 25.90) --
	(109.90, 25.90) --
	(109.90, 25.90) --
	(109.90, 25.90) --
	(109.90, 25.90) --
	(109.90, 25.90) --
	(109.90, 25.90) --
	(109.83, 25.90) --
	(109.83, 25.90) --
	(109.83, 25.90) --
	(109.76, 25.90) --
	(109.76, 25.90) --
	(109.76, 25.90) --
	(109.69, 25.90) --
	(109.69, 25.90) --
	(109.69, 25.90) --
	(109.62, 25.90) --
	(109.62, 25.90) --
	(109.62, 25.90) --
	(109.55, 25.90) --
	(109.55, 25.90) --
	(109.55, 25.90) --
	(109.48, 25.90) --
	(109.48, 25.90) --
	(109.48, 25.90) --
	(109.41, 25.90) --
	(109.41, 25.90) --
	(109.41, 25.90) --
	(109.34, 25.90) --
	(109.34, 25.90) --
	(109.34, 25.90) --
	(109.27, 25.90) --
	(109.27, 25.90) --
	(109.27, 25.90) --
	(109.20, 25.90) --
	(109.20, 25.90) --
	(109.20, 25.90) --
	(109.13, 25.90) --
	(109.13, 25.90) --
	(109.13, 25.90) --
	(109.06, 25.90) --
	(109.06, 25.90) --
	(109.06, 25.90) --
	(108.99, 25.90) --
	(108.99, 25.90) --
	(108.99, 25.90) --
	(108.92, 25.90) --
	(108.92, 25.90) --
	(108.92, 25.90) --
	(108.85, 25.90) --
	(108.85, 25.90) --
	(108.85, 25.90) --
	(108.78, 25.90) --
	(108.78, 25.90) --
	(108.78, 25.90) --
	(108.71, 25.90) --
	(108.71, 25.90) --
	(108.71, 25.90) --
	(108.64, 25.90) --
	(108.64, 25.90) --
	(108.64, 25.90) --
	(108.57, 25.90) --
	(108.57, 25.90) --
	(108.57, 25.90) --
	(108.50, 25.90) --
	(108.50, 25.90) --
	(108.50, 25.90) --
	(108.43, 25.90) --
	(108.43, 25.90) --
	(108.43, 25.90) --
	(108.36, 25.90) --
	(108.36, 25.90) --
	(108.36, 25.90) --
	(108.29, 25.90) --
	(108.29, 25.90) --
	(108.29, 25.90) --
	(108.22, 25.90) --
	(108.22, 25.90) --
	(108.22, 25.90) --
	(108.15, 25.90) --
	(108.15, 25.90) --
	(108.15, 25.90) --
	(108.08, 25.90) --
	(108.08, 25.90) --
	(108.08, 25.90) --
	(108.01, 25.90) --
	(108.01, 25.90) --
	(108.01, 25.90) --
	(107.94, 25.90) --
	(107.94, 25.90) --
	(107.94, 25.90) --
	(107.87, 25.90) --
	(107.87, 25.90) --
	(107.87, 25.90) --
	(107.80, 25.90) --
	(107.80, 25.90) --
	(107.80, 25.90) --
	(107.73, 25.90) --
	(107.73, 25.90) --
	(107.73, 25.90) --
	(107.66, 25.90) --
	(107.66, 25.90) --
	(107.66, 25.90) --
	(107.59, 25.90) --
	(107.59, 25.90) --
	(107.59, 25.90) --
	(107.52, 25.90) --
	(107.52, 25.90) --
	(107.52, 25.90) --
	(107.45, 25.90) --
	(107.45, 25.90) --
	(107.45, 25.90) --
	(107.41, 25.90) --
	(107.41, 25.90) --
	(107.41, 25.90) --
	(107.38, 25.90) --
	(107.38, 25.90) --
	(107.38, 25.90) --
	(107.31, 25.90) --
	(107.31, 25.90) --
	(107.31, 25.90) --
	(107.24, 25.90) --
	(107.24, 25.90) --
	(107.24, 25.90) --
	(107.17, 25.90) --
	(107.17, 25.90) --
	(107.17, 25.90) --
	(107.10, 25.90) --
	(107.10, 25.90) --
	(107.10, 25.90) --
	(107.03, 25.90) --
	(107.03, 25.90) --
	(107.03, 25.90) --
	(106.96, 25.90) --
	(106.96, 25.90) --
	(106.96, 25.90) --
	(106.89, 25.90) --
	(106.89, 25.90) --
	(106.89, 25.90) --
	(106.82, 25.90) --
	(106.82, 25.90) --
	(106.82, 25.90) --
	(106.75, 25.90) --
	(106.75, 25.90) --
	(106.75, 25.90) --
	(106.68, 25.90) --
	(106.68, 25.90) --
	(106.68, 25.90) --
	(106.61, 25.90) --
	(106.61, 25.90) --
	(106.61, 25.90) --
	(106.54, 25.90) --
	(106.54, 25.90) --
	(106.54, 25.90) --
	(106.47, 25.90) --
	(106.47, 25.90) --
	(106.47, 25.90) --
	(106.40, 25.90) --
	(106.40, 25.90) --
	(106.40, 25.90) --
	(106.33, 25.90) --
	(106.33, 25.90) --
	(106.33, 25.90) --
	(106.26, 25.90) --
	(106.26, 25.90) --
	(106.26, 25.90) --
	(106.19, 25.90) --
	(106.19, 25.90) --
	(106.19, 25.90) --
	(106.12, 25.90) --
	(106.12, 25.90) --
	(106.12, 25.90) --
	(106.05, 25.90) --
	(106.05, 25.90) --
	(106.05, 25.90) --
	(105.98, 25.90) --
	(105.98, 25.90) --
	(105.98, 25.90) --
	(105.91, 25.90) --
	(105.91, 25.90) --
	(105.91, 25.90) --
	(105.84, 25.90) --
	(105.84, 25.90) --
	(105.84, 25.90) --
	(105.77, 25.90) --
	(105.77, 25.90) --
	(105.77, 25.90) --
	(105.70, 25.90) --
	(105.70, 25.90) --
	(105.70, 25.90) --
	(105.63, 25.90) --
	(105.63, 25.90) --
	(105.63, 25.90) --
	(105.56, 25.90) --
	(105.56, 25.90) --
	(105.56, 25.90) --
	(105.49, 25.90) --
	(105.49, 25.90) --
	(105.49, 25.90) --
	(105.42, 25.90) --
	(105.42, 25.90) --
	(105.42, 25.90) --
	(105.34, 25.90) --
	(105.34, 25.90) --
	(105.34, 25.90) --
	(105.28, 25.90) --
	(105.28, 25.90) --
	(105.28, 25.90) --
	(105.20, 25.90) --
	(105.20, 25.90) --
	(105.20, 25.90) --
	(105.14, 25.90) --
	(105.14, 25.90) --
	(105.14, 25.90) --
	(105.06, 25.90) --
	(105.06, 25.90) --
	(105.06, 25.90) --
	(104.99, 25.90) --
	(104.99, 25.90) --
	(104.99, 25.90) --
	(104.92, 25.90) --
	(104.92, 25.90) --
	(104.92, 25.90) --
	(104.85, 25.90) --
	(104.85, 25.90) --
	(104.85, 25.90) --
	(104.78, 25.90) --
	(104.78, 25.90) --
	(104.78, 25.90) --
	(104.71, 25.90) --
	(104.71, 25.90) --
	(104.71, 25.90) --
	(104.64, 25.90) --
	(104.64, 25.90) --
	(104.64, 25.90) --
	(104.57, 25.90) --
	(104.57, 25.90) --
	(104.57, 25.90) --
	(104.50, 25.90) --
	(104.50, 25.90) --
	(104.50, 25.90) --
	(104.44, 25.90) --
	(104.44, 25.90) --
	(104.44, 25.90) --
	(104.43, 25.90) --
	(104.43, 25.90) --
	(104.43, 25.90) --
	(104.36, 25.90) --
	(104.36, 25.90) --
	(104.36, 25.90) --
	(104.29, 25.90) --
	(104.29, 25.90) --
	(104.29, 25.90) --
	(104.22, 25.90) --
	(104.22, 25.90) --
	(104.22, 25.90) --
	(104.15, 25.90) --
	(104.15, 25.90) --
	(104.15, 25.90) --
	(104.08, 25.90) --
	(104.08, 25.90) --
	(104.08, 25.90) --
	(104.01, 25.90) --
	(104.01, 25.90) --
	(104.01, 25.90) --
	(103.94, 25.90) --
	(103.94, 25.90) --
	(103.94, 25.90) --
	(103.87, 25.90) --
	(103.87, 25.90) --
	(103.87, 25.90) --
	(103.80, 25.90) --
	(103.80, 25.90) --
	(103.80, 25.90) --
	(103.73, 25.90) --
	(103.73, 25.90) --
	(103.73, 25.90) --
	(103.66, 25.90) --
	(103.66, 25.90) --
	(103.66, 25.90) --
	(103.59, 25.90) --
	(103.59, 25.90) --
	(103.59, 25.90) --
	(103.52, 25.90) --
	(103.52, 25.90) --
	(103.52, 25.90) --
	(103.45, 25.90) --
	(103.45, 25.90) --
	(103.45, 25.90) --
	(103.37, 25.90) --
	(103.37, 25.90) --
	(103.37, 25.90) --
	(103.31, 25.90) --
	(103.31, 25.90) --
	(103.31, 25.90) --
	(103.23, 25.90) --
	(103.23, 25.90) --
	(103.23, 25.90) --
	(103.16, 25.90) --
	(103.16, 25.90) --
	(103.16, 25.90) --
	(103.09, 25.90) --
	(103.09, 25.90) --
	(103.09, 25.90) --
	(103.02, 25.90) --
	(103.02, 25.90) --
	(103.02, 25.90) --
	(102.95, 25.90) --
	(102.95, 25.90) --
	(102.95, 25.90) --
	(102.88, 25.90) --
	(102.88, 25.90) --
	(102.88, 25.90) --
	(102.81, 25.90) --
	(102.81, 25.90) --
	(102.81, 25.90) --
	(102.74, 25.90) --
	(102.74, 25.90) --
	(102.74, 25.90) --
	(102.67, 25.90) --
	(102.67, 25.90) --
	(102.67, 25.90) --
	(102.60, 25.90) --
	(102.60, 25.90) --
	(102.60, 25.90) --
	(102.53, 25.90) --
	(102.53, 25.90) --
	(102.53, 25.90) --
	(102.46, 25.90) --
	(102.46, 25.90) --
	(102.46, 25.90) --
	(102.39, 25.90) --
	(102.39, 25.90) --
	(102.39, 25.90) --
	(102.32, 25.90) --
	(102.32, 25.90) --
	(102.32, 25.90) --
	(102.25, 25.90) --
	(102.25, 25.90) --
	(102.25, 25.90) --
	(102.18, 25.90) --
	(102.18, 25.90) --
	(102.18, 25.90) --
	(102.11, 25.90) --
	(102.11, 25.90) --
	(102.11, 25.90) --
	(102.04, 25.90) --
	(102.04, 25.90) --
	(102.04, 25.90) --
	(101.97, 25.90) --
	(101.97, 25.90) --
	(101.97, 25.90) --
	(101.90, 25.90) --
	(101.90, 25.90) --
	(101.90, 25.90) --
	(101.82, 25.90) --
	(101.82, 25.90) --
	(101.82, 25.90) --
	(101.75, 25.90) --
	(101.75, 25.90) --
	(101.75, 25.90) --
	(101.68, 25.90) --
	(101.68, 25.90) --
	(101.68, 25.90) --
	(101.66, 25.90) --
	(101.66, 25.90) --
	(101.66, 25.90) --
	(101.61, 25.90) --
	(101.61, 25.90) --
	(101.61, 25.90) --
	(101.54, 25.90) --
	(101.54, 25.90) --
	(101.54, 25.90) --
	(101.47, 25.90) --
	(101.47, 25.90) --
	(101.47, 25.90) --
	(101.40, 25.90) --
	(101.40, 25.90) --
	(101.40, 25.90) --
	(101.33, 25.90) --
	(101.33, 25.90) --
	(101.33, 25.90) --
	(101.26, 25.90) --
	(101.26, 25.90) --
	(101.26, 25.90) --
	(101.19, 25.90) --
	(101.19, 25.90) --
	(101.19, 25.90) --
	(101.12, 25.90) --
	(101.12, 25.90) --
	(101.12, 25.90) --
	(101.05, 25.90) --
	(101.05, 25.90) --
	(101.05, 25.90) --
	(100.98, 25.90) --
	(100.98, 25.90) --
	(100.98, 25.90) --
	(100.91, 25.90) --
	(100.91, 25.90) --
	(100.91, 25.90) --
	(100.84, 25.90) --
	(100.84, 25.90) --
	(100.84, 25.90) --
	(100.77, 25.90) --
	(100.77, 25.90) --
	(100.77, 25.90) --
	(100.70, 25.90) --
	(100.70, 25.90) --
	(100.70, 25.90) --
	(100.62, 25.90) --
	(100.62, 25.90) --
	(100.62, 25.90) --
	(100.55, 25.90) --
	(100.55, 25.90) --
	(100.55, 25.90) --
	(100.48, 25.90) --
	(100.48, 25.90) --
	(100.48, 25.90) --
	(100.41, 25.90) --
	(100.41, 25.90) --
	(100.41, 25.90) --
	(100.34, 25.90) --
	(100.34, 25.90) --
	(100.34, 25.90) --
	(100.27, 25.90) --
	(100.27, 25.90) --
	(100.27, 25.90) --
	(100.20, 25.90) --
	(100.20, 25.90) --
	(100.20, 25.90) --
	(100.13, 25.90) --
	(100.13, 25.90) --
	(100.13, 25.90) --
	(100.06, 25.90) --
	(100.06, 25.90) --
	(100.06, 25.90) --
	( 99.99, 25.90) --
	( 99.99, 25.90) --
	( 99.99, 25.90) --
	( 99.92, 25.90) --
	( 99.92, 25.90) --
	( 99.92, 25.90) --
	( 99.85, 25.90) --
	( 99.85, 25.90) --
	( 99.85, 25.90) --
	( 99.78, 25.90) --
	( 99.78, 25.90) --
	( 99.78, 25.90) --
	( 99.71, 25.90) --
	( 99.71, 25.90) --
	( 99.71, 25.90) --
	( 99.64, 25.90) --
	( 99.64, 25.90) --
	( 99.64, 25.90) --
	( 99.56, 25.90) --
	( 99.56, 25.90) --
	( 99.56, 25.90) --
	( 99.49, 25.90) --
	( 99.49, 25.90) --
	( 99.49, 25.90) --
	( 99.42, 25.90) --
	( 99.42, 25.90) --
	( 99.42, 25.90) --
	( 99.35, 25.90) --
	( 99.35, 25.90) --
	( 99.35, 25.90) --
	( 99.28, 25.90) --
	( 99.28, 25.90) --
	( 99.28, 25.90) --
	( 99.21, 25.90) --
	( 99.21, 25.90) --
	( 99.21, 25.90) --
	( 99.14, 25.90) --
	( 99.14, 25.90) --
	( 99.14, 25.90) --
	( 99.07, 25.90) --
	( 99.07, 25.90) --
	( 99.07, 25.90) --
	( 99.07, 25.90) --
	( 99.07, 25.90) --
	( 99.07, 25.90) --
	( 99.00, 25.90) --
	( 99.00, 25.90) --
	( 99.00, 25.90) --
	( 98.93, 25.90) --
	( 98.93, 25.90) --
	( 98.93, 25.90) --
	( 98.86, 25.90) --
	( 98.86, 25.90) --
	( 98.86, 25.90) --
	( 98.79, 25.90) --
	( 98.79, 25.90) --
	( 98.79, 25.90) --
	( 98.72, 25.90) --
	( 98.72, 25.90) --
	( 98.72, 25.90) --
	( 98.65, 25.90) --
	( 98.65, 25.90) --
	( 98.65, 25.90) --
	( 98.57, 25.90) --
	( 98.57, 25.90) --
	( 98.57, 25.90) --
	( 98.50, 25.90) --
	( 98.50, 25.90) --
	( 98.50, 25.90) --
	( 98.43, 25.90) --
	( 98.43, 25.90) --
	( 98.43, 25.90) --
	( 98.36, 25.90) --
	( 98.36, 25.90) --
	( 98.36, 25.90) --
	( 98.29, 25.90) --
	( 98.29, 25.90) --
	( 98.29, 25.90) --
	( 98.22, 25.90) --
	( 98.22, 25.90) --
	( 98.22, 25.90) --
	( 98.15, 25.90) --
	( 98.15, 25.90) --
	( 98.15, 25.90) --
	( 98.08, 25.90) --
	( 98.08, 25.90) --
	( 98.08, 25.90) --
	( 98.01, 25.90) --
	( 98.01, 25.90) --
	( 98.01, 25.90) --
	( 97.94, 25.90) --
	( 97.94, 25.90) --
	( 97.94, 25.90) --
	( 97.87, 25.90) --
	( 97.87, 25.90) --
	( 97.87, 25.90) --
	( 97.80, 25.90) --
	( 97.80, 25.90) --
	( 97.80, 25.90) --
	( 97.72, 25.90) --
	( 97.72, 25.90) --
	( 97.72, 25.90) --
	( 97.65, 25.90) --
	( 97.65, 25.90) --
	( 97.65, 25.90) --
	( 97.58, 25.90) --
	( 97.58, 25.90) --
	( 97.58, 25.90) --
	( 97.51, 25.90) --
	( 97.51, 25.90) --
	( 97.51, 25.90) --
	( 97.44, 25.90) --
	( 97.44, 25.90) --
	( 97.44, 25.90) --
	( 97.37, 25.90) --
	( 97.37, 25.90) --
	( 97.37, 25.90) --
	( 97.30, 25.90) --
	( 97.30, 25.90) --
	( 97.30, 25.90) --
	( 97.23, 25.90) --
	( 97.23, 25.90) --
	( 97.23, 25.90) --
	( 97.16, 25.90) --
	( 97.16, 25.90) --
	( 97.16, 25.90) --
	( 97.09, 25.90) --
	( 97.09, 25.90) --
	( 97.09, 25.90) --
	( 97.02, 25.90) --
	( 97.02, 25.90) --
	( 97.02, 25.90) --
	( 96.94, 25.90) --
	( 96.94, 25.90) --
	( 96.94, 25.90) --
	( 96.87, 25.90) --
	( 96.87, 25.90) --
	( 96.87, 25.90) --
	( 96.80, 25.90) --
	( 96.80, 25.90) --
	( 96.80, 25.90) --
	( 96.77, 25.90) --
	( 96.77, 25.90) --
	( 96.77, 25.90) --
	( 96.73, 25.90) --
	( 96.73, 25.90) --
	( 96.73, 25.90) --
	( 96.66, 25.90) --
	( 96.66, 25.90) --
	( 96.66, 25.90) --
	( 96.59, 25.90) --
	( 96.59, 25.90) --
	( 96.59, 25.90) --
	( 96.52, 25.90) --
	( 96.52, 25.90) --
	( 96.52, 25.90) --
	( 96.45, 25.90) --
	( 96.45, 25.90) --
	( 96.45, 25.90) --
	( 96.38, 25.90) --
	( 96.38, 25.90) --
	( 96.38, 25.90) --
	( 96.31, 25.90) --
	( 96.31, 25.90) --
	( 96.31, 25.90) --
	( 96.24, 25.90) --
	( 96.24, 25.90) --
	( 96.24, 25.90) --
	( 96.16, 25.90) --
	( 96.16, 25.90) --
	( 96.16, 25.90) --
	( 96.09, 25.90) --
	( 96.09, 25.90) --
	( 96.09, 25.90) --
	( 96.02, 25.90) --
	( 96.02, 25.90) --
	( 96.02, 25.90) --
	( 95.95, 25.90) --
	( 95.95, 25.90) --
	( 95.95, 25.90) --
	( 95.88, 25.90) --
	( 95.88, 25.90) --
	( 95.88, 25.90) --
	( 95.81, 25.90) --
	( 95.81, 25.90) --
	( 95.81, 25.90) --
	( 95.74, 25.90) --
	( 95.74, 25.90) --
	( 95.74, 25.90) --
	( 95.67, 25.90) --
	( 95.67, 25.90) --
	( 95.67, 25.90) --
	( 95.60, 25.90) --
	( 95.60, 25.90) --
	( 95.60, 25.90) --
	( 95.53, 25.90) --
	( 95.53, 25.90) --
	( 95.53, 25.90) --
	( 95.45, 25.90) --
	( 95.45, 25.90) --
	( 95.45, 25.90) --
	( 95.38, 25.90) --
	( 95.38, 25.90) --
	( 95.38, 25.90) --
	( 95.31, 25.90) --
	( 95.31, 25.90) --
	( 95.31, 25.90) --
	( 95.24, 25.90) --
	( 95.24, 25.90) --
	( 95.24, 25.90) --
	( 95.17, 25.90) --
	( 95.17, 25.90) --
	( 95.17, 25.90) --
	( 95.10, 25.90) --
	( 95.10, 25.90) --
	( 95.10, 25.90) --
	( 95.03, 25.90) --
	( 95.03, 25.90) --
	( 95.03, 25.90) --
	( 94.96, 25.90) --
	( 94.96, 25.90) --
	( 94.96, 25.90) --
	( 94.95, 25.90) --
	( 94.95, 25.90) --
	( 94.95, 25.90) --
	( 94.89, 25.90) --
	( 94.89, 25.90) --
	( 94.89, 25.90) --
	( 94.82, 25.90) --
	( 94.82, 25.90) --
	( 94.82, 25.90) --
	( 94.75, 25.90) --
	( 94.75, 25.90) --
	( 94.75, 25.90) --
	( 94.67, 25.90) --
	( 94.67, 25.90) --
	( 94.67, 25.90) --
	( 94.60, 25.90) --
	( 94.60, 25.90) --
	( 94.60, 25.90) --
	( 94.53, 25.90) --
	( 94.53, 25.90) --
	( 94.53, 25.90) --
	( 94.46, 25.90) --
	( 94.46, 25.90) --
	( 94.46, 25.90) --
	( 94.39, 25.90) --
	( 94.39, 25.90) --
	( 94.39, 25.90) --
	( 94.32, 25.90) --
	( 94.32, 25.90) --
	( 94.32, 25.90) --
	( 94.25, 25.90) --
	( 94.25, 25.90) --
	( 94.25, 25.90) --
	( 94.18, 25.90) --
	( 94.18, 25.90) --
	( 94.18, 25.90) --
	( 94.10, 25.90) --
	( 94.10, 25.90) --
	( 94.10, 25.90) --
	( 94.03, 25.90) --
	( 94.03, 25.90) --
	( 94.03, 25.90) --
	( 93.96, 25.90) --
	( 93.96, 25.90) --
	( 93.96, 25.90) --
	( 93.89, 25.90) --
	( 93.89, 25.90) --
	( 93.89, 25.90) --
	( 93.82, 25.90) --
	( 93.82, 25.90) --
	( 93.82, 25.90) --
	( 93.75, 25.90) --
	( 93.75, 25.90) --
	( 93.75, 25.90) --
	( 93.68, 25.90) --
	( 93.68, 25.90) --
	( 93.68, 25.90) --
	( 93.61, 25.90) --
	( 93.61, 25.90) --
	( 93.61, 25.90) --
	( 93.54, 25.90) --
	( 93.54, 25.90) --
	( 93.54, 25.90) --
	( 93.46, 25.90) --
	( 93.46, 25.90) --
	( 93.46, 25.90) --
	( 93.39, 25.90) --
	( 93.39, 25.90) --
	( 93.39, 25.90) --
	( 93.32, 25.90) --
	( 93.32, 25.90) --
	( 93.32, 25.90) --
	( 93.25, 25.90) --
	( 93.25, 25.90) --
	( 93.25, 25.90) --
	( 93.18, 25.90) --
	( 93.18, 25.90) --
	( 93.18, 25.90) --
	( 93.11, 25.90) --
	( 93.11, 25.90) --
	( 93.11, 25.90) --
	( 93.04, 25.90) --
	( 93.04, 25.90) --
	( 93.04, 25.90) --
	( 92.97, 25.90) --
	( 92.97, 25.90) --
	( 92.97, 25.90) --
	( 92.90, 25.90) --
	( 92.90, 25.90) --
	( 92.90, 25.90) --
	( 92.83, 25.90) --
	( 92.83, 25.90) --
	( 92.83, 25.90) --
	( 92.75, 25.90) --
	( 92.75, 25.90) --
	( 92.75, 25.90) --
	( 92.68, 25.90) --
	( 92.68, 25.90) --
	( 92.68, 25.90) --
	( 92.65, 25.90) --
	( 92.65, 25.90) --
	( 92.65, 25.90) --
	( 92.61, 25.90) --
	( 92.61, 25.90) --
	( 92.61, 25.90) --
	( 92.54, 25.90) --
	( 92.54, 25.90) --
	( 92.54, 25.90) --
	( 92.47, 25.90) --
	( 92.47, 25.90) --
	( 92.47, 25.90) --
	( 92.40, 25.90) --
	( 92.40, 25.90) --
	( 92.40, 25.90) --
	( 92.33, 25.90) --
	( 92.33, 25.90) --
	( 92.33, 25.90) --
	( 92.26, 25.90) --
	( 92.26, 25.90) --
	( 92.26, 25.90) --
	( 92.18, 25.90) --
	( 92.18, 25.90) --
	( 92.18, 25.90) --
	( 92.11, 25.90) --
	( 92.11, 25.90) --
	( 92.11, 25.90) --
	( 92.04, 25.90) --
	( 92.04, 25.90) --
	( 92.04, 25.90) --
	( 91.97, 25.90) --
	( 91.97, 25.90) --
	( 91.97, 25.90) --
	( 91.90, 25.90) --
	( 91.90, 25.90) --
	( 91.90, 25.90) --
	( 91.83, 25.90) --
	( 91.83, 25.90) --
	( 91.83, 25.90) --
	( 91.76, 25.90) --
	( 91.76, 25.90) --
	( 91.76, 25.90) --
	( 91.69, 25.90) --
	( 91.69, 25.90) --
	( 91.69, 25.90) --
	( 91.61, 25.90) --
	( 91.61, 25.90) --
	( 91.61, 25.90) --
	( 91.54, 25.90) --
	( 91.54, 25.90) --
	( 91.54, 25.90) --
	( 91.47, 25.90) --
	( 91.47, 25.90) --
	( 91.47, 25.90) --
	( 91.40, 25.90) --
	( 91.40, 25.90) --
	( 91.40, 25.90) --
	( 91.33, 25.90) --
	( 91.33, 25.90) --
	( 91.33, 25.90) --
	( 91.26, 25.90) --
	( 91.26, 25.90) --
	( 91.26, 25.90) --
	( 91.19, 25.90) --
	( 91.19, 25.90) --
	( 91.19, 25.90) --
	( 91.12, 25.90) --
	( 91.12, 25.90) --
	( 91.12, 25.90) --
	( 91.04, 25.90) --
	( 91.04, 25.90) --
	( 91.04, 25.90) --
	( 90.97, 25.90) --
	( 90.97, 25.90) --
	( 90.97, 25.90) --
	( 90.90, 25.90) --
	( 90.90, 25.90) --
	( 90.90, 25.90) --
	( 90.83, 25.90) --
	( 90.83, 25.90) --
	( 90.83, 25.90) --
	( 90.76, 25.90) --
	( 90.76, 25.90) --
	( 90.76, 25.90) --
	( 90.69, 25.90) --
	( 90.69, 25.90) --
	( 90.69, 25.90) --
	( 90.64, 25.90) --
	( 90.64, 25.90) --
	( 90.64, 25.90) --
	( 90.62, 25.90) --
	( 90.62, 25.90) --
	( 90.62, 25.90) --
	( 90.54, 25.90) --
	( 90.54, 25.90) --
	( 90.54, 25.90) --
	( 90.47, 25.90) --
	( 90.47, 25.90) --
	( 90.47, 25.90) --
	( 90.40, 25.90) --
	( 90.40, 25.90) --
	( 90.40, 25.90) --
	( 90.33, 25.90) --
	( 90.33, 25.90) --
	( 90.33, 25.90) --
	( 90.26, 25.90) --
	( 90.26, 25.90) --
	( 90.26, 25.90) --
	( 90.19, 25.90) --
	( 90.19, 25.90) --
	( 90.19, 25.90) --
	( 90.12, 25.90) --
	( 90.12, 25.90) --
	( 90.12, 25.90) --
	( 90.04, 25.90) --
	( 90.04, 25.90) --
	( 90.04, 25.90) --
	( 89.97, 25.90) --
	( 89.97, 25.90) --
	( 89.97, 25.90) --
	( 89.90, 25.90) --
	( 89.90, 25.90) --
	( 89.90, 25.90) --
	( 89.83, 25.90) --
	( 89.83, 25.90) --
	( 89.83, 25.90) --
	( 89.76, 25.90) --
	( 89.76, 25.90) --
	( 89.76, 25.90) --
	( 89.69, 25.90) --
	( 89.69, 25.90) --
	( 89.69, 25.90) --
	( 89.62, 25.90) --
	( 89.62, 25.90) --
	( 89.62, 25.90) --
	( 89.55, 25.90) --
	( 89.55, 25.90) --
	( 89.55, 25.90) --
	( 89.47, 25.90) --
	( 89.47, 25.90) --
	( 89.47, 25.90) --
	( 89.40, 25.90) --
	( 89.40, 25.90) --
	( 89.40, 25.90) --
	( 89.33, 25.90) --
	( 89.33, 25.90) --
	( 89.33, 25.90) --
	( 89.26, 25.90) --
	( 89.26, 25.90) --
	( 89.26, 25.90) --
	( 89.19, 25.90) --
	( 89.19, 25.90) --
	( 89.19, 25.90) --
	( 89.12, 25.90) --
	( 89.12, 25.90) --
	( 89.12, 25.90) --
	( 89.05, 25.90) --
	( 89.05, 25.90) --
	( 89.05, 25.90) --
	( 88.97, 25.90) --
	( 88.97, 25.90) --
	( 88.97, 25.90) --
	( 88.92, 25.90) --
	( 88.92, 25.90) --
	( 88.92, 25.90) --
	( 88.90, 25.90) --
	( 88.90, 25.90) --
	( 88.90, 25.90) --
	( 88.83, 25.90) --
	( 88.83, 25.90) --
	( 88.83, 25.90) --
	( 88.76, 25.90) --
	( 88.76, 25.90) --
	( 88.76, 25.90) --
	( 88.69, 25.90) --
	( 88.69, 25.90) --
	( 88.69, 25.90) --
	( 88.62, 25.90) --
	( 88.62, 25.90) --
	( 88.62, 25.90) --
	( 88.54, 25.90) --
	( 88.54, 25.90) --
	( 88.54, 25.90) --
	( 88.47, 25.90) --
	( 88.47, 25.90) --
	( 88.47, 25.90) --
	( 88.40, 25.90) --
	( 88.40, 25.90) --
	( 88.40, 25.90) --
	( 88.33, 25.90) --
	( 88.33, 25.90) --
	( 88.33, 25.90) --
	( 88.26, 25.90) --
	( 88.26, 25.90) --
	( 88.26, 25.90) --
	( 88.19, 25.90) --
	( 88.19, 25.90) --
	( 88.19, 25.90) --
	( 88.12, 25.90) --
	( 88.12, 25.90) --
	( 88.12, 25.90) --
	( 88.04, 25.90) --
	( 88.04, 25.90) --
	( 88.04, 25.90) --
	( 87.97, 25.90) --
	( 87.97, 25.90) --
	( 87.97, 25.90) --
	( 87.90, 25.90) --
	( 87.90, 25.90) --
	( 87.90, 25.90) --
	( 87.83, 25.90) --
	( 87.83, 25.90) --
	( 87.83, 25.90) --
	( 87.76, 25.90) --
	( 87.76, 25.90) --
	( 87.76, 25.90) --
	( 87.69, 25.90) --
	( 87.69, 25.90) --
	( 87.69, 25.90) --
	( 87.62, 25.90) --
	( 87.62, 25.90) --
	( 87.62, 25.90) --
	( 87.54, 25.90) --
	( 87.54, 25.90) --
	( 87.54, 25.90) --
	( 87.47, 25.90) --
	( 87.47, 25.90) --
	( 87.47, 25.90) --
	( 87.40, 25.90) --
	( 87.40, 25.90) --
	( 87.40, 25.90) --
	( 87.33, 25.90) --
	( 87.33, 25.90) --
	( 87.33, 25.90) --
	( 87.26, 25.90) --
	( 87.26, 25.90) --
	( 87.26, 25.90) --
	( 87.19, 25.90) --
	( 87.19, 25.90) --
	( 87.19, 25.90) --
	( 87.11, 25.90) --
	( 87.11, 25.90) --
	( 87.11, 25.90) --
	( 87.04, 25.90) --
	( 87.04, 25.90) --
	( 87.04, 25.90) --
	( 86.97, 25.90) --
	( 86.97, 25.90) --
	( 86.97, 25.90) --
	( 86.90, 25.90) --
	( 86.90, 25.90) --
	( 86.90, 25.90) --
	( 86.90, 25.90) --
	( 86.90, 25.90) --
	( 86.90, 25.90) --
	( 86.83, 25.90) --
	( 86.83, 25.90) --
	( 86.83, 25.90) --
	( 86.76, 25.90) --
	( 86.76, 25.90) --
	( 86.76, 25.90) --
	( 86.68, 25.90) --
	( 86.68, 25.90) --
	( 86.68, 25.90) --
	( 86.61, 25.90) --
	( 86.61, 25.90) --
	( 86.61, 25.90) --
	( 86.54, 25.90) --
	( 86.54, 25.90) --
	( 86.54, 25.90) --
	( 86.47, 25.90) --
	( 86.47, 25.90) --
	( 86.47, 25.90) --
	( 86.40, 25.90) --
	( 86.40, 25.90) --
	( 86.40, 25.90) --
	( 86.33, 25.90) --
	( 86.33, 25.90) --
	( 86.33, 25.90) --
	( 86.25, 25.90) --
	( 86.25, 25.90) --
	( 86.25, 25.90) --
	( 86.18, 25.90) --
	( 86.18, 25.90) --
	( 86.18, 25.90) --
	( 86.11, 25.90) --
	( 86.11, 25.90) --
	( 86.11, 25.90) --
	( 86.04, 25.90) --
	( 86.04, 25.90) --
	( 86.04, 25.90) --
	( 85.97, 25.90) --
	( 85.97, 25.90) --
	( 85.97, 25.90) --
	( 85.90, 25.90) --
	( 85.90, 25.90) --
	( 85.90, 25.90) --
	( 85.83, 25.90) --
	( 85.83, 25.90) --
	( 85.83, 25.90) --
	( 85.75, 25.90) --
	( 85.75, 25.90) --
	( 85.75, 25.90) --
	( 85.68, 25.90) --
	( 85.68, 25.90) --
	( 85.68, 25.90) --
	( 85.61, 25.90) --
	( 85.61, 25.90) --
	( 85.61, 25.90) --
	( 85.54, 25.90) --
	( 85.54, 25.90) --
	( 85.54, 25.90) --
	( 85.47, 25.90) --
	( 85.47, 25.90) --
	( 85.47, 25.90) --
	( 85.47, 25.90) --
	( 85.47, 25.90) --
	( 85.47, 25.90) --
	( 85.40, 25.90) --
	( 85.40, 25.90) --
	( 85.40, 25.90) --
	( 85.32, 25.90) --
	( 85.32, 25.90) --
	( 85.32, 25.90) --
	( 85.25, 25.90) --
	( 85.25, 25.90) --
	( 85.25, 25.90) --
	( 85.18, 25.90) --
	( 85.18, 25.90) --
	( 85.18, 25.90) --
	( 85.11, 25.90) --
	( 85.11, 25.90) --
	( 85.11, 25.90) --
	( 85.04, 25.90) --
	( 85.04, 25.90) --
	( 85.04, 25.90) --
	( 84.96, 25.90) --
	( 84.96, 25.90) --
	( 84.96, 25.90) --
	( 84.89, 25.90) --
	( 84.89, 25.90) --
	( 84.89, 25.90) --
	( 84.82, 25.90) --
	( 84.82, 25.90) --
	( 84.82, 25.90) --
	( 84.75, 25.90) --
	( 84.75, 25.90) --
	( 84.75, 25.90) --
	( 84.68, 25.90) --
	( 84.68, 25.90) --
	( 84.68, 25.90) --
	( 84.61, 25.90) --
	( 84.61, 25.90) --
	( 84.61, 25.90) --
	( 84.54, 25.90) --
	( 84.54, 25.90) --
	( 84.54, 25.90) --
	( 84.46, 25.90) --
	( 84.46, 25.90) --
	( 84.46, 25.90) --
	( 84.39, 25.90) --
	( 84.39, 25.90) --
	( 84.39, 25.90) --
	( 84.32, 25.90) --
	( 84.32, 25.90) --
	( 84.32, 25.90) --
	( 84.25, 25.90) --
	( 84.25, 25.90) --
	( 84.25, 25.90) --
	( 84.18, 25.90) --
	( 84.18, 25.90) --
	( 84.18, 25.90) --
	( 84.12, 25.90) --
	( 84.12, 25.90) --
	( 84.12, 25.90) --
	( 84.10, 25.90) --
	( 84.10, 25.90) --
	( 84.10, 25.90) --
	( 84.03, 25.90) --
	( 84.03, 25.90) --
	( 84.03, 25.90) --
	( 83.96, 25.90) --
	( 83.96, 25.90) --
	( 83.96, 25.90) --
	( 83.89, 25.90) --
	( 83.89, 25.90) --
	( 83.89, 25.90) --
	( 83.82, 25.90) --
	( 83.82, 25.90) --
	( 83.82, 25.90) --
	( 83.75, 25.90) --
	( 83.75, 25.90) --
	( 83.75, 25.90) --
	( 83.67, 25.90) --
	( 83.67, 25.90) --
	( 83.67, 25.90) --
	( 83.60, 25.90) --
	( 83.60, 25.90) --
	( 83.60, 25.90) --
	( 83.53, 25.90) --
	( 83.53, 25.90) --
	( 83.53, 25.90) --
	( 83.46, 25.90) --
	( 83.46, 25.90) --
	( 83.46, 25.90) --
	( 83.39, 25.90) --
	( 83.39, 25.90) --
	( 83.39, 25.90) --
	( 83.31, 25.90) --
	( 83.31, 25.90) --
	( 83.31, 25.90) --
	( 83.24, 25.90) --
	( 83.24, 25.90) --
	( 83.24, 25.90) --
	( 83.17, 25.90) --
	( 83.17, 25.90) --
	( 83.17, 25.90) --
	( 83.10, 25.90) --
	( 83.10, 25.90) --
	( 83.10, 25.90) --
	( 83.03, 25.90) --
	( 83.03, 25.90) --
	( 83.03, 25.90) --
	( 82.96, 25.90) --
	( 82.96, 25.90) --
	( 82.96, 25.90) --
	( 82.88, 25.90) --
	( 82.88, 25.90) --
	( 82.88, 25.90) --
	( 82.81, 25.90) --
	( 82.81, 25.90) --
	( 82.81, 25.90) --
	( 82.74, 25.90) --
	( 82.74, 25.90) --
	( 82.74, 25.90) --
	( 82.67, 25.90) --
	( 82.67, 25.90) --
	( 82.67, 25.90) --
	( 82.60, 25.90) --
	( 82.60, 25.90) --
	( 82.60, 25.90) --
	( 82.59, 25.90) --
	( 82.59, 25.90) --
	( 82.59, 25.90) --
	( 82.52, 25.90) --
	( 82.52, 25.90) --
	( 82.52, 25.90) --
	( 82.45, 25.90) --
	( 82.45, 25.90) --
	( 82.45, 25.90) --
	( 82.38, 25.90) --
	( 82.38, 25.90) --
	( 82.38, 25.90) --
	( 82.31, 25.90) --
	( 82.31, 25.90) --
	( 82.31, 25.90) --
	( 82.24, 25.90) --
	( 82.24, 25.90) --
	( 82.24, 25.90) --
	( 82.16, 25.90) --
	( 82.16, 25.90) --
	( 82.16, 25.90) --
	( 82.09, 25.90) --
	( 82.09, 25.90) --
	( 82.09, 25.90) --
	( 82.02, 25.90) --
	( 82.02, 25.90) --
	( 82.02, 25.90) --
	( 81.95, 25.90) --
	( 81.95, 25.90) --
	( 81.95, 25.90) --
	( 81.88, 25.90) --
	( 81.88, 25.90) --
	( 81.88, 25.90) --
	( 81.81, 25.90) --
	( 81.81, 25.90) --
	( 81.81, 25.90) --
	( 81.73, 25.90) --
	( 81.73, 25.90) --
	( 81.73, 25.90) --
	( 81.66, 25.90) --
	( 81.66, 25.90) --
	( 81.66, 25.90) --
	( 81.59, 25.90) --
	( 81.59, 25.90) --
	( 81.59, 25.90) --
	( 81.52, 25.90) --
	( 81.52, 25.90) --
	( 81.52, 25.90) --
	( 81.45, 25.90) --
	( 81.45, 25.90) --
	( 81.45, 25.90) --
	( 81.44, 25.90) --
	( 81.44, 25.90) --
	( 81.44, 25.90) --
	( 81.37, 25.90) --
	( 81.37, 25.90) --
	( 81.37, 25.90) --
	( 81.30, 25.90) --
	( 81.30, 25.90) --
	( 81.30, 25.90) --
	( 81.23, 25.90) --
	( 81.23, 25.90) --
	( 81.23, 25.90) --
	( 81.16, 25.90) --
	( 81.16, 25.90) --
	( 81.16, 25.90) --
	( 81.09, 25.90) --
	( 81.09, 25.90) --
	( 81.09, 25.90) --
	( 81.01, 25.90) --
	( 81.01, 25.90) --
	( 81.01, 25.90) --
	( 80.94, 25.90) --
	( 80.94, 25.90) --
	( 80.94, 25.90) --
	( 80.87, 25.90) --
	( 80.87, 25.90) --
	( 80.87, 25.90) --
	( 80.80, 25.90) --
	( 80.80, 25.90) --
	( 80.80, 25.90) --
	( 80.73, 25.90) --
	( 80.73, 25.90) --
	( 80.73, 25.90) --
	( 80.65, 25.90) --
	( 80.65, 25.90) --
	( 80.65, 25.90) --
	( 80.58, 25.90) --
	( 80.58, 25.90) --
	( 80.58, 25.90) --
	( 80.51, 25.90) --
	( 80.51, 25.90) --
	( 80.51, 25.90) --
	( 80.48, 25.90) --
	( 80.48, 25.90) --
	( 80.48, 25.90) --
	( 80.44, 25.90) --
	( 80.44, 25.90) --
	( 80.44, 25.90) --
	( 80.37, 25.90) --
	( 80.37, 25.90) --
	( 80.37, 25.90) --
	( 80.29, 25.90) --
	( 80.29, 25.90) --
	( 80.29, 25.90) --
	( 80.22, 25.90) --
	( 80.22, 25.90) --
	( 80.22, 25.90) --
	( 80.15, 25.90) --
	( 80.15, 25.90) --
	( 80.15, 25.90) --
	( 80.08, 25.90) --
	( 80.08, 25.90) --
	( 80.08, 25.90) --
	( 80.01, 25.90) --
	( 80.01, 25.90) --
	( 80.01, 25.90) --
	( 79.93, 25.90) --
	( 79.93, 25.90) --
	( 79.93, 25.90) --
	( 79.86, 25.90) --
	( 79.86, 25.90) --
	( 79.86, 25.90) --
	( 79.79, 25.90) --
	( 79.79, 25.90) --
	( 79.79, 25.90) --
	( 79.72, 25.90) --
	( 79.72, 25.90) --
	( 79.72, 25.90) --
	( 79.65, 25.90) --
	( 79.65, 25.90) --
	( 79.65, 25.90) --
	( 79.62, 25.90) --
	( 79.62, 25.90) --
	( 79.62, 25.90) --
	( 79.57, 25.90) --
	( 79.57, 25.90) --
	( 79.57, 25.90) --
	( 79.50, 25.90) --
	( 79.50, 25.90) --
	( 79.50, 25.90) --
	( 79.43, 25.90) --
	( 79.43, 25.90) --
	( 79.43, 25.90) --
	( 79.36, 25.90) --
	( 79.36, 25.90) --
	( 79.36, 25.90) --
	( 79.29, 25.90) --
	( 79.29, 25.90) --
	( 79.29, 25.90) --
	( 79.21, 25.90) --
	( 79.21, 25.90) --
	( 79.21, 25.90) --
	( 79.14, 25.90) --
	( 79.14, 25.90) --
	( 79.14, 25.90) --
	( 79.07, 25.90) --
	( 79.07, 25.90) --
	( 79.07, 25.90) --
	( 79.00, 25.90) --
	( 79.00, 25.90) --
	( 79.00, 25.90) --
	( 78.93, 25.90) --
	( 78.93, 25.90) --
	( 78.93, 25.90) --
	( 78.85, 25.90) --
	( 78.85, 25.90) --
	( 78.85, 25.90) --
	( 78.78, 25.90) --
	( 78.78, 25.90) --
	( 78.78, 25.90) --
	( 78.76, 25.90) --
	( 78.76, 25.90) --
	( 78.76, 25.90) --
	( 78.71, 25.90) --
	( 78.71, 25.90) --
	( 78.71, 25.90) --
	( 78.64, 25.90) --
	( 78.64, 25.90) --
	( 78.64, 25.90) --
	( 78.56, 25.90) --
	( 78.56, 25.90) --
	( 78.56, 25.90) --
	( 78.49, 25.90) --
	( 78.49, 25.90) --
	( 78.49, 25.90) --
	( 78.42, 25.90) --
	( 78.42, 25.90) --
	( 78.42, 25.90) --
	( 78.35, 25.90) --
	( 78.35, 25.90) --
	( 78.35, 25.90) --
	( 78.28, 25.90) --
	( 78.28, 25.90) --
	( 78.28, 25.90) --
	( 78.20, 25.90) --
	( 78.20, 25.90) --
	( 78.20, 25.90) --
	( 78.13, 25.90) --
	( 78.13, 25.90) --
	( 78.13, 25.90) --
	( 78.06, 25.90) --
	( 78.06, 25.90) --
	( 78.06, 25.90) --
	( 77.99, 25.90) --
	( 77.99, 25.90) --
	( 77.99, 25.90) --
	( 77.91, 25.90) --
	( 77.91, 25.90) --
	( 77.91, 25.90) --
	( 77.84, 25.90) --
	( 77.84, 25.90) --
	( 77.84, 25.90) --
	( 77.77, 25.90) --
	( 77.77, 25.90) --
	( 77.77, 25.90) --
	( 77.70, 25.90) --
	( 77.70, 25.90) --
	( 77.70, 25.90) --
	( 77.70, 25.90) --
	( 77.70, 25.90) --
	( 77.70, 25.90) --
	( 77.63, 25.90) --
	( 77.63, 25.90) --
	( 77.63, 25.90) --
	( 77.55, 25.90) --
	( 77.55, 25.90) --
	( 77.55, 25.90) --
	( 77.48, 25.90) --
	( 77.48, 25.90) --
	( 77.48, 25.90) --
	( 77.41, 25.90) --
	( 77.41, 25.90) --
	( 77.41, 25.90) --
	( 77.34, 25.90) --
	( 77.34, 25.90) --
	( 77.34, 25.90) --
	( 77.27, 25.90) --
	( 77.27, 25.90) --
	( 77.27, 25.90) --
	( 77.19, 25.90) --
	( 77.19, 25.90) --
	( 77.19, 25.90) --
	( 77.12, 25.90) --
	( 77.12, 25.90) --
	( 77.12, 25.90) --
	( 77.05, 25.90) --
	( 77.05, 25.90) --
	( 77.05, 25.90) --
	( 76.98, 25.90) --
	( 76.98, 25.90) --
	( 76.98, 25.90) --
	( 76.90, 25.90) --
	( 76.90, 25.90) --
	( 76.90, 25.90) --
	( 76.83, 25.90) --
	( 76.83, 25.90) --
	( 76.83, 25.90) --
	( 76.76, 25.90) --
	( 76.76, 25.90) --
	( 76.76, 25.90) --
	( 76.75, 25.90) --
	( 76.75, 25.90) --
	( 76.75, 25.90) --
	( 76.69, 25.90) --
	( 76.69, 25.90) --
	( 76.69, 25.90) --
	( 76.61, 25.90) --
	( 76.61, 25.90) --
	( 76.61, 25.90) --
	( 76.54, 25.90) --
	( 76.54, 25.90) --
	( 76.54, 25.90) --
	( 76.47, 25.90) --
	( 76.47, 25.90) --
	( 76.47, 25.90) --
	( 76.40, 25.90) --
	( 76.40, 25.90) --
	( 76.40, 25.90) --
	( 76.33, 25.90) --
	( 76.33, 25.90) --
	( 76.33, 25.90) --
	( 76.25, 25.90) --
	( 76.25, 25.90) --
	( 76.25, 25.90) --
	( 76.18, 25.90) --
	( 76.18, 25.90) --
	( 76.18, 25.90) --
	( 76.11, 25.90) --
	( 76.11, 25.90) --
	( 76.11, 25.90) --
	( 76.04, 25.90) --
	( 76.04, 25.90) --
	( 76.04, 25.90) --
	( 75.96, 25.90) --
	( 75.96, 25.90) --
	( 75.96, 25.90) --
	( 75.89, 25.90) --
	( 75.89, 25.90) --
	( 75.89, 25.90) --
	( 75.82, 25.90) --
	( 75.82, 25.90) --
	( 75.82, 25.90) --
	( 75.79, 25.90) --
	( 75.79, 25.90) --
	( 75.79, 25.90) --
	( 75.75, 25.90) --
	( 75.75, 25.90) --
	( 75.75, 25.90) --
	( 75.67, 25.90) --
	( 75.67, 25.90) --
	( 75.67, 25.90) --
	( 75.60, 25.90) --
	( 75.60, 25.90) --
	( 75.60, 25.90) --
	( 75.53, 25.90) --
	( 75.53, 25.90) --
	( 75.53, 25.90) --
	( 75.46, 25.90) --
	( 75.46, 25.90) --
	( 75.46, 25.90) --
	( 75.39, 25.90) --
	( 75.39, 25.90) --
	( 75.39, 25.90) --
	( 75.31, 25.90) --
	( 75.31, 25.90) --
	( 75.31, 25.90) --
	( 75.24, 25.90) --
	( 75.24, 25.90) --
	( 75.24, 25.90) --
	( 75.17, 25.90) --
	( 75.17, 25.90) --
	( 75.17, 25.90) --
	( 75.10, 25.90) --
	( 75.10, 25.90) --
	( 75.10, 25.90) --
	( 75.02, 25.90) --
	( 75.02, 25.90) --
	( 75.02, 25.90) --
	( 74.95, 25.90) --
	( 74.95, 25.90) --
	( 74.95, 25.90) --
	( 74.92, 25.90) --
	( 74.92, 25.90) --
	( 74.92, 25.90) --
	( 74.88, 25.90) --
	( 74.88, 25.90) --
	( 74.88, 25.90) --
	( 74.81, 25.90) --
	( 74.81, 25.90) --
	( 74.81, 25.90) --
	( 74.73, 25.90) --
	( 74.73, 25.90) --
	( 74.73, 25.90) --
	( 74.66, 25.90) --
	( 74.66, 25.90) --
	( 74.66, 25.90) --
	( 74.59, 25.90) --
	( 74.59, 25.90) --
	( 74.59, 25.90) --
	( 74.52, 25.90) --
	( 74.52, 25.90) --
	( 74.52, 25.90) --
	( 74.44, 25.90) --
	( 74.44, 25.90) --
	( 74.44, 25.90) --
	( 74.37, 25.90) --
	( 74.37, 25.90) --
	( 74.37, 25.90) --
	( 74.30, 25.90) --
	( 74.30, 25.90) --
	( 74.30, 25.90) --
	( 74.23, 25.90) --
	( 74.23, 25.90) --
	( 74.23, 25.90) --
	( 74.16, 25.90) --
	( 74.16, 25.90) --
	( 74.16, 25.90) --
	( 74.08, 25.90) --
	( 74.08, 25.90) --
	( 74.08, 25.90) --
	( 74.01, 25.90) --
	( 74.01, 25.90) --
	( 74.01, 25.90) --
	( 73.97, 25.90) --
	( 73.97, 25.90) --
	( 73.97, 25.90) --
	( 73.94, 25.90) --
	( 73.94, 25.90) --
	( 73.94, 25.90) --
	( 73.86, 25.90) --
	( 73.86, 25.90) --
	( 73.86, 25.90) --
	( 73.79, 25.90) --
	( 73.79, 25.90) --
	( 73.79, 25.90) --
	( 73.72, 25.90) --
	( 73.72, 25.90) --
	( 73.72, 25.90) --
	( 73.65, 25.90) --
	( 73.65, 25.90) --
	( 73.65, 25.90) --
	( 73.58, 25.90) --
	( 73.58, 25.90) --
	( 73.58, 25.90) --
	( 73.50, 25.90) --
	( 73.50, 25.90) --
	( 73.50, 25.90) --
	( 73.43, 25.90) --
	( 73.43, 25.90) --
	( 73.43, 25.90) --
	( 73.36, 25.90) --
	( 73.36, 25.90) --
	( 73.36, 25.90) --
	( 73.28, 25.90) --
	( 73.28, 25.90) --
	( 73.28, 25.90) --
	( 73.21, 25.90) --
	( 73.21, 25.90) --
	( 73.21, 25.90) --
	( 73.20, 25.90) --
	( 73.20, 25.90) --
	( 73.20, 25.90) --
	( 73.14, 25.90) --
	( 73.14, 25.90) --
	( 73.14, 25.90) --
	( 73.07, 25.90) --
	( 73.07, 25.90) --
	( 73.07, 25.90) --
	( 73.00, 25.90) --
	( 73.00, 25.90) --
	( 73.00, 25.90) --
	( 72.92, 25.90) --
	( 72.92, 25.90) --
	( 72.92, 25.90) --
	( 72.85, 25.90) --
	( 72.85, 25.90) --
	( 72.85, 25.90) --
	( 72.78, 25.90) --
	( 72.78, 25.90) --
	( 72.78, 25.90) --
	( 72.71, 25.90) --
	( 72.71, 25.90) --
	( 72.71, 25.90) --
	( 72.63, 25.90) --
	( 72.63, 25.90) --
	( 72.63, 25.90) --
	( 72.56, 25.90) --
	( 72.56, 25.90) --
	( 72.56, 25.90) --
	( 72.49, 25.90) --
	( 72.49, 25.90) --
	( 72.49, 25.90) --
	( 72.42, 25.90) --
	( 72.42, 25.90) --
	( 72.42, 25.90) --
	( 72.34, 25.90) --
	( 72.34, 25.90) --
	( 72.34, 25.90) --
	( 72.27, 25.90) --
	( 72.27, 25.90) --
	( 72.27, 25.90) --
	( 72.20, 25.90) --
	( 72.20, 25.90) --
	( 72.20, 25.90) --
	( 72.13, 25.90) --
	( 72.13, 25.90) --
	( 72.13, 25.90) --
	( 72.05, 25.90) --
	( 72.05, 25.90) --
	( 72.05, 25.90) --
	( 71.98, 25.90) --
	( 71.98, 25.90) --
	( 71.98, 25.90) --
	( 71.95, 25.90) --
	( 71.95, 25.90) --
	( 71.95, 25.90) --
	( 71.91, 25.90) --
	( 71.91, 25.90) --
	( 71.91, 25.90) --
	( 71.83, 25.90) --
	( 71.83, 25.90) --
	( 71.83, 25.90) --
	( 71.76, 25.90) --
	( 71.76, 25.90) --
	( 71.76, 25.90) --
	( 71.69, 25.90) --
	( 71.69, 25.90) --
	( 71.69, 25.90) --
	( 71.62, 25.90) --
	( 71.62, 25.90) --
	( 71.62, 25.90) --
	( 71.54, 25.90) --
	( 71.54, 25.90) --
	( 71.54, 25.90) --
	( 71.47, 25.90) --
	( 71.47, 25.90) --
	( 71.47, 25.90) --
	( 71.40, 25.90) --
	( 71.40, 25.90) --
	( 71.40, 25.90) --
	( 71.33, 25.90) --
	( 71.33, 25.90) --
	( 71.33, 25.90) --
	( 71.26, 25.90) --
	( 71.26, 25.90) --
	( 71.26, 25.90) --
	( 71.19, 25.90) --
	( 71.19, 25.90) --
	( 71.19, 25.90) --
	( 71.18, 25.90) --
	( 71.18, 25.90) --
	( 71.18, 25.90) --
	( 71.11, 25.90) --
	( 71.11, 25.90) --
	( 71.11, 25.90) --
	( 71.04, 25.90) --
	( 71.04, 25.90) --
	( 71.04, 25.90) --
	( 70.96, 25.90) --
	( 70.96, 25.90) --
	( 70.96, 25.90) --
	( 70.89, 25.90) --
	( 70.89, 25.90) --
	( 70.89, 25.90) --
	( 70.82, 25.90) --
	( 70.82, 25.90) --
	( 70.82, 25.90) --
	( 70.75, 25.90) --
	( 70.75, 25.90) --
	( 70.75, 25.90) --
	( 70.67, 25.90) --
	( 70.67, 25.90) --
	( 70.67, 25.90) --
	( 70.60, 25.90) --
	( 70.60, 25.90) --
	( 70.60, 25.90) --
	( 70.53, 25.90) --
	( 70.53, 25.90) --
	( 70.53, 25.90) --
	( 70.46, 25.90) --
	( 70.46, 25.90) --
	( 70.46, 25.90) --
	( 70.42, 25.90) --
	( 70.42, 25.90) --
	( 70.42, 25.90) --
	( 70.38, 25.90) --
	( 70.38, 25.90) --
	( 70.38, 25.90) --
	( 70.31, 25.90) --
	( 70.31, 25.90) --
	( 70.31, 25.90) --
	( 70.24, 25.90) --
	( 70.24, 25.90) --
	( 70.24, 25.90) --
	( 70.17, 25.90) --
	( 70.17, 25.90) --
	( 70.17, 25.90) --
	( 70.09, 25.90) --
	( 70.09, 25.90) --
	( 70.09, 25.90) --
	( 70.02, 25.90) --
	( 70.02, 25.90) --
	( 70.02, 25.90) --
	( 69.95, 25.90) --
	( 69.95, 25.90) --
	( 69.95, 25.90) --
	( 69.87, 25.90) --
	( 69.87, 25.90) --
	( 69.87, 25.90) --
	( 69.80, 25.90) --
	( 69.80, 25.90) --
	( 69.80, 25.90) --
	( 69.73, 25.90) --
	( 69.73, 25.90) --
	( 69.73, 25.90) --
	( 69.66, 25.90) --
	( 69.66, 25.90) --
	( 69.66, 25.90) --
	( 69.58, 25.90) --
	( 69.58, 25.90) --
	( 69.58, 25.90) --
	( 69.51, 25.90) --
	( 69.51, 25.90) --
	( 69.51, 25.90) --
	( 69.44, 25.90) --
	( 69.44, 25.90) --
	( 69.44, 25.90) --
	( 69.37, 25.90) --
	( 69.37, 25.90) --
	( 69.37, 25.90) --
	( 69.29, 25.90) --
	( 69.29, 25.90) --
	( 69.29, 25.90) --
	( 69.27, 25.90) --
	( 69.27, 25.90) --
	( 69.27, 25.90) --
	( 69.22, 25.90) --
	( 69.22, 25.90) --
	( 69.22, 25.90) --
	( 69.15, 25.90) --
	( 69.15, 25.90) --
	( 69.15, 25.90) --
	( 69.07, 25.90) --
	( 69.07, 25.90) --
	( 69.07, 25.90) --
	( 69.00, 25.90) --
	( 69.00, 25.90) --
	( 69.00, 25.90) --
	( 68.93, 25.90) --
	( 68.93, 25.90) --
	( 68.93, 25.90) --
	( 68.86, 25.90) --
	( 68.86, 25.90) --
	( 68.86, 25.90) --
	( 68.78, 25.90) --
	( 68.78, 25.90) --
	( 68.78, 25.90) --
	( 68.71, 25.90) --
	( 68.71, 25.90) --
	( 68.71, 25.90) --
	( 68.64, 25.90) --
	( 68.64, 25.90) --
	( 68.64, 25.90) --
	( 68.56, 25.90) --
	( 68.56, 25.90) --
	( 68.56, 25.90) --
	( 68.49, 25.90) --
	( 68.49, 25.90) --
	( 68.49, 25.90) --
	( 68.42, 25.90) --
	( 68.42, 25.90) --
	( 68.42, 25.90) --
	( 68.35, 25.90) --
	( 68.35, 25.90) --
	( 68.35, 25.90) --
	( 68.27, 25.90) --
	( 68.27, 25.90) --
	( 68.27, 25.90) --
	( 68.22, 25.90) --
	( 68.22, 25.90) --
	( 68.22, 25.90) --
	( 68.20, 25.90) --
	( 68.20, 25.90) --
	( 68.20, 25.90) --
	( 68.13, 25.90) --
	( 68.13, 25.90) --
	( 68.13, 25.90) --
	( 68.05, 25.90) --
	( 68.05, 25.90) --
	( 68.05, 25.90) --
	( 67.98, 25.90) --
	( 67.98, 25.90) --
	( 67.98, 25.90) --
	( 67.91, 25.90) --
	( 67.91, 25.90) --
	( 67.91, 25.90) --
	( 67.84, 25.90) --
	( 67.84, 25.90) --
	( 67.84, 25.90) --
	( 67.76, 25.90) --
	( 67.76, 25.90) --
	( 67.76, 25.90) --
	( 67.69, 25.90) --
	( 67.69, 25.90) --
	( 67.69, 25.90) --
	( 67.62, 25.90) --
	( 67.62, 25.90) --
	( 67.62, 25.90) --
	( 67.55, 25.90) --
	( 67.55, 25.90) --
	( 67.55, 25.90) --
	( 67.47, 25.90) --
	( 67.47, 25.90) --
	( 67.47, 25.90) --
	( 67.40, 25.90) --
	( 67.40, 25.90) --
	( 67.40, 25.90) --
	( 67.33, 25.90) --
	( 67.33, 25.90) --
	( 67.33, 25.90) --
	( 67.25, 25.90) --
	( 67.25, 25.90) --
	( 67.25, 25.90) --
	( 67.18, 25.90) --
	( 67.18, 25.90) --
	( 67.18, 25.90) --
	( 67.11, 25.90) --
	( 67.11, 25.90) --
	( 67.11, 25.90) --
	( 67.04, 25.90) --
	( 67.04, 25.90) --
	( 67.04, 25.90) --
	( 66.96, 25.90) --
	( 66.96, 25.90) --
	( 66.96, 25.90) --
	( 66.89, 25.90) --
	( 66.89, 25.90) --
	( 66.89, 25.90) --
	( 66.82, 25.90) --
	( 66.82, 25.90) --
	( 66.82, 25.90) --
	( 66.74, 25.90) --
	( 66.74, 25.90) --
	( 66.74, 25.90) --
	( 66.67, 25.90) --
	( 66.67, 25.90) --
	( 66.67, 25.90) --
	( 66.60, 25.90) --
	( 66.60, 25.90) --
	( 66.60, 25.90) --
	( 66.53, 25.90) --
	( 66.53, 25.90) --
	( 66.53, 25.90) --
	( 66.45, 25.90) --
	( 66.45, 25.90) --
	( 66.45, 25.90) --
	( 66.40, 25.90) --
	( 66.40, 25.90) --
	( 66.40, 25.90) --
	( 66.38, 25.90) --
	( 66.38, 25.90) --
	( 66.38, 25.90) --
	( 66.31, 25.90) --
	( 66.31, 25.90) --
	( 66.31, 25.90) --
	( 66.23, 25.90) --
	( 66.23, 25.90) --
	( 66.23, 25.90) --
	( 66.16, 25.90) --
	( 66.16, 25.90) --
	( 66.16, 25.90) --
	( 66.09, 25.90) --
	( 66.09, 25.90) --
	( 66.09, 25.90) --
	( 66.02, 25.90) --
	( 66.02, 25.90) --
	( 66.02, 25.90) --
	( 65.94, 25.90) --
	( 65.94, 25.90) --
	( 65.94, 25.90) --
	( 65.87, 25.90) --
	( 65.87, 25.90) --
	( 65.87, 25.90) --
	( 65.80, 25.90) --
	( 65.80, 25.90) --
	( 65.80, 25.90) --
	( 65.72, 25.90) --
	( 65.72, 25.90) --
	( 65.72, 25.90) --
	( 65.65, 25.90) --
	( 65.65, 25.90) --
	( 65.65, 25.90) --
	( 65.58, 25.90) --
	( 65.58, 25.90) --
	( 65.58, 25.90) --
	( 65.50, 25.90) --
	( 65.50, 25.90) --
	( 65.50, 25.90) --
	( 65.43, 25.90) --
	( 65.43, 25.90) --
	( 65.43, 25.90) --
	( 65.36, 25.90) --
	( 65.36, 25.90) --
	( 65.36, 25.90) --
	( 65.29, 25.90) --
	( 65.29, 25.90) --
	( 65.29, 25.90) --
	( 65.21, 25.90) --
	( 65.21, 25.90) --
	( 65.21, 25.90) --
	( 65.14, 25.90) --
	( 65.14, 25.90) --
	( 65.14, 25.90) --
	( 65.07, 25.90) --
	( 65.07, 25.90) --
	( 65.07, 25.90) --
	( 64.99, 25.90) --
	( 64.99, 25.90) --
	( 64.99, 25.90) --
	( 64.92, 25.90) --
	( 64.92, 25.90) --
	( 64.92, 25.90) --
	( 64.85, 25.90) --
	( 64.85, 25.90) --
	( 64.85, 25.90) --
	( 64.77, 25.90) --
	( 64.77, 25.90) --
	( 64.77, 25.90) --
	( 64.70, 25.90) --
	( 64.70, 25.90) --
	( 64.70, 25.90) --
	( 64.63, 25.90) --
	( 64.63, 25.90) --
	( 64.63, 25.90) --
	( 64.57, 25.90) --
	( 64.57, 25.90) --
	( 64.57, 25.90) --
	( 64.56, 25.90) --
	( 64.56, 25.90) --
	( 64.56, 25.90) --
	( 64.48, 25.90) --
	( 64.48, 25.90) --
	( 64.48, 25.90) --
	( 64.41, 25.90) --
	( 64.41, 25.90) --
	( 64.41, 25.90) --
	( 64.34, 25.90) --
	( 64.34, 25.90) --
	( 64.34, 25.90) --
	( 64.26, 25.90) --
	( 64.26, 25.90) --
	( 64.26, 25.90) --
	( 64.19, 25.90) --
	( 64.19, 25.90) --
	( 64.19, 25.90) --
	( 64.12, 25.90) --
	( 64.12, 25.90) --
	( 64.12, 25.90) --
	( 64.04, 25.90) --
	( 64.04, 25.90) --
	( 64.04, 25.90) --
	( 63.97, 25.90) --
	( 63.97, 25.90) --
	( 63.97, 25.90) --
	( 63.90, 25.90) --
	( 63.90, 25.90) --
	( 63.90, 25.90) --
	( 63.83, 25.90) --
	( 63.83, 25.90) --
	( 63.83, 25.90) --
	( 63.75, 25.90) --
	( 63.75, 25.90) --
	( 63.75, 25.90) --
	( 63.68, 25.90) --
	( 63.68, 25.90) --
	( 63.68, 25.90) --
	( 63.61, 25.90) --
	( 63.61, 25.90) --
	( 63.61, 25.90) --
	( 63.53, 25.90) --
	( 63.53, 25.90) --
	( 63.53, 25.90) --
	( 63.46, 25.90) --
	( 63.46, 25.90) --
	( 63.46, 25.90) --
	( 63.39, 25.90) --
	( 63.39, 25.90) --
	( 63.39, 25.90) --
	( 63.31, 25.90) --
	( 63.31, 25.90) --
	( 63.31, 25.90) --
	( 63.24, 25.90) --
	( 63.24, 25.90) --
	( 63.24, 25.90) --
	( 63.23, 25.90) --
	( 63.23, 25.90) --
	( 63.23, 25.90) --
	( 63.17, 25.90) --
	( 63.17, 25.90) --
	( 63.17, 25.90) --
	( 63.09, 25.90) --
	( 63.09, 25.90) --
	( 63.09, 25.90) --
	( 63.02, 25.90) --
	( 63.02, 25.90) --
	( 63.02, 25.90) --
	( 62.95, 25.90) --
	( 62.95, 25.90) --
	( 62.95, 25.90) --
	( 62.87, 25.90) --
	( 62.87, 25.90) --
	( 62.87, 25.90) --
	( 62.80, 25.90) --
	( 62.80, 25.90) --
	( 62.80, 25.90) --
	( 62.73, 25.90) --
	( 62.73, 25.90) --
	( 62.73, 25.90) --
	( 62.65, 25.90) --
	( 62.65, 25.90) --
	( 62.65, 25.90) --
	( 62.58, 25.90) --
	( 62.58, 25.90) --
	( 62.58, 25.90) --
	( 62.51, 25.90) --
	( 62.51, 25.90) --
	( 62.51, 25.90) --
	( 62.44, 25.90) --
	( 62.44, 25.90) --
	( 62.44, 25.90) --
	( 62.36, 25.90) --
	( 62.36, 25.90) --
	( 62.36, 25.90) --
	( 62.29, 25.90) --
	( 62.29, 25.90) --
	( 62.29, 25.90) --
	( 62.22, 25.90) --
	( 62.22, 25.90) --
	( 62.22, 25.90) --
	( 62.14, 25.90) --
	( 62.14, 25.90) --
	( 62.14, 25.90) --
	( 62.07, 25.90) --
	( 62.07, 25.90) --
	( 62.07, 25.90) --
	( 62.00, 25.90) --
	( 62.00, 25.90) --
	( 62.00, 25.90) --
	( 61.92, 25.90) --
	( 61.92, 25.90) --
	( 61.92, 25.90) --
	( 61.85, 25.90) --
	( 61.85, 25.90) --
	( 61.85, 25.90) --
	( 61.78, 25.90) --
	( 61.78, 25.90) --
	( 61.78, 25.90) --
	( 61.70, 25.90) --
	( 61.70, 25.90) --
	( 61.70, 25.90) --
	( 61.63, 25.90) --
	( 61.63, 25.90) --
	( 61.63, 25.90) --
	( 61.56, 25.90) --
	( 61.56, 25.90) --
	( 61.56, 25.90) --
	( 61.48, 25.90) --
	( 61.48, 25.90) --
	( 61.48, 25.90) --
	( 61.41, 25.90) --
	( 61.41, 25.90) --
	( 61.41, 25.90) --
	( 61.41, 25.90) --
	( 61.41, 25.90) --
	( 61.41, 25.90) --
	( 61.34, 25.90) --
	( 61.34, 25.90) --
	( 61.34, 25.90) --
	( 61.26, 25.90) --
	( 61.26, 25.90) --
	( 61.26, 25.90) --
	( 61.19, 25.90) --
	( 61.19, 25.90) --
	( 61.19, 25.90) --
	( 61.12, 25.90) --
	( 61.12, 25.90) --
	( 61.12, 25.90) --
	( 61.04, 25.90) --
	( 61.04, 25.90) --
	( 61.04, 25.90) --
	( 60.97, 25.90) --
	( 60.97, 25.90) --
	( 60.97, 25.90) --
	( 60.90, 25.90) --
	( 60.90, 25.90) --
	( 60.90, 25.90) --
	( 60.83, 25.90) --
	( 60.83, 25.90) --
	( 60.83, 25.90) --
	( 60.75, 25.90) --
	( 60.75, 25.90) --
	( 60.75, 25.90) --
	( 60.68, 25.90) --
	( 60.68, 25.90) --
	( 60.68, 25.90) --
	( 60.61, 25.90) --
	( 60.61, 25.90) --
	( 60.61, 25.90) --
	( 60.53, 25.90) --
	( 60.53, 25.90) --
	( 60.53, 25.90) --
	( 60.46, 25.90) --
	( 60.46, 25.90) --
	( 60.46, 25.90) --
	( 60.45, 25.90) --
	( 60.45, 25.90) --
	( 60.45, 25.90) --
	( 60.39, 25.90) --
	( 60.39, 25.90) --
	( 60.39, 25.90) --
	( 60.31, 25.90) --
	( 60.31, 25.90) --
	( 60.31, 25.90) --
	( 60.24, 25.90) --
	( 60.24, 25.90) --
	( 60.24, 25.90) --
	( 60.17, 25.90) --
	( 60.17, 25.90) --
	( 60.17, 25.90) --
	( 60.09, 25.90) --
	( 60.09, 25.90) --
	( 60.09, 25.90) --
	( 60.02, 25.90) --
	( 60.02, 25.90) --
	( 60.02, 25.90) --
	( 59.95, 25.90) --
	( 59.95, 25.90) --
	( 59.95, 25.90) --
	( 59.87, 25.90) --
	( 59.87, 25.90) --
	( 59.87, 25.90) --
	( 59.80, 25.90) --
	( 59.80, 25.90) --
	( 59.80, 25.90) --
	( 59.73, 25.90) --
	( 59.73, 25.90) --
	( 59.73, 25.90) --
	( 59.65, 25.90) --
	( 59.65, 25.90) --
	( 59.65, 25.90) --
	( 59.58, 25.90) --
	( 59.58, 25.90) --
	( 59.58, 25.90) --
	( 59.51, 25.90) --
	( 59.51, 25.90) --
	( 59.51, 25.90) --
	( 59.50, 25.90) --
	( 59.50, 25.90) --
	( 59.50, 25.90) --
	( 59.43, 25.90) --
	( 59.43, 25.90) --
	( 59.43, 25.90) --
	( 59.36, 25.90) --
	( 59.36, 25.90) --
	( 59.36, 25.90) --
	( 59.29, 25.90) --
	( 59.29, 25.90) --
	( 59.29, 25.90) --
	( 59.21, 25.90) --
	( 59.21, 25.90) --
	( 59.21, 25.90) --
	( 59.14, 25.90) --
	( 59.14, 25.90) --
	( 59.14, 25.90) --
	( 59.07, 25.90) --
	( 59.07, 25.90) --
	( 59.07, 25.90) --
	( 58.99, 25.90) --
	( 58.99, 25.90) --
	( 58.99, 25.90) --
	( 58.92, 25.90) --
	( 58.92, 25.90) --
	( 58.92, 25.90) --
	( 58.92, 25.90) --
	( 58.92, 25.90) --
	( 58.92, 25.90) --
	( 58.85, 25.90) --
	( 58.85, 25.90) --
	( 58.85, 25.90) --
	( 58.77, 25.90) --
	( 58.77, 25.90) --
	( 58.77, 25.90) --
	( 58.70, 25.90) --
	( 58.70, 25.90) --
	( 58.70, 25.90) --
	( 58.63, 25.90) --
	( 58.63, 25.90) --
	( 58.63, 25.90) --
	( 58.55, 25.90) --
	( 58.55, 25.90) --
	( 58.55, 25.90) --
	( 58.48, 25.90) --
	( 58.48, 25.90) --
	( 58.48, 25.90) --
	( 58.41, 25.90) --
	( 58.41, 25.90) --
	( 58.41, 25.90) --
	( 58.35, 25.90) --
	( 58.35, 25.90) --
	( 58.35, 25.90) --
	( 58.33, 25.90) --
	( 58.33, 25.90) --
	( 58.33, 25.90) --
	( 58.26, 25.90) --
	( 58.26, 25.90) --
	( 58.26, 25.90) --
	( 58.18, 25.90) --
	( 58.18, 25.90) --
	( 58.18, 25.90) --
	( 58.11, 25.90) --
	( 58.11, 25.90) --
	( 58.11, 25.90) --
	( 58.04, 25.90) --
	( 58.04, 25.90) --
	( 58.04, 25.90) --
	( 57.97, 25.90) --
	( 57.97, 25.90) --
	( 57.97, 25.90) --
	( 57.89, 25.90) --
	( 57.89, 25.90) --
	( 57.89, 25.90) --
	( 57.87, 25.90) --
	( 57.87, 25.90) --
	( 57.87, 25.90) --
	( 57.82, 25.90) --
	( 57.82, 25.90) --
	( 57.82, 25.90) --
	( 57.74, 25.90) --
	( 57.74, 25.90) --
	( 57.74, 25.90) --
	( 57.67, 25.90) --
	( 57.67, 25.90) --
	( 57.67, 25.90) --
	( 57.60, 25.90) --
	( 57.60, 25.90) --
	( 57.60, 25.90) --
	( 57.52, 25.90) --
	( 57.52, 25.90) --
	( 57.52, 25.90) --
	( 57.45, 25.90) --
	( 57.45, 25.90) --
	( 57.45, 25.90) --
	( 57.38, 25.90) --
	( 57.38, 25.90) --
	( 57.38, 25.90) --
	( 57.30, 25.90) --
	( 57.30, 25.90) --
	( 57.30, 25.90) --
	( 57.23, 25.90) --
	( 57.23, 25.90) --
	( 57.23, 25.90) --
	( 57.20, 25.90) --
	( 57.20, 25.90) --
	( 57.20, 25.90) --
	( 57.16, 25.90) --
	( 57.16, 25.90) --
	( 57.16, 25.90) --
	( 57.08, 25.90) --
	( 57.08, 25.90) --
	( 57.08, 25.90) --
	( 57.01, 25.90) --
	( 57.01, 25.90) --
	( 57.01, 25.90) --
	( 56.94, 25.90) --
	( 56.94, 25.90) --
	( 56.94, 25.90) --
	( 56.91, 25.90) --
	( 56.91, 25.90) --
	( 56.91, 25.90) --
	( 56.86, 25.90) --
	( 56.86, 25.90) --
	( 56.86, 25.90) --
	( 56.79, 25.90) --
	( 56.79, 25.90) --
	( 56.79, 25.90) --
	( 56.72, 25.90) --
	( 56.72, 25.90) --
	( 56.72, 25.90) --
	( 56.64, 25.90) --
	( 56.64, 25.90) --
	( 56.64, 25.90) --
	( 56.57, 25.90) --
	( 56.57, 25.90) --
	( 56.57, 25.90) --
	( 56.50, 25.90) --
	( 56.50, 25.90) --
	( 56.50, 25.90) --
	( 56.43, 25.90) --
	( 56.43, 25.90) --
	( 56.43, 25.90) --
	( 56.42, 25.90) --
	( 56.42, 25.90) --
	( 56.42, 25.90) --
	( 56.35, 25.90) --
	( 56.35, 25.90) --
	( 56.35, 25.90) --
	( 56.27, 25.90) --
	( 56.27, 25.90) --
	( 56.27, 25.90) --
	( 56.20, 25.90) --
	( 56.20, 25.90) --
	( 56.20, 25.90) --
	( 56.14, 25.90) --
	( 56.14, 25.90) --
	( 56.14, 25.90) --
	( 56.13, 25.90) --
	( 56.13, 25.90) --
	( 56.13, 25.90) --
	( 56.06, 25.90) --
	( 56.06, 25.90) --
	( 56.06, 25.90) --
	( 55.98, 25.90) --
	( 55.98, 25.90) --
	( 55.98, 25.90) --
	( 55.91, 25.90) --
	( 55.91, 25.90) --
	( 55.91, 25.90) --
	( 55.83, 25.90) --
	( 55.83, 25.90) --
	( 55.83, 25.90) --
	( 55.76, 25.90) --
	( 55.76, 25.90) --
	( 55.76, 25.90) --
	( 55.69, 25.90) --
	( 55.69, 25.90) --
	( 55.69, 25.90) --
	( 55.61, 25.90) --
	( 55.61, 25.90) --
	( 55.61, 25.90) --
	( 55.57, 25.90) --
	( 55.57, 25.90) --
	( 55.57, 25.90) --
	( 55.54, 25.90) --
	( 55.54, 25.90) --
	( 55.54, 25.90) --
	( 55.47, 25.90) --
	( 55.47, 25.90) --
	( 55.47, 25.90) --
	( 55.39, 25.90) --
	( 55.39, 25.90) --
	( 55.39, 25.90) --
	( 55.32, 25.90) --
	( 55.32, 25.90) --
	( 55.32, 25.90) --
	( 55.28, 25.90) --
	( 55.28, 25.90) --
	( 55.28, 25.90) --
	( 55.25, 25.90) --
	( 55.25, 25.90) --
	( 55.25, 25.90) --
	( 55.17, 25.90) --
	( 55.17, 25.90) --
	( 55.17, 25.90) --
	( 55.10, 25.90) --
	( 55.10, 25.90) --
	( 55.10, 25.90) --
	( 55.02, 25.90) --
	( 55.02, 25.90) --
	( 55.02, 25.90) --
	( 54.95, 25.90) --
	( 54.95, 25.90) --
	( 54.95, 25.90) --
	( 54.88, 25.90) --
	( 54.88, 25.90) --
	( 54.88, 25.90) --
	( 54.80, 25.90) --
	( 54.80, 25.90) --
	( 54.80, 25.90) --
	( 54.73, 25.90) --
	( 54.73, 25.90) --
	( 54.73, 25.90) --
	( 54.66, 25.90) --
	( 54.66, 25.90) --
	( 54.66, 25.90) --
	( 54.58, 25.90) --
	( 54.58, 25.90) --
	( 54.58, 25.90) --
	( 54.51, 25.90) --
	( 54.51, 25.90) --
	( 54.51, 25.90) --
	( 54.51, 25.90) --
	( 54.51, 25.90) --
	( 54.51, 25.90) --
	( 54.44, 25.90) --
	( 54.44, 25.90) --
	( 54.44, 25.90) --
	( 54.36, 25.90) --
	( 54.36, 25.90) --
	( 54.36, 25.90) --
	( 54.29, 25.90) --
	( 54.29, 25.90) --
	( 54.29, 25.90) --
	( 54.21, 25.90) --
	( 54.21, 25.90) --
	( 54.21, 25.90) --
	( 54.14, 25.90) --
	( 54.14, 25.90) --
	( 54.14, 25.90) --
	( 54.07, 25.90) --
	( 54.07, 25.90) --
	( 54.07, 25.90) --
	( 53.99, 25.90) --
	( 53.99, 25.90) --
	( 53.99, 25.90) --
	( 53.92, 25.90) --
	( 53.92, 25.90) --
	( 53.92, 25.90) --
	( 53.85, 25.90) --
	( 53.85, 25.90) --
	( 53.85, 25.90) --
	( 53.77, 25.90) --
	( 53.77, 25.90) --
	( 53.77, 25.90) --
	( 53.75, 25.90) --
	( 53.75, 25.90) --
	( 53.75, 25.90) --
	( 53.70, 25.90) --
	( 53.70, 25.90) --
	( 53.70, 25.90) --
	( 53.62, 25.90) --
	( 53.62, 25.90) --
	( 53.62, 25.90) --
	( 53.55, 25.90) --
	( 53.55, 25.90) --
	( 53.55, 25.90) --
	( 53.48, 25.90) --
	( 53.48, 25.90) --
	( 53.48, 25.90) --
	( 53.41, 25.90) --
	( 53.41, 25.90) --
	( 53.41, 25.90) --
	( 53.33, 25.90) --
	( 53.33, 25.90) --
	( 53.33, 25.90) --
	( 53.26, 25.90) --
	( 53.26, 25.90) --
	( 53.26, 25.90) --
	( 53.18, 25.90) --
	( 53.18, 25.90) --
	( 53.18, 25.90) --
	( 53.17, 25.90) --
	( 53.17, 25.90) --
	( 53.17, 25.90) --
	( 53.11, 25.90) --
	( 53.11, 25.90) --
	( 53.11, 25.90) --
	( 53.04, 25.90) --
	( 53.04, 25.90) --
	( 53.04, 25.90) --
	( 52.96, 25.90) --
	( 52.96, 25.90) --
	( 52.96, 25.90) --
	( 52.89, 25.90) --
	( 52.89, 25.90) --
	( 52.89, 25.90) --
	( 52.81, 25.90) --
	( 52.81, 25.90) --
	( 52.81, 25.90) --
	( 52.74, 25.90) --
	( 52.74, 25.90) --
	( 52.74, 25.90) --
	( 52.69, 25.90) --
	( 52.69, 25.90) --
	( 52.69, 25.90) --
	( 52.67, 25.90) --
	( 52.67, 25.90) --
	( 52.67, 25.90) --
	( 52.59, 25.90) --
	( 52.59, 25.90) --
	( 52.59, 25.90) --
	( 52.52, 25.90) --
	( 52.52, 25.90) --
	( 52.52, 25.90) --
	( 52.45, 25.90) --
	( 52.45, 25.90) --
	( 52.45, 25.90) --
	( 52.37, 25.90) --
	( 52.37, 25.90) --
	( 52.37, 25.90) --
	( 52.30, 25.90) --
	( 52.30, 25.90) --
	( 52.30, 25.90) --
	( 52.22, 25.90) --
	( 52.22, 25.90) --
	( 52.22, 25.90) --
	( 52.21, 25.90) --
	( 52.21, 25.90) --
	( 52.21, 25.90) --
	( 52.15, 25.90) --
	( 52.15, 25.90) --
	( 52.15, 25.90) --
	( 52.08, 25.90) --
	( 52.08, 25.90) --
	( 52.08, 25.90) --
	( 52.00, 25.90) --
	( 52.00, 25.90) --
	( 52.00, 25.90) --
	( 51.93, 25.90) --
	( 51.93, 25.90) --
	( 51.93, 25.90) --
	( 51.86, 25.90) --
	( 51.86, 25.90) --
	( 51.86, 25.90) --
	( 51.78, 25.90) --
	( 51.78, 25.90) --
	( 51.78, 25.90) --
	( 51.73, 25.90) --
	( 51.73, 25.90) --
	( 51.73, 25.90) --
	( 51.71, 25.90) --
	( 51.71, 25.90) --
	( 51.71, 25.90) --
	( 51.63, 25.90) --
	( 51.63, 25.90) --
	( 51.63, 25.90) --
	( 51.56, 25.90) --
	( 51.56, 25.90) --
	( 51.56, 25.90) --
	( 51.49, 25.90) --
	( 51.49, 25.90) --
	( 51.49, 25.90) --
	( 51.41, 25.90) --
	( 51.41, 25.90) --
	( 51.41, 25.90) --
	( 51.35, 25.90) --
	( 51.35, 25.90) --
	( 51.35, 25.90) --
	( 51.34, 25.90) --
	( 51.34, 25.90) --
	( 51.34, 25.90) --
	( 51.27, 25.90) --
	( 51.27, 25.90) --
	( 51.27, 25.90) --
	( 51.19, 25.90) --
	( 51.19, 25.90) --
	( 51.19, 25.90) --
	( 51.12, 25.90) --
	( 51.12, 25.90) --
	( 51.12, 25.90) --
	( 51.04, 25.90) --
	( 51.04, 25.90) --
	( 51.04, 25.90) --
	( 50.97, 25.90) --
	( 50.97, 25.90) --
	( 50.97, 25.90) --
	( 50.97, 25.90) --
	( 50.97, 25.90) --
	( 50.97, 25.90) --
	( 50.90, 25.90) --
	( 50.90, 25.90) --
	( 50.90, 25.90) --
	( 50.87, 25.90) --
	( 50.87, 25.90) --
	( 50.87, 25.90) --
	( 50.82, 25.90) --
	( 50.82, 25.90) --
	( 50.82, 25.90) --
	( 50.75, 25.90) --
	( 50.75, 25.90) --
	( 50.75, 25.90) --
	( 50.67, 25.90) --
	( 50.67, 25.90) --
	( 50.67, 25.90) --
	( 50.60, 25.90) --
	( 50.60, 25.90) --
	( 50.60, 25.90) --
	( 50.53, 25.90) --
	( 50.53, 25.90) --
	( 50.53, 25.90) --
	( 50.49, 25.90) --
	( 50.49, 25.90) --
	( 50.49, 25.90) --
	( 50.45, 25.90) --
	( 50.45, 25.90) --
	( 50.45, 25.90) --
	( 50.38, 25.90) --
	( 50.38, 25.90) --
	( 50.38, 25.90) --
	( 50.30, 25.90) --
	( 50.30, 25.90) --
	( 50.30, 25.90) --
	( 50.23, 25.90) --
	( 50.23, 25.90) --
	( 50.23, 25.90) --
	( 50.16, 25.90) --
	( 50.16, 25.90) --
	( 50.16, 25.90) --
	( 50.10, 25.90) --
	( 50.10, 25.90) --
	( 50.10, 25.90) --
	( 50.08, 25.90) --
	( 50.08, 25.90) --
	( 50.08, 25.90) --
	( 50.01, 25.90) --
	( 50.01, 25.90) --
	( 50.01, 25.90) --
	( 49.94, 25.90) --
	( 49.94, 25.90) --
	( 49.94, 25.90) --
	( 49.91, 25.90) --
	( 49.91, 25.90) --
	( 49.91, 25.90) --
	( 49.86, 25.90) --
	( 49.86, 25.90) --
	( 49.86, 25.90) --
	( 49.79, 25.90) --
	( 49.79, 25.90) --
	( 49.79, 25.90) --
	( 49.72, 25.90) --
	( 49.72, 25.90) --
	( 49.72, 25.90) --
	( 49.71, 25.90) --
	( 49.71, 25.90) --
	( 49.71, 25.90) --
	( 49.64, 25.90) --
	( 49.64, 25.90) --
	( 49.64, 25.90) --
	( 49.57, 25.90) --
	( 49.57, 25.90) --
	( 49.57, 25.90) --
	( 49.53, 25.90) --
	( 49.53, 25.90) --
	( 49.53, 25.90) --
	( 49.49, 25.90) --
	( 49.49, 25.90) --
	( 49.49, 25.90) --
	( 49.42, 25.90) --
	( 49.42, 25.90) --
	( 49.42, 25.90) --
	( 49.34, 25.90) --
	( 49.34, 25.90) --
	( 49.34, 25.90) --
	( 49.34, 25.90) --
	( 49.34, 25.90) --
	( 49.34, 25.90) --
	( 49.27, 25.90) --
	( 49.27, 25.90) --
	( 49.27, 25.90) --
	( 49.20, 25.90) --
	( 49.20, 25.90) --
	( 49.20, 25.90) --
	( 49.12, 25.90) --
	( 49.12, 25.90) --
	( 49.12, 25.90) --
	( 49.05, 25.90) --
	( 49.05, 25.90) --
	( 49.05, 25.90) --
	( 49.05, 25.90) --
	( 49.05, 25.90) --
	( 49.05, 25.90) --
	( 48.97, 25.90) --
	( 48.97, 25.90) --
	( 48.97, 25.90) --
	( 48.95, 25.90) --
	( 48.95, 25.90) --
	( 48.95, 25.90) --
	( 48.90, 25.90) --
	( 48.90, 25.90) --
	( 48.90, 25.90) --
	( 48.83, 25.90) --
	( 48.83, 25.90) --
	( 48.83, 25.90) --
	( 48.76, 25.90) --
	( 48.76, 25.90) --
	( 48.76, 25.90) --
	( 48.75, 25.90) --
	( 48.75, 25.90) --
	( 48.75, 25.90) --
	( 48.68, 25.90) --
	( 48.68, 25.90) --
	( 48.68, 25.90) --
	( 48.67, 25.90) --
	( 48.67, 25.90) --
	( 48.67, 25.90) --
	( 48.60, 25.90) --
	( 48.60, 25.90) --
	( 48.60, 25.90) --
	( 48.53, 25.90) --
	( 48.53, 25.90) --
	( 48.53, 25.90) --
	( 48.47, 25.90) --
	( 48.47, 25.90) --
	( 48.47, 25.90) --
	( 48.46, 25.90) --
	( 48.46, 25.90) --
	( 48.46, 25.90) --
	( 48.38, 25.90) --
	( 48.38, 25.90) --
	( 48.38, 25.90) --
	( 48.31, 25.90) --
	( 48.31, 25.90) --
	( 48.31, 25.90) --
	( 48.28, 25.90) --
	( 48.28, 25.90) --
	( 48.28, 25.90) --
	( 48.23, 25.90) --
	( 48.23, 25.90) --
	( 48.23, 25.90) --
	( 48.19, 25.90) --
	( 48.19, 25.90) --
	( 48.19, 25.90) --
	( 48.16, 25.90) --
	( 48.16, 25.90) --
	( 48.16, 25.90) --
	( 48.09, 25.90) --
	( 48.09, 25.90) --
	( 48.09, 25.90) --
	( 48.01, 25.90) --
	( 48.01, 25.90) --
	( 48.01, 25.90) --
	( 48.00, 25.90) --
	( 48.00, 25.90) --
	( 48.00, 25.90) --
	( 47.94, 25.90) --
	( 47.94, 25.90) --
	( 47.94, 25.90) --
	( 47.90, 25.90) --
	( 47.90, 25.90) --
	( 47.90, 25.90) --
	( 47.86, 25.90) --
	( 47.86, 25.90) --
	( 47.86, 25.90) --
	( 47.80, 25.90) --
	( 47.80, 25.90) --
	( 47.80, 25.90) --
	( 47.79, 25.90) --
	( 47.79, 25.90) --
	( 47.79, 25.90) --
	( 47.72, 25.90) --
	( 47.72, 25.90) --
	( 47.72, 25.90) --
	( 47.64, 25.90) --
	( 47.64, 25.90) --
	( 47.64, 25.90) --
	( 47.61, 25.90) --
	( 47.61, 25.90) --
	( 47.61, 25.90) --
	( 47.57, 25.90) --
	( 47.57, 25.90) --
	( 47.57, 25.90) --
	( 47.52, 25.90) --
	( 47.52, 25.90) --
	( 47.52, 25.90) --
	( 47.50, 25.90) --
	( 47.50, 25.90) --
	( 47.50, 25.90) --
	( 47.42, 25.90) --
	( 47.42, 25.90) --
	( 47.42, 25.90) --
	( 47.35, 25.90) --
	( 47.35, 25.90) --
	( 47.35, 25.90) --
	( 47.32, 25.90) --
	( 47.32, 25.90) --
	( 47.32, 25.90) --
	( 47.27, 25.90) --
	( 47.27, 25.90) --
	( 47.27, 25.90) --
	( 47.23, 25.90) --
	( 47.23, 25.90) --
	( 47.23, 25.90) --
	( 47.20, 25.90) --
	( 47.20, 25.90) --
	( 47.20, 25.90) --
	( 47.13, 25.90) --
	( 47.13, 25.90) --
	( 47.13, 25.90) --
	( 47.13, 25.90) --
	( 47.13, 25.90) --
	( 47.13, 25.90) --
	( 47.05, 25.90) --
	( 47.05, 25.90) --
	( 47.05, 25.90) --
	( 46.98, 25.90) --
	( 46.98, 25.90) --
	( 46.98, 25.90) --
	( 46.94, 25.90) --
	( 46.94, 25.90) --
	( 46.94, 25.90) --
	( 46.90, 25.90) --
	( 46.90, 25.90) --
	( 46.90, 25.90) --
	( 46.85, 25.90) --
	( 46.85, 25.90) --
	( 46.85, 25.90) --
	( 46.83, 25.90) --
	( 46.83, 25.90) --
	( 46.83, 25.90) --
	( 46.75, 25.90) --
	( 46.75, 25.90) --
	( 46.75, 25.90) --
	( 46.75, 25.90) --
	( 46.75, 25.90) --
	( 46.75, 25.90) --
	( 46.68, 25.90) --
	( 46.68, 25.90) --
	( 46.68, 25.90) --
	( 46.61, 25.90) --
	( 46.61, 25.90) --
	( 46.61, 25.90) --
	( 46.53, 25.90) --
	( 46.53, 25.90) --
	( 46.53, 25.90) --
	( 46.46, 25.90) --
	( 46.46, 25.90) --
	( 46.46, 25.90) --
	( 46.46, 25.90) --
	( 46.46, 25.90) --
	( 46.46, 25.90) --
	( 46.38, 25.90) --
	( 46.38, 25.90) --
	( 46.38, 25.90) --
	( 46.31, 25.90) --
	( 46.31, 25.90) --
	( 46.31, 25.90) --
	( 46.24, 25.90) --
	( 46.24, 25.90) --
	( 46.24, 25.90) --
	( 46.17, 25.90) --
	( 46.17, 25.90) --
	( 46.17, 25.90) --
	( 46.16, 25.90) --
	( 46.16, 25.90) --
	( 46.16, 25.90) --
	( 46.09, 25.90) --
	( 46.09, 25.90) --
	( 46.09, 25.90) --
	( 46.08, 25.90) --
	( 46.08, 25.90) --
	( 46.08, 25.90) --
	( 46.01, 25.90) --
	( 46.01, 25.90) --
	( 46.01, 25.90) --
	( 45.98, 25.90) --
	( 45.98, 25.90) --
	( 45.98, 25.90) --
	( 45.94, 25.90) --
	( 45.94, 25.90) --
	( 45.94, 25.90) --
	( 45.87, 25.90) --
	( 45.87, 25.90) --
	( 45.87, 25.90) --
	( 45.79, 25.90) --
	( 45.79, 25.90) --
	( 45.79, 25.90) --
	( 45.79, 25.90) --
	( 45.79, 25.90) --
	( 45.79, 25.90) --
	( 45.72, 25.90) --
	( 45.72, 25.90) --
	( 45.72, 25.90) --
	( 45.70, 25.90) --
	( 45.70, 25.90) --
	( 45.70, 25.90) --
	( 45.64, 25.90) --
	( 45.64, 25.90) --
	( 45.64, 25.90) --
	( 45.60, 25.90) --
	( 45.60, 25.90) --
	( 45.60, 25.90) --
	( 45.57, 25.90) --
	( 45.57, 25.90) --
	( 45.57, 25.90) --
	( 45.50, 25.90) --
	( 45.50, 25.90) --
	( 45.50, 25.90) --
	( 45.49, 25.90) --
	( 45.49, 25.90) --
	( 45.49, 25.90) --
	( 45.42, 25.90) --
	( 45.42, 25.90) --
	( 45.42, 25.90) --
	( 45.41, 25.90) --
	( 45.41, 25.90) --
	( 45.41, 25.90) --
	( 45.35, 25.90) --
	( 45.35, 25.90) --
	( 45.35, 25.90) --
	( 45.31, 25.90) --
	( 45.31, 25.90) --
	( 45.31, 25.90) --
	( 45.27, 25.90) --
	( 45.27, 25.90) --
	( 45.27, 25.90) --
	( 45.20, 25.90) --
	( 45.20, 25.90) --
	( 45.20, 25.90) --
	( 45.12, 25.90) --
	( 45.12, 25.90) --
	( 45.12, 25.90) --
	( 45.05, 25.90) --
	( 45.05, 25.90) --
	( 45.05, 25.90) --
	( 45.02, 25.90) --
	( 45.02, 25.90) --
	( 45.02, 25.90) --
	( 44.98, 25.90) --
	( 44.98, 25.90) --
	( 44.98, 25.90) --
	( 44.93, 25.90) --
	( 44.93, 25.90) --
	( 44.93, 25.90) --
	( 44.90, 25.90) --
	( 44.90, 25.90) --
	( 44.90, 25.90) --
	( 44.83, 25.90) --
	( 44.83, 25.90) --
	( 44.83, 25.90) --
	( 44.75, 25.90) --
	( 44.75, 25.90) --
	( 44.75, 25.90) --
	( 44.74, 25.90) --
	( 44.74, 25.90) --
	( 44.74, 25.90) --
	( 44.68, 25.90) --
	( 44.68, 25.90) --
	( 44.68, 25.90) --
	( 44.64, 25.90) --
	( 44.64, 25.90) --
	( 44.64, 25.90) --
	( 44.60, 25.90) --
	( 44.60, 25.90) --
	( 44.60, 25.90) --
	( 44.53, 25.90) --
	( 44.53, 25.90) --
	( 44.53, 25.90) --
	( 44.46, 25.90) --
	( 44.46, 25.90) --
	( 44.46, 25.90) --
	( 44.45, 25.90) --
	( 44.45, 25.90) --
	( 44.45, 25.90) --
	( 44.38, 25.90) --
	( 44.38, 25.90) --
	( 44.38, 25.90) --
	( 44.31, 25.90) --
	( 44.31, 25.90) --
	( 44.31, 25.90) --
	( 44.23, 25.90) --
	( 44.23, 25.90) --
	( 44.23, 25.90) --
	( 44.16, 25.90) --
	( 44.16, 25.90) --
	( 44.16, 25.90) --
	( 44.16, 25.90) --
	( 44.16, 25.90) --
	( 44.16, 25.90) --
	( 44.08, 25.90) --
	( 44.08, 25.90) --
	( 44.08, 25.90) --
	( 44.01, 25.90) --
	( 44.01, 25.90) --
	( 44.01, 25.90) --
	( 43.97, 25.90) --
	( 43.97, 25.90) --
	( 43.97, 25.90) --
	( 43.94, 25.90) --
	( 43.94, 25.90) --
	( 43.94, 25.90) --
	( 43.86, 25.90) --
	( 43.86, 25.90) --
	( 43.86, 25.90) --
	( 43.79, 25.90) --
	( 43.79, 25.90) --
	( 43.79, 25.90) --
	( 43.71, 25.90) --
	( 43.71, 25.90) --
	( 43.71, 25.90) --
	( 43.64, 25.90) --
	( 43.64, 25.90) --
	( 43.64, 25.90) --
	( 43.59, 25.90) --
	( 43.59, 25.90) --
	( 43.59, 25.90) --
	( 43.56, 25.90) --
	( 43.56, 25.90) --
	( 43.56, 25.90) --
	( 43.49, 25.90) --
	( 43.49, 25.90) --
	( 43.49, 25.90) --
	( 43.49, 25.90) --
	( 43.49, 25.90) --
	( 43.49, 25.90) --
	( 43.42, 25.90) --
	( 43.42, 25.90) --
	( 43.42, 25.90) --
	( 43.34, 25.90) --
	( 43.34, 25.90) --
	( 43.34, 25.90) --
	( 43.30, 25.90) --
	( 43.30, 25.90) --
	( 43.30, 25.90) --
	( 43.27, 25.90) --
	( 43.27, 25.90) --
	( 43.27, 25.90) --
	( 43.19, 25.90) --
	( 43.19, 25.90) --
	( 43.19, 25.90) --
	( 43.12, 25.90) --
	( 43.12, 25.90) --
	( 43.12, 25.90) --
	( 43.04, 25.90) --
	( 43.04, 25.90) --
	( 43.04, 25.90) --
	( 43.01, 25.90) --
	( 43.01, 25.90) --
	( 43.01, 25.90) --
	( 42.97, 25.90) --
	( 42.97, 25.90) --
	( 42.97, 25.90) --
	( 42.90, 25.90) --
	( 42.90, 25.90) --
	( 42.90, 25.90) --
	( 42.82, 25.90) --
	( 42.82, 25.90) --
	( 42.82, 25.90) --
	( 42.82, 25.90) --
	( 42.82, 25.90) --
	( 42.82, 25.90) --
	( 42.75, 25.90) --
	( 42.75, 25.90) --
	( 42.75, 25.90) --
	( 42.67, 25.90) --
	( 42.67, 25.90) --
	( 42.67, 25.90) --
	( 42.60, 25.90) --
	( 42.60, 25.90) --
	( 42.60, 25.90) --
	( 42.53, 25.90) --
	( 42.53, 25.90) --
	( 42.53, 25.90) --
	( 42.52, 25.90) --
	( 42.52, 25.90) --
	( 42.52, 25.90) --
	( 42.45, 25.90) --
	( 42.45, 25.90) --
	( 42.45, 25.90) --
	( 42.38, 25.90) --
	( 42.38, 25.90) --
	( 42.38, 25.90) --
	( 42.30, 25.90) --
	( 42.30, 25.90) --
	( 42.30, 25.90) --
	( 42.23, 25.90) --
	( 42.23, 25.90) --
	( 42.23, 25.90) --
	( 42.21, 25.90) --
	( 42.21, 25.90) --
	( 42.21, 25.90) --
	( 42.15, 25.90) --
	( 42.15, 25.90) --
	( 42.15, 25.90) --
	( 42.08, 25.90) --
	( 42.08, 25.90) --
	( 42.08, 25.90) --
	( 42.05, 25.90) --
	( 42.05, 25.90) --
	( 42.05, 25.90) --
	( 42.00, 25.90) --
	( 42.00, 25.90) --
	( 42.00, 25.90) --
	( 41.93, 25.90) --
	( 41.93, 25.90) --
	( 41.93, 25.90) --
	( 41.86, 25.90) --
	( 41.86, 25.90) --
	( 41.86, 25.90) --
	( 41.78, 25.90) --
	( 41.78, 25.90) --
	( 41.78, 25.90) --
	( 41.71, 25.90) --
	( 41.71, 25.90) --
	( 41.71, 25.90) --
	( 41.67, 25.90) --
	( 41.67, 25.90) --
	( 41.67, 25.90) --
	( 41.63, 25.90) --
	( 41.63, 25.90) --
	( 41.63, 25.90) --
	( 41.56, 25.90) --
	( 41.56, 25.90) --
	( 41.56, 25.90) --
	( 41.48, 25.90) --
	( 41.48, 25.90) --
	( 41.48, 25.90) --
	( 41.41, 25.90) --
	( 41.41, 25.90) --
	( 41.41, 25.90) --
	( 41.33, 25.90) --
	( 41.33, 25.90) --
	( 41.33, 25.90) --
	( 41.29, 25.90) --
	( 41.29, 25.90) --
	( 41.29, 25.90) --
	( 41.26, 25.90) --
	( 41.26, 25.90) --
	( 41.26, 25.90) --
	( 41.19, 25.90) --
	( 41.19, 25.90) --
	( 41.19, 25.90) --
	( 41.11, 25.90) --
	( 41.11, 25.90) --
	( 41.11, 25.90) --
	( 41.04, 25.90) --
	( 41.04, 25.90) --
	( 41.04, 25.90) --
	( 40.96, 25.90) --
	( 40.96, 25.90) --
	( 40.96, 25.90) --
	( 40.92, 25.90) --
	( 40.92, 25.90) --
	( 40.92, 25.90) --
	( 40.89, 25.90) --
	( 40.89, 25.90) --
	( 40.89, 25.90) --
	( 40.81, 25.90) --
	( 40.81, 25.90) --
	( 40.81, 25.90) --
	( 40.74, 25.90) --
	( 40.74, 25.90) --
	( 40.74, 25.90) --
	( 40.67, 25.90) --
	( 40.67, 25.90) --
	( 40.67, 25.90) --
	( 40.59, 25.90) --
	( 40.59, 25.90) --
	( 40.59, 25.90) --
	( 40.52, 25.90) --
	( 40.52, 25.90) --
	( 40.52, 25.90) --
	( 40.52, 25.90) --
	( 40.52, 25.90) --
	( 40.52, 25.90) --
	( 40.44, 25.90) --
	( 40.44, 25.90) --
	( 40.44, 25.90) --
	( 40.37, 25.90) --
	( 40.37, 25.90) --
	( 40.37, 25.90) --
	( 40.35, 25.90) --
	( 40.35, 25.90) --
	( 40.35, 25.90) --
	( 40.29, 25.90) --
	( 40.29, 25.90) --
	( 40.29, 25.90) --
	( 40.22, 25.90) --
	( 40.22, 25.90) --
	( 40.22, 25.90) --
	( 40.14, 25.90) --
	( 40.14, 25.90) --
	( 40.14, 25.90) --
	( 40.07, 25.90) --
	( 40.07, 25.90) --
	( 40.07, 25.90) --
	( 40.03, 25.90) --
	( 40.03, 25.90) --
	( 40.03, 25.90) --
	( 40.00, 25.90) --
	( 40.00, 25.90) --
	( 40.00, 25.90) --
	( 39.92, 25.90) --
	( 39.92, 25.90) --
	( 39.92, 25.90) --
	( 39.91, 25.90) --
	( 39.91, 25.90) --
	( 39.91, 25.90) --
	( 39.87, 25.90) --
	( 39.87, 25.90) --
	( 39.87, 25.90) --
	( 39.85, 25.90) --
	( 39.85, 25.90) --
	( 39.85, 25.90) --
	( 39.83, 25.90) --
	( 39.83, 25.90) --
	( 39.83, 25.90) --
	( 39.79, 25.90) --
	( 39.79, 25.90) --
	( 39.79, 25.90) --
	( 39.77, 25.90) --
	( 39.77, 25.90) --
	( 39.77, 25.90) --
	( 39.71, 25.90) --
	( 39.71, 25.90) --
	( 39.71, 25.90) --
	( 39.70, 25.90) --
	( 39.70, 25.90) --
	( 39.70, 25.90) --
	( 39.67, 25.90) --
	( 39.67, 25.90) --
	( 39.67, 25.90) --
	( 39.62, 25.90) --
	( 39.62, 25.90) --
	( 39.62, 25.90) --
	( 39.55, 25.90) --
	( 39.55, 25.90) --
	( 39.55, 25.90) --
	( 39.51, 25.90) --
	( 39.51, 25.90) --
	( 39.51, 25.90) --
	( 39.47, 25.90) --
	( 39.47, 25.90) --
	( 39.47, 25.90) --
	( 39.47, 25.90) --
	( 39.47, 25.90) --
	( 39.47, 25.90) --
	( 39.40, 25.90) --
	( 39.40, 25.90) --
	( 39.40, 25.90) --
	( 39.33, 25.90) --
	( 39.33, 25.90) --
	( 39.33, 25.90) --
	( 39.25, 25.90) --
	( 39.25, 25.90) --
	( 39.25, 25.90) --
	( 39.18, 25.90) --
	( 39.18, 25.90) --
	( 39.18, 25.90) --
	( 39.14, 25.90) --
	( 39.14, 25.90) --
	( 39.14, 25.90) --
	( 39.10, 25.90) --
	( 39.10, 25.90) --
	( 39.10, 25.90) --
	( 39.03, 25.90) --
	( 39.03, 25.90) --
	( 39.03, 25.90) --
	( 38.99, 25.90) --
	( 38.99, 25.90) --
	( 38.99, 25.90) --
	( 38.95, 25.90) --
	( 38.95, 25.90) --
	( 38.95, 25.90) --
	( 38.88, 25.90) --
	( 38.88, 25.90) --
	( 38.88, 25.90) --
	( 38.82, 25.90) --
	( 38.82, 25.90) --
	( 38.82, 25.90) --
	( 38.80, 25.90) --
	( 38.80, 25.90) --
	( 38.80, 25.90) --
	( 38.73, 25.90) --
	( 38.73, 25.90) --
	( 38.73, 25.90) --
	( 38.65, 25.90) --
	( 38.65, 25.90) --
	( 38.65, 25.90) --
	( 38.58, 25.90) --
	( 38.58, 25.90) --
	( 38.58, 25.90) --
	( 38.51, 25.90) --
	( 38.51, 25.90) --
	( 38.51, 25.90) --
	( 38.43, 25.90) --
	( 38.43, 25.90) --
	( 38.43, 25.90) --
	( 38.36, 25.90) --
	( 38.36, 25.90) --
	( 38.36, 25.90) --
	( 38.33, 25.90) --
	( 38.33, 25.90) --
	( 38.33, 25.90) --
	( 38.28, 25.90) --
	( 38.28, 25.90) --
	( 38.28, 25.90) --
	( 38.21, 25.90) --
	( 38.21, 25.90) --
	( 38.21, 25.90) --
	( 38.13, 25.90) --
	( 38.13, 25.90) --
	( 38.13, 25.90) --
	( 38.06, 25.90) --
	( 38.06, 25.90) --
	( 38.06, 25.90) --
	( 37.98, 25.90) --
	( 37.98, 25.90) --
	( 37.98, 25.90) --
	( 37.97, 25.90) --
	( 37.97, 25.90) --
	( 37.97, 25.90) --
	( 37.91, 25.90) --
	( 37.91, 25.90) --
	( 37.91, 25.90) --
	( 37.83, 25.90) --
	( 37.83, 25.90) --
	( 37.83, 25.90) --
	( 37.76, 25.90) --
	( 37.76, 25.90) --
	( 37.76, 25.90) --
	( 37.69, 25.90) --
	( 37.69, 25.90) --
	( 37.69, 25.90) --
	( 37.65, 25.90) --
	( 37.65, 25.90) --
	( 37.65, 25.90) --
	( 37.61, 25.90) --
	( 37.61, 25.90) --
	( 37.61, 25.90) --
	( 37.61, 25.90) --
	( 37.61, 25.90) --
	( 37.61, 25.90) --
	( 37.54, 25.90) --
	( 37.54, 25.90) --
	( 37.54, 25.90) --
	( 37.46, 25.90) --
	( 37.46, 25.90) --
	( 37.46, 25.90) --
	( 37.39, 25.90) --
	( 37.39, 25.90) --
	( 37.39, 25.90) --
	( 37.32, 25.90) --
	( 37.32, 25.90) --
	( 37.32, 25.90) --
	( 37.31, 25.90) --
	( 37.31, 25.90) --
	( 37.31, 25.90) --
	( 37.24, 25.90) --
	( 37.24, 25.90) --
	( 37.24, 25.90) --
	( 37.16, 25.90) --
	( 37.16, 25.90) --
	( 37.16, 25.90) --
	( 37.16, 25.90) --
	( 37.16, 25.90) --
	( 37.16, 25.90) --
	( 37.09, 25.90) --
	( 37.09, 25.90) --
	( 37.09, 25.90) --
	( 37.01, 25.90) --
	( 37.01, 25.90) --
	( 37.01, 25.90) --
	( 36.94, 25.90) --
	( 36.94, 25.90) --
	( 36.94, 25.90) --
	( 36.92, 25.90) --
	( 36.92, 25.90) --
	( 36.92, 25.90) --
	( 36.86, 25.90) --
	( 36.86, 25.90) --
	( 36.86, 25.90) --
	( 36.79, 25.90) --
	( 36.79, 25.90) --
	( 36.79, 25.90) --
	( 36.76, 25.90) --
	( 36.76, 25.90) --
	( 36.76, 25.90) --
	( 36.71, 25.90) --
	( 36.71, 25.90) --
	( 36.71, 25.90) --
	( 36.64, 25.90) --
	( 36.64, 25.90) --
	( 36.64, 25.90) --
	( 36.57, 25.90) --
	( 36.57, 25.90) --
	( 36.57, 25.90) --
	( 36.49, 25.90) --
	( 36.49, 25.90) --
	( 36.49, 25.90) --
	( 36.43, 25.90) --
	( 36.43, 25.90) --
	( 36.43, 25.90) --
	( 36.42, 25.90) --
	( 36.42, 25.90) --
	( 36.42, 25.90) --
	( 36.34, 25.90) --
	( 36.34, 25.90) --
	( 36.34, 25.90) --
	( 36.27, 25.90) --
	( 36.27, 25.90) --
	( 36.27, 25.90) --
	( 36.23, 25.90) --
	( 36.23, 25.90) --
	( 36.23, 25.90) --
	( 36.19, 25.90) --
	( 36.19, 25.90) --
	( 36.19, 25.90) --
	( 36.15, 25.90) --
	( 36.15, 25.90) --
	( 36.15, 25.90) --
	( 36.12, 25.90) --
	( 36.12, 25.90) --
	( 36.12, 25.90) --
	( 36.11, 25.90) --
	( 36.11, 25.90) --
	( 36.11, 25.90) --
	( 36.04, 25.90) --
	( 36.04, 25.90) --
	( 36.04, 25.90) --
	( 35.97, 25.90) --
	( 35.97, 25.90) --
	( 35.97, 25.90) --
	( 35.89, 25.90) --
	( 35.89, 25.90) --
	( 35.89, 25.90) --
	( 35.87, 25.90) --
	( 35.87, 25.90) --
	( 35.87, 25.90) --
	( 35.83, 25.90) --
	( 35.83, 25.90) --
	( 35.83, 25.90) --
	( 35.82, 25.90) --
	( 35.82, 25.90) --
	( 35.82, 25.90) --
	( 35.82, 25.90) --
	( 35.82, 25.90) --
	( 35.82, 25.90) --
	( 35.74, 25.90) --
	( 35.74, 25.90) --
	( 35.74, 25.90) --
	( 35.71, 25.90) --
	( 35.71, 25.90) --
	( 35.71, 25.90) --
	( 35.67, 25.90) --
	( 35.67, 25.90) --
	( 35.67, 25.90) --
	( 35.60, 25.90) --
	( 35.60, 25.90) --
	( 35.60, 25.90) --
	( 35.58, 25.90) --
	( 35.58, 25.90) --
	( 35.58, 25.90) --
	( 35.52, 25.90) --
	( 35.52, 25.90) --
	( 35.52, 25.90) --
	( 35.50, 25.90) --
	( 35.50, 25.90) --
	( 35.50, 25.90) --
	( 35.45, 25.90) --
	( 35.45, 25.90) --
	( 35.45, 25.90) --
	( 35.38, 25.90) --
	( 35.38, 25.90) --
	( 35.38, 25.90) --
	( 35.37, 25.90) --
	( 35.37, 25.90) --
	( 35.37, 25.90) --
	( 35.34, 25.90) --
	( 35.34, 25.90) --
	( 35.34, 25.90) --
	( 35.30, 25.90) --
	( 35.30, 25.90) --
	( 35.30, 25.90) --
	( 35.22, 25.90) --
	( 35.22, 25.90) --
	( 35.22, 25.90) --
	( 35.18, 25.90) --
	( 35.18, 25.90) --
	( 35.18, 25.90) --
	( 35.15, 25.90) --
	( 35.15, 25.90) --
	( 35.15, 25.90) --
	( 35.07, 25.90) --
	( 35.07, 25.90) --
	( 35.07, 25.90) --
	( 35.06, 25.90) --
	( 35.06, 25.90) --
	( 35.06, 25.90) --
	( 35.00, 25.90) --
	( 35.00, 25.90) --
	( 35.00, 25.90) --
	( 34.98, 25.90) --
	( 34.98, 25.90) --
	( 34.98, 25.90) --
	( 34.92, 25.90) --
	( 34.92, 25.90) --
	( 34.92, 25.90) --
	( 34.87, 25.90) --
	( 34.87, 25.90) --
	( 34.87, 25.90) --
	( 34.86, 25.90) --
	( 34.86, 25.90) --
	( 34.86, 25.90) --
	( 34.85, 25.90) --
	( 34.85, 25.90) --
	( 34.85, 25.90) --
	( 34.82, 25.90) --
	( 34.82, 25.90) --
	( 34.82, 25.90) --
	( 34.77, 25.90) --
	( 34.77, 25.90) --
	( 34.77, 25.90) --
	( 34.74, 25.90) --
	( 34.74, 25.90) --
	( 34.74, 25.90) --
	( 34.70, 25.90) --
	( 34.70, 25.90) --
	( 34.70, 25.90) --
	( 34.70, 25.90) --
	( 34.70, 25.90) --
	( 34.70, 25.90) --
	( 34.62, 25.90) --
	( 34.62, 25.90) --
	( 34.62, 25.90) --
	( 34.61, 25.90) --
	( 34.61, 25.90) --
	( 34.61, 25.90) --
	( 34.55, 25.90) --
	( 34.55, 25.90) --
	( 34.55, 25.90) --
	( 34.53, 25.90) --
	( 34.53, 25.90) --
	( 34.53, 25.90) --
	( 34.47, 25.90) --
	( 34.47, 25.90) --
	( 34.47, 25.90) --
	( 34.41, 25.90) --
	( 34.41, 25.90) --
	( 34.41, 25.90) --
	( 34.40, 25.90) --
	( 34.40, 25.90) --
	( 34.40, 25.90) --
	( 34.32, 25.90) --
	( 34.32, 25.90) --
	( 34.32, 25.90) --
	( 34.25, 25.90) --
	( 34.25, 25.90) --
	( 34.25, 25.90) --
	( 34.25, 25.90) --
	( 34.25, 25.90) --
	( 34.25, 25.90) --
	( 34.21, 25.90) --
	( 34.21, 25.90) --
	( 34.21, 25.90) --
	( 34.17, 25.90) --
	( 34.17, 25.90) --
	( 34.17, 25.90) --
	( 34.10, 25.90) --
	( 34.10, 25.90) --
	( 34.10, 25.90) --
	( 34.09, 25.90) --
	( 34.09, 25.90) --
	( 34.09, 25.90) --
	( 34.05, 25.90) --
	( 34.05, 25.90) --
	( 34.05, 25.90) --
	( 34.03, 25.90) --
	( 34.03, 25.90) --
	( 34.03, 25.90) --
	( 33.95, 25.90) --
	( 33.95, 25.90) --
	( 33.95, 25.90) --
	( 33.89, 25.90) --
	( 33.89, 25.90) --
	( 33.89, 25.90) --
	( 33.88, 25.90) --
	( 33.88, 25.90) --
	( 33.88, 25.90) --
	( 33.85, 25.90) --
	( 33.85, 25.90) --
	( 33.85, 25.90) --
	( 33.80, 25.90) --
	( 33.80, 25.90) --
	( 33.80, 25.90) --
	( 33.73, 25.90) --
	( 33.73, 25.90) --
	( 33.73, 25.90) --
	( 33.65, 25.90) --
	( 33.65, 25.90) --
	( 33.65, 25.90) --
	( 33.64, 25.90) --
	( 33.64, 25.90) --
	( 33.64, 25.90) --
	( 33.58, 25.90) --
	( 33.58, 25.90) --
	( 33.58, 25.90) --
	( 33.50, 25.90) --
	( 33.50, 25.90) --
	( 33.50, 25.90) --
	( 33.44, 25.90) --
	( 33.44, 25.90) --
	( 33.44, 25.90) --
	( 33.43, 25.90) --
	( 33.43, 25.90) --
	( 33.43, 25.90) --
	( 33.35, 25.90) --
	( 33.35, 25.90) --
	( 33.35, 25.90) --
	( 33.28, 25.90) --
	( 33.28, 25.90) --
	( 33.28, 25.90) --
	( 33.28, 25.90) --
	( 33.28, 25.90) --
	( 33.28, 25.90) --
	( 33.20, 25.90) --
	( 33.20, 25.90) --
	( 33.20, 25.90) --
	( 33.20, 25.90) --
	( 33.20, 25.90) --
	( 33.20, 25.90) --
	( 33.16, 25.90) --
	( 33.16, 25.90) --
	( 33.16, 25.90) --
	( 33.13, 25.90) --
	( 33.13, 25.90) --
	( 33.13, 25.90) --
	( 33.12, 25.90) --
	( 33.12, 25.90) --
	( 33.12, 25.90) --
	( 33.08, 25.90) --
	( 33.08, 25.90) --
	( 33.08, 25.90) --
	( 33.05, 25.90) --
	( 33.05, 25.90) --
	( 33.05, 25.90) --
	( 33.00, 25.90) --
	( 33.00, 25.90) --
	( 33.00, 25.90) --
	( 32.98, 25.90) --
	( 32.98, 25.90) --
	( 32.98, 25.90) --
	( 32.96, 25.90) --
	( 32.96, 25.90) --
	( 32.96, 25.90) --
	( 32.90, 25.90) --
	( 32.90, 25.90) --
	( 32.90, 25.90) --
	( 32.83, 25.90) --
	( 32.83, 25.90) --
	( 32.83, 25.90) --
	( 32.75, 25.90) --
	( 32.75, 25.90) --
	( 32.75, 25.90) --
	( 32.71, 25.90) --
	( 32.71, 25.90) --
	( 32.71, 25.90) --
	( 32.68, 25.90) --
	( 32.68, 25.90) --
	( 32.68, 25.90) --
	( 32.67, 25.90) --
	( 32.67, 25.90) --
	( 32.67, 25.90) --
	( 32.60, 25.90) --
	( 32.60, 25.90) --
	( 32.60, 25.90) --
	( 32.59, 25.90) --
	( 32.59, 25.90) --
	( 32.59, 25.90) --
	( 32.55, 25.90) --
	( 32.55, 25.90) --
	( 32.55, 25.90) --
	( 32.53, 25.90) --
	( 32.53, 25.90) --
	( 32.53, 25.90) --
	( 32.47, 25.90) --
	( 32.47, 25.90) --
	( 32.47, 25.90) --
	( 32.47, 25.90) --
	( 32.47, 25.90) --
	( 32.47, 25.90) --
	( 32.45, 25.90) --
	( 32.45, 25.90) --
	( 32.45, 25.90) --
	( 32.43, 25.90) --
	( 32.43, 25.90) --
	( 32.43, 25.90) --
	( 32.39, 25.90) --
	( 32.39, 25.90) --
	( 32.39, 25.90) --
	( 32.38, 25.90) --
	( 32.38, 25.90) --
	( 32.38, 25.90) --
	( 32.30, 25.90) --
	( 32.30, 25.90) --
	( 32.30, 25.90) --
	( 32.27, 25.90) --
	( 32.27, 25.90) --
	( 32.27, 25.90) --
	( 32.23, 25.90) --
	( 32.23, 25.90) --
	( 32.23, 25.90) --
	( 32.23, 25.90) --
	( 32.23, 25.90) --
	( 32.23, 25.90) --
	( 32.15, 25.90) --
	( 32.15, 25.90) --
	( 32.15, 25.90) --
	( 32.11, 25.90) --
	( 32.11, 25.90) --
	( 32.11, 25.90) --
	( 32.08, 25.90) --
	( 32.08, 25.90) --
	( 32.08, 25.90) --
	( 32.00, 25.90) --
	( 32.00, 25.90) --
	( 32.00, 25.90) --
	( 31.95, 25.90) --
	( 31.95, 25.90) --
	( 31.95, 25.90) --
	( 31.93, 25.90) --
	( 31.93, 25.90) --
	( 31.93, 25.90) --
	( 31.87, 25.90) --
	( 31.87, 25.90) --
	( 31.87, 25.90) --
	( 31.85, 25.90) --
	( 31.85, 25.90) --
	( 31.85, 25.90) --
	( 31.78, 25.90) --
	( 31.78, 25.90) --
	( 31.78, 25.90) --
	( 31.74, 25.90) --
	( 31.74, 25.90) --
	( 31.74, 25.90) --
	( 31.71, 25.90) --
	( 31.71, 25.90) --
	( 31.71, 25.90) --
	( 31.70, 25.90) --
	( 31.70, 25.90) --
	( 31.70, 25.90) --
	( 31.63, 25.90) --
	( 31.63, 25.90) --
	( 31.63, 25.90) --
	( 31.58, 25.90) --
	( 31.58, 25.90) --
	( 31.58, 25.90) --
	( 31.56, 25.90) --
	( 31.56, 25.90) --
	( 31.56, 25.90) --
	( 31.48, 25.90) --
	( 31.48, 25.90) --
	( 31.48, 25.90) --
	( 31.46, 25.90) --
	( 31.46, 25.90) --
	( 31.46, 25.90) --
	( 31.42, 25.90) --
	( 31.42, 25.90) --
	( 31.42, 25.90) --
	( 31.41, 25.90) --
	( 31.41, 25.90) --
	( 31.41, 25.90) --
	( 31.33, 25.90) --
	( 31.33, 25.90) --
	( 31.33, 25.90) --
	( 31.30, 25.90) --
	( 31.30, 25.90) --
	( 31.30, 25.90) --
	( 31.26, 25.90) --
	( 31.26, 25.90) --
	( 31.26, 25.90) --
	( 31.18, 25.90) --
	( 31.18, 25.90) --
	( 31.18, 25.90) --
	( 31.14, 25.90) --
	( 31.14, 25.90) --
	( 31.14, 25.90) --
	( 31.10, 25.90) --
	( 31.10, 25.90) --
	( 31.10, 25.90) --
	( 31.03, 25.90) --
	( 31.03, 25.90) --
	( 31.03, 25.90) --
	( 30.96, 25.90) --
	( 30.96, 25.90) --
	( 30.96, 25.90) --
	( 30.94, 25.90) --
	( 30.94, 25.90) --
	( 30.94, 25.90) --
	( 30.88, 25.90) --
	( 30.88, 25.90) --
	( 30.88, 25.90) --
	( 30.81, 25.90) --
	( 30.81, 25.90) --
	( 30.81, 25.90) --
	( 30.73, 25.90) --
	( 30.73, 25.90) --
	( 30.73, 25.90) --
	( 30.73, 25.90) --
	( 30.73, 25.90) --
	( 30.73, 25.90) --
	( 30.66, 25.90) --
	( 30.66, 25.90) --
	( 30.66, 25.90) --
	( 30.58, 25.90) --
	( 30.58, 25.90) --
	( 30.58, 25.90) --
	( 30.51, 25.90) --
	( 30.51, 25.90) --
	( 30.51, 25.90) --
	( 30.43, 25.90) --
	( 30.43, 25.90) --
	( 30.43, 25.90) --
	( 30.41, 25.90) --
	( 30.41, 25.90) --
	( 30.41, 25.90) --
	( 30.36, 25.90) --
	( 30.36, 25.90) --
	( 30.36, 25.90) --
	( 30.28, 25.90) --
	( 30.28, 25.90) --
	( 30.28, 25.90) --
	( 30.21, 25.90) --
	( 30.21, 25.90) --
	( 30.21, 25.90) --
	( 30.13, 25.90) --
	( 30.13, 25.90) --
	( 30.13, 25.90) --
	( 30.06, 25.90) --
	( 30.06, 25.90) --
	( 30.06, 25.90) --
	( 29.98, 25.90) --
	( 29.98, 25.90) --
	( 29.98, 25.90) --
	( 29.93, 25.90) --
	( 29.93, 25.90) --
	( 29.93, 25.90) --
	( 29.91, 25.90) --
	( 29.91, 25.90) --
	( 29.91, 25.90) --
	( 29.83, 25.90) --
	( 29.83, 25.90) --
	( 29.83, 25.90) --
	( 29.76, 25.90) --
	( 29.76, 25.90) --
	( 29.76, 25.90) --
	( 29.68, 25.90) --
	( 29.68, 25.90) --
	( 29.68, 25.90) --
	( 29.61, 25.90) --
	( 29.61, 25.90) --
	( 29.61, 25.90) --
	( 29.53, 25.90) --
	( 29.53, 25.90) --
	( 29.53, 25.90) --
	( 29.46, 25.90) --
	( 29.46, 25.90) --
	( 29.46, 25.90) --
	( 29.40, 25.90) --
	( 29.40, 25.90) --
	( 29.40, 25.90) --
	( 29.38, 25.90) --
	( 29.38, 25.90) --
	( 29.38, 25.90) --
	( 29.31, 25.90) --
	( 29.31, 25.90) --
	( 29.31, 25.90) --
	( 29.23, 25.90) --
	( 29.23, 25.90) --
	( 29.23, 25.90) --
	( 29.20, 25.90) --
	( 29.20, 25.90) --
	( 29.20, 25.90) --
	( 29.16, 25.90) --
	( 29.16, 25.90) --
	( 29.16, 25.90) --
	( 29.08, 25.90) --
	( 29.08, 25.90) --
	( 29.08, 25.90) --
	( 29.01, 25.90) --
	( 29.01, 25.90) --
	( 29.01, 25.90) --
	( 28.93, 25.90) --
	( 28.93, 25.90) --
	( 28.93, 25.90) --
	( 28.86, 25.90) --
	( 28.86, 25.90) --
	( 28.86, 25.90) --
	( 28.78, 25.90) --
	( 28.78, 25.90) --
	( 28.78, 25.90) --
	( 28.71, 25.90) --
	( 28.71, 25.90) --
	( 28.71, 25.90) --
	( 28.63, 25.90) --
	( 28.63, 25.90) --
	( 28.63, 25.90) --
	( 28.56, 25.90) --
	( 28.56, 25.90) --
	( 28.56, 25.90) --
	( 28.48, 25.90) --
	( 28.48, 25.90) --
	( 28.48, 25.90) --
	( 28.40, 25.90) --
	( 28.40, 25.90) --
	( 28.40, 25.90) --
	( 28.39, 25.90) --
	( 28.39, 25.90) --
	( 28.39, 25.90) --
	( 28.33, 25.90) --
	( 28.33, 25.90) --
	( 28.33, 25.90) --
	( 28.25, 25.90) --
	( 28.25, 25.90) --
	( 28.25, 25.90) --
	( 28.19, 25.90) --
	( 28.19, 25.90) --
	( 28.19, 25.90) --
	( 28.18, 25.90) --
	( 28.18, 25.90) --
	( 28.18, 25.90) --
	( 28.10, 25.90) --
	( 28.10, 25.90) --
	( 28.10, 25.90) --
	( 28.03, 25.90) --
	( 28.03, 25.90) --
	( 28.03, 25.90) --
	( 28.03, 25.90) --
	( 28.03, 25.90) --
	( 28.03, 25.90) --
	( 27.95, 25.90) --
	( 27.95, 25.90) --
	( 27.95, 25.90) --
	( 27.88, 25.90) --
	( 27.88, 25.90) --
	( 27.88, 25.90) --
	( 27.80, 25.90) --
	( 27.80, 25.90) --
	( 27.80, 25.90) --
	( 27.73, 25.90) --
	( 27.73, 25.90) --
	( 27.73, 25.90) --
	( 27.66, 25.90) --
	( 27.66, 25.90) --
	( 27.66, 25.90) --
	( 27.65, 25.90) --
	( 27.65, 25.90) --
	( 27.65, 25.90) --
	( 27.58, 25.90) --
	( 27.58, 25.90) --
	( 27.58, 25.90) --
	( 27.50, 25.90) --
	( 27.50, 25.90) --
	( 27.50, 25.90) --
	( 27.43, 25.90) --
	( 27.43, 25.90) --
	( 27.43, 25.90) --
	( 27.39, 25.90) --
	( 27.39, 25.90) --
	( 27.39, 25.90) --
	( 27.35, 25.90) --
	( 27.35, 25.90) --
	( 27.35, 25.90) --
	( 27.28, 25.90) --
	( 27.28, 25.90) --
	( 27.28, 25.90) --
	( 27.20, 25.90) --
	( 27.20, 25.90) --
	( 27.20, 25.90) --
	( 27.13, 25.90) --
	( 27.13, 25.90) --
	( 27.13, 25.90) --
	( 27.06, 25.90) --
	( 27.06, 25.90) --
	( 27.06, 25.90) --
	( 27.05, 25.90) --
	( 27.05, 25.90) --
	( 27.05, 25.90) --
	( 26.98, 25.90) --
	( 26.98, 25.90) --
	( 26.98, 25.90) --
	( 26.90, 25.90) --
	( 26.90, 25.90) --
	( 26.90, 25.90) --
	( 26.83, 25.90) --
	( 26.83, 25.90) --
	( 26.83, 25.90) --
	( 26.75, 25.90) --
	( 26.75, 25.90) --
	( 26.75, 25.90) --
	( 26.68, 25.90) --
	( 26.68, 25.90) --
	( 26.68, 25.90) --
	( 26.60, 25.90) --
	( 26.60, 25.90) --
	( 26.60, 25.90) --
	( 26.53, 25.90) --
	( 26.53, 25.90) --
	( 26.53, 25.90) --
	( 26.45, 25.90) --
	( 26.45, 25.90) --
	( 26.45, 25.90) --
	( 26.38, 25.90) --
	( 26.38, 25.90) --
	( 26.38, 25.90) --
	( 26.30, 25.90) --
	( 26.30, 25.90) --
	( 26.30, 25.90) --
	( 26.22, 25.90) --
	( 26.22, 25.90) --
	( 26.22, 25.90) --
	( 26.15, 25.90) --
	( 26.15, 25.90) --
	( 26.15, 25.90) --
	( 26.07, 25.90) --
	( 26.07, 25.90) --
	( 26.07, 25.90) --
	( 26.00, 25.90) --
	( 26.00, 25.90) --
	( 26.00, 25.90) --
	( 25.92, 25.90) --
	( 25.92, 25.90) --
	( 25.92, 25.90) --
	( 25.85, 25.90) --
	( 25.85, 25.90) --
	( 25.85, 25.90) --
	( 25.80, 25.90) --
	( 25.80, 25.90) --
	( 25.80, 25.90) --
	( 25.77, 25.90) --
	( 25.77, 25.90) --
	( 25.77, 25.90) --
	( 25.70, 25.90) --
	( 25.70, 25.90) --
	( 25.70, 25.90) --
	( 25.62, 25.90) --
	( 25.62, 25.90) --
	( 25.62, 25.90) --
	( 25.55, 25.90) --
	( 25.55, 25.90) --
	( 25.55, 25.90) --
	( 25.47, 25.90) --
	( 25.47, 25.90) --
	( 25.47, 25.90) --
	( 25.40, 25.90) --
	( 25.40, 25.90) --
	( 25.40, 25.90) --
	( 25.32, 25.90) --
	( 25.32, 25.90) --
	( 25.32, 25.90) --
	( 25.25, 25.90) --
	( 25.25, 25.90) --
	( 25.25, 25.90) --
	( 25.17, 25.90) --
	( 25.17, 25.90) --
	( 25.17, 25.90) --
	( 25.12, 25.90) --
	( 25.12, 25.90) --
	( 25.12, 25.90) --
	( 25.10, 25.90) --
	( 25.10, 25.90) --
	( 25.10, 25.90) --
	( 25.02, 25.90) --
	( 25.02, 25.90) --
	( 25.02, 25.90) --
	( 24.95, 25.90) --
	( 24.95, 25.90) --
	( 24.95, 25.90) --
	( 24.87, 25.90) --
	( 24.87, 25.90) --
	( 24.87, 25.90) --
	( 24.83, 25.90) --
	( 24.83, 25.90) --
	( 24.83, 25.90) --
	( 24.80, 25.90) --
	( 24.80, 25.90) --
	( 24.80, 25.90) --
	( 24.75, 25.90) --
	( 24.75, 25.90) --
	( 24.75, 25.90) --
	( 24.72, 25.90) --
	( 24.72, 25.90) --
	( 24.72, 25.90) --
	( 24.65, 25.90) --
	( 24.65, 25.90) --
	( 24.65, 25.90) --
	( 24.63, 25.90) --
	( 24.63, 25.90) --
	( 24.63, 25.90) --
	( 24.57, 25.90) --
	( 24.57, 25.90) --
	( 24.57, 25.90) --
	( 24.55, 25.90) --
	( 24.55, 25.90) --
	( 24.55, 25.90) --
	( 24.51, 25.90) --
	( 24.51, 25.90) --
	( 24.51, 25.90) --
	( 24.50, 25.90) --
	( 24.50, 25.90) --
	( 24.50, 25.90) --
	( 24.42, 25.90) --
	( 24.42, 25.90) --
	( 24.42, 25.90) --
	cycle;
\definecolor{drawColor}{RGB}{0,176,246}

\path[draw=drawColor,line width= 0.6pt,line join=round] ( 24.42, 25.94) --
	( 24.42, 25.94) --
	( 24.50, 25.94) --
	( 24.50, 25.94) --
	( 24.50, 25.94) --
	( 24.51, 25.94) --
	( 24.51, 25.94) --
	( 24.51, 25.94) --
	( 24.55, 25.94) --
	( 24.55, 25.94) --
	( 24.55, 25.94) --
	( 24.57, 25.94) --
	( 24.57, 25.94) --
	( 24.57, 25.94) --
	( 24.63, 25.94) --
	( 24.63, 25.94) --
	( 24.63, 25.94) --
	( 24.65, 25.94) --
	( 24.65, 25.94) --
	( 24.65, 25.94) --
	( 24.72, 25.94) --
	( 24.72, 25.94) --
	( 24.72, 25.94) --
	( 24.75, 25.94) --
	( 24.75, 25.94) --
	( 24.75, 25.94) --
	( 24.80, 25.94) --
	( 24.80, 25.94) --
	( 24.80, 25.94) --
	( 24.83, 25.94) --
	( 24.83, 25.94) --
	( 24.83, 25.94) --
	( 24.87, 25.94) --
	( 24.87, 25.94) --
	( 24.87, 25.94) --
	( 24.95, 25.94) --
	( 24.95, 25.94) --
	( 24.95, 25.94) --
	( 25.02, 25.94) --
	( 25.02, 25.94) --
	( 25.02, 25.94) --
	( 25.10, 25.94) --
	( 25.10, 25.94) --
	( 25.10, 25.94) --
	( 25.12, 25.94) --
	( 25.12, 25.94) --
	( 25.12, 25.94) --
	( 25.17, 25.94) --
	( 25.17, 25.94) --
	( 25.17, 25.94) --
	( 25.25, 25.94) --
	( 25.25, 25.94) --
	( 25.25, 25.94) --
	( 25.32, 25.94) --
	( 25.32, 25.94) --
	( 25.32, 25.94) --
	( 25.40, 25.94) --
	( 25.40, 25.94) --
	( 25.40, 25.94) --
	( 25.47, 25.94) --
	( 25.47, 25.94) --
	( 25.47, 25.94) --
	( 25.55, 25.94) --
	( 25.55, 25.94) --
	( 25.55, 25.94) --
	( 25.62, 25.94) --
	( 25.62, 25.94) --
	( 25.62, 25.94) --
	( 25.70, 25.94) --
	( 25.70, 25.94) --
	( 25.70, 25.94) --
	( 25.77, 25.94) --
	( 25.77, 25.94) --
	( 25.77, 25.94) --
	( 25.80, 25.94) --
	( 25.80, 25.94) --
	( 25.80, 25.94) --
	( 25.85, 25.94) --
	( 25.85, 25.94) --
	( 25.85, 25.94) --
	( 25.92, 25.94) --
	( 25.92, 25.94) --
	( 25.92, 25.94) --
	( 26.00, 25.94) --
	( 26.00, 25.94) --
	( 26.00, 25.94) --
	( 26.07, 25.94) --
	( 26.07, 25.94) --
	( 26.07, 25.94) --
	( 26.15, 25.94) --
	( 26.15, 25.94) --
	( 26.15, 25.94) --
	( 26.22, 25.94) --
	( 26.22, 25.94) --
	( 26.22, 25.94) --
	( 26.30, 25.94) --
	( 26.30, 25.94) --
	( 26.30, 25.94) --
	( 26.38, 25.94) --
	( 26.38, 25.94) --
	( 26.38, 25.94) --
	( 26.45, 25.94) --
	( 26.45, 25.94) --
	( 26.45, 25.94) --
	( 26.53, 25.94) --
	( 26.53, 25.94) --
	( 26.53, 25.94) --
	( 26.60, 25.94) --
	( 26.60, 25.94) --
	( 26.60, 25.94) --
	( 26.68, 25.94) --
	( 26.68, 25.94) --
	( 26.68, 25.94) --
	( 26.75, 25.94) --
	( 26.75, 25.94) --
	( 26.75, 25.94) --
	( 26.83, 25.94) --
	( 26.83, 25.94) --
	( 26.83, 25.94) --
	( 26.90, 25.94) --
	( 26.90, 25.94) --
	( 26.90, 25.94) --
	( 26.98, 25.94) --
	( 26.98, 25.94) --
	( 26.98, 25.94) --
	( 27.05, 25.94) --
	( 27.05, 25.94) --
	( 27.05, 25.94) --
	( 27.06, 25.94) --
	( 27.06, 25.94) --
	( 27.06, 25.94) --
	( 27.13, 25.94) --
	( 27.13, 25.94) --
	( 27.13, 25.94) --
	( 27.20, 25.94) --
	( 27.20, 25.94) --
	( 27.20, 25.94) --
	( 27.28, 25.94) --
	( 27.28, 25.94) --
	( 27.28, 25.94) --
	( 27.35, 25.94) --
	( 27.35, 25.94) --
	( 27.35, 25.94) --
	( 27.39, 25.94) --
	( 27.39, 25.94) --
	( 27.39, 25.94) --
	( 27.43, 25.94) --
	( 27.43, 25.94) --
	( 27.43, 25.94) --
	( 27.50, 25.94) --
	( 27.50, 25.94) --
	( 27.50, 25.94) --
	( 27.58, 25.94) --
	( 27.58, 25.94) --
	( 27.58, 25.94) --
	( 27.65, 25.95) --
	( 27.65, 25.95) --
	( 27.65, 25.95) --
	( 27.66, 25.95) --
	( 27.66, 25.95) --
	( 27.66, 25.95) --
	( 27.73, 25.95) --
	( 27.73, 25.95) --
	( 27.73, 25.95) --
	( 27.80, 25.95) --
	( 27.80, 25.95) --
	( 27.80, 25.95) --
	( 27.88, 25.95) --
	( 27.88, 25.95) --
	( 27.88, 25.95) --
	( 27.95, 25.95) --
	( 27.95, 25.95) --
	( 27.95, 25.95) --
	( 28.03, 25.95) --
	( 28.03, 25.95) --
	( 28.03, 25.95) --
	( 28.03, 25.95) --
	( 28.03, 25.95) --
	( 28.03, 25.95) --
	( 28.10, 25.95) --
	( 28.10, 25.95) --
	( 28.10, 25.95) --
	( 28.18, 25.96) --
	( 28.18, 25.96) --
	( 28.18, 25.96) --
	( 28.19, 25.96) --
	( 28.19, 25.96) --
	( 28.19, 25.96) --
	( 28.25, 25.96) --
	( 28.25, 25.96) --
	( 28.25, 25.96) --
	( 28.33, 25.96) --
	( 28.33, 25.96) --
	( 28.33, 25.96) --
	( 28.39, 25.96) --
	( 28.39, 25.96) --
	( 28.39, 25.96) --
	( 28.40, 25.96) --
	( 28.40, 25.96) --
	( 28.40, 25.96) --
	( 28.48, 25.96) --
	( 28.48, 25.96) --
	( 28.48, 25.96) --
	( 28.56, 25.96) --
	( 28.56, 25.96) --
	( 28.56, 25.96) --
	( 28.63, 25.96) --
	( 28.63, 25.96) --
	( 28.63, 25.96) --
	( 28.71, 25.97) --
	( 28.71, 25.97) --
	( 28.71, 25.97) --
	( 28.78, 25.97) --
	( 28.78, 25.97) --
	( 28.78, 25.97) --
	( 28.86, 25.97) --
	( 28.86, 25.97) --
	( 28.86, 25.97) --
	( 28.93, 25.97) --
	( 28.93, 25.97) --
	( 28.93, 25.97) --
	( 29.01, 25.97) --
	( 29.01, 25.97) --
	( 29.01, 25.97) --
	( 29.08, 25.97) --
	( 29.08, 25.97) --
	( 29.08, 25.97) --
	( 29.16, 25.97) --
	( 29.16, 25.97) --
	( 29.16, 25.97) --
	( 29.20, 25.97) --
	( 29.20, 25.97) --
	( 29.20, 25.97) --
	( 29.23, 25.98) --
	( 29.23, 25.98) --
	( 29.23, 25.98) --
	( 29.31, 25.98) --
	( 29.31, 25.98) --
	( 29.31, 25.98) --
	( 29.38, 25.98) --
	( 29.38, 25.98) --
	( 29.38, 25.98) --
	( 29.40, 25.98) --
	( 29.40, 25.98) --
	( 29.40, 25.98) --
	( 29.46, 25.98) --
	( 29.46, 25.98) --
	( 29.46, 25.98) --
	( 29.53, 25.98) --
	( 29.53, 25.98) --
	( 29.53, 25.98) --
	( 29.61, 25.98) --
	( 29.61, 25.98) --
	( 29.61, 25.98) --
	( 29.68, 25.98) --
	( 29.68, 25.98) --
	( 29.68, 25.98) --
	( 29.76, 25.98) --
	( 29.76, 25.98) --
	( 29.76, 25.98) --
	( 29.83, 25.98) --
	( 29.83, 25.98) --
	( 29.83, 25.98) --
	( 29.91, 25.98) --
	( 29.91, 25.98) --
	( 29.91, 25.98) --
	( 29.93, 25.98) --
	( 29.93, 25.98) --
	( 29.93, 25.98) --
	( 29.98, 25.98) --
	( 29.98, 25.98) --
	( 29.98, 25.98) --
	( 30.06, 25.97) --
	( 30.06, 25.97) --
	( 30.06, 25.97) --
	( 30.13, 25.97) --
	( 30.13, 25.97) --
	( 30.13, 25.97) --
	( 30.21, 25.97) --
	( 30.21, 25.97) --
	( 30.21, 25.97) --
	( 30.28, 25.97) --
	( 30.28, 25.97) --
	( 30.28, 25.97) --
	( 30.36, 25.97) --
	( 30.36, 25.97) --
	( 30.36, 25.97) --
	( 30.41, 25.97) --
	( 30.41, 25.97) --
	( 30.41, 25.97) --
	( 30.43, 25.97) --
	( 30.43, 25.97) --
	( 30.43, 25.97) --
	( 30.51, 25.97) --
	( 30.51, 25.97) --
	( 30.51, 25.97) --
	( 30.58, 25.97) --
	( 30.58, 25.97) --
	( 30.58, 25.97) --
	( 30.66, 25.97) --
	( 30.66, 25.97) --
	( 30.66, 25.97) --
	( 30.73, 25.97) --
	( 30.73, 25.97) --
	( 30.73, 25.97) --
	( 30.73, 25.97) --
	( 30.73, 25.97) --
	( 30.73, 25.97) --
	( 30.81, 25.97) --
	( 30.81, 25.97) --
	( 30.81, 25.97) --
	( 30.88, 25.97) --
	( 30.88, 25.97) --
	( 30.88, 25.97) --
	( 30.94, 25.97) --
	( 30.94, 25.97) --
	( 30.94, 25.97) --
	( 30.96, 25.97) --
	( 30.96, 25.97) --
	( 30.96, 25.97) --
	( 31.03, 25.97) --
	( 31.03, 25.97) --
	( 31.03, 25.97) --
	( 31.10, 25.97) --
	( 31.10, 25.97) --
	( 31.10, 25.97) --
	( 31.14, 25.97) --
	( 31.14, 25.97) --
	( 31.14, 25.97) --
	( 31.18, 25.97) --
	( 31.18, 25.97) --
	( 31.18, 25.97) --
	( 31.26, 25.97) --
	( 31.26, 25.97) --
	( 31.26, 25.97) --
	( 31.30, 25.97) --
	( 31.30, 25.97) --
	( 31.30, 25.97) --
	( 31.33, 25.97) --
	( 31.33, 25.97) --
	( 31.33, 25.97) --
	( 31.41, 25.97) --
	( 31.41, 25.97) --
	( 31.41, 25.97) --
	( 31.42, 25.97) --
	( 31.42, 25.97) --
	( 31.42, 25.97) --
	( 31.46, 25.97) --
	( 31.46, 25.97) --
	( 31.46, 25.97) --
	( 31.48, 25.97) --
	( 31.48, 25.97) --
	( 31.48, 25.97) --
	( 31.56, 25.97) --
	( 31.56, 25.97) --
	( 31.56, 25.97) --
	( 31.58, 25.97) --
	( 31.58, 25.97) --
	( 31.58, 25.97) --
	( 31.63, 25.96) --
	( 31.63, 25.96) --
	( 31.63, 25.96) --
	( 31.70, 25.96) --
	( 31.70, 25.96) --
	( 31.70, 25.96) --
	( 31.71, 25.96) --
	( 31.71, 25.96) --
	( 31.71, 25.96) --
	( 31.74, 25.96) --
	( 31.74, 25.96) --
	( 31.74, 25.96) --
	( 31.78, 25.96) --
	( 31.78, 25.96) --
	( 31.78, 25.96) --
	( 31.85, 25.96) --
	( 31.85, 25.96) --
	( 31.85, 25.96) --
	( 31.87, 25.96) --
	( 31.87, 25.96) --
	( 31.87, 25.96) --
	( 31.93, 25.96) --
	( 31.93, 25.96) --
	( 31.93, 25.96) --
	( 31.95, 25.96) --
	( 31.95, 25.96) --
	( 31.95, 25.96) --
	( 32.00, 25.96) --
	( 32.00, 25.96) --
	( 32.00, 25.96) --
	( 32.08, 25.96) --
	( 32.08, 25.96) --
	( 32.08, 25.96) --
	( 32.11, 25.96) --
	( 32.11, 25.96) --
	( 32.11, 25.96) --
	( 32.15, 25.96) --
	( 32.15, 25.96) --
	( 32.15, 25.96) --
	( 32.23, 25.96) --
	( 32.23, 25.96) --
	( 32.23, 25.96) --
	( 32.23, 25.96) --
	( 32.23, 25.96) --
	( 32.23, 25.96) --
	( 32.27, 25.96) --
	( 32.27, 25.96) --
	( 32.27, 25.96) --
	( 32.30, 25.96) --
	( 32.30, 25.96) --
	( 32.30, 25.96) --
	( 32.38, 25.96) --
	( 32.38, 25.96) --
	( 32.38, 25.96) --
	( 32.39, 25.96) --
	( 32.39, 25.96) --
	( 32.39, 25.96) --
	( 32.43, 25.96) --
	( 32.43, 25.96) --
	( 32.43, 25.96) --
	( 32.45, 25.96) --
	( 32.45, 25.96) --
	( 32.45, 25.96) --
	( 32.47, 25.96) --
	( 32.47, 25.96) --
	( 32.47, 25.96) --
	( 32.47, 25.96) --
	( 32.47, 25.96) --
	( 32.47, 25.96) --
	( 32.53, 25.96) --
	( 32.53, 25.96) --
	( 32.53, 25.96) --
	( 32.55, 25.96) --
	( 32.55, 25.96) --
	( 32.55, 25.96) --
	( 32.59, 25.96) --
	( 32.59, 25.96) --
	( 32.59, 25.96) --
	( 32.60, 25.96) --
	( 32.60, 25.96) --
	( 32.60, 25.96) --
	( 32.67, 25.96) --
	( 32.67, 25.96) --
	( 32.67, 25.96) --
	( 32.68, 25.96) --
	( 32.68, 25.96) --
	( 32.68, 25.96) --
	( 32.71, 25.96) --
	( 32.71, 25.96) --
	( 32.71, 25.96) --
	( 32.75, 25.96) --
	( 32.75, 25.96) --
	( 32.75, 25.96) --
	( 32.83, 25.96) --
	( 32.83, 25.96) --
	( 32.83, 25.96) --
	( 32.90, 25.96) --
	( 32.90, 25.96) --
	( 32.90, 25.96) --
	( 32.96, 25.96) --
	( 32.96, 25.96) --
	( 32.96, 25.96) --
	( 32.98, 25.96) --
	( 32.98, 25.96) --
	( 32.98, 25.96) --
	( 33.00, 25.96) --
	( 33.00, 25.96) --
	( 33.00, 25.96) --
	( 33.05, 25.95) --
	( 33.05, 25.95) --
	( 33.05, 25.95) --
	( 33.08, 25.95) --
	( 33.08, 25.95) --
	( 33.08, 25.95) --
	( 33.12, 25.95) --
	( 33.12, 25.95) --
	( 33.12, 25.95) --
	( 33.13, 25.95) --
	( 33.13, 25.95) --
	( 33.13, 25.95) --
	( 33.16, 25.95) --
	( 33.16, 25.95) --
	( 33.16, 25.95) --
	( 33.20, 25.95) --
	( 33.20, 25.95) --
	( 33.20, 25.95) --
	( 33.20, 25.95) --
	( 33.20, 25.95) --
	( 33.20, 25.95) --
	( 33.28, 25.95) --
	( 33.28, 25.95) --
	( 33.28, 25.95) --
	( 33.28, 25.95) --
	( 33.28, 25.95) --
	( 33.28, 25.95) --
	( 33.35, 25.95) --
	( 33.35, 25.95) --
	( 33.35, 25.95) --
	( 33.43, 25.95) --
	( 33.43, 25.95) --
	( 33.43, 25.95) --
	( 33.44, 25.95) --
	( 33.44, 25.95) --
	( 33.44, 25.95) --
	( 33.50, 25.95) --
	( 33.50, 25.95) --
	( 33.50, 25.95) --
	( 33.58, 25.95) --
	( 33.58, 25.95) --
	( 33.58, 25.95) --
	( 33.64, 25.95) --
	( 33.64, 25.95) --
	( 33.64, 25.95) --
	( 33.65, 25.95) --
	( 33.65, 25.95) --
	( 33.65, 25.95) --
	( 33.73, 25.95) --
	( 33.73, 25.95) --
	( 33.73, 25.95) --
	( 33.80, 25.95) --
	( 33.80, 25.95) --
	( 33.80, 25.95) --
	( 33.85, 25.95) --
	( 33.85, 25.95) --
	( 33.85, 25.95) --
	( 33.88, 25.95) --
	( 33.88, 25.95) --
	( 33.88, 25.95) --
	( 33.89, 25.95) --
	( 33.89, 25.95) --
	( 33.89, 25.95) --
	( 33.95, 25.95) --
	( 33.95, 25.95) --
	( 33.95, 25.95) --
	( 34.03, 25.95) --
	( 34.03, 25.95) --
	( 34.03, 25.95) --
	( 34.05, 25.95) --
	( 34.05, 25.95) --
	( 34.05, 25.95) --
	( 34.09, 25.95) --
	( 34.09, 25.95) --
	( 34.09, 25.95) --
	( 34.10, 25.95) --
	( 34.10, 25.95) --
	( 34.10, 25.95) --
	( 34.17, 25.95) --
	( 34.17, 25.95) --
	( 34.17, 25.95) --
	( 34.21, 25.95) --
	( 34.21, 25.95) --
	( 34.21, 25.95) --
	( 34.25, 25.94) --
	( 34.25, 25.94) --
	( 34.25, 25.94) --
	( 34.25, 25.94) --
	( 34.25, 25.94) --
	( 34.25, 25.94) --
	( 34.32, 25.94) --
	( 34.32, 25.94) --
	( 34.32, 25.94) --
	( 34.40, 25.94) --
	( 34.40, 25.94) --
	( 34.40, 25.94) --
	( 34.41, 25.94) --
	( 34.41, 25.94) --
	( 34.41, 25.94) --
	( 34.47, 25.94) --
	( 34.47, 25.94) --
	( 34.47, 25.94) --
	( 34.53, 25.94) --
	( 34.53, 25.94) --
	( 34.53, 25.94) --
	( 34.55, 25.94) --
	( 34.55, 25.94) --
	( 34.55, 25.94) --
	( 34.61, 25.94) --
	( 34.61, 25.94) --
	( 34.61, 25.94) --
	( 34.62, 25.94) --
	( 34.62, 25.94) --
	( 34.62, 25.94) --
	( 34.70, 25.94) --
	( 34.70, 25.94) --
	( 34.70, 25.94) --
	( 34.70, 25.94) --
	( 34.70, 25.94) --
	( 34.70, 25.94) --
	( 34.74, 25.94) --
	( 34.74, 25.94) --
	( 34.74, 25.94) --
	( 34.77, 25.94) --
	( 34.77, 25.94) --
	( 34.77, 25.94) --
	( 34.82, 25.94) --
	( 34.82, 25.94) --
	( 34.82, 25.94) --
	( 34.85, 25.94) --
	( 34.85, 25.94) --
	( 34.85, 25.94) --
	( 34.86, 25.94) --
	( 34.86, 25.94) --
	( 34.86, 25.94) --
	( 34.87, 25.94) --
	( 34.87, 25.94) --
	( 34.87, 25.94) --
	( 34.92, 25.94) --
	( 34.92, 25.94) --
	( 34.92, 25.94) --
	( 34.98, 25.94) --
	( 34.98, 25.94) --
	( 34.98, 25.94) --
	( 35.00, 25.95) --
	( 35.00, 25.95) --
	( 35.00, 25.95) --
	( 35.06, 25.95) --
	( 35.06, 25.95) --
	( 35.06, 25.95) --
	( 35.07, 25.95) --
	( 35.07, 25.95) --
	( 35.07, 25.95) --
	( 35.15, 25.95) --
	( 35.15, 25.95) --
	( 35.15, 25.95) --
	( 35.18, 25.95) --
	( 35.18, 25.95) --
	( 35.18, 25.95) --
	( 35.22, 25.95) --
	( 35.22, 25.95) --
	( 35.22, 25.95) --
	( 35.30, 25.96) --
	( 35.30, 25.96) --
	( 35.30, 25.96) --
	( 35.34, 25.96) --
	( 35.34, 25.96) --
	( 35.34, 25.96) --
	( 35.37, 25.96) --
	( 35.37, 25.96) --
	( 35.37, 25.96) --
	( 35.38, 25.96) --
	( 35.38, 25.96) --
	( 35.38, 25.96) --
	( 35.45, 25.96) --
	( 35.45, 25.96) --
	( 35.45, 25.96) --
	( 35.50, 25.97) --
	( 35.50, 25.97) --
	( 35.50, 25.97) --
	( 35.52, 25.97) --
	( 35.52, 25.97) --
	( 35.52, 25.97) --
	( 35.58, 25.97) --
	( 35.58, 25.97) --
	( 35.58, 25.97) --
	( 35.60, 25.97) --
	( 35.60, 25.97) --
	( 35.60, 25.97) --
	( 35.67, 25.97) --
	( 35.67, 25.97) --
	( 35.67, 25.97) --
	( 35.71, 25.97) --
	( 35.71, 25.97) --
	( 35.71, 25.97) --
	( 35.74, 25.98) --
	( 35.74, 25.98) --
	( 35.74, 25.98) --
	( 35.82, 25.98) --
	( 35.82, 25.98) --
	( 35.82, 25.98) --
	( 35.82, 25.98) --
	( 35.82, 25.98) --
	( 35.82, 25.98) --
	( 35.83, 25.98) --
	( 35.83, 25.98) --
	( 35.83, 25.98) --
	( 35.87, 25.98) --
	( 35.87, 25.98) --
	( 35.87, 25.98) --
	( 35.89, 25.98) --
	( 35.89, 25.98) --
	( 35.89, 25.98) --
	( 35.97, 25.98) --
	( 35.97, 25.98) --
	( 35.97, 25.98) --
	( 36.04, 25.98) --
	( 36.04, 25.98) --
	( 36.04, 25.98) --
	( 36.11, 25.98) --
	( 36.11, 25.98) --
	( 36.11, 25.98) --
	( 36.12, 25.98) --
	( 36.12, 25.98) --
	( 36.12, 25.98) --
	( 36.15, 25.98) --
	( 36.15, 25.98) --
	( 36.15, 25.98) --
	( 36.19, 25.97) --
	( 36.19, 25.97) --
	( 36.19, 25.97) --
	( 36.23, 25.97) --
	( 36.23, 25.97) --
	( 36.23, 25.97) --
	( 36.27, 25.97) --
	( 36.27, 25.97) --
	( 36.27, 25.97) --
	( 36.34, 25.97) --
	( 36.34, 25.97) --
	( 36.34, 25.97) --
	( 36.42, 25.97) --
	( 36.42, 25.97) --
	( 36.42, 25.97) --
	( 36.43, 25.97) --
	( 36.43, 25.97) --
	( 36.43, 25.97) --
	( 36.49, 25.97) --
	( 36.49, 25.97) --
	( 36.49, 25.97) --
	( 36.57, 25.97) --
	( 36.57, 25.97) --
	( 36.57, 25.97) --
	( 36.64, 25.97) --
	( 36.64, 25.97) --
	( 36.64, 25.97) --
	( 36.71, 25.97) --
	( 36.71, 25.97) --
	( 36.71, 25.97) --
	( 36.76, 25.97) --
	( 36.76, 25.97) --
	( 36.76, 25.97) --
	( 36.79, 25.97) --
	( 36.79, 25.97) --
	( 36.79, 25.97) --
	( 36.86, 25.97) --
	( 36.86, 25.97) --
	( 36.86, 25.97) --
	( 36.92, 25.97) --
	( 36.92, 25.97) --
	( 36.92, 25.97) --
	( 36.94, 25.97) --
	( 36.94, 25.97) --
	( 36.94, 25.97) --
	( 37.01, 25.97) --
	( 37.01, 25.97) --
	( 37.01, 25.97) --
	( 37.09, 25.97) --
	( 37.09, 25.97) --
	( 37.09, 25.97) --
	( 37.16, 25.96) --
	( 37.16, 25.96) --
	( 37.16, 25.96) --
	( 37.16, 25.96) --
	( 37.16, 25.96) --
	( 37.16, 25.96) --
	( 37.24, 25.96) --
	( 37.24, 25.96) --
	( 37.24, 25.96) --
	( 37.31, 25.96) --
	( 37.31, 25.96) --
	( 37.31, 25.96) --
	( 37.32, 25.96) --
	( 37.32, 25.96) --
	( 37.32, 25.96) --
	( 37.39, 25.96) --
	( 37.39, 25.96) --
	( 37.39, 25.96) --
	( 37.46, 25.96) --
	( 37.46, 25.96) --
	( 37.46, 25.96) --
	( 37.54, 25.96) --
	( 37.54, 25.96) --
	( 37.54, 25.96) --
	( 37.61, 25.96) --
	( 37.61, 25.96) --
	( 37.61, 25.96) --
	( 37.61, 25.96) --
	( 37.61, 25.96) --
	( 37.61, 25.96) --
	( 37.65, 25.96) --
	( 37.65, 25.96) --
	( 37.65, 25.96) --
	( 37.69, 25.96) --
	( 37.69, 25.96) --
	( 37.69, 25.96) --
	( 37.76, 25.97) --
	( 37.76, 25.97) --
	( 37.76, 25.97) --
	( 37.83, 25.98) --
	( 37.83, 25.98) --
	( 37.83, 25.98) --
	( 37.91, 25.99) --
	( 37.91, 25.99) --
	( 37.91, 25.99) --
	( 37.97, 25.99) --
	( 37.97, 25.99) --
	( 37.97, 25.99) --
	( 37.98, 25.99) --
	( 37.98, 25.99) --
	( 37.98, 25.99) --
	( 38.06, 26.00) --
	( 38.06, 26.00) --
	( 38.06, 26.00) --
	( 38.13, 26.01) --
	( 38.13, 26.01) --
	( 38.13, 26.01) --
	( 38.21, 26.02) --
	( 38.21, 26.02) --
	( 38.21, 26.02) --
	( 38.28, 26.02) --
	( 38.28, 26.02) --
	( 38.28, 26.02) --
	( 38.33, 26.03) --
	( 38.33, 26.03) --
	( 38.33, 26.03) --
	( 38.36, 26.03) --
	( 38.36, 26.03) --
	( 38.36, 26.03) --
	( 38.43, 26.04) --
	( 38.43, 26.04) --
	( 38.43, 26.04) --
	( 38.51, 26.05) --
	( 38.51, 26.05) --
	( 38.51, 26.05) --
	( 38.58, 26.05) --
	( 38.58, 26.05) --
	( 38.58, 26.05) --
	( 38.65, 26.06) --
	( 38.65, 26.06) --
	( 38.65, 26.06) --
	( 38.73, 26.07) --
	( 38.73, 26.07) --
	( 38.73, 26.07) --
	( 38.80, 26.08) --
	( 38.80, 26.08) --
	( 38.80, 26.08) --
	( 38.82, 26.08) --
	( 38.82, 26.08) --
	( 38.82, 26.08) --
	( 38.88, 26.08) --
	( 38.88, 26.08) --
	( 38.88, 26.08) --
	( 38.95, 26.09) --
	( 38.95, 26.09) --
	( 38.95, 26.09) --
	( 38.99, 26.10) --
	( 38.99, 26.10) --
	( 38.99, 26.10) --
	( 39.03, 26.10) --
	( 39.03, 26.10) --
	( 39.03, 26.10) --
	( 39.10, 26.12) --
	( 39.10, 26.12) --
	( 39.10, 26.12) --
	( 39.14, 26.13) --
	( 39.14, 26.13) --
	( 39.14, 26.13) --
	( 39.18, 26.13) --
	( 39.18, 26.13) --
	( 39.18, 26.13) --
	( 39.25, 26.15) --
	( 39.25, 26.15) --
	( 39.25, 26.15) --
	( 39.33, 26.16) --
	( 39.33, 26.16) --
	( 39.33, 26.16) --
	( 39.40, 26.18) --
	( 39.40, 26.18) --
	( 39.40, 26.18) --
	( 39.47, 26.19) --
	( 39.47, 26.19) --
	( 39.47, 26.19) --
	( 39.47, 26.19) --
	( 39.47, 26.19) --
	( 39.47, 26.19) --
	( 39.51, 26.20) --
	( 39.51, 26.20) --
	( 39.51, 26.20) --
	( 39.55, 26.21) --
	( 39.55, 26.21) --
	( 39.55, 26.21) --
	( 39.62, 26.22) --
	( 39.62, 26.22) --
	( 39.62, 26.22) --
	( 39.67, 26.23) --
	( 39.67, 26.23) --
	( 39.67, 26.23) --
	( 39.70, 26.24) --
	( 39.70, 26.24) --
	( 39.70, 26.24) --
	( 39.71, 26.24) --
	( 39.71, 26.24) --
	( 39.71, 26.24) --
	( 39.77, 26.25) --
	( 39.77, 26.25) --
	( 39.77, 26.25) --
	( 39.79, 26.26) --
	( 39.79, 26.26) --
	( 39.79, 26.26) --
	( 39.83, 26.27) --
	( 39.83, 26.27) --
	( 39.83, 26.27) --
	( 39.85, 26.27) --
	( 39.85, 26.27) --
	( 39.85, 26.27) --
	( 39.87, 26.27) --
	( 39.87, 26.27) --
	( 39.87, 26.27) --
	( 39.91, 26.28) --
	( 39.91, 26.28) --
	( 39.91, 26.28) --
	( 39.92, 26.28) --
	( 39.92, 26.28) --
	( 39.92, 26.28) --
	( 40.00, 26.30) --
	( 40.00, 26.30) --
	( 40.00, 26.30) --
	( 40.03, 26.31) --
	( 40.03, 26.31) --
	( 40.03, 26.31) --
	( 40.07, 26.31) --
	( 40.07, 26.31) --
	( 40.07, 26.31) --
	( 40.14, 26.33) --
	( 40.14, 26.33) --
	( 40.14, 26.33) --
	( 40.22, 26.34) --
	( 40.22, 26.34) --
	( 40.22, 26.34) --
	( 40.29, 26.36) --
	( 40.29, 26.36) --
	( 40.29, 26.36) --
	( 40.35, 26.37) --
	( 40.35, 26.37) --
	( 40.35, 26.37) --
	( 40.37, 26.37) --
	( 40.37, 26.37) --
	( 40.37, 26.37) --
	( 40.44, 26.39) --
	( 40.44, 26.39) --
	( 40.44, 26.39) --
	( 40.52, 26.40) --
	( 40.52, 26.40) --
	( 40.52, 26.40) --
	( 40.52, 26.41) --
	( 40.52, 26.41) --
	( 40.52, 26.41) --
	( 40.59, 26.44) --
	( 40.59, 26.44) --
	( 40.59, 26.44) --
	( 40.67, 26.48) --
	( 40.67, 26.48) --
	( 40.67, 26.48) --
	( 40.74, 26.52) --
	( 40.74, 26.52) --
	( 40.74, 26.52) --
	( 40.81, 26.55) --
	( 40.81, 26.55) --
	( 40.81, 26.55) --
	( 40.89, 26.59) --
	( 40.89, 26.59) --
	( 40.89, 26.59) --
	( 40.92, 26.61) --
	( 40.92, 26.61) --
	( 40.92, 26.61) --
	( 40.96, 26.63) --
	( 40.96, 26.63) --
	( 40.96, 26.63) --
	( 41.04, 26.67) --
	( 41.04, 26.67) --
	( 41.04, 26.67) --
	( 41.11, 26.70) --
	( 41.11, 26.70) --
	( 41.11, 26.70) --
	( 41.19, 26.74) --
	( 41.19, 26.74) --
	( 41.19, 26.74) --
	( 41.26, 26.78) --
	( 41.26, 26.78) --
	( 41.26, 26.78) --
	( 41.29, 26.79) --
	( 41.29, 26.79) --
	( 41.29, 26.79) --
	( 41.33, 26.85) --
	( 41.33, 26.85) --
	( 41.33, 26.85) --
	( 41.41, 26.94) --
	( 41.41, 26.94) --
	( 41.41, 26.94) --
	( 41.48, 27.03) --
	( 41.48, 27.03) --
	( 41.48, 27.03) --
	( 41.56, 27.12) --
	( 41.56, 27.12) --
	( 41.56, 27.12) --
	( 41.63, 27.21) --
	( 41.63, 27.21) --
	( 41.63, 27.21) --
	( 41.67, 27.26) --
	( 41.67, 27.26) --
	( 41.67, 27.26) --
	( 41.71, 27.31) --
	( 41.71, 27.31) --
	( 41.71, 27.31) --
	( 41.78, 27.42) --
	( 41.78, 27.42) --
	( 41.78, 27.42) --
	( 41.86, 27.53) --
	( 41.86, 27.53) --
	( 41.86, 27.53) --
	( 41.93, 27.64) --
	( 41.93, 27.64) --
	( 41.93, 27.64) --
	( 42.00, 27.75) --
	( 42.00, 27.75) --
	( 42.00, 27.75) --
	( 42.05, 27.82) --
	( 42.05, 27.82) --
	( 42.05, 27.82) --
	( 42.08, 27.86) --
	( 42.08, 27.86) --
	( 42.08, 27.86) --
	( 42.15, 27.97) --
	( 42.15, 27.97) --
	( 42.15, 27.97) --
	( 42.21, 28.07) --
	( 42.21, 28.07) --
	( 42.21, 28.07) --
	( 42.23, 28.09) --
	( 42.23, 28.09) --
	( 42.23, 28.09) --
	( 42.30, 28.20) --
	( 42.30, 28.20) --
	( 42.30, 28.20) --
	( 42.38, 28.32) --
	( 42.38, 28.32) --
	( 42.38, 28.32) --
	( 42.45, 28.43) --
	( 42.45, 28.43) --
	( 42.45, 28.43) --
	( 42.52, 28.55) --
	( 42.52, 28.55) --
	( 42.52, 28.55) --
	( 42.53, 28.56) --
	( 42.53, 28.56) --
	( 42.53, 28.56) --
	( 42.60, 28.67) --
	( 42.60, 28.67) --
	( 42.60, 28.67) --
	( 42.67, 28.80) --
	( 42.67, 28.80) --
	( 42.67, 28.80) --
	( 42.75, 28.94) --
	( 42.75, 28.94) --
	( 42.75, 28.94) --
	( 42.82, 29.06) --
	( 42.82, 29.06) --
	( 42.82, 29.06) --
	( 42.82, 29.07) --
	( 42.82, 29.07) --
	( 42.82, 29.07) --
	( 42.90, 29.29) --
	( 42.90, 29.29) --
	( 42.90, 29.29) --
	( 42.97, 29.52) --
	( 42.97, 29.52) --
	( 42.97, 29.52) --
	( 43.01, 29.65) --
	( 43.01, 29.65) --
	( 43.01, 29.65) --
	( 43.04, 29.71) --
	( 43.04, 29.71) --
	( 43.04, 29.71) --
	( 43.12, 29.87) --
	( 43.12, 29.87) --
	( 43.12, 29.87) --
	( 43.19, 30.02) --
	( 43.19, 30.02) --
	( 43.19, 30.02) --
	( 43.27, 30.18) --
	( 43.27, 30.18) --
	( 43.27, 30.18) --
	( 43.30, 30.25) --
	( 43.30, 30.25) --
	( 43.30, 30.25) --
	( 43.34, 30.36) --
	( 43.34, 30.36) --
	( 43.34, 30.36) --
	( 43.42, 30.56) --
	( 43.42, 30.56) --
	( 43.42, 30.56) --
	( 43.49, 30.77) --
	( 43.49, 30.77) --
	( 43.49, 30.77) --
	( 43.49, 30.77) --
	( 43.49, 30.77) --
	( 43.49, 30.77) --
	( 43.56, 31.30) --
	( 43.56, 31.30) --
	( 43.56, 31.30) --
	( 43.59, 31.47) --
	( 43.59, 31.47) --
	( 43.59, 31.47) --
	( 43.64, 31.55) --
	( 43.64, 31.55) --
	( 43.64, 31.56) --
	( 43.71, 31.68) --
	( 43.71, 31.68) --
	( 43.71, 31.68) --
	( 43.79, 31.80) --
	( 43.79, 31.80) --
	( 43.79, 31.80) --
	( 43.86, 31.93) --
	( 43.86, 31.93) --
	( 43.86, 31.93) --
	( 43.94, 32.05) --
	( 43.94, 32.05) --
	( 43.94, 32.05) --
	( 43.97, 32.11) --
	( 43.97, 32.69) --
	( 43.97, 32.69) --
	( 44.01, 32.81) --
	( 44.01, 32.81) --
	( 44.01, 32.81) --
	( 44.08, 33.04) --
	( 44.08, 33.04) --
	( 44.08, 33.04) --
	( 44.16, 33.26) --
	( 44.16, 33.26) --
	( 44.16, 33.26) --
	( 44.16, 33.27) --
	( 44.16, 33.27) --
	( 44.16, 33.27) --
	( 44.23, 33.41) --
	( 44.23, 33.41) --
	( 44.23, 33.41) --
	( 44.31, 33.55) --
	( 44.31, 33.55) --
	( 44.31, 33.55) --
	( 44.38, 33.69) --
	( 44.38, 33.69) --
	( 44.38, 33.69) --
	( 44.45, 33.82) --
	( 44.45, 34.42) --
	( 44.45, 34.42) --
	( 44.46, 34.44) --
	( 44.46, 34.44) --
	( 44.46, 34.44) --
	( 44.53, 34.69) --
	( 44.53, 34.69) --
	( 44.53, 34.69) --
	( 44.60, 34.93) --
	( 44.60, 34.93) --
	( 44.60, 34.93) --
	( 44.64, 35.06) --
	( 44.64, 35.06) --
	( 44.64, 35.06) --
	( 44.68, 35.27) --
	( 44.68, 35.27) --
	( 44.68, 35.27) --
	( 44.74, 35.60) --
	( 44.74, 36.14) --
	( 44.74, 36.14) --
	( 44.75, 36.19) --
	( 44.75, 36.19) --
	( 44.75, 36.19) --
	( 44.83, 36.40) --
	( 44.83, 36.40) --
	( 44.83, 36.40) --
	( 44.90, 36.61) --
	( 44.90, 36.61) --
	( 44.90, 36.61) --
	( 44.93, 36.69) --
	( 44.93, 36.69) --
	( 44.93, 36.69) --
	( 44.98, 36.97) --
	( 44.98, 36.97) --
	( 44.98, 36.97) --
	( 45.02, 37.27) --
	( 45.02, 37.83) --
	( 45.02, 37.83) --
	( 45.05, 37.87) --
	( 45.05, 37.87) --
	( 45.05, 37.87) --
	( 45.12, 37.99) --
	( 45.12, 37.99) --
	( 45.12, 37.99) --
	( 45.20, 38.11) --
	( 45.20, 38.11) --
	( 45.20, 38.11) --
	( 45.27, 38.23) --
	( 45.27, 38.23) --
	( 45.27, 38.23) --
	( 45.31, 38.30) --
	( 45.31, 38.84) --
	( 45.31, 38.84) --
	( 45.35, 39.02) --
	( 45.35, 39.02) --
	( 45.35, 39.02) --
	( 45.41, 39.36) --
	( 45.41, 39.83) --
	( 45.41, 39.83) --
	( 45.42, 39.89) --
	( 45.42, 39.89) --
	( 45.42, 39.89) --
	( 45.49, 40.27) --
	( 45.49, 40.27) --
	( 45.49, 40.27) --
	( 45.50, 40.31) --
	( 45.50, 40.31) --
	( 45.50, 40.31) --
	( 45.57, 40.58) --
	( 45.57, 40.58) --
	( 45.57, 40.58) --
	( 45.60, 40.70) --
	( 45.60, 40.70) --
	( 45.60, 40.70) --
	( 45.64, 40.96) --
	( 45.64, 40.96) --
	( 45.64, 40.96) --
	( 45.70, 41.27) --
	( 45.70, 41.27) --
	( 45.70, 41.27) --
	( 45.72, 41.37) --
	( 45.72, 41.37) --
	( 45.72, 41.37) --
	( 45.79, 41.71) --
	( 45.79, 41.71) --
	( 45.79, 41.71) --
	( 45.79, 41.71) --
	( 45.79, 41.71) --
	( 45.79, 41.71) --
	( 45.87, 41.87) --
	( 45.87, 41.87) --
	( 45.87, 41.87) --
	( 45.94, 42.03) --
	( 45.94, 42.03) --
	( 45.94, 42.03) --
	( 45.98, 42.12) --
	( 45.98, 42.12) --
	( 45.98, 42.12) --
	( 46.01, 42.23) --
	( 46.01, 42.23) --
	( 46.01, 42.23) --
	( 46.08, 42.49) --
	( 46.08, 42.86) --
	( 46.08, 42.86) --
	( 46.09, 42.90) --
	( 46.09, 42.90) --
	( 46.09, 42.90) --
	( 46.16, 43.29) --
	( 46.16, 43.29) --
	( 46.16, 43.29) --
	( 46.17, 43.36) --
	( 46.17, 43.36) --
	( 46.17, 43.36) --
	( 46.24, 43.40) --
	( 46.24, 43.40) --
	( 46.24, 43.40) --
	( 46.31, 43.45) --
	( 46.31, 43.45) --
	( 46.31, 43.45) --
	( 46.38, 43.50) --
	( 46.38, 43.50) --
	( 46.38, 43.50) --
	( 46.46, 43.55) --
	( 46.46, 43.55) --
	( 46.46, 43.55) --
	( 46.46, 43.55) --
	( 46.46, 43.55) --
	( 46.46, 43.55) --
	( 46.53, 43.50) --
	( 46.53, 43.50) --
	( 46.53, 43.50) --
	( 46.61, 43.44) --
	( 46.61, 43.44) --
	( 46.61, 43.44) --
	( 46.68, 43.38) --
	( 46.68, 43.38) --
	( 46.68, 43.38) --
	( 46.75, 43.32) --
	( 46.75, 43.32) --
	( 46.75, 43.32) --
	( 46.75, 43.31) --
	( 46.75, 43.31) --
	( 46.75, 43.31) --
	( 46.83, 43.15) --
	( 46.83, 43.15) --
	( 46.83, 43.15) --
	( 46.85, 43.11) --
	( 46.85, 43.11) --
	( 46.85, 43.11) --
	( 46.90, 42.96) --
	( 46.90, 42.96) --
	( 46.90, 42.96) --
	( 46.94, 42.86) --
	( 46.94, 42.86) --
	( 46.94, 42.86) --
	( 46.98, 42.81) --
	( 46.98, 42.81) --
	( 46.98, 42.81) --
	( 47.05, 42.72) --
	( 47.05, 42.72) --
	( 47.05, 42.72) --
	( 47.13, 42.63) --
	( 47.13, 42.63) --
	( 47.13, 42.63) --
	( 47.13, 42.62) --
	( 47.13, 42.35) --
	( 47.13, 42.35) --
	( 47.20, 42.14) --
	( 47.20, 42.14) --
	( 47.20, 42.14) --
	( 47.23, 42.04) --
	( 47.23, 41.05) --
	( 47.23, 41.05) --
	( 47.27, 41.31) --
	( 47.27, 41.31) --
	( 47.27, 41.31) --
	( 47.32, 41.61) --
	( 47.32, 41.61) --
	( 47.32, 41.61) --
	( 47.35, 41.49) --
	( 47.35, 41.49) --
	( 47.35, 41.49) --
	( 47.42, 41.08) --
	( 47.42, 41.08) --
	( 47.42, 41.08) --
	( 47.50, 40.67) --
	( 47.50, 40.67) --
	( 47.50, 40.67) --
	( 47.52, 40.55) --
	( 47.52, 40.55) --
	( 47.52, 40.55) --
	( 47.57, 40.33) --
	( 47.57, 40.33) --
	( 47.57, 40.33) --
	( 47.61, 40.16) --
	( 47.61, 39.67) --
	( 47.61, 39.67) --
	( 47.64, 39.62) --
	( 47.64, 39.62) --
	( 47.64, 39.62) --
	( 47.72, 39.49) --
	( 47.72, 39.49) --
	( 47.72, 39.49) --
	( 47.79, 39.37) --
	( 47.79, 39.37) --
	( 47.79, 39.37) --
	( 47.80, 39.34) --
	( 47.80, 38.82) --
	( 47.80, 38.82) --
	( 47.86, 38.54) --
	( 47.86, 38.54) --
	( 47.86, 38.54) --
	( 47.90, 38.37) --
	( 47.90, 38.37) --
	( 47.90, 38.37) --
	( 47.94, 38.18) --
	( 47.94, 38.18) --
	( 47.94, 38.18) --
	( 48.00, 37.91) --
	( 48.00, 37.91) --
	( 48.00, 37.91) --
	( 48.01, 37.87) --
	( 48.01, 37.87) --
	( 48.01, 37.87) --
	( 48.09, 37.69) --
	( 48.09, 37.69) --
	( 48.09, 37.69) --
	( 48.16, 37.51) --
	( 48.16, 37.51) --
	( 48.16, 37.51) --
	( 48.19, 37.44) --
	( 48.19, 37.44) --
	( 48.19, 37.44) --
	( 48.23, 37.18) --
	( 48.23, 37.18) --
	( 48.23, 37.18) --
	( 48.28, 36.90) --
	( 48.28, 36.28) --
	( 48.28, 36.28) --
	( 48.31, 36.20) --
	( 48.31, 36.20) --
	( 48.31, 36.20) --
	( 48.38, 35.99) --
	( 48.38, 35.99) --
	( 48.38, 35.99) --
	( 48.46, 35.77) --
	( 48.46, 35.77) --
	( 48.46, 35.77) --
	( 48.47, 35.72) --
	( 48.47, 35.72) --
	( 48.47, 35.72) --
	( 48.53, 35.51) --
	( 48.53, 35.51) --
	( 48.53, 35.51) --
	( 48.60, 35.24) --
	( 48.60, 35.24) --
	( 48.60, 35.24) --
	( 48.67, 35.02) --
	( 48.67, 35.02) --
	( 48.67, 35.02) --
	( 48.68, 34.93) --
	( 48.68, 34.93) --
	( 48.68, 34.93) --
	( 48.75, 34.39) --
	( 48.75, 34.39) --
	( 48.75, 34.39) --
	( 48.76, 34.32) --
	( 48.76, 34.32) --
	( 48.76, 34.32) --
	( 48.83, 34.13) --
	( 48.83, 34.13) --
	( 48.83, 34.13) --
	( 48.90, 33.91) --
	( 48.90, 33.91) --
	( 48.90, 33.91) --
	( 48.95, 33.76) --
	( 48.95, 33.76) --
	( 48.95, 33.76) --
	( 48.97, 33.63) --
	( 48.97, 33.63) --
	( 48.97, 33.63) --
	( 49.05, 33.15) --
	( 49.05, 33.15) --
	( 49.05, 33.15) --
	( 49.05, 33.14) --
	( 49.05, 33.14) --
	( 49.05, 33.14) --
	( 49.12, 33.02) --
	( 49.12, 33.02) --
	( 49.12, 33.02) --
	( 49.20, 32.91) --
	( 49.20, 32.91) --
	( 49.20, 32.91) --
	( 49.27, 32.79) --
	( 49.27, 32.79) --
	( 49.27, 32.79) --
	( 49.34, 32.69) --
	( 49.34, 32.69) --
	( 49.34, 32.69) --
	( 49.34, 32.67) --
	( 49.34, 32.67) --
	( 49.34, 32.67) --
	( 49.42, 32.43) --
	( 49.42, 32.43) --
	( 49.42, 32.43) --
	( 49.49, 32.19) --
	( 49.49, 32.19) --
	( 49.49, 32.19) --
	( 49.53, 32.07) --
	( 49.53, 32.07) --
	( 49.53, 32.07) --
	( 49.57, 31.94) --
	( 49.57, 31.94) --
	( 49.57, 31.94) --
	( 49.64, 31.70) --
	( 49.64, 31.70) --
	( 49.64, 31.70) --
	( 49.71, 31.45) --
	( 49.71, 31.45) --
	( 49.71, 31.45) --
	( 49.72, 31.43) --
	( 49.72, 31.43) --
	( 49.72, 31.43) --
	( 49.79, 31.27) --
	( 49.79, 31.27) --
	( 49.79, 31.27) --
	( 49.86, 31.09) --
	( 49.86, 31.09) --
	( 49.86, 31.09) --
	( 49.91, 30.96) --
	( 49.91, 30.96) --
	( 49.91, 30.96) --
	( 49.94, 30.90) --
	( 49.94, 30.90) --
	( 49.94, 30.90) --
	( 50.01, 30.72) --
	( 50.01, 30.72) --
	( 50.01, 30.72) --
	( 50.08, 30.53) --
	( 50.08, 30.53) --
	( 50.08, 30.53) --
	( 50.10, 30.48) --
	( 50.10, 30.48) --
	( 50.10, 30.48) --
	( 50.16, 30.41) --
	( 50.16, 30.41) --
	( 50.16, 30.41) --
	( 50.23, 30.32) --
	( 50.23, 30.32) --
	( 50.23, 30.32) --
	( 50.30, 30.24) --
	( 50.30, 30.24) --
	( 50.30, 30.24) --
	( 50.38, 30.15) --
	( 50.38, 30.15) --
	( 50.38, 30.15) --
	( 50.45, 30.06) --
	( 50.45, 30.06) --
	( 50.45, 30.06) --
	( 50.49, 30.01) --
	( 50.49, 30.01) --
	( 50.49, 30.01) --
	( 50.53, 29.97) --
	( 50.53, 29.97) --
	( 50.53, 29.97) --
	( 50.60, 29.89) --
	( 50.60, 29.89) --
	( 50.60, 29.89) --
	( 50.67, 29.81) --
	( 50.67, 29.81) --
	( 50.67, 29.81) --
	( 50.75, 29.72) --
	( 50.75, 29.72) --
	( 50.75, 29.72) --
	( 50.82, 29.64) --
	( 50.82, 29.64) --
	( 50.82, 29.64) --
	( 50.87, 29.59) --
	( 50.87, 29.59) --
	( 50.87, 29.59) --
	( 50.90, 29.51) --
	( 50.90, 29.51) --
	( 50.90, 29.51) --
	( 50.97, 29.30) --
	( 50.97, 29.30) --
	( 50.97, 29.30) --
	( 50.97, 29.29) --
	( 50.97, 29.29) --
	( 50.97, 29.29) --
	( 51.04, 29.21) --
	( 51.04, 29.21) --
	( 51.04, 29.21) --
	( 51.12, 29.14) --
	( 51.12, 29.14) --
	( 51.12, 29.14) --
	( 51.19, 29.06) --
	( 51.19, 29.06) --
	( 51.19, 29.06) --
	( 51.27, 28.98) --
	( 51.27, 28.98) --
	( 51.27, 28.98) --
	( 51.34, 28.90) --
	( 51.34, 28.90) --
	( 51.34, 28.90) --
	( 51.35, 28.89) --
	( 51.35, 28.89) --
	( 51.35, 28.89) --
	( 51.41, 28.84) --
	( 51.41, 28.84) --
	( 51.41, 28.84) --
	( 51.49, 28.79) --
	( 51.49, 28.79) --
	( 51.49, 28.79) --
	( 51.56, 28.74) --
	( 51.56, 28.74) --
	( 51.56, 28.74) --
	( 51.63, 28.69) --
	( 51.63, 28.69) --
	( 51.63, 28.69) --
	( 51.71, 28.63) --
	( 51.71, 28.63) --
	( 51.71, 28.63) --
	( 51.73, 28.62) --
	( 51.73, 28.62) --
	( 51.73, 28.62) --
	( 51.78, 28.60) --
	( 51.78, 28.60) --
	( 51.78, 28.60) --
	( 51.86, 28.56) --
	( 51.86, 28.56) --
	( 51.86, 28.56) --
	( 51.93, 28.53) --
	( 51.93, 28.53) --
	( 51.93, 28.53) --
	( 52.00, 28.50) --
	( 52.00, 28.50) --
	( 52.00, 28.50) --
	( 52.08, 28.46) --
	( 52.08, 28.46) --
	( 52.08, 28.46) --
	( 52.15, 28.43) --
	( 52.15, 28.43) --
	( 52.15, 28.43) --
	( 52.21, 28.40) --
	( 52.21, 28.40) --
	( 52.21, 28.40) --
	( 52.22, 28.40) --
	( 52.22, 28.40) --
	( 52.22, 28.40) --
	( 52.30, 28.38) --
	( 52.30, 28.38) --
	( 52.30, 28.38) --
	( 52.37, 28.35) --
	( 52.37, 28.35) --
	( 52.37, 28.35) --
	( 52.45, 28.33) --
	( 52.45, 28.33) --
	( 52.45, 28.33) --
	( 52.52, 28.30) --
	( 52.52, 28.30) --
	( 52.52, 28.30) --
	( 52.59, 28.28) --
	( 52.59, 28.28) --
	( 52.59, 28.28) --
	( 52.67, 28.26) --
	( 52.67, 28.26) --
	( 52.67, 28.26) --
	( 52.69, 28.25) --
	( 52.69, 28.25) --
	( 52.69, 28.25) --
	( 52.74, 28.24) --
	( 52.74, 28.24) --
	( 52.74, 28.24) --
	( 52.81, 28.23) --
	( 52.81, 28.23) --
	( 52.81, 28.23) --
	( 52.89, 28.22) --
	( 52.89, 28.22) --
	( 52.89, 28.22) --
	( 52.96, 28.22) --
	( 52.96, 28.22) --
	( 52.96, 28.22) --
	( 53.04, 28.21) --
	( 53.04, 28.21) --
	( 53.04, 28.21) --
	( 53.11, 28.20) --
	( 53.11, 28.20) --
	( 53.11, 28.20) --
	( 53.17, 28.19) --
	( 53.17, 28.19) --
	( 53.17, 28.19) --
	( 53.18, 28.19) --
	( 53.18, 28.19) --
	( 53.18, 28.19) --
	( 53.26, 28.20) --
	( 53.26, 28.20) --
	( 53.26, 28.20) --
	( 53.33, 28.22) --
	( 53.33, 28.22) --
	( 53.33, 28.22) --
	( 53.41, 28.23) --
	( 53.41, 28.23) --
	( 53.41, 28.23) --
	( 53.48, 28.24) --
	( 53.48, 28.24) --
	( 53.48, 28.24) --
	( 53.55, 28.25) --
	( 53.55, 28.25) --
	( 53.55, 28.25) --
	( 53.62, 28.27) --
	( 53.62, 28.27) --
	( 53.62, 28.27) --
	( 53.70, 28.28) --
	( 53.70, 28.28) --
	( 53.70, 28.28) --
	( 53.75, 28.29) --
	( 53.75, 28.29) --
	( 53.75, 28.29) --
	( 53.77, 28.30) --
	( 53.77, 28.30) --
	( 53.77, 28.30) --
	( 53.85, 28.34) --
	( 53.85, 28.34) --
	( 53.85, 28.34) --
	( 53.92, 28.37) --
	( 53.92, 28.37) --
	( 53.92, 28.37) --
	( 53.99, 28.41) --
	( 53.99, 28.41) --
	( 53.99, 28.41) --
	( 54.07, 28.44) --
	( 54.07, 28.44) --
	( 54.07, 28.44) --
	( 54.14, 28.48) --
	( 54.14, 28.48) --
	( 54.14, 28.48) --
	( 54.21, 28.51) --
	( 54.21, 28.51) --
	( 54.21, 28.51) --
	( 54.29, 28.55) --
	( 54.29, 28.55) --
	( 54.29, 28.55) --
	( 54.36, 28.58) --
	( 54.36, 28.58) --
	( 54.36, 28.58) --
	( 54.44, 28.62) --
	( 54.44, 28.62) --
	( 54.44, 28.62) --
	( 54.51, 28.65) --
	( 54.51, 28.65) --
	( 54.51, 28.65) --
	( 54.51, 28.66) --
	( 54.51, 28.66) --
	( 54.51, 28.66) --
	( 54.58, 28.69) --
	( 54.58, 28.69) --
	( 54.58, 28.69) --
	( 54.66, 28.74) --
	( 54.66, 28.74) --
	( 54.66, 28.74) --
	( 54.73, 28.78) --
	( 54.73, 28.78) --
	( 54.73, 28.78) --
	( 54.80, 28.82) --
	( 54.80, 28.82) --
	( 54.80, 28.82) --
	( 54.88, 28.86) --
	( 54.88, 28.86) --
	( 54.88, 28.86) --
	( 54.95, 28.90) --
	( 54.95, 28.90) --
	( 54.95, 28.90) --
	( 55.02, 28.94) --
	( 55.02, 28.94) --
	( 55.02, 28.94) --
	( 55.10, 28.98) --
	( 55.10, 28.98) --
	( 55.10, 28.98) --
	( 55.17, 29.02) --
	( 55.17, 29.02) --
	( 55.17, 29.02) --
	( 55.25, 29.06) --
	( 55.25, 29.06) --
	( 55.25, 29.06) --
	( 55.28, 29.08) --
	( 55.28, 29.08) --
	( 55.28, 29.08) --
	( 55.32, 29.14) --
	( 55.32, 29.14) --
	( 55.32, 29.14) --
	( 55.39, 29.26) --
	( 55.39, 29.26) --
	( 55.39, 29.26) --
	( 55.47, 29.37) --
	( 55.47, 29.37) --
	( 55.47, 29.37) --
	( 55.54, 29.49) --
	( 55.54, 29.49) --
	( 55.54, 29.49) --
	( 55.57, 29.53) --
	( 55.57, 29.53) --
	( 55.57, 29.53) --
	( 55.61, 29.56) --
	( 55.61, 29.56) --
	( 55.61, 29.56) --
	( 55.69, 29.62) --
	( 55.69, 29.62) --
	( 55.69, 29.62) --
	( 55.76, 29.67) --
	( 55.76, 29.67) --
	( 55.76, 29.67) --
	( 55.83, 29.73) --
	( 55.83, 29.73) --
	( 55.83, 29.73) --
	( 55.91, 29.78) --
	( 55.91, 29.78) --
	( 55.91, 29.78) --
	( 55.98, 29.84) --
	( 55.98, 29.84) --
	( 55.98, 29.84) --
	( 56.06, 29.89) --
	( 56.06, 29.89) --
	( 56.06, 29.89) --
	( 56.13, 29.95) --
	( 56.13, 29.95) --
	( 56.13, 29.95) --
	( 56.14, 29.96) --
	( 56.14, 29.96) --
	( 56.14, 29.96) --
	( 56.20, 30.03) --
	( 56.20, 30.03) --
	( 56.20, 30.03) --
	( 56.27, 30.13) --
	( 56.27, 30.13) --
	( 56.27, 30.13) --
	( 56.35, 30.22) --
	( 56.35, 30.22) --
	( 56.35, 30.22) --
	( 56.42, 30.31) --
	( 56.42, 30.31) --
	( 56.42, 30.31) --
	( 56.43, 30.32) --
	( 56.43, 30.32) --
	( 56.43, 30.32) --
	( 56.50, 30.39) --
	( 56.50, 30.39) --
	( 56.50, 30.39) --
	( 56.57, 30.45) --
	( 56.57, 30.45) --
	( 56.57, 30.45) --
	( 56.64, 30.52) --
	( 56.64, 30.52) --
	( 56.64, 30.52) --
	( 56.72, 30.59) --
	( 56.72, 30.59) --
	( 56.72, 30.59) --
	( 56.79, 30.66) --
	( 56.79, 30.66) --
	( 56.79, 30.66) --
	( 56.86, 30.73) --
	( 56.86, 30.73) --
	( 56.86, 30.73) --
	( 56.91, 30.77) --
	( 56.91, 30.77) --
	( 56.91, 30.77) --
	( 56.94, 30.81) --
	( 56.94, 30.81) --
	( 56.94, 30.81) --
	( 57.01, 30.93) --
	( 57.01, 30.93) --
	( 57.01, 30.93) --
	( 57.08, 31.04) --
	( 57.08, 31.04) --
	( 57.08, 31.04) --
	( 57.16, 31.16) --
	( 57.16, 31.16) --
	( 57.16, 31.16) --
	( 57.20, 31.22) --
	( 57.20, 31.22) --
	( 57.20, 31.22) --
	( 57.23, 31.24) --
	( 57.23, 31.24) --
	( 57.23, 31.24) --
	( 57.30, 31.30) --
	( 57.30, 31.30) --
	( 57.30, 31.30) --
	( 57.38, 31.36) --
	( 57.38, 31.36) --
	( 57.38, 31.36) --
	( 57.45, 31.42) --
	( 57.45, 31.42) --
	( 57.45, 31.42) --
	( 57.52, 31.47) --
	( 57.52, 31.47) --
	( 57.52, 31.47) --
	( 57.60, 31.53) --
	( 57.60, 31.53) --
	( 57.60, 31.53) --
	( 57.67, 31.59) --
	( 57.67, 31.59) --
	( 57.67, 31.59) --
	( 57.74, 31.64) --
	( 57.74, 31.64) --
	( 57.74, 31.64) --
	( 57.82, 31.70) --
	( 57.82, 31.70) --
	( 57.82, 31.70) --
	( 57.87, 31.74) --
	( 57.87, 31.74) --
	( 57.87, 31.74) --
	( 57.89, 31.76) --
	( 57.89, 31.76) --
	( 57.89, 31.76) --
	( 57.97, 31.82) --
	( 57.97, 31.82) --
	( 57.97, 31.82) --
	( 58.04, 31.89) --
	( 58.04, 31.89) --
	( 58.04, 31.89) --
	( 58.11, 31.95) --
	( 58.11, 31.95) --
	( 58.11, 31.95) --
	( 58.18, 32.01) --
	( 58.18, 32.01) --
	( 58.18, 32.01) --
	( 58.26, 32.07) --
	( 58.26, 32.07) --
	( 58.26, 32.07) --
	( 58.33, 32.14) --
	( 58.33, 32.14) --
	( 58.33, 32.14) --
	( 58.35, 32.15) --
	( 58.35, 32.15) --
	( 58.35, 32.15) --
	( 58.41, 32.20) --
	( 58.41, 32.20) --
	( 58.41, 32.20) --
	( 58.48, 32.26) --
	( 58.48, 32.26) --
	( 58.48, 32.26) --
	( 58.55, 32.32) --
	( 58.55, 32.32) --
	( 58.55, 32.32) --
	( 58.63, 32.37) --
	( 58.63, 32.37) --
	( 58.63, 32.37) --
	( 58.70, 32.43) --
	( 58.70, 32.43) --
	( 58.70, 32.43) --
	( 58.77, 32.49) --
	( 58.77, 32.49) --
	( 58.77, 32.49) --
	( 58.85, 32.55) --
	( 58.85, 32.55) --
	( 58.85, 32.55) --
	( 58.92, 32.61) --
	( 58.92, 32.61) --
	( 58.92, 32.61) --
	( 58.92, 32.61) --
	( 58.92, 32.61) --
	( 58.92, 32.61) --
	( 58.99, 32.66) --
	( 58.99, 32.66) --
	( 58.99, 32.66) --
	( 59.07, 32.72) --
	( 59.07, 32.72) --
	( 59.07, 32.72) --
	( 59.14, 32.77) --
	( 59.14, 32.77) --
	( 59.14, 32.77) --
	( 59.21, 32.82) --
	( 59.21, 32.82) --
	( 59.21, 32.82) --
	( 59.29, 32.87) --
	( 59.29, 32.87) --
	( 59.29, 32.87) --
	( 59.36, 32.92) --
	( 59.36, 32.92) --
	( 59.36, 32.92) --
	( 59.43, 32.98) --
	( 59.43, 32.98) --
	( 59.43, 32.98) --
	( 59.50, 33.02) --
	( 59.50, 33.02) --
	( 59.50, 33.02) --
	( 59.51, 33.03) --
	( 59.51, 33.03) --
	( 59.51, 33.03) --
	( 59.58, 33.07) --
	( 59.58, 33.07) --
	( 59.58, 33.07) --
	( 59.65, 33.12) --
	( 59.65, 33.12) --
	( 59.65, 33.12) --
	( 59.73, 33.16) --
	( 59.73, 33.16) --
	( 59.73, 33.16) --
	( 59.80, 33.21) --
	( 59.80, 33.21) --
	( 59.80, 33.21) --
	( 59.87, 33.25) --
	( 59.87, 33.25) --
	( 59.87, 33.25) --
	( 59.95, 33.29) --
	( 59.95, 33.29) --
	( 59.95, 33.29) --
	( 60.02, 33.34) --
	( 60.02, 33.34) --
	( 60.02, 33.34) --
	( 60.09, 33.38) --
	( 60.09, 33.38) --
	( 60.09, 33.38) --
	( 60.17, 33.43) --
	( 60.17, 33.43) --
	( 60.17, 33.43) --
	( 60.24, 33.47) --
	( 60.24, 33.47) --
	( 60.24, 33.47) --
	( 60.31, 33.52) --
	( 60.31, 33.52) --
	( 60.31, 33.52) --
	( 60.39, 33.56) --
	( 60.39, 33.56) --
	( 60.39, 33.56) --
	( 60.45, 33.60) --
	( 60.45, 33.60) --
	( 60.45, 33.60) --
	( 60.46, 33.60) --
	( 60.46, 33.60) --
	( 60.46, 33.60) --
	( 60.53, 33.64) --
	( 60.53, 33.64) --
	( 60.53, 33.64) --
	( 60.61, 33.67) --
	( 60.61, 33.67) --
	( 60.61, 33.67) --
	( 60.68, 33.70) --
	( 60.68, 33.70) --
	( 60.68, 33.70) --
	( 60.75, 33.73) --
	( 60.75, 33.73) --
	( 60.75, 33.73) --
	( 60.83, 33.76) --
	( 60.83, 33.76) --
	( 60.83, 33.76) --
	( 60.90, 33.79) --
	( 60.90, 33.79) --
	( 60.90, 33.79) --
	( 60.97, 33.82) --
	( 60.97, 33.82) --
	( 60.97, 33.82) --
	( 61.04, 33.85) --
	( 61.04, 33.85) --
	( 61.04, 33.85) --
	( 61.12, 33.89) --
	( 61.12, 33.89) --
	( 61.12, 33.89) --
	( 61.19, 33.92) --
	( 61.19, 33.92) --
	( 61.19, 33.92) --
	( 61.26, 33.95) --
	( 61.26, 33.95) --
	( 61.26, 33.95) --
	( 61.34, 33.98) --
	( 61.34, 33.98) --
	( 61.34, 33.98) --
	( 61.41, 34.01) --
	( 61.41, 34.01) --
	( 61.41, 34.01) --
	( 61.41, 34.01) --
	( 61.41, 34.01) --
	( 61.41, 34.01) --
	( 61.48, 34.03) --
	( 61.48, 34.03) --
	( 61.48, 34.03) --
	( 61.56, 34.04) --
	( 61.56, 34.04) --
	( 61.56, 34.04) --
	( 61.63, 34.06) --
	( 61.63, 34.06) --
	( 61.63, 34.06) --
	( 61.70, 34.08) --
	( 61.70, 34.08) --
	( 61.70, 34.08) --
	( 61.78, 34.10) --
	( 61.78, 34.10) --
	( 61.78, 34.10) --
	( 61.85, 34.11) --
	( 61.85, 34.11) --
	( 61.85, 34.11) --
	( 61.92, 34.13) --
	( 61.92, 34.13) --
	( 61.92, 34.13) --
	( 62.00, 34.15) --
	( 62.00, 34.15) --
	( 62.00, 34.15) --
	( 62.07, 34.16) --
	( 62.07, 34.16) --
	( 62.07, 34.16) --
	( 62.14, 34.18) --
	( 62.14, 34.18) --
	( 62.14, 34.18) --
	( 62.22, 34.20) --
	( 62.22, 34.20) --
	( 62.22, 34.20) --
	( 62.29, 34.22) --
	( 62.29, 34.22) --
	( 62.29, 34.22) --
	( 62.36, 34.23) --
	( 62.36, 34.23) --
	( 62.36, 34.23) --
	( 62.44, 34.25) --
	( 62.44, 34.25) --
	( 62.44, 34.25) --
	( 62.51, 34.27) --
	( 62.51, 34.27) --
	( 62.51, 34.27) --
	( 62.58, 34.28) --
	( 62.58, 34.28) --
	( 62.58, 34.28) --
	( 62.65, 34.30) --
	( 62.65, 34.30) --
	( 62.65, 34.30) --
	( 62.73, 34.32) --
	( 62.73, 34.32) --
	( 62.73, 34.32) --
	( 62.80, 34.34) --
	( 62.80, 34.34) --
	( 62.80, 34.34) --
	( 62.87, 34.35) --
	( 62.87, 34.35) --
	( 62.87, 34.35) --
	( 62.95, 34.37) --
	( 62.95, 34.37) --
	( 62.95, 34.37) --
	( 63.02, 34.39) --
	( 63.02, 34.39) --
	( 63.02, 34.39) --
	( 63.09, 34.40) --
	( 63.09, 34.40) --
	( 63.09, 34.40) --
	( 63.17, 34.42) --
	( 63.17, 34.42) --
	( 63.17, 34.42) --
	( 63.23, 34.44) --
	( 63.23, 34.44) --
	( 63.23, 34.44) --
	( 63.24, 34.44) --
	( 63.24, 34.44) --
	( 63.24, 34.44) --
	( 63.31, 34.44) --
	( 63.31, 34.44) --
	( 63.31, 34.44) --
	( 63.39, 34.45) --
	( 63.39, 34.45) --
	( 63.39, 34.45) --
	( 63.46, 34.46) --
	( 63.46, 34.46) --
	( 63.46, 34.46) --
	( 63.53, 34.47) --
	( 63.53, 34.47) --
	( 63.53, 34.47) --
	( 63.61, 34.47) --
	( 63.61, 34.47) --
	( 63.61, 34.47) --
	( 63.68, 34.48) --
	( 63.68, 34.48) --
	( 63.68, 34.48) --
	( 63.75, 34.49) --
	( 63.75, 34.49) --
	( 63.75, 34.49) --
	( 63.83, 34.50) --
	( 63.83, 34.50) --
	( 63.83, 34.50) --
	( 63.90, 34.50) --
	( 63.90, 34.50) --
	( 63.90, 34.50) --
	( 63.97, 34.51) --
	( 63.97, 34.51) --
	( 63.97, 34.51) --
	( 64.04, 34.52) --
	( 64.04, 34.52) --
	( 64.04, 34.52) --
	( 64.12, 34.53) --
	( 64.12, 34.53) --
	( 64.12, 34.53) --
	( 64.19, 34.53) --
	( 64.19, 34.53) --
	( 64.19, 34.53) --
	( 64.26, 34.54) --
	( 64.26, 34.54) --
	( 64.26, 34.54) --
	( 64.34, 34.55) --
	( 64.34, 34.55) --
	( 64.34, 34.55) --
	( 64.41, 34.56) --
	( 64.41, 34.56) --
	( 64.41, 34.56) --
	( 64.48, 34.56) --
	( 64.48, 34.56) --
	( 64.48, 34.56) --
	( 64.56, 34.57) --
	( 64.56, 34.57) --
	( 64.56, 34.57) --
	( 64.57, 34.57) --
	( 64.57, 34.57) --
	( 64.57, 34.57) --
	( 64.63, 34.57) --
	( 64.63, 34.57) --
	( 64.63, 34.57) --
	( 64.70, 34.56) --
	( 64.70, 34.56) --
	( 64.70, 34.56) --
	( 64.77, 34.56) --
	( 64.77, 34.56) --
	( 64.77, 34.56) --
	( 64.85, 34.56) --
	( 64.85, 34.56) --
	( 64.85, 34.56) --
	( 64.92, 34.55) --
	( 64.92, 34.55) --
	( 64.92, 34.55) --
	( 64.99, 34.55) --
	( 64.99, 34.55) --
	( 64.99, 34.55) --
	( 65.07, 34.54) --
	( 65.07, 34.54) --
	( 65.07, 34.54) --
	( 65.14, 34.54) --
	( 65.14, 34.54) --
	( 65.14, 34.54) --
	( 65.21, 34.53) --
	( 65.21, 34.53) --
	( 65.21, 34.53) --
	( 65.29, 34.53) --
	( 65.29, 34.53) --
	( 65.29, 34.53) --
	( 65.36, 34.52) --
	( 65.36, 34.52) --
	( 65.36, 34.52) --
	( 65.43, 34.52) --
	( 65.43, 34.52) --
	( 65.43, 34.52) --
	( 65.50, 34.51) --
	( 65.50, 34.51) --
	( 65.50, 34.51) --
	( 65.58, 34.51) --
	( 65.58, 34.51) --
	( 65.58, 34.51) --
	( 65.65, 34.50) --
	( 65.65, 34.50) --
	( 65.65, 34.50) --
	( 65.72, 34.50) --
	( 65.72, 34.50) --
	( 65.72, 34.50) --
	( 65.80, 34.49) --
	( 65.80, 34.49) --
	( 65.80, 34.49) --
	( 65.87, 34.49) --
	( 65.87, 34.49) --
	( 65.87, 34.49) --
	( 65.94, 34.49) --
	( 65.94, 34.49) --
	( 65.94, 34.49) --
	( 66.02, 34.48) --
	( 66.02, 34.48) --
	( 66.02, 34.48) --
	( 66.09, 34.48) --
	( 66.09, 34.48) --
	( 66.09, 34.48) --
	( 66.16, 34.47) --
	( 66.16, 34.47) --
	( 66.16, 34.47) --
	( 66.23, 34.47) --
	( 66.23, 34.47) --
	( 66.23, 34.47) --
	( 66.31, 34.46) --
	( 66.31, 34.46) --
	( 66.31, 34.46) --
	( 66.38, 34.46) --
	( 66.38, 34.46) --
	( 66.38, 34.46) --
	( 66.40, 34.46) --
	( 66.40, 34.46) --
	( 66.40, 34.46) --
	( 66.45, 34.45) --
	( 66.45, 34.45) --
	( 66.45, 34.45) --
	( 66.53, 34.44) --
	( 66.53, 34.44) --
	( 66.53, 34.44) --
	( 66.60, 34.43) --
	( 66.60, 34.43) --
	( 66.60, 34.43) --
	( 66.67, 34.42) --
	( 66.67, 34.42) --
	( 66.67, 34.42) --
	( 66.74, 34.40) --
	( 66.74, 34.40) --
	( 66.74, 34.40) --
	( 66.82, 34.39) --
	( 66.82, 34.39) --
	( 66.82, 34.39) --
	( 66.89, 34.38) --
	( 66.89, 34.38) --
	( 66.89, 34.38) --
	( 66.96, 34.37) --
	( 66.96, 34.37) --
	( 66.96, 34.37) --
	( 67.04, 34.36) --
	( 67.04, 34.36) --
	( 67.04, 34.36) --
	( 67.11, 34.35) --
	( 67.11, 34.35) --
	( 67.11, 34.35) --
	( 67.18, 34.34) --
	( 67.18, 34.34) --
	( 67.18, 34.34) --
	( 67.25, 34.33) --
	( 67.25, 34.33) --
	( 67.25, 34.33) --
	( 67.33, 34.32) --
	( 67.33, 34.32) --
	( 67.33, 34.32) --
	( 67.40, 34.31) --
	( 67.40, 34.31) --
	( 67.40, 34.31) --
	( 67.47, 34.30) --
	( 67.47, 34.30) --
	( 67.47, 34.30) --
	( 67.55, 34.28) --
	( 67.55, 34.28) --
	( 67.55, 34.28) --
	( 67.62, 34.27) --
	( 67.62, 34.27) --
	( 67.62, 34.27) --
	( 67.69, 34.26) --
	( 67.69, 34.26) --
	( 67.69, 34.26) --
	( 67.76, 34.25) --
	( 67.76, 34.25) --
	( 67.76, 34.25) --
	( 67.84, 34.24) --
	( 67.84, 34.24) --
	( 67.84, 34.24) --
	( 67.91, 34.23) --
	( 67.91, 34.23) --
	( 67.91, 34.23) --
	( 67.98, 34.22) --
	( 67.98, 34.22) --
	( 67.98, 34.22) --
	( 68.05, 34.21) --
	( 68.05, 34.21) --
	( 68.05, 34.21) --
	( 68.13, 34.20) --
	( 68.13, 34.20) --
	( 68.13, 34.20) --
	( 68.20, 34.19) --
	( 68.20, 34.19) --
	( 68.20, 34.19) --
	( 68.22, 34.18) --
	( 68.22, 34.18) --
	( 68.22, 34.18) --
	( 68.27, 34.17) --
	( 68.27, 34.17) --
	( 68.27, 34.17) --
	( 68.35, 34.14) --
	( 68.35, 34.14) --
	( 68.35, 34.14) --
	( 68.42, 34.12) --
	( 68.42, 34.12) --
	( 68.42, 34.12) --
	( 68.49, 34.10) --
	( 68.49, 34.10) --
	( 68.49, 34.10) --
	( 68.56, 34.08) --
	( 68.56, 34.08) --
	( 68.56, 34.08) --
	( 68.64, 34.05) --
	( 68.64, 34.05) --
	( 68.64, 34.05) --
	( 68.71, 34.03) --
	( 68.71, 34.03) --
	( 68.71, 34.03) --
	( 68.78, 34.01) --
	( 68.78, 34.01) --
	( 68.78, 34.01) --
	( 68.86, 33.98) --
	( 68.86, 33.98) --
	( 68.86, 33.98) --
	( 68.93, 33.96) --
	( 68.93, 33.96) --
	( 68.93, 33.96) --
	( 69.00, 33.94) --
	( 69.00, 33.94) --
	( 69.00, 33.94) --
	( 69.07, 33.92) --
	( 69.07, 33.92) --
	( 69.07, 33.92) --
	( 69.15, 33.89) --
	( 69.15, 33.89) --
	( 69.15, 33.89) --
	( 69.22, 33.87) --
	( 69.22, 33.87) --
	( 69.22, 33.87) --
	( 69.27, 33.85) --
	( 69.27, 33.85) --
	( 69.27, 33.85) --
	( 69.29, 33.85) --
	( 69.29, 33.85) --
	( 69.29, 33.85) --
	( 69.37, 33.83) --
	( 69.37, 33.83) --
	( 69.37, 33.83) --
	( 69.44, 33.81) --
	( 69.44, 33.81) --
	( 69.44, 33.81) --
	( 69.51, 33.79) --
	( 69.51, 33.79) --
	( 69.51, 33.79) --
	( 69.58, 33.77) --
	( 69.58, 33.77) --
	( 69.58, 33.77) --
	( 69.66, 33.74) --
	( 69.66, 33.74) --
	( 69.66, 33.74) --
	( 69.73, 33.72) --
	( 69.73, 33.72) --
	( 69.73, 33.72) --
	( 69.80, 33.70) --
	( 69.80, 33.70) --
	( 69.80, 33.70) --
	( 69.87, 33.68) --
	( 69.87, 33.68) --
	( 69.87, 33.68) --
	( 69.95, 33.66) --
	( 69.95, 33.66) --
	( 69.95, 33.66) --
	( 70.02, 33.64) --
	( 70.02, 33.64) --
	( 70.02, 33.64) --
	( 70.09, 33.62) --
	( 70.09, 33.62) --
	( 70.09, 33.62) --
	( 70.17, 33.60) --
	( 70.17, 33.60) --
	( 70.17, 33.60) --
	( 70.24, 33.58) --
	( 70.24, 33.58) --
	( 70.24, 33.58) --
	( 70.31, 33.56) --
	( 70.31, 33.56) --
	( 70.31, 33.56) --
	( 70.38, 33.54) --
	( 70.38, 33.54) --
	( 70.38, 33.54) --
	( 70.42, 33.53) --
	( 70.42, 33.53) --
	( 70.42, 33.53) --
	( 70.46, 33.51) --
	( 70.46, 33.51) --
	( 70.46, 33.51) --
	( 70.53, 33.47) --
	( 70.53, 33.47) --
	( 70.53, 33.47) --
	( 70.60, 33.43) --
	( 70.60, 33.43) --
	( 70.60, 33.43) --
	( 70.67, 33.40) --
	( 70.67, 33.40) --
	( 70.67, 33.40) --
	( 70.75, 33.36) --
	( 70.75, 33.36) --
	( 70.75, 33.36) --
	( 70.82, 33.32) --
	( 70.82, 33.32) --
	( 70.82, 33.32) --
	( 70.89, 33.29) --
	( 70.89, 33.29) --
	( 70.89, 33.29) --
	( 70.96, 33.25) --
	( 70.96, 33.25) --
	( 70.96, 33.25) --
	( 71.04, 33.21) --
	( 71.04, 33.21) --
	( 71.04, 33.21) --
	( 71.11, 33.18) --
	( 71.11, 33.18) --
	( 71.11, 33.18) --
	( 71.18, 33.14) --
	( 71.18, 33.14) --
	( 71.18, 33.14) --
	( 71.19, 33.14) --
	( 71.19, 33.14) --
	( 71.19, 33.14) --
	( 71.26, 33.11) --
	( 71.26, 33.11) --
	( 71.26, 33.11) --
	( 71.33, 33.08) --
	( 71.33, 33.08) --
	( 71.33, 33.08) --
	( 71.40, 33.06) --
	( 71.40, 33.06) --
	( 71.40, 33.06) --
	( 71.47, 33.03) --
	( 71.47, 33.03) --
	( 71.47, 33.03) --
	( 71.54, 33.00) --
	( 71.54, 33.00) --
	( 71.54, 33.00) --
	( 71.62, 32.97) --
	( 71.62, 32.97) --
	( 71.62, 32.97) --
	( 71.69, 32.95) --
	( 71.69, 32.95) --
	( 71.69, 32.95) --
	( 71.76, 32.92) --
	( 71.76, 32.92) --
	( 71.76, 32.92) --
	( 71.83, 32.89) --
	( 71.83, 32.89) --
	( 71.83, 32.89) --
	( 71.91, 32.86) --
	( 71.91, 32.86) --
	( 71.91, 32.86) --
	( 71.95, 32.85) --
	( 71.95, 32.85) --
	( 71.95, 32.85) --
	( 71.98, 32.83) --
	( 71.98, 32.83) --
	( 71.98, 32.83) --
	( 72.05, 32.80) --
	( 72.05, 32.80) --
	( 72.05, 32.80) --
	( 72.13, 32.77) --
	( 72.13, 32.77) --
	( 72.13, 32.77) --
	( 72.20, 32.74) --
	( 72.20, 32.74) --
	( 72.20, 32.74) --
	( 72.27, 32.70) --
	( 72.27, 32.70) --
	( 72.27, 32.70) --
	( 72.34, 32.67) --
	( 72.34, 32.67) --
	( 72.34, 32.67) --
	( 72.42, 32.64) --
	( 72.42, 32.64) --
	( 72.42, 32.64) --
	( 72.49, 32.60) --
	( 72.49, 32.60) --
	( 72.49, 32.60) --
	( 72.56, 32.57) --
	( 72.56, 32.57) --
	( 72.56, 32.57) --
	( 72.63, 32.54) --
	( 72.63, 32.54) --
	( 72.63, 32.54) --
	( 72.71, 32.51) --
	( 72.71, 32.51) --
	( 72.71, 32.51) --
	( 72.78, 32.47) --
	( 72.78, 32.47) --
	( 72.78, 32.47) --
	( 72.85, 32.44) --
	( 72.85, 32.44) --
	( 72.85, 32.44) --
	( 72.92, 32.41) --
	( 72.92, 32.41) --
	( 72.92, 32.41) --
	( 73.00, 32.38) --
	( 73.00, 32.38) --
	( 73.00, 32.38) --
	( 73.07, 32.34) --
	( 73.07, 32.34) --
	( 73.07, 32.34) --
	( 73.14, 32.31) --
	( 73.14, 32.31) --
	( 73.14, 32.31) --
	( 73.20, 32.28) --
	( 73.20, 32.28) --
	( 73.20, 32.28) --
	( 73.21, 32.28) --
	( 73.21, 32.28) --
	( 73.21, 32.28) --
	( 73.28, 32.24) --
	( 73.28, 32.24) --
	( 73.28, 32.24) --
	( 73.36, 32.20) --
	( 73.36, 32.20) --
	( 73.36, 32.20) --
	( 73.43, 32.16) --
	( 73.43, 32.16) --
	( 73.43, 32.16) --
	( 73.50, 32.12) --
	( 73.50, 32.12) --
	( 73.50, 32.12) --
	( 73.58, 32.08) --
	( 73.58, 32.08) --
	( 73.58, 32.08) --
	( 73.65, 32.05) --
	( 73.65, 32.05) --
	( 73.65, 32.05) --
	( 73.72, 32.01) --
	( 73.72, 32.01) --
	( 73.72, 32.01) --
	( 73.79, 31.97) --
	( 73.79, 31.97) --
	( 73.79, 31.97) --
	( 73.86, 31.93) --
	( 73.86, 31.93) --
	( 73.86, 31.93) --
	( 73.94, 31.89) --
	( 73.94, 31.89) --
	( 73.94, 31.89) --
	( 73.97, 31.88) --
	( 73.97, 31.88) --
	( 73.97, 31.88) --
	( 74.01, 31.86) --
	( 74.01, 31.86) --
	( 74.01, 31.86) --
	( 74.08, 31.82) --
	( 74.08, 31.82) --
	( 74.08, 31.82) --
	( 74.16, 31.79) --
	( 74.16, 31.79) --
	( 74.16, 31.79) --
	( 74.23, 31.76) --
	( 74.23, 31.76) --
	( 74.23, 31.76) --
	( 74.30, 31.73) --
	( 74.30, 31.73) --
	( 74.30, 31.73) --
	( 74.37, 31.70) --
	( 74.37, 31.70) --
	( 74.37, 31.70) --
	( 74.44, 31.66) --
	( 74.44, 31.66) --
	( 74.44, 31.66) --
	( 74.52, 31.63) --
	( 74.52, 31.63) --
	( 74.52, 31.63) --
	( 74.59, 31.60) --
	( 74.59, 31.60) --
	( 74.59, 31.60) --
	( 74.66, 31.57) --
	( 74.66, 31.57) --
	( 74.66, 31.57) --
	( 74.73, 31.53) --
	( 74.73, 31.53) --
	( 74.73, 31.53) --
	( 74.81, 31.50) --
	( 74.81, 31.50) --
	( 74.81, 31.50) --
	( 74.88, 31.47) --
	( 74.88, 31.47) --
	( 74.88, 31.47) --
	( 74.92, 31.45) --
	( 74.92, 31.45) --
	( 74.92, 31.45) --
	( 74.95, 31.44) --
	( 74.95, 31.44) --
	( 74.95, 31.44) --
	( 75.02, 31.42) --
	( 75.02, 31.42) --
	( 75.02, 31.42) --
	( 75.10, 31.39) --
	( 75.10, 31.39) --
	( 75.10, 31.39) --
	( 75.17, 31.37) --
	( 75.17, 31.37) --
	( 75.17, 31.37) --
	( 75.24, 31.34) --
	( 75.24, 31.34) --
	( 75.24, 31.34) --
	( 75.31, 31.32) --
	( 75.31, 31.32) --
	( 75.31, 31.32) --
	( 75.39, 31.29) --
	( 75.39, 31.29) --
	( 75.39, 31.29) --
	( 75.46, 31.27) --
	( 75.46, 31.27) --
	( 75.46, 31.27) --
	( 75.53, 31.25) --
	( 75.53, 31.25) --
	( 75.53, 31.25) --
	( 75.60, 31.22) --
	( 75.60, 31.22) --
	( 75.60, 31.22) --
	( 75.67, 31.20) --
	( 75.67, 31.20) --
	( 75.67, 31.20) --
	( 75.75, 31.17) --
	( 75.75, 31.17) --
	( 75.75, 31.17) --
	( 75.79, 31.16) --
	( 75.79, 31.16) --
	( 75.79, 31.16) --
	( 75.82, 31.14) --
	( 75.82, 31.14) --
	( 75.82, 31.14) --
	( 75.89, 31.11) --
	( 75.89, 31.11) --
	( 75.89, 31.11) --
	( 75.96, 31.07) --
	( 75.96, 31.07) --
	( 75.96, 31.07) --
	( 76.04, 31.03) --
	( 76.04, 31.03) --
	( 76.04, 31.03) --
	( 76.11, 31.00) --
	( 76.11, 31.00) --
	( 76.11, 31.00) --
	( 76.18, 30.96) --
	( 76.18, 30.96) --
	( 76.18, 30.96) --
	( 76.25, 30.92) --
	( 76.25, 30.92) --
	( 76.25, 30.92) --
	( 76.33, 30.89) --
	( 76.33, 30.89) --
	( 76.33, 30.89) --
	( 76.40, 30.85) --
	( 76.40, 30.85) --
	( 76.40, 30.85) --
	( 76.47, 30.81) --
	( 76.47, 30.81) --
	( 76.47, 30.81) --
	( 76.54, 30.78) --
	( 76.54, 30.78) --
	( 76.54, 30.78) --
	( 76.61, 30.74) --
	( 76.61, 30.74) --
	( 76.61, 30.74) --
	( 76.69, 30.70) --
	( 76.69, 30.70) --
	( 76.69, 30.70) --
	( 76.75, 30.67) --
	( 76.75, 30.67) --
	( 76.75, 30.67) --
	( 76.76, 30.67) --
	( 76.76, 30.67) --
	( 76.76, 30.67) --
	( 76.83, 30.64) --
	( 76.83, 30.64) --
	( 76.83, 30.64) --
	( 76.90, 30.61) --
	( 76.90, 30.61) --
	( 76.90, 30.61) --
	( 76.98, 30.58) --
	( 76.98, 30.58) --
	( 76.98, 30.58) --
	( 77.05, 30.56) --
	( 77.05, 30.56) --
	( 77.05, 30.56) --
	( 77.12, 30.53) --
	( 77.12, 30.53) --
	( 77.12, 30.53) --
	( 77.19, 30.50) --
	( 77.19, 30.50) --
	( 77.19, 30.50) --
	( 77.27, 30.47) --
	( 77.27, 30.47) --
	( 77.27, 30.47) --
	( 77.34, 30.45) --
	( 77.34, 30.45) --
	( 77.34, 30.45) --
	( 77.41, 30.42) --
	( 77.41, 30.42) --
	( 77.41, 30.42) --
	( 77.48, 30.39) --
	( 77.48, 30.39) --
	( 77.48, 30.39) --
	( 77.55, 30.36) --
	( 77.55, 30.36) --
	( 77.55, 30.36) --
	( 77.63, 30.34) --
	( 77.63, 30.34) --
	( 77.63, 30.34) --
	( 77.70, 30.31) --
	( 77.70, 30.31) --
	( 77.70, 30.31) --
	( 77.70, 30.30) --
	( 77.70, 30.30) --
	( 77.70, 30.30) --
	( 77.77, 30.28) --
	( 77.77, 30.28) --
	( 77.77, 30.28) --
	( 77.84, 30.24) --
	( 77.84, 30.24) --
	( 77.84, 30.24) --
	( 77.91, 30.21) --
	( 77.91, 30.21) --
	( 77.91, 30.21) --
	( 77.99, 30.18) --
	( 77.99, 30.18) --
	( 77.99, 30.18) --
	( 78.06, 30.15) --
	( 78.06, 30.15) --
	( 78.06, 30.15) --
	( 78.13, 30.12) --
	( 78.13, 30.12) --
	( 78.13, 30.12) --
	( 78.20, 30.08) --
	( 78.20, 30.08) --
	( 78.20, 30.08) --
	( 78.28, 30.05) --
	( 78.28, 30.05) --
	( 78.28, 30.05) --
	( 78.35, 30.02) --
	( 78.35, 30.02) --
	( 78.35, 30.02) --
	( 78.42, 29.99) --
	( 78.42, 29.99) --
	( 78.42, 29.99) --
	( 78.49, 29.96) --
	( 78.49, 29.96) --
	( 78.49, 29.96) --
	( 78.56, 29.93) --
	( 78.56, 29.93) --
	( 78.56, 29.93) --
	( 78.64, 29.89) --
	( 78.64, 29.89) --
	( 78.64, 29.89) --
	( 78.71, 29.86) --
	( 78.71, 29.86) --
	( 78.71, 29.86) --
	( 78.76, 29.84) --
	( 78.76, 29.84) --
	( 78.76, 29.84) --
	( 78.78, 29.83) --
	( 78.78, 29.83) --
	( 78.78, 29.83) --
	( 78.85, 29.79) --
	( 78.85, 29.79) --
	( 78.85, 29.79) --
	( 78.93, 29.76) --
	( 78.93, 29.76) --
	( 78.93, 29.76) --
	( 79.00, 29.73) --
	( 79.00, 29.73) --
	( 79.00, 29.73) --
	( 79.07, 29.69) --
	( 79.07, 29.69) --
	( 79.07, 29.69) --
	( 79.14, 29.66) --
	( 79.14, 29.66) --
	( 79.14, 29.66) --
	( 79.21, 29.62) --
	( 79.21, 29.62) --
	( 79.21, 29.62) --
	( 79.29, 29.59) --
	( 79.29, 29.59) --
	( 79.29, 29.59) --
	( 79.36, 29.56) --
	( 79.36, 29.56) --
	( 79.36, 29.56) --
	( 79.43, 29.52) --
	( 79.43, 29.52) --
	( 79.43, 29.52) --
	( 79.50, 29.49) --
	( 79.50, 29.49) --
	( 79.50, 29.49) --
	( 79.57, 29.45) --
	( 79.57, 29.45) --
	( 79.57, 29.45) --
	( 79.62, 29.43) --
	( 79.62, 29.43) --
	( 79.62, 29.43) --
	( 79.65, 29.42) --
	( 79.65, 29.42) --
	( 79.65, 29.42) --
	( 79.72, 29.40) --
	( 79.72, 29.40) --
	( 79.72, 29.40) --
	( 79.79, 29.37) --
	( 79.79, 29.37) --
	( 79.79, 29.37) --
	( 79.86, 29.35) --
	( 79.86, 29.35) --
	( 79.86, 29.35) --
	( 79.93, 29.32) --
	( 79.93, 29.32) --
	( 79.93, 29.32) --
	( 80.01, 29.29) --
	( 80.01, 29.29) --
	( 80.01, 29.29) --
	( 80.08, 29.27) --
	( 80.08, 29.27) --
	( 80.08, 29.27) --
	( 80.15, 29.24) --
	( 80.15, 29.24) --
	( 80.15, 29.24) --
	( 80.22, 29.22) --
	( 80.22, 29.22) --
	( 80.22, 29.22) --
	( 80.29, 29.19) --
	( 80.29, 29.19) --
	( 80.29, 29.19) --
	( 80.37, 29.16) --
	( 80.37, 29.16) --
	( 80.37, 29.16) --
	( 80.44, 29.14) --
	( 80.44, 29.14) --
	( 80.44, 29.14) --
	( 80.48, 29.12) --
	( 80.48, 29.12) --
	( 80.48, 29.12) --
	( 80.51, 29.11) --
	( 80.51, 29.11) --
	( 80.51, 29.11) --
	( 80.58, 29.08) --
	( 80.58, 29.08) --
	( 80.58, 29.08) --
	( 80.65, 29.05) --
	( 80.65, 29.05) --
	( 80.65, 29.05) --
	( 80.73, 29.02) --
	( 80.73, 29.02) --
	( 80.73, 29.02) --
	( 80.80, 28.99) --
	( 80.80, 28.99) --
	( 80.80, 28.99) --
	( 80.87, 28.96) --
	( 80.87, 28.96) --
	( 80.87, 28.96) --
	( 80.94, 28.93) --
	( 80.94, 28.93) --
	( 80.94, 28.93) --
	( 81.01, 28.90) --
	( 81.01, 28.90) --
	( 81.01, 28.90) --
	( 81.09, 28.87) --
	( 81.09, 28.87) --
	( 81.09, 28.87) --
	( 81.16, 28.83) --
	( 81.16, 28.83) --
	( 81.16, 28.83) --
	( 81.23, 28.80) --
	( 81.23, 28.80) --
	( 81.23, 28.80) --
	( 81.30, 28.77) --
	( 81.30, 28.77) --
	( 81.30, 28.77) --
	( 81.37, 28.74) --
	( 81.37, 28.74) --
	( 81.37, 28.74) --
	( 81.44, 28.71) --
	( 81.44, 28.71) --
	( 81.44, 28.71) --
	( 81.45, 28.71) --
	( 81.45, 28.71) --
	( 81.45, 28.71) --
	( 81.52, 28.69) --
	( 81.52, 28.69) --
	( 81.52, 28.69) --
	( 81.59, 28.67) --
	( 81.59, 28.67) --
	( 81.59, 28.67) --
	( 81.66, 28.65) --
	( 81.66, 28.65) --
	( 81.66, 28.65) --
	( 81.73, 28.63) --
	( 81.73, 28.63) --
	( 81.73, 28.63) --
	( 81.81, 28.60) --
	( 81.81, 28.60) --
	( 81.81, 28.60) --
	( 81.88, 28.58) --
	( 81.88, 28.58) --
	( 81.88, 28.58) --
	( 81.95, 28.56) --
	( 81.95, 28.56) --
	( 81.95, 28.56) --
	( 82.02, 28.54) --
	( 82.02, 28.54) --
	( 82.02, 28.54) --
	( 82.09, 28.52) --
	( 82.09, 28.52) --
	( 82.09, 28.52) --
	( 82.16, 28.49) --
	( 82.16, 28.49) --
	( 82.16, 28.49) --
	( 82.24, 28.47) --
	( 82.24, 28.47) --
	( 82.24, 28.47) --
	( 82.31, 28.45) --
	( 82.31, 28.45) --
	( 82.31, 28.45) --
	( 82.38, 28.43) --
	( 82.38, 28.43) --
	( 82.38, 28.43) --
	( 82.45, 28.41) --
	( 82.45, 28.41) --
	( 82.45, 28.41) --
	( 82.52, 28.39) --
	( 82.52, 28.39) --
	( 82.52, 28.39) --
	( 82.59, 28.36) --
	( 82.59, 28.36) --
	( 82.59, 28.36) --
	( 82.60, 28.36) --
	( 82.60, 28.36) --
	( 82.60, 28.36) --
	( 82.67, 28.35) --
	( 82.67, 28.35) --
	( 82.67, 28.35) --
	( 82.74, 28.33) --
	( 82.74, 28.33) --
	( 82.74, 28.33) --
	( 82.81, 28.31) --
	( 82.81, 28.31) --
	( 82.81, 28.31) --
	( 82.88, 28.30) --
	( 82.88, 28.30) --
	( 82.88, 28.30) --
	( 82.96, 28.28) --
	( 82.96, 28.28) --
	( 82.96, 28.28) --
	( 83.03, 28.27) --
	( 83.03, 28.27) --
	( 83.03, 28.27) --
	( 83.10, 28.25) --
	( 83.10, 28.25) --
	( 83.10, 28.25) --
	( 83.17, 28.23) --
	( 83.17, 28.23) --
	( 83.17, 28.23) --
	( 83.24, 28.22) --
	( 83.24, 28.22) --
	( 83.24, 28.22) --
	( 83.31, 28.20) --
	( 83.31, 28.20) --
	( 83.31, 28.20) --
	( 83.39, 28.18) --
	( 83.39, 28.18) --
	( 83.39, 28.18) --
	( 83.46, 28.17) --
	( 83.46, 28.17) --
	( 83.46, 28.17) --
	( 83.53, 28.15) --
	( 83.53, 28.15) --
	( 83.53, 28.15) --
	( 83.60, 28.13) --
	( 83.60, 28.13) --
	( 83.60, 28.13) --
	( 83.67, 28.12) --
	( 83.67, 28.12) --
	( 83.67, 28.12) --
	( 83.75, 28.10) --
	( 83.75, 28.10) --
	( 83.75, 28.10) --
	( 83.82, 28.09) --
	( 83.82, 28.09) --
	( 83.82, 28.09) --
	( 83.89, 28.07) --
	( 83.89, 28.07) --
	( 83.89, 28.07) --
	( 83.96, 28.05) --
	( 83.96, 28.05) --
	( 83.96, 28.05) --
	( 84.03, 28.04) --
	( 84.03, 28.04) --
	( 84.03, 28.04) --
	( 84.10, 28.02) --
	( 84.10, 28.02) --
	( 84.10, 28.02) --
	( 84.12, 28.02) --
	( 84.12, 28.02) --
	( 84.12, 28.02) --
	( 84.18, 28.00) --
	( 84.18, 28.00) --
	( 84.18, 28.00) --
	( 84.25, 27.99) --
	( 84.25, 27.99) --
	( 84.25, 27.99) --
	( 84.32, 27.97) --
	( 84.32, 27.97) --
	( 84.32, 27.97) --
	( 84.39, 27.95) --
	( 84.39, 27.95) --
	( 84.39, 27.95) --
	( 84.46, 27.93) --
	( 84.46, 27.93) --
	( 84.46, 27.93) --
	( 84.54, 27.91) --
	( 84.54, 27.91) --
	( 84.54, 27.91) --
	( 84.61, 27.90) --
	( 84.61, 27.90) --
	( 84.61, 27.90) --
	( 84.68, 27.88) --
	( 84.68, 27.88) --
	( 84.68, 27.88) --
	( 84.75, 27.86) --
	( 84.75, 27.86) --
	( 84.75, 27.86) --
	( 84.82, 27.84) --
	( 84.82, 27.84) --
	( 84.82, 27.84) --
	( 84.89, 27.83) --
	( 84.89, 27.83) --
	( 84.89, 27.83) --
	( 84.96, 27.81) --
	( 84.96, 27.81) --
	( 84.96, 27.81) --
	( 85.04, 27.79) --
	( 85.04, 27.79) --
	( 85.04, 27.79) --
	( 85.11, 27.77) --
	( 85.11, 27.77) --
	( 85.11, 27.77) --
	( 85.18, 27.76) --
	( 85.18, 27.76) --
	( 85.18, 27.76) --
	( 85.25, 27.74) --
	( 85.25, 27.74) --
	( 85.25, 27.74) --
	( 85.32, 27.72) --
	( 85.32, 27.72) --
	( 85.32, 27.72) --
	( 85.40, 27.70) --
	( 85.40, 27.70) --
	( 85.40, 27.70) --
	( 85.47, 27.69) --
	( 85.47, 27.69) --
	( 85.47, 27.69) --
	( 85.47, 27.69) --
	( 85.47, 27.69) --
	( 85.47, 27.69) --
	( 85.54, 27.67) --
	( 85.54, 27.67) --
	( 85.54, 27.67) --
	( 85.61, 27.65) --
	( 85.61, 27.65) --
	( 85.61, 27.65) --
	( 85.68, 27.63) --
	( 85.68, 27.63) --
	( 85.68, 27.63) --
	( 85.75, 27.61) --
	( 85.75, 27.61) --
	( 85.75, 27.61) --
	( 85.83, 27.59) --
	( 85.83, 27.59) --
	( 85.83, 27.59) --
	( 85.90, 27.57) --
	( 85.90, 27.57) --
	( 85.90, 27.57) --
	( 85.97, 27.55) --
	( 85.97, 27.55) --
	( 85.97, 27.55) --
	( 86.04, 27.53) --
	( 86.04, 27.53) --
	( 86.04, 27.53) --
	( 86.11, 27.51) --
	( 86.11, 27.51) --
	( 86.11, 27.51) --
	( 86.18, 27.49) --
	( 86.18, 27.49) --
	( 86.18, 27.49) --
	( 86.25, 27.47) --
	( 86.25, 27.47) --
	( 86.25, 27.47) --
	( 86.33, 27.45) --
	( 86.33, 27.45) --
	( 86.33, 27.45) --
	( 86.40, 27.43) --
	( 86.40, 27.43) --
	( 86.40, 27.43) --
	( 86.47, 27.42) --
	( 86.47, 27.42) --
	( 86.47, 27.42) --
	( 86.54, 27.40) --
	( 86.54, 27.40) --
	( 86.54, 27.40) --
	( 86.61, 27.38) --
	( 86.61, 27.38) --
	( 86.61, 27.38) --
	( 86.68, 27.36) --
	( 86.68, 27.36) --
	( 86.68, 27.36) --
	( 86.76, 27.34) --
	( 86.76, 27.34) --
	( 86.76, 27.34) --
	( 86.83, 27.32) --
	( 86.83, 27.32) --
	( 86.83, 27.32) --
	( 86.90, 27.30) --
	( 86.90, 27.30) --
	( 86.90, 27.30) --
	( 86.90, 27.30) --
	( 86.90, 27.30) --
	( 86.90, 27.30) --
	( 86.97, 27.29) --
	( 86.97, 27.29) --
	( 86.97, 27.29) --
	( 87.04, 27.27) --
	( 87.04, 27.27) --
	( 87.04, 27.27) --
	( 87.11, 27.26) --
	( 87.11, 27.26) --
	( 87.11, 27.26) --
	( 87.19, 27.25) --
	( 87.19, 27.25) --
	( 87.19, 27.25) --
	( 87.26, 27.24) --
	( 87.26, 27.24) --
	( 87.26, 27.24) --
	( 87.33, 27.22) --
	( 87.33, 27.22) --
	( 87.33, 27.22) --
	( 87.40, 27.21) --
	( 87.40, 27.21) --
	( 87.40, 27.21) --
	( 87.47, 27.20) --
	( 87.47, 27.20) --
	( 87.47, 27.20) --
	( 87.54, 27.19) --
	( 87.54, 27.19) --
	( 87.54, 27.19) --
	( 87.62, 27.17) --
	( 87.62, 27.17) --
	( 87.62, 27.17) --
	( 87.69, 27.16) --
	( 87.69, 27.16) --
	( 87.69, 27.16) --
	( 87.76, 27.15) --
	( 87.76, 27.15) --
	( 87.76, 27.15) --
	( 87.83, 27.14) --
	( 87.83, 27.14) --
	( 87.83, 27.14) --
	( 87.90, 27.12) --
	( 87.90, 27.12) --
	( 87.90, 27.12) --
	( 87.97, 27.11) --
	( 87.97, 27.11) --
	( 87.97, 27.11) --
	( 88.04, 27.10) --
	( 88.04, 27.10) --
	( 88.04, 27.10) --
	( 88.12, 27.09) --
	( 88.12, 27.09) --
	( 88.12, 27.09) --
	( 88.19, 27.08) --
	( 88.19, 27.08) --
	( 88.19, 27.08) --
	( 88.26, 27.06) --
	( 88.26, 27.06) --
	( 88.26, 27.06) --
	( 88.33, 27.05) --
	( 88.33, 27.05) --
	( 88.33, 27.05) --
	( 88.40, 27.04) --
	( 88.40, 27.04) --
	( 88.40, 27.04) --
	( 88.47, 27.03) --
	( 88.47, 27.03) --
	( 88.47, 27.03) --
	( 88.54, 27.01) --
	( 88.54, 27.01) --
	( 88.54, 27.01) --
	( 88.62, 27.00) --
	( 88.62, 27.00) --
	( 88.62, 27.00) --
	( 88.69, 26.99) --
	( 88.69, 26.99) --
	( 88.69, 26.99) --
	( 88.76, 26.98) --
	( 88.76, 26.98) --
	( 88.76, 26.98) --
	( 88.83, 26.96) --
	( 88.83, 26.96) --
	( 88.83, 26.96) --
	( 88.90, 26.95) --
	( 88.90, 26.95) --
	( 88.90, 26.95) --
	( 88.92, 26.95) --
	( 88.92, 26.95) --
	( 88.92, 26.95) --
	( 88.97, 26.94) --
	( 88.97, 26.94) --
	( 88.97, 26.94) --
	( 89.05, 26.93) --
	( 89.05, 26.93) --
	( 89.05, 26.93) --
	( 89.12, 26.93) --
	( 89.12, 26.93) --
	( 89.12, 26.93) --
	( 89.19, 26.92) --
	( 89.19, 26.92) --
	( 89.19, 26.92) --
	( 89.26, 26.91) --
	( 89.26, 26.91) --
	( 89.26, 26.91) --
	( 89.33, 26.90) --
	( 89.33, 26.90) --
	( 89.33, 26.90) --
	( 89.40, 26.89) --
	( 89.40, 26.89) --
	( 89.40, 26.89) --
	( 89.47, 26.89) --
	( 89.47, 26.89) --
	( 89.47, 26.89) --
	( 89.55, 26.88) --
	( 89.55, 26.88) --
	( 89.55, 26.88) --
	( 89.62, 26.87) --
	( 89.62, 26.87) --
	( 89.62, 26.87) --
	( 89.69, 26.86) --
	( 89.69, 26.86) --
	( 89.69, 26.86) --
	( 89.76, 26.85) --
	( 89.76, 26.85) --
	( 89.76, 26.85) --
	( 89.83, 26.85) --
	( 89.83, 26.85) --
	( 89.83, 26.85) --
	( 89.90, 26.84) --
	( 89.90, 26.84) --
	( 89.90, 26.84) --
	( 89.97, 26.83) --
	( 89.97, 26.83) --
	( 89.97, 26.83) --
	( 90.04, 26.82) --
	( 90.04, 26.82) --
	( 90.04, 26.82) --
	( 90.12, 26.81) --
	( 90.12, 26.81) --
	( 90.12, 26.81) --
	( 90.19, 26.81) --
	( 90.19, 26.81) --
	( 90.19, 26.81) --
	( 90.26, 26.80) --
	( 90.26, 26.80) --
	( 90.26, 26.80) --
	( 90.33, 26.79) --
	( 90.33, 26.79) --
	( 90.33, 26.79) --
	( 90.40, 26.78) --
	( 90.40, 26.78) --
	( 90.40, 26.78) --
	( 90.47, 26.77) --
	( 90.47, 26.77) --
	( 90.47, 26.77) --
	( 90.54, 26.77) --
	( 90.54, 26.77) --
	( 90.54, 26.77) --
	( 90.62, 26.76) --
	( 90.62, 26.76) --
	( 90.62, 26.76) --
	( 90.64, 26.75) --
	( 90.64, 26.75) --
	( 90.64, 26.75) --
	( 90.69, 26.75) --
	( 90.69, 26.75) --
	( 90.69, 26.75) --
	( 90.76, 26.74) --
	( 90.76, 26.74) --
	( 90.76, 26.74) --
	( 90.83, 26.74) --
	( 90.83, 26.74) --
	( 90.83, 26.74) --
	( 90.90, 26.73) --
	( 90.90, 26.73) --
	( 90.90, 26.73) --
	( 90.97, 26.72) --
	( 90.97, 26.72) --
	( 90.97, 26.72) --
	( 91.04, 26.72) --
	( 91.04, 26.72) --
	( 91.04, 26.72) --
	( 91.12, 26.71) --
	( 91.12, 26.71) --
	( 91.12, 26.71) --
	( 91.19, 26.70) --
	( 91.19, 26.70) --
	( 91.19, 26.70) --
	( 91.26, 26.70) --
	( 91.26, 26.70) --
	( 91.26, 26.70) --
	( 91.33, 26.69) --
	( 91.33, 26.69) --
	( 91.33, 26.69) --
	( 91.40, 26.68) --
	( 91.40, 26.68) --
	( 91.40, 26.68) --
	( 91.47, 26.67) --
	( 91.47, 26.67) --
	( 91.47, 26.67) --
	( 91.54, 26.67) --
	( 91.54, 26.67) --
	( 91.54, 26.67) --
	( 91.61, 26.66) --
	( 91.61, 26.66) --
	( 91.61, 26.66) --
	( 91.69, 26.65) --
	( 91.69, 26.65) --
	( 91.69, 26.65) --
	( 91.76, 26.65) --
	( 91.76, 26.65) --
	( 91.76, 26.65) --
	( 91.83, 26.64) --
	( 91.83, 26.64) --
	( 91.83, 26.64) --
	( 91.90, 26.63) --
	( 91.90, 26.63) --
	( 91.90, 26.63) --
	( 91.97, 26.63) --
	( 91.97, 26.63) --
	( 91.97, 26.63) --
	( 92.04, 26.62) --
	( 92.04, 26.62) --
	( 92.04, 26.62) --
	( 92.11, 26.61) --
	( 92.11, 26.61) --
	( 92.11, 26.61) --
	( 92.18, 26.61) --
	( 92.18, 26.61) --
	( 92.18, 26.61) --
	( 92.26, 26.60) --
	( 92.26, 26.60) --
	( 92.26, 26.60) --
	( 92.33, 26.59) --
	( 92.33, 26.59) --
	( 92.33, 26.59) --
	( 92.40, 26.59) --
	( 92.40, 26.59) --
	( 92.40, 26.59) --
	( 92.47, 26.58) --
	( 92.47, 26.58) --
	( 92.47, 26.58) --
	( 92.54, 26.57) --
	( 92.54, 26.57) --
	( 92.54, 26.57) --
	( 92.61, 26.56) --
	( 92.61, 26.56) --
	( 92.61, 26.56) --
	( 92.65, 26.56) --
	( 92.65, 26.56) --
	( 92.65, 26.56) --
	( 92.68, 26.56) --
	( 92.68, 26.56) --
	( 92.68, 26.56) --
	( 92.75, 26.55) --
	( 92.75, 26.55) --
	( 92.75, 26.55) --
	( 92.83, 26.55) --
	( 92.83, 26.55) --
	( 92.83, 26.55) --
	( 92.90, 26.55) --
	( 92.90, 26.55) --
	( 92.90, 26.55) --
	( 92.97, 26.54) --
	( 92.97, 26.54) --
	( 92.97, 26.54) --
	( 93.04, 26.54) --
	( 93.04, 26.54) --
	( 93.04, 26.54) --
	( 93.11, 26.53) --
	( 93.11, 26.53) --
	( 93.11, 26.53) --
	( 93.18, 26.53) --
	( 93.18, 26.53) --
	( 93.18, 26.53) --
	( 93.25, 26.53) --
	( 93.25, 26.53) --
	( 93.25, 26.53) --
	( 93.32, 26.52) --
	( 93.32, 26.52) --
	( 93.32, 26.52) --
	( 93.39, 26.52) --
	( 93.39, 26.52) --
	( 93.39, 26.52) --
	( 93.46, 26.51) --
	( 93.46, 26.51) --
	( 93.46, 26.51) --
	( 93.54, 26.51) --
	( 93.54, 26.51) --
	( 93.54, 26.51) --
	( 93.61, 26.50) --
	( 93.61, 26.50) --
	( 93.61, 26.50) --
	( 93.68, 26.50) --
	( 93.68, 26.50) --
	( 93.68, 26.50) --
	( 93.75, 26.50) --
	( 93.75, 26.50) --
	( 93.75, 26.50) --
	( 93.82, 26.49) --
	( 93.82, 26.49) --
	( 93.82, 26.49) --
	( 93.89, 26.49) --
	( 93.89, 26.49) --
	( 93.89, 26.49) --
	( 93.96, 26.48) --
	( 93.96, 26.48) --
	( 93.96, 26.48) --
	( 94.03, 26.48) --
	( 94.03, 26.48) --
	( 94.03, 26.48) --
	( 94.10, 26.48) --
	( 94.10, 26.48) --
	( 94.10, 26.48) --
	( 94.18, 26.47) --
	( 94.18, 26.47) --
	( 94.18, 26.47) --
	( 94.25, 26.47) --
	( 94.25, 26.47) --
	( 94.25, 26.47) --
	( 94.32, 26.46) --
	( 94.32, 26.46) --
	( 94.32, 26.46) --
	( 94.39, 26.46) --
	( 94.39, 26.46) --
	( 94.39, 26.46) --
	( 94.46, 26.45) --
	( 94.46, 26.45) --
	( 94.46, 26.45) --
	( 94.53, 26.45) --
	( 94.53, 26.45) --
	( 94.53, 26.45) --
	( 94.60, 26.45) --
	( 94.60, 26.45) --
	( 94.60, 26.45) --
	( 94.67, 26.44) --
	( 94.67, 26.44) --
	( 94.67, 26.44) --
	( 94.75, 26.44) --
	( 94.75, 26.44) --
	( 94.75, 26.44) --
	( 94.82, 26.43) --
	( 94.82, 26.43) --
	( 94.82, 26.43) --
	( 94.89, 26.43) --
	( 94.89, 26.43) --
	( 94.89, 26.43) --
	( 94.95, 26.42) --
	( 94.95, 26.42) --
	( 94.95, 26.42) --
	( 94.96, 26.42) --
	( 94.96, 26.42) --
	( 94.96, 26.42) --
	( 95.03, 26.42) --
	( 95.03, 26.42) --
	( 95.03, 26.42) --
	( 95.10, 26.41) --
	( 95.10, 26.41) --
	( 95.10, 26.41) --
	( 95.17, 26.41) --
	( 95.17, 26.41) --
	( 95.17, 26.41) --
	( 95.24, 26.40) --
	( 95.24, 26.40) --
	( 95.24, 26.40) --
	( 95.31, 26.39) --
	( 95.31, 26.39) --
	( 95.31, 26.39) --
	( 95.38, 26.39) --
	( 95.38, 26.39) --
	( 95.38, 26.39) --
	( 95.45, 26.38) --
	( 95.45, 26.38) --
	( 95.45, 26.38) --
	( 95.53, 26.38) --
	( 95.53, 26.38) --
	( 95.53, 26.38) --
	( 95.60, 26.37) --
	( 95.60, 26.37) --
	( 95.60, 26.37) --
	( 95.67, 26.36) --
	( 95.67, 26.36) --
	( 95.67, 26.36) --
	( 95.74, 26.36) --
	( 95.74, 26.36) --
	( 95.74, 26.36) --
	( 95.81, 26.35) --
	( 95.81, 26.35) --
	( 95.81, 26.35) --
	( 95.88, 26.35) --
	( 95.88, 26.35) --
	( 95.88, 26.35) --
	( 95.95, 26.34) --
	( 95.95, 26.34) --
	( 95.95, 26.34) --
	( 96.02, 26.33) --
	( 96.02, 26.33) --
	( 96.02, 26.33) --
	( 96.09, 26.33) --
	( 96.09, 26.33) --
	( 96.09, 26.33) --
	( 96.16, 26.32) --
	( 96.16, 26.32) --
	( 96.16, 26.32) --
	( 96.24, 26.32) --
	( 96.24, 26.32) --
	( 96.24, 26.32) --
	( 96.31, 26.31) --
	( 96.31, 26.31) --
	( 96.31, 26.31) --
	( 96.38, 26.30) --
	( 96.38, 26.30) --
	( 96.38, 26.30) --
	( 96.45, 26.30) --
	( 96.45, 26.30) --
	( 96.45, 26.30) --
	( 96.52, 26.29) --
	( 96.52, 26.29) --
	( 96.52, 26.29) --
	( 96.59, 26.29) --
	( 96.59, 26.29) --
	( 96.59, 26.29) --
	( 96.66, 26.28) --
	( 96.66, 26.28) --
	( 96.66, 26.28) --
	( 96.73, 26.27) --
	( 96.73, 26.27) --
	( 96.73, 26.27) --
	( 96.77, 26.27) --
	( 96.77, 26.27) --
	( 96.77, 26.27) --
	( 96.80, 26.27) --
	( 96.80, 26.27) --
	( 96.80, 26.27) --
	( 96.87, 26.27) --
	( 96.87, 26.27) --
	( 96.87, 26.27) --
	( 96.94, 26.27) --
	( 96.94, 26.27) --
	( 96.94, 26.27) --
	( 97.02, 26.26) --
	( 97.02, 26.26) --
	( 97.02, 26.26) --
	( 97.09, 26.26) --
	( 97.09, 26.26) --
	( 97.09, 26.26) --
	( 97.16, 26.26) --
	( 97.16, 26.26) --
	( 97.16, 26.26) --
	( 97.23, 26.26) --
	( 97.23, 26.26) --
	( 97.23, 26.26) --
	( 97.30, 26.26) --
	( 97.30, 26.26) --
	( 97.30, 26.26) --
	( 97.37, 26.25) --
	( 97.37, 26.25) --
	( 97.37, 26.25) --
	( 97.44, 26.25) --
	( 97.44, 26.25) --
	( 97.44, 26.25) --
	( 97.51, 26.25) --
	( 97.51, 26.25) --
	( 97.51, 26.25) --
	( 97.58, 26.25) --
	( 97.58, 26.25) --
	( 97.58, 26.25) --
	( 97.65, 26.25) --
	( 97.65, 26.25) --
	( 97.65, 26.25) --
	( 97.72, 26.25) --
	( 97.72, 26.25) --
	( 97.72, 26.25) --
	( 97.80, 26.24) --
	( 97.80, 26.24) --
	( 97.80, 26.24) --
	( 97.87, 26.24) --
	( 97.87, 26.24) --
	( 97.87, 26.24) --
	( 97.94, 26.24) --
	( 97.94, 26.24) --
	( 97.94, 26.24) --
	( 98.01, 26.24) --
	( 98.01, 26.24) --
	( 98.01, 26.24) --
	( 98.08, 26.24) --
	( 98.08, 26.24) --
	( 98.08, 26.24) --
	( 98.15, 26.24) --
	( 98.15, 26.24) --
	( 98.15, 26.24) --
	( 98.22, 26.23) --
	( 98.22, 26.23) --
	( 98.22, 26.23) --
	( 98.29, 26.23) --
	( 98.29, 26.23) --
	( 98.29, 26.23) --
	( 98.36, 26.23) --
	( 98.36, 26.23) --
	( 98.36, 26.23) --
	( 98.43, 26.23) --
	( 98.43, 26.23) --
	( 98.43, 26.23) --
	( 98.50, 26.23) --
	( 98.50, 26.23) --
	( 98.50, 26.23) --
	( 98.57, 26.22) --
	( 98.57, 26.22) --
	( 98.57, 26.22) --
	( 98.65, 26.22) --
	( 98.65, 26.22) --
	( 98.65, 26.22) --
	( 98.72, 26.22) --
	( 98.72, 26.22) --
	( 98.72, 26.22) --
	( 98.79, 26.22) --
	( 98.79, 26.22) --
	( 98.79, 26.22) --
	( 98.86, 26.22) --
	( 98.86, 26.22) --
	( 98.86, 26.22) --
	( 98.93, 26.22) --
	( 98.93, 26.22) --
	( 98.93, 26.22) --
	( 99.00, 26.21) --
	( 99.00, 26.21) --
	( 99.00, 26.21) --
	( 99.07, 26.21) --
	( 99.07, 26.21) --
	( 99.07, 26.21) --
	( 99.07, 26.21) --
	( 99.07, 26.21) --
	( 99.07, 26.21) --
	( 99.14, 26.21) --
	( 99.14, 26.21) --
	( 99.14, 26.21) --
	( 99.21, 26.21) --
	( 99.21, 26.21) --
	( 99.21, 26.21) --
	( 99.28, 26.20) --
	( 99.28, 26.20) --
	( 99.28, 26.20) --
	( 99.35, 26.20) --
	( 99.35, 26.20) --
	( 99.35, 26.20) --
	( 99.42, 26.20) --
	( 99.42, 26.20) --
	( 99.42, 26.20) --
	( 99.49, 26.20) --
	( 99.49, 26.20) --
	( 99.49, 26.20) --
	( 99.56, 26.19) --
	( 99.56, 26.19) --
	( 99.56, 26.19) --
	( 99.64, 26.19) --
	( 99.64, 26.19) --
	( 99.64, 26.19) --
	( 99.71, 26.19) --
	( 99.71, 26.19) --
	( 99.71, 26.19) --
	( 99.78, 26.19) --
	( 99.78, 26.19) --
	( 99.78, 26.19) --
	( 99.85, 26.18) --
	( 99.85, 26.18) --
	( 99.85, 26.18) --
	( 99.92, 26.18) --
	( 99.92, 26.18) --
	( 99.92, 26.18) --
	( 99.99, 26.18) --
	( 99.99, 26.18) --
	( 99.99, 26.18) --
	(100.06, 26.17) --
	(100.06, 26.17) --
	(100.06, 26.17) --
	(100.13, 26.17) --
	(100.13, 26.17) --
	(100.13, 26.17) --
	(100.20, 26.17) --
	(100.20, 26.17) --
	(100.20, 26.17) --
	(100.27, 26.17) --
	(100.27, 26.17) --
	(100.27, 26.17) --
	(100.34, 26.16) --
	(100.34, 26.16) --
	(100.34, 26.16) --
	(100.41, 26.16) --
	(100.41, 26.16) --
	(100.41, 26.16) --
	(100.48, 26.16) --
	(100.48, 26.16) --
	(100.48, 26.16) --
	(100.55, 26.16) --
	(100.55, 26.16) --
	(100.55, 26.16) --
	(100.62, 26.15) --
	(100.62, 26.15) --
	(100.62, 26.15) --
	(100.70, 26.15) --
	(100.70, 26.15) --
	(100.70, 26.15) --
	(100.77, 26.15) --
	(100.77, 26.15) --
	(100.77, 26.15) --
	(100.84, 26.15) --
	(100.84, 26.15) --
	(100.84, 26.15) --
	(100.91, 26.14) --
	(100.91, 26.14) --
	(100.91, 26.14) --
	(100.98, 26.14) --
	(100.98, 26.14) --
	(100.98, 26.14) --
	(101.05, 26.14) --
	(101.05, 26.14) --
	(101.05, 26.14) --
	(101.12, 26.13) --
	(101.12, 26.13) --
	(101.12, 26.13) --
	(101.19, 26.13) --
	(101.19, 26.13) --
	(101.19, 26.13) --
	(101.26, 26.13) --
	(101.26, 26.13) --
	(101.26, 26.13) --
	(101.33, 26.13) --
	(101.33, 26.13) --
	(101.33, 26.13) --
	(101.40, 26.12) --
	(101.40, 26.12) --
	(101.40, 26.12) --
	(101.47, 26.12) --
	(101.47, 26.12) --
	(101.47, 26.12) --
	(101.54, 26.12) --
	(101.54, 26.12) --
	(101.54, 26.12) --
	(101.61, 26.12) --
	(101.61, 26.12) --
	(101.61, 26.12) --
	(101.66, 26.11) --
	(101.66, 26.11) --
	(101.66, 26.11) --
	(101.68, 26.11) --
	(101.68, 26.11) --
	(101.68, 26.11) --
	(101.75, 26.11) --
	(101.75, 26.11) --
	(101.75, 26.11) --
	(101.82, 26.11) --
	(101.82, 26.11) --
	(101.82, 26.11) --
	(101.90, 26.11) --
	(101.90, 26.11) --
	(101.90, 26.11) --
	(101.97, 26.11) --
	(101.97, 26.11) --
	(101.97, 26.11) --
	(102.04, 26.11) --
	(102.04, 26.11) --
	(102.04, 26.11) --
	(102.11, 26.11) --
	(102.11, 26.11) --
	(102.11, 26.11) --
	(102.18, 26.11) --
	(102.18, 26.11) --
	(102.18, 26.11) --
	(102.25, 26.11) --
	(102.25, 26.11) --
	(102.25, 26.11) --
	(102.32, 26.11) --
	(102.32, 26.11) --
	(102.32, 26.11) --
	(102.39, 26.10) --
	(102.39, 26.10) --
	(102.39, 26.10) --
	(102.46, 26.10) --
	(102.46, 26.10) --
	(102.46, 26.10) --
	(102.53, 26.10) --
	(102.53, 26.10) --
	(102.53, 26.10) --
	(102.60, 26.10) --
	(102.60, 26.10) --
	(102.60, 26.10) --
	(102.67, 26.10) --
	(102.67, 26.10) --
	(102.67, 26.10) --
	(102.74, 26.10) --
	(102.74, 26.10) --
	(102.74, 26.10) --
	(102.81, 26.10) --
	(102.81, 26.10) --
	(102.81, 26.10) --
	(102.88, 26.10) --
	(102.88, 26.10) --
	(102.88, 26.10) --
	(102.95, 26.10) --
	(102.95, 26.10) --
	(102.95, 26.10) --
	(103.02, 26.10) --
	(103.02, 26.10) --
	(103.02, 26.10) --
	(103.09, 26.09) --
	(103.09, 26.09) --
	(103.09, 26.09) --
	(103.16, 26.09) --
	(103.16, 26.09) --
	(103.16, 26.09) --
	(103.23, 26.09) --
	(103.23, 26.09) --
	(103.23, 26.09) --
	(103.31, 26.09) --
	(103.31, 26.09) --
	(103.31, 26.09) --
	(103.37, 26.09) --
	(103.37, 26.09) --
	(103.37, 26.09) --
	(103.45, 26.09) --
	(103.45, 26.09) --
	(103.45, 26.09) --
	(103.52, 26.09) --
	(103.52, 26.09) --
	(103.52, 26.09) --
	(103.59, 26.09) --
	(103.59, 26.09) --
	(103.59, 26.09) --
	(103.66, 26.09) --
	(103.66, 26.09) --
	(103.66, 26.09) --
	(103.73, 26.09) --
	(103.73, 26.09) --
	(103.73, 26.09) --
	(103.80, 26.08) --
	(103.80, 26.08) --
	(103.80, 26.08) --
	(103.87, 26.08) --
	(103.87, 26.08) --
	(103.87, 26.08) --
	(103.94, 26.08) --
	(103.94, 26.08) --
	(103.94, 26.08) --
	(104.01, 26.08) --
	(104.01, 26.08) --
	(104.01, 26.08) --
	(104.08, 26.08) --
	(104.08, 26.08) --
	(104.08, 26.08) --
	(104.15, 26.08) --
	(104.15, 26.08) --
	(104.15, 26.08) --
	(104.22, 26.08) --
	(104.22, 26.08) --
	(104.22, 26.08) --
	(104.29, 26.08) --
	(104.29, 26.08) --
	(104.29, 26.08) --
	(104.36, 26.08) --
	(104.36, 26.08) --
	(104.36, 26.08) --
	(104.43, 26.08) --
	(104.43, 26.08) --
	(104.43, 26.08) --
	(104.44, 26.08) --
	(104.44, 26.08) --
	(104.44, 26.08) --
	(104.50, 26.08) --
	(104.50, 26.08) --
	(104.50, 26.08) --
	(104.57, 26.07) --
	(104.57, 26.07) --
	(104.57, 26.07) --
	(104.64, 26.07) --
	(104.64, 26.07) --
	(104.64, 26.07) --
	(104.71, 26.07) --
	(104.71, 26.07) --
	(104.71, 26.07) --
	(104.78, 26.07) --
	(104.78, 26.07) --
	(104.78, 26.07) --
	(104.85, 26.07) --
	(104.85, 26.07) --
	(104.85, 26.07) --
	(104.92, 26.07) --
	(104.92, 26.07) --
	(104.92, 26.07) --
	(104.99, 26.07) --
	(104.99, 26.07) --
	(104.99, 26.07) --
	(105.06, 26.07) --
	(105.06, 26.07) --
	(105.06, 26.07) --
	(105.14, 26.07) --
	(105.14, 26.07) --
	(105.14, 26.07) --
	(105.20, 26.07) --
	(105.20, 26.07) --
	(105.20, 26.07) --
	(105.28, 26.07) --
	(105.28, 26.07) --
	(105.28, 26.07) --
	(105.34, 26.07) --
	(105.34, 26.07) --
	(105.34, 26.07) --
	(105.42, 26.07) --
	(105.42, 26.07) --
	(105.42, 26.07) --
	(105.49, 26.07) --
	(105.49, 26.07) --
	(105.49, 26.07) --
	(105.56, 26.07) --
	(105.56, 26.07) --
	(105.56, 26.07) --
	(105.63, 26.07) --
	(105.63, 26.07) --
	(105.63, 26.07) --
	(105.70, 26.07) --
	(105.70, 26.07) --
	(105.70, 26.07) --
	(105.77, 26.07) --
	(105.77, 26.07) --
	(105.77, 26.07) --
	(105.84, 26.07) --
	(105.84, 26.07) --
	(105.84, 26.07) --
	(105.91, 26.07) --
	(105.91, 26.07) --
	(105.91, 26.07) --
	(105.98, 26.07) --
	(105.98, 26.07) --
	(105.98, 26.07) --
	(106.05, 26.07) --
	(106.05, 26.07) --
	(106.05, 26.07) --
	(106.12, 26.06) --
	(106.12, 26.06) --
	(106.12, 26.06) --
	(106.19, 26.06) --
	(106.19, 26.06) --
	(106.19, 26.06) --
	(106.26, 26.06) --
	(106.26, 26.06) --
	(106.26, 26.06) --
	(106.33, 26.06) --
	(106.33, 26.06) --
	(106.33, 26.06) --
	(106.40, 26.06) --
	(106.40, 26.06) --
	(106.40, 26.06) --
	(106.47, 26.06) --
	(106.47, 26.06) --
	(106.47, 26.06) --
	(106.54, 26.06) --
	(106.54, 26.06) --
	(106.54, 26.06) --
	(106.61, 26.06) --
	(106.61, 26.06) --
	(106.61, 26.06) --
	(106.68, 26.06) --
	(106.68, 26.06) --
	(106.68, 26.06) --
	(106.75, 26.06) --
	(106.75, 26.06) --
	(106.75, 26.06) --
	(106.82, 26.06) --
	(106.82, 26.06) --
	(106.82, 26.06) --
	(106.89, 26.06) --
	(106.89, 26.06) --
	(106.89, 26.06) --
	(106.96, 26.06) --
	(106.96, 26.06) --
	(106.96, 26.06) --
	(107.03, 26.06) --
	(107.03, 26.06) --
	(107.03, 26.06) --
	(107.10, 26.06) --
	(107.10, 26.06) --
	(107.10, 26.06) --
	(107.17, 26.06) --
	(107.17, 26.06) --
	(107.17, 26.06) --
	(107.24, 26.06) --
	(107.24, 26.06) --
	(107.24, 26.06) --
	(107.31, 26.06) --
	(107.31, 26.06) --
	(107.31, 26.06) --
	(107.38, 26.06) --
	(107.38, 26.06) --
	(107.38, 26.06) --
	(107.41, 26.06) --
	(107.41, 26.06) --
	(107.41, 26.06) --
	(107.45, 26.06) --
	(107.45, 26.06) --
	(107.45, 26.06) --
	(107.52, 26.06) --
	(107.52, 26.06) --
	(107.52, 26.06) --
	(107.59, 26.06) --
	(107.59, 26.06) --
	(107.59, 26.06) --
	(107.66, 26.06) --
	(107.66, 26.06) --
	(107.66, 26.06) --
	(107.73, 26.06) --
	(107.73, 26.06) --
	(107.73, 26.06) --
	(107.80, 26.06) --
	(107.80, 26.06) --
	(107.80, 26.06) --
	(107.87, 26.06) --
	(107.87, 26.06) --
	(107.87, 26.06) --
	(107.94, 26.06) --
	(107.94, 26.06) --
	(107.94, 26.06) --
	(108.01, 26.06) --
	(108.01, 26.06) --
	(108.01, 26.06) --
	(108.08, 26.06) --
	(108.08, 26.06) --
	(108.08, 26.06) --
	(108.15, 26.06) --
	(108.15, 26.06) --
	(108.15, 26.06) --
	(108.22, 26.06) --
	(108.22, 26.06) --
	(108.22, 26.06) --
	(108.29, 26.06) --
	(108.29, 26.06) --
	(108.29, 26.06) --
	(108.36, 26.06) --
	(108.36, 26.06) --
	(108.36, 26.06) --
	(108.43, 26.06) --
	(108.43, 26.06) --
	(108.43, 26.06) --
	(108.50, 26.06) --
	(108.50, 26.06) --
	(108.50, 26.06) --
	(108.57, 26.06) --
	(108.57, 26.06) --
	(108.57, 26.06) --
	(108.64, 26.06) --
	(108.64, 26.06) --
	(108.64, 26.06) --
	(108.71, 26.06) --
	(108.71, 26.06) --
	(108.71, 26.06) --
	(108.78, 26.06) --
	(108.78, 26.06) --
	(108.78, 26.06) --
	(108.85, 26.06) --
	(108.85, 26.06) --
	(108.85, 26.06) --
	(108.92, 26.06) --
	(108.92, 26.06) --
	(108.92, 26.06) --
	(108.99, 26.06) --
	(108.99, 26.06) --
	(108.99, 26.06) --
	(109.06, 26.06) --
	(109.06, 26.06) --
	(109.06, 26.06) --
	(109.13, 26.06) --
	(109.13, 26.06) --
	(109.13, 26.06) --
	(109.20, 26.06) --
	(109.20, 26.06) --
	(109.20, 26.06) --
	(109.27, 26.06) --
	(109.27, 26.06) --
	(109.27, 26.06) --
	(109.34, 26.06) --
	(109.34, 26.06) --
	(109.34, 26.06) --
	(109.41, 26.06) --
	(109.41, 26.06) --
	(109.41, 26.06) --
	(109.48, 26.06) --
	(109.48, 26.06) --
	(109.48, 26.06) --
	(109.55, 26.06) --
	(109.55, 26.06) --
	(109.55, 26.06) --
	(109.62, 26.06) --
	(109.62, 26.06) --
	(109.62, 26.06) --
	(109.69, 26.06) --
	(109.69, 26.06) --
	(109.69, 26.06) --
	(109.76, 26.06) --
	(109.76, 26.06) --
	(109.76, 26.06) --
	(109.83, 26.06) --
	(109.83, 26.06) --
	(109.83, 26.06) --
	(109.90, 26.06) --
	(109.90, 26.06) --
	(109.90, 26.06) --
	(109.90, 26.06) --
	(109.90, 26.06) --
	(109.90, 26.06) --
	(109.97, 26.06) --
	(109.97, 26.06) --
	(109.97, 26.06) --
	(110.04, 26.06) --
	(110.04, 26.06) --
	(110.04, 26.06) --
	(110.11, 26.06) --
	(110.11, 26.06) --
	(110.11, 26.06) --
	(110.18, 26.06) --
	(110.18, 26.06) --
	(110.18, 26.06) --
	(110.25, 26.06) --
	(110.25, 26.06) --
	(110.25, 26.06) --
	(110.32, 26.06) --
	(110.32, 26.06) --
	(110.32, 26.06) --
	(110.39, 26.06) --
	(110.39, 26.06) --
	(110.39, 26.06) --
	(110.46, 26.06) --
	(110.46, 26.06) --
	(110.46, 26.06) --
	(110.53, 26.06) --
	(110.53, 26.06) --
	(110.53, 26.06) --
	(110.60, 26.06) --
	(110.60, 26.06) --
	(110.60, 26.06) --
	(110.67, 26.06) --
	(110.67, 26.06) --
	(110.67, 26.06) --
	(110.74, 26.07) --
	(110.74, 26.07) --
	(110.74, 26.07) --
	(110.81, 26.07) --
	(110.81, 26.07) --
	(110.81, 26.07) --
	(110.88, 26.07) --
	(110.88, 26.07) --
	(110.88, 26.07) --
	(110.95, 26.07) --
	(110.95, 26.07) --
	(110.95, 26.07) --
	(111.02, 26.07) --
	(111.02, 26.07) --
	(111.02, 26.07) --
	(111.09, 26.07) --
	(111.09, 26.07) --
	(111.09, 26.07) --
	(111.16, 26.07) --
	(111.16, 26.07) --
	(111.16, 26.07) --
	(111.23, 26.07) --
	(111.23, 26.07) --
	(111.23, 26.07) --
	(111.30, 26.07) --
	(111.30, 26.07) --
	(111.30, 26.07) --
	(111.37, 26.07) --
	(111.37, 26.07) --
	(111.37, 26.07) --
	(111.44, 26.07) --
	(111.44, 26.07) --
	(111.44, 26.07) --
	(111.51, 26.07) --
	(111.51, 26.07) --
	(111.51, 26.07) --
	(111.58, 26.07) --
	(111.58, 26.07) --
	(111.58, 26.07) --
	(111.65, 26.08) --
	(111.65, 26.08) --
	(111.65, 26.08) --
	(111.72, 26.08) --
	(111.72, 26.08) --
	(111.72, 26.08) --
	(111.72, 26.08) --
	(111.72, 26.08);

\node[text=drawColor,anchor=base west,inner sep=0pt, outer sep=0pt, scale=  0.57] at ( 62.97, 43.55) {White LED};
\definecolor{drawColor}{gray}{0.20}

\path[draw=drawColor,line width= 0.5pt,line join=round,line cap=round] ( 16.95, 25.90) rectangle (141.29, 61.21);
\end{scope}
\begin{scope}
\path[clip] (145.79, 65.71) rectangle (270.13,101.01);
\definecolor{fillColor}{RGB}{255,255,255}

\path[fill=fillColor] (145.79, 65.71) rectangle (270.13,101.01);
\definecolor{fillColor}{RGB}{231,107,243}

\path[fill=fillColor] (133.48, 65.72) --
	(133.48, 65.72) --
	(133.55, 65.72) --
	(133.55, 65.72) --
	(133.55, 65.72) --
	(133.63, 65.71) --
	(133.63, 65.71) --
	(133.63, 65.71) --
	(133.70, 65.72) --
	(133.70, 65.72) --
	(133.70, 65.72) --
	(133.78, 65.71) --
	(133.78, 65.71) --
	(133.78, 65.71) --
	(133.86, 65.71) --
	(133.86, 65.71) --
	(133.86, 65.71) --
	(133.93, 65.72) --
	(133.93, 65.72) --
	(133.93, 65.72) --
	(134.01, 65.71) --
	(134.01, 65.71) --
	(134.01, 65.71) --
	(134.08, 65.72) --
	(134.08, 65.72) --
	(134.08, 65.72) --
	(134.16, 65.71) --
	(134.16, 65.71) --
	(134.16, 65.71) --
	(134.24, 65.72) --
	(134.24, 65.72) --
	(134.24, 65.72) --
	(134.31, 65.72) --
	(134.31, 65.72) --
	(134.31, 65.72) --
	(134.39, 65.72) --
	(134.39, 65.72) --
	(134.39, 65.72) --
	(134.47, 65.72) --
	(134.47, 65.72) --
	(134.47, 65.72) --
	(134.54, 65.72) --
	(134.54, 65.72) --
	(134.54, 65.72) --
	(134.62, 65.72) --
	(134.62, 65.72) --
	(134.62, 65.72) --
	(134.69, 65.72) --
	(134.69, 65.72) --
	(134.70, 65.72) --
	(134.77, 65.71) --
	(134.77, 65.71) --
	(134.77, 65.71) --
	(134.85, 65.71) --
	(134.85, 65.71) --
	(134.85, 65.71) --
	(134.92, 65.72) --
	(134.92, 65.72) --
	(134.92, 65.72) --
	(135.00, 65.72) --
	(135.00, 65.72) --
	(135.00, 65.72) --
	(135.08, 65.72) --
	(135.08, 65.72) --
	(135.08, 65.72) --
	(135.15, 65.72) --
	(135.15, 65.72) --
	(135.15, 65.72) --
	(135.23, 65.71) --
	(135.23, 65.71) --
	(135.23, 65.71) --
	(135.31, 65.72) --
	(135.31, 65.72) --
	(135.31, 65.72) --
	(135.38, 65.72) --
	(135.38, 65.72) --
	(135.38, 65.72) --
	(135.46, 65.72) --
	(135.46, 65.72) --
	(135.46, 65.72) --
	(135.53, 65.73) --
	(135.53, 65.73) --
	(135.53, 65.73) --
	(135.61, 65.73) --
	(135.61, 65.73) --
	(135.61, 65.73) --
	(135.69, 65.72) --
	(135.69, 65.72) --
	(135.69, 65.72) --
	(135.76, 65.73) --
	(135.76, 65.73) --
	(135.76, 65.73) --
	(135.84, 65.73) --
	(135.84, 65.73) --
	(135.84, 65.73) --
	(135.91, 65.73) --
	(135.91, 65.73) --
	(135.91, 65.73) --
	(135.99, 65.72) --
	(135.99, 65.72) --
	(135.99, 65.72) --
	(136.07, 65.72) --
	(136.07, 65.72) --
	(136.07, 65.72) --
	(136.14, 65.72) --
	(136.14, 65.72) --
	(136.14, 65.72) --
	(136.22, 65.72) --
	(136.22, 65.72) --
	(136.22, 65.72) --
	(136.30, 65.72) --
	(136.30, 65.72) --
	(136.30, 65.72) --
	(136.37, 65.72) --
	(136.37, 65.72) --
	(136.37, 65.72) --
	(136.45, 65.73) --
	(136.45, 65.73) --
	(136.45, 65.73) --
	(136.52, 65.72) --
	(136.52, 65.72) --
	(136.52, 65.72) --
	(136.60, 65.72) --
	(136.60, 65.72) --
	(136.60, 65.72) --
	(136.68, 65.72) --
	(136.68, 65.72) --
	(136.68, 65.72) --
	(136.75, 65.72) --
	(136.75, 65.72) --
	(136.75, 65.72) --
	(136.83, 65.73) --
	(136.83, 65.73) --
	(136.83, 65.73) --
	(136.90, 65.73) --
	(136.90, 65.73) --
	(136.90, 65.73) --
	(136.98, 65.72) --
	(136.98, 65.72) --
	(136.98, 65.72) --
	(137.06, 65.72) --
	(137.06, 65.72) --
	(137.06, 65.72) --
	(137.13, 65.72) --
	(137.13, 65.72) --
	(137.13, 65.72) --
	(137.21, 65.73) --
	(137.21, 65.73) --
	(137.21, 65.73) --
	(137.29, 65.72) --
	(137.29, 65.72) --
	(137.29, 65.72) --
	(137.36, 65.72) --
	(137.36, 65.72) --
	(137.36, 65.72) --
	(137.44, 65.72) --
	(137.44, 65.72) --
	(137.44, 65.72) --
	(137.51, 65.72) --
	(137.51, 65.72) --
	(137.51, 65.72) --
	(137.59, 65.72) --
	(137.59, 65.72) --
	(137.59, 65.72) --
	(137.66, 65.72) --
	(137.66, 65.72) --
	(137.66, 65.72) --
	(137.74, 65.73) --
	(137.74, 65.73) --
	(137.74, 65.73) --
	(137.82, 65.72) --
	(137.82, 65.72) --
	(137.82, 65.72) --
	(137.89, 65.72) --
	(137.89, 65.72) --
	(137.89, 65.72) --
	(137.97, 65.71) --
	(137.97, 65.71) --
	(137.97, 65.71) --
	(138.05, 65.73) --
	(138.05, 65.73) --
	(138.05, 65.73) --
	(138.12, 65.73) --
	(138.12, 65.73) --
	(138.12, 65.73) --
	(138.20, 65.73) --
	(138.20, 65.73) --
	(138.20, 65.73) --
	(138.27, 65.72) --
	(138.27, 65.72) --
	(138.27, 65.72) --
	(138.35, 65.72) --
	(138.35, 65.72) --
	(138.35, 65.72) --
	(138.43, 65.73) --
	(138.43, 65.73) --
	(138.43, 65.73) --
	(138.50, 65.73) --
	(138.50, 65.73) --
	(138.50, 65.73) --
	(138.58, 65.72) --
	(138.58, 65.72) --
	(138.58, 65.72) --
	(138.65, 65.72) --
	(138.65, 65.72) --
	(138.65, 65.72) --
	(138.73, 65.72) --
	(138.73, 65.72) --
	(138.73, 65.72) --
	(138.81, 65.72) --
	(138.81, 65.72) --
	(138.81, 65.72) --
	(138.88, 65.73) --
	(138.88, 65.73) --
	(138.88, 65.73) --
	(138.96, 65.73) --
	(138.96, 65.73) --
	(138.96, 65.73) --
	(139.03, 65.73) --
	(139.03, 65.73) --
	(139.03, 65.73) --
	(139.11, 65.72) --
	(139.11, 65.72) --
	(139.11, 65.72) --
	(139.19, 65.73) --
	(139.19, 65.73) --
	(139.19, 65.73) --
	(139.26, 65.72) --
	(139.26, 65.72) --
	(139.26, 65.72) --
	(139.34, 65.72) --
	(139.34, 65.72) --
	(139.34, 65.72) --
	(139.42, 65.73) --
	(139.42, 65.73) --
	(139.42, 65.73) --
	(139.49, 65.72) --
	(139.49, 65.72) --
	(139.49, 65.72) --
	(139.57, 65.72) --
	(139.57, 65.72) --
	(139.57, 65.72) --
	(139.64, 65.73) --
	(139.64, 65.73) --
	(139.64, 65.73) --
	(139.72, 65.73) --
	(139.72, 65.73) --
	(139.72, 65.73) --
	(139.79, 65.72) --
	(139.79, 65.72) --
	(139.79, 65.72) --
	(139.87, 65.72) --
	(139.87, 65.72) --
	(139.87, 65.72) --
	(139.95, 65.72) --
	(139.95, 65.72) --
	(139.95, 65.72) --
	(140.02, 65.73) --
	(140.02, 65.73) --
	(140.02, 65.73) --
	(140.10, 65.72) --
	(140.10, 65.72) --
	(140.10, 65.72) --
	(140.17, 65.72) --
	(140.17, 65.72) --
	(140.17, 65.72) --
	(140.25, 65.73) --
	(140.25, 65.73) --
	(140.25, 65.73) --
	(140.33, 65.73) --
	(140.33, 65.73) --
	(140.33, 65.73) --
	(140.40, 65.72) --
	(140.40, 65.72) --
	(140.40, 65.72) --
	(140.48, 65.72) --
	(140.48, 65.72) --
	(140.48, 65.72) --
	(140.55, 65.72) --
	(140.55, 65.72) --
	(140.55, 65.72) --
	(140.63, 65.72) --
	(140.63, 65.72) --
	(140.63, 65.72) --
	(140.71, 65.72) --
	(140.71, 65.72) --
	(140.71, 65.72) --
	(140.78, 65.73) --
	(140.78, 65.73) --
	(140.78, 65.73) --
	(140.86, 65.73) --
	(140.86, 65.73) --
	(140.86, 65.73) --
	(140.93, 65.72) --
	(140.93, 65.72) --
	(140.93, 65.72) --
	(141.01, 65.72) --
	(141.01, 65.72) --
	(141.01, 65.72) --
	(141.08, 65.72) --
	(141.08, 65.72) --
	(141.08, 65.72) --
	(141.16, 65.72) --
	(141.16, 65.72) --
	(141.16, 65.72) --
	(141.24, 65.73) --
	(141.24, 65.73) --
	(141.24, 65.73) --
	(141.31, 65.73) --
	(141.31, 65.73) --
	(141.31, 65.73) --
	(141.39, 65.72) --
	(141.39, 65.72) --
	(141.39, 65.72) --
	(141.46, 65.72) --
	(141.46, 65.72) --
	(141.46, 65.72) --
	(141.54, 65.72) --
	(141.54, 65.72) --
	(141.54, 65.72) --
	(141.62, 65.72) --
	(141.62, 65.72) --
	(141.62, 65.72) --
	(141.69, 65.73) --
	(141.69, 65.73) --
	(141.69, 65.73) --
	(141.77, 65.72) --
	(141.77, 65.72) --
	(141.77, 65.72) --
	(141.84, 65.73) --
	(141.84, 65.73) --
	(141.84, 65.73) --
	(141.92, 65.73) --
	(141.92, 65.73) --
	(141.92, 65.73) --
	(142.00, 65.73) --
	(142.00, 65.73) --
	(142.00, 65.73) --
	(142.07, 65.73) --
	(142.07, 65.73) --
	(142.07, 65.73) --
	(142.15, 65.73) --
	(142.15, 65.73) --
	(142.15, 65.73) --
	(142.22, 65.73) --
	(142.22, 65.73) --
	(142.22, 65.73) --
	(142.30, 65.73) --
	(142.30, 65.73) --
	(142.30, 65.73) --
	(142.37, 65.72) --
	(142.37, 65.72) --
	(142.37, 65.72) --
	(142.45, 65.72) --
	(142.45, 65.72) --
	(142.45, 65.72) --
	(142.53, 65.73) --
	(142.53, 65.73) --
	(142.53, 65.73) --
	(142.60, 65.73) --
	(142.60, 65.73) --
	(142.60, 65.73) --
	(142.68, 65.73) --
	(142.68, 65.73) --
	(142.68, 65.73) --
	(142.75, 65.72) --
	(142.75, 65.72) --
	(142.75, 65.72) --
	(142.83, 65.72) --
	(142.83, 65.72) --
	(142.83, 65.72) --
	(142.91, 65.72) --
	(142.91, 65.72) --
	(142.91, 65.72) --
	(142.98, 65.72) --
	(142.98, 65.72) --
	(142.98, 65.72) --
	(143.06, 65.73) --
	(143.06, 65.73) --
	(143.06, 65.73) --
	(143.13, 65.72) --
	(143.13, 65.72) --
	(143.13, 65.72) --
	(143.21, 65.72) --
	(143.21, 65.72) --
	(143.21, 65.72) --
	(143.29, 65.72) --
	(143.29, 65.72) --
	(143.29, 65.72) --
	(143.36, 65.73) --
	(143.36, 65.73) --
	(143.36, 65.73) --
	(143.44, 65.72) --
	(143.44, 65.72) --
	(143.44, 65.72) --
	(143.51, 65.72) --
	(143.51, 65.72) --
	(143.51, 65.72) --
	(143.59, 65.73) --
	(143.59, 65.73) --
	(143.59, 65.73) --
	(143.66, 65.73) --
	(143.66, 65.73) --
	(143.66, 65.73) --
	(143.74, 65.72) --
	(143.74, 65.72) --
	(143.74, 65.72) --
	(143.82, 65.72) --
	(143.82, 65.72) --
	(143.82, 65.72) --
	(143.89, 65.73) --
	(143.89, 65.73) --
	(143.89, 65.73) --
	(143.97, 65.73) --
	(143.97, 65.73) --
	(143.97, 65.73) --
	(144.04, 65.72) --
	(144.04, 65.72) --
	(144.04, 65.72) --
	(144.12, 65.72) --
	(144.12, 65.72) --
	(144.12, 65.72) --
	(144.19, 65.72) --
	(144.19, 65.72) --
	(144.19, 65.72) --
	(144.27, 65.72) --
	(144.27, 65.72) --
	(144.27, 65.72) --
	(144.35, 65.72) --
	(144.35, 65.72) --
	(144.35, 65.72) --
	(144.42, 65.73) --
	(144.42, 65.73) --
	(144.42, 65.73) --
	(144.50, 65.72) --
	(144.50, 65.72) --
	(144.50, 65.72) --
	(144.57, 65.72) --
	(144.57, 65.72) --
	(144.57, 65.72) --
	(144.65, 65.73) --
	(144.65, 65.73) --
	(144.65, 65.73) --
	(144.72, 65.72) --
	(144.72, 65.72) --
	(144.72, 65.72) --
	(144.80, 65.72) --
	(144.80, 65.72) --
	(144.80, 65.72) --
	(144.88, 65.73) --
	(144.88, 65.73) --
	(144.88, 65.73) --
	(144.95, 65.73) --
	(144.95, 65.73) --
	(144.95, 65.73) --
	(145.03, 65.72) --
	(145.03, 65.72) --
	(145.03, 65.72) --
	(145.10, 65.73) --
	(145.10, 65.73) --
	(145.10, 65.73) --
	(145.18, 65.73) --
	(145.18, 65.73) --
	(145.18, 65.73) --
	(145.26, 65.73) --
	(145.26, 65.73) --
	(145.26, 65.73) --
	(145.33, 65.73) --
	(145.33, 65.73) --
	(145.33, 65.73) --
	(145.41, 65.72) --
	(145.41, 65.72) --
	(145.41, 65.72) --
	(145.48, 65.73) --
	(145.48, 65.73) --
	(145.48, 65.73) --
	(145.56, 65.73) --
	(145.56, 65.73) --
	(145.56, 65.73) --
	(145.63, 65.73) --
	(145.63, 65.73) --
	(145.63, 65.73) --
	(145.71, 65.73) --
	(145.71, 65.73) --
	(145.71, 65.73) --
	(145.79, 65.73) --
	(145.79, 65.73) --
	(145.79, 65.73) --
	(145.86, 65.73) --
	(145.86, 65.73) --
	(145.86, 65.73) --
	(145.94, 65.72) --
	(145.94, 65.72) --
	(145.94, 65.72) --
	(146.01, 65.73) --
	(146.01, 65.73) --
	(146.01, 65.73) --
	(146.09, 65.72) --
	(146.09, 65.72) --
	(146.09, 65.72) --
	(146.16, 65.72) --
	(146.16, 65.72) --
	(146.16, 65.72) --
	(146.24, 65.73) --
	(146.24, 65.73) --
	(146.24, 65.73) --
	(146.32, 65.73) --
	(146.32, 65.73) --
	(146.32, 65.73) --
	(146.39, 65.72) --
	(146.39, 65.72) --
	(146.39, 65.72) --
	(146.47, 65.72) --
	(146.47, 65.72) --
	(146.47, 65.72) --
	(146.54, 65.73) --
	(146.54, 65.73) --
	(146.54, 65.73) --
	(146.62, 65.73) --
	(146.62, 65.73) --
	(146.62, 65.73) --
	(146.69, 65.72) --
	(146.69, 65.72) --
	(146.69, 65.72) --
	(146.77, 65.72) --
	(146.77, 65.72) --
	(146.77, 65.72) --
	(146.84, 65.72) --
	(146.84, 65.72) --
	(146.84, 65.72) --
	(146.92, 65.73) --
	(146.92, 65.73) --
	(146.92, 65.73) --
	(146.99, 65.72) --
	(146.99, 65.72) --
	(146.99, 65.72) --
	(147.07, 65.72) --
	(147.07, 65.72) --
	(147.07, 65.72) --
	(147.15, 65.73) --
	(147.15, 65.73) --
	(147.15, 65.73) --
	(147.22, 65.73) --
	(147.22, 65.73) --
	(147.22, 65.73) --
	(147.30, 65.72) --
	(147.30, 65.72) --
	(147.30, 65.72) --
	(147.37, 65.73) --
	(147.37, 65.73) --
	(147.37, 65.73) --
	(147.45, 65.73) --
	(147.45, 65.73) --
	(147.45, 65.73) --
	(147.52, 65.73) --
	(147.52, 65.73) --
	(147.52, 65.73) --
	(147.60, 65.73) --
	(147.60, 65.73) --
	(147.60, 65.73) --
	(147.68, 65.73) --
	(147.68, 65.73) --
	(147.68, 65.73) --
	(147.75, 65.73) --
	(147.75, 65.73) --
	(147.75, 65.73) --
	(147.83, 65.72) --
	(147.83, 65.72) --
	(147.83, 65.72) --
	(147.90, 65.72) --
	(147.90, 65.72) --
	(147.90, 65.72) --
	(147.98, 65.73) --
	(147.98, 65.73) --
	(147.98, 65.73) --
	(148.05, 65.73) --
	(148.05, 65.73) --
	(148.05, 65.73) --
	(148.13, 65.73) --
	(148.13, 65.73) --
	(148.13, 65.73) --
	(148.20, 65.73) --
	(148.20, 65.73) --
	(148.20, 65.73) --
	(148.28, 65.73) --
	(148.28, 65.73) --
	(148.28, 65.73) --
	(148.36, 65.73) --
	(148.36, 65.73) --
	(148.36, 65.73) --
	(148.43, 65.72) --
	(148.43, 65.72) --
	(148.43, 65.72) --
	(148.51, 65.73) --
	(148.51, 65.73) --
	(148.51, 65.73) --
	(148.58, 65.72) --
	(148.58, 65.72) --
	(148.58, 65.72) --
	(148.66, 65.72) --
	(148.66, 65.72) --
	(148.66, 65.72) --
	(148.73, 65.73) --
	(148.73, 65.73) --
	(148.73, 65.73) --
	(148.81, 65.73) --
	(148.81, 65.73) --
	(148.81, 65.73) --
	(148.88, 65.73) --
	(148.88, 65.73) --
	(148.88, 65.73) --
	(148.96, 65.72) --
	(148.96, 65.72) --
	(148.96, 65.72) --
	(149.04, 65.73) --
	(149.04, 65.73) --
	(149.04, 65.73) --
	(149.11, 65.73) --
	(149.11, 65.73) --
	(149.11, 65.73) --
	(149.19, 65.73) --
	(149.19, 65.73) --
	(149.19, 65.73) --
	(149.26, 65.73) --
	(149.26, 65.73) --
	(149.26, 65.73) --
	(149.34, 65.72) --
	(149.34, 65.72) --
	(149.34, 65.72) --
	(149.41, 65.72) --
	(149.41, 65.72) --
	(149.41, 65.72) --
	(149.49, 65.73) --
	(149.49, 65.73) --
	(149.49, 65.73) --
	(149.56, 65.72) --
	(149.56, 65.72) --
	(149.56, 65.72) --
	(149.64, 65.73) --
	(149.64, 65.73) --
	(149.64, 65.73) --
	(149.72, 65.73) --
	(149.72, 65.73) --
	(149.72, 65.73) --
	(149.79, 65.73) --
	(149.79, 65.73) --
	(149.79, 65.73) --
	(149.87, 65.72) --
	(149.87, 65.72) --
	(149.87, 65.72) --
	(149.94, 65.73) --
	(149.94, 65.73) --
	(149.94, 65.73) --
	(150.02, 65.72) --
	(150.02, 65.72) --
	(150.02, 65.72) --
	(150.09, 65.73) --
	(150.09, 65.73) --
	(150.09, 65.73) --
	(150.17, 65.73) --
	(150.17, 65.73) --
	(150.17, 65.73) --
	(150.24, 65.73) --
	(150.24, 65.73) --
	(150.24, 65.73) --
	(150.32, 65.73) --
	(150.32, 65.73) --
	(150.32, 65.73) --
	(150.39, 65.73) --
	(150.39, 65.73) --
	(150.39, 65.73) --
	(150.47, 65.73) --
	(150.47, 65.73) --
	(150.47, 65.73) --
	(150.55, 65.73) --
	(150.55, 65.73) --
	(150.55, 65.73) --
	(150.62, 65.73) --
	(150.62, 65.73) --
	(150.62, 65.73) --
	(150.70, 65.72) --
	(150.70, 65.72) --
	(150.70, 65.72) --
	(150.77, 65.73) --
	(150.77, 65.73) --
	(150.77, 65.73) --
	(150.85, 65.72) --
	(150.85, 65.72) --
	(150.85, 65.72) --
	(150.92, 65.73) --
	(150.92, 65.73) --
	(150.92, 65.73) --
	(151.00, 65.73) --
	(151.00, 65.73) --
	(151.00, 65.73) --
	(151.07, 65.72) --
	(151.07, 65.72) --
	(151.07, 65.72) --
	(151.15, 65.73) --
	(151.15, 65.73) --
	(151.15, 65.73) --
	(151.22, 65.73) --
	(151.22, 65.73) --
	(151.22, 65.73) --
	(151.30, 65.73) --
	(151.30, 65.73) --
	(151.30, 65.73) --
	(151.37, 65.72) --
	(151.37, 65.72) --
	(151.37, 65.72) --
	(151.45, 65.72) --
	(151.45, 65.72) --
	(151.45, 65.72) --
	(151.53, 65.73) --
	(151.53, 65.73) --
	(151.53, 65.73) --
	(151.60, 65.73) --
	(151.60, 65.73) --
	(151.60, 65.73) --
	(151.68, 65.73) --
	(151.68, 65.73) --
	(151.68, 65.73) --
	(151.75, 65.72) --
	(151.75, 65.72) --
	(151.75, 65.72) --
	(151.83, 65.73) --
	(151.83, 65.73) --
	(151.83, 65.73) --
	(151.90, 65.73) --
	(151.90, 65.73) --
	(151.90, 65.73) --
	(151.98, 65.73) --
	(151.98, 65.73) --
	(151.98, 65.73) --
	(152.05, 65.73) --
	(152.05, 65.73) --
	(152.05, 65.73) --
	(152.13, 65.73) --
	(152.13, 65.73) --
	(152.13, 65.73) --
	(152.20, 65.73) --
	(152.20, 65.73) --
	(152.20, 65.73) --
	(152.28, 65.73) --
	(152.28, 65.73) --
	(152.28, 65.73) --
	(152.35, 65.73) --
	(152.35, 65.73) --
	(152.35, 65.73) --
	(152.43, 65.73) --
	(152.43, 65.73) --
	(152.43, 65.73) --
	(152.51, 65.72) --
	(152.51, 65.72) --
	(152.51, 65.72) --
	(152.58, 65.73) --
	(152.58, 65.73) --
	(152.58, 65.73) --
	(152.66, 65.73) --
	(152.66, 65.73) --
	(152.66, 65.73) --
	(152.73, 65.73) --
	(152.73, 65.73) --
	(152.73, 65.73) --
	(152.81, 65.72) --
	(152.81, 65.72) --
	(152.81, 65.72) --
	(152.88, 65.73) --
	(152.88, 65.73) --
	(152.88, 65.73) --
	(152.96, 65.73) --
	(152.96, 65.73) --
	(152.96, 65.73) --
	(153.03, 65.73) --
	(153.03, 65.73) --
	(153.03, 65.73) --
	(153.11, 65.73) --
	(153.11, 65.73) --
	(153.11, 65.73) --
	(153.11, 65.73) --
	(153.11, 65.73) --
	(153.11, 65.73) --
	(153.18, 65.73) --
	(153.18, 65.73) --
	(153.18, 65.73) --
	(153.26, 65.72) --
	(153.26, 65.72) --
	(153.26, 65.72) --
	(153.26, 65.72) --
	(153.26, 65.72) --
	(153.26, 65.72) --
	(153.33, 65.73) --
	(153.33, 65.73) --
	(153.33, 65.73) --
	(153.35, 65.73) --
	(153.35, 65.73) --
	(153.35, 65.73) --
	(153.39, 65.73) --
	(153.39, 65.73) --
	(153.39, 65.73) --
	(153.41, 65.73) --
	(153.41, 65.73) --
	(153.41, 65.73) --
	(153.47, 65.73) --
	(153.47, 65.73) --
	(153.47, 65.73) --
	(153.48, 65.73) --
	(153.48, 65.73) --
	(153.48, 65.73) --
	(153.56, 65.72) --
	(153.56, 65.72) --
	(153.56, 65.72) --
	(153.59, 65.72) --
	(153.59, 65.72) --
	(153.59, 65.72) --
	(153.63, 65.73) --
	(153.63, 65.73) --
	(153.63, 65.73) --
	(153.67, 65.73) --
	(153.67, 65.73) --
	(153.67, 65.73) --
	(153.71, 65.73) --
	(153.71, 65.73) --
	(153.71, 65.73) --
	(153.79, 65.73) --
	(153.79, 65.73) --
	(153.79, 65.73) --
	(153.86, 65.72) --
	(153.86, 65.72) --
	(153.86, 65.72) --
	(153.94, 65.73) --
	(153.94, 65.73) --
	(153.94, 65.73) --
	(153.95, 65.73) --
	(153.95, 65.73) --
	(153.95, 65.73) --
	(154.01, 65.73) --
	(154.01, 65.73) --
	(154.01, 65.73) --
	(154.09, 65.73) --
	(154.09, 65.73) --
	(154.09, 65.73) --
	(154.16, 65.72) --
	(154.16, 65.72) --
	(154.16, 65.72) --
	(154.24, 65.73) --
	(154.24, 65.73) --
	(154.24, 65.73) --
	(154.31, 65.73) --
	(154.31, 65.73) --
	(154.31, 65.73) --
	(154.39, 65.73) --
	(154.39, 65.73) --
	(154.39, 65.73) --
	(154.46, 65.73) --
	(154.46, 65.73) --
	(154.46, 65.73) --
	(154.54, 65.73) --
	(154.54, 65.73) --
	(154.54, 65.73) --
	(154.61, 65.73) --
	(154.61, 65.73) --
	(154.61, 65.73) --
	(154.64, 65.73) --
	(154.64, 65.73) --
	(154.64, 65.73) --
	(154.69, 65.73) --
	(154.69, 65.73) --
	(154.69, 65.73) --
	(154.76, 65.73) --
	(154.76, 65.73) --
	(154.76, 65.73) --
	(154.84, 65.73) --
	(154.84, 65.73) --
	(154.84, 65.73) --
	(154.91, 65.73) --
	(154.91, 65.73) --
	(154.91, 65.73) --
	(154.99, 65.72) --
	(154.99, 65.72) --
	(154.99, 65.72) --
	(155.06, 65.73) --
	(155.06, 65.73) --
	(155.06, 65.73) --
	(155.14, 65.73) --
	(155.14, 65.73) --
	(155.14, 65.73) --
	(155.22, 65.73) --
	(155.22, 65.73) --
	(155.22, 65.73) --
	(155.29, 65.73) --
	(155.29, 65.73) --
	(155.29, 65.73) --
	(155.37, 65.73) --
	(155.37, 65.73) --
	(155.37, 65.73) --
	(155.44, 65.73) --
	(155.44, 65.73) --
	(155.44, 65.73) --
	(155.52, 65.73) --
	(155.52, 65.73) --
	(155.52, 65.73) --
	(155.59, 65.73) --
	(155.59, 65.73) --
	(155.59, 65.73) --
	(155.67, 65.73) --
	(155.67, 65.73) --
	(155.67, 65.73) --
	(155.74, 65.73) --
	(155.74, 65.73) --
	(155.74, 65.73) --
	(155.82, 65.73) --
	(155.82, 65.73) --
	(155.82, 65.73) --
	(155.89, 65.73) --
	(155.89, 65.73) --
	(155.89, 65.73) --
	(155.90, 65.73) --
	(155.90, 65.73) --
	(155.90, 65.73) --
	(155.97, 65.73) --
	(155.97, 65.73) --
	(155.97, 65.73) --
	(156.04, 65.73) --
	(156.04, 65.73) --
	(156.04, 65.73) --
	(156.12, 65.73) --
	(156.12, 65.73) --
	(156.12, 65.73) --
	(156.19, 65.73) --
	(156.19, 65.73) --
	(156.19, 65.73) --
	(156.23, 65.73) --
	(156.23, 65.73) --
	(156.23, 65.73) --
	(156.27, 65.74) --
	(156.27, 65.74) --
	(156.27, 65.74) --
	(156.34, 65.73) --
	(156.34, 65.73) --
	(156.34, 65.73) --
	(156.42, 65.73) --
	(156.42, 65.73) --
	(156.42, 65.73) --
	(156.49, 65.73) --
	(156.49, 65.73) --
	(156.49, 65.73) --
	(156.50, 65.73) --
	(156.50, 65.73) --
	(156.50, 65.73) --
	(156.57, 65.73) --
	(156.57, 65.73) --
	(156.57, 65.73) --
	(156.64, 65.73) --
	(156.64, 65.73) --
	(156.64, 65.73) --
	(156.72, 65.73) --
	(156.72, 65.73) --
	(156.72, 65.73) --
	(156.79, 65.74) --
	(156.79, 65.74) --
	(156.79, 65.74) --
	(156.87, 65.73) --
	(156.87, 65.73) --
	(156.87, 65.73) --
	(156.87, 65.73) --
	(156.87, 65.73) --
	(156.87, 65.73) --
	(156.94, 65.73) --
	(156.94, 65.73) --
	(156.94, 65.73) --
	(157.02, 65.73) --
	(157.02, 65.73) --
	(157.02, 65.73) --
	(157.03, 65.73) --
	(157.03, 65.73) --
	(157.03, 65.73) --
	(157.09, 65.73) --
	(157.09, 65.73) --
	(157.09, 65.73) --
	(157.17, 65.73) --
	(157.17, 65.73) --
	(157.17, 65.73) --
	(157.23, 65.73) --
	(157.23, 65.73) --
	(157.23, 65.73) --
	(157.24, 65.73) --
	(157.24, 65.73) --
	(157.24, 65.73) --
	(157.32, 65.73) --
	(157.32, 65.73) --
	(157.32, 65.73) --
	(157.39, 65.73) --
	(157.39, 65.73) --
	(157.39, 65.73) --
	(157.47, 65.73) --
	(157.47, 65.73) --
	(157.47, 65.73) --
	(157.54, 65.73) --
	(157.54, 65.73) --
	(157.54, 65.73) --
	(157.62, 65.73) --
	(157.62, 65.73) --
	(157.62, 65.73) --
	(157.69, 65.73) --
	(157.69, 65.73) --
	(157.69, 65.73) --
	(157.77, 65.73) --
	(157.77, 65.73) --
	(157.77, 65.73) --
	(157.84, 65.73) --
	(157.84, 65.73) --
	(157.84, 65.73) --
	(157.92, 65.73) --
	(157.92, 65.73) --
	(157.92, 65.73) --
	(157.99, 65.73) --
	(157.99, 65.73) --
	(157.99, 65.73) --
	(158.04, 65.73) --
	(158.04, 65.73) --
	(158.04, 65.73) --
	(158.07, 65.73) --
	(158.07, 65.73) --
	(158.07, 65.73) --
	(158.14, 65.73) --
	(158.15, 65.73) --
	(158.15, 65.73) --
	(158.22, 65.73) --
	(158.22, 65.73) --
	(158.22, 65.73) --
	(158.24, 65.73) --
	(158.24, 65.73) --
	(158.24, 65.73) --
	(158.30, 65.73) --
	(158.30, 65.73) --
	(158.30, 65.73) --
	(158.37, 65.73) --
	(158.37, 65.73) --
	(158.37, 65.73) --
	(158.45, 65.73) --
	(158.45, 65.73) --
	(158.45, 65.73) --
	(158.52, 65.72) --
	(158.52, 65.72) --
	(158.52, 65.72) --
	(158.60, 65.73) --
	(158.60, 65.73) --
	(158.60, 65.73) --
	(158.67, 65.74) --
	(158.67, 65.74) --
	(158.67, 65.74) --
	(158.75, 65.73) --
	(158.75, 65.73) --
	(158.75, 65.73) --
	(158.76, 65.73) --
	(158.76, 65.73) --
	(158.76, 65.73) --
	(158.82, 65.74) --
	(158.82, 65.74) --
	(158.82, 65.74) --
	(158.89, 65.73) --
	(158.89, 65.73) --
	(158.89, 65.73) --
	(158.97, 65.74) --
	(158.97, 65.74) --
	(158.97, 65.74) --
	(159.04, 65.73) --
	(159.04, 65.73) --
	(159.04, 65.73) --
	(159.12, 65.73) --
	(159.12, 65.73) --
	(159.12, 65.73) --
	(159.19, 65.73) --
	(159.19, 65.73) --
	(159.19, 65.73) --
	(159.25, 65.72) --
	(159.25, 65.72) --
	(159.25, 65.72) --
	(159.27, 65.72) --
	(159.27, 65.72) --
	(159.27, 65.72) --
	(159.34, 65.73) --
	(159.34, 65.73) --
	(159.34, 65.73) --
	(159.42, 65.73) --
	(159.42, 65.73) --
	(159.42, 65.73) --
	(159.49, 65.74) --
	(159.49, 65.74) --
	(159.49, 65.74) --
	(159.57, 65.73) --
	(159.57, 65.73) --
	(159.57, 65.73) --
	(159.57, 65.73) --
	(159.57, 65.73) --
	(159.57, 65.73) --
	(159.65, 65.73) --
	(159.65, 65.73) --
	(159.65, 65.73) --
	(159.72, 65.73) --
	(159.72, 65.73) --
	(159.72, 65.73) --
	(159.78, 65.73) --
	(159.78, 65.73) --
	(159.78, 65.73) --
	(159.80, 65.73) --
	(159.80, 65.73) --
	(159.80, 65.73) --
	(159.87, 65.73) --
	(159.87, 65.73) --
	(159.87, 65.73) --
	(159.94, 65.73) --
	(159.94, 65.73) --
	(159.94, 65.73) --
	(159.98, 65.73) --
	(159.98, 65.73) --
	(159.98, 65.73) --
	(160.02, 65.73) --
	(160.02, 65.73) --
	(160.02, 65.73) --
	(160.09, 65.74) --
	(160.09, 65.74) --
	(160.09, 65.74) --
	(160.14, 65.74) --
	(160.14, 65.74) --
	(160.14, 65.74) --
	(160.17, 65.74) --
	(160.17, 65.74) --
	(160.17, 65.74) --
	(160.24, 65.74) --
	(160.24, 65.74) --
	(160.24, 65.74) --
	(160.26, 65.74) --
	(160.26, 65.74) --
	(160.26, 65.74) --
	(160.30, 65.73) --
	(160.30, 65.73) --
	(160.30, 65.73) --
	(160.32, 65.73) --
	(160.32, 65.73) --
	(160.32, 65.73) --
	(160.39, 65.73) --
	(160.39, 65.73) --
	(160.39, 65.73) --
	(160.42, 65.73) --
	(160.42, 65.73) --
	(160.42, 65.73) --
	(160.47, 65.73) --
	(160.47, 65.73) --
	(160.47, 65.73) --
	(160.54, 65.73) --
	(160.54, 65.73) --
	(160.54, 65.73) --
	(160.54, 65.73) --
	(160.54, 65.73) --
	(160.54, 65.73) --
	(160.58, 65.73) --
	(160.58, 65.73) --
	(160.58, 65.73) --
	(160.62, 65.73) --
	(160.62, 65.73) --
	(160.62, 65.73) --
	(160.69, 65.74) --
	(160.69, 65.74) --
	(160.69, 65.74) --
	(160.71, 65.73) --
	(160.71, 65.73) --
	(160.71, 65.73) --
	(160.77, 65.73) --
	(160.77, 65.73) --
	(160.77, 65.73) --
	(160.79, 65.73) --
	(160.79, 65.73) --
	(160.79, 65.73) --
	(160.84, 65.73) --
	(160.84, 65.73) --
	(160.84, 65.73) --
	(160.92, 65.73) --
	(160.92, 65.73) --
	(160.92, 65.73) --
	(160.95, 65.73) --
	(160.95, 65.73) --
	(160.95, 65.73) --
	(160.99, 65.74) --
	(160.99, 65.74) --
	(160.99, 65.74) --
	(161.07, 65.74) --
	(161.07, 65.74) --
	(161.07, 65.74) --
	(161.07, 65.74) --
	(161.07, 65.74) --
	(161.07, 65.74) --
	(161.11, 65.74) --
	(161.11, 65.74) --
	(161.11, 65.74) --
	(161.14, 65.74) --
	(161.14, 65.74) --
	(161.14, 65.74) --
	(161.22, 65.74) --
	(161.22, 65.74) --
	(161.22, 65.74) --
	(161.23, 65.74) --
	(161.23, 65.74) --
	(161.23, 65.74) --
	(161.27, 65.74) --
	(161.27, 65.74) --
	(161.27, 65.74) --
	(161.29, 65.74) --
	(161.29, 65.74) --
	(161.29, 65.74) --
	(161.31, 65.74) --
	(161.31, 65.74) --
	(161.31, 65.74) --
	(161.31, 65.73) --
	(161.31, 65.73) --
	(161.31, 65.73) --
	(161.37, 65.73) --
	(161.37, 65.73) --
	(161.37, 65.73) --
	(161.39, 65.73) --
	(161.39, 65.73) --
	(161.39, 65.73) --
	(161.43, 65.73) --
	(161.43, 65.73) --
	(161.43, 65.73) --
	(161.44, 65.73) --
	(161.44, 65.73) --
	(161.44, 65.73) --
	(161.51, 65.74) --
	(161.51, 65.74) --
	(161.51, 65.74) --
	(161.52, 65.74) --
	(161.52, 65.74) --
	(161.52, 65.74) --
	(161.55, 65.74) --
	(161.55, 65.74) --
	(161.55, 65.74) --
	(161.59, 65.74) --
	(161.59, 65.74) --
	(161.59, 65.74) --
	(161.67, 65.74) --
	(161.67, 65.74) --
	(161.67, 65.74) --
	(161.74, 65.74) --
	(161.74, 65.74) --
	(161.74, 65.74) --
	(161.80, 65.74) --
	(161.80, 65.74) --
	(161.80, 65.74) --
	(161.82, 65.74) --
	(161.82, 65.74) --
	(161.82, 65.74) --
	(161.84, 65.74) --
	(161.84, 65.74) --
	(161.84, 65.74) --
	(161.89, 65.74) --
	(161.89, 65.74) --
	(161.89, 65.74) --
	(161.92, 65.74) --
	(161.92, 65.74) --
	(161.92, 65.74) --
	(161.96, 65.74) --
	(161.96, 65.74) --
	(161.96, 65.74) --
	(161.97, 65.74) --
	(161.97, 65.74) --
	(161.97, 65.74) --
	(162.00, 65.74) --
	(162.00, 65.74) --
	(162.00, 65.74) --
	(162.04, 65.74) --
	(162.04, 65.74) --
	(162.04, 65.74) --
	(162.04, 65.74) --
	(162.04, 65.74) --
	(162.04, 65.74) --
	(162.12, 65.75) --
	(162.12, 65.75) --
	(162.12, 65.75) --
	(162.12, 65.75) --
	(162.12, 65.75) --
	(162.12, 65.75) --
	(162.19, 65.75) --
	(162.19, 65.75) --
	(162.19, 65.75) --
	(162.27, 65.74) --
	(162.27, 65.74) --
	(162.27, 65.74) --
	(162.28, 65.74) --
	(162.28, 65.74) --
	(162.28, 65.74) --
	(162.34, 65.75) --
	(162.34, 65.75) --
	(162.34, 65.75) --
	(162.42, 65.76) --
	(162.42, 65.76) --
	(162.42, 65.76) --
	(162.48, 65.75) --
	(162.48, 65.75) --
	(162.48, 65.75) --
	(162.49, 65.75) --
	(162.49, 65.75) --
	(162.49, 65.75) --
	(162.56, 65.75) --
	(162.56, 65.75) --
	(162.56, 65.75) --
	(162.64, 65.75) --
	(162.64, 65.75) --
	(162.64, 65.75) --
	(162.69, 65.76) --
	(162.69, 65.76) --
	(162.69, 65.76) --
	(162.71, 65.76) --
	(162.71, 65.76) --
	(162.71, 65.76) --
	(162.73, 65.76) --
	(162.73, 65.76) --
	(162.73, 65.76) --
	(162.79, 65.76) --
	(162.79, 65.76) --
	(162.79, 65.76) --
	(162.87, 65.76) --
	(162.87, 65.76) --
	(162.87, 65.76) --
	(162.89, 65.76) --
	(162.89, 65.76) --
	(162.89, 65.76) --
	(162.93, 65.76) --
	(162.93, 65.76) --
	(162.93, 65.76) --
	(162.94, 65.76) --
	(162.94, 65.76) --
	(162.94, 65.76) --
	(163.01, 65.76) --
	(163.01, 65.76) --
	(163.01, 65.76) --
	(163.05, 65.76) --
	(163.05, 65.76) --
	(163.05, 65.76) --
	(163.09, 65.76) --
	(163.09, 65.76) --
	(163.09, 65.76) --
	(163.09, 65.76) --
	(163.09, 65.76) --
	(163.09, 65.76) --
	(163.16, 65.77) --
	(163.16, 65.77) --
	(163.16, 65.77) --
	(163.24, 65.77) --
	(163.24, 65.77) --
	(163.24, 65.77) --
	(163.25, 65.76) --
	(163.25, 65.76) --
	(163.25, 65.76) --
	(163.31, 65.76) --
	(163.31, 65.76) --
	(163.31, 65.76) --
	(163.37, 65.76) --
	(163.37, 65.76) --
	(163.37, 65.76) --
	(163.39, 65.76) --
	(163.39, 65.76) --
	(163.39, 65.76) --
	(163.45, 65.76) --
	(163.45, 65.76) --
	(163.45, 65.76) --
	(163.46, 65.76) --
	(163.46, 65.76) --
	(163.46, 65.76) --
	(163.53, 65.77) --
	(163.53, 65.77) --
	(163.53, 65.77) --
	(163.54, 65.77) --
	(163.54, 65.77) --
	(163.54, 65.77) --
	(163.57, 65.77) --
	(163.57, 65.77) --
	(163.57, 65.77) --
	(163.61, 65.76) --
	(163.61, 65.76) --
	(163.61, 65.76) --
	(163.66, 65.76) --
	(163.66, 65.76) --
	(163.66, 65.76) --
	(163.69, 65.77) --
	(163.69, 65.77) --
	(163.69, 65.77) --
	(163.70, 65.77) --
	(163.70, 65.77) --
	(163.70, 65.77) --
	(163.71, 65.77) --
	(163.71, 65.77) --
	(163.71, 65.77) --
	(163.76, 65.76) --
	(163.76, 65.76) --
	(163.76, 65.76) --
	(163.82, 65.76) --
	(163.82, 65.76) --
	(163.82, 65.76) --
	(163.84, 65.76) --
	(163.84, 65.76) --
	(163.84, 65.76) --
	(163.90, 65.77) --
	(163.90, 65.77) --
	(163.90, 65.77) --
	(163.91, 65.77) --
	(163.91, 65.77) --
	(163.91, 65.77) --
	(163.99, 65.77) --
	(163.99, 65.77) --
	(163.99, 65.77) --
	(164.02, 65.77) --
	(164.02, 65.77) --
	(164.02, 65.77) --
	(164.06, 65.77) --
	(164.06, 65.77) --
	(164.06, 65.77) --
	(164.14, 65.78) --
	(164.14, 65.78) --
	(164.14, 65.78) --
	(164.18, 65.77) --
	(164.18, 65.77) --
	(164.18, 65.77) --
	(164.21, 65.77) --
	(164.21, 65.77) --
	(164.21, 65.77) --
	(164.22, 65.77) --
	(164.22, 65.77) --
	(164.22, 65.77) --
	(164.28, 65.77) --
	(164.28, 65.77) --
	(164.28, 65.77) --
	(164.34, 65.78) --
	(164.34, 65.78) --
	(164.34, 65.78) --
	(164.36, 65.78) --
	(164.36, 65.78) --
	(164.36, 65.78) --
	(164.42, 65.78) --
	(164.42, 65.78) --
	(164.42, 65.78) --
	(164.43, 65.78) --
	(164.43, 65.78) --
	(164.43, 65.78) --
	(164.51, 65.79) --
	(164.51, 65.79) --
	(164.51, 65.79) --
	(164.55, 65.79) --
	(164.55, 65.79) --
	(164.55, 65.79) --
	(164.58, 65.80) --
	(164.58, 65.80) --
	(164.58, 65.80) --
	(164.66, 65.80) --
	(164.66, 65.80) --
	(164.66, 65.80) --
	(164.66, 65.80) --
	(164.66, 65.80) --
	(164.66, 65.80) --
	(164.67, 65.80) --
	(164.67, 65.80) --
	(164.67, 65.80) --
	(164.71, 65.80) --
	(164.71, 65.80) --
	(164.71, 65.80) --
	(164.73, 65.80) --
	(164.73, 65.80) --
	(164.73, 65.80) --
	(164.81, 65.82) --
	(164.81, 65.82) --
	(164.81, 65.82) --
	(164.88, 65.82) --
	(164.88, 65.82) --
	(164.88, 65.82) --
	(164.95, 65.82) --
	(164.95, 65.82) --
	(164.95, 65.82) --
	(164.96, 65.83) --
	(164.96, 65.83) --
	(164.96, 65.83) --
	(164.99, 65.83) --
	(164.99, 65.83) --
	(164.99, 65.83) --
	(165.03, 65.84) --
	(165.03, 65.84) --
	(165.03, 65.84) --
	(165.07, 65.85) --
	(165.07, 65.85) --
	(165.07, 65.85) --
	(165.11, 65.85) --
	(165.11, 65.85) --
	(165.11, 65.85) --
	(165.18, 65.86) --
	(165.18, 65.86) --
	(165.18, 65.86) --
	(165.25, 65.87) --
	(165.25, 65.87) --
	(165.25, 65.87) --
	(165.27, 65.87) --
	(165.27, 65.87) --
	(165.27, 65.87) --
	(165.33, 65.89) --
	(165.33, 65.89) --
	(165.33, 65.89) --
	(165.41, 65.89) --
	(165.41, 65.89) --
	(165.41, 65.89) --
	(165.48, 65.90) --
	(165.48, 65.90) --
	(165.48, 65.90) --
	(165.55, 65.92) --
	(165.55, 65.92) --
	(165.55, 65.92) --
	(165.60, 65.93) --
	(165.60, 65.93) --
	(165.60, 65.93) --
	(165.63, 65.93) --
	(165.63, 65.93) --
	(165.63, 65.93) --
	(165.70, 65.95) --
	(165.70, 65.95) --
	(165.70, 65.95) --
	(165.76, 65.96) --
	(165.76, 65.96) --
	(165.76, 65.96) --
	(165.78, 65.96) --
	(165.78, 65.96) --
	(165.78, 65.96) --
	(165.85, 65.98) --
	(165.85, 65.98) --
	(165.85, 65.98) --
	(165.93, 66.00) --
	(165.93, 66.00) --
	(165.93, 66.00) --
	(166.00, 66.01) --
	(166.00, 66.01) --
	(166.00, 66.01) --
	(166.00, 66.01) --
	(166.00, 66.01) --
	(166.00, 66.01) --
	(166.08, 66.02) --
	(166.08, 66.02) --
	(166.08, 66.02) --
	(166.15, 66.05) --
	(166.15, 66.05) --
	(166.15, 66.05) --
	(166.16, 66.05) --
	(166.16, 66.05) --
	(166.16, 66.05) --
	(166.23, 66.07) --
	(166.23, 66.07) --
	(166.23, 66.07) --
	(166.30, 66.09) --
	(166.30, 66.09) --
	(166.30, 66.09) --
	(166.38, 66.11) --
	(166.38, 66.11) --
	(166.38, 66.11) --
	(166.44, 66.13) --
	(166.44, 66.13) --
	(166.44, 66.13) --
	(166.45, 66.13) --
	(166.45, 66.13) --
	(166.45, 66.13) --
	(166.48, 66.14) --
	(166.48, 66.14) --
	(166.48, 66.14) --
	(166.52, 66.15) --
	(166.52, 66.15) --
	(166.52, 66.15) --
	(166.60, 66.18) --
	(166.60, 66.18) --
	(166.60, 66.18) --
	(166.67, 66.19) --
	(166.67, 66.19) --
	(166.67, 66.19) --
	(166.75, 66.21) --
	(166.75, 66.21) --
	(166.75, 66.21) --
	(166.81, 66.24) --
	(166.81, 66.24) --
	(166.81, 66.24) --
	(166.82, 66.24) --
	(166.82, 66.24) --
	(166.82, 66.24) --
	(166.90, 66.26) --
	(166.90, 66.26) --
	(166.90, 66.26) --
	(166.97, 66.29) --
	(166.97, 66.29) --
	(166.97, 66.29) --
	(167.05, 66.32) --
	(167.05, 66.32) --
	(167.05, 66.32) --
	(167.12, 66.32) --
	(167.12, 66.32) --
	(167.12, 66.32) --
	(167.17, 66.34) --
	(167.17, 66.34) --
	(167.17, 66.34) --
	(167.20, 66.35) --
	(167.20, 66.35) --
	(167.20, 66.35) --
	(167.27, 66.36) --
	(167.27, 66.36) --
	(167.27, 66.36) --
	(167.34, 66.38) --
	(167.34, 66.38) --
	(167.34, 66.38) --
	(167.42, 66.41) --
	(167.42, 66.41) --
	(167.42, 66.41) --
	(167.49, 66.42) --
	(167.49, 66.42) --
	(167.49, 66.42) --
	(167.57, 66.44) --
	(167.57, 66.44) --
	(167.57, 66.44) --
	(167.64, 66.43) --
	(167.64, 66.43) --
	(167.64, 66.43) --
	(167.66, 66.44) --
	(167.66, 66.44) --
	(167.66, 66.44) --
	(167.72, 66.46) --
	(167.72, 66.46) --
	(167.72, 66.46) --
	(167.79, 66.47) --
	(167.79, 66.47) --
	(167.79, 66.47) --
	(167.83, 66.47) --
	(167.83, 66.47) --
	(167.83, 66.47) --
	(167.87, 66.48) --
	(167.87, 66.48) --
	(167.87, 66.48) --
	(167.94, 66.49) --
	(167.94, 66.49) --
	(167.94, 66.49) --
	(167.98, 66.51) --
	(167.98, 66.51) --
	(167.98, 66.51) --
	(168.01, 66.52) --
	(168.01, 66.52) --
	(168.01, 66.52) --
	(168.09, 66.53) --
	(168.09, 66.53) --
	(168.09, 66.53) --
	(168.16, 66.55) --
	(168.16, 66.55) --
	(168.16, 66.55) --
	(168.24, 66.54) --
	(168.24, 66.54) --
	(168.24, 66.54) --
	(168.30, 66.57) --
	(168.30, 66.57) --
	(168.30, 66.57) --
	(168.31, 66.57) --
	(168.31, 66.57) --
	(168.31, 66.57) --
	(168.34, 66.57) --
	(168.34, 66.57) --
	(168.34, 66.57) --
	(168.39, 66.57) --
	(168.39, 66.57) --
	(168.39, 66.57) --
	(168.46, 66.57) --
	(168.46, 66.57) --
	(168.46, 66.57) --
	(168.51, 66.59) --
	(168.51, 66.59) --
	(168.51, 66.59) --
	(168.54, 66.61) --
	(168.54, 66.61) --
	(168.54, 66.61) --
	(168.55, 66.61) --
	(168.55, 66.61) --
	(168.55, 66.61) --
	(168.61, 66.63) --
	(168.61, 66.63) --
	(168.61, 66.63) --
	(168.63, 66.63) --
	(168.63, 66.63) --
	(168.63, 66.63) --
	(168.67, 66.64) --
	(168.67, 66.64) --
	(168.67, 66.64) --
	(168.69, 66.64) --
	(168.69, 66.64) --
	(168.69, 66.64) --
	(168.71, 66.64) --
	(168.71, 66.64) --
	(168.71, 66.64) --
	(168.75, 66.65) --
	(168.75, 66.65) --
	(168.75, 66.65) --
	(168.76, 66.65) --
	(168.76, 66.65) --
	(168.76, 66.65) --
	(168.83, 66.65) --
	(168.83, 66.65) --
	(168.83, 66.65) --
	(168.87, 66.67) --
	(168.87, 66.67) --
	(168.87, 66.67) --
	(168.91, 66.69) --
	(168.91, 66.69) --
	(168.91, 66.69) --
	(168.98, 66.69) --
	(168.98, 66.69) --
	(168.98, 66.69) --
	(169.06, 66.71) --
	(169.06, 66.71) --
	(169.06, 66.71) --
	(169.13, 66.73) --
	(169.13, 66.73) --
	(169.13, 66.73) --
	(169.19, 66.73) --
	(169.19, 66.73) --
	(169.19, 66.73) --
	(169.21, 66.73) --
	(169.21, 66.73) --
	(169.21, 66.73) --
	(169.28, 66.73) --
	(169.28, 66.73) --
	(169.28, 66.73) --
	(169.36, 66.75) --
	(169.36, 66.75) --
	(169.36, 66.75) --
	(169.36, 66.75) --
	(169.36, 66.75) --
	(169.36, 66.75) --
	(169.43, 66.75) --
	(169.43, 66.75) --
	(169.43, 66.75) --
	(169.51, 66.78) --
	(169.51, 66.78) --
	(169.51, 66.78) --
	(169.58, 66.80) --
	(169.58, 66.80) --
	(169.58, 66.80) --
	(169.65, 66.85) --
	(169.65, 66.85) --
	(169.65, 66.85) --
	(169.73, 66.81) --
	(169.73, 66.81) --
	(169.73, 66.81) --
	(169.76, 66.82) --
	(169.76, 66.82) --
	(169.76, 66.82) --
	(169.80, 66.83) --
	(169.80, 66.83) --
	(169.80, 66.83) --
	(169.88, 66.83) --
	(169.88, 66.83) --
	(169.88, 66.83) --
	(169.95, 66.83) --
	(169.95, 66.83) --
	(169.95, 66.83) --
	(170.03, 66.83) --
	(170.03, 66.83) --
	(170.03, 66.83) --
	(170.10, 66.85) --
	(170.10, 66.85) --
	(170.10, 66.85) --
	(170.13, 66.86) --
	(170.13, 66.86) --
	(170.13, 66.86) --
	(170.17, 66.87) --
	(170.17, 66.87) --
	(170.17, 66.87) --
	(170.25, 66.88) --
	(170.25, 66.88) --
	(170.25, 66.88) --
	(170.32, 66.89) --
	(170.32, 66.89) --
	(170.32, 66.89) --
	(170.40, 66.91) --
	(170.40, 66.91) --
	(170.40, 66.91) --
	(170.47, 66.94) --
	(170.47, 66.94) --
	(170.47, 66.94) --
	(170.51, 66.95) --
	(170.51, 66.95) --
	(170.51, 66.95) --
	(170.55, 66.95) --
	(170.55, 66.95) --
	(170.55, 66.95) --
	(170.62, 66.97) --
	(170.62, 66.97) --
	(170.62, 66.97) --
	(170.70, 67.02) --
	(170.70, 67.02) --
	(170.70, 67.02) --
	(170.77, 67.04) --
	(170.77, 67.04) --
	(170.77, 67.04) --
	(170.84, 67.08) --
	(170.84, 67.08) --
	(170.84, 67.08) --
	(170.89, 67.06) --
	(170.89, 67.06) --
	(170.89, 67.06) --
	(170.92, 67.06) --
	(170.92, 67.06) --
	(170.92, 67.06) --
	(170.99, 67.09) --
	(170.99, 67.09) --
	(170.99, 67.09) --
	(171.05, 67.11) --
	(171.05, 67.11) --
	(171.05, 67.11) --
	(171.07, 67.11) --
	(171.07, 67.11) --
	(171.07, 67.11) --
	(171.14, 67.07) --
	(171.14, 67.07) --
	(171.14, 67.07) --
	(171.22, 67.08) --
	(171.22, 67.08) --
	(171.22, 67.08) --
	(171.29, 67.08) --
	(171.29, 67.08) --
	(171.29, 67.08) --
	(171.36, 67.12) --
	(171.36, 67.12) --
	(171.36, 67.12) --
	(171.37, 67.12) --
	(171.37, 67.12) --
	(171.37, 67.12) --
	(171.44, 67.15) --
	(171.44, 67.15) --
	(171.44, 67.15) --
	(171.51, 67.13) --
	(171.51, 67.13) --
	(171.51, 67.13) --
	(171.59, 67.19) --
	(171.59, 67.19) --
	(171.59, 67.19) --
	(171.66, 67.16) --
	(171.66, 67.16) --
	(171.66, 67.16) --
	(171.66, 67.16) --
	(171.66, 67.16) --
	(171.66, 67.16) --
	(171.74, 67.18) --
	(171.74, 67.18) --
	(171.74, 67.18) --
	(171.81, 67.22) --
	(171.81, 67.22) --
	(171.81, 67.22) --
	(171.85, 67.25) --
	(171.85, 67.25) --
	(171.85, 67.25) --
	(171.88, 67.28) --
	(171.88, 67.28) --
	(171.88, 67.28) --
	(171.96, 67.29) --
	(171.96, 67.29) --
	(171.96, 67.29) --
	(172.03, 67.28) --
	(172.03, 67.28) --
	(172.03, 67.28) --
	(172.11, 67.34) --
	(172.11, 67.34) --
	(172.11, 67.34) --
	(172.14, 67.34) --
	(172.14, 67.34) --
	(172.14, 67.34) --
	(172.18, 67.33) --
	(172.18, 67.33) --
	(172.18, 67.33) --
	(172.26, 67.33) --
	(172.26, 67.33) --
	(172.26, 67.33) --
	(172.33, 67.36) --
	(172.33, 67.36) --
	(172.33, 67.36) --
	(172.33, 67.36) --
	(172.33, 67.36) --
	(172.33, 67.36) --
	(172.40, 67.40) --
	(172.40, 67.40) --
	(172.40, 67.40) --
	(172.43, 67.41) --
	(172.43, 67.41) --
	(172.43, 67.41) --
	(172.48, 67.42) --
	(172.48, 67.42) --
	(172.48, 67.42) --
	(172.55, 67.42) --
	(172.55, 67.42) --
	(172.55, 67.42) --
	(172.63, 67.49) --
	(172.63, 67.49) --
	(172.63, 67.49) --
	(172.70, 67.42) --
	(172.70, 67.42) --
	(172.70, 67.42) --
	(172.78, 67.46) --
	(172.78, 67.46) --
	(172.78, 67.46) --
	(172.81, 67.45) --
	(172.81, 67.45) --
	(172.81, 67.45) --
	(172.85, 67.44) --
	(172.85, 67.44) --
	(172.85, 67.44) --
	(172.92, 67.53) --
	(172.92, 67.53) --
	(172.92, 67.53) --
	(173.00, 67.51) --
	(173.00, 67.51) --
	(173.00, 67.51) --
	(173.00, 67.51) --
	(173.00, 67.51) --
	(173.00, 67.51) --
	(173.07, 67.51) --
	(173.07, 67.51) --
	(173.07, 67.51) --
	(173.15, 67.53) --
	(173.15, 67.53) --
	(173.15, 67.53) --
	(173.22, 67.57) --
	(173.22, 67.57) --
	(173.22, 67.57) --
	(173.29, 67.59) --
	(173.29, 67.59) --
	(173.29, 67.59) --
	(173.29, 67.59) --
	(173.29, 67.59) --
	(173.29, 67.59) --
	(173.37, 67.60) --
	(173.37, 67.60) --
	(173.37, 67.60) --
	(173.44, 67.63) --
	(173.44, 67.63) --
	(173.44, 67.63) --
	(173.48, 67.60) --
	(173.48, 67.60) --
	(173.48, 67.60) --
	(173.52, 67.57) --
	(173.52, 67.57) --
	(173.52, 67.57) --
	(173.58, 67.60) --
	(173.58, 67.60) --
	(173.58, 67.60) --
	(173.59, 67.60) --
	(173.59, 67.60) --
	(173.59, 67.60) --
	(173.67, 67.69) --
	(173.67, 67.69) --
	(173.67, 67.69) --
	(173.74, 67.68) --
	(173.74, 67.68) --
	(173.74, 67.68) --
	(173.77, 67.69) --
	(173.77, 67.69) --
	(173.77, 67.69) --
	(173.81, 67.70) --
	(173.81, 67.70) --
	(173.81, 67.70) --
	(173.86, 67.77) --
	(173.86, 67.77) --
	(173.86, 67.77) --
	(173.89, 67.80) --
	(173.89, 67.80) --
	(173.89, 67.80) --
	(173.96, 67.69) --
	(173.96, 67.69) --
	(173.96, 67.69) --
	(174.04, 67.67) --
	(174.04, 67.67) --
	(174.04, 67.67) --
	(174.11, 67.67) --
	(174.11, 67.67) --
	(174.11, 67.67) --
	(174.15, 67.68) --
	(174.15, 67.68) --
	(174.15, 67.68) --
	(174.18, 67.70) --
	(174.18, 67.70) --
	(174.18, 67.70) --
	(174.25, 67.69) --
	(174.25, 67.69) --
	(174.25, 67.69) --
	(174.26, 67.69) --
	(174.26, 67.69) --
	(174.26, 67.69) --
	(174.33, 67.72) --
	(174.33, 67.72) --
	(174.33, 67.72) --
	(174.34, 67.72) --
	(174.34, 67.72) --
	(174.34, 67.72) --
	(174.41, 67.71) --
	(174.41, 67.71) --
	(174.41, 67.71) --
	(174.44, 67.72) --
	(174.44, 67.72) --
	(174.44, 67.72) --
	(174.48, 67.74) --
	(174.48, 67.74) --
	(174.48, 67.74) --
	(174.53, 67.77) --
	(174.53, 67.77) --
	(174.53, 67.77) --
	(174.56, 67.78) --
	(174.56, 67.78) --
	(174.56, 67.78) --
	(174.63, 67.84) --
	(174.63, 67.84) --
	(174.63, 67.84) --
	(174.63, 67.84) --
	(174.63, 67.84) --
	(174.63, 67.84) --
	(174.70, 67.79) --
	(174.70, 67.79) --
	(174.70, 67.79) --
	(174.78, 67.79) --
	(174.78, 67.79) --
	(174.78, 67.79) --
	(174.82, 67.82) --
	(174.82, 67.82) --
	(174.82, 67.82) --
	(174.85, 67.83) --
	(174.85, 67.83) --
	(174.85, 67.83) --
	(174.92, 67.86) --
	(174.92, 67.86) --
	(174.92, 67.86) --
	(174.93, 67.86) --
	(174.93, 67.86) --
	(174.93, 67.86) --
	(175.00, 67.84) --
	(175.00, 67.84) --
	(175.00, 67.84) --
	(175.01, 67.86) --
	(175.01, 67.86) --
	(175.01, 67.86) --
	(175.07, 67.92) --
	(175.07, 67.92) --
	(175.07, 67.92) --
	(175.15, 67.91) --
	(175.15, 67.91) --
	(175.15, 67.91) --
	(175.22, 67.93) --
	(175.22, 67.93) --
	(175.22, 67.93) --
	(175.30, 67.95) --
	(175.30, 67.95) --
	(175.30, 67.95) --
	(175.30, 67.96) --
	(175.30, 67.96) --
	(175.30, 67.96) --
	(175.37, 68.06) --
	(175.37, 68.06) --
	(175.37, 68.06) --
	(175.45, 68.08) --
	(175.45, 68.08) --
	(175.45, 68.08) --
	(175.52, 68.15) --
	(175.52, 68.15) --
	(175.52, 68.15) --
	(175.59, 68.34) --
	(175.59, 68.34) --
	(175.59, 68.34) --
	(175.59, 68.35) --
	(175.59, 68.35) --
	(175.59, 68.35) --
	(175.67, 68.50) --
	(175.67, 68.50) --
	(175.67, 68.50) --
	(175.68, 68.55) --
	(175.68, 68.55) --
	(175.68, 68.55) --
	(175.74, 68.71) --
	(175.74, 68.71) --
	(175.74, 68.71) --
	(175.78, 68.70) --
	(175.78, 68.70) --
	(175.78, 68.70) --
	(175.82, 68.68) --
	(175.82, 68.68) --
	(175.82, 68.68) --
	(175.89, 68.50) --
	(175.89, 68.50) --
	(175.89, 68.50) --
	(175.96, 68.41) --
	(175.96, 68.41) --
	(175.96, 68.41) --
	(175.97, 68.43) --
	(175.97, 68.43) --
	(175.97, 68.43) --
	(176.04, 68.52) --
	(176.04, 68.52) --
	(176.04, 68.52) --
	(176.07, 68.55) --
	(176.07, 68.55) --
	(176.07, 68.55) --
	(176.11, 68.60) --
	(176.11, 68.60) --
	(176.11, 68.60) --
	(176.16, 68.62) --
	(176.16, 68.62) --
	(176.16, 68.62) --
	(176.19, 68.63) --
	(176.19, 68.63) --
	(176.19, 68.63) --
	(176.26, 68.68) --
	(176.26, 68.68) --
	(176.26, 68.68) --
	(176.33, 68.50) --
	(176.33, 68.50) --
	(176.33, 68.50) --
	(176.36, 68.51) --
	(176.36, 68.51) --
	(176.36, 68.51) --
	(176.41, 68.55) --
	(176.41, 68.55) --
	(176.41, 68.55) --
	(176.45, 68.56) --
	(176.45, 68.56) --
	(176.45, 68.56) --
	(176.48, 68.56) --
	(176.48, 68.56) --
	(176.48, 68.56) --
	(176.56, 68.66) --
	(176.56, 68.66) --
	(176.56, 68.66) --
	(176.63, 68.74) --
	(176.63, 68.74) --
	(176.63, 68.74) --
	(176.64, 68.77) --
	(176.64, 68.77) --
	(176.64, 68.77) --
	(176.70, 68.90) --
	(176.70, 68.90) --
	(176.70, 68.90) --
	(176.74, 68.89) --
	(176.74, 68.89) --
	(176.74, 68.89) --
	(176.78, 68.88) --
	(176.78, 68.88) --
	(176.78, 68.88) --
	(176.83, 68.93) --
	(176.83, 68.93) --
	(176.83, 68.93) --
	(176.85, 68.95) --
	(176.85, 68.95) --
	(176.85, 68.95) --
	(176.93, 69.07) --
	(176.93, 69.07) --
	(176.93, 69.07) --
	(177.00, 69.24) --
	(177.00, 69.24) --
	(177.00, 69.24) --
	(177.03, 69.25) --
	(177.03, 69.25) --
	(177.03, 69.25) --
	(177.07, 69.28) --
	(177.07, 69.28) --
	(177.07, 69.28) --
	(177.12, 69.20) --
	(177.12, 69.20) --
	(177.12, 69.20) --
	(177.15, 69.16) --
	(177.15, 69.16) --
	(177.15, 69.16) --
	(177.22, 69.20) --
	(177.22, 69.20) --
	(177.22, 69.20) --
	(177.30, 69.24) --
	(177.30, 69.24) --
	(177.30, 69.24) --
	(177.31, 69.27) --
	(177.31, 69.27) --
	(177.31, 69.27) --
	(177.37, 69.38) --
	(177.37, 69.38) --
	(177.37, 69.38) --
	(177.44, 69.59) --
	(177.44, 69.59) --
	(177.44, 69.59) --
	(177.51, 69.53) --
	(177.51, 69.53) --
	(177.51, 69.53) --
	(177.52, 69.51) --
	(177.52, 69.51) --
	(177.52, 69.51) --
	(177.59, 69.68) --
	(177.59, 69.68) --
	(177.59, 69.68) --
	(177.60, 69.70) --
	(177.60, 69.70) --
	(177.60, 69.70) --
	(177.67, 69.86) --
	(177.67, 69.86) --
	(177.67, 69.86) --
	(177.74, 70.15) --
	(177.74, 70.15) --
	(177.74, 70.15) --
	(177.79, 70.22) --
	(177.79, 70.22) --
	(177.79, 70.22) --
	(177.81, 70.24) --
	(177.81, 70.24) --
	(177.81, 70.24) --
	(177.89, 70.27) --
	(177.89, 70.27) --
	(177.89, 70.27) --
	(177.89, 70.27) --
	(177.89, 70.27) --
	(177.89, 70.27) --
	(177.96, 70.19) --
	(177.96, 70.19) --
	(177.96, 70.19) --
	(178.04, 69.94) --
	(178.04, 69.94) --
	(178.04, 69.94) --
	(178.11, 70.13) --
	(178.11, 70.13) --
	(178.11, 70.13) --
	(178.18, 70.14) --
	(178.18, 70.14) --
	(178.18, 70.14) --
	(178.18, 70.15) --
	(178.18, 70.15) --
	(178.18, 70.15) --
	(178.26, 70.31) --
	(178.26, 70.31) --
	(178.26, 70.31) --
	(178.33, 70.54) --
	(178.33, 70.54) --
	(178.33, 70.54) --
	(178.37, 70.76) --
	(178.37, 70.76) --
	(178.37, 70.76) --
	(178.41, 71.00) --
	(178.41, 71.00) --
	(178.41, 71.00) --
	(178.48, 71.66) --
	(178.48, 71.66) --
	(178.48, 71.66) --
	(178.55, 72.00) --
	(178.55, 72.00) --
	(178.55, 72.00) --
	(178.56, 71.99) --
	(178.56, 71.99) --
	(178.56, 71.99) --
	(178.63, 71.85) --
	(178.63, 71.85) --
	(178.63, 71.85) --
	(178.70, 71.60) --
	(178.70, 71.60) --
	(178.70, 71.60) --
	(178.75, 71.57) --
	(178.75, 71.57) --
	(178.75, 71.57) --
	(178.78, 71.56) --
	(178.78, 71.56) --
	(178.78, 71.56) --
	(178.85, 71.30) --
	(178.85, 71.30) --
	(178.85, 71.30) --
	(178.92, 70.99) --
	(178.92, 70.99) --
	(178.92, 70.99) --
	(178.94, 70.98) --
	(178.94, 70.98) --
	(178.94, 70.98) --
	(179.00, 70.95) --
	(179.00, 70.95) --
	(179.00, 70.95) --
	(179.07, 70.99) --
	(179.07, 70.99) --
	(179.07, 70.99) --
	(179.14, 70.81) --
	(179.14, 70.81) --
	(179.14, 70.81) --
	(179.22, 70.85) --
	(179.22, 70.85) --
	(179.22, 70.85) --
	(179.29, 70.77) --
	(179.29, 70.77) --
	(179.29, 70.77) --
	(179.33, 70.92) --
	(179.33, 70.92) --
	(179.33, 70.92) --
	(179.37, 71.11) --
	(179.37, 71.11) --
	(179.37, 71.11) --
	(179.44, 71.43) --
	(179.44, 71.43) --
	(179.44, 71.43) --
	(179.51, 71.76) --
	(179.51, 71.76) --
	(179.51, 71.76) --
	(179.59, 71.51) --
	(179.59, 71.51) --
	(179.59, 71.51) --
	(179.66, 71.11) --
	(179.66, 71.11) --
	(179.66, 71.11) --
	(179.71, 70.94) --
	(179.71, 70.94) --
	(179.71, 70.94) --
	(179.73, 70.84) --
	(179.73, 70.84) --
	(179.73, 70.84) --
	(179.81, 70.74) --
	(179.81, 70.74) --
	(179.81, 70.74) --
	(179.81, 70.73) --
	(179.81, 70.73) --
	(179.81, 70.73) --
	(179.88, 70.79) --
	(179.88, 70.79) --
	(179.88, 70.79) --
	(179.96, 70.68) --
	(179.96, 70.68) --
	(179.96, 70.68) --
	(180.03, 70.70) --
	(180.03, 70.70) --
	(180.03, 70.70) --
	(180.10, 70.75) --
	(180.10, 70.75) --
	(180.10, 70.75) --
	(180.18, 70.75) --
	(180.18, 70.75) --
	(180.18, 70.75) --
	(180.19, 70.78) --
	(180.19, 70.78) --
	(180.19, 70.78) --
	(180.25, 70.99) --
	(180.25, 70.99) --
	(180.25, 70.99) --
	(180.33, 71.03) --
	(180.33, 71.03) --
	(180.33, 71.03) --
	(180.40, 71.16) --
	(180.40, 71.16) --
	(180.40, 71.16) --
	(180.47, 71.22) --
	(180.47, 71.22) --
	(180.47, 71.22) --
	(180.55, 71.48) --
	(180.55, 71.48) --
	(180.55, 71.48) --
	(180.57, 71.56) --
	(180.57, 71.56) --
	(180.57, 71.56) --
	(180.62, 71.71) --
	(180.62, 71.71) --
	(180.62, 71.71) --
	(180.70, 72.11) --
	(180.70, 72.11) --
	(180.70, 72.11) --
	(180.77, 72.27) --
	(180.77, 72.27) --
	(180.77, 72.27) --
	(180.84, 72.20) --
	(180.84, 72.20) --
	(180.84, 72.20) --
	(180.92, 72.13) --
	(180.92, 72.13) --
	(180.92, 72.13) --
	(180.99, 72.14) --
	(180.99, 72.14) --
	(180.99, 72.14) --
	(181.05, 72.26) --
	(181.05, 72.26) --
	(181.05, 72.26) --
	(181.06, 72.29) --
	(181.06, 72.29) --
	(181.06, 72.29) --
	(181.14, 72.20) --
	(181.14, 72.20) --
	(181.14, 72.20) --
	(181.21, 72.14) --
	(181.21, 72.14) --
	(181.21, 72.14) --
	(181.28, 72.07) --
	(181.28, 72.07) --
	(181.28, 72.07) --
	(181.36, 71.95) --
	(181.36, 71.95) --
	(181.36, 71.95) --
	(181.43, 71.95) --
	(181.43, 71.95) --
	(181.43, 71.95) --
	(181.51, 71.83) --
	(181.51, 71.83) --
	(181.51, 71.83) --
	(181.53, 71.82) --
	(181.53, 71.82) --
	(181.53, 71.82) --
	(181.58, 71.79) --
	(181.58, 71.79) --
	(181.58, 71.79) --
	(181.65, 71.79) --
	(181.65, 71.79) --
	(181.65, 71.79) --
	(181.73, 71.75) --
	(181.73, 71.75) --
	(181.73, 71.75) --
	(181.80, 71.77) --
	(181.80, 71.77) --
	(181.80, 71.77) --
	(181.87, 71.82) --
	(181.87, 71.82) --
	(181.87, 71.82) --
	(181.95, 71.91) --
	(181.95, 71.91) --
	(181.95, 71.91) --
	(182.01, 71.95) --
	(182.01, 71.95) --
	(182.01, 71.95) --
	(182.02, 71.95) --
	(182.02, 71.95) --
	(182.02, 71.95) --
	(182.10, 71.98) --
	(182.10, 71.98) --
	(182.10, 71.98) --
	(182.17, 72.11) --
	(182.17, 72.11) --
	(182.17, 72.11) --
	(182.24, 72.21) --
	(182.24, 72.21) --
	(182.24, 72.21) --
	(182.32, 72.37) --
	(182.32, 72.37) --
	(182.32, 72.37) --
	(182.39, 72.63) --
	(182.39, 72.63) --
	(182.39, 72.63) --
	(182.46, 73.08) --
	(182.46, 73.08) --
	(182.46, 73.08) --
	(182.54, 73.22) --
	(182.54, 73.22) --
	(182.54, 73.22) --
	(182.58, 73.18) --
	(182.58, 73.18) --
	(182.58, 73.18) --
	(182.61, 73.15) --
	(182.61, 73.15) --
	(182.61, 73.15) --
	(182.68, 73.04) --
	(182.69, 73.04) --
	(182.69, 73.04) --
	(182.76, 72.78) --
	(182.76, 72.78) --
	(182.76, 72.78) --
	(182.83, 72.59) --
	(182.83, 72.59) --
	(182.83, 72.59) --
	(182.91, 72.43) --
	(182.91, 72.43) --
	(182.91, 72.43) --
	(182.98, 72.32) --
	(182.98, 72.32) --
	(182.98, 72.32) --
	(183.05, 72.30) --
	(183.05, 72.30) --
	(183.05, 72.30) --
	(183.13, 72.18) --
	(183.13, 72.18) --
	(183.13, 72.18) --
	(183.20, 72.21) --
	(183.20, 72.21) --
	(183.20, 72.21) --
	(183.27, 72.14) --
	(183.27, 72.14) --
	(183.27, 72.14) --
	(183.35, 72.12) --
	(183.35, 72.12) --
	(183.35, 72.12) --
	(183.35, 72.12) --
	(183.35, 72.12) --
	(183.35, 72.12) --
	(183.42, 72.11) --
	(183.42, 72.11) --
	(183.42, 72.11) --
	(183.50, 72.10) --
	(183.50, 72.10) --
	(183.50, 72.10) --
	(183.57, 72.07) --
	(183.57, 72.07) --
	(183.57, 72.07) --
	(183.64, 72.11) --
	(183.64, 72.11) --
	(183.64, 72.11) --
	(183.72, 72.06) --
	(183.72, 72.06) --
	(183.72, 72.06) --
	(183.79, 72.10) --
	(183.79, 72.10) --
	(183.79, 72.10) --
	(183.86, 72.07) --
	(183.86, 72.07) --
	(183.86, 72.07) --
	(183.94, 72.07) --
	(183.94, 72.07) --
	(183.94, 72.07) --
	(184.01, 72.08) --
	(184.01, 72.08) --
	(184.01, 72.08) --
	(184.09, 72.06) --
	(184.09, 72.06) --
	(184.09, 72.06) --
	(184.12, 72.07) --
	(184.12, 72.07) --
	(184.12, 72.07) --
	(184.16, 72.09) --
	(184.16, 72.09) --
	(184.16, 72.09) --
	(184.23, 72.13) --
	(184.23, 72.13) --
	(184.23, 72.13) --
	(184.30, 72.14) --
	(184.30, 72.14) --
	(184.30, 72.14) --
	(184.38, 72.16) --
	(184.38, 72.16) --
	(184.38, 72.16) --
	(184.41, 72.13) --
	(184.41, 72.13) --
	(184.41, 72.13) --
	(184.45, 72.08) --
	(184.45, 72.08) --
	(184.45, 72.08) --
	(184.53, 72.08) --
	(184.53, 72.08) --
	(184.53, 72.08) --
	(184.60, 72.06) --
	(184.60, 72.06) --
	(184.60, 72.06) --
	(184.67, 72.04) --
	(184.67, 72.04) --
	(184.67, 72.04) --
	(184.75, 72.05) --
	(184.75, 72.05) --
	(184.75, 72.05) --
	(184.82, 72.05) --
	(184.82, 72.05) --
	(184.82, 72.05) --
	(184.89, 72.06) --
	(184.89, 72.06) --
	(184.89, 72.06) --
	(184.97, 72.01) --
	(184.97, 72.01) --
	(184.97, 72.01) --
	(184.98, 72.01) --
	(184.98, 72.01) --
	(184.98, 72.01) --
	(185.04, 72.03) --
	(185.04, 72.03) --
	(185.04, 72.03) --
	(185.11, 72.09) --
	(185.11, 72.09) --
	(185.11, 72.09) --
	(185.19, 72.05) --
	(185.19, 72.05) --
	(185.19, 72.05) --
	(185.26, 72.05) --
	(185.26, 72.05) --
	(185.26, 72.05) --
	(185.27, 72.05) --
	(185.27, 72.05) --
	(185.27, 72.05) --
	(185.33, 72.08) --
	(185.33, 72.08) --
	(185.33, 72.08) --
	(185.41, 72.09) --
	(185.41, 72.09) --
	(185.41, 72.09) --
	(185.48, 72.08) --
	(185.48, 72.08) --
	(185.48, 72.08) --
	(185.55, 72.06) --
	(185.55, 72.06) --
	(185.55, 72.06) --
	(185.63, 72.10) --
	(185.63, 72.10) --
	(185.63, 72.10) --
	(185.70, 72.17) --
	(185.70, 72.17) --
	(185.70, 72.17) --
	(185.75, 72.18) --
	(185.75, 72.18) --
	(185.75, 72.18) --
	(185.78, 72.19) --
	(185.78, 72.19) --
	(185.78, 72.19) --
	(185.85, 72.20) --
	(185.85, 72.20) --
	(185.85, 72.20) --
	(185.92, 72.25) --
	(185.92, 72.25) --
	(185.92, 72.25) --
	(186.00, 72.27) --
	(186.00, 72.27) --
	(186.00, 72.27) --
	(186.03, 72.25) --
	(186.03, 72.25) --
	(186.03, 72.25) --
	(186.07, 72.23) --
	(186.07, 72.23) --
	(186.07, 72.23) --
	(186.14, 72.25) --
	(186.14, 72.25) --
	(186.14, 72.25) --
	(186.22, 72.28) --
	(186.22, 72.28) --
	(186.22, 72.28) --
	(186.29, 72.30) --
	(186.29, 72.30) --
	(186.29, 72.30) --
	(186.36, 72.32) --
	(186.36, 72.32) --
	(186.36, 72.32) --
	(186.44, 72.32) --
	(186.44, 72.32) --
	(186.44, 72.32) --
	(186.51, 72.34) --
	(186.51, 72.34) --
	(186.51, 72.34) --
	(186.58, 72.35) --
	(186.58, 72.35) --
	(186.58, 72.35) --
	(186.66, 72.39) --
	(186.66, 72.39) --
	(186.66, 72.39) --
	(186.71, 72.41) --
	(186.71, 72.41) --
	(186.71, 72.41) --
	(186.73, 72.42) --
	(186.73, 72.42) --
	(186.73, 72.42) --
	(186.80, 72.49) --
	(186.80, 72.49) --
	(186.80, 72.49) --
	(186.88, 72.48) --
	(186.88, 72.48) --
	(186.88, 72.48) --
	(186.95, 72.46) --
	(186.95, 72.46) --
	(186.95, 72.46) --
	(187.02, 72.49) --
	(187.02, 72.49) --
	(187.02, 72.49) --
	(187.10, 72.57) --
	(187.10, 72.57) --
	(187.10, 72.57) --
	(187.17, 72.55) --
	(187.17, 72.55) --
	(187.17, 72.55) --
	(187.18, 72.56) --
	(187.18, 72.56) --
	(187.18, 72.56) --
	(187.25, 72.58) --
	(187.25, 72.58) --
	(187.25, 72.58) --
	(187.32, 72.58) --
	(187.32, 72.58) --
	(187.32, 72.58) --
	(187.39, 72.62) --
	(187.39, 72.62) --
	(187.39, 72.62) --
	(187.46, 72.64) --
	(187.46, 72.64) --
	(187.46, 72.64) --
	(187.54, 72.73) --
	(187.54, 72.73) --
	(187.54, 72.73) --
	(187.61, 72.75) --
	(187.61, 72.75) --
	(187.61, 72.75) --
	(187.68, 72.77) --
	(187.68, 72.77) --
	(187.68, 72.77) --
	(187.76, 72.76) --
	(187.76, 72.76) --
	(187.76, 72.76) --
	(187.76, 72.76) --
	(187.76, 72.76) --
	(187.76, 72.76) --
	(187.83, 72.79) --
	(187.83, 72.79) --
	(187.83, 72.79) --
	(187.91, 72.78) --
	(187.91, 72.78) --
	(187.91, 72.78) --
	(187.98, 72.86) --
	(187.98, 72.86) --
	(187.98, 72.86) --
	(188.05, 72.88) --
	(188.05, 72.88) --
	(188.05, 72.88) --
	(188.13, 72.91) --
	(188.13, 72.91) --
	(188.13, 72.91) --
	(188.20, 72.93) --
	(188.20, 72.93) --
	(188.20, 72.93) --
	(188.27, 72.96) --
	(188.27, 72.96) --
	(188.27, 72.96) --
	(188.33, 73.02) --
	(188.33, 73.02) --
	(188.33, 73.02) --
	(188.34, 73.02) --
	(188.34, 73.02) --
	(188.34, 73.02) --
	(188.42, 73.02) --
	(188.42, 73.02) --
	(188.42, 73.02) --
	(188.49, 73.09) --
	(188.49, 73.09) --
	(188.49, 73.09) --
	(188.56, 73.14) --
	(188.56, 73.14) --
	(188.56, 73.14) --
	(188.64, 73.13) --
	(188.64, 73.13) --
	(188.64, 73.13) --
	(188.71, 73.12) --
	(188.71, 73.12) --
	(188.71, 73.12) --
	(188.79, 73.15) --
	(188.79, 73.15) --
	(188.79, 73.15) --
	(188.86, 73.15) --
	(188.86, 73.15) --
	(188.86, 73.15) --
	(188.93, 73.16) --
	(188.93, 73.16) --
	(188.93, 73.16) --
	(189.01, 73.19) --
	(189.01, 73.19) --
	(189.01, 73.19) --
	(189.08, 73.21) --
	(189.08, 73.21) --
	(189.08, 73.21) --
	(189.15, 73.29) --
	(189.15, 73.29) --
	(189.15, 73.29) --
	(189.22, 73.25) --
	(189.22, 73.25) --
	(189.22, 73.25) --
	(189.29, 73.35) --
	(189.29, 73.35) --
	(189.29, 73.35) --
	(189.30, 73.36) --
	(189.30, 73.36) --
	(189.30, 73.36) --
	(189.37, 73.37) --
	(189.37, 73.37) --
	(189.37, 73.37) --
	(189.44, 73.35) --
	(189.44, 73.35) --
	(189.44, 73.35) --
	(189.52, 73.39) --
	(189.52, 73.39) --
	(189.52, 73.39) --
	(189.59, 73.38) --
	(189.59, 73.38) --
	(189.59, 73.38) --
	(189.66, 73.45) --
	(189.66, 73.45) --
	(189.66, 73.45) --
	(189.74, 73.41) --
	(189.74, 73.41) --
	(189.74, 73.41) --
	(189.81, 73.45) --
	(189.81, 73.45) --
	(189.81, 73.45) --
	(189.88, 73.48) --
	(189.88, 73.48) --
	(189.88, 73.48) --
	(189.96, 73.46) --
	(189.96, 73.46) --
	(189.96, 73.46) --
	(190.03, 73.47) --
	(190.03, 73.47) --
	(190.03, 73.47) --
	(190.10, 73.49) --
	(190.10, 73.49) --
	(190.10, 73.49) --
	(190.18, 73.49) --
	(190.18, 73.49) --
	(190.18, 73.49) --
	(190.25, 73.48) --
	(190.25, 73.48) --
	(190.25, 73.48) --
	(190.25, 73.48) --
	(190.25, 73.48) --
	(190.25, 73.48) --
	(190.32, 73.56) --
	(190.32, 73.56) --
	(190.32, 73.56) --
	(190.40, 73.51) --
	(190.40, 73.51) --
	(190.40, 73.51) --
	(190.47, 73.57) --
	(190.47, 73.57) --
	(190.47, 73.57) --
	(190.54, 73.53) --
	(190.54, 73.53) --
	(190.54, 73.53) --
	(190.62, 73.59) --
	(190.62, 73.59) --
	(190.62, 73.59) --
	(190.69, 73.53) --
	(190.69, 73.53) --
	(190.69, 73.53) --
	(190.76, 73.54) --
	(190.76, 73.54) --
	(190.76, 73.54) --
	(190.84, 73.54) --
	(190.84, 73.54) --
	(190.84, 73.54) --
	(190.91, 73.52) --
	(190.91, 73.52) --
	(190.91, 73.52) --
	(190.98, 73.51) --
	(190.98, 73.51) --
	(190.98, 73.51) --
	(191.06, 73.52) --
	(191.06, 73.52) --
	(191.06, 73.52) --
	(191.13, 73.46) --
	(191.13, 73.46) --
	(191.13, 73.46) --
	(191.20, 73.50) --
	(191.20, 73.50) --
	(191.20, 73.50) --
	(191.27, 73.48) --
	(191.27, 73.48) --
	(191.27, 73.48) --
	(191.35, 73.45) --
	(191.35, 73.45) --
	(191.35, 73.45) --
	(191.42, 73.49) --
	(191.42, 73.49) --
	(191.42, 73.49) --
	(191.49, 73.47) --
	(191.49, 73.47) --
	(191.49, 73.47) --
	(191.57, 73.46) --
	(191.57, 73.46) --
	(191.57, 73.46) --
	(191.64, 73.41) --
	(191.64, 73.41) --
	(191.64, 73.41) --
	(191.71, 73.43) --
	(191.71, 73.43) --
	(191.71, 73.43) --
	(191.79, 73.38) --
	(191.79, 73.38) --
	(191.79, 73.38) --
	(191.86, 73.44) --
	(191.86, 73.44) --
	(191.86, 73.44) --
	(191.93, 73.42) --
	(191.93, 73.42) --
	(191.93, 73.42) --
	(192.01, 73.38) --
	(192.01, 73.38) --
	(192.01, 73.38) --
	(192.07, 73.38) --
	(192.07, 73.38) --
	(192.07, 73.38) --
	(192.08, 73.38) --
	(192.08, 73.38) --
	(192.08, 73.38) --
	(192.15, 73.33) --
	(192.15, 73.33) --
	(192.15, 73.33) --
	(192.23, 73.33) --
	(192.23, 73.33) --
	(192.23, 73.33) --
	(192.30, 73.35) --
	(192.30, 73.35) --
	(192.30, 73.35) --
	(192.37, 73.32) --
	(192.37, 73.32) --
	(192.37, 73.32) --
	(192.44, 73.31) --
	(192.44, 73.31) --
	(192.44, 73.31) --
	(192.52, 73.27) --
	(192.52, 73.27) --
	(192.52, 73.27) --
	(192.59, 73.21) --
	(192.59, 73.21) --
	(192.59, 73.21) --
	(192.66, 73.25) --
	(192.66, 73.25) --
	(192.66, 73.25) --
	(192.74, 73.22) --
	(192.74, 73.22) --
	(192.74, 73.22) --
	(192.81, 73.19) --
	(192.81, 73.19) --
	(192.81, 73.19) --
	(192.88, 73.15) --
	(192.88, 73.15) --
	(192.88, 73.15) --
	(192.96, 73.12) --
	(192.96, 73.12) --
	(192.96, 73.12) --
	(193.03, 73.09) --
	(193.03, 73.09) --
	(193.03, 73.09) --
	(193.10, 73.08) --
	(193.10, 73.08) --
	(193.10, 73.08) --
	(193.17, 73.08) --
	(193.17, 73.08) --
	(193.17, 73.08) --
	(193.25, 73.04) --
	(193.25, 73.04) --
	(193.25, 73.04) --
	(193.32, 73.03) --
	(193.32, 73.03) --
	(193.32, 73.03) --
	(193.39, 73.01) --
	(193.39, 73.01) --
	(193.39, 73.01) --
	(193.41, 72.99) --
	(193.41, 72.99) --
	(193.41, 72.99) --
	(193.47, 72.94) --
	(193.47, 72.94) --
	(193.47, 72.94) --
	(193.54, 72.89) --
	(193.54, 72.89) --
	(193.54, 72.89) --
	(193.61, 72.87) --
	(193.61, 72.87) --
	(193.61, 72.87) --
	(193.69, 72.92) --
	(193.69, 72.92) --
	(193.69, 72.92) --
	(193.76, 72.88) --
	(193.76, 72.88) --
	(193.76, 72.88) --
	(193.83, 72.86) --
	(193.83, 72.86) --
	(193.83, 72.86) --
	(193.91, 72.78) --
	(193.91, 72.78) --
	(193.91, 72.78) --
	(193.98, 72.78) --
	(193.98, 72.78) --
	(193.98, 72.78) --
	(194.05, 72.73) --
	(194.05, 72.73) --
	(194.05, 72.73) --
	(194.12, 72.70) --
	(194.12, 72.70) --
	(194.12, 72.70) --
	(194.20, 72.63) --
	(194.20, 72.63) --
	(194.20, 72.63) --
	(194.27, 72.70) --
	(194.27, 72.70) --
	(194.27, 72.70) --
	(194.34, 72.59) --
	(194.34, 72.59) --
	(194.34, 72.59) --
	(194.42, 72.61) --
	(194.42, 72.61) --
	(194.42, 72.61) --
	(194.49, 72.57) --
	(194.49, 72.57) --
	(194.49, 72.57) --
	(194.56, 72.57) --
	(194.56, 72.57) --
	(194.56, 72.57) --
	(194.63, 72.55) --
	(194.63, 72.55) --
	(194.63, 72.55) --
	(194.71, 72.48) --
	(194.71, 72.48) --
	(194.71, 72.48) --
	(194.78, 72.47) --
	(194.78, 72.47) --
	(194.78, 72.47) --
	(194.85, 72.45) --
	(194.85, 72.45) --
	(194.85, 72.45) --
	(194.93, 72.47) --
	(194.93, 72.47) --
	(194.93, 72.47) --
	(195.00, 72.42) --
	(195.00, 72.42) --
	(195.00, 72.42) --
	(195.07, 72.39) --
	(195.07, 72.39) --
	(195.07, 72.39) --
	(195.15, 72.39) --
	(195.15, 72.39) --
	(195.15, 72.39) --
	(195.22, 72.37) --
	(195.22, 72.37) --
	(195.22, 72.37) --
	(195.23, 72.37) --
	(195.23, 72.37) --
	(195.23, 72.37) --
	(195.29, 72.33) --
	(195.29, 72.33) --
	(195.29, 72.33) --
	(195.36, 72.30) --
	(195.36, 72.30) --
	(195.36, 72.30) --
	(195.44, 72.25) --
	(195.44, 72.25) --
	(195.44, 72.25) --
	(195.51, 72.23) --
	(195.51, 72.23) --
	(195.51, 72.23) --
	(195.58, 72.20) --
	(195.58, 72.20) --
	(195.58, 72.20) --
	(195.66, 72.20) --
	(195.66, 72.20) --
	(195.66, 72.20) --
	(195.73, 72.16) --
	(195.73, 72.16) --
	(195.73, 72.16) --
	(195.80, 72.18) --
	(195.80, 72.18) --
	(195.80, 72.18) --
	(195.87, 72.11) --
	(195.87, 72.11) --
	(195.87, 72.11) --
	(195.95, 72.08) --
	(195.95, 72.08) --
	(195.95, 72.08) --
	(196.02, 72.04) --
	(196.02, 72.04) --
	(196.02, 72.04) --
	(196.09, 72.03) --
	(196.09, 72.03) --
	(196.09, 72.03) --
	(196.17, 72.04) --
	(196.17, 72.04) --
	(196.17, 72.04) --
	(196.24, 71.98) --
	(196.24, 71.98) --
	(196.24, 71.98) --
	(196.31, 71.98) --
	(196.31, 71.98) --
	(196.31, 71.98) --
	(196.38, 71.93) --
	(196.38, 71.93) --
	(196.38, 71.93) --
	(196.46, 71.88) --
	(196.46, 71.88) --
	(196.46, 71.88) --
	(196.53, 71.86) --
	(196.53, 71.86) --
	(196.53, 71.86) --
	(196.60, 71.85) --
	(196.60, 71.85) --
	(196.60, 71.85) --
	(196.68, 71.82) --
	(196.68, 71.82) --
	(196.68, 71.82) --
	(196.75, 71.78) --
	(196.75, 71.78) --
	(196.75, 71.78) --
	(196.82, 71.80) --
	(196.82, 71.80) --
	(196.82, 71.80) --
	(196.89, 71.74) --
	(196.89, 71.74) --
	(196.89, 71.74) --
	(196.97, 71.72) --
	(196.97, 71.72) --
	(196.97, 71.72) --
	(197.04, 71.71) --
	(197.04, 71.71) --
	(197.04, 71.71) --
	(197.06, 71.71) --
	(197.06, 71.71) --
	(197.06, 71.71) --
	(197.11, 71.72) --
	(197.11, 71.72) --
	(197.11, 71.72) --
	(197.19, 71.70) --
	(197.19, 71.70) --
	(197.19, 71.70) --
	(197.26, 71.68) --
	(197.26, 71.68) --
	(197.26, 71.68) --
	(197.33, 71.67) --
	(197.33, 71.67) --
	(197.33, 71.67) --
	(197.40, 71.67) --
	(197.40, 71.67) --
	(197.40, 71.67) --
	(197.48, 71.64) --
	(197.48, 71.64) --
	(197.48, 71.64) --
	(197.55, 71.61) --
	(197.55, 71.61) --
	(197.55, 71.61) --
	(197.62, 71.61) --
	(197.62, 71.61) --
	(197.62, 71.61) --
	(197.69, 71.54) --
	(197.69, 71.54) --
	(197.69, 71.54) --
	(197.77, 71.54) --
	(197.77, 71.54) --
	(197.77, 71.54) --
	(197.84, 71.52) --
	(197.84, 71.52) --
	(197.84, 71.52) --
	(197.91, 71.53) --
	(197.91, 71.53) --
	(197.91, 71.53) --
	(197.99, 71.50) --
	(197.99, 71.50) --
	(197.99, 71.50) --
	(198.06, 71.47) --
	(198.06, 71.47) --
	(198.06, 71.47) --
	(198.11, 71.45) --
	(198.11, 71.45) --
	(198.11, 71.45) --
	(198.13, 71.44) --
	(198.13, 71.44) --
	(198.13, 71.44) --
	(198.20, 71.42) --
	(198.20, 71.42) --
	(198.20, 71.42) --
	(198.28, 71.42) --
	(198.28, 71.42) --
	(198.28, 71.42) --
	(198.35, 71.37) --
	(198.35, 71.37) --
	(198.35, 71.37) --
	(198.42, 71.36) --
	(198.42, 71.36) --
	(198.42, 71.36) --
	(198.50, 71.30) --
	(198.50, 71.30) --
	(198.50, 71.30) --
	(198.57, 71.29) --
	(198.57, 71.29) --
	(198.57, 71.29) --
	(198.64, 71.29) --
	(198.64, 71.29) --
	(198.64, 71.29) --
	(198.71, 71.21) --
	(198.71, 71.21) --
	(198.71, 71.21) --
	(198.79, 71.19) --
	(198.79, 71.19) --
	(198.79, 71.19) --
	(198.86, 71.21) --
	(198.86, 71.21) --
	(198.86, 71.21) --
	(198.93, 71.17) --
	(198.93, 71.17) --
	(198.93, 71.17) --
	(199.00, 71.16) --
	(199.00, 71.16) --
	(199.00, 71.16) --
	(199.08, 71.15) --
	(199.08, 71.15) --
	(199.08, 71.15) --
	(199.15, 71.14) --
	(199.15, 71.14) --
	(199.15, 71.14) --
	(199.22, 71.08) --
	(199.22, 71.08) --
	(199.22, 71.08) --
	(199.26, 71.09) --
	(199.26, 71.09) --
	(199.26, 71.09) --
	(199.29, 71.09) --
	(199.29, 71.09) --
	(199.29, 71.09) --
	(199.37, 71.06) --
	(199.37, 71.06) --
	(199.37, 71.06) --
	(199.44, 71.00) --
	(199.44, 71.00) --
	(199.44, 71.00) --
	(199.51, 71.02) --
	(199.51, 71.02) --
	(199.51, 71.02) --
	(199.59, 71.03) --
	(199.59, 71.03) --
	(199.59, 71.03) --
	(199.66, 71.03) --
	(199.66, 71.03) --
	(199.66, 71.03) --
	(199.73, 71.02) --
	(199.73, 71.02) --
	(199.73, 71.02) --
	(199.80, 70.98) --
	(199.80, 70.98) --
	(199.80, 70.98) --
	(199.88, 70.99) --
	(199.88, 70.99) --
	(199.88, 70.99) --
	(199.95, 70.95) --
	(199.95, 70.95) --
	(199.95, 70.95) --
	(200.02, 70.97) --
	(200.02, 70.97) --
	(200.02, 70.97) --
	(200.03, 70.97) --
	(200.03, 70.97) --
	(200.03, 70.97) --
	(200.09, 70.99) --
	(200.09, 70.99) --
	(200.09, 70.99) --
	(200.17, 70.96) --
	(200.17, 70.96) --
	(200.17, 70.96) --
	(200.24, 71.00) --
	(200.24, 71.00) --
	(200.24, 71.00) --
	(200.31, 70.99) --
	(200.31, 70.99) --
	(200.31, 70.99) --
	(200.38, 70.99) --
	(200.38, 70.99) --
	(200.38, 70.99) --
	(200.46, 71.00) --
	(200.46, 71.00) --
	(200.46, 71.00) --
	(200.53, 71.02) --
	(200.53, 71.02) --
	(200.53, 71.02) --
	(200.60, 70.98) --
	(200.60, 70.98) --
	(200.60, 70.98) --
	(200.67, 71.00) --
	(200.67, 71.00) --
	(200.67, 71.00) --
	(200.75, 71.00) --
	(200.75, 71.00) --
	(200.75, 71.00) --
	(200.79, 71.01) --
	(200.79, 71.01) --
	(200.79, 71.01) --
	(200.82, 71.01) --
	(200.82, 71.01) --
	(200.82, 71.01) --
	(200.89, 71.00) --
	(200.89, 71.00) --
	(200.89, 71.00) --
	(200.96, 71.02) --
	(200.96, 71.02) --
	(200.96, 71.02) --
	(201.04, 71.04) --
	(201.04, 71.04) --
	(201.04, 71.04) --
	(201.11, 71.02) --
	(201.11, 71.02) --
	(201.11, 71.02) --
	(201.18, 71.02) --
	(201.18, 71.02) --
	(201.18, 71.02) --
	(201.26, 71.05) --
	(201.26, 71.05) --
	(201.26, 71.05) --
	(201.33, 71.06) --
	(201.33, 71.06) --
	(201.33, 71.06) --
	(201.40, 71.06) --
	(201.40, 71.06) --
	(201.40, 71.06) --
	(201.47, 71.08) --
	(201.47, 71.08) --
	(201.47, 71.08) --
	(201.54, 71.08) --
	(201.54, 71.08) --
	(201.54, 71.08) --
	(201.62, 71.11) --
	(201.62, 71.11) --
	(201.62, 71.11) --
	(201.69, 71.09) --
	(201.69, 71.09) --
	(201.69, 71.09) --
	(201.76, 71.15) --
	(201.76, 71.15) --
	(201.76, 71.15) --
	(201.83, 71.15) --
	(201.83, 71.15) --
	(201.83, 71.15) --
	(201.91, 71.17) --
	(201.91, 71.17) --
	(201.91, 71.17) --
	(201.98, 71.20) --
	(201.98, 71.20) --
	(201.98, 71.20) --
	(202.04, 71.21) --
	(202.04, 71.21) --
	(202.04, 71.21) --
	(202.05, 71.22) --
	(202.05, 71.22) --
	(202.05, 71.22) --
	(202.12, 71.24) --
	(202.12, 71.24) --
	(202.12, 71.24) --
	(202.20, 71.25) --
	(202.20, 71.25) --
	(202.20, 71.25) --
	(202.27, 71.29) --
	(202.27, 71.29) --
	(202.27, 71.29) --
	(202.34, 71.31) --
	(202.34, 71.31) --
	(202.34, 71.31) --
	(202.41, 71.32) --
	(202.41, 71.32) --
	(202.41, 71.32) --
	(202.49, 71.32) --
	(202.49, 71.32) --
	(202.49, 71.32) --
	(202.56, 71.33) --
	(202.56, 71.33) --
	(202.56, 71.33) --
	(202.63, 71.36) --
	(202.63, 71.36) --
	(202.63, 71.36) --
	(202.70, 71.36) --
	(202.70, 71.36) --
	(202.70, 71.36) --
	(202.78, 71.41) --
	(202.78, 71.41) --
	(202.78, 71.41) --
	(202.81, 71.43) --
	(202.81, 71.43) --
	(202.81, 71.43) --
	(202.85, 71.45) --
	(202.85, 71.45) --
	(202.85, 71.45) --
	(202.92, 71.49) --
	(202.92, 71.49) --
	(202.92, 71.49) --
	(202.99, 71.53) --
	(202.99, 71.53) --
	(202.99, 71.53) --
	(203.07, 71.58) --
	(203.07, 71.58) --
	(203.07, 71.58) --
	(203.14, 71.57) --
	(203.14, 71.57) --
	(203.14, 71.57) --
	(203.21, 71.55) --
	(203.21, 71.55) --
	(203.21, 71.55) --
	(203.28, 71.59) --
	(203.28, 71.59) --
	(203.28, 71.59) --
	(203.36, 71.56) --
	(203.36, 71.56) --
	(203.36, 71.56) --
	(203.43, 71.57) --
	(203.43, 71.57) --
	(203.43, 71.57) --
	(203.50, 71.57) --
	(203.50, 71.57) --
	(203.50, 71.57) --
	(203.57, 71.48) --
	(203.57, 71.48) --
	(203.57, 71.48) --
	(203.64, 71.50) --
	(203.65, 71.50) --
	(203.65, 71.50) --
	(203.72, 71.47) --
	(203.72, 71.47) --
	(203.72, 71.47) --
	(203.76, 71.47) --
	(203.76, 71.47) --
	(203.76, 71.47) --
	(203.79, 71.46) --
	(203.79, 71.46) --
	(203.79, 71.46) --
	(203.86, 71.44) --
	(203.86, 71.44) --
	(203.86, 71.44) --
	(203.94, 71.42) --
	(203.94, 71.42) --
	(203.94, 71.42) --
	(204.01, 71.35) --
	(204.01, 71.35) --
	(204.01, 71.35) --
	(204.08, 71.34) --
	(204.08, 71.34) --
	(204.08, 71.34) --
	(204.15, 71.30) --
	(204.15, 71.30) --
	(204.15, 71.30) --
	(204.22, 71.30) --
	(204.22, 71.30) --
	(204.22, 71.30) --
	(204.30, 71.30) --
	(204.30, 71.30) --
	(204.30, 71.30) --
	(204.37, 71.29) --
	(204.37, 71.29) --
	(204.37, 71.29) --
	(204.44, 71.25) --
	(204.44, 71.25) --
	(204.44, 71.25) --
	(204.51, 71.24) --
	(204.51, 71.24) --
	(204.51, 71.24) --
	(204.59, 71.24) --
	(204.59, 71.24) --
	(204.59, 71.24) --
	(204.63, 71.26) --
	(204.63, 71.26) --
	(204.63, 71.26) --
	(204.66, 71.27) --
	(204.66, 71.27) --
	(204.66, 71.27) --
	(204.73, 71.23) --
	(204.73, 71.23) --
	(204.73, 71.23) --
	(204.80, 71.23) --
	(204.80, 71.23) --
	(204.80, 71.23) --
	(204.88, 71.20) --
	(204.88, 71.20) --
	(204.88, 71.20) --
	(204.95, 71.22) --
	(204.95, 71.22) --
	(204.95, 71.22) --
	(205.02, 71.22) --
	(205.02, 71.22) --
	(205.02, 71.22) --
	(205.09, 71.22) --
	(205.09, 71.22) --
	(205.09, 71.22) --
	(205.16, 71.18) --
	(205.16, 71.18) --
	(205.16, 71.18) --
	(205.24, 71.18) --
	(205.24, 71.18) --
	(205.24, 71.18) --
	(205.31, 71.19) --
	(205.31, 71.19) --
	(205.31, 71.19) --
	(205.38, 71.13) --
	(205.38, 71.13) --
	(205.38, 71.13) --
	(205.45, 71.12) --
	(205.45, 71.12) --
	(205.45, 71.12) --
	(205.53, 71.07) --
	(205.53, 71.07) --
	(205.53, 71.07) --
	(205.58, 71.05) --
	(205.58, 71.05) --
	(205.58, 71.05) --
	(205.60, 71.05) --
	(205.60, 71.05) --
	(205.60, 71.05) --
	(205.67, 71.01) --
	(205.67, 71.01) --
	(205.67, 71.01) --
	(205.74, 70.98) --
	(205.74, 70.98) --
	(205.74, 70.98) --
	(205.82, 70.95) --
	(205.82, 70.95) --
	(205.82, 70.95) --
	(205.89, 70.91) --
	(205.89, 70.91) --
	(205.89, 70.91) --
	(205.96, 70.84) --
	(205.96, 70.84) --
	(205.96, 70.84) --
	(206.03, 70.84) --
	(206.03, 70.84) --
	(206.03, 70.84) --
	(206.10, 70.82) --
	(206.10, 70.82) --
	(206.10, 70.82) --
	(206.18, 70.80) --
	(206.18, 70.80) --
	(206.18, 70.80) --
	(206.25, 70.77) --
	(206.25, 70.77) --
	(206.25, 70.77) --
	(206.32, 70.75) --
	(206.32, 70.75) --
	(206.32, 70.75) --
	(206.39, 70.73) --
	(206.39, 70.73) --
	(206.39, 70.73) --
	(206.46, 70.71) --
	(206.46, 70.71) --
	(206.46, 70.71) --
	(206.54, 70.69) --
	(206.54, 70.69) --
	(206.54, 70.69) --
	(206.54, 70.68) --
	(206.54, 70.68) --
	(206.54, 70.68) --
	(206.61, 70.67) --
	(206.61, 70.67) --
	(206.61, 70.67) --
	(206.68, 70.67) --
	(206.68, 70.67) --
	(206.68, 70.67) --
	(206.75, 70.65) --
	(206.75, 70.65) --
	(206.75, 70.65) --
	(206.83, 70.66) --
	(206.83, 70.66) --
	(206.83, 70.66) --
	(206.90, 70.60) --
	(206.90, 70.60) --
	(206.90, 70.60) --
	(206.97, 70.63) --
	(206.97, 70.63) --
	(206.97, 70.63) --
	(207.04, 70.61) --
	(207.04, 70.61) --
	(207.04, 70.61) --
	(207.12, 70.58) --
	(207.12, 70.58) --
	(207.12, 70.58) --
	(207.19, 70.61) --
	(207.19, 70.61) --
	(207.19, 70.61) --
	(207.26, 70.62) --
	(207.26, 70.62) --
	(207.26, 70.62) --
	(207.33, 70.62) --
	(207.33, 70.62) --
	(207.33, 70.62) --
	(207.40, 70.68) --
	(207.40, 70.68) --
	(207.40, 70.68) --
	(207.48, 70.72) --
	(207.48, 70.72) --
	(207.48, 70.72) --
	(207.55, 70.77) --
	(207.55, 70.77) --
	(207.55, 70.77) --
	(207.60, 70.82) --
	(207.60, 70.82) --
	(207.60, 70.82) --
	(207.62, 70.84) --
	(207.62, 70.84) --
	(207.62, 70.84) --
	(207.69, 70.86) --
	(207.69, 70.86) --
	(207.69, 70.86) --
	(207.76, 70.88) --
	(207.76, 70.88) --
	(207.76, 70.88) --
	(207.84, 70.90) --
	(207.84, 70.90) --
	(207.84, 70.90) --
	(207.91, 70.89) --
	(207.91, 70.89) --
	(207.91, 70.89) --
	(207.98, 70.84) --
	(207.98, 70.84) --
	(207.98, 70.84) --
	(208.05, 70.82) --
	(208.05, 70.82) --
	(208.05, 70.82) --
	(208.12, 70.75) --
	(208.12, 70.75) --
	(208.12, 70.75) --
	(208.20, 70.68) --
	(208.20, 70.68) --
	(208.20, 70.68) --
	(208.27, 70.61) --
	(208.27, 70.61) --
	(208.27, 70.61) --
	(208.34, 70.60) --
	(208.34, 70.60) --
	(208.34, 70.60) --
	(208.41, 70.56) --
	(208.41, 70.56) --
	(208.41, 70.56) --
	(208.46, 70.53) --
	(208.46, 70.53) --
	(208.46, 70.53) --
	(208.48, 70.51) --
	(208.48, 70.51) --
	(208.48, 70.51) --
	(208.56, 70.51) --
	(208.56, 70.51) --
	(208.56, 70.51) --
	(208.63, 70.49) --
	(208.63, 70.49) --
	(208.63, 70.49) --
	(208.70, 70.46) --
	(208.70, 70.46) --
	(208.70, 70.46) --
	(208.77, 70.45) --
	(208.77, 70.45) --
	(208.77, 70.45) --
	(208.84, 70.46) --
	(208.84, 70.46) --
	(208.84, 70.46) --
	(208.92, 70.44) --
	(208.92, 70.44) --
	(208.92, 70.44) --
	(208.99, 70.44) --
	(208.99, 70.44) --
	(208.99, 70.44) --
	(209.06, 70.40) --
	(209.06, 70.40) --
	(209.06, 70.40) --
	(209.13, 70.43) --
	(209.13, 70.43) --
	(209.13, 70.43) --
	(209.20, 70.43) --
	(209.20, 70.43) --
	(209.20, 70.43) --
	(209.28, 70.45) --
	(209.28, 70.45) --
	(209.28, 70.45) --
	(209.32, 70.45) --
	(209.32, 70.45) --
	(209.32, 70.45) --
	(209.35, 70.45) --
	(209.35, 70.45) --
	(209.35, 70.45) --
	(209.42, 70.50) --
	(209.42, 70.50) --
	(209.42, 70.50) --
	(209.49, 70.48) --
	(209.49, 70.48) --
	(209.49, 70.48) --
	(209.56, 70.47) --
	(209.56, 70.47) --
	(209.56, 70.47) --
	(209.64, 70.48) --
	(209.64, 70.48) --
	(209.64, 70.48) --
	(209.71, 70.51) --
	(209.71, 70.51) --
	(209.71, 70.51) --
	(209.78, 70.55) --
	(209.78, 70.55) --
	(209.78, 70.55) --
	(209.85, 70.54) --
	(209.85, 70.54) --
	(209.85, 70.54) --
	(209.92, 70.57) --
	(209.92, 70.57) --
	(209.92, 70.57) --
	(210.00, 70.57) --
	(210.00, 70.57) --
	(210.00, 70.57) --
	(210.07, 70.61) --
	(210.07, 70.61) --
	(210.07, 70.61) --
	(210.14, 70.59) --
	(210.14, 70.59) --
	(210.14, 70.59) --
	(210.21, 70.66) --
	(210.21, 70.66) --
	(210.21, 70.66) --
	(210.28, 70.66) --
	(210.28, 70.66) --
	(210.28, 70.66) --
	(210.29, 70.66) --
	(210.29, 70.66) --
	(210.29, 70.66) --
	(210.36, 70.72) --
	(210.36, 70.72) --
	(210.36, 70.72) --
	(210.43, 70.75) --
	(210.43, 70.75) --
	(210.43, 70.75) --
	(210.50, 70.81) --
	(210.50, 70.81) --
	(210.50, 70.81) --
	(210.57, 70.82) --
	(210.57, 70.82) --
	(210.57, 70.82) --
	(210.65, 70.88) --
	(210.65, 70.88) --
	(210.65, 70.88) --
	(210.72, 70.88) --
	(210.72, 70.88) --
	(210.72, 70.88) --
	(210.79, 70.93) --
	(210.79, 70.93) --
	(210.79, 70.93) --
	(210.86, 71.00) --
	(210.86, 71.00) --
	(210.86, 71.00) --
	(210.93, 71.05) --
	(210.93, 71.05) --
	(210.93, 71.05) --
	(211.00, 71.10) --
	(211.00, 71.10) --
	(211.00, 71.10) --
	(211.08, 71.05) --
	(211.08, 71.05) --
	(211.08, 71.05) --
	(211.15, 71.07) --
	(211.15, 71.07) --
	(211.15, 71.07) --
	(211.22, 71.13) --
	(211.22, 71.13) --
	(211.22, 71.13) --
	(211.29, 71.14) --
	(211.29, 71.14) --
	(211.29, 71.14) --
	(211.36, 71.14) --
	(211.36, 71.14) --
	(211.36, 71.14) --
	(211.43, 71.16) --
	(211.43, 71.16) --
	(211.43, 71.16) --
	(211.43, 71.16) --
	(211.43, 71.16) --
	(211.43, 71.16) --
	(211.51, 71.16) --
	(211.51, 71.16) --
	(211.51, 71.16) --
	(211.58, 71.22) --
	(211.58, 71.22) --
	(211.58, 71.22) --
	(211.65, 71.30) --
	(211.65, 71.30) --
	(211.65, 71.30) --
	(211.72, 71.33) --
	(211.72, 71.33) --
	(211.72, 71.33) --
	(211.79, 71.38) --
	(211.79, 71.38) --
	(211.79, 71.38) --
	(211.87, 71.47) --
	(211.87, 71.47) --
	(211.87, 71.47) --
	(211.94, 71.51) --
	(211.94, 71.51) --
	(211.94, 71.51) --
	(212.01, 71.59) --
	(212.01, 71.59) --
	(212.01, 71.59) --
	(212.08, 71.60) --
	(212.08, 71.60) --
	(212.08, 71.60) --
	(212.15, 71.64) --
	(212.15, 71.64) --
	(212.15, 71.64) --
	(212.23, 71.66) --
	(212.23, 71.66) --
	(212.23, 71.66) --
	(212.30, 71.67) --
	(212.30, 71.67) --
	(212.30, 71.67) --
	(212.37, 71.65) --
	(212.37, 71.65) --
	(212.37, 71.65) --
	(212.44, 71.70) --
	(212.44, 71.70) --
	(212.44, 71.70) --
	(212.51, 71.68) --
	(212.51, 71.68) --
	(212.51, 71.68) --
	(212.58, 71.73) --
	(212.58, 71.73) --
	(212.58, 71.73) --
	(212.66, 71.78) --
	(212.66, 71.78) --
	(212.66, 71.78) --
	(212.73, 71.80) --
	(212.73, 71.80) --
	(212.73, 71.80) --
	(212.80, 71.82) --
	(212.80, 71.82) --
	(212.80, 71.82) --
	(212.87, 71.88) --
	(212.87, 71.88) --
	(212.87, 71.88) --
	(212.94, 71.94) --
	(212.94, 71.94) --
	(212.94, 71.94) --
	(212.96, 71.96) --
	(212.96, 71.96) --
	(212.96, 71.96) --
	(213.02, 72.02) --
	(213.02, 72.02) --
	(213.02, 72.02) --
	(213.09, 72.15) --
	(213.09, 72.15) --
	(213.09, 72.15) --
	(213.16, 72.32) --
	(213.16, 72.32) --
	(213.16, 72.32) --
	(213.23, 72.58) --
	(213.23, 72.58) --
	(213.23, 72.58) --
	(213.30, 72.90) --
	(213.30, 72.90) --
	(213.30, 72.90) --
	(213.37, 73.24) --
	(213.37, 73.24) --
	(213.37, 73.24) --
	(213.45, 73.43) --
	(213.45, 73.43) --
	(213.45, 73.43) --
	(213.52, 73.36) --
	(213.52, 73.36) --
	(213.52, 73.36) --
	(213.59, 73.32) --
	(213.59, 73.32) --
	(213.59, 73.32) --
	(213.66, 73.30) --
	(213.66, 73.30) --
	(213.66, 73.30) --
	(213.73, 73.30) --
	(213.73, 73.30) --
	(213.73, 73.30) --
	(213.80, 73.30) --
	(213.80, 73.30) --
	(213.80, 73.30) --
	(213.88, 73.43) --
	(213.88, 73.43) --
	(213.88, 73.43) --
	(213.95, 73.49) --
	(213.95, 73.49) --
	(213.95, 73.49) --
	(214.02, 73.58) --
	(214.02, 73.58) --
	(214.02, 73.58) --
	(214.09, 73.71) --
	(214.09, 73.71) --
	(214.09, 73.71) --
	(214.16, 73.87) --
	(214.16, 73.87) --
	(214.16, 73.87) --
	(214.23, 74.09) --
	(214.23, 74.09) --
	(214.23, 74.09) --
	(214.31, 74.22) --
	(214.31, 74.22) --
	(214.31, 74.22) --
	(214.31, 74.22) --
	(214.31, 74.22) --
	(214.31, 74.22) --
	(214.38, 74.33) --
	(214.38, 74.33) --
	(214.38, 74.33) --
	(214.45, 74.35) --
	(214.45, 74.35) --
	(214.45, 74.35) --
	(214.52, 74.32) --
	(214.52, 74.32) --
	(214.52, 74.32) --
	(214.59, 74.16) --
	(214.59, 74.16) --
	(214.59, 74.16) --
	(214.66, 74.10) --
	(214.66, 74.10) --
	(214.66, 74.10) --
	(214.74, 73.87) --
	(214.74, 73.87) --
	(214.74, 73.87) --
	(214.81, 73.73) --
	(214.81, 73.73) --
	(214.81, 73.73) --
	(214.88, 73.62) --
	(214.88, 73.62) --
	(214.88, 73.62) --
	(214.95, 73.51) --
	(214.95, 73.51) --
	(214.95, 73.51) --
	(215.02, 73.38) --
	(215.02, 73.38) --
	(215.02, 73.38) --
	(215.09, 73.31) --
	(215.09, 73.31) --
	(215.09, 73.31) --
	(215.17, 73.28) --
	(215.17, 73.28) --
	(215.17, 73.28) --
	(215.24, 73.26) --
	(215.24, 73.26) --
	(215.24, 73.26) --
	(215.31, 73.18) --
	(215.31, 73.18) --
	(215.31, 73.18) --
	(215.38, 73.12) --
	(215.38, 73.12) --
	(215.38, 73.12) --
	(215.45, 73.11) --
	(215.45, 73.11) --
	(215.45, 73.11) --
	(215.52, 73.04) --
	(215.52, 73.04) --
	(215.52, 73.04) --
	(215.60, 73.10) --
	(215.60, 73.10) --
	(215.60, 73.10) --
	(215.67, 73.11) --
	(215.67, 73.11) --
	(215.67, 73.11) --
	(215.74, 73.12) --
	(215.74, 73.12) --
	(215.74, 73.12) --
	(215.74, 73.12) --
	(215.74, 73.12) --
	(215.74, 73.12) --
	(215.81, 73.15) --
	(215.81, 73.15) --
	(215.81, 73.15) --
	(215.88, 73.17) --
	(215.88, 73.17) --
	(215.88, 73.17) --
	(215.95, 73.22) --
	(215.95, 73.22) --
	(215.95, 73.22) --
	(216.03, 73.16) --
	(216.03, 73.16) --
	(216.03, 73.16) --
	(216.10, 73.15) --
	(216.10, 73.15) --
	(216.10, 73.15) --
	(216.17, 73.12) --
	(216.17, 73.12) --
	(216.17, 73.12) --
	(216.24, 73.10) --
	(216.24, 73.10) --
	(216.24, 73.10) --
	(216.31, 73.05) --
	(216.31, 73.05) --
	(216.31, 73.05) --
	(216.38, 73.02) --
	(216.38, 73.02) --
	(216.38, 73.02) --
	(216.46, 72.95) --
	(216.46, 72.95) --
	(216.46, 72.95) --
	(216.53, 72.93) --
	(216.53, 72.93) --
	(216.53, 72.93) --
	(216.60, 72.94) --
	(216.60, 72.94) --
	(216.60, 72.94) --
	(216.67, 72.96) --
	(216.67, 72.96) --
	(216.67, 72.96) --
	(216.74, 72.92) --
	(216.74, 72.92) --
	(216.74, 72.92) --
	(216.81, 72.90) --
	(216.81, 72.90) --
	(216.81, 72.90) --
	(216.88, 72.87) --
	(216.88, 72.87) --
	(216.88, 72.87) --
	(216.96, 72.88) --
	(216.96, 72.88) --
	(216.96, 72.88) --
	(217.03, 72.86) --
	(217.03, 72.86) --
	(217.03, 72.86) --
	(217.10, 72.88) --
	(217.10, 72.88) --
	(217.10, 72.88) --
	(217.17, 72.93) --
	(217.17, 72.93) --
	(217.17, 72.93) --
	(217.24, 73.03) --
	(217.24, 73.03) --
	(217.24, 73.03) --
	(217.31, 73.10) --
	(217.31, 73.10) --
	(217.31, 73.10) --
	(217.38, 73.17) --
	(217.38, 73.17) --
	(217.38, 73.17) --
	(217.46, 73.25) --
	(217.46, 73.25) --
	(217.46, 73.25) --
	(217.53, 73.38) --
	(217.53, 73.38) --
	(217.53, 73.38) --
	(217.60, 73.53) --
	(217.60, 73.53) --
	(217.60, 73.53) --
	(217.67, 73.60) --
	(217.67, 73.60) --
	(217.67, 73.60) --
	(217.74, 73.79) --
	(217.74, 73.79) --
	(217.74, 73.79) --
	(217.76, 73.83) --
	(217.76, 73.83) --
	(217.76, 73.83) --
	(217.81, 74.01) --
	(217.81, 74.01) --
	(217.81, 74.01) --
	(217.88, 74.16) --
	(217.88, 74.16) --
	(217.88, 74.16) --
	(217.96, 74.39) --
	(217.96, 74.39) --
	(217.96, 74.39) --
	(218.03, 74.64) --
	(218.03, 74.64) --
	(218.03, 74.64) --
	(218.10, 74.81) --
	(218.10, 74.81) --
	(218.10, 74.81) --
	(218.17, 74.94) --
	(218.17, 74.94) --
	(218.17, 74.94) --
	(218.24, 74.93) --
	(218.24, 74.93) --
	(218.24, 74.93) --
	(218.31, 74.85) --
	(218.31, 74.85) --
	(218.31, 74.85) --
	(218.38, 74.70) --
	(218.38, 74.70) --
	(218.38, 74.70) --
	(218.46, 74.47) --
	(218.46, 74.47) --
	(218.46, 74.47) --
	(218.53, 74.32) --
	(218.53, 74.32) --
	(218.53, 74.32) --
	(218.60, 74.11) --
	(218.60, 74.11) --
	(218.60, 74.11) --
	(218.67, 73.91) --
	(218.67, 73.91) --
	(218.67, 73.91) --
	(218.74, 73.84) --
	(218.74, 73.84) --
	(218.74, 73.84) --
	(218.81, 73.66) --
	(218.81, 73.66) --
	(218.81, 73.66) --
	(218.88, 73.48) --
	(218.88, 73.48) --
	(218.88, 73.48) --
	(218.95, 73.32) --
	(218.95, 73.32) --
	(218.95, 73.32) --
	(219.03, 73.20) --
	(219.03, 73.20) --
	(219.03, 73.20) --
	(219.10, 73.16) --
	(219.10, 73.16) --
	(219.10, 73.16) --
	(219.17, 73.02) --
	(219.17, 73.02) --
	(219.17, 73.02) --
	(219.24, 72.96) --
	(219.24, 72.96) --
	(219.24, 72.96) --
	(219.31, 72.89) --
	(219.31, 72.89) --
	(219.31, 72.89) --
	(219.38, 72.81) --
	(219.38, 72.81) --
	(219.38, 72.81) --
	(219.46, 72.67) --
	(219.46, 72.67) --
	(219.46, 72.67) --
	(219.48, 72.63) --
	(219.48, 72.63) --
	(219.48, 72.63) --
	(219.53, 72.57) --
	(219.53, 72.57) --
	(219.53, 72.57) --
	(219.60, 72.49) --
	(219.60, 72.49) --
	(219.60, 72.49) --
	(219.67, 72.50) --
	(219.67, 72.50) --
	(219.67, 72.50) --
	(219.74, 72.41) --
	(219.74, 72.41) --
	(219.74, 72.41) --
	(219.81, 72.37) --
	(219.81, 72.37) --
	(219.81, 72.37) --
	(219.88, 72.35) --
	(219.88, 72.35) --
	(219.88, 72.35) --
	(219.95, 72.36) --
	(219.95, 72.36) --
	(219.95, 72.36) --
	(220.03, 72.35) --
	(220.03, 72.35) --
	(220.03, 72.35) --
	(220.10, 72.39) --
	(220.10, 72.39) --
	(220.10, 72.39) --
	(220.17, 72.41) --
	(220.17, 72.41) --
	(220.17, 72.41) --
	(220.24, 72.43) --
	(220.24, 72.43) --
	(220.24, 72.43) --
	(220.31, 72.46) --
	(220.31, 72.46) --
	(220.31, 72.46) --
	(220.38, 72.51) --
	(220.38, 72.51) --
	(220.38, 72.51) --
	(220.45, 72.60) --
	(220.45, 72.60) --
	(220.45, 72.60) --
	(220.52, 72.76) --
	(220.52, 72.76) --
	(220.52, 72.76) --
	(220.60, 72.90) --
	(220.60, 72.90) --
	(220.60, 72.90) --
	(220.67, 73.08) --
	(220.67, 73.08) --
	(220.67, 73.08) --
	(220.74, 73.22) --
	(220.74, 73.22) --
	(220.74, 73.22) --
	(220.81, 73.29) --
	(220.81, 73.29) --
	(220.81, 73.29) --
	(220.88, 73.27) --
	(220.88, 73.27) --
	(220.88, 73.27) --
	(220.95, 73.24) --
	(220.95, 73.24) --
	(220.95, 73.24) --
	(221.02, 73.23) --
	(221.02, 73.23) --
	(221.02, 73.23) --
	(221.09, 73.25) --
	(221.09, 73.25) --
	(221.09, 73.25) --
	(221.17, 73.38) --
	(221.17, 73.38) --
	(221.17, 73.38) --
	(221.24, 73.60) --
	(221.24, 73.60) --
	(221.24, 73.60) --
	(221.31, 73.82) --
	(221.31, 73.82) --
	(221.31, 73.82) --
	(221.38, 73.88) --
	(221.38, 73.88) --
	(221.38, 73.88) --
	(221.45, 73.91) --
	(221.45, 73.91) --
	(221.45, 73.91) --
	(221.49, 73.97) --
	(221.49, 73.97) --
	(221.49, 73.97) --
	(221.52, 74.00) --
	(221.52, 74.00) --
	(221.52, 74.00) --
	(221.59, 73.85) --
	(221.59, 73.85) --
	(221.59, 73.85) --
	(221.66, 73.65) --
	(221.66, 73.65) --
	(221.66, 73.65) --
	(221.74, 73.37) --
	(221.74, 73.37) --
	(221.74, 73.37) --
	(221.81, 73.02) --
	(221.81, 73.02) --
	(221.81, 73.02) --
	(221.88, 72.87) --
	(221.88, 72.87) --
	(221.88, 72.87) --
	(221.95, 72.65) --
	(221.95, 72.65) --
	(221.95, 72.65) --
	(222.02, 72.51) --
	(222.02, 72.51) --
	(222.02, 72.51) --
	(222.09, 72.49) --
	(222.09, 72.49) --
	(222.09, 72.49) --
	(222.16, 72.45) --
	(222.16, 72.45) --
	(222.16, 72.45) --
	(222.23, 72.45) --
	(222.23, 72.45) --
	(222.23, 72.45) --
	(222.30, 72.53) --
	(222.30, 72.53) --
	(222.30, 72.53) --
	(222.37, 72.65) --
	(222.37, 72.65) --
	(222.37, 72.65) --
	(222.45, 72.71) --
	(222.45, 72.71) --
	(222.45, 72.71) --
	(222.52, 72.88) --
	(222.52, 72.88) --
	(222.52, 72.88) --
	(222.59, 73.05) --
	(222.59, 73.05) --
	(222.59, 73.05) --
	(222.66, 73.13) --
	(222.66, 73.13) --
	(222.66, 73.13) --
	(222.73, 73.19) --
	(222.73, 73.19) --
	(222.73, 73.19) --
	(222.80, 73.16) --
	(222.80, 73.16) --
	(222.80, 73.16) --
	(222.87, 73.06) --
	(222.87, 73.06) --
	(222.87, 73.06) --
	(222.94, 72.95) --
	(222.94, 72.95) --
	(222.94, 72.95) --
	(223.02, 72.89) --
	(223.02, 72.89) --
	(223.02, 72.89) --
	(223.09, 72.72) --
	(223.09, 72.72) --
	(223.09, 72.72) --
	(223.16, 72.60) --
	(223.16, 72.60) --
	(223.16, 72.60) --
	(223.23, 72.50) --
	(223.23, 72.50) --
	(223.23, 72.50) --
	(223.30, 72.36) --
	(223.30, 72.36) --
	(223.30, 72.36) --
	(223.37, 72.26) --
	(223.37, 72.26) --
	(223.37, 72.26) --
	(223.44, 72.19) --
	(223.44, 72.19) --
	(223.44, 72.19) --
	(223.51, 72.11) --
	(223.51, 72.11) --
	(223.51, 72.11) --
	(223.58, 72.02) --
	(223.58, 72.02) --
	(223.58, 72.02) --
	(223.66, 72.01) --
	(223.66, 72.01) --
	(223.66, 72.01) --
	(223.73, 71.96) --
	(223.73, 71.96) --
	(223.73, 71.96) --
	(223.79, 71.95) --
	(223.79, 71.95) --
	(223.79, 71.95) --
	(223.80, 71.95) --
	(223.80, 71.95) --
	(223.80, 71.95) --
	(223.87, 71.92) --
	(223.87, 71.92) --
	(223.87, 71.92) --
	(223.94, 71.97) --
	(223.94, 71.97) --
	(223.94, 71.97) --
	(224.01, 72.07) --
	(224.01, 72.07) --
	(224.01, 72.07) --
	(224.08, 72.18) --
	(224.08, 72.18) --
	(224.08, 72.18) --
	(224.15, 72.17) --
	(224.15, 72.17) --
	(224.15, 72.17) --
	(224.22, 72.17) --
	(224.22, 72.17) --
	(224.22, 72.17) --
	(224.29, 72.17) --
	(224.29, 72.17) --
	(224.29, 72.17) --
	(224.36, 72.07) --
	(224.36, 72.07) --
	(224.36, 72.07) --
	(224.44, 72.03) --
	(224.44, 72.03) --
	(224.44, 72.03) --
	(224.51, 71.94) --
	(224.51, 71.94) --
	(224.51, 71.94) --
	(224.58, 71.87) --
	(224.58, 71.87) --
	(224.58, 71.87) --
	(224.65, 71.90) --
	(224.65, 71.90) --
	(224.65, 71.90) --
	(224.72, 71.92) --
	(224.72, 71.92) --
	(224.72, 71.92) --
	(224.79, 71.90) --
	(224.79, 71.90) --
	(224.79, 71.90) --
	(224.86, 71.83) --
	(224.86, 71.83) --
	(224.86, 71.83) --
	(224.93, 71.81) --
	(224.93, 71.81) --
	(224.93, 71.81) --
	(225.00, 71.76) --
	(225.00, 71.76) --
	(225.00, 71.76) --
	(225.07, 71.72) --
	(225.07, 71.72) --
	(225.07, 71.72) --
	(225.15, 71.67) --
	(225.15, 71.67) --
	(225.15, 71.67) --
	(225.22, 71.66) --
	(225.22, 71.66) --
	(225.22, 71.66) --
	(225.29, 71.67) --
	(225.29, 71.67) --
	(225.29, 71.67) --
	(225.36, 71.68) --
	(225.36, 71.68) --
	(225.36, 71.68) --
	(225.43, 71.66) --
	(225.43, 71.66) --
	(225.43, 71.66) --
	(225.50, 71.73) --
	(225.50, 71.73) --
	(225.50, 71.73) --
	(225.57, 71.75) --
	(225.57, 71.75) --
	(225.57, 71.75) --
	(225.61, 71.78) --
	(225.61, 71.78) --
	(225.61, 71.78) --
	(225.64, 71.80) --
	(225.64, 71.80) --
	(225.64, 71.80) --
	(225.71, 71.90) --
	(225.71, 71.90) --
	(225.71, 71.90) --
	(225.78, 71.94) --
	(225.78, 71.94) --
	(225.78, 71.94) --
	(225.85, 72.05) --
	(225.85, 72.05) --
	(225.85, 72.05) --
	(225.93, 72.18) --
	(225.93, 72.18) --
	(225.93, 72.18) --
	(226.00, 72.25) --
	(226.00, 72.25) --
	(226.00, 72.25) --
	(226.07, 72.31) --
	(226.07, 72.31) --
	(226.07, 72.31) --
	(226.14, 72.35) --
	(226.14, 72.35) --
	(226.14, 72.35) --
	(226.21, 72.54) --
	(226.21, 72.54) --
	(226.21, 72.54) --
	(226.28, 72.91) --
	(226.28, 72.91) --
	(226.28, 72.91) --
	(226.35, 73.85) --
	(226.35, 73.85) --
	(226.35, 73.85) --
	(226.42, 75.69) --
	(226.42, 75.69) --
	(226.42, 75.69) --
	(226.49, 77.82) --
	(226.49, 77.82) --
	(226.49, 77.82) --
	(226.56, 78.62) --
	(226.56, 78.62) --
	(226.56, 78.62) --
	(226.63, 77.77) --
	(226.63, 77.77) --
	(226.63, 77.77) --
	(226.71, 76.26) --
	(226.71, 76.26) --
	(226.71, 76.26) --
	(226.78, 74.90) --
	(226.78, 74.90) --
	(226.78, 74.90) --
	(226.85, 73.79) --
	(226.85, 73.79) --
	(226.85, 73.79) --
	(226.92, 72.98) --
	(226.92, 72.98) --
	(226.92, 72.98) --
	(226.99, 72.36) --
	(226.99, 72.36) --
	(226.99, 72.36) --
	(227.06, 71.88) --
	(227.06, 71.88) --
	(227.06, 71.88) --
	(227.13, 71.59) --
	(227.13, 71.59) --
	(227.13, 71.59) --
	(227.20, 71.30) --
	(227.20, 71.30) --
	(227.20, 71.30) --
	(227.27, 71.08) --
	(227.27, 71.08) --
	(227.27, 71.08) --
	(227.34, 70.94) --
	(227.34, 70.94) --
	(227.34, 70.94) --
	(227.41, 70.76) --
	(227.41, 70.76) --
	(227.41, 70.76) --
	(227.48, 70.66) --
	(227.48, 70.66) --
	(227.48, 70.66) --
	(227.56, 70.56) --
	(227.56, 70.56) --
	(227.56, 70.56) --
	(227.62, 70.46) --
	(227.62, 70.46) --
	(227.62, 70.46) --
	(227.70, 70.38) --
	(227.70, 70.38) --
	(227.70, 70.38) --
	(227.77, 70.32) --
	(227.77, 70.32) --
	(227.77, 70.32) --
	(227.84, 70.23) --
	(227.84, 70.23) --
	(227.84, 70.23) --
	(227.91, 70.13) --
	(227.91, 70.13) --
	(227.91, 70.13) --
	(227.91, 70.12) --
	(227.91, 70.12) --
	(227.91, 70.12) --
	(227.98, 70.03) --
	(227.98, 70.03) --
	(227.98, 70.03) --
	(228.05, 69.97) --
	(228.05, 69.97) --
	(228.05, 69.97) --
	(228.12, 69.90) --
	(228.12, 69.90) --
	(228.12, 69.90) --
	(228.19, 69.83) --
	(228.19, 69.83) --
	(228.19, 69.83) --
	(228.26, 69.76) --
	(228.26, 69.76) --
	(228.26, 69.76) --
	(228.33, 69.71) --
	(228.33, 69.71) --
	(228.33, 69.71) --
	(228.40, 69.65) --
	(228.40, 69.65) --
	(228.40, 69.65) --
	(228.47, 69.58) --
	(228.47, 69.58) --
	(228.47, 69.58) --
	(228.55, 69.58) --
	(228.55, 69.58) --
	(228.55, 69.58) --
	(228.62, 69.49) --
	(228.62, 69.49) --
	(228.62, 69.49) --
	(228.69, 69.44) --
	(228.69, 69.44) --
	(228.69, 69.44) --
	(228.76, 69.41) --
	(228.76, 69.41) --
	(228.76, 69.41) --
	(228.83, 69.39) --
	(228.83, 69.39) --
	(228.83, 69.39) --
	(228.90, 69.36) --
	(228.90, 69.36) --
	(228.90, 69.36) --
	(228.97, 69.30) --
	(228.97, 69.30) --
	(228.97, 69.30) --
	(229.04, 69.29) --
	(229.04, 69.29) --
	(229.04, 69.29) --
	(229.11, 69.29) --
	(229.11, 69.29) --
	(229.11, 69.29) --
	(229.18, 69.28) --
	(229.18, 69.28) --
	(229.18, 69.28) --
	(229.25, 69.31) --
	(229.25, 69.31) --
	(229.25, 69.31) --
	(229.32, 69.33) --
	(229.32, 69.33) --
	(229.32, 69.33) --
	(229.39, 69.31) --
	(229.39, 69.31) --
	(229.39, 69.31) --
	(229.46, 69.28) --
	(229.46, 69.28) --
	(229.46, 69.28) --
	(229.53, 69.27) --
	(229.53, 69.27) --
	(229.53, 69.27) --
	(229.60, 69.22) --
	(229.60, 69.22) --
	(229.60, 69.22) --
	(229.68, 69.18) --
	(229.68, 69.18) --
	(229.68, 69.18) --
	(229.75, 69.13) --
	(229.75, 69.13) --
	(229.75, 69.13) --
	(229.82, 69.09) --
	(229.82, 69.09) --
	(229.82, 69.09) --
	(229.89, 69.06) --
	(229.89, 69.06) --
	(229.89, 69.06) --
	(229.96, 69.06) --
	(229.96, 69.06) --
	(229.96, 69.06) --
	(230.03, 69.07) --
	(230.03, 69.07) --
	(230.03, 69.07) --
	(230.10, 69.03) --
	(230.10, 69.03) --
	(230.10, 69.03) --
	(230.17, 69.06) --
	(230.17, 69.06) --
	(230.17, 69.06) --
	(230.24, 69.08) --
	(230.24, 69.08) --
	(230.24, 69.08) --
	(230.31, 69.22) --
	(230.31, 69.22) --
	(230.31, 69.22) --
	(230.38, 69.42) --
	(230.38, 69.42) --
	(230.38, 69.42) --
	(230.45, 69.69) --
	(230.45, 69.69) --
	(230.45, 69.69) --
	(230.50, 69.86) --
	(230.50, 69.86) --
	(230.50, 69.86) --
	(230.52, 69.94) --
	(230.52, 69.94) --
	(230.52, 69.94) --
	(230.59, 70.07) --
	(230.59, 70.07) --
	(230.59, 70.07) --
	(230.66, 70.00) --
	(230.66, 70.00) --
	(230.66, 70.00) --
	(230.73, 69.87) --
	(230.73, 69.87) --
	(230.73, 69.87) --
	(230.80, 69.67) --
	(230.80, 69.67) --
	(230.80, 69.67) --
	(230.88, 69.50) --
	(230.88, 69.50) --
	(230.88, 69.50) --
	(230.95, 69.30) --
	(230.95, 69.30) --
	(230.95, 69.30) --
	(231.02, 69.16) --
	(231.02, 69.16) --
	(231.02, 69.16) --
	(231.09, 69.09) --
	(231.09, 69.09) --
	(231.09, 69.09) --
	(231.16, 68.96) --
	(231.16, 68.96) --
	(231.16, 68.96) --
	(231.23, 68.91) --
	(231.23, 68.91) --
	(231.23, 68.91) --
	(231.30, 68.86) --
	(231.30, 68.86) --
	(231.30, 68.86) --
	(231.37, 68.81) --
	(231.37, 68.81) --
	(231.37, 68.81) --
	(231.44, 68.81) --
	(231.44, 68.81) --
	(231.44, 68.81) --
	(231.51, 68.86) --
	(231.51, 68.86) --
	(231.51, 68.86) --
	(231.58, 68.91) --
	(231.58, 68.91) --
	(231.58, 68.91) --
	(231.65, 69.02) --
	(231.65, 69.02) --
	(231.65, 69.02) --
	(231.72, 69.13) --
	(231.72, 69.13) --
	(231.72, 69.13) --
	(231.79, 69.28) --
	(231.79, 69.28) --
	(231.79, 69.28) --
	(231.86, 69.36) --
	(231.86, 69.36) --
	(231.86, 69.36) --
	(231.93, 69.33) --
	(231.93, 69.33) --
	(231.93, 69.33) --
	(232.00, 69.28) --
	(232.00, 69.28) --
	(232.00, 69.28) --
	(232.07, 69.17) --
	(232.07, 69.17) --
	(232.07, 69.17) --
	(232.14, 69.12) --
	(232.14, 69.12) --
	(232.14, 69.12) --
	(232.21, 69.03) --
	(232.21, 69.03) --
	(232.21, 69.03) --
	(232.28, 68.94) --
	(232.28, 68.94) --
	(232.28, 68.94) --
	(232.36, 68.88) --
	(232.36, 68.88) --
	(232.36, 68.88) --
	(232.43, 68.81) --
	(232.43, 68.81) --
	(232.43, 68.81) --
	(232.50, 68.76) --
	(232.50, 68.76) --
	(232.50, 68.76) --
	(232.57, 68.73) --
	(232.57, 68.73) --
	(232.57, 68.73) --
	(232.64, 68.67) --
	(232.64, 68.67) --
	(232.64, 68.67) --
	(232.71, 68.62) --
	(232.71, 68.62) --
	(232.71, 68.62) --
	(232.78, 68.57) --
	(232.78, 68.57) --
	(232.78, 68.57) --
	(232.85, 68.55) --
	(232.85, 68.55) --
	(232.85, 68.55) --
	(232.92, 68.50) --
	(232.92, 68.50) --
	(232.92, 68.50) --
	(232.99, 68.47) --
	(232.99, 68.47) --
	(232.99, 68.47) --
	(233.06, 68.46) --
	(233.06, 68.46) --
	(233.06, 68.46) --
	(233.13, 68.47) --
	(233.13, 68.47) --
	(233.13, 68.47) --
	(233.20, 68.49) --
	(233.20, 68.49) --
	(233.20, 68.49) --
	(233.27, 68.52) --
	(233.27, 68.52) --
	(233.27, 68.52) --
	(233.28, 68.53) --
	(233.28, 68.53) --
	(233.28, 68.53) --
	(233.34, 68.56) --
	(233.34, 68.56) --
	(233.34, 68.56) --
	(233.41, 68.57) --
	(233.41, 68.57) --
	(233.41, 68.57) --
	(233.48, 68.55) --
	(233.48, 68.55) --
	(233.48, 68.55) --
	(233.55, 68.61) --
	(233.55, 68.61) --
	(233.55, 68.61) --
	(233.62, 68.60) --
	(233.62, 68.60) --
	(233.62, 68.60) --
	(233.69, 68.58) --
	(233.69, 68.58) --
	(233.69, 68.58) --
	(233.76, 68.56) --
	(233.76, 68.56) --
	(233.76, 68.56) --
	(233.83, 68.56) --
	(233.83, 68.56) --
	(233.83, 68.56) --
	(233.90, 68.55) --
	(233.90, 68.55) --
	(233.90, 68.55) --
	(233.97, 68.52) --
	(233.97, 68.52) --
	(233.97, 68.52) --
	(234.04, 68.49) --
	(234.04, 68.49) --
	(234.04, 68.49) --
	(234.11, 68.49) --
	(234.11, 68.49) --
	(234.11, 68.49) --
	(234.18, 68.44) --
	(234.18, 68.44) --
	(234.18, 68.44) --
	(234.25, 68.42) --
	(234.26, 68.42) --
	(234.26, 68.42) --
	(234.32, 68.42) --
	(234.32, 68.42) --
	(234.32, 68.42) --
	(234.40, 68.45) --
	(234.40, 68.45) --
	(234.40, 68.45) --
	(234.47, 68.40) --
	(234.47, 68.40) --
	(234.47, 68.40) --
	(234.54, 68.41) --
	(234.54, 68.41) --
	(234.54, 68.41) --
	(234.61, 68.40) --
	(234.61, 68.40) --
	(234.61, 68.40) --
	(234.68, 68.44) --
	(234.68, 68.44) --
	(234.68, 68.44) --
	(234.75, 68.44) --
	(234.75, 68.44) --
	(234.75, 68.44) --
	(234.82, 68.48) --
	(234.82, 68.48) --
	(234.82, 68.48) --
	(234.89, 68.51) --
	(234.89, 68.51) --
	(234.89, 68.51) --
	(234.96, 68.60) --
	(234.96, 68.60) --
	(234.96, 68.60) --
	(235.03, 68.70) --
	(235.03, 68.70) --
	(235.03, 68.70) --
	(235.10, 68.79) --
	(235.10, 68.79) --
	(235.10, 68.79) --
	(235.17, 68.91) --
	(235.17, 68.91) --
	(235.17, 68.91) --
	(235.24, 69.07) --
	(235.24, 69.07) --
	(235.24, 69.07) --
	(235.31, 69.21) --
	(235.31, 69.21) --
	(235.31, 69.21) --
	(235.38, 69.45) --
	(235.38, 69.45) --
	(235.38, 69.45) --
	(235.45, 69.81) --
	(235.45, 69.81) --
	(235.45, 69.81) --
	(235.52, 70.29) --
	(235.52, 70.29) --
	(235.52, 70.29) --
	(235.59, 70.98) --
	(235.59, 70.98) --
	(235.59, 70.98) --
	(235.66, 71.90) --
	(235.66, 71.90) --
	(235.66, 71.90) --
	(235.73, 73.16) --
	(235.73, 73.16) --
	(235.73, 73.16) --
	(235.80, 75.40) --
	(235.80, 75.40) --
	(235.80, 75.40) --
	(235.87, 78.92) --
	(235.87, 78.92) --
	(235.87, 78.92) --
	(235.94, 82.69) --
	(235.94, 82.69) --
	(235.94, 82.69) --
	(236.01, 82.69) --
	(236.01, 82.69) --
	(236.01, 82.69) --
	(236.08, 82.69) --
	(236.08, 82.69) --
	(236.08, 82.69) --
	(236.15, 82.69) --
	(236.15, 82.69) --
	(236.15, 82.69) --
	(236.22, 82.69) --
	(236.22, 82.69) --
	(236.22, 82.69) --
	(236.25, 82.69) --
	(236.25, 82.69) --
	(236.25, 82.69) --
	(236.29, 82.69) --
	(236.29, 82.69) --
	(236.29, 82.69) --
	(236.36, 80.81) --
	(236.36, 80.81) --
	(236.36, 80.81) --
	(236.43, 78.39) --
	(236.43, 78.39) --
	(236.43, 78.39) --
	(236.50, 76.91) --
	(236.50, 76.90) --
	(236.50, 76.90) --
	(236.57, 76.42) --
	(236.57, 76.42) --
	(236.57, 76.42) --
	(236.64, 76.85) --
	(236.64, 76.85) --
	(236.64, 76.85) --
	(236.71, 77.89) --
	(236.71, 77.89) --
	(236.71, 77.89) --
	(236.78, 79.20) --
	(236.78, 79.20) --
	(236.78, 79.20) --
	(236.85, 79.49) --
	(236.85, 79.49) --
	(236.85, 79.49) --
	(236.92, 78.40) --
	(236.92, 78.40) --
	(236.92, 78.40) --
	(236.99, 76.41) --
	(236.99, 76.41) --
	(236.99, 76.41) --
	(237.06, 74.44) --
	(237.06, 74.44) --
	(237.06, 74.44) --
	(237.13, 72.66) --
	(237.13, 72.66) --
	(237.13, 72.66) --
	(237.20, 71.41) --
	(237.20, 71.41) --
	(237.20, 71.41) --
	(237.27, 70.59) --
	(237.27, 70.59) --
	(237.27, 70.59) --
	(237.34, 70.08) --
	(237.34, 70.08) --
	(237.34, 70.08) --
	(237.41, 69.77) --
	(237.41, 69.77) --
	(237.41, 69.77) --
	(237.48, 69.61) --
	(237.48, 69.61) --
	(237.48, 69.61) --
	(237.55, 69.68) --
	(237.55, 69.68) --
	(237.55, 69.68) --
	(237.62, 69.98) --
	(237.62, 69.98) --
	(237.62, 69.98) --
	(237.69, 70.55) --
	(237.69, 70.55) --
	(237.69, 70.55) --
	(237.76, 71.47) --
	(237.76, 71.47) --
	(237.76, 71.47) --
	(237.83, 72.68) --
	(237.83, 72.68) --
	(237.83, 72.68) --
	(237.90, 73.50) --
	(237.90, 73.50) --
	(237.90, 73.50) --
	(237.97, 73.43) --
	(237.97, 73.43) --
	(237.97, 73.43) --
	(238.04, 72.65) --
	(238.04, 72.65) --
	(238.04, 72.65) --
	(238.11, 71.63) --
	(238.11, 71.63) --
	(238.11, 71.63) --
	(238.18, 70.62) --
	(238.18, 70.62) --
	(238.18, 70.62) --
	(238.25, 69.82) --
	(238.25, 69.82) --
	(238.25, 69.82) --
	(238.32, 69.24) --
	(238.32, 69.24) --
	(238.32, 69.24) --
	(238.39, 68.78) --
	(238.39, 68.78) --
	(238.39, 68.78) --
	(238.46, 68.53) --
	(238.46, 68.53) --
	(238.46, 68.53) --
	(238.53, 68.32) --
	(238.53, 68.32) --
	(238.53, 68.32) --
	(238.60, 68.23) --
	(238.60, 68.23) --
	(238.60, 68.23) --
	(238.67, 68.25) --
	(238.67, 68.25) --
	(238.67, 68.25) --
	(238.74, 68.38) --
	(238.74, 68.38) --
	(238.74, 68.38) --
	(238.74, 68.38) --
	(238.74, 68.38) --
	(238.74, 68.38) --
	(238.81, 68.72) --
	(238.81, 68.72) --
	(238.81, 68.72) --
	(238.88, 69.26) --
	(238.88, 69.26) --
	(238.88, 69.26) --
	(238.95, 69.46) --
	(238.95, 69.46) --
	(238.95, 69.46) --
	(239.02, 69.21) --
	(239.02, 69.21) --
	(239.02, 69.21) --
	(239.09, 68.82) --
	(239.09, 68.82) --
	(239.09, 68.82) --
	(239.16, 68.38) --
	(239.16, 68.38) --
	(239.16, 68.38) --
	(239.23, 67.99) --
	(239.23, 67.99) --
	(239.23, 67.99) --
	(239.30, 67.70) --
	(239.30, 67.70) --
	(239.30, 67.70) --
	(239.37, 67.50) --
	(239.37, 67.50) --
	(239.37, 67.50) --
	(239.44, 67.34) --
	(239.44, 67.34) --
	(239.44, 67.34) --
	(239.51, 67.24) --
	(239.51, 67.24) --
	(239.51, 67.24) --
	(239.58, 67.15) --
	(239.58, 67.15) --
	(239.58, 67.15) --
	(239.65, 67.09) --
	(239.65, 67.09) --
	(239.65, 67.09) --
	(239.72, 67.03) --
	(239.72, 67.03) --
	(239.72, 67.03) --
	(239.79, 66.98) --
	(239.79, 66.98) --
	(239.79, 66.98) --
	(239.86, 66.97) --
	(239.86, 66.97) --
	(239.86, 66.97) --
	(239.93, 66.92) --
	(239.93, 66.92) --
	(239.93, 66.92) --
	(240.00, 66.89) --
	(240.00, 66.89) --
	(240.00, 66.89) --
	(240.07, 66.86) --
	(240.07, 66.86) --
	(240.07, 66.86) --
	(240.14, 66.83) --
	(240.14, 66.83) --
	(240.14, 66.83) --
	(240.21, 66.81) --
	(240.21, 66.81) --
	(240.21, 66.81) --
	(240.28, 66.78) --
	(240.28, 66.78) --
	(240.28, 66.78) --
	(240.35, 66.76) --
	(240.35, 66.76) --
	(240.35, 66.76) --
	(240.42, 66.76) --
	(240.42, 66.76) --
	(240.42, 66.76) --
	(240.49, 66.74) --
	(240.49, 66.74) --
	(240.49, 66.74) --
	(240.56, 66.73) --
	(240.56, 66.73) --
	(240.56, 66.73) --
	(240.56, 66.73) --
	(240.56, 66.73) --
	(240.56, 66.73) --
	(240.63, 66.72) --
	(240.63, 66.72) --
	(240.63, 66.72) --
	(240.70, 66.73) --
	(240.70, 66.73) --
	(240.70, 66.73) --
	(240.77, 66.73) --
	(240.77, 66.73) --
	(240.77, 66.73) --
	(240.84, 66.71) --
	(240.84, 66.71) --
	(240.84, 66.71) --
	(240.91, 66.70) --
	(240.91, 66.70) --
	(240.91, 66.70) --
	(240.98, 66.71) --
	(240.98, 66.71) --
	(240.98, 66.71) --
	(241.05, 66.68) --
	(241.05, 66.68) --
	(241.05, 66.68) --
	(241.12, 66.69) --
	(241.12, 66.69) --
	(241.12, 66.69) --
	(241.19, 66.68) --
	(241.19, 66.68) --
	(241.19, 66.68) --
	(241.26, 66.69) --
	(241.26, 66.69) --
	(241.26, 66.69) --
	(241.33, 66.68) --
	(241.33, 66.68) --
	(241.33, 66.68) --
	(241.40, 66.68) --
	(241.40, 66.68) --
	(241.40, 66.68) --
	(241.47, 66.68) --
	(241.47, 66.68) --
	(241.47, 66.68) --
	(241.54, 66.68) --
	(241.54, 66.68) --
	(241.54, 66.68) --
	(241.61, 66.69) --
	(241.61, 66.69) --
	(241.61, 66.69) --
	(241.68, 66.70) --
	(241.68, 66.70) --
	(241.68, 66.70) --
	(241.75, 66.69) --
	(241.75, 66.69) --
	(241.75, 66.69) --
	(241.82, 66.69) --
	(241.82, 66.69) --
	(241.82, 66.69) --
	(241.89, 66.68) --
	(241.89, 66.68) --
	(241.89, 66.68) --
	(241.96, 66.69) --
	(241.96, 66.69) --
	(241.96, 66.69) --
	(242.03, 66.68) --
	(242.03, 66.68) --
	(242.03, 66.68) --
	(242.09, 66.68) --
	(242.09, 66.68) --
	(242.09, 66.68) --
	(242.16, 66.66) --
	(242.16, 66.66) --
	(242.16, 66.66) --
	(242.23, 66.68) --
	(242.23, 66.68) --
	(242.23, 66.68) --
	(242.30, 66.68) --
	(242.30, 66.68) --
	(242.30, 66.68) --
	(242.37, 66.67) --
	(242.37, 66.67) --
	(242.37, 66.67) --
	(242.44, 66.67) --
	(242.44, 66.67) --
	(242.44, 66.67) --
	(242.51, 66.67) --
	(242.51, 66.67) --
	(242.51, 66.67) --
	(242.58, 66.69) --
	(242.58, 66.69) --
	(242.58, 66.69) --
	(242.65, 66.72) --
	(242.65, 66.72) --
	(242.65, 66.72) --
	(242.72, 66.74) --
	(242.72, 66.74) --
	(242.72, 66.74) --
	(242.79, 66.77) --
	(242.79, 66.77) --
	(242.79, 66.77) --
	(242.86, 66.80) --
	(242.86, 66.80) --
	(242.86, 66.80) --
	(242.93, 66.80) --
	(242.93, 66.80) --
	(242.93, 66.80) --
	(243.00, 66.81) --
	(243.00, 66.81) --
	(243.00, 66.81) --
	(243.07, 66.82) --
	(243.07, 66.82) --
	(243.07, 66.82) --
	(243.14, 66.82) --
	(243.14, 66.82) --
	(243.14, 66.82) --
	(243.21, 66.83) --
	(243.21, 66.83) --
	(243.21, 66.83) --
	(243.28, 66.83) --
	(243.28, 66.83) --
	(243.28, 66.83) --
	(243.35, 66.84) --
	(243.35, 66.84) --
	(243.35, 66.84) --
	(243.42, 66.84) --
	(243.42, 66.84) --
	(243.42, 66.84) --
	(243.49, 66.86) --
	(243.49, 66.86) --
	(243.49, 66.86) --
	(243.56, 66.87) --
	(243.56, 66.87) --
	(243.56, 66.87) --
	(243.63, 66.89) --
	(243.63, 66.89) --
	(243.63, 66.89) --
	(243.70, 66.91) --
	(243.70, 66.91) --
	(243.70, 66.91) --
	(243.77, 66.92) --
	(243.77, 66.92) --
	(243.77, 66.92) --
	(243.84, 66.94) --
	(243.84, 66.94) --
	(243.84, 66.94) --
	(243.90, 66.96) --
	(243.91, 66.96) --
	(243.91, 66.96) --
	(243.98, 66.98) --
	(243.98, 66.98) --
	(243.98, 66.98) --
	(244.05, 67.00) --
	(244.05, 67.00) --
	(244.05, 67.00) --
	(244.11, 67.00) --
	(244.11, 67.00) --
	(244.11, 67.00) --
	(244.18, 67.04) --
	(244.18, 67.04) --
	(244.18, 67.04) --
	(244.25, 67.08) --
	(244.25, 67.08) --
	(244.25, 67.08) --
	(244.32, 67.11) --
	(244.32, 67.11) --
	(244.32, 67.11) --
	(244.39, 67.13) --
	(244.39, 67.13) --
	(244.39, 67.13) --
	(244.46, 67.16) --
	(244.46, 67.16) --
	(244.46, 67.16) --
	(244.53, 67.18) --
	(244.53, 67.18) --
	(244.53, 67.18) --
	(244.60, 67.20) --
	(244.60, 67.20) --
	(244.60, 67.20) --
	(244.67, 67.22) --
	(244.67, 67.22) --
	(244.67, 67.22) --
	(244.74, 67.25) --
	(244.74, 67.25) --
	(244.74, 67.25) --
	(244.81, 67.30) --
	(244.81, 67.30) --
	(244.81, 67.30) --
	(244.88, 67.37) --
	(244.88, 67.37) --
	(244.88, 67.37) --
	(244.95, 67.48) --
	(244.95, 67.48) --
	(244.95, 67.48) --
	(245.02, 67.62) --
	(245.02, 67.62) --
	(245.02, 67.62) --
	(245.09, 67.84) --
	(245.09, 67.84) --
	(245.09, 67.84) --
	(245.16, 68.20) --
	(245.16, 68.20) --
	(245.16, 68.20) --
	(245.23, 68.74) --
	(245.23, 68.74) --
	(245.23, 68.74) --
	(245.30, 69.71) --
	(245.30, 69.71) --
	(245.30, 69.71) --
	(245.37, 71.46) --
	(245.37, 71.46) --
	(245.37, 71.46) --
	(245.44, 73.83) --
	(245.44, 73.83) --
	(245.44, 73.83) --
	(245.51, 75.56) --
	(245.51, 75.56) --
	(245.51, 75.56) --
	(245.57, 76.33) --
	(245.57, 76.33) --
	(245.57, 76.33) --
	(245.64, 76.13) --
	(245.64, 76.13) --
	(245.64, 76.13) --
	(245.71, 75.37) --
	(245.71, 75.37) --
	(245.71, 75.37) --
	(245.78, 74.34) --
	(245.78, 74.34) --
	(245.78, 74.34) --
	(245.85, 73.11) --
	(245.85, 73.11) --
	(245.85, 73.11) --
	(245.92, 71.87) --
	(245.92, 71.87) --
	(245.92, 71.87) --
	(245.99, 70.85) --
	(245.99, 70.85) --
	(245.99, 70.85) --
	(246.06, 70.08) --
	(246.06, 70.08) --
	(246.06, 70.08) --
	(246.13, 69.51) --
	(246.13, 69.51) --
	(246.13, 69.51) --
	(246.20, 69.07) --
	(246.20, 69.07) --
	(246.20, 69.07) --
	(246.27, 68.73) --
	(246.27, 68.73) --
	(246.27, 68.73) --
	(246.34, 68.47) --
	(246.34, 68.47) --
	(246.34, 68.47) --
	(246.41, 68.20) --
	(246.41, 68.20) --
	(246.41, 68.20) --
	(246.48, 67.99) --
	(246.48, 67.99) --
	(246.48, 67.99) --
	(246.55, 67.83) --
	(246.55, 67.83) --
	(246.55, 67.83) --
	(246.61, 67.67) --
	(246.61, 67.67) --
	(246.61, 67.67) --
	(246.68, 67.51) --
	(246.68, 67.51) --
	(246.68, 67.51) --
	(246.75, 67.43) --
	(246.75, 67.43) --
	(246.75, 67.43) --
	(246.82, 67.36) --
	(246.82, 67.36) --
	(246.82, 67.36) --
	(246.89, 67.34) --
	(246.89, 67.34) --
	(246.89, 67.34) --
	(246.96, 67.32) --
	(246.96, 67.32) --
	(246.96, 67.32) --
	(247.03, 67.33) --
	(247.03, 67.33) --
	(247.03, 67.33) --
	(247.10, 67.31) --
	(247.10, 67.31) --
	(247.10, 67.31) --
	(247.17, 67.34) --
	(247.17, 67.34) --
	(247.17, 67.34) --
	(247.24, 67.41) --
	(247.24, 67.41) --
	(247.24, 67.41) --
	(247.31, 67.54) --
	(247.31, 67.54) --
	(247.31, 67.54) --
	(247.38, 67.72) --
	(247.38, 67.72) --
	(247.38, 67.72) --
	(247.45, 68.03) --
	(247.45, 68.03) --
	(247.45, 68.03) --
	(247.52, 68.64) --
	(247.52, 68.64) --
	(247.52, 68.64) --
	(247.59, 69.43) --
	(247.59, 69.43) --
	(247.59, 69.43) --
	(247.65, 70.01) --
	(247.65, 70.01) --
	(247.65, 70.01) --
	(247.72, 70.01) --
	(247.72, 70.01) --
	(247.72, 70.01) --
	(247.79, 69.78) --
	(247.79, 69.78) --
	(247.79, 69.78) --
	(247.86, 69.39) --
	(247.86, 69.39) --
	(247.86, 69.39) --
	(247.93, 68.97) --
	(247.93, 68.97) --
	(247.93, 68.97) --
	(248.00, 68.51) --
	(248.00, 68.51) --
	(248.00, 68.51) --
	(248.07, 68.12) --
	(248.07, 68.12) --
	(248.07, 68.12) --
	(248.14, 67.83) --
	(248.14, 67.83) --
	(248.14, 67.83) --
	(248.21, 67.61) --
	(248.21, 67.61) --
	(248.21, 67.61) --
	(248.28, 67.46) --
	(248.28, 67.46) --
	(248.28, 67.46) --
	(248.35, 67.34) --
	(248.35, 67.34) --
	(248.35, 67.34) --
	(248.42, 67.24) --
	(248.42, 67.24) --
	(248.42, 67.24) --
	(248.49, 67.13) --
	(248.49, 67.13) --
	(248.49, 67.13) --
	(248.56, 67.04) --
	(248.56, 67.04) --
	(248.56, 67.04) --
	(248.63, 66.97) --
	(248.63, 66.97) --
	(248.63, 66.97) --
	(248.69, 66.93) --
	(248.69, 66.93) --
	(248.69, 66.93) --
	(248.76, 66.94) --
	(248.76, 66.94) --
	(248.76, 66.94) --
	(248.83, 66.97) --
	(248.83, 66.97) --
	(248.83, 66.97) --
	(248.90, 67.09) --
	(248.90, 67.09) --
	(248.90, 67.09) --
	(248.97, 67.38) --
	(248.97, 67.38) --
	(248.97, 67.38) --
	(249.04, 68.13) --
	(249.04, 68.13) --
	(249.04, 68.13) --
	(249.11, 69.21) --
	(249.11, 69.21) --
	(249.11, 69.21) --
	(249.18, 69.88) --
	(249.18, 69.88) --
	(249.18, 69.88) --
	(249.25, 69.78) --
	(249.25, 69.78) --
	(249.25, 69.78) --
	(249.32, 69.46) --
	(249.32, 69.46) --
	(249.32, 69.46) --
	(249.39, 69.04) --
	(249.39, 69.04) --
	(249.39, 69.04) --
	(249.46, 68.55) --
	(249.46, 68.55) --
	(249.46, 68.55) --
	(249.52, 68.01) --
	(249.52, 68.01) --
	(249.52, 68.01) --
	(249.59, 67.58) --
	(249.59, 67.58) --
	(249.59, 67.58) --
	(249.66, 67.25) --
	(249.66, 67.25) --
	(249.66, 67.25) --
	(249.73, 67.01) --
	(249.73, 67.01) --
	(249.73, 67.01) --
	(249.80, 66.84) --
	(249.80, 66.84) --
	(249.80, 66.84) --
	(249.87, 66.73) --
	(249.87, 66.73) --
	(249.87, 66.73) --
	(249.94, 66.65) --
	(249.94, 66.65) --
	(249.94, 66.65) --
	(250.01, 66.59) --
	(250.01, 66.59) --
	(250.01, 66.59) --
	(250.08, 66.55) --
	(250.08, 66.55) --
	(250.08, 66.55) --
	(250.15, 66.53) --
	(250.15, 66.53) --
	(250.15, 66.53) --
	(250.22, 66.52) --
	(250.22, 66.52) --
	(250.22, 66.52) --
	(250.28, 66.53) --
	(250.28, 66.53) --
	(250.28, 66.53) --
	(250.35, 66.54) --
	(250.35, 66.54) --
	(250.35, 66.54) --
	(250.42, 66.56) --
	(250.42, 66.56) --
	(250.42, 66.56) --
	(250.49, 66.63) --
	(250.49, 66.63) --
	(250.49, 66.63) --
	(250.56, 66.69) --
	(250.56, 66.69) --
	(250.56, 66.69) --
	(250.63, 66.79) --
	(250.63, 66.79) --
	(250.63, 66.79) --
	(250.70, 66.99) --
	(250.70, 66.99) --
	(250.70, 66.99) --
	(250.77, 67.30) --
	(250.77, 67.30) --
	(250.77, 67.30) --
	(250.84, 67.77) --
	(250.84, 67.77) --
	(250.84, 67.77) --
	(250.91, 68.49) --
	(250.91, 68.49) --
	(250.91, 68.49) --
	(250.97, 69.35) --
	(250.97, 69.35) --
	(250.97, 69.35) --
	(251.04, 69.90) --
	(251.04, 69.90) --
	(251.04, 69.90) --
	(251.11, 70.03) --
	(251.11, 70.03) --
	(251.11, 70.03) --
	(251.18, 69.84) --
	(251.18, 69.84) --
	(251.18, 69.84) --
	(251.25, 69.56) --
	(251.25, 69.56) --
	(251.25, 69.56) --
	(251.32, 69.11) --
	(251.32, 69.11) --
	(251.32, 69.11) --
	(251.39, 68.61) --
	(251.39, 68.61) --
	(251.39, 68.61) --
	(251.46, 68.14) --
	(251.46, 68.14) --
	(251.46, 68.14) --
	(251.53, 67.75) --
	(251.53, 67.75) --
	(251.53, 67.75) --
	(251.60, 67.45) --
	(251.60, 67.45) --
	(251.60, 67.45) --
	(251.66, 67.25) --
	(251.66, 67.25) --
	(251.66, 67.25) --
	(251.73, 67.10) --
	(251.73, 67.10) --
	(251.73, 67.10) --
	(251.80, 66.99) --
	(251.80, 66.99) --
	(251.80, 66.99) --
	(251.87, 66.90) --
	(251.87, 66.90) --
	(251.87, 66.90) --
	(251.94, 66.80) --
	(251.94, 66.80) --
	(251.94, 66.80) --
	(252.01, 66.73) --
	(252.01, 66.73) --
	(252.01, 66.73) --
	(252.08, 66.66) --
	(252.08, 66.66) --
	(252.08, 66.66) --
	(252.15, 66.58) --
	(252.15, 66.58) --
	(252.15, 66.58) --
	(252.22, 66.51) --
	(252.22, 66.51) --
	(252.22, 66.51) --
	(252.29, 66.47) --
	(252.29, 66.47) --
	(252.29, 66.47) --
	(252.35, 66.41) --
	(252.35, 66.41) --
	(252.35, 66.41) --
	(252.42, 66.36) --
	(252.42, 66.36) --
	(252.42, 66.36) --
	(252.49, 66.33) --
	(252.49, 66.33) --
	(252.49, 66.33) --
	(252.56, 66.30) --
	(252.56, 66.30) --
	(252.56, 66.30) --
	(252.63, 66.27) --
	(252.63, 66.27) --
	(252.63, 66.27) --
	(252.70, 66.24) --
	(252.70, 66.24) --
	(252.70, 66.24) --
	(252.77, 66.21) --
	(252.77, 66.21) --
	(252.77, 66.21) --
	(252.84, 66.20) --
	(252.84, 66.20) --
	(252.84, 66.20) --
	(252.91, 66.19) --
	(252.91, 66.19) --
	(252.91, 66.19) --
	(252.97, 66.17) --
	(252.97, 66.17) --
	(252.97, 66.17) --
	(253.04, 66.16) --
	(253.04, 66.16) --
	(253.04, 66.16) --
	(253.11, 66.16) --
	(253.11, 66.16) --
	(253.11, 66.16) --
	(253.18, 66.16) --
	(253.18, 66.16) --
	(253.18, 66.16) --
	(253.25, 66.16) --
	(253.25, 66.16) --
	(253.25, 66.16) --
	(253.32, 66.17) --
	(253.32, 66.17) --
	(253.32, 66.17) --
	(253.39, 66.17) --
	(253.39, 66.17) --
	(253.39, 66.17) --
	(253.46, 66.17) --
	(253.46, 66.17) --
	(253.46, 66.17) --
	(253.53, 66.18) --
	(253.53, 66.18) --
	(253.53, 66.18) --
	(253.59, 66.17) --
	(253.59, 66.17) --
	(253.59, 66.17) --
	(253.66, 66.17) --
	(253.66, 66.17) --
	(253.66, 66.17) --
	(253.73, 66.15) --
	(253.73, 66.15) --
	(253.73, 66.15) --
	(253.80, 66.14) --
	(253.80, 66.14) --
	(253.80, 66.14) --
	(253.87, 66.13) --
	(253.87, 66.13) --
	(253.87, 66.13) --
	(253.94, 66.13) --
	(253.94, 66.13) --
	(253.94, 66.13) --
	(254.01, 66.12) --
	(254.01, 66.12) --
	(254.01, 66.12) --
	(254.08, 66.13) --
	(254.08, 66.13) --
	(254.08, 66.13) --
	(254.14, 66.12) --
	(254.14, 66.12) --
	(254.14, 66.12) --
	(254.21, 66.14) --
	(254.21, 66.14) --
	(254.21, 66.14) --
	(254.28, 66.15) --
	(254.28, 66.15) --
	(254.28, 66.15) --
	(254.35, 66.19) --
	(254.35, 66.19) --
	(254.35, 66.19) --
	(254.42, 66.24) --
	(254.42, 66.24) --
	(254.42, 66.24) --
	(254.49, 66.30) --
	(254.49, 66.30) --
	(254.49, 66.30) --
	(254.56, 66.35) --
	(254.56, 66.35) --
	(254.56, 66.35) --
	(254.63, 66.36) --
	(254.63, 66.36) --
	(254.63, 66.36) --
	(254.70, 66.36) --
	(254.70, 66.36) --
	(254.70, 66.36) --
	(254.76, 66.34) --
	(254.76, 66.34) --
	(254.76, 66.34) --
	(254.83, 66.31) --
	(254.83, 66.31) --
	(254.83, 66.31) --
	(254.90, 66.28) --
	(254.90, 66.28) --
	(254.90, 66.28) --
	(254.97, 66.25) --
	(254.97, 66.25) --
	(254.97, 66.25) --
	(255.04, 66.22) --
	(255.04, 66.22) --
	(255.04, 66.22) --
	(255.11, 66.19) --
	(255.11, 66.19) --
	(255.11, 66.19) --
	(255.18, 66.16) --
	(255.18, 66.16) --
	(255.18, 66.16) --
	(255.24, 66.15) --
	(255.24, 66.15) --
	(255.24, 66.15) --
	(255.31, 66.12) --
	(255.31, 66.12) --
	(255.31, 66.12) --
	(255.38, 66.10) --
	(255.38, 66.10) --
	(255.38, 66.10) --
	(255.45, 66.09) --
	(255.45, 66.09) --
	(255.45, 66.09) --
	(255.52, 66.10) --
	(255.52, 66.10) --
	(255.52, 66.10) --
	(255.59, 66.09) --
	(255.59, 66.09) --
	(255.59, 66.09) --
	(255.66, 66.11) --
	(255.66, 66.11) --
	(255.66, 66.11) --
	(255.73, 66.11) --
	(255.73, 66.11) --
	(255.73, 66.11) --
	(255.79, 66.11) --
	(255.79, 66.11) --
	(255.79, 66.11) --
	(255.86, 66.11) --
	(255.86, 66.11) --
	(255.86, 66.11) --
	(255.93, 66.11) --
	(255.93, 66.11) --
	(255.93, 66.11) --
	(256.00, 66.11) --
	(256.00, 66.11) --
	(256.00, 66.11) --
	(256.07, 66.11) --
	(256.07, 66.11) --
	(256.07, 66.11) --
	(256.14, 66.10) --
	(256.14, 66.10) --
	(256.14, 66.10) --
	(256.20, 66.11) --
	(256.20, 66.11) --
	(256.20, 66.11) --
	(256.27, 66.12) --
	(256.27, 66.12) --
	(256.27, 66.12) --
	(256.34, 66.13) --
	(256.34, 66.13) --
	(256.34, 66.13) --
	(256.41, 66.16) --
	(256.41, 66.16) --
	(256.41, 66.16) --
	(256.48, 66.20) --
	(256.48, 66.20) --
	(256.48, 66.20) --
	(256.55, 66.23) --
	(256.55, 66.23) --
	(256.55, 66.23) --
	(256.62, 66.26) --
	(256.62, 66.26) --
	(256.62, 66.26) --
	(256.69, 66.29) --
	(256.69, 66.29) --
	(256.69, 66.29) --
	(256.75, 66.30) --
	(256.75, 66.30) --
	(256.75, 66.30) --
	(256.82, 66.32) --
	(256.82, 66.32) --
	(256.82, 66.32) --
	(256.89, 66.28) --
	(256.89, 66.28) --
	(256.89, 66.28) --
	(256.96, 66.25) --
	(256.96, 66.25) --
	(256.96, 66.25) --
	(257.03, 66.23) --
	(257.03, 66.23) --
	(257.03, 66.23) --
	(257.10, 66.19) --
	(257.10, 66.19) --
	(257.10, 66.19) --
	(257.17, 66.16) --
	(257.17, 66.16) --
	(257.17, 66.16) --
	(257.24, 66.12) --
	(257.24, 66.12) --
	(257.24, 66.12) --
	(257.30, 66.09) --
	(257.30, 66.09) --
	(257.30, 66.09) --
	(257.37, 66.06) --
	(257.37, 66.06) --
	(257.37, 66.06) --
	(257.44, 66.04) --
	(257.44, 66.04) --
	(257.44, 66.04) --
	(257.51, 66.03) --
	(257.51, 66.03) --
	(257.51, 66.03) --
	(257.58, 66.00) --
	(257.58, 66.00) --
	(257.58, 66.00) --
	(257.65, 65.98) --
	(257.65, 65.98) --
	(257.65, 65.98) --
	(257.71, 65.98) --
	(257.71, 65.98) --
	(257.71, 65.98) --
	(257.78, 65.96) --
	(257.78, 65.96) --
	(257.78, 65.96) --
	(257.85, 65.96) --
	(257.85, 65.96) --
	(257.85, 65.96) --
	(257.92, 65.95) --
	(257.92, 65.95) --
	(257.92, 65.95) --
	(257.99, 65.94) --
	(257.99, 65.94) --
	(257.99, 65.94) --
	(258.06, 65.93) --
	(258.06, 65.93) --
	(258.06, 65.93) --
	(258.12, 65.92) --
	(258.12, 65.92) --
	(258.12, 65.92) --
	(258.19, 65.93) --
	(258.19, 65.93) --
	(258.19, 65.93) --
	(258.26, 65.92) --
	(258.26, 65.92) --
	(258.26, 65.92) --
	(258.33, 65.91) --
	(258.33, 65.91) --
	(258.33, 65.91) --
	(258.40, 65.91) --
	(258.40, 65.91) --
	(258.40, 65.91) --
	(258.47, 65.91) --
	(258.47, 65.91) --
	(258.47, 65.91) --
	(258.54, 65.90) --
	(258.54, 65.90) --
	(258.54, 65.90) --
	(258.60, 65.90) --
	(258.60, 65.90) --
	(258.60, 65.90) --
	(258.67, 65.90) --
	(258.67, 65.90) --
	(258.67, 65.90) --
	(258.74, 65.91) --
	(258.74, 65.91) --
	(258.74, 65.91) --
	(258.81, 65.90) --
	(258.81, 65.90) --
	(258.81, 65.90) --
	(258.88, 65.90) --
	(258.88, 65.90) --
	(258.88, 65.90) --
	(258.95, 65.91) --
	(258.95, 65.91) --
	(258.95, 65.91) --
	(259.01, 65.91) --
	(259.01, 65.91) --
	(259.01, 65.91) --
	(259.08, 65.91) --
	(259.08, 65.91) --
	(259.08, 65.91) --
	(259.15, 65.91) --
	(259.15, 65.91) --
	(259.15, 65.91) --
	(259.22, 65.93) --
	(259.22, 65.93) --
	(259.22, 65.93) --
	(259.29, 65.95) --
	(259.29, 65.95) --
	(259.29, 65.95) --
	(259.36, 65.95) --
	(259.36, 65.95) --
	(259.36, 65.95) --
	(259.42, 65.97) --
	(259.42, 65.97) --
	(259.42, 65.97) --
	(259.49, 65.98) --
	(259.49, 65.98) --
	(259.49, 65.98) --
	(259.56, 65.99) --
	(259.56, 65.99) --
	(259.56, 65.99) --
	(259.63, 66.00) --
	(259.63, 66.00) --
	(259.63, 66.00) --
	(259.70, 66.01) --
	(259.70, 66.01) --
	(259.70, 66.01) --
	(259.77, 66.01) --
	(259.77, 66.01) --
	(259.77, 66.01) --
	(259.83, 66.02) --
	(259.83, 66.02) --
	(259.83, 66.02) --
	(259.90, 66.01) --
	(259.90, 66.01) --
	(259.90, 66.01) --
	(259.97, 66.01) --
	(259.97, 66.01) --
	(259.97, 66.01) --
	(260.04, 66.02) --
	(260.04, 66.02) --
	(260.04, 66.02) --
	(260.11, 66.01) --
	(260.11, 66.01) --
	(260.11, 66.01) --
	(260.18, 66.01) --
	(260.18, 66.01) --
	(260.18, 66.01) --
	(260.25, 66.01) --
	(260.25, 66.01) --
	(260.25, 66.01) --
	(260.31, 65.99) --
	(260.31, 65.99) --
	(260.31, 65.99) --
	(260.38, 65.99) --
	(260.38, 65.99) --
	(260.38, 65.99) --
	(260.45, 65.99) --
	(260.45, 65.99) --
	(260.45, 65.99) --
	(260.52, 65.98) --
	(260.52, 65.98) --
	(260.52, 65.98) --
	(260.59, 65.97) --
	(260.59, 65.97) --
	(260.59, 65.97) --
	(260.66, 65.97) --
	(260.66, 65.97) --
	(260.66, 65.97) --
	(260.72, 65.98) --
	(260.72, 65.98) --
	(260.72, 65.98) --
	(260.79, 65.99) --
	(260.79, 65.99) --
	(260.79, 65.99) --
	(260.86, 66.01) --
	(260.86, 66.01) --
	(260.86, 66.01) --
	(260.93, 66.04) --
	(260.93, 66.04) --
	(260.93, 66.04) --
	(261.00, 66.09) --
	(261.00, 66.09) --
	(261.00, 66.09) --
	(261.06, 66.17) --
	(261.06, 66.17) --
	(261.06, 66.17) --
	(261.13, 66.30) --
	(261.13, 66.30) --
	(261.13, 66.30) --
	(261.20, 66.51) --
	(261.20, 66.51) --
	(261.20, 66.51) --
	(261.27, 66.70) --
	(261.27, 66.70) --
	(261.27, 66.70) --
	(261.34, 66.82) --
	(261.34, 66.82) --
	(261.34, 66.82) --
	(261.40, 66.80) --
	(261.40, 66.80) --
	(261.40, 66.80) --
	(261.47, 66.74) --
	(261.47, 66.74) --
	(261.47, 66.74) --
	(261.54, 66.65) --
	(261.54, 66.65) --
	(261.54, 66.65) --
	(261.61, 66.53) --
	(261.61, 66.53) --
	(261.61, 66.53) --
	(261.68, 66.40) --
	(261.68, 66.40) --
	(261.68, 66.40) --
	(261.75, 66.27) --
	(261.75, 66.27) --
	(261.75, 66.27) --
	(261.81, 66.16) --
	(261.81, 66.16) --
	(261.81, 66.16) --
	(261.88, 66.07) --
	(261.88, 66.07) --
	(261.88, 66.07) --
	(261.95, 66.01) --
	(261.95, 66.01) --
	(261.95, 66.01) --
	(262.02, 65.97) --
	(262.02, 65.97) --
	(262.02, 65.97) --
	(262.09, 65.94) --
	(262.09, 65.94) --
	(262.09, 65.94) --
	(262.16, 65.91) --
	(262.16, 65.91) --
	(262.16, 65.91) --
	(262.22, 65.91) --
	(262.22, 65.91) --
	(262.22, 65.91) --
	(262.29, 65.91) --
	(262.29, 65.91) --
	(262.29, 65.91) --
	(262.36, 65.90) --
	(262.36, 65.90) --
	(262.36, 65.90) --
	(262.43, 65.89) --
	(262.43, 65.89) --
	(262.43, 65.89) --
	(262.50, 65.89) --
	(262.50, 65.89) --
	(262.50, 65.89) --
	(262.56, 65.91) --
	(262.56, 65.91) --
	(262.56, 65.91) --
	(262.63, 65.90) --
	(262.63, 65.90) --
	(262.63, 65.90) --
	(262.70, 65.92) --
	(262.70, 65.92) --
	(262.70, 65.92) --
	(262.77, 65.94) --
	(262.77, 65.94) --
	(262.77, 65.94) --
	(262.84, 65.96) --
	(262.84, 65.96) --
	(262.84, 65.96) --
	(262.90, 66.00) --
	(262.90, 66.00) --
	(262.90, 66.00) --
	(262.97, 66.03) --
	(262.97, 66.03) --
	(262.97, 66.03) --
	(263.04, 66.11) --
	(263.04, 66.11) --
	(263.04, 66.11) --
	(263.11, 66.24) --
	(263.11, 66.24) --
	(263.11, 66.24) --
	(263.18, 66.40) --
	(263.18, 66.40) --
	(263.18, 66.40) --
	(263.25, 66.56) --
	(263.25, 66.56) --
	(263.25, 66.56) --
	(263.31, 66.69) --
	(263.31, 66.69) --
	(263.31, 66.69) --
	(263.38, 66.73) --
	(263.38, 66.73) --
	(263.38, 66.73) --
	(263.45, 66.72) --
	(263.45, 66.72) --
	(263.45, 66.72) --
	(263.52, 66.67) --
	(263.52, 66.67) --
	(263.52, 66.67) --
	(263.58, 66.59) --
	(263.58, 66.59) --
	(263.58, 66.59) --
	(263.65, 66.49) --
	(263.65, 66.49) --
	(263.65, 66.49) --
	(263.72, 66.37) --
	(263.72, 66.37) --
	(263.72, 66.37) --
	(263.79, 66.27) --
	(263.79, 66.27) --
	(263.79, 66.27) --
	(263.86, 66.17) --
	(263.86, 66.17) --
	(263.86, 66.17) --
	(263.93, 66.11) --
	(263.93, 66.11) --
	(263.93, 66.11) --
	(263.99, 66.05) --
	(263.99, 66.05) --
	(263.99, 66.05) --
	(264.06, 66.00) --
	(264.06, 66.00) --
	(264.06, 66.00) --
	(264.13, 65.97) --
	(264.13, 65.97) --
	(264.13, 65.97) --
	(264.20, 65.95) --
	(264.20, 65.95) --
	(264.20, 65.95) --
	(264.26, 65.92) --
	(264.26, 65.92) --
	(264.26, 65.92) --
	(264.33, 65.91) --
	(264.33, 65.91) --
	(264.33, 65.91) --
	(264.40, 65.89) --
	(264.40, 65.89) --
	(264.40, 65.89) --
	(264.47, 65.87) --
	(264.47, 65.87) --
	(264.47, 65.87) --
	(264.54, 65.87) --
	(264.54, 65.87) --
	(264.54, 65.87) --
	(264.61, 65.86) --
	(264.61, 65.86) --
	(264.61, 65.86) --
	(264.67, 65.86) --
	(264.67, 65.86) --
	(264.67, 65.86) --
	(264.74, 65.85) --
	(264.74, 65.85) --
	(264.74, 65.85) --
	(264.81, 65.85) --
	(264.81, 65.85) --
	(264.81, 65.85) --
	(264.88, 65.85) --
	(264.88, 65.85) --
	(264.88, 65.85) --
	(264.94, 65.84) --
	(264.94, 65.84) --
	(264.94, 65.84) --
	(265.01, 65.84) --
	(265.01, 65.84) --
	(265.01, 65.84) --
	(265.08, 65.84) --
	(265.08, 65.84) --
	(265.08, 65.84) --
	(265.15, 65.84) --
	(265.15, 65.84) --
	(265.15, 65.84) --
	(265.22, 65.83) --
	(265.22, 65.83) --
	(265.22, 65.83) --
	(265.28, 65.83) --
	(265.28, 65.83) --
	(265.28, 65.83) --
	(265.35, 65.83) --
	(265.35, 65.83) --
	(265.35, 65.83) --
	(265.42, 65.82) --
	(265.42, 65.82) --
	(265.42, 65.82) --
	(265.49, 65.82) --
	(265.49, 65.82) --
	(265.49, 65.82) --
	(265.56, 65.82) --
	(265.56, 65.82) --
	(265.56, 65.82) --
	(265.62, 65.81) --
	(265.62, 65.81) --
	(265.62, 65.81) --
	(265.69, 65.82) --
	(265.69, 65.82) --
	(265.69, 65.82) --
	(265.76, 65.81) --
	(265.76, 65.81) --
	(265.76, 65.81) --
	(265.83, 65.81) --
	(265.83, 65.81) --
	(265.83, 65.81) --
	(265.90, 65.81) --
	(265.90, 65.81) --
	(265.90, 65.81) --
	(265.96, 65.81) --
	(265.96, 65.81) --
	(265.96, 65.81) --
	(266.03, 65.81) --
	(266.03, 65.81) --
	(266.03, 65.81) --
	(266.10, 65.83) --
	(266.10, 65.83) --
	(266.10, 65.83) --
	(266.17, 65.83) --
	(266.17, 65.83) --
	(266.17, 65.83) --
	(266.23, 65.83) --
	(266.23, 65.83) --
	(266.23, 65.83) --
	(266.30, 65.84) --
	(266.30, 65.84) --
	(266.30, 65.84) --
	(266.37, 65.85) --
	(266.37, 65.85) --
	(266.37, 65.85) --
	(266.44, 65.85) --
	(266.44, 65.85) --
	(266.44, 65.85) --
	(266.51, 65.85) --
	(266.51, 65.85) --
	(266.51, 65.85) --
	(266.57, 65.85) --
	(266.57, 65.85) --
	(266.57, 65.85) --
	(266.64, 65.86) --
	(266.64, 65.86) --
	(266.64, 65.86) --
	(266.71, 65.85) --
	(266.71, 65.85) --
	(266.71, 65.85) --
	(266.78, 65.84) --
	(266.78, 65.84) --
	(266.78, 65.84) --
	(266.84, 65.84) --
	(266.84, 65.84) --
	(266.84, 65.84) --
	(266.91, 65.82) --
	(266.91, 65.82) --
	(266.91, 65.82) --
	(266.98, 65.82) --
	(266.98, 65.82) --
	(266.98, 65.82) --
	(267.05, 65.81) --
	(267.05, 65.81) --
	(267.05, 65.81) --
	(267.11, 65.81) --
	(267.11, 65.81) --
	(267.11, 65.81) --
	(267.18, 65.80) --
	(267.18, 65.80) --
	(267.18, 65.80) --
	(267.25, 65.79) --
	(267.25, 65.79) --
	(267.25, 65.79) --
	(267.32, 65.79) --
	(267.32, 65.79) --
	(267.32, 65.79) --
	(267.39, 65.79) --
	(267.39, 65.79) --
	(267.39, 65.79) --
	(267.45, 65.79) --
	(267.45, 65.79) --
	(267.45, 65.79) --
	(267.52, 65.78) --
	(267.52, 65.78) --
	(267.52, 65.78) --
	(267.59, 65.79) --
	(267.59, 65.79) --
	(267.59, 65.79) --
	(267.66, 65.78) --
	(267.66, 65.78) --
	(267.66, 65.78) --
	(267.72, 65.78) --
	(267.72, 65.78) --
	(267.72, 65.78) --
	(267.79, 65.78) --
	(267.79, 65.78) --
	(267.79, 65.78) --
	(267.86, 65.77) --
	(267.86, 65.77) --
	(267.86, 65.77) --
	(267.93, 65.77) --
	(267.93, 65.77) --
	(267.93, 65.77) --
	(267.99, 65.76) --
	(267.99, 65.76) --
	(267.99, 65.76) --
	(268.06, 65.77) --
	(268.06, 65.77) --
	(268.06, 65.77) --
	(268.13, 65.76) --
	(268.13, 65.76) --
	(268.13, 65.76) --
	(268.20, 65.77) --
	(268.20, 65.77) --
	(268.20, 65.77) --
	(268.27, 65.76) --
	(268.27, 65.76) --
	(268.27, 65.76) --
	(268.33, 65.76) --
	(268.33, 65.76) --
	(268.33, 65.76) --
	(268.40, 65.76) --
	(268.40, 65.76) --
	(268.40, 65.76) --
	(268.47, 65.76) --
	(268.47, 65.76) --
	(268.47, 65.76) --
	(268.54, 65.76) --
	(268.54, 65.76) --
	(268.54, 65.76) --
	(268.60, 65.76) --
	(268.60, 65.76) --
	(268.60, 65.76) --
	(268.67, 65.76) --
	(268.67, 65.76) --
	(268.67, 65.76) --
	(268.74, 65.76) --
	(268.74, 65.76) --
	(268.74, 65.76) --
	(268.81, 65.76) --
	(268.81, 65.76) --
	(268.81, 65.76) --
	(268.87, 65.75) --
	(268.87, 65.75) --
	(268.87, 65.75) --
	(268.94, 65.76) --
	(268.94, 65.76) --
	(268.94, 65.76) --
	(269.01, 65.75) --
	(269.01, 65.75) --
	(269.01, 65.75) --
	(269.08, 65.76) --
	(269.08, 65.76) --
	(269.08, 65.76) --
	(269.14, 65.75) --
	(269.14, 65.75) --
	(269.14, 65.75) --
	(269.21, 65.76) --
	(269.21, 65.76) --
	(269.21, 65.76) --
	(269.28, 65.75) --
	(269.28, 65.75) --
	(269.28, 65.75) --
	(269.35, 65.75) --
	(269.35, 65.75) --
	(269.35, 65.75) --
	(269.42, 65.75) --
	(269.42, 65.75) --
	(269.42, 65.75) --
	(269.48, 65.75) --
	(269.48, 65.75) --
	(269.48, 65.75) --
	(269.55, 65.75) --
	(269.55, 65.75) --
	(269.55, 65.75) --
	(269.62, 65.75) --
	(269.62, 65.75) --
	(269.62, 65.75) --
	(269.69, 65.74) --
	(269.69, 65.74) --
	(269.69, 65.74) --
	(269.75, 65.75) --
	(269.75, 65.75) --
	(269.75, 65.75) --
	(269.82, 65.75) --
	(269.82, 65.75) --
	(269.82, 65.75) --
	(269.89, 65.75) --
	(269.89, 65.75) --
	(269.89, 65.75) --
	(269.95, 65.74) --
	(269.95, 65.74) --
	(269.95, 65.74) --
	(270.02, 65.75) --
	(270.02, 65.75) --
	(270.02, 65.75) --
	(270.09, 65.75) --
	(270.09, 65.75) --
	(270.09, 65.75) --
	(270.16, 65.74) --
	(270.16, 65.74) --
	(270.16, 65.74) --
	(270.22, 65.74) --
	(270.22, 65.74) --
	(270.22, 65.74) --
	(270.29, 65.75) --
	(270.29, 65.75) --
	(270.29, 65.75) --
	(270.36, 65.75) --
	(270.36, 65.75) --
	(270.36, 65.75) --
	(270.43, 65.75) --
	(270.43, 65.75) --
	(270.43, 65.75) --
	(270.49, 65.74) --
	(270.49, 65.74) --
	(270.49, 65.74) --
	(270.56, 65.74) --
	(270.56, 65.74) --
	(270.56, 65.74) --
	(270.63, 65.74) --
	(270.63, 65.74) --
	(270.63, 65.74) --
	(270.70, 65.74) --
	(270.70, 65.74) --
	(270.70, 65.74) --
	(270.76, 65.74) --
	(270.76, 65.74) --
	(270.76, 65.74) --
	(270.83, 65.74) --
	(270.83, 65.74) --
	(270.83, 65.74) --
	(270.90, 65.75) --
	(270.90, 65.75) --
	(270.90, 65.75) --
	(270.97, 65.74) --
	(270.97, 65.74) --
	(270.97, 65.74) --
	(271.03, 65.74) --
	(271.03, 65.74) --
	(271.03, 65.74) --
	(271.10, 65.74) --
	(271.10, 65.74) --
	(271.10, 65.74) --
	(271.17, 65.74) --
	(271.17, 65.74) --
	(271.17, 65.74) --
	(271.24, 65.74) --
	(271.24, 65.74) --
	(271.24, 65.74) --
	(271.30, 65.75) --
	(271.30, 65.75) --
	(271.30, 65.75) --
	(271.37, 65.73) --
	(271.37, 65.73) --
	(271.37, 65.73) --
	(271.44, 65.74) --
	(271.44, 65.74) --
	(271.44, 65.74) --
	(271.51, 65.74) --
	(271.51, 65.74) --
	(271.51, 65.74) --
	(271.57, 65.74) --
	(271.57, 65.74) --
	(271.57, 65.74) --
	(271.64, 65.74) --
	(271.64, 65.74) --
	(271.64, 65.74) --
	(271.71, 65.74) --
	(271.71, 65.74) --
	(271.71, 65.74) --
	(271.77, 65.74) --
	(271.77, 65.74) --
	(271.77, 65.74) --
	(271.84, 65.74) --
	(271.84, 65.74) --
	(271.84, 65.74) --
	(271.91, 65.73) --
	(271.91, 65.73) --
	(271.91, 65.73) --
	(271.98, 65.74) --
	(271.98, 65.74) --
	(271.98, 65.74) --
	(272.04, 65.74) --
	(272.04, 65.74) --
	(272.04, 65.74) --
	(272.11, 65.74) --
	(272.11, 65.74) --
	(272.11, 65.74) --
	(272.18, 65.74) --
	(272.18, 65.74) --
	(272.18, 65.74) --
	(272.25, 65.74) --
	(272.25, 65.74) --
	(272.25, 65.74) --
	(272.31, 65.74) --
	(272.31, 65.74) --
	(272.31, 65.74) --
	(272.38, 65.73) --
	(272.38, 65.73) --
	(272.38, 65.73) --
	(272.45, 65.74) --
	(272.45, 65.74) --
	(272.45, 65.74) --
	(272.52, 65.74) --
	(272.52, 65.74) --
	(272.52, 65.74) --
	(272.58, 65.73) --
	(272.58, 65.73) --
	(272.58, 65.73) --
	(272.65, 65.74) --
	(272.65, 65.74) --
	(272.65, 65.74) --
	(272.72, 65.74) --
	(272.72, 65.74) --
	(272.72, 65.74) --
	(272.79, 65.73) --
	(272.79, 65.73) --
	(272.79, 65.73) --
	(272.85, 65.74) --
	(272.85, 65.74) --
	(272.85, 65.74) --
	(272.92, 65.73) --
	(272.92, 65.73) --
	(272.92, 65.73) --
	(272.99, 65.74) --
	(272.99, 65.74) --
	(272.99, 65.74) --
	(273.05, 65.73) --
	(273.05, 65.73) --
	(273.05, 65.73) --
	(273.12, 65.74) --
	(273.12, 65.74) --
	(273.12, 65.74) --
	(273.19, 65.74) --
	(273.19, 65.74) --
	(273.19, 65.74) --
	(273.26, 65.74) --
	(273.26, 65.74) --
	(273.26, 65.74) --
	(273.32, 65.74) --
	(273.32, 65.74) --
	(273.32, 65.74) --
	(273.39, 65.74) --
	(273.39, 65.74) --
	(273.39, 65.74) --
	(273.46, 65.73) --
	(273.46, 65.73) --
	(273.46, 65.73) --
	(273.53, 65.74) --
	(273.53, 65.74) --
	(273.53, 65.74) --
	(273.59, 65.73) --
	(273.59, 65.73) --
	(273.59, 65.73) --
	(273.66, 65.74) --
	(273.66, 65.74) --
	(273.66, 65.74) --
	(273.73, 65.74) --
	(273.73, 65.74) --
	(273.73, 65.74) --
	(273.79, 65.73) --
	(273.79, 65.73) --
	(273.79, 65.73) --
	(273.86, 65.73) --
	(273.86, 65.73) --
	(273.86, 65.73) --
	(273.93, 65.73) --
	(273.93, 65.73) --
	(273.93, 65.73) --
	(274.00, 65.73) --
	(274.00, 65.73) --
	(274.00, 65.73) --
	(274.06, 65.73) --
	(274.06, 65.73) --
	(274.06, 65.73) --
	(274.13, 65.73) --
	(274.13, 65.73) --
	(274.13, 65.73) --
	(274.20, 65.73) --
	(274.20, 65.73) --
	(274.20, 65.73) --
	(274.26, 65.73) --
	(274.26, 65.73) --
	(274.26, 65.73) --
	(274.33, 65.72) --
	(274.33, 65.72) --
	(274.33, 65.72) --
	(274.40, 65.73) --
	(274.40, 65.73) --
	(274.40, 65.73) --
	(274.46, 65.73) --
	(274.46, 65.73) --
	(274.46, 65.73) --
	(274.53, 65.73) --
	(274.53, 65.73) --
	(274.53, 65.73) --
	(274.60, 65.72) --
	(274.60, 65.72) --
	(274.60, 65.72) --
	(274.67, 65.73) --
	(274.67, 65.73) --
	(274.67, 65.73) --
	(274.73, 65.72) --
	(274.73, 65.72) --
	(274.73, 65.72) --
	(274.80, 65.73) --
	(274.80, 65.73) --
	(274.80, 65.73) --
	(274.87, 65.73) --
	(274.87, 65.73) --
	(274.87, 65.73) --
	(274.93, 65.73) --
	(274.93, 65.73) --
	(274.93, 65.73) --
	(275.00, 65.73) --
	(275.00, 65.73) --
	(275.00, 65.73) --
	(275.07, 65.73) --
	(275.07, 65.73) --
	(275.07, 65.73) --
	(275.14, 65.73) --
	(275.14, 65.73) --
	(275.14, 65.73) --
	(275.20, 65.72) --
	(275.20, 65.72) --
	(275.20, 65.72) --
	(275.27, 65.72) --
	(275.27, 65.72) --
	(275.27, 65.72) --
	(275.34, 65.72) --
	(275.34, 65.72) --
	(275.34, 65.72) --
	(275.40, 65.72) --
	(275.40, 65.72) --
	(275.40, 65.72) --
	(275.47, 65.73) --
	(275.47, 65.73) --
	(275.47, 65.73) --
	(275.54, 65.72) --
	(275.54, 65.72) --
	(275.54, 65.72) --
	(275.60, 65.72) --
	(275.60, 65.72) --
	(275.61, 65.72) --
	(275.67, 65.72) --
	(275.67, 65.72) --
	(275.67, 65.72) --
	(275.74, 65.73) --
	(275.74, 65.73) --
	(275.74, 65.73) --
	(275.81, 65.73) --
	(275.81, 65.73) --
	(275.81, 65.73) --
	(275.87, 65.72) --
	(275.87, 65.72) --
	(275.87, 65.72) --
	(275.94, 65.72) --
	(275.94, 65.72) --
	(275.94, 65.72) --
	(276.01, 65.73) --
	(276.01, 65.73) --
	(276.01, 65.73) --
	(276.07, 65.72) --
	(276.07, 65.72) --
	(276.07, 65.72) --
	(276.14, 65.73) --
	(276.14, 65.73) --
	(276.14, 65.73) --
	(276.21, 65.72) --
	(276.21, 65.72) --
	(276.21, 65.72) --
	(276.28, 65.72) --
	(276.28, 65.72) --
	(276.28, 65.72) --
	(276.34, 65.73) --
	(276.34, 65.73) --
	(276.34, 65.73) --
	(276.41, 65.73) --
	(276.41, 65.73) --
	(276.41, 65.73) --
	(276.48, 65.72) --
	(276.48, 65.72) --
	(276.48, 65.72) --
	(276.54, 65.72) --
	(276.54, 65.72) --
	(276.54, 65.72) --
	(276.61, 65.72) --
	(276.61, 65.72) --
	(276.61, 65.72) --
	(276.68, 65.73) --
	(276.68, 65.73) --
	(276.68, 65.73) --
	(276.74, 65.72) --
	(276.74, 65.72) --
	(276.74, 65.72) --
	(276.81, 65.72) --
	(276.81, 65.72) --
	(276.81, 65.72) --
	(276.88, 65.72) --
	(276.88, 65.72) --
	(276.88, 65.72) --
	(276.94, 65.72) --
	(276.94, 65.72) --
	(276.94, 65.72) --
	(277.01, 65.72) --
	(277.01, 65.72) --
	(277.01, 65.72) --
	(277.08, 65.73) --
	(277.08, 65.73) --
	(277.08, 65.73) --
	(277.15, 65.73) --
	(277.15, 65.73) --
	(277.15, 65.73) --
	(277.21, 65.72) --
	(277.21, 65.72) --
	(277.21, 65.72) --
	(277.28, 65.72) --
	(277.28, 65.72) --
	(277.28, 65.72) --
	(277.35, 65.73) --
	(277.35, 65.73) --
	(277.35, 65.73) --
	(277.41, 65.73) --
	(277.41, 65.73) --
	(277.41, 65.73) --
	(277.48, 65.73) --
	(277.48, 65.73) --
	(277.48, 65.73) --
	(277.55, 65.72) --
	(277.55, 65.72) --
	(277.55, 65.72) --
	(277.61, 65.73) --
	(277.61, 65.73) --
	(277.61, 65.73) --
	(277.68, 65.72) --
	(277.68, 65.72) --
	(277.68, 65.72) --
	(277.75, 65.72) --
	(277.75, 65.72) --
	(277.75, 65.72) --
	(277.81, 65.72) --
	(277.81, 65.72) --
	(277.81, 65.72) --
	(277.88, 65.72) --
	(277.88, 65.72) --
	(277.88, 65.72) --
	(277.95, 65.72) --
	(277.95, 65.72) --
	(277.95, 65.72) --
	(278.01, 65.73) --
	(278.01, 65.73) --
	(278.01, 65.73) --
	(278.08, 65.72) --
	(278.08, 65.72) --
	(278.08, 65.72) --
	(278.15, 65.72) --
	(278.15, 65.72) --
	(278.15, 65.72) --
	(278.21, 65.72) --
	(278.21, 65.72) --
	(278.21, 65.72) --
	(278.28, 65.72) --
	(278.28, 65.72) --
	(278.28, 65.72) --
	(278.35, 65.73) --
	(278.35, 65.73) --
	(278.35, 65.73) --
	(278.41, 65.72) --
	(278.41, 65.72) --
	(278.41, 65.72) --
	(278.48, 65.72) --
	(278.48, 65.72) --
	(278.48, 65.72) --
	(278.55, 65.72) --
	(278.55, 65.72) --
	(278.55, 65.72) --
	(278.61, 65.72) --
	(278.61, 65.72) --
	(278.61, 65.72) --
	(278.68, 65.72) --
	(278.68, 65.72) --
	(278.68, 65.72) --
	(278.75, 65.72) --
	(278.75, 65.72) --
	(278.75, 65.72) --
	(278.82, 65.73) --
	(278.82, 65.73) --
	(278.82, 65.73) --
	(278.88, 65.72) --
	(278.88, 65.72) --
	(278.88, 65.72) --
	(278.95, 65.72) --
	(278.95, 65.72) --
	(278.95, 65.72) --
	(279.02, 65.72) --
	(279.02, 65.72) --
	(279.02, 65.72) --
	(279.08, 65.73) --
	(279.08, 65.73) --
	(279.08, 65.73) --
	(279.15, 65.72) --
	(279.15, 65.72) --
	(279.15, 65.72) --
	(279.22, 65.72) --
	(279.22, 65.72) --
	(279.22, 65.72) --
	(279.28, 65.72) --
	(279.28, 65.72) --
	(279.28, 65.72) --
	(279.35, 65.72) --
	(279.35, 65.72) --
	(279.35, 65.72) --
	(279.42, 65.72) --
	(279.42, 65.72) --
	(279.42, 65.72) --
	(279.48, 65.73) --
	(279.48, 65.73) --
	(279.48, 65.73) --
	(279.55, 65.73) --
	(279.55, 65.73) --
	(279.55, 65.73) --
	(279.62, 65.72) --
	(279.62, 65.72) --
	(279.62, 65.72) --
	(279.68, 65.72) --
	(279.68, 65.72) --
	(279.68, 65.72) --
	(279.75, 65.72) --
	(279.75, 65.72) --
	(279.75, 65.72) --
	(279.82, 65.73) --
	(279.82, 65.73) --
	(279.82, 65.73) --
	(279.88, 65.72) --
	(279.88, 65.72) --
	(279.88, 65.72) --
	(279.95, 65.72) --
	(279.95, 65.72) --
	(279.95, 65.72) --
	(280.02, 65.73) --
	(280.02, 65.73) --
	(280.02, 65.73) --
	(280.08, 65.72) --
	(280.08, 65.72) --
	(280.08, 65.72) --
	(280.15, 65.72) --
	(280.15, 65.72) --
	(280.15, 65.72) --
	(280.22, 65.73) --
	(280.22, 65.73) --
	(280.22, 65.73) --
	(280.28, 65.73) --
	(280.28, 65.73) --
	(280.28, 65.73) --
	(280.35, 65.72) --
	(280.35, 65.72) --
	(280.35, 65.72) --
	(280.42, 65.72) --
	(280.42, 65.72) --
	(280.42, 65.72) --
	(280.48, 65.72) --
	(280.48, 65.72) --
	(280.48, 65.71) --
	(280.48, 65.71) --
	(280.48, 65.71) --
	(280.42, 65.71) --
	(280.42, 65.71) --
	(280.42, 65.71) --
	(280.35, 65.71) --
	(280.35, 65.71) --
	(280.35, 65.71) --
	(280.28, 65.71) --
	(280.28, 65.71) --
	(280.28, 65.71) --
	(280.22, 65.71) --
	(280.22, 65.71) --
	(280.22, 65.71) --
	(280.15, 65.71) --
	(280.15, 65.71) --
	(280.15, 65.71) --
	(280.08, 65.71) --
	(280.08, 65.71) --
	(280.08, 65.71) --
	(280.02, 65.71) --
	(280.02, 65.71) --
	(280.02, 65.71) --
	(279.95, 65.71) --
	(279.95, 65.71) --
	(279.95, 65.71) --
	(279.88, 65.71) --
	(279.88, 65.71) --
	(279.88, 65.71) --
	(279.82, 65.71) --
	(279.82, 65.71) --
	(279.82, 65.71) --
	(279.75, 65.71) --
	(279.75, 65.71) --
	(279.75, 65.71) --
	(279.68, 65.71) --
	(279.68, 65.71) --
	(279.68, 65.71) --
	(279.62, 65.71) --
	(279.62, 65.71) --
	(279.62, 65.71) --
	(279.55, 65.71) --
	(279.55, 65.71) --
	(279.55, 65.71) --
	(279.48, 65.71) --
	(279.48, 65.71) --
	(279.48, 65.71) --
	(279.42, 65.71) --
	(279.42, 65.71) --
	(279.42, 65.71) --
	(279.35, 65.71) --
	(279.35, 65.71) --
	(279.35, 65.71) --
	(279.28, 65.71) --
	(279.28, 65.71) --
	(279.28, 65.71) --
	(279.22, 65.71) --
	(279.22, 65.71) --
	(279.22, 65.71) --
	(279.15, 65.71) --
	(279.15, 65.71) --
	(279.15, 65.71) --
	(279.08, 65.71) --
	(279.08, 65.71) --
	(279.08, 65.71) --
	(279.02, 65.71) --
	(279.02, 65.71) --
	(279.02, 65.71) --
	(278.95, 65.71) --
	(278.95, 65.71) --
	(278.95, 65.71) --
	(278.88, 65.71) --
	(278.88, 65.71) --
	(278.88, 65.71) --
	(278.82, 65.71) --
	(278.82, 65.71) --
	(278.82, 65.71) --
	(278.75, 65.71) --
	(278.75, 65.71) --
	(278.75, 65.71) --
	(278.68, 65.71) --
	(278.68, 65.71) --
	(278.68, 65.71) --
	(278.61, 65.71) --
	(278.61, 65.71) --
	(278.61, 65.71) --
	(278.55, 65.71) --
	(278.55, 65.71) --
	(278.55, 65.71) --
	(278.48, 65.71) --
	(278.48, 65.71) --
	(278.48, 65.71) --
	(278.41, 65.71) --
	(278.41, 65.71) --
	(278.41, 65.71) --
	(278.35, 65.71) --
	(278.35, 65.71) --
	(278.35, 65.71) --
	(278.28, 65.71) --
	(278.28, 65.71) --
	(278.28, 65.71) --
	(278.21, 65.71) --
	(278.21, 65.71) --
	(278.21, 65.71) --
	(278.15, 65.71) --
	(278.15, 65.71) --
	(278.15, 65.71) --
	(278.08, 65.71) --
	(278.08, 65.71) --
	(278.08, 65.71) --
	(278.01, 65.71) --
	(278.01, 65.71) --
	(278.01, 65.71) --
	(277.95, 65.71) --
	(277.95, 65.71) --
	(277.95, 65.71) --
	(277.88, 65.71) --
	(277.88, 65.71) --
	(277.88, 65.71) --
	(277.81, 65.71) --
	(277.81, 65.71) --
	(277.81, 65.71) --
	(277.75, 65.71) --
	(277.75, 65.71) --
	(277.75, 65.71) --
	(277.68, 65.71) --
	(277.68, 65.71) --
	(277.68, 65.71) --
	(277.61, 65.71) --
	(277.61, 65.71) --
	(277.61, 65.71) --
	(277.55, 65.71) --
	(277.55, 65.71) --
	(277.55, 65.71) --
	(277.48, 65.71) --
	(277.48, 65.71) --
	(277.48, 65.71) --
	(277.41, 65.71) --
	(277.41, 65.71) --
	(277.41, 65.71) --
	(277.35, 65.71) --
	(277.35, 65.71) --
	(277.35, 65.71) --
	(277.28, 65.71) --
	(277.28, 65.71) --
	(277.28, 65.71) --
	(277.21, 65.71) --
	(277.21, 65.71) --
	(277.21, 65.71) --
	(277.15, 65.71) --
	(277.15, 65.71) --
	(277.15, 65.71) --
	(277.08, 65.71) --
	(277.08, 65.71) --
	(277.08, 65.71) --
	(277.01, 65.71) --
	(277.01, 65.71) --
	(277.01, 65.71) --
	(276.94, 65.71) --
	(276.94, 65.71) --
	(276.94, 65.71) --
	(276.88, 65.71) --
	(276.88, 65.71) --
	(276.88, 65.71) --
	(276.81, 65.71) --
	(276.81, 65.71) --
	(276.81, 65.71) --
	(276.74, 65.71) --
	(276.74, 65.71) --
	(276.74, 65.71) --
	(276.68, 65.71) --
	(276.68, 65.71) --
	(276.68, 65.71) --
	(276.61, 65.71) --
	(276.61, 65.71) --
	(276.61, 65.71) --
	(276.54, 65.71) --
	(276.54, 65.71) --
	(276.54, 65.71) --
	(276.48, 65.71) --
	(276.48, 65.71) --
	(276.48, 65.71) --
	(276.41, 65.71) --
	(276.41, 65.71) --
	(276.41, 65.71) --
	(276.34, 65.71) --
	(276.34, 65.71) --
	(276.34, 65.71) --
	(276.28, 65.71) --
	(276.28, 65.71) --
	(276.28, 65.71) --
	(276.21, 65.71) --
	(276.21, 65.71) --
	(276.21, 65.71) --
	(276.14, 65.71) --
	(276.14, 65.71) --
	(276.14, 65.71) --
	(276.07, 65.71) --
	(276.07, 65.71) --
	(276.07, 65.71) --
	(276.01, 65.71) --
	(276.01, 65.71) --
	(276.01, 65.71) --
	(275.94, 65.71) --
	(275.94, 65.71) --
	(275.94, 65.71) --
	(275.87, 65.71) --
	(275.87, 65.71) --
	(275.87, 65.71) --
	(275.81, 65.71) --
	(275.81, 65.71) --
	(275.81, 65.71) --
	(275.74, 65.71) --
	(275.74, 65.71) --
	(275.74, 65.71) --
	(275.67, 65.71) --
	(275.67, 65.71) --
	(275.67, 65.71) --
	(275.61, 65.71) --
	(275.60, 65.71) --
	(275.60, 65.71) --
	(275.54, 65.71) --
	(275.54, 65.71) --
	(275.54, 65.71) --
	(275.47, 65.71) --
	(275.47, 65.71) --
	(275.47, 65.71) --
	(275.40, 65.71) --
	(275.40, 65.71) --
	(275.40, 65.71) --
	(275.34, 65.71) --
	(275.34, 65.71) --
	(275.34, 65.71) --
	(275.27, 65.71) --
	(275.27, 65.71) --
	(275.27, 65.71) --
	(275.20, 65.71) --
	(275.20, 65.71) --
	(275.20, 65.71) --
	(275.14, 65.71) --
	(275.14, 65.71) --
	(275.14, 65.71) --
	(275.07, 65.71) --
	(275.07, 65.71) --
	(275.07, 65.71) --
	(275.00, 65.71) --
	(275.00, 65.71) --
	(275.00, 65.71) --
	(274.93, 65.71) --
	(274.93, 65.71) --
	(274.93, 65.71) --
	(274.87, 65.71) --
	(274.87, 65.71) --
	(274.87, 65.71) --
	(274.80, 65.71) --
	(274.80, 65.71) --
	(274.80, 65.71) --
	(274.73, 65.71) --
	(274.73, 65.71) --
	(274.73, 65.71) --
	(274.67, 65.71) --
	(274.67, 65.71) --
	(274.67, 65.71) --
	(274.60, 65.71) --
	(274.60, 65.71) --
	(274.60, 65.71) --
	(274.53, 65.71) --
	(274.53, 65.71) --
	(274.53, 65.71) --
	(274.46, 65.71) --
	(274.46, 65.71) --
	(274.46, 65.71) --
	(274.40, 65.71) --
	(274.40, 65.71) --
	(274.40, 65.71) --
	(274.33, 65.71) --
	(274.33, 65.71) --
	(274.33, 65.71) --
	(274.26, 65.71) --
	(274.26, 65.71) --
	(274.26, 65.71) --
	(274.20, 65.71) --
	(274.20, 65.71) --
	(274.20, 65.71) --
	(274.13, 65.71) --
	(274.13, 65.71) --
	(274.13, 65.71) --
	(274.06, 65.71) --
	(274.06, 65.71) --
	(274.06, 65.71) --
	(274.00, 65.71) --
	(274.00, 65.71) --
	(274.00, 65.71) --
	(273.93, 65.71) --
	(273.93, 65.71) --
	(273.93, 65.71) --
	(273.86, 65.71) --
	(273.86, 65.71) --
	(273.86, 65.71) --
	(273.79, 65.71) --
	(273.79, 65.71) --
	(273.79, 65.71) --
	(273.73, 65.71) --
	(273.73, 65.71) --
	(273.73, 65.71) --
	(273.66, 65.71) --
	(273.66, 65.71) --
	(273.66, 65.71) --
	(273.59, 65.71) --
	(273.59, 65.71) --
	(273.59, 65.71) --
	(273.53, 65.71) --
	(273.53, 65.71) --
	(273.53, 65.71) --
	(273.46, 65.71) --
	(273.46, 65.71) --
	(273.46, 65.71) --
	(273.39, 65.71) --
	(273.39, 65.71) --
	(273.39, 65.71) --
	(273.32, 65.71) --
	(273.32, 65.71) --
	(273.32, 65.71) --
	(273.26, 65.71) --
	(273.26, 65.71) --
	(273.26, 65.71) --
	(273.19, 65.71) --
	(273.19, 65.71) --
	(273.19, 65.71) --
	(273.12, 65.71) --
	(273.12, 65.71) --
	(273.12, 65.71) --
	(273.05, 65.71) --
	(273.05, 65.71) --
	(273.05, 65.71) --
	(272.99, 65.71) --
	(272.99, 65.71) --
	(272.99, 65.71) --
	(272.92, 65.71) --
	(272.92, 65.71) --
	(272.92, 65.71) --
	(272.85, 65.71) --
	(272.85, 65.71) --
	(272.85, 65.71) --
	(272.79, 65.71) --
	(272.79, 65.71) --
	(272.79, 65.71) --
	(272.72, 65.71) --
	(272.72, 65.71) --
	(272.72, 65.71) --
	(272.65, 65.71) --
	(272.65, 65.71) --
	(272.65, 65.71) --
	(272.58, 65.71) --
	(272.58, 65.71) --
	(272.58, 65.71) --
	(272.52, 65.71) --
	(272.52, 65.71) --
	(272.52, 65.71) --
	(272.45, 65.71) --
	(272.45, 65.71) --
	(272.45, 65.71) --
	(272.38, 65.71) --
	(272.38, 65.71) --
	(272.38, 65.71) --
	(272.31, 65.71) --
	(272.31, 65.71) --
	(272.31, 65.71) --
	(272.25, 65.71) --
	(272.25, 65.71) --
	(272.25, 65.71) --
	(272.18, 65.71) --
	(272.18, 65.71) --
	(272.18, 65.71) --
	(272.11, 65.71) --
	(272.11, 65.71) --
	(272.11, 65.71) --
	(272.04, 65.71) --
	(272.04, 65.71) --
	(272.04, 65.71) --
	(271.98, 65.71) --
	(271.98, 65.71) --
	(271.98, 65.71) --
	(271.91, 65.71) --
	(271.91, 65.71) --
	(271.91, 65.71) --
	(271.84, 65.71) --
	(271.84, 65.71) --
	(271.84, 65.71) --
	(271.77, 65.71) --
	(271.77, 65.71) --
	(271.77, 65.71) --
	(271.71, 65.71) --
	(271.71, 65.71) --
	(271.71, 65.71) --
	(271.64, 65.71) --
	(271.64, 65.71) --
	(271.64, 65.71) --
	(271.57, 65.71) --
	(271.57, 65.71) --
	(271.57, 65.71) --
	(271.51, 65.71) --
	(271.51, 65.71) --
	(271.51, 65.71) --
	(271.44, 65.71) --
	(271.44, 65.71) --
	(271.44, 65.71) --
	(271.37, 65.71) --
	(271.37, 65.71) --
	(271.37, 65.71) --
	(271.30, 65.71) --
	(271.30, 65.71) --
	(271.30, 65.71) --
	(271.24, 65.71) --
	(271.24, 65.71) --
	(271.24, 65.71) --
	(271.17, 65.71) --
	(271.17, 65.71) --
	(271.17, 65.71) --
	(271.10, 65.71) --
	(271.10, 65.71) --
	(271.10, 65.71) --
	(271.03, 65.71) --
	(271.03, 65.71) --
	(271.03, 65.71) --
	(270.97, 65.71) --
	(270.97, 65.71) --
	(270.97, 65.71) --
	(270.90, 65.71) --
	(270.90, 65.71) --
	(270.90, 65.71) --
	(270.83, 65.71) --
	(270.83, 65.71) --
	(270.83, 65.71) --
	(270.76, 65.71) --
	(270.76, 65.71) --
	(270.76, 65.71) --
	(270.70, 65.71) --
	(270.70, 65.71) --
	(270.70, 65.71) --
	(270.63, 65.71) --
	(270.63, 65.71) --
	(270.63, 65.71) --
	(270.56, 65.71) --
	(270.56, 65.71) --
	(270.56, 65.71) --
	(270.49, 65.71) --
	(270.49, 65.71) --
	(270.49, 65.71) --
	(270.43, 65.71) --
	(270.43, 65.71) --
	(270.43, 65.71) --
	(270.36, 65.71) --
	(270.36, 65.71) --
	(270.36, 65.71) --
	(270.29, 65.71) --
	(270.29, 65.71) --
	(270.29, 65.71) --
	(270.22, 65.71) --
	(270.22, 65.71) --
	(270.22, 65.71) --
	(270.16, 65.71) --
	(270.16, 65.71) --
	(270.16, 65.71) --
	(270.09, 65.71) --
	(270.09, 65.71) --
	(270.09, 65.71) --
	(270.02, 65.71) --
	(270.02, 65.71) --
	(270.02, 65.71) --
	(269.95, 65.71) --
	(269.95, 65.71) --
	(269.95, 65.71) --
	(269.89, 65.71) --
	(269.89, 65.71) --
	(269.89, 65.71) --
	(269.82, 65.71) --
	(269.82, 65.71) --
	(269.82, 65.71) --
	(269.75, 65.71) --
	(269.75, 65.71) --
	(269.75, 65.71) --
	(269.69, 65.71) --
	(269.69, 65.71) --
	(269.69, 65.71) --
	(269.62, 65.71) --
	(269.62, 65.71) --
	(269.62, 65.71) --
	(269.55, 65.71) --
	(269.55, 65.71) --
	(269.55, 65.71) --
	(269.48, 65.71) --
	(269.48, 65.71) --
	(269.48, 65.71) --
	(269.42, 65.71) --
	(269.42, 65.71) --
	(269.42, 65.71) --
	(269.35, 65.71) --
	(269.35, 65.71) --
	(269.35, 65.71) --
	(269.28, 65.71) --
	(269.28, 65.71) --
	(269.28, 65.71) --
	(269.21, 65.71) --
	(269.21, 65.71) --
	(269.21, 65.71) --
	(269.14, 65.71) --
	(269.14, 65.71) --
	(269.14, 65.71) --
	(269.08, 65.71) --
	(269.08, 65.71) --
	(269.08, 65.71) --
	(269.01, 65.71) --
	(269.01, 65.71) --
	(269.01, 65.71) --
	(268.94, 65.71) --
	(268.94, 65.71) --
	(268.94, 65.71) --
	(268.87, 65.71) --
	(268.87, 65.71) --
	(268.87, 65.71) --
	(268.81, 65.71) --
	(268.81, 65.71) --
	(268.81, 65.71) --
	(268.74, 65.71) --
	(268.74, 65.71) --
	(268.74, 65.71) --
	(268.67, 65.71) --
	(268.67, 65.71) --
	(268.67, 65.71) --
	(268.60, 65.71) --
	(268.60, 65.71) --
	(268.60, 65.71) --
	(268.54, 65.71) --
	(268.54, 65.71) --
	(268.54, 65.71) --
	(268.47, 65.71) --
	(268.47, 65.71) --
	(268.47, 65.71) --
	(268.40, 65.71) --
	(268.40, 65.71) --
	(268.40, 65.71) --
	(268.33, 65.71) --
	(268.33, 65.71) --
	(268.33, 65.71) --
	(268.27, 65.71) --
	(268.27, 65.71) --
	(268.27, 65.71) --
	(268.20, 65.71) --
	(268.20, 65.71) --
	(268.20, 65.71) --
	(268.13, 65.71) --
	(268.13, 65.71) --
	(268.13, 65.71) --
	(268.06, 65.71) --
	(268.06, 65.71) --
	(268.06, 65.71) --
	(267.99, 65.71) --
	(267.99, 65.71) --
	(267.99, 65.71) --
	(267.93, 65.71) --
	(267.93, 65.71) --
	(267.93, 65.71) --
	(267.86, 65.71) --
	(267.86, 65.71) --
	(267.86, 65.71) --
	(267.79, 65.71) --
	(267.79, 65.71) --
	(267.79, 65.71) --
	(267.72, 65.71) --
	(267.72, 65.71) --
	(267.72, 65.71) --
	(267.66, 65.71) --
	(267.66, 65.71) --
	(267.66, 65.71) --
	(267.59, 65.71) --
	(267.59, 65.71) --
	(267.59, 65.71) --
	(267.52, 65.71) --
	(267.52, 65.71) --
	(267.52, 65.71) --
	(267.45, 65.71) --
	(267.45, 65.71) --
	(267.45, 65.71) --
	(267.39, 65.71) --
	(267.39, 65.71) --
	(267.39, 65.71) --
	(267.32, 65.71) --
	(267.32, 65.71) --
	(267.32, 65.71) --
	(267.25, 65.71) --
	(267.25, 65.71) --
	(267.25, 65.71) --
	(267.18, 65.71) --
	(267.18, 65.71) --
	(267.18, 65.71) --
	(267.11, 65.71) --
	(267.11, 65.71) --
	(267.11, 65.71) --
	(267.05, 65.71) --
	(267.05, 65.71) --
	(267.05, 65.71) --
	(266.98, 65.71) --
	(266.98, 65.71) --
	(266.98, 65.71) --
	(266.91, 65.71) --
	(266.91, 65.71) --
	(266.91, 65.71) --
	(266.84, 65.71) --
	(266.84, 65.71) --
	(266.84, 65.71) --
	(266.78, 65.71) --
	(266.78, 65.71) --
	(266.78, 65.71) --
	(266.71, 65.71) --
	(266.71, 65.71) --
	(266.71, 65.71) --
	(266.64, 65.71) --
	(266.64, 65.71) --
	(266.64, 65.71) --
	(266.57, 65.71) --
	(266.57, 65.71) --
	(266.57, 65.71) --
	(266.51, 65.71) --
	(266.51, 65.71) --
	(266.51, 65.71) --
	(266.44, 65.71) --
	(266.44, 65.71) --
	(266.44, 65.71) --
	(266.37, 65.71) --
	(266.37, 65.71) --
	(266.37, 65.71) --
	(266.30, 65.71) --
	(266.30, 65.71) --
	(266.30, 65.71) --
	(266.23, 65.71) --
	(266.23, 65.71) --
	(266.23, 65.71) --
	(266.17, 65.71) --
	(266.17, 65.71) --
	(266.17, 65.71) --
	(266.10, 65.71) --
	(266.10, 65.71) --
	(266.10, 65.71) --
	(266.03, 65.71) --
	(266.03, 65.71) --
	(266.03, 65.71) --
	(265.96, 65.71) --
	(265.96, 65.71) --
	(265.96, 65.71) --
	(265.90, 65.71) --
	(265.90, 65.71) --
	(265.90, 65.71) --
	(265.83, 65.71) --
	(265.83, 65.71) --
	(265.83, 65.71) --
	(265.76, 65.71) --
	(265.76, 65.71) --
	(265.76, 65.71) --
	(265.69, 65.71) --
	(265.69, 65.71) --
	(265.69, 65.71) --
	(265.62, 65.71) --
	(265.62, 65.71) --
	(265.62, 65.71) --
	(265.56, 65.71) --
	(265.56, 65.71) --
	(265.56, 65.71) --
	(265.49, 65.71) --
	(265.49, 65.71) --
	(265.49, 65.71) --
	(265.42, 65.71) --
	(265.42, 65.71) --
	(265.42, 65.71) --
	(265.35, 65.71) --
	(265.35, 65.71) --
	(265.35, 65.71) --
	(265.28, 65.71) --
	(265.28, 65.71) --
	(265.28, 65.71) --
	(265.22, 65.71) --
	(265.22, 65.71) --
	(265.22, 65.71) --
	(265.15, 65.71) --
	(265.15, 65.71) --
	(265.15, 65.71) --
	(265.08, 65.71) --
	(265.08, 65.71) --
	(265.08, 65.71) --
	(265.01, 65.71) --
	(265.01, 65.71) --
	(265.01, 65.71) --
	(264.94, 65.71) --
	(264.94, 65.71) --
	(264.94, 65.71) --
	(264.88, 65.71) --
	(264.88, 65.71) --
	(264.88, 65.71) --
	(264.81, 65.71) --
	(264.81, 65.71) --
	(264.81, 65.71) --
	(264.74, 65.71) --
	(264.74, 65.71) --
	(264.74, 65.71) --
	(264.67, 65.71) --
	(264.67, 65.71) --
	(264.67, 65.71) --
	(264.61, 65.71) --
	(264.61, 65.71) --
	(264.61, 65.71) --
	(264.54, 65.71) --
	(264.54, 65.71) --
	(264.54, 65.71) --
	(264.47, 65.71) --
	(264.47, 65.71) --
	(264.47, 65.71) --
	(264.40, 65.71) --
	(264.40, 65.71) --
	(264.40, 65.71) --
	(264.33, 65.71) --
	(264.33, 65.71) --
	(264.33, 65.71) --
	(264.26, 65.71) --
	(264.26, 65.71) --
	(264.26, 65.71) --
	(264.20, 65.71) --
	(264.20, 65.71) --
	(264.20, 65.71) --
	(264.13, 65.71) --
	(264.13, 65.71) --
	(264.13, 65.71) --
	(264.06, 65.71) --
	(264.06, 65.71) --
	(264.06, 65.71) --
	(263.99, 65.71) --
	(263.99, 65.71) --
	(263.99, 65.71) --
	(263.93, 65.71) --
	(263.93, 65.71) --
	(263.93, 65.71) --
	(263.86, 65.71) --
	(263.86, 65.71) --
	(263.86, 65.71) --
	(263.79, 65.71) --
	(263.79, 65.71) --
	(263.79, 65.71) --
	(263.72, 65.71) --
	(263.72, 65.71) --
	(263.72, 65.71) --
	(263.65, 65.71) --
	(263.65, 65.71) --
	(263.65, 65.71) --
	(263.58, 65.71) --
	(263.58, 65.71) --
	(263.58, 65.71) --
	(263.52, 65.71) --
	(263.52, 65.71) --
	(263.52, 65.71) --
	(263.45, 65.71) --
	(263.45, 65.71) --
	(263.45, 65.71) --
	(263.38, 65.71) --
	(263.38, 65.71) --
	(263.38, 65.71) --
	(263.31, 65.71) --
	(263.31, 65.71) --
	(263.31, 65.71) --
	(263.25, 65.71) --
	(263.25, 65.71) --
	(263.25, 65.71) --
	(263.18, 65.71) --
	(263.18, 65.71) --
	(263.18, 65.71) --
	(263.11, 65.71) --
	(263.11, 65.71) --
	(263.11, 65.71) --
	(263.04, 65.71) --
	(263.04, 65.71) --
	(263.04, 65.71) --
	(262.97, 65.71) --
	(262.97, 65.71) --
	(262.97, 65.71) --
	(262.90, 65.71) --
	(262.90, 65.71) --
	(262.90, 65.71) --
	(262.84, 65.71) --
	(262.84, 65.71) --
	(262.84, 65.71) --
	(262.77, 65.71) --
	(262.77, 65.71) --
	(262.77, 65.71) --
	(262.70, 65.71) --
	(262.70, 65.71) --
	(262.70, 65.71) --
	(262.63, 65.71) --
	(262.63, 65.71) --
	(262.63, 65.71) --
	(262.56, 65.71) --
	(262.56, 65.71) --
	(262.56, 65.71) --
	(262.50, 65.71) --
	(262.50, 65.71) --
	(262.50, 65.71) --
	(262.43, 65.71) --
	(262.43, 65.71) --
	(262.43, 65.71) --
	(262.36, 65.71) --
	(262.36, 65.71) --
	(262.36, 65.71) --
	(262.29, 65.71) --
	(262.29, 65.71) --
	(262.29, 65.71) --
	(262.22, 65.71) --
	(262.22, 65.71) --
	(262.22, 65.71) --
	(262.16, 65.71) --
	(262.16, 65.71) --
	(262.16, 65.71) --
	(262.09, 65.71) --
	(262.09, 65.71) --
	(262.09, 65.71) --
	(262.02, 65.71) --
	(262.02, 65.71) --
	(262.02, 65.71) --
	(261.95, 65.71) --
	(261.95, 65.71) --
	(261.95, 65.71) --
	(261.88, 65.71) --
	(261.88, 65.71) --
	(261.88, 65.71) --
	(261.81, 65.71) --
	(261.81, 65.71) --
	(261.81, 65.71) --
	(261.75, 65.71) --
	(261.75, 65.71) --
	(261.75, 65.71) --
	(261.68, 65.71) --
	(261.68, 65.71) --
	(261.68, 65.71) --
	(261.61, 65.71) --
	(261.61, 65.71) --
	(261.61, 65.71) --
	(261.54, 65.71) --
	(261.54, 65.71) --
	(261.54, 65.71) --
	(261.47, 65.71) --
	(261.47, 65.71) --
	(261.47, 65.71) --
	(261.40, 65.71) --
	(261.40, 65.71) --
	(261.40, 65.71) --
	(261.34, 65.71) --
	(261.34, 65.71) --
	(261.34, 65.71) --
	(261.27, 65.71) --
	(261.27, 65.71) --
	(261.27, 65.71) --
	(261.20, 65.71) --
	(261.20, 65.71) --
	(261.20, 65.71) --
	(261.13, 65.71) --
	(261.13, 65.71) --
	(261.13, 65.71) --
	(261.06, 65.71) --
	(261.06, 65.71) --
	(261.06, 65.71) --
	(261.00, 65.71) --
	(261.00, 65.71) --
	(261.00, 65.71) --
	(260.93, 65.71) --
	(260.93, 65.71) --
	(260.93, 65.71) --
	(260.86, 65.71) --
	(260.86, 65.71) --
	(260.86, 65.71) --
	(260.79, 65.71) --
	(260.79, 65.71) --
	(260.79, 65.71) --
	(260.72, 65.71) --
	(260.72, 65.71) --
	(260.72, 65.71) --
	(260.66, 65.71) --
	(260.66, 65.71) --
	(260.66, 65.71) --
	(260.59, 65.71) --
	(260.59, 65.71) --
	(260.59, 65.71) --
	(260.52, 65.71) --
	(260.52, 65.71) --
	(260.52, 65.71) --
	(260.45, 65.71) --
	(260.45, 65.71) --
	(260.45, 65.71) --
	(260.38, 65.71) --
	(260.38, 65.71) --
	(260.38, 65.71) --
	(260.31, 65.71) --
	(260.31, 65.71) --
	(260.31, 65.71) --
	(260.25, 65.71) --
	(260.25, 65.71) --
	(260.25, 65.71) --
	(260.18, 65.71) --
	(260.18, 65.71) --
	(260.18, 65.71) --
	(260.11, 65.71) --
	(260.11, 65.71) --
	(260.11, 65.71) --
	(260.04, 65.71) --
	(260.04, 65.71) --
	(260.04, 65.71) --
	(259.97, 65.71) --
	(259.97, 65.71) --
	(259.97, 65.71) --
	(259.90, 65.71) --
	(259.90, 65.71) --
	(259.90, 65.71) --
	(259.83, 65.71) --
	(259.83, 65.71) --
	(259.83, 65.71) --
	(259.77, 65.71) --
	(259.77, 65.71) --
	(259.77, 65.71) --
	(259.70, 65.71) --
	(259.70, 65.71) --
	(259.70, 65.71) --
	(259.63, 65.71) --
	(259.63, 65.71) --
	(259.63, 65.71) --
	(259.56, 65.71) --
	(259.56, 65.71) --
	(259.56, 65.71) --
	(259.49, 65.71) --
	(259.49, 65.71) --
	(259.49, 65.71) --
	(259.42, 65.71) --
	(259.42, 65.71) --
	(259.42, 65.71) --
	(259.36, 65.71) --
	(259.36, 65.71) --
	(259.36, 65.71) --
	(259.29, 65.71) --
	(259.29, 65.71) --
	(259.29, 65.71) --
	(259.22, 65.71) --
	(259.22, 65.71) --
	(259.22, 65.71) --
	(259.15, 65.71) --
	(259.15, 65.71) --
	(259.15, 65.71) --
	(259.08, 65.71) --
	(259.08, 65.71) --
	(259.08, 65.71) --
	(259.01, 65.71) --
	(259.01, 65.71) --
	(259.01, 65.71) --
	(258.95, 65.71) --
	(258.95, 65.71) --
	(258.95, 65.71) --
	(258.88, 65.71) --
	(258.88, 65.71) --
	(258.88, 65.71) --
	(258.81, 65.71) --
	(258.81, 65.71) --
	(258.81, 65.71) --
	(258.74, 65.71) --
	(258.74, 65.71) --
	(258.74, 65.71) --
	(258.67, 65.71) --
	(258.67, 65.71) --
	(258.67, 65.71) --
	(258.60, 65.71) --
	(258.60, 65.71) --
	(258.60, 65.71) --
	(258.54, 65.71) --
	(258.54, 65.71) --
	(258.54, 65.71) --
	(258.47, 65.71) --
	(258.47, 65.71) --
	(258.47, 65.71) --
	(258.40, 65.71) --
	(258.40, 65.71) --
	(258.40, 65.71) --
	(258.33, 65.71) --
	(258.33, 65.71) --
	(258.33, 65.71) --
	(258.26, 65.71) --
	(258.26, 65.71) --
	(258.26, 65.71) --
	(258.19, 65.71) --
	(258.19, 65.71) --
	(258.19, 65.71) --
	(258.12, 65.71) --
	(258.12, 65.71) --
	(258.12, 65.71) --
	(258.06, 65.71) --
	(258.06, 65.71) --
	(258.06, 65.71) --
	(257.99, 65.71) --
	(257.99, 65.71) --
	(257.99, 65.71) --
	(257.92, 65.71) --
	(257.92, 65.71) --
	(257.92, 65.71) --
	(257.85, 65.71) --
	(257.85, 65.71) --
	(257.85, 65.71) --
	(257.78, 65.71) --
	(257.78, 65.71) --
	(257.78, 65.71) --
	(257.71, 65.71) --
	(257.71, 65.71) --
	(257.71, 65.71) --
	(257.65, 65.71) --
	(257.65, 65.71) --
	(257.65, 65.71) --
	(257.58, 65.71) --
	(257.58, 65.71) --
	(257.58, 65.71) --
	(257.51, 65.71) --
	(257.51, 65.71) --
	(257.51, 65.71) --
	(257.44, 65.71) --
	(257.44, 65.71) --
	(257.44, 65.71) --
	(257.37, 65.71) --
	(257.37, 65.71) --
	(257.37, 65.71) --
	(257.30, 65.71) --
	(257.30, 65.71) --
	(257.30, 65.71) --
	(257.24, 65.71) --
	(257.24, 65.71) --
	(257.24, 65.71) --
	(257.17, 65.71) --
	(257.17, 65.71) --
	(257.17, 65.71) --
	(257.10, 65.71) --
	(257.10, 65.71) --
	(257.10, 65.71) --
	(257.03, 65.71) --
	(257.03, 65.71) --
	(257.03, 65.71) --
	(256.96, 65.71) --
	(256.96, 65.71) --
	(256.96, 65.71) --
	(256.89, 65.71) --
	(256.89, 65.71) --
	(256.89, 65.71) --
	(256.82, 65.71) --
	(256.82, 65.71) --
	(256.82, 65.71) --
	(256.75, 65.71) --
	(256.75, 65.71) --
	(256.75, 65.71) --
	(256.69, 65.71) --
	(256.69, 65.71) --
	(256.69, 65.71) --
	(256.62, 65.71) --
	(256.62, 65.71) --
	(256.62, 65.71) --
	(256.55, 65.71) --
	(256.55, 65.71) --
	(256.55, 65.71) --
	(256.48, 65.71) --
	(256.48, 65.71) --
	(256.48, 65.71) --
	(256.41, 65.71) --
	(256.41, 65.71) --
	(256.41, 65.71) --
	(256.34, 65.71) --
	(256.34, 65.71) --
	(256.34, 65.71) --
	(256.27, 65.71) --
	(256.27, 65.71) --
	(256.27, 65.71) --
	(256.20, 65.71) --
	(256.20, 65.71) --
	(256.20, 65.71) --
	(256.14, 65.71) --
	(256.14, 65.71) --
	(256.14, 65.71) --
	(256.07, 65.71) --
	(256.07, 65.71) --
	(256.07, 65.71) --
	(256.00, 65.71) --
	(256.00, 65.71) --
	(256.00, 65.71) --
	(255.93, 65.71) --
	(255.93, 65.71) --
	(255.93, 65.71) --
	(255.86, 65.71) --
	(255.86, 65.71) --
	(255.86, 65.71) --
	(255.79, 65.71) --
	(255.79, 65.71) --
	(255.79, 65.71) --
	(255.73, 65.71) --
	(255.73, 65.71) --
	(255.73, 65.71) --
	(255.66, 65.71) --
	(255.66, 65.71) --
	(255.66, 65.71) --
	(255.59, 65.71) --
	(255.59, 65.71) --
	(255.59, 65.71) --
	(255.52, 65.71) --
	(255.52, 65.71) --
	(255.52, 65.71) --
	(255.45, 65.71) --
	(255.45, 65.71) --
	(255.45, 65.71) --
	(255.38, 65.71) --
	(255.38, 65.71) --
	(255.38, 65.71) --
	(255.31, 65.71) --
	(255.31, 65.71) --
	(255.31, 65.71) --
	(255.24, 65.71) --
	(255.24, 65.71) --
	(255.24, 65.71) --
	(255.18, 65.71) --
	(255.18, 65.71) --
	(255.18, 65.71) --
	(255.11, 65.71) --
	(255.11, 65.71) --
	(255.11, 65.71) --
	(255.04, 65.71) --
	(255.04, 65.71) --
	(255.04, 65.71) --
	(254.97, 65.71) --
	(254.97, 65.71) --
	(254.97, 65.71) --
	(254.90, 65.71) --
	(254.90, 65.71) --
	(254.90, 65.71) --
	(254.83, 65.71) --
	(254.83, 65.71) --
	(254.83, 65.71) --
	(254.76, 65.71) --
	(254.76, 65.71) --
	(254.76, 65.71) --
	(254.70, 65.71) --
	(254.70, 65.71) --
	(254.70, 65.71) --
	(254.63, 65.71) --
	(254.63, 65.71) --
	(254.63, 65.71) --
	(254.56, 65.71) --
	(254.56, 65.71) --
	(254.56, 65.71) --
	(254.49, 65.71) --
	(254.49, 65.71) --
	(254.49, 65.71) --
	(254.42, 65.71) --
	(254.42, 65.71) --
	(254.42, 65.71) --
	(254.35, 65.71) --
	(254.35, 65.71) --
	(254.35, 65.71) --
	(254.28, 65.71) --
	(254.28, 65.71) --
	(254.28, 65.71) --
	(254.21, 65.71) --
	(254.21, 65.71) --
	(254.21, 65.71) --
	(254.14, 65.71) --
	(254.14, 65.71) --
	(254.14, 65.71) --
	(254.08, 65.71) --
	(254.08, 65.71) --
	(254.08, 65.71) --
	(254.01, 65.71) --
	(254.01, 65.71) --
	(254.01, 65.71) --
	(253.94, 65.71) --
	(253.94, 65.71) --
	(253.94, 65.71) --
	(253.87, 65.71) --
	(253.87, 65.71) --
	(253.87, 65.71) --
	(253.80, 65.71) --
	(253.80, 65.71) --
	(253.80, 65.71) --
	(253.73, 65.71) --
	(253.73, 65.71) --
	(253.73, 65.71) --
	(253.66, 65.71) --
	(253.66, 65.71) --
	(253.66, 65.71) --
	(253.59, 65.71) --
	(253.59, 65.71) --
	(253.59, 65.71) --
	(253.53, 65.71) --
	(253.53, 65.71) --
	(253.53, 65.71) --
	(253.46, 65.71) --
	(253.46, 65.71) --
	(253.46, 65.71) --
	(253.39, 65.71) --
	(253.39, 65.71) --
	(253.39, 65.71) --
	(253.32, 65.71) --
	(253.32, 65.71) --
	(253.32, 65.71) --
	(253.25, 65.71) --
	(253.25, 65.71) --
	(253.25, 65.71) --
	(253.18, 65.71) --
	(253.18, 65.71) --
	(253.18, 65.71) --
	(253.11, 65.71) --
	(253.11, 65.71) --
	(253.11, 65.71) --
	(253.04, 65.71) --
	(253.04, 65.71) --
	(253.04, 65.71) --
	(252.97, 65.71) --
	(252.97, 65.71) --
	(252.97, 65.71) --
	(252.91, 65.71) --
	(252.91, 65.71) --
	(252.91, 65.71) --
	(252.84, 65.71) --
	(252.84, 65.71) --
	(252.84, 65.71) --
	(252.77, 65.71) --
	(252.77, 65.71) --
	(252.77, 65.71) --
	(252.70, 65.71) --
	(252.70, 65.71) --
	(252.70, 65.71) --
	(252.63, 65.71) --
	(252.63, 65.71) --
	(252.63, 65.71) --
	(252.56, 65.71) --
	(252.56, 65.71) --
	(252.56, 65.71) --
	(252.49, 65.71) --
	(252.49, 65.71) --
	(252.49, 65.71) --
	(252.42, 65.71) --
	(252.42, 65.71) --
	(252.42, 65.71) --
	(252.35, 65.71) --
	(252.35, 65.71) --
	(252.35, 65.71) --
	(252.29, 65.71) --
	(252.29, 65.71) --
	(252.29, 65.71) --
	(252.22, 65.71) --
	(252.22, 65.71) --
	(252.22, 65.71) --
	(252.15, 65.71) --
	(252.15, 65.71) --
	(252.15, 65.71) --
	(252.08, 65.71) --
	(252.08, 65.71) --
	(252.08, 65.71) --
	(252.01, 65.71) --
	(252.01, 65.71) --
	(252.01, 65.71) --
	(251.94, 65.71) --
	(251.94, 65.71) --
	(251.94, 65.71) --
	(251.87, 65.71) --
	(251.87, 65.71) --
	(251.87, 65.71) --
	(251.80, 65.71) --
	(251.80, 65.71) --
	(251.80, 65.71) --
	(251.73, 65.71) --
	(251.73, 65.71) --
	(251.73, 65.71) --
	(251.66, 65.71) --
	(251.66, 65.71) --
	(251.66, 65.71) --
	(251.60, 65.71) --
	(251.60, 65.71) --
	(251.60, 65.71) --
	(251.53, 65.71) --
	(251.53, 65.71) --
	(251.53, 65.71) --
	(251.46, 65.71) --
	(251.46, 65.71) --
	(251.46, 65.71) --
	(251.39, 65.71) --
	(251.39, 65.71) --
	(251.39, 65.71) --
	(251.32, 65.71) --
	(251.32, 65.71) --
	(251.32, 65.71) --
	(251.25, 65.71) --
	(251.25, 65.71) --
	(251.25, 65.71) --
	(251.18, 65.71) --
	(251.18, 65.71) --
	(251.18, 65.71) --
	(251.11, 65.71) --
	(251.11, 65.71) --
	(251.11, 65.71) --
	(251.04, 65.71) --
	(251.04, 65.71) --
	(251.04, 65.71) --
	(250.97, 65.71) --
	(250.97, 65.71) --
	(250.97, 65.71) --
	(250.91, 65.71) --
	(250.91, 65.71) --
	(250.91, 65.71) --
	(250.84, 65.71) --
	(250.84, 65.71) --
	(250.84, 65.71) --
	(250.77, 65.71) --
	(250.77, 65.71) --
	(250.77, 65.71) --
	(250.70, 65.71) --
	(250.70, 65.71) --
	(250.70, 65.71) --
	(250.63, 65.71) --
	(250.63, 65.71) --
	(250.63, 65.71) --
	(250.56, 65.71) --
	(250.56, 65.71) --
	(250.56, 65.71) --
	(250.49, 65.71) --
	(250.49, 65.71) --
	(250.49, 65.71) --
	(250.42, 65.71) --
	(250.42, 65.71) --
	(250.42, 65.71) --
	(250.35, 65.71) --
	(250.35, 65.71) --
	(250.35, 65.71) --
	(250.28, 65.71) --
	(250.28, 65.71) --
	(250.28, 65.71) --
	(250.22, 65.71) --
	(250.22, 65.71) --
	(250.22, 65.71) --
	(250.15, 65.71) --
	(250.15, 65.71) --
	(250.15, 65.71) --
	(250.08, 65.71) --
	(250.08, 65.71) --
	(250.08, 65.71) --
	(250.01, 65.71) --
	(250.01, 65.71) --
	(250.01, 65.71) --
	(249.94, 65.71) --
	(249.94, 65.71) --
	(249.94, 65.71) --
	(249.87, 65.71) --
	(249.87, 65.71) --
	(249.87, 65.71) --
	(249.80, 65.71) --
	(249.80, 65.71) --
	(249.80, 65.71) --
	(249.73, 65.71) --
	(249.73, 65.71) --
	(249.73, 65.71) --
	(249.66, 65.71) --
	(249.66, 65.71) --
	(249.66, 65.71) --
	(249.59, 65.71) --
	(249.59, 65.71) --
	(249.59, 65.71) --
	(249.52, 65.71) --
	(249.52, 65.71) --
	(249.52, 65.71) --
	(249.46, 65.71) --
	(249.46, 65.71) --
	(249.46, 65.71) --
	(249.39, 65.71) --
	(249.39, 65.71) --
	(249.39, 65.71) --
	(249.32, 65.71) --
	(249.32, 65.71) --
	(249.32, 65.71) --
	(249.25, 65.71) --
	(249.25, 65.71) --
	(249.25, 65.71) --
	(249.18, 65.71) --
	(249.18, 65.71) --
	(249.18, 65.71) --
	(249.11, 65.71) --
	(249.11, 65.71) --
	(249.11, 65.71) --
	(249.04, 65.71) --
	(249.04, 65.71) --
	(249.04, 65.71) --
	(248.97, 65.71) --
	(248.97, 65.71) --
	(248.97, 65.71) --
	(248.90, 65.71) --
	(248.90, 65.71) --
	(248.90, 65.71) --
	(248.83, 65.71) --
	(248.83, 65.71) --
	(248.83, 65.71) --
	(248.76, 65.71) --
	(248.76, 65.71) --
	(248.76, 65.71) --
	(248.69, 65.71) --
	(248.69, 65.71) --
	(248.69, 65.71) --
	(248.63, 65.71) --
	(248.63, 65.71) --
	(248.63, 65.71) --
	(248.56, 65.71) --
	(248.56, 65.71) --
	(248.56, 65.71) --
	(248.49, 65.71) --
	(248.49, 65.71) --
	(248.49, 65.71) --
	(248.42, 65.71) --
	(248.42, 65.71) --
	(248.42, 65.71) --
	(248.35, 65.71) --
	(248.35, 65.71) --
	(248.35, 65.71) --
	(248.28, 65.71) --
	(248.28, 65.71) --
	(248.28, 65.71) --
	(248.21, 65.71) --
	(248.21, 65.71) --
	(248.21, 65.71) --
	(248.14, 65.71) --
	(248.14, 65.71) --
	(248.14, 65.71) --
	(248.07, 65.71) --
	(248.07, 65.71) --
	(248.07, 65.71) --
	(248.00, 65.71) --
	(248.00, 65.71) --
	(248.00, 65.71) --
	(247.93, 65.71) --
	(247.93, 65.71) --
	(247.93, 65.71) --
	(247.86, 65.71) --
	(247.86, 65.71) --
	(247.86, 65.71) --
	(247.79, 65.71) --
	(247.79, 65.71) --
	(247.79, 65.71) --
	(247.72, 65.71) --
	(247.72, 65.71) --
	(247.72, 65.71) --
	(247.65, 65.71) --
	(247.65, 65.71) --
	(247.65, 65.71) --
	(247.59, 65.71) --
	(247.59, 65.71) --
	(247.59, 65.71) --
	(247.52, 65.71) --
	(247.52, 65.71) --
	(247.52, 65.71) --
	(247.45, 65.71) --
	(247.45, 65.71) --
	(247.45, 65.71) --
	(247.38, 65.71) --
	(247.38, 65.71) --
	(247.38, 65.71) --
	(247.31, 65.71) --
	(247.31, 65.71) --
	(247.31, 65.71) --
	(247.24, 65.71) --
	(247.24, 65.71) --
	(247.24, 65.71) --
	(247.17, 65.71) --
	(247.17, 65.71) --
	(247.17, 65.71) --
	(247.10, 65.71) --
	(247.10, 65.71) --
	(247.10, 65.71) --
	(247.03, 65.71) --
	(247.03, 65.71) --
	(247.03, 65.71) --
	(246.96, 65.71) --
	(246.96, 65.71) --
	(246.96, 65.71) --
	(246.89, 65.71) --
	(246.89, 65.71) --
	(246.89, 65.71) --
	(246.82, 65.71) --
	(246.82, 65.71) --
	(246.82, 65.71) --
	(246.75, 65.71) --
	(246.75, 65.71) --
	(246.75, 65.71) --
	(246.68, 65.71) --
	(246.68, 65.71) --
	(246.68, 65.71) --
	(246.61, 65.71) --
	(246.61, 65.71) --
	(246.61, 65.71) --
	(246.55, 65.71) --
	(246.55, 65.71) --
	(246.55, 65.71) --
	(246.48, 65.71) --
	(246.48, 65.71) --
	(246.48, 65.71) --
	(246.41, 65.71) --
	(246.41, 65.71) --
	(246.41, 65.71) --
	(246.34, 65.71) --
	(246.34, 65.71) --
	(246.34, 65.71) --
	(246.27, 65.71) --
	(246.27, 65.71) --
	(246.27, 65.71) --
	(246.20, 65.71) --
	(246.20, 65.71) --
	(246.20, 65.71) --
	(246.13, 65.71) --
	(246.13, 65.71) --
	(246.13, 65.71) --
	(246.06, 65.71) --
	(246.06, 65.71) --
	(246.06, 65.71) --
	(245.99, 65.71) --
	(245.99, 65.71) --
	(245.99, 65.71) --
	(245.92, 65.71) --
	(245.92, 65.71) --
	(245.92, 65.71) --
	(245.85, 65.71) --
	(245.85, 65.71) --
	(245.85, 65.71) --
	(245.78, 65.71) --
	(245.78, 65.71) --
	(245.78, 65.71) --
	(245.71, 65.71) --
	(245.71, 65.71) --
	(245.71, 65.71) --
	(245.64, 65.71) --
	(245.64, 65.71) --
	(245.64, 65.71) --
	(245.57, 65.71) --
	(245.57, 65.71) --
	(245.57, 65.71) --
	(245.51, 65.71) --
	(245.51, 65.71) --
	(245.51, 65.71) --
	(245.44, 65.71) --
	(245.44, 65.71) --
	(245.44, 65.71) --
	(245.37, 65.71) --
	(245.37, 65.71) --
	(245.37, 65.71) --
	(245.30, 65.71) --
	(245.30, 65.71) --
	(245.30, 65.71) --
	(245.23, 65.71) --
	(245.23, 65.71) --
	(245.23, 65.71) --
	(245.16, 65.71) --
	(245.16, 65.71) --
	(245.16, 65.71) --
	(245.09, 65.71) --
	(245.09, 65.71) --
	(245.09, 65.71) --
	(245.02, 65.71) --
	(245.02, 65.71) --
	(245.02, 65.71) --
	(244.95, 65.71) --
	(244.95, 65.71) --
	(244.95, 65.71) --
	(244.88, 65.71) --
	(244.88, 65.71) --
	(244.88, 65.71) --
	(244.81, 65.71) --
	(244.81, 65.71) --
	(244.81, 65.71) --
	(244.74, 65.71) --
	(244.74, 65.71) --
	(244.74, 65.71) --
	(244.67, 65.71) --
	(244.67, 65.71) --
	(244.67, 65.71) --
	(244.60, 65.71) --
	(244.60, 65.71) --
	(244.60, 65.71) --
	(244.53, 65.71) --
	(244.53, 65.71) --
	(244.53, 65.71) --
	(244.46, 65.71) --
	(244.46, 65.71) --
	(244.46, 65.71) --
	(244.39, 65.71) --
	(244.39, 65.71) --
	(244.39, 65.71) --
	(244.32, 65.71) --
	(244.32, 65.71) --
	(244.32, 65.71) --
	(244.25, 65.71) --
	(244.25, 65.71) --
	(244.25, 65.71) --
	(244.18, 65.71) --
	(244.18, 65.71) --
	(244.18, 65.71) --
	(244.11, 65.71) --
	(244.11, 65.71) --
	(244.11, 65.71) --
	(244.05, 65.71) --
	(244.05, 65.71) --
	(244.05, 65.71) --
	(243.98, 65.71) --
	(243.98, 65.71) --
	(243.98, 65.71) --
	(243.91, 65.71) --
	(243.91, 65.71) --
	(243.90, 65.71) --
	(243.84, 65.71) --
	(243.84, 65.71) --
	(243.84, 65.71) --
	(243.77, 65.71) --
	(243.77, 65.71) --
	(243.77, 65.71) --
	(243.70, 65.71) --
	(243.70, 65.71) --
	(243.70, 65.71) --
	(243.63, 65.71) --
	(243.63, 65.71) --
	(243.63, 65.71) --
	(243.56, 65.71) --
	(243.56, 65.71) --
	(243.56, 65.71) --
	(243.49, 65.71) --
	(243.49, 65.71) --
	(243.49, 65.71) --
	(243.42, 65.71) --
	(243.42, 65.71) --
	(243.42, 65.71) --
	(243.35, 65.71) --
	(243.35, 65.71) --
	(243.35, 65.71) --
	(243.28, 65.71) --
	(243.28, 65.71) --
	(243.28, 65.71) --
	(243.21, 65.71) --
	(243.21, 65.71) --
	(243.21, 65.71) --
	(243.14, 65.71) --
	(243.14, 65.71) --
	(243.14, 65.71) --
	(243.07, 65.71) --
	(243.07, 65.71) --
	(243.07, 65.71) --
	(243.00, 65.71) --
	(243.00, 65.71) --
	(243.00, 65.71) --
	(242.93, 65.71) --
	(242.93, 65.71) --
	(242.93, 65.71) --
	(242.86, 65.71) --
	(242.86, 65.71) --
	(242.86, 65.71) --
	(242.79, 65.71) --
	(242.79, 65.71) --
	(242.79, 65.71) --
	(242.72, 65.71) --
	(242.72, 65.71) --
	(242.72, 65.71) --
	(242.65, 65.71) --
	(242.65, 65.71) --
	(242.65, 65.71) --
	(242.58, 65.71) --
	(242.58, 65.71) --
	(242.58, 65.71) --
	(242.51, 65.71) --
	(242.51, 65.71) --
	(242.51, 65.71) --
	(242.44, 65.71) --
	(242.44, 65.71) --
	(242.44, 65.71) --
	(242.37, 65.71) --
	(242.37, 65.71) --
	(242.37, 65.71) --
	(242.30, 65.71) --
	(242.30, 65.71) --
	(242.30, 65.71) --
	(242.23, 65.71) --
	(242.23, 65.71) --
	(242.23, 65.71) --
	(242.16, 65.71) --
	(242.16, 65.71) --
	(242.16, 65.71) --
	(242.09, 65.71) --
	(242.09, 65.71) --
	(242.09, 65.71) --
	(242.03, 65.71) --
	(242.03, 65.71) --
	(242.03, 65.71) --
	(241.96, 65.71) --
	(241.96, 65.71) --
	(241.96, 65.71) --
	(241.89, 65.71) --
	(241.89, 65.71) --
	(241.89, 65.71) --
	(241.82, 65.71) --
	(241.82, 65.71) --
	(241.82, 65.71) --
	(241.75, 65.71) --
	(241.75, 65.71) --
	(241.75, 65.71) --
	(241.68, 65.71) --
	(241.68, 65.71) --
	(241.68, 65.71) --
	(241.61, 65.71) --
	(241.61, 65.71) --
	(241.61, 65.71) --
	(241.54, 65.71) --
	(241.54, 65.71) --
	(241.54, 65.71) --
	(241.47, 65.71) --
	(241.47, 65.71) --
	(241.47, 65.71) --
	(241.40, 65.71) --
	(241.40, 65.71) --
	(241.40, 65.71) --
	(241.33, 65.71) --
	(241.33, 65.71) --
	(241.33, 65.71) --
	(241.26, 65.71) --
	(241.26, 65.71) --
	(241.26, 65.71) --
	(241.19, 65.71) --
	(241.19, 65.71) --
	(241.19, 65.71) --
	(241.12, 65.71) --
	(241.12, 65.71) --
	(241.12, 65.71) --
	(241.05, 65.71) --
	(241.05, 65.71) --
	(241.05, 65.71) --
	(240.98, 65.71) --
	(240.98, 65.71) --
	(240.98, 65.71) --
	(240.91, 65.71) --
	(240.91, 65.71) --
	(240.91, 65.71) --
	(240.84, 65.71) --
	(240.84, 65.71) --
	(240.84, 65.71) --
	(240.77, 65.71) --
	(240.77, 65.71) --
	(240.77, 65.71) --
	(240.70, 65.71) --
	(240.70, 65.71) --
	(240.70, 65.71) --
	(240.63, 65.71) --
	(240.63, 65.71) --
	(240.63, 65.71) --
	(240.56, 65.71) --
	(240.56, 65.71) --
	(240.56, 65.71) --
	(240.56, 65.71) --
	(240.56, 65.71) --
	(240.56, 65.71) --
	(240.49, 65.71) --
	(240.49, 65.71) --
	(240.49, 65.71) --
	(240.42, 65.71) --
	(240.42, 65.71) --
	(240.42, 65.71) --
	(240.35, 65.71) --
	(240.35, 65.71) --
	(240.35, 65.71) --
	(240.28, 65.71) --
	(240.28, 65.71) --
	(240.28, 65.71) --
	(240.21, 65.71) --
	(240.21, 65.71) --
	(240.21, 65.71) --
	(240.14, 65.71) --
	(240.14, 65.71) --
	(240.14, 65.71) --
	(240.07, 65.71) --
	(240.07, 65.71) --
	(240.07, 65.71) --
	(240.00, 65.71) --
	(240.00, 65.71) --
	(240.00, 65.71) --
	(239.93, 65.71) --
	(239.93, 65.71) --
	(239.93, 65.71) --
	(239.86, 65.71) --
	(239.86, 65.71) --
	(239.86, 65.71) --
	(239.79, 65.71) --
	(239.79, 65.71) --
	(239.79, 65.71) --
	(239.72, 65.71) --
	(239.72, 65.71) --
	(239.72, 65.71) --
	(239.65, 65.71) --
	(239.65, 65.71) --
	(239.65, 65.71) --
	(239.58, 65.71) --
	(239.58, 65.71) --
	(239.58, 65.71) --
	(239.51, 65.71) --
	(239.51, 65.71) --
	(239.51, 65.71) --
	(239.44, 65.71) --
	(239.44, 65.71) --
	(239.44, 65.71) --
	(239.37, 65.71) --
	(239.37, 65.71) --
	(239.37, 65.71) --
	(239.30, 65.71) --
	(239.30, 65.71) --
	(239.30, 65.71) --
	(239.23, 65.71) --
	(239.23, 65.71) --
	(239.23, 65.71) --
	(239.16, 65.71) --
	(239.16, 65.71) --
	(239.16, 65.71) --
	(239.09, 65.71) --
	(239.09, 65.71) --
	(239.09, 65.71) --
	(239.02, 65.71) --
	(239.02, 65.71) --
	(239.02, 65.71) --
	(238.95, 65.71) --
	(238.95, 65.71) --
	(238.95, 65.71) --
	(238.88, 65.71) --
	(238.88, 65.71) --
	(238.88, 65.71) --
	(238.81, 65.71) --
	(238.81, 65.71) --
	(238.81, 65.71) --
	(238.74, 65.71) --
	(238.74, 65.71) --
	(238.74, 65.71) --
	(238.74, 65.71) --
	(238.74, 65.71) --
	(238.74, 65.71) --
	(238.67, 65.71) --
	(238.67, 65.71) --
	(238.67, 65.71) --
	(238.60, 65.71) --
	(238.60, 65.71) --
	(238.60, 65.71) --
	(238.53, 65.71) --
	(238.53, 65.71) --
	(238.53, 65.71) --
	(238.46, 65.71) --
	(238.46, 65.71) --
	(238.46, 65.71) --
	(238.39, 65.71) --
	(238.39, 65.71) --
	(238.39, 65.71) --
	(238.32, 65.71) --
	(238.32, 65.71) --
	(238.32, 65.71) --
	(238.25, 65.71) --
	(238.25, 65.71) --
	(238.25, 65.71) --
	(238.18, 65.71) --
	(238.18, 65.71) --
	(238.18, 65.71) --
	(238.11, 65.71) --
	(238.11, 65.71) --
	(238.11, 65.71) --
	(238.04, 65.71) --
	(238.04, 65.71) --
	(238.04, 65.71) --
	(237.97, 65.71) --
	(237.97, 65.71) --
	(237.97, 65.71) --
	(237.90, 65.71) --
	(237.90, 65.71) --
	(237.90, 65.71) --
	(237.83, 65.71) --
	(237.83, 65.71) --
	(237.83, 65.71) --
	(237.76, 65.71) --
	(237.76, 65.71) --
	(237.76, 65.71) --
	(237.69, 65.71) --
	(237.69, 65.71) --
	(237.69, 65.71) --
	(237.62, 65.71) --
	(237.62, 65.71) --
	(237.62, 65.71) --
	(237.55, 65.71) --
	(237.55, 65.71) --
	(237.55, 65.71) --
	(237.48, 65.71) --
	(237.48, 65.71) --
	(237.48, 65.71) --
	(237.41, 65.71) --
	(237.41, 65.71) --
	(237.41, 65.71) --
	(237.34, 65.71) --
	(237.34, 65.71) --
	(237.34, 65.71) --
	(237.27, 65.71) --
	(237.27, 65.71) --
	(237.27, 65.71) --
	(237.20, 65.71) --
	(237.20, 65.71) --
	(237.20, 65.71) --
	(237.13, 65.71) --
	(237.13, 65.71) --
	(237.13, 65.71) --
	(237.06, 65.71) --
	(237.06, 65.71) --
	(237.06, 65.71) --
	(236.99, 65.71) --
	(236.99, 65.71) --
	(236.99, 65.71) --
	(236.92, 65.71) --
	(236.92, 65.71) --
	(236.92, 65.71) --
	(236.85, 65.71) --
	(236.85, 65.71) --
	(236.85, 65.71) --
	(236.78, 65.71) --
	(236.78, 65.71) --
	(236.78, 65.71) --
	(236.71, 65.71) --
	(236.71, 65.71) --
	(236.71, 65.71) --
	(236.64, 65.71) --
	(236.64, 65.71) --
	(236.64, 65.71) --
	(236.57, 65.71) --
	(236.57, 65.71) --
	(236.57, 65.71) --
	(236.50, 65.71) --
	(236.50, 65.71) --
	(236.50, 65.71) --
	(236.43, 65.71) --
	(236.43, 65.71) --
	(236.43, 65.71) --
	(236.36, 65.71) --
	(236.36, 65.71) --
	(236.36, 65.71) --
	(236.29, 65.71) --
	(236.29, 65.71) --
	(236.29, 65.71) --
	(236.25, 65.71) --
	(236.25, 65.71) --
	(236.25, 65.71) --
	(236.22, 65.71) --
	(236.22, 65.71) --
	(236.22, 65.71) --
	(236.15, 65.71) --
	(236.15, 65.71) --
	(236.15, 65.71) --
	(236.08, 65.71) --
	(236.08, 65.71) --
	(236.08, 65.71) --
	(236.01, 65.71) --
	(236.01, 65.71) --
	(236.01, 65.71) --
	(235.94, 65.71) --
	(235.94, 65.71) --
	(235.94, 65.71) --
	(235.87, 65.71) --
	(235.87, 65.71) --
	(235.87, 65.71) --
	(235.80, 65.71) --
	(235.80, 65.71) --
	(235.80, 65.71) --
	(235.73, 65.71) --
	(235.73, 65.71) --
	(235.73, 65.71) --
	(235.66, 65.71) --
	(235.66, 65.71) --
	(235.66, 65.71) --
	(235.59, 65.71) --
	(235.59, 65.71) --
	(235.59, 65.71) --
	(235.52, 65.71) --
	(235.52, 65.71) --
	(235.52, 65.71) --
	(235.45, 65.71) --
	(235.45, 65.71) --
	(235.45, 65.71) --
	(235.38, 65.71) --
	(235.38, 65.71) --
	(235.38, 65.71) --
	(235.31, 65.71) --
	(235.31, 65.71) --
	(235.31, 65.71) --
	(235.24, 65.71) --
	(235.24, 65.71) --
	(235.24, 65.71) --
	(235.17, 65.71) --
	(235.17, 65.71) --
	(235.17, 65.71) --
	(235.10, 65.71) --
	(235.10, 65.71) --
	(235.10, 65.71) --
	(235.03, 65.71) --
	(235.03, 65.71) --
	(235.03, 65.71) --
	(234.96, 65.71) --
	(234.96, 65.71) --
	(234.96, 65.71) --
	(234.89, 65.71) --
	(234.89, 65.71) --
	(234.89, 65.71) --
	(234.82, 65.71) --
	(234.82, 65.71) --
	(234.82, 65.71) --
	(234.75, 65.71) --
	(234.75, 65.71) --
	(234.75, 65.71) --
	(234.68, 65.71) --
	(234.68, 65.71) --
	(234.68, 65.71) --
	(234.61, 65.71) --
	(234.61, 65.71) --
	(234.61, 65.71) --
	(234.54, 65.71) --
	(234.54, 65.71) --
	(234.54, 65.71) --
	(234.47, 65.71) --
	(234.47, 65.71) --
	(234.47, 65.71) --
	(234.40, 65.71) --
	(234.40, 65.71) --
	(234.40, 65.71) --
	(234.32, 65.71) --
	(234.32, 65.71) --
	(234.32, 65.71) --
	(234.26, 65.71) --
	(234.26, 65.71) --
	(234.25, 65.71) --
	(234.18, 65.71) --
	(234.18, 65.71) --
	(234.18, 65.71) --
	(234.11, 65.71) --
	(234.11, 65.71) --
	(234.11, 65.71) --
	(234.04, 65.71) --
	(234.04, 65.71) --
	(234.04, 65.71) --
	(233.97, 65.71) --
	(233.97, 65.71) --
	(233.97, 65.71) --
	(233.90, 65.71) --
	(233.90, 65.71) --
	(233.90, 65.71) --
	(233.83, 65.71) --
	(233.83, 65.71) --
	(233.83, 65.71) --
	(233.76, 65.71) --
	(233.76, 65.71) --
	(233.76, 65.71) --
	(233.69, 65.71) --
	(233.69, 65.71) --
	(233.69, 65.71) --
	(233.62, 65.71) --
	(233.62, 65.71) --
	(233.62, 65.71) --
	(233.55, 65.71) --
	(233.55, 65.71) --
	(233.55, 65.71) --
	(233.48, 65.71) --
	(233.48, 65.71) --
	(233.48, 65.71) --
	(233.41, 65.71) --
	(233.41, 65.71) --
	(233.41, 65.71) --
	(233.34, 65.71) --
	(233.34, 65.71) --
	(233.34, 65.71) --
	(233.28, 65.71) --
	(233.28, 65.71) --
	(233.28, 65.71) --
	(233.27, 65.71) --
	(233.27, 65.71) --
	(233.27, 65.71) --
	(233.20, 65.71) --
	(233.20, 65.71) --
	(233.20, 65.71) --
	(233.13, 65.71) --
	(233.13, 65.71) --
	(233.13, 65.71) --
	(233.06, 65.71) --
	(233.06, 65.71) --
	(233.06, 65.71) --
	(232.99, 65.71) --
	(232.99, 65.71) --
	(232.99, 65.71) --
	(232.92, 65.71) --
	(232.92, 65.71) --
	(232.92, 65.71) --
	(232.85, 65.71) --
	(232.85, 65.71) --
	(232.85, 65.71) --
	(232.78, 65.71) --
	(232.78, 65.71) --
	(232.78, 65.71) --
	(232.71, 65.71) --
	(232.71, 65.71) --
	(232.71, 65.71) --
	(232.64, 65.71) --
	(232.64, 65.71) --
	(232.64, 65.71) --
	(232.57, 65.71) --
	(232.57, 65.71) --
	(232.57, 65.71) --
	(232.50, 65.71) --
	(232.50, 65.71) --
	(232.50, 65.71) --
	(232.43, 65.71) --
	(232.43, 65.71) --
	(232.43, 65.71) --
	(232.36, 65.71) --
	(232.36, 65.71) --
	(232.36, 65.71) --
	(232.28, 65.71) --
	(232.28, 65.71) --
	(232.28, 65.71) --
	(232.21, 65.71) --
	(232.21, 65.71) --
	(232.21, 65.71) --
	(232.14, 65.71) --
	(232.14, 65.71) --
	(232.14, 65.71) --
	(232.07, 65.71) --
	(232.07, 65.71) --
	(232.07, 65.71) --
	(232.00, 65.71) --
	(232.00, 65.71) --
	(232.00, 65.71) --
	(231.93, 65.71) --
	(231.93, 65.71) --
	(231.93, 65.71) --
	(231.86, 65.71) --
	(231.86, 65.71) --
	(231.86, 65.71) --
	(231.79, 65.71) --
	(231.79, 65.71) --
	(231.79, 65.71) --
	(231.72, 65.71) --
	(231.72, 65.71) --
	(231.72, 65.71) --
	(231.65, 65.71) --
	(231.65, 65.71) --
	(231.65, 65.71) --
	(231.58, 65.71) --
	(231.58, 65.71) --
	(231.58, 65.71) --
	(231.51, 65.71) --
	(231.51, 65.71) --
	(231.51, 65.71) --
	(231.44, 65.71) --
	(231.44, 65.71) --
	(231.44, 65.71) --
	(231.37, 65.71) --
	(231.37, 65.71) --
	(231.37, 65.71) --
	(231.30, 65.71) --
	(231.30, 65.71) --
	(231.30, 65.71) --
	(231.23, 65.71) --
	(231.23, 65.71) --
	(231.23, 65.71) --
	(231.16, 65.71) --
	(231.16, 65.71) --
	(231.16, 65.71) --
	(231.09, 65.71) --
	(231.09, 65.71) --
	(231.09, 65.71) --
	(231.02, 65.71) --
	(231.02, 65.71) --
	(231.02, 65.71) --
	(230.95, 65.71) --
	(230.95, 65.71) --
	(230.95, 65.71) --
	(230.88, 65.71) --
	(230.88, 65.71) --
	(230.88, 65.71) --
	(230.80, 65.71) --
	(230.80, 65.71) --
	(230.80, 65.71) --
	(230.73, 65.71) --
	(230.73, 65.71) --
	(230.73, 65.71) --
	(230.66, 65.71) --
	(230.66, 65.71) --
	(230.66, 65.71) --
	(230.59, 65.71) --
	(230.59, 65.71) --
	(230.59, 65.71) --
	(230.52, 65.71) --
	(230.52, 65.71) --
	(230.52, 65.71) --
	(230.50, 65.71) --
	(230.50, 65.71) --
	(230.50, 65.71) --
	(230.45, 65.71) --
	(230.45, 65.71) --
	(230.45, 65.71) --
	(230.38, 65.71) --
	(230.38, 65.71) --
	(230.38, 65.71) --
	(230.31, 65.71) --
	(230.31, 65.71) --
	(230.31, 65.71) --
	(230.24, 65.71) --
	(230.24, 65.71) --
	(230.24, 65.71) --
	(230.17, 65.71) --
	(230.17, 65.71) --
	(230.17, 65.71) --
	(230.10, 65.71) --
	(230.10, 65.71) --
	(230.10, 65.71) --
	(230.03, 65.71) --
	(230.03, 65.71) --
	(230.03, 65.71) --
	(229.96, 65.71) --
	(229.96, 65.71) --
	(229.96, 65.71) --
	(229.89, 65.71) --
	(229.89, 65.71) --
	(229.89, 65.71) --
	(229.82, 65.71) --
	(229.82, 65.71) --
	(229.82, 65.71) --
	(229.75, 65.71) --
	(229.75, 65.71) --
	(229.75, 65.71) --
	(229.68, 65.71) --
	(229.68, 65.71) --
	(229.68, 65.71) --
	(229.60, 65.71) --
	(229.60, 65.71) --
	(229.60, 65.71) --
	(229.53, 65.71) --
	(229.53, 65.71) --
	(229.53, 65.71) --
	(229.46, 65.71) --
	(229.46, 65.71) --
	(229.46, 65.71) --
	(229.39, 65.71) --
	(229.39, 65.71) --
	(229.39, 65.71) --
	(229.32, 65.71) --
	(229.32, 65.71) --
	(229.32, 65.71) --
	(229.25, 65.71) --
	(229.25, 65.71) --
	(229.25, 65.71) --
	(229.18, 65.71) --
	(229.18, 65.71) --
	(229.18, 65.71) --
	(229.11, 65.71) --
	(229.11, 65.71) --
	(229.11, 65.71) --
	(229.04, 65.71) --
	(229.04, 65.71) --
	(229.04, 65.71) --
	(228.97, 65.71) --
	(228.97, 65.71) --
	(228.97, 65.71) --
	(228.90, 65.71) --
	(228.90, 65.71) --
	(228.90, 65.71) --
	(228.83, 65.71) --
	(228.83, 65.71) --
	(228.83, 65.71) --
	(228.76, 65.71) --
	(228.76, 65.71) --
	(228.76, 65.71) --
	(228.69, 65.71) --
	(228.69, 65.71) --
	(228.69, 65.71) --
	(228.62, 65.71) --
	(228.62, 65.71) --
	(228.62, 65.71) --
	(228.55, 65.71) --
	(228.55, 65.71) --
	(228.55, 65.71) --
	(228.47, 65.71) --
	(228.47, 65.71) --
	(228.47, 65.71) --
	(228.40, 65.71) --
	(228.40, 65.71) --
	(228.40, 65.71) --
	(228.33, 65.71) --
	(228.33, 65.71) --
	(228.33, 65.71) --
	(228.26, 65.71) --
	(228.26, 65.71) --
	(228.26, 65.71) --
	(228.19, 65.71) --
	(228.19, 65.71) --
	(228.19, 65.71) --
	(228.12, 65.71) --
	(228.12, 65.71) --
	(228.12, 65.71) --
	(228.05, 65.71) --
	(228.05, 65.71) --
	(228.05, 65.71) --
	(227.98, 65.71) --
	(227.98, 65.71) --
	(227.98, 65.71) --
	(227.91, 65.71) --
	(227.91, 65.71) --
	(227.91, 65.71) --
	(227.91, 65.71) --
	(227.91, 65.71) --
	(227.91, 65.71) --
	(227.84, 65.71) --
	(227.84, 65.71) --
	(227.84, 65.71) --
	(227.77, 65.71) --
	(227.77, 65.71) --
	(227.77, 65.71) --
	(227.70, 65.71) --
	(227.70, 65.71) --
	(227.70, 65.71) --
	(227.62, 65.71) --
	(227.62, 65.71) --
	(227.62, 65.71) --
	(227.56, 65.71) --
	(227.56, 65.71) --
	(227.56, 65.71) --
	(227.48, 65.71) --
	(227.48, 65.71) --
	(227.48, 65.71) --
	(227.41, 65.71) --
	(227.41, 65.71) --
	(227.41, 65.71) --
	(227.34, 65.71) --
	(227.34, 65.71) --
	(227.34, 65.71) --
	(227.27, 65.71) --
	(227.27, 65.71) --
	(227.27, 65.71) --
	(227.20, 65.71) --
	(227.20, 65.71) --
	(227.20, 65.71) --
	(227.13, 65.71) --
	(227.13, 65.71) --
	(227.13, 65.71) --
	(227.06, 65.71) --
	(227.06, 65.71) --
	(227.06, 65.71) --
	(226.99, 65.71) --
	(226.99, 65.71) --
	(226.99, 65.71) --
	(226.92, 65.71) --
	(226.92, 65.71) --
	(226.92, 65.71) --
	(226.85, 65.71) --
	(226.85, 65.71) --
	(226.85, 65.71) --
	(226.78, 65.71) --
	(226.78, 65.71) --
	(226.78, 65.71) --
	(226.71, 65.71) --
	(226.71, 65.71) --
	(226.71, 65.71) --
	(226.63, 65.71) --
	(226.63, 65.71) --
	(226.63, 65.71) --
	(226.56, 65.71) --
	(226.56, 65.71) --
	(226.56, 65.71) --
	(226.49, 65.71) --
	(226.49, 65.71) --
	(226.49, 65.71) --
	(226.42, 65.71) --
	(226.42, 65.71) --
	(226.42, 65.71) --
	(226.35, 65.71) --
	(226.35, 65.71) --
	(226.35, 65.71) --
	(226.28, 65.71) --
	(226.28, 65.71) --
	(226.28, 65.71) --
	(226.21, 65.71) --
	(226.21, 65.71) --
	(226.21, 65.71) --
	(226.14, 65.71) --
	(226.14, 65.71) --
	(226.14, 65.71) --
	(226.07, 65.71) --
	(226.07, 65.71) --
	(226.07, 65.71) --
	(226.00, 65.71) --
	(226.00, 65.71) --
	(226.00, 65.71) --
	(225.93, 65.71) --
	(225.93, 65.71) --
	(225.93, 65.71) --
	(225.85, 65.71) --
	(225.85, 65.71) --
	(225.85, 65.71) --
	(225.78, 65.71) --
	(225.78, 65.71) --
	(225.78, 65.71) --
	(225.71, 65.71) --
	(225.71, 65.71) --
	(225.71, 65.71) --
	(225.64, 65.71) --
	(225.64, 65.71) --
	(225.64, 65.71) --
	(225.61, 65.71) --
	(225.61, 65.71) --
	(225.61, 65.71) --
	(225.57, 65.71) --
	(225.57, 65.71) --
	(225.57, 65.71) --
	(225.50, 65.71) --
	(225.50, 65.71) --
	(225.50, 65.71) --
	(225.43, 65.71) --
	(225.43, 65.71) --
	(225.43, 65.71) --
	(225.36, 65.71) --
	(225.36, 65.71) --
	(225.36, 65.71) --
	(225.29, 65.71) --
	(225.29, 65.71) --
	(225.29, 65.71) --
	(225.22, 65.71) --
	(225.22, 65.71) --
	(225.22, 65.71) --
	(225.15, 65.71) --
	(225.15, 65.71) --
	(225.15, 65.71) --
	(225.07, 65.71) --
	(225.07, 65.71) --
	(225.07, 65.71) --
	(225.00, 65.71) --
	(225.00, 65.71) --
	(225.00, 65.71) --
	(224.93, 65.71) --
	(224.93, 65.71) --
	(224.93, 65.71) --
	(224.86, 65.71) --
	(224.86, 65.71) --
	(224.86, 65.71) --
	(224.79, 65.71) --
	(224.79, 65.71) --
	(224.79, 65.71) --
	(224.72, 65.71) --
	(224.72, 65.71) --
	(224.72, 65.71) --
	(224.65, 65.71) --
	(224.65, 65.71) --
	(224.65, 65.71) --
	(224.58, 65.71) --
	(224.58, 65.71) --
	(224.58, 65.71) --
	(224.51, 65.71) --
	(224.51, 65.71) --
	(224.51, 65.71) --
	(224.44, 65.71) --
	(224.44, 65.71) --
	(224.44, 65.71) --
	(224.36, 65.71) --
	(224.36, 65.71) --
	(224.36, 65.71) --
	(224.29, 65.71) --
	(224.29, 65.71) --
	(224.29, 65.71) --
	(224.22, 65.71) --
	(224.22, 65.71) --
	(224.22, 65.71) --
	(224.15, 65.71) --
	(224.15, 65.71) --
	(224.15, 65.71) --
	(224.08, 65.71) --
	(224.08, 65.71) --
	(224.08, 65.71) --
	(224.01, 65.71) --
	(224.01, 65.71) --
	(224.01, 65.71) --
	(223.94, 65.71) --
	(223.94, 65.71) --
	(223.94, 65.71) --
	(223.87, 65.71) --
	(223.87, 65.71) --
	(223.87, 65.71) --
	(223.80, 65.71) --
	(223.80, 65.71) --
	(223.80, 65.71) --
	(223.79, 65.71) --
	(223.79, 65.71) --
	(223.79, 65.71) --
	(223.73, 65.71) --
	(223.73, 65.71) --
	(223.73, 65.71) --
	(223.66, 65.71) --
	(223.66, 65.71) --
	(223.66, 65.71) --
	(223.58, 65.71) --
	(223.58, 65.71) --
	(223.58, 65.71) --
	(223.51, 65.71) --
	(223.51, 65.71) --
	(223.51, 65.71) --
	(223.44, 65.71) --
	(223.44, 65.71) --
	(223.44, 65.71) --
	(223.37, 65.71) --
	(223.37, 65.71) --
	(223.37, 65.71) --
	(223.30, 65.71) --
	(223.30, 65.71) --
	(223.30, 65.71) --
	(223.23, 65.71) --
	(223.23, 65.71) --
	(223.23, 65.71) --
	(223.16, 65.71) --
	(223.16, 65.71) --
	(223.16, 65.71) --
	(223.09, 65.71) --
	(223.09, 65.71) --
	(223.09, 65.71) --
	(223.02, 65.71) --
	(223.02, 65.71) --
	(223.02, 65.71) --
	(222.94, 65.71) --
	(222.94, 65.71) --
	(222.94, 65.71) --
	(222.87, 65.71) --
	(222.87, 65.71) --
	(222.87, 65.71) --
	(222.80, 65.71) --
	(222.80, 65.71) --
	(222.80, 65.71) --
	(222.73, 65.71) --
	(222.73, 65.71) --
	(222.73, 65.71) --
	(222.66, 65.71) --
	(222.66, 65.71) --
	(222.66, 65.71) --
	(222.59, 65.71) --
	(222.59, 65.71) --
	(222.59, 65.71) --
	(222.52, 65.71) --
	(222.52, 65.71) --
	(222.52, 65.71) --
	(222.45, 65.71) --
	(222.45, 65.71) --
	(222.45, 65.71) --
	(222.37, 65.71) --
	(222.37, 65.71) --
	(222.37, 65.71) --
	(222.30, 65.71) --
	(222.30, 65.71) --
	(222.30, 65.71) --
	(222.23, 65.71) --
	(222.23, 65.71) --
	(222.23, 65.71) --
	(222.16, 65.71) --
	(222.16, 65.71) --
	(222.16, 65.71) --
	(222.09, 65.71) --
	(222.09, 65.71) --
	(222.09, 65.71) --
	(222.02, 65.71) --
	(222.02, 65.71) --
	(222.02, 65.71) --
	(221.95, 65.71) --
	(221.95, 65.71) --
	(221.95, 65.71) --
	(221.88, 65.71) --
	(221.88, 65.71) --
	(221.88, 65.71) --
	(221.81, 65.71) --
	(221.81, 65.71) --
	(221.81, 65.71) --
	(221.74, 65.71) --
	(221.74, 65.71) --
	(221.74, 65.71) --
	(221.66, 65.71) --
	(221.66, 65.71) --
	(221.66, 65.71) --
	(221.59, 65.71) --
	(221.59, 65.71) --
	(221.59, 65.71) --
	(221.52, 65.71) --
	(221.52, 65.71) --
	(221.52, 65.71) --
	(221.49, 65.71) --
	(221.49, 65.71) --
	(221.49, 65.71) --
	(221.45, 65.71) --
	(221.45, 65.71) --
	(221.45, 65.71) --
	(221.38, 65.71) --
	(221.38, 65.71) --
	(221.38, 65.71) --
	(221.31, 65.71) --
	(221.31, 65.71) --
	(221.31, 65.71) --
	(221.24, 65.71) --
	(221.24, 65.71) --
	(221.24, 65.71) --
	(221.17, 65.71) --
	(221.17, 65.71) --
	(221.17, 65.71) --
	(221.09, 65.71) --
	(221.09, 65.71) --
	(221.09, 65.71) --
	(221.02, 65.71) --
	(221.02, 65.71) --
	(221.02, 65.71) --
	(220.95, 65.71) --
	(220.95, 65.71) --
	(220.95, 65.71) --
	(220.88, 65.71) --
	(220.88, 65.71) --
	(220.88, 65.71) --
	(220.81, 65.71) --
	(220.81, 65.71) --
	(220.81, 65.71) --
	(220.74, 65.71) --
	(220.74, 65.71) --
	(220.74, 65.71) --
	(220.67, 65.71) --
	(220.67, 65.71) --
	(220.67, 65.71) --
	(220.60, 65.71) --
	(220.60, 65.71) --
	(220.60, 65.71) --
	(220.52, 65.71) --
	(220.52, 65.71) --
	(220.52, 65.71) --
	(220.45, 65.71) --
	(220.45, 65.71) --
	(220.45, 65.71) --
	(220.38, 65.71) --
	(220.38, 65.71) --
	(220.38, 65.71) --
	(220.31, 65.71) --
	(220.31, 65.71) --
	(220.31, 65.71) --
	(220.24, 65.71) --
	(220.24, 65.71) --
	(220.24, 65.71) --
	(220.17, 65.71) --
	(220.17, 65.71) --
	(220.17, 65.71) --
	(220.10, 65.71) --
	(220.10, 65.71) --
	(220.10, 65.71) --
	(220.03, 65.71) --
	(220.03, 65.71) --
	(220.03, 65.71) --
	(219.95, 65.71) --
	(219.95, 65.71) --
	(219.95, 65.71) --
	(219.88, 65.71) --
	(219.88, 65.71) --
	(219.88, 65.71) --
	(219.81, 65.71) --
	(219.81, 65.71) --
	(219.81, 65.71) --
	(219.74, 65.71) --
	(219.74, 65.71) --
	(219.74, 65.71) --
	(219.67, 65.71) --
	(219.67, 65.71) --
	(219.67, 65.71) --
	(219.60, 65.71) --
	(219.60, 65.71) --
	(219.60, 65.71) --
	(219.53, 65.71) --
	(219.53, 65.71) --
	(219.53, 65.71) --
	(219.48, 65.71) --
	(219.48, 65.71) --
	(219.48, 65.71) --
	(219.46, 65.71) --
	(219.46, 65.71) --
	(219.46, 65.71) --
	(219.38, 65.71) --
	(219.38, 65.71) --
	(219.38, 65.71) --
	(219.31, 65.71) --
	(219.31, 65.71) --
	(219.31, 65.71) --
	(219.24, 65.71) --
	(219.24, 65.71) --
	(219.24, 65.71) --
	(219.17, 65.71) --
	(219.17, 65.71) --
	(219.17, 65.71) --
	(219.10, 65.71) --
	(219.10, 65.71) --
	(219.10, 65.71) --
	(219.03, 65.71) --
	(219.03, 65.71) --
	(219.03, 65.71) --
	(218.95, 65.71) --
	(218.95, 65.71) --
	(218.95, 65.71) --
	(218.88, 65.71) --
	(218.88, 65.71) --
	(218.88, 65.71) --
	(218.81, 65.71) --
	(218.81, 65.71) --
	(218.81, 65.71) --
	(218.74, 65.71) --
	(218.74, 65.71) --
	(218.74, 65.71) --
	(218.67, 65.71) --
	(218.67, 65.71) --
	(218.67, 65.71) --
	(218.60, 65.71) --
	(218.60, 65.71) --
	(218.60, 65.71) --
	(218.53, 65.71) --
	(218.53, 65.71) --
	(218.53, 65.71) --
	(218.46, 65.71) --
	(218.46, 65.71) --
	(218.46, 65.71) --
	(218.38, 65.71) --
	(218.38, 65.71) --
	(218.38, 65.71) --
	(218.31, 65.71) --
	(218.31, 65.71) --
	(218.31, 65.71) --
	(218.24, 65.71) --
	(218.24, 65.71) --
	(218.24, 65.71) --
	(218.17, 65.71) --
	(218.17, 65.71) --
	(218.17, 65.71) --
	(218.10, 65.71) --
	(218.10, 65.71) --
	(218.10, 65.71) --
	(218.03, 65.71) --
	(218.03, 65.71) --
	(218.03, 65.71) --
	(217.96, 65.71) --
	(217.96, 65.71) --
	(217.96, 65.71) --
	(217.88, 65.71) --
	(217.88, 65.71) --
	(217.88, 65.71) --
	(217.81, 65.71) --
	(217.81, 65.71) --
	(217.81, 65.71) --
	(217.76, 65.71) --
	(217.76, 65.71) --
	(217.76, 65.71) --
	(217.74, 65.71) --
	(217.74, 65.71) --
	(217.74, 65.71) --
	(217.67, 65.71) --
	(217.67, 65.71) --
	(217.67, 65.71) --
	(217.60, 65.71) --
	(217.60, 65.71) --
	(217.60, 65.71) --
	(217.53, 65.71) --
	(217.53, 65.71) --
	(217.53, 65.71) --
	(217.46, 65.71) --
	(217.46, 65.71) --
	(217.46, 65.71) --
	(217.38, 65.71) --
	(217.38, 65.71) --
	(217.38, 65.71) --
	(217.31, 65.71) --
	(217.31, 65.71) --
	(217.31, 65.71) --
	(217.24, 65.71) --
	(217.24, 65.71) --
	(217.24, 65.71) --
	(217.17, 65.71) --
	(217.17, 65.71) --
	(217.17, 65.71) --
	(217.10, 65.71) --
	(217.10, 65.71) --
	(217.10, 65.71) --
	(217.03, 65.71) --
	(217.03, 65.71) --
	(217.03, 65.71) --
	(216.96, 65.71) --
	(216.96, 65.71) --
	(216.96, 65.71) --
	(216.88, 65.71) --
	(216.88, 65.71) --
	(216.88, 65.71) --
	(216.81, 65.71) --
	(216.81, 65.71) --
	(216.81, 65.71) --
	(216.74, 65.71) --
	(216.74, 65.71) --
	(216.74, 65.71) --
	(216.67, 65.71) --
	(216.67, 65.71) --
	(216.67, 65.71) --
	(216.60, 65.71) --
	(216.60, 65.71) --
	(216.60, 65.71) --
	(216.53, 65.71) --
	(216.53, 65.71) --
	(216.53, 65.71) --
	(216.46, 65.71) --
	(216.46, 65.71) --
	(216.46, 65.71) --
	(216.38, 65.71) --
	(216.38, 65.71) --
	(216.38, 65.71) --
	(216.31, 65.71) --
	(216.31, 65.71) --
	(216.31, 65.71) --
	(216.24, 65.71) --
	(216.24, 65.71) --
	(216.24, 65.71) --
	(216.17, 65.71) --
	(216.17, 65.71) --
	(216.17, 65.71) --
	(216.10, 65.71) --
	(216.10, 65.71) --
	(216.10, 65.71) --
	(216.03, 65.71) --
	(216.03, 65.71) --
	(216.03, 65.71) --
	(215.95, 65.71) --
	(215.95, 65.71) --
	(215.95, 65.71) --
	(215.88, 65.71) --
	(215.88, 65.71) --
	(215.88, 65.71) --
	(215.81, 65.71) --
	(215.81, 65.71) --
	(215.81, 65.71) --
	(215.74, 65.71) --
	(215.74, 65.71) --
	(215.74, 65.71) --
	(215.74, 65.71) --
	(215.74, 65.71) --
	(215.74, 65.71) --
	(215.67, 65.71) --
	(215.67, 65.71) --
	(215.67, 65.71) --
	(215.60, 65.71) --
	(215.60, 65.71) --
	(215.60, 65.71) --
	(215.52, 65.71) --
	(215.52, 65.71) --
	(215.52, 65.71) --
	(215.45, 65.71) --
	(215.45, 65.71) --
	(215.45, 65.71) --
	(215.38, 65.71) --
	(215.38, 65.71) --
	(215.38, 65.71) --
	(215.31, 65.71) --
	(215.31, 65.71) --
	(215.31, 65.71) --
	(215.24, 65.71) --
	(215.24, 65.71) --
	(215.24, 65.71) --
	(215.17, 65.71) --
	(215.17, 65.71) --
	(215.17, 65.71) --
	(215.09, 65.71) --
	(215.09, 65.71) --
	(215.09, 65.71) --
	(215.02, 65.71) --
	(215.02, 65.71) --
	(215.02, 65.71) --
	(214.95, 65.71) --
	(214.95, 65.71) --
	(214.95, 65.71) --
	(214.88, 65.71) --
	(214.88, 65.71) --
	(214.88, 65.71) --
	(214.81, 65.71) --
	(214.81, 65.71) --
	(214.81, 65.71) --
	(214.74, 65.71) --
	(214.74, 65.71) --
	(214.74, 65.71) --
	(214.66, 65.71) --
	(214.66, 65.71) --
	(214.66, 65.71) --
	(214.59, 65.71) --
	(214.59, 65.71) --
	(214.59, 65.71) --
	(214.52, 65.71) --
	(214.52, 65.71) --
	(214.52, 65.71) --
	(214.45, 65.71) --
	(214.45, 65.71) --
	(214.45, 65.71) --
	(214.38, 65.71) --
	(214.38, 65.71) --
	(214.38, 65.71) --
	(214.31, 65.71) --
	(214.31, 65.71) --
	(214.31, 65.71) --
	(214.31, 65.71) --
	(214.31, 65.71) --
	(214.31, 65.71) --
	(214.23, 65.71) --
	(214.23, 65.71) --
	(214.23, 65.71) --
	(214.16, 65.71) --
	(214.16, 65.71) --
	(214.16, 65.71) --
	(214.09, 65.71) --
	(214.09, 65.71) --
	(214.09, 65.71) --
	(214.02, 65.71) --
	(214.02, 65.71) --
	(214.02, 65.71) --
	(213.95, 65.71) --
	(213.95, 65.71) --
	(213.95, 65.71) --
	(213.88, 65.71) --
	(213.88, 65.71) --
	(213.88, 65.71) --
	(213.80, 65.71) --
	(213.80, 65.71) --
	(213.80, 65.71) --
	(213.73, 65.71) --
	(213.73, 65.71) --
	(213.73, 65.71) --
	(213.66, 65.71) --
	(213.66, 65.71) --
	(213.66, 65.71) --
	(213.59, 65.71) --
	(213.59, 65.71) --
	(213.59, 65.71) --
	(213.52, 65.71) --
	(213.52, 65.71) --
	(213.52, 65.71) --
	(213.45, 65.71) --
	(213.45, 65.71) --
	(213.45, 65.71) --
	(213.37, 65.71) --
	(213.37, 65.71) --
	(213.37, 65.71) --
	(213.30, 65.71) --
	(213.30, 65.71) --
	(213.30, 65.71) --
	(213.23, 65.71) --
	(213.23, 65.71) --
	(213.23, 65.71) --
	(213.16, 65.71) --
	(213.16, 65.71) --
	(213.16, 65.71) --
	(213.09, 65.71) --
	(213.09, 65.71) --
	(213.09, 65.71) --
	(213.02, 65.71) --
	(213.02, 65.71) --
	(213.02, 65.71) --
	(212.96, 65.71) --
	(212.96, 65.71) --
	(212.96, 65.71) --
	(212.94, 65.71) --
	(212.94, 65.71) --
	(212.94, 65.71) --
	(212.87, 65.71) --
	(212.87, 65.71) --
	(212.87, 65.71) --
	(212.80, 65.71) --
	(212.80, 65.71) --
	(212.80, 65.71) --
	(212.73, 65.71) --
	(212.73, 65.71) --
	(212.73, 65.71) --
	(212.66, 65.71) --
	(212.66, 65.71) --
	(212.66, 65.71) --
	(212.58, 65.71) --
	(212.58, 65.71) --
	(212.58, 65.71) --
	(212.51, 65.71) --
	(212.51, 65.71) --
	(212.51, 65.71) --
	(212.44, 65.71) --
	(212.44, 65.71) --
	(212.44, 65.71) --
	(212.37, 65.71) --
	(212.37, 65.71) --
	(212.37, 65.71) --
	(212.30, 65.71) --
	(212.30, 65.71) --
	(212.30, 65.71) --
	(212.23, 65.71) --
	(212.23, 65.71) --
	(212.23, 65.71) --
	(212.15, 65.71) --
	(212.15, 65.71) --
	(212.15, 65.71) --
	(212.08, 65.71) --
	(212.08, 65.71) --
	(212.08, 65.71) --
	(212.01, 65.71) --
	(212.01, 65.71) --
	(212.01, 65.71) --
	(211.94, 65.71) --
	(211.94, 65.71) --
	(211.94, 65.71) --
	(211.87, 65.71) --
	(211.87, 65.71) --
	(211.87, 65.71) --
	(211.79, 65.71) --
	(211.79, 65.71) --
	(211.79, 65.71) --
	(211.72, 65.71) --
	(211.72, 65.71) --
	(211.72, 65.71) --
	(211.65, 65.71) --
	(211.65, 65.71) --
	(211.65, 65.71) --
	(211.58, 65.71) --
	(211.58, 65.71) --
	(211.58, 65.71) --
	(211.51, 65.71) --
	(211.51, 65.71) --
	(211.51, 65.71) --
	(211.43, 65.71) --
	(211.43, 65.71) --
	(211.43, 65.71) --
	(211.43, 65.71) --
	(211.43, 65.71) --
	(211.43, 65.71) --
	(211.36, 65.71) --
	(211.36, 65.71) --
	(211.36, 65.71) --
	(211.29, 65.71) --
	(211.29, 65.71) --
	(211.29, 65.71) --
	(211.22, 65.71) --
	(211.22, 65.71) --
	(211.22, 65.71) --
	(211.15, 65.71) --
	(211.15, 65.71) --
	(211.15, 65.71) --
	(211.08, 65.71) --
	(211.08, 65.71) --
	(211.08, 65.71) --
	(211.00, 65.71) --
	(211.00, 65.71) --
	(211.00, 65.71) --
	(210.93, 65.71) --
	(210.93, 65.71) --
	(210.93, 65.71) --
	(210.86, 65.71) --
	(210.86, 65.71) --
	(210.86, 65.71) --
	(210.79, 65.71) --
	(210.79, 65.71) --
	(210.79, 65.71) --
	(210.72, 65.71) --
	(210.72, 65.71) --
	(210.72, 65.71) --
	(210.65, 65.71) --
	(210.65, 65.71) --
	(210.65, 65.71) --
	(210.57, 65.71) --
	(210.57, 65.71) --
	(210.57, 65.71) --
	(210.50, 65.71) --
	(210.50, 65.71) --
	(210.50, 65.71) --
	(210.43, 65.71) --
	(210.43, 65.71) --
	(210.43, 65.71) --
	(210.36, 65.71) --
	(210.36, 65.71) --
	(210.36, 65.71) --
	(210.29, 65.71) --
	(210.29, 65.71) --
	(210.29, 65.71) --
	(210.28, 65.71) --
	(210.28, 65.71) --
	(210.28, 65.71) --
	(210.21, 65.71) --
	(210.21, 65.71) --
	(210.21, 65.71) --
	(210.14, 65.71) --
	(210.14, 65.71) --
	(210.14, 65.71) --
	(210.07, 65.71) --
	(210.07, 65.71) --
	(210.07, 65.71) --
	(210.00, 65.71) --
	(210.00, 65.71) --
	(210.00, 65.71) --
	(209.92, 65.71) --
	(209.92, 65.71) --
	(209.92, 65.71) --
	(209.85, 65.71) --
	(209.85, 65.71) --
	(209.85, 65.71) --
	(209.78, 65.71) --
	(209.78, 65.71) --
	(209.78, 65.71) --
	(209.71, 65.71) --
	(209.71, 65.71) --
	(209.71, 65.71) --
	(209.64, 65.71) --
	(209.64, 65.71) --
	(209.64, 65.71) --
	(209.56, 65.71) --
	(209.56, 65.71) --
	(209.56, 65.71) --
	(209.49, 65.71) --
	(209.49, 65.71) --
	(209.49, 65.71) --
	(209.42, 65.71) --
	(209.42, 65.71) --
	(209.42, 65.71) --
	(209.35, 65.71) --
	(209.35, 65.71) --
	(209.35, 65.71) --
	(209.32, 65.71) --
	(209.32, 65.71) --
	(209.32, 65.71) --
	(209.28, 65.71) --
	(209.28, 65.71) --
	(209.28, 65.71) --
	(209.20, 65.71) --
	(209.20, 65.71) --
	(209.20, 65.71) --
	(209.13, 65.71) --
	(209.13, 65.71) --
	(209.13, 65.71) --
	(209.06, 65.71) --
	(209.06, 65.71) --
	(209.06, 65.71) --
	(208.99, 65.71) --
	(208.99, 65.71) --
	(208.99, 65.71) --
	(208.92, 65.71) --
	(208.92, 65.71) --
	(208.92, 65.71) --
	(208.84, 65.71) --
	(208.84, 65.71) --
	(208.84, 65.71) --
	(208.77, 65.71) --
	(208.77, 65.71) --
	(208.77, 65.71) --
	(208.70, 65.71) --
	(208.70, 65.71) --
	(208.70, 65.71) --
	(208.63, 65.71) --
	(208.63, 65.71) --
	(208.63, 65.71) --
	(208.56, 65.71) --
	(208.56, 65.71) --
	(208.56, 65.71) --
	(208.48, 65.71) --
	(208.48, 65.71) --
	(208.48, 65.71) --
	(208.46, 65.71) --
	(208.46, 65.71) --
	(208.46, 65.71) --
	(208.41, 65.71) --
	(208.41, 65.71) --
	(208.41, 65.71) --
	(208.34, 65.71) --
	(208.34, 65.71) --
	(208.34, 65.71) --
	(208.27, 65.71) --
	(208.27, 65.71) --
	(208.27, 65.71) --
	(208.20, 65.71) --
	(208.20, 65.71) --
	(208.20, 65.71) --
	(208.12, 65.71) --
	(208.12, 65.71) --
	(208.12, 65.71) --
	(208.05, 65.71) --
	(208.05, 65.71) --
	(208.05, 65.71) --
	(207.98, 65.71) --
	(207.98, 65.71) --
	(207.98, 65.71) --
	(207.91, 65.71) --
	(207.91, 65.71) --
	(207.91, 65.71) --
	(207.84, 65.71) --
	(207.84, 65.71) --
	(207.84, 65.71) --
	(207.76, 65.71) --
	(207.76, 65.71) --
	(207.76, 65.71) --
	(207.69, 65.71) --
	(207.69, 65.71) --
	(207.69, 65.71) --
	(207.62, 65.71) --
	(207.62, 65.71) --
	(207.62, 65.71) --
	(207.60, 65.71) --
	(207.60, 65.71) --
	(207.60, 65.71) --
	(207.55, 65.71) --
	(207.55, 65.71) --
	(207.55, 65.71) --
	(207.48, 65.71) --
	(207.48, 65.71) --
	(207.48, 65.71) --
	(207.40, 65.71) --
	(207.40, 65.71) --
	(207.40, 65.71) --
	(207.33, 65.71) --
	(207.33, 65.71) --
	(207.33, 65.71) --
	(207.26, 65.71) --
	(207.26, 65.71) --
	(207.26, 65.71) --
	(207.19, 65.71) --
	(207.19, 65.71) --
	(207.19, 65.71) --
	(207.12, 65.71) --
	(207.12, 65.71) --
	(207.12, 65.71) --
	(207.04, 65.71) --
	(207.04, 65.71) --
	(207.04, 65.71) --
	(206.97, 65.71) --
	(206.97, 65.71) --
	(206.97, 65.71) --
	(206.90, 65.71) --
	(206.90, 65.71) --
	(206.90, 65.71) --
	(206.83, 65.71) --
	(206.83, 65.71) --
	(206.83, 65.71) --
	(206.75, 65.71) --
	(206.75, 65.71) --
	(206.75, 65.71) --
	(206.68, 65.71) --
	(206.68, 65.71) --
	(206.68, 65.71) --
	(206.61, 65.71) --
	(206.61, 65.71) --
	(206.61, 65.71) --
	(206.54, 65.71) --
	(206.54, 65.71) --
	(206.54, 65.71) --
	(206.54, 65.71) --
	(206.54, 65.71) --
	(206.54, 65.71) --
	(206.46, 65.71) --
	(206.46, 65.71) --
	(206.46, 65.71) --
	(206.39, 65.71) --
	(206.39, 65.71) --
	(206.39, 65.71) --
	(206.32, 65.71) --
	(206.32, 65.71) --
	(206.32, 65.71) --
	(206.25, 65.71) --
	(206.25, 65.71) --
	(206.25, 65.71) --
	(206.18, 65.71) --
	(206.18, 65.71) --
	(206.18, 65.71) --
	(206.10, 65.71) --
	(206.10, 65.71) --
	(206.10, 65.71) --
	(206.03, 65.71) --
	(206.03, 65.71) --
	(206.03, 65.71) --
	(205.96, 65.71) --
	(205.96, 65.71) --
	(205.96, 65.71) --
	(205.89, 65.71) --
	(205.89, 65.71) --
	(205.89, 65.71) --
	(205.82, 65.71) --
	(205.82, 65.71) --
	(205.82, 65.71) --
	(205.74, 65.71) --
	(205.74, 65.71) --
	(205.74, 65.71) --
	(205.67, 65.71) --
	(205.67, 65.71) --
	(205.67, 65.71) --
	(205.60, 65.71) --
	(205.60, 65.71) --
	(205.60, 65.71) --
	(205.58, 65.71) --
	(205.58, 65.71) --
	(205.58, 65.71) --
	(205.53, 65.71) --
	(205.53, 65.71) --
	(205.53, 65.71) --
	(205.45, 65.71) --
	(205.45, 65.71) --
	(205.45, 65.71) --
	(205.38, 65.71) --
	(205.38, 65.71) --
	(205.38, 65.71) --
	(205.31, 65.71) --
	(205.31, 65.71) --
	(205.31, 65.71) --
	(205.24, 65.71) --
	(205.24, 65.71) --
	(205.24, 65.71) --
	(205.16, 65.71) --
	(205.16, 65.71) --
	(205.16, 65.71) --
	(205.09, 65.71) --
	(205.09, 65.71) --
	(205.09, 65.71) --
	(205.02, 65.71) --
	(205.02, 65.71) --
	(205.02, 65.71) --
	(204.95, 65.71) --
	(204.95, 65.71) --
	(204.95, 65.71) --
	(204.88, 65.71) --
	(204.88, 65.71) --
	(204.88, 65.71) --
	(204.80, 65.71) --
	(204.80, 65.71) --
	(204.80, 65.71) --
	(204.73, 65.71) --
	(204.73, 65.71) --
	(204.73, 65.71) --
	(204.66, 65.71) --
	(204.66, 65.71) --
	(204.66, 65.71) --
	(204.63, 65.71) --
	(204.63, 65.71) --
	(204.63, 65.71) --
	(204.59, 65.71) --
	(204.59, 65.71) --
	(204.59, 65.71) --
	(204.51, 65.71) --
	(204.51, 65.71) --
	(204.51, 65.71) --
	(204.44, 65.71) --
	(204.44, 65.71) --
	(204.44, 65.71) --
	(204.37, 65.71) --
	(204.37, 65.71) --
	(204.37, 65.71) --
	(204.30, 65.71) --
	(204.30, 65.71) --
	(204.30, 65.71) --
	(204.22, 65.71) --
	(204.22, 65.71) --
	(204.22, 65.71) --
	(204.15, 65.71) --
	(204.15, 65.71) --
	(204.15, 65.71) --
	(204.08, 65.71) --
	(204.08, 65.71) --
	(204.08, 65.71) --
	(204.01, 65.71) --
	(204.01, 65.71) --
	(204.01, 65.71) --
	(203.94, 65.71) --
	(203.94, 65.71) --
	(203.94, 65.71) --
	(203.86, 65.71) --
	(203.86, 65.71) --
	(203.86, 65.71) --
	(203.79, 65.71) --
	(203.79, 65.71) --
	(203.79, 65.71) --
	(203.76, 65.71) --
	(203.76, 65.71) --
	(203.76, 65.71) --
	(203.72, 65.71) --
	(203.72, 65.71) --
	(203.72, 65.71) --
	(203.65, 65.71) --
	(203.65, 65.71) --
	(203.64, 65.71) --
	(203.57, 65.71) --
	(203.57, 65.71) --
	(203.57, 65.71) --
	(203.50, 65.71) --
	(203.50, 65.71) --
	(203.50, 65.71) --
	(203.43, 65.71) --
	(203.43, 65.71) --
	(203.43, 65.71) --
	(203.36, 65.71) --
	(203.36, 65.71) --
	(203.36, 65.71) --
	(203.28, 65.71) --
	(203.28, 65.71) --
	(203.28, 65.71) --
	(203.21, 65.71) --
	(203.21, 65.71) --
	(203.21, 65.71) --
	(203.14, 65.71) --
	(203.14, 65.71) --
	(203.14, 65.71) --
	(203.07, 65.71) --
	(203.07, 65.71) --
	(203.07, 65.71) --
	(202.99, 65.71) --
	(202.99, 65.71) --
	(202.99, 65.71) --
	(202.92, 65.71) --
	(202.92, 65.71) --
	(202.92, 65.71) --
	(202.85, 65.71) --
	(202.85, 65.71) --
	(202.85, 65.71) --
	(202.81, 65.71) --
	(202.81, 65.71) --
	(202.81, 65.71) --
	(202.78, 65.71) --
	(202.78, 65.71) --
	(202.78, 65.71) --
	(202.70, 65.71) --
	(202.70, 65.71) --
	(202.70, 65.71) --
	(202.63, 65.71) --
	(202.63, 65.71) --
	(202.63, 65.71) --
	(202.56, 65.71) --
	(202.56, 65.71) --
	(202.56, 65.71) --
	(202.49, 65.71) --
	(202.49, 65.71) --
	(202.49, 65.71) --
	(202.41, 65.71) --
	(202.41, 65.71) --
	(202.41, 65.71) --
	(202.34, 65.71) --
	(202.34, 65.71) --
	(202.34, 65.71) --
	(202.27, 65.71) --
	(202.27, 65.71) --
	(202.27, 65.71) --
	(202.20, 65.71) --
	(202.20, 65.71) --
	(202.20, 65.71) --
	(202.12, 65.71) --
	(202.12, 65.71) --
	(202.12, 65.71) --
	(202.05, 65.71) --
	(202.05, 65.71) --
	(202.05, 65.71) --
	(202.04, 65.71) --
	(202.04, 65.71) --
	(202.04, 65.71) --
	(201.98, 65.71) --
	(201.98, 65.71) --
	(201.98, 65.71) --
	(201.91, 65.71) --
	(201.91, 65.71) --
	(201.91, 65.71) --
	(201.83, 65.71) --
	(201.83, 65.71) --
	(201.83, 65.71) --
	(201.76, 65.71) --
	(201.76, 65.71) --
	(201.76, 65.71) --
	(201.69, 65.71) --
	(201.69, 65.71) --
	(201.69, 65.71) --
	(201.62, 65.71) --
	(201.62, 65.71) --
	(201.62, 65.71) --
	(201.54, 65.71) --
	(201.54, 65.71) --
	(201.54, 65.71) --
	(201.47, 65.71) --
	(201.47, 65.71) --
	(201.47, 65.71) --
	(201.40, 65.71) --
	(201.40, 65.71) --
	(201.40, 65.71) --
	(201.33, 65.71) --
	(201.33, 65.71) --
	(201.33, 65.71) --
	(201.26, 65.71) --
	(201.26, 65.71) --
	(201.26, 65.71) --
	(201.18, 65.71) --
	(201.18, 65.71) --
	(201.18, 65.71) --
	(201.11, 65.71) --
	(201.11, 65.71) --
	(201.11, 65.71) --
	(201.04, 65.71) --
	(201.04, 65.71) --
	(201.04, 65.71) --
	(200.96, 65.71) --
	(200.96, 65.71) --
	(200.96, 65.71) --
	(200.89, 65.71) --
	(200.89, 65.71) --
	(200.89, 65.71) --
	(200.82, 65.71) --
	(200.82, 65.71) --
	(200.82, 65.71) --
	(200.79, 65.71) --
	(200.79, 65.71) --
	(200.79, 65.71) --
	(200.75, 65.71) --
	(200.75, 65.71) --
	(200.75, 65.71) --
	(200.67, 65.71) --
	(200.67, 65.71) --
	(200.67, 65.71) --
	(200.60, 65.71) --
	(200.60, 65.71) --
	(200.60, 65.71) --
	(200.53, 65.71) --
	(200.53, 65.71) --
	(200.53, 65.71) --
	(200.46, 65.71) --
	(200.46, 65.71) --
	(200.46, 65.71) --
	(200.38, 65.71) --
	(200.38, 65.71) --
	(200.38, 65.71) --
	(200.31, 65.71) --
	(200.31, 65.71) --
	(200.31, 65.71) --
	(200.24, 65.71) --
	(200.24, 65.71) --
	(200.24, 65.71) --
	(200.17, 65.71) --
	(200.17, 65.71) --
	(200.17, 65.71) --
	(200.09, 65.71) --
	(200.09, 65.71) --
	(200.09, 65.71) --
	(200.03, 65.71) --
	(200.03, 65.71) --
	(200.03, 65.71) --
	(200.02, 65.71) --
	(200.02, 65.71) --
	(200.02, 65.71) --
	(199.95, 65.71) --
	(199.95, 65.71) --
	(199.95, 65.71) --
	(199.88, 65.71) --
	(199.88, 65.71) --
	(199.88, 65.71) --
	(199.80, 65.71) --
	(199.80, 65.71) --
	(199.80, 65.71) --
	(199.73, 65.71) --
	(199.73, 65.71) --
	(199.73, 65.71) --
	(199.66, 65.71) --
	(199.66, 65.71) --
	(199.66, 65.71) --
	(199.59, 65.71) --
	(199.59, 65.71) --
	(199.59, 65.71) --
	(199.51, 65.71) --
	(199.51, 65.71) --
	(199.51, 65.71) --
	(199.44, 65.71) --
	(199.44, 65.71) --
	(199.44, 65.71) --
	(199.37, 65.71) --
	(199.37, 65.71) --
	(199.37, 65.71) --
	(199.29, 65.71) --
	(199.29, 65.71) --
	(199.29, 65.71) --
	(199.26, 65.71) --
	(199.26, 65.71) --
	(199.26, 65.71) --
	(199.22, 65.71) --
	(199.22, 65.71) --
	(199.22, 65.71) --
	(199.15, 65.71) --
	(199.15, 65.71) --
	(199.15, 65.71) --
	(199.08, 65.71) --
	(199.08, 65.71) --
	(199.08, 65.71) --
	(199.00, 65.71) --
	(199.00, 65.71) --
	(199.00, 65.71) --
	(198.93, 65.71) --
	(198.93, 65.71) --
	(198.93, 65.71) --
	(198.86, 65.71) --
	(198.86, 65.71) --
	(198.86, 65.71) --
	(198.79, 65.71) --
	(198.79, 65.71) --
	(198.79, 65.71) --
	(198.71, 65.71) --
	(198.71, 65.71) --
	(198.71, 65.71) --
	(198.64, 65.71) --
	(198.64, 65.71) --
	(198.64, 65.71) --
	(198.57, 65.71) --
	(198.57, 65.71) --
	(198.57, 65.71) --
	(198.50, 65.71) --
	(198.50, 65.71) --
	(198.50, 65.71) --
	(198.42, 65.71) --
	(198.42, 65.71) --
	(198.42, 65.71) --
	(198.35, 65.71) --
	(198.35, 65.71) --
	(198.35, 65.71) --
	(198.28, 65.71) --
	(198.28, 65.71) --
	(198.28, 65.71) --
	(198.20, 65.71) --
	(198.20, 65.71) --
	(198.20, 65.71) --
	(198.13, 65.71) --
	(198.13, 65.71) --
	(198.13, 65.71) --
	(198.11, 65.71) --
	(198.11, 65.71) --
	(198.11, 65.71) --
	(198.06, 65.71) --
	(198.06, 65.71) --
	(198.06, 65.71) --
	(197.99, 65.71) --
	(197.99, 65.71) --
	(197.99, 65.71) --
	(197.91, 65.71) --
	(197.91, 65.71) --
	(197.91, 65.71) --
	(197.84, 65.71) --
	(197.84, 65.71) --
	(197.84, 65.71) --
	(197.77, 65.71) --
	(197.77, 65.71) --
	(197.77, 65.71) --
	(197.69, 65.71) --
	(197.69, 65.71) --
	(197.69, 65.71) --
	(197.62, 65.71) --
	(197.62, 65.71) --
	(197.62, 65.71) --
	(197.55, 65.71) --
	(197.55, 65.71) --
	(197.55, 65.71) --
	(197.48, 65.71) --
	(197.48, 65.71) --
	(197.48, 65.71) --
	(197.40, 65.71) --
	(197.40, 65.71) --
	(197.40, 65.71) --
	(197.33, 65.71) --
	(197.33, 65.71) --
	(197.33, 65.71) --
	(197.26, 65.71) --
	(197.26, 65.71) --
	(197.26, 65.71) --
	(197.19, 65.71) --
	(197.19, 65.71) --
	(197.19, 65.71) --
	(197.11, 65.71) --
	(197.11, 65.71) --
	(197.11, 65.71) --
	(197.06, 65.71) --
	(197.06, 65.71) --
	(197.06, 65.71) --
	(197.04, 65.71) --
	(197.04, 65.71) --
	(197.04, 65.71) --
	(196.97, 65.71) --
	(196.97, 65.71) --
	(196.97, 65.71) --
	(196.89, 65.71) --
	(196.89, 65.71) --
	(196.89, 65.71) --
	(196.82, 65.71) --
	(196.82, 65.71) --
	(196.82, 65.71) --
	(196.75, 65.71) --
	(196.75, 65.71) --
	(196.75, 65.71) --
	(196.68, 65.71) --
	(196.68, 65.71) --
	(196.68, 65.71) --
	(196.60, 65.71) --
	(196.60, 65.71) --
	(196.60, 65.71) --
	(196.53, 65.71) --
	(196.53, 65.71) --
	(196.53, 65.71) --
	(196.46, 65.71) --
	(196.46, 65.71) --
	(196.46, 65.71) --
	(196.38, 65.71) --
	(196.38, 65.71) --
	(196.38, 65.71) --
	(196.31, 65.71) --
	(196.31, 65.71) --
	(196.31, 65.71) --
	(196.24, 65.71) --
	(196.24, 65.71) --
	(196.24, 65.71) --
	(196.17, 65.71) --
	(196.17, 65.71) --
	(196.17, 65.71) --
	(196.09, 65.71) --
	(196.09, 65.71) --
	(196.09, 65.71) --
	(196.02, 65.71) --
	(196.02, 65.71) --
	(196.02, 65.71) --
	(195.95, 65.71) --
	(195.95, 65.71) --
	(195.95, 65.71) --
	(195.87, 65.71) --
	(195.87, 65.71) --
	(195.87, 65.71) --
	(195.80, 65.71) --
	(195.80, 65.71) --
	(195.80, 65.71) --
	(195.73, 65.71) --
	(195.73, 65.71) --
	(195.73, 65.71) --
	(195.66, 65.71) --
	(195.66, 65.71) --
	(195.66, 65.71) --
	(195.58, 65.71) --
	(195.58, 65.71) --
	(195.58, 65.71) --
	(195.51, 65.71) --
	(195.51, 65.71) --
	(195.51, 65.71) --
	(195.44, 65.71) --
	(195.44, 65.71) --
	(195.44, 65.71) --
	(195.36, 65.71) --
	(195.36, 65.71) --
	(195.36, 65.71) --
	(195.29, 65.71) --
	(195.29, 65.71) --
	(195.29, 65.71) --
	(195.23, 65.71) --
	(195.23, 65.71) --
	(195.23, 65.71) --
	(195.22, 65.71) --
	(195.22, 65.71) --
	(195.22, 65.71) --
	(195.15, 65.71) --
	(195.15, 65.71) --
	(195.15, 65.71) --
	(195.07, 65.71) --
	(195.07, 65.71) --
	(195.07, 65.71) --
	(195.00, 65.71) --
	(195.00, 65.71) --
	(195.00, 65.71) --
	(194.93, 65.71) --
	(194.93, 65.71) --
	(194.93, 65.71) --
	(194.85, 65.71) --
	(194.85, 65.71) --
	(194.85, 65.71) --
	(194.78, 65.71) --
	(194.78, 65.71) --
	(194.78, 65.71) --
	(194.71, 65.71) --
	(194.71, 65.71) --
	(194.71, 65.71) --
	(194.63, 65.71) --
	(194.63, 65.71) --
	(194.63, 65.71) --
	(194.56, 65.71) --
	(194.56, 65.71) --
	(194.56, 65.71) --
	(194.49, 65.71) --
	(194.49, 65.71) --
	(194.49, 65.71) --
	(194.42, 65.71) --
	(194.42, 65.71) --
	(194.42, 65.71) --
	(194.34, 65.71) --
	(194.34, 65.71) --
	(194.34, 65.71) --
	(194.27, 65.71) --
	(194.27, 65.71) --
	(194.27, 65.71) --
	(194.20, 65.71) --
	(194.20, 65.71) --
	(194.20, 65.71) --
	(194.12, 65.71) --
	(194.12, 65.71) --
	(194.12, 65.71) --
	(194.05, 65.71) --
	(194.05, 65.71) --
	(194.05, 65.71) --
	(193.98, 65.71) --
	(193.98, 65.71) --
	(193.98, 65.71) --
	(193.91, 65.71) --
	(193.91, 65.71) --
	(193.91, 65.71) --
	(193.83, 65.71) --
	(193.83, 65.71) --
	(193.83, 65.71) --
	(193.76, 65.71) --
	(193.76, 65.71) --
	(193.76, 65.71) --
	(193.69, 65.71) --
	(193.69, 65.71) --
	(193.69, 65.71) --
	(193.61, 65.71) --
	(193.61, 65.71) --
	(193.61, 65.71) --
	(193.54, 65.71) --
	(193.54, 65.71) --
	(193.54, 65.71) --
	(193.47, 65.71) --
	(193.47, 65.71) --
	(193.47, 65.71) --
	(193.41, 65.71) --
	(193.41, 65.71) --
	(193.41, 65.71) --
	(193.39, 65.71) --
	(193.39, 65.71) --
	(193.39, 65.71) --
	(193.32, 65.71) --
	(193.32, 65.71) --
	(193.32, 65.71) --
	(193.25, 65.71) --
	(193.25, 65.71) --
	(193.25, 65.71) --
	(193.17, 65.71) --
	(193.17, 65.71) --
	(193.17, 65.71) --
	(193.10, 65.71) --
	(193.10, 65.71) --
	(193.10, 65.71) --
	(193.03, 65.71) --
	(193.03, 65.71) --
	(193.03, 65.71) --
	(192.96, 65.71) --
	(192.96, 65.71) --
	(192.96, 65.71) --
	(192.88, 65.71) --
	(192.88, 65.71) --
	(192.88, 65.71) --
	(192.81, 65.71) --
	(192.81, 65.71) --
	(192.81, 65.71) --
	(192.74, 65.71) --
	(192.74, 65.71) --
	(192.74, 65.71) --
	(192.66, 65.71) --
	(192.66, 65.71) --
	(192.66, 65.71) --
	(192.59, 65.71) --
	(192.59, 65.71) --
	(192.59, 65.71) --
	(192.52, 65.71) --
	(192.52, 65.71) --
	(192.52, 65.71) --
	(192.44, 65.71) --
	(192.44, 65.71) --
	(192.44, 65.71) --
	(192.37, 65.71) --
	(192.37, 65.71) --
	(192.37, 65.71) --
	(192.30, 65.71) --
	(192.30, 65.71) --
	(192.30, 65.71) --
	(192.23, 65.71) --
	(192.23, 65.71) --
	(192.23, 65.71) --
	(192.15, 65.71) --
	(192.15, 65.71) --
	(192.15, 65.71) --
	(192.08, 65.71) --
	(192.08, 65.71) --
	(192.08, 65.71) --
	(192.07, 65.71) --
	(192.07, 65.71) --
	(192.07, 65.71) --
	(192.01, 65.71) --
	(192.01, 65.71) --
	(192.01, 65.71) --
	(191.93, 65.71) --
	(191.93, 65.71) --
	(191.93, 65.71) --
	(191.86, 65.71) --
	(191.86, 65.71) --
	(191.86, 65.71) --
	(191.79, 65.71) --
	(191.79, 65.71) --
	(191.79, 65.71) --
	(191.71, 65.71) --
	(191.71, 65.71) --
	(191.71, 65.71) --
	(191.64, 65.71) --
	(191.64, 65.71) --
	(191.64, 65.71) --
	(191.57, 65.71) --
	(191.57, 65.71) --
	(191.57, 65.71) --
	(191.49, 65.71) --
	(191.49, 65.71) --
	(191.49, 65.71) --
	(191.42, 65.71) --
	(191.42, 65.71) --
	(191.42, 65.71) --
	(191.35, 65.71) --
	(191.35, 65.71) --
	(191.35, 65.71) --
	(191.27, 65.71) --
	(191.27, 65.71) --
	(191.27, 65.71) --
	(191.20, 65.71) --
	(191.20, 65.71) --
	(191.20, 65.71) --
	(191.13, 65.71) --
	(191.13, 65.71) --
	(191.13, 65.71) --
	(191.06, 65.71) --
	(191.06, 65.71) --
	(191.06, 65.71) --
	(190.98, 65.71) --
	(190.98, 65.71) --
	(190.98, 65.71) --
	(190.91, 65.71) --
	(190.91, 65.71) --
	(190.91, 65.71) --
	(190.84, 65.71) --
	(190.84, 65.71) --
	(190.84, 65.71) --
	(190.76, 65.71) --
	(190.76, 65.71) --
	(190.76, 65.71) --
	(190.69, 65.71) --
	(190.69, 65.71) --
	(190.69, 65.71) --
	(190.62, 65.71) --
	(190.62, 65.71) --
	(190.62, 65.71) --
	(190.54, 65.71) --
	(190.54, 65.71) --
	(190.54, 65.71) --
	(190.47, 65.71) --
	(190.47, 65.71) --
	(190.47, 65.71) --
	(190.40, 65.71) --
	(190.40, 65.71) --
	(190.40, 65.71) --
	(190.32, 65.71) --
	(190.32, 65.71) --
	(190.32, 65.71) --
	(190.25, 65.71) --
	(190.25, 65.71) --
	(190.25, 65.71) --
	(190.25, 65.71) --
	(190.25, 65.71) --
	(190.25, 65.71) --
	(190.18, 65.71) --
	(190.18, 65.71) --
	(190.18, 65.71) --
	(190.10, 65.71) --
	(190.10, 65.71) --
	(190.10, 65.71) --
	(190.03, 65.71) --
	(190.03, 65.71) --
	(190.03, 65.71) --
	(189.96, 65.71) --
	(189.96, 65.71) --
	(189.96, 65.71) --
	(189.88, 65.71) --
	(189.88, 65.71) --
	(189.88, 65.71) --
	(189.81, 65.71) --
	(189.81, 65.71) --
	(189.81, 65.71) --
	(189.74, 65.71) --
	(189.74, 65.71) --
	(189.74, 65.71) --
	(189.66, 65.71) --
	(189.66, 65.71) --
	(189.66, 65.71) --
	(189.59, 65.71) --
	(189.59, 65.71) --
	(189.59, 65.71) --
	(189.52, 65.71) --
	(189.52, 65.71) --
	(189.52, 65.71) --
	(189.44, 65.71) --
	(189.44, 65.71) --
	(189.44, 65.71) --
	(189.37, 65.71) --
	(189.37, 65.71) --
	(189.37, 65.71) --
	(189.30, 65.71) --
	(189.30, 65.71) --
	(189.30, 65.71) --
	(189.29, 65.71) --
	(189.29, 65.71) --
	(189.29, 65.71) --
	(189.22, 65.71) --
	(189.22, 65.71) --
	(189.22, 65.71) --
	(189.15, 65.71) --
	(189.15, 65.71) --
	(189.15, 65.71) --
	(189.08, 65.71) --
	(189.08, 65.71) --
	(189.08, 65.71) --
	(189.01, 65.71) --
	(189.01, 65.71) --
	(189.01, 65.71) --
	(188.93, 65.71) --
	(188.93, 65.71) --
	(188.93, 65.71) --
	(188.86, 65.71) --
	(188.86, 65.71) --
	(188.86, 65.71) --
	(188.79, 65.71) --
	(188.79, 65.71) --
	(188.79, 65.71) --
	(188.71, 65.71) --
	(188.71, 65.71) --
	(188.71, 65.71) --
	(188.64, 65.71) --
	(188.64, 65.71) --
	(188.64, 65.71) --
	(188.56, 65.71) --
	(188.56, 65.71) --
	(188.56, 65.71) --
	(188.49, 65.71) --
	(188.49, 65.71) --
	(188.49, 65.71) --
	(188.42, 65.71) --
	(188.42, 65.71) --
	(188.42, 65.71) --
	(188.34, 65.71) --
	(188.34, 65.71) --
	(188.34, 65.71) --
	(188.33, 65.71) --
	(188.33, 65.71) --
	(188.33, 65.71) --
	(188.27, 65.71) --
	(188.27, 65.71) --
	(188.27, 65.71) --
	(188.20, 65.71) --
	(188.20, 65.71) --
	(188.20, 65.71) --
	(188.13, 65.71) --
	(188.13, 65.71) --
	(188.13, 65.71) --
	(188.05, 65.71) --
	(188.05, 65.71) --
	(188.05, 65.71) --
	(187.98, 65.71) --
	(187.98, 65.71) --
	(187.98, 65.71) --
	(187.91, 65.71) --
	(187.91, 65.71) --
	(187.91, 65.71) --
	(187.83, 65.71) --
	(187.83, 65.71) --
	(187.83, 65.71) --
	(187.76, 65.71) --
	(187.76, 65.71) --
	(187.76, 65.71) --
	(187.76, 65.71) --
	(187.76, 65.71) --
	(187.76, 65.71) --
	(187.68, 65.71) --
	(187.68, 65.71) --
	(187.68, 65.71) --
	(187.61, 65.71) --
	(187.61, 65.71) --
	(187.61, 65.71) --
	(187.54, 65.71) --
	(187.54, 65.71) --
	(187.54, 65.71) --
	(187.46, 65.71) --
	(187.46, 65.71) --
	(187.46, 65.71) --
	(187.39, 65.71) --
	(187.39, 65.71) --
	(187.39, 65.71) --
	(187.32, 65.71) --
	(187.32, 65.71) --
	(187.32, 65.71) --
	(187.25, 65.71) --
	(187.25, 65.71) --
	(187.25, 65.71) --
	(187.18, 65.71) --
	(187.18, 65.71) --
	(187.18, 65.71) --
	(187.17, 65.71) --
	(187.17, 65.71) --
	(187.17, 65.71) --
	(187.10, 65.71) --
	(187.10, 65.71) --
	(187.10, 65.71) --
	(187.02, 65.71) --
	(187.02, 65.71) --
	(187.02, 65.71) --
	(186.95, 65.71) --
	(186.95, 65.71) --
	(186.95, 65.71) --
	(186.88, 65.71) --
	(186.88, 65.71) --
	(186.88, 65.71) --
	(186.80, 65.71) --
	(186.80, 65.71) --
	(186.80, 65.71) --
	(186.73, 65.71) --
	(186.73, 65.71) --
	(186.73, 65.71) --
	(186.71, 65.71) --
	(186.71, 65.71) --
	(186.71, 65.71) --
	(186.66, 65.71) --
	(186.66, 65.71) --
	(186.66, 65.71) --
	(186.58, 65.71) --
	(186.58, 65.71) --
	(186.58, 65.71) --
	(186.51, 65.71) --
	(186.51, 65.71) --
	(186.51, 65.71) --
	(186.44, 65.71) --
	(186.44, 65.71) --
	(186.44, 65.71) --
	(186.36, 65.71) --
	(186.36, 65.71) --
	(186.36, 65.71) --
	(186.29, 65.71) --
	(186.29, 65.71) --
	(186.29, 65.71) --
	(186.22, 65.71) --
	(186.22, 65.71) --
	(186.22, 65.71) --
	(186.14, 65.71) --
	(186.14, 65.71) --
	(186.14, 65.71) --
	(186.07, 65.71) --
	(186.07, 65.71) --
	(186.07, 65.71) --
	(186.03, 65.71) --
	(186.03, 65.71) --
	(186.03, 65.71) --
	(186.00, 65.71) --
	(186.00, 65.71) --
	(186.00, 65.71) --
	(185.92, 65.71) --
	(185.92, 65.71) --
	(185.92, 65.71) --
	(185.85, 65.71) --
	(185.85, 65.71) --
	(185.85, 65.71) --
	(185.78, 65.71) --
	(185.78, 65.71) --
	(185.78, 65.71) --
	(185.75, 65.71) --
	(185.75, 65.71) --
	(185.75, 65.71) --
	(185.70, 65.71) --
	(185.70, 65.71) --
	(185.70, 65.71) --
	(185.63, 65.71) --
	(185.63, 65.71) --
	(185.63, 65.71) --
	(185.55, 65.71) --
	(185.55, 65.71) --
	(185.55, 65.71) --
	(185.48, 65.71) --
	(185.48, 65.71) --
	(185.48, 65.71) --
	(185.41, 65.71) --
	(185.41, 65.71) --
	(185.41, 65.71) --
	(185.33, 65.71) --
	(185.33, 65.71) --
	(185.33, 65.71) --
	(185.27, 65.71) --
	(185.27, 65.71) --
	(185.27, 65.71) --
	(185.26, 65.71) --
	(185.26, 65.71) --
	(185.26, 65.71) --
	(185.19, 65.71) --
	(185.19, 65.71) --
	(185.19, 65.71) --
	(185.11, 65.71) --
	(185.11, 65.71) --
	(185.11, 65.71) --
	(185.04, 65.71) --
	(185.04, 65.71) --
	(185.04, 65.71) --
	(184.98, 65.71) --
	(184.98, 65.71) --
	(184.98, 65.71) --
	(184.97, 65.71) --
	(184.97, 65.71) --
	(184.97, 65.71) --
	(184.89, 65.71) --
	(184.89, 65.71) --
	(184.89, 65.71) --
	(184.82, 65.71) --
	(184.82, 65.71) --
	(184.82, 65.71) --
	(184.75, 65.71) --
	(184.75, 65.71) --
	(184.75, 65.71) --
	(184.67, 65.71) --
	(184.67, 65.71) --
	(184.67, 65.71) --
	(184.60, 65.71) --
	(184.60, 65.71) --
	(184.60, 65.71) --
	(184.53, 65.71) --
	(184.53, 65.71) --
	(184.53, 65.71) --
	(184.45, 65.71) --
	(184.45, 65.71) --
	(184.45, 65.71) --
	(184.41, 65.71) --
	(184.41, 65.71) --
	(184.41, 65.71) --
	(184.38, 65.71) --
	(184.38, 65.71) --
	(184.38, 65.71) --
	(184.30, 65.71) --
	(184.30, 65.71) --
	(184.30, 65.71) --
	(184.23, 65.71) --
	(184.23, 65.71) --
	(184.23, 65.71) --
	(184.16, 65.71) --
	(184.16, 65.71) --
	(184.16, 65.71) --
	(184.12, 65.71) --
	(184.12, 65.71) --
	(184.12, 65.71) --
	(184.09, 65.71) --
	(184.09, 65.71) --
	(184.09, 65.71) --
	(184.01, 65.71) --
	(184.01, 65.71) --
	(184.01, 65.71) --
	(183.94, 65.71) --
	(183.94, 65.71) --
	(183.94, 65.71) --
	(183.86, 65.71) --
	(183.86, 65.71) --
	(183.86, 65.71) --
	(183.79, 65.71) --
	(183.79, 65.71) --
	(183.79, 65.71) --
	(183.72, 65.71) --
	(183.72, 65.71) --
	(183.72, 65.71) --
	(183.64, 65.71) --
	(183.64, 65.71) --
	(183.64, 65.71) --
	(183.57, 65.71) --
	(183.57, 65.71) --
	(183.57, 65.71) --
	(183.50, 65.71) --
	(183.50, 65.71) --
	(183.50, 65.71) --
	(183.42, 65.71) --
	(183.42, 65.71) --
	(183.42, 65.71) --
	(183.35, 65.71) --
	(183.35, 65.71) --
	(183.35, 65.71) --
	(183.35, 65.71) --
	(183.35, 65.71) --
	(183.35, 65.71) --
	(183.27, 65.71) --
	(183.27, 65.71) --
	(183.27, 65.71) --
	(183.20, 65.71) --
	(183.20, 65.71) --
	(183.20, 65.71) --
	(183.13, 65.71) --
	(183.13, 65.71) --
	(183.13, 65.71) --
	(183.05, 65.71) --
	(183.05, 65.71) --
	(183.05, 65.71) --
	(182.98, 65.71) --
	(182.98, 65.71) --
	(182.98, 65.71) --
	(182.91, 65.71) --
	(182.91, 65.71) --
	(182.91, 65.71) --
	(182.83, 65.71) --
	(182.83, 65.71) --
	(182.83, 65.71) --
	(182.76, 65.71) --
	(182.76, 65.71) --
	(182.76, 65.71) --
	(182.69, 65.71) --
	(182.69, 65.71) --
	(182.68, 65.71) --
	(182.61, 65.71) --
	(182.61, 65.71) --
	(182.61, 65.71) --
	(182.58, 65.71) --
	(182.58, 65.71) --
	(182.58, 65.71) --
	(182.54, 65.71) --
	(182.54, 65.71) --
	(182.54, 65.71) --
	(182.46, 65.71) --
	(182.46, 65.71) --
	(182.46, 65.71) --
	(182.39, 65.71) --
	(182.39, 65.71) --
	(182.39, 65.71) --
	(182.32, 65.71) --
	(182.32, 65.71) --
	(182.32, 65.71) --
	(182.24, 65.71) --
	(182.24, 65.71) --
	(182.24, 65.71) --
	(182.17, 65.71) --
	(182.17, 65.71) --
	(182.17, 65.71) --
	(182.10, 65.71) --
	(182.10, 65.71) --
	(182.10, 65.71) --
	(182.02, 65.71) --
	(182.02, 65.71) --
	(182.02, 65.71) --
	(182.01, 65.71) --
	(182.01, 65.71) --
	(182.01, 65.71) --
	(181.95, 65.71) --
	(181.95, 65.71) --
	(181.95, 65.71) --
	(181.87, 65.71) --
	(181.87, 65.71) --
	(181.87, 65.71) --
	(181.80, 65.71) --
	(181.80, 65.71) --
	(181.80, 65.71) --
	(181.73, 65.71) --
	(181.73, 65.71) --
	(181.73, 65.71) --
	(181.65, 65.71) --
	(181.65, 65.71) --
	(181.65, 65.71) --
	(181.58, 65.71) --
	(181.58, 65.71) --
	(181.58, 65.71) --
	(181.53, 65.71) --
	(181.53, 65.71) --
	(181.53, 65.71) --
	(181.51, 65.71) --
	(181.51, 65.71) --
	(181.51, 65.71) --
	(181.43, 65.71) --
	(181.43, 65.71) --
	(181.43, 65.71) --
	(181.36, 65.71) --
	(181.36, 65.71) --
	(181.36, 65.71) --
	(181.28, 65.71) --
	(181.28, 65.71) --
	(181.28, 65.71) --
	(181.21, 65.71) --
	(181.21, 65.71) --
	(181.21, 65.71) --
	(181.14, 65.71) --
	(181.14, 65.71) --
	(181.14, 65.71) --
	(181.06, 65.71) --
	(181.06, 65.71) --
	(181.06, 65.71) --
	(181.05, 65.71) --
	(181.05, 65.71) --
	(181.05, 65.71) --
	(180.99, 65.71) --
	(180.99, 65.71) --
	(180.99, 65.71) --
	(180.92, 65.71) --
	(180.92, 65.71) --
	(180.92, 65.71) --
	(180.84, 65.71) --
	(180.84, 65.71) --
	(180.84, 65.71) --
	(180.77, 65.71) --
	(180.77, 65.71) --
	(180.77, 65.71) --
	(180.70, 65.71) --
	(180.70, 65.71) --
	(180.70, 65.71) --
	(180.62, 65.71) --
	(180.62, 65.71) --
	(180.62, 65.71) --
	(180.57, 65.71) --
	(180.57, 65.71) --
	(180.57, 65.71) --
	(180.55, 65.71) --
	(180.55, 65.71) --
	(180.55, 65.71) --
	(180.47, 65.71) --
	(180.47, 65.71) --
	(180.47, 65.71) --
	(180.40, 65.71) --
	(180.40, 65.71) --
	(180.40, 65.71) --
	(180.33, 65.71) --
	(180.33, 65.71) --
	(180.33, 65.71) --
	(180.25, 65.71) --
	(180.25, 65.71) --
	(180.25, 65.71) --
	(180.19, 65.71) --
	(180.19, 65.71) --
	(180.19, 65.71) --
	(180.18, 65.71) --
	(180.18, 65.71) --
	(180.18, 65.71) --
	(180.10, 65.71) --
	(180.10, 65.71) --
	(180.10, 65.71) --
	(180.03, 65.71) --
	(180.03, 65.71) --
	(180.03, 65.71) --
	(179.96, 65.71) --
	(179.96, 65.71) --
	(179.96, 65.71) --
	(179.88, 65.71) --
	(179.88, 65.71) --
	(179.88, 65.71) --
	(179.81, 65.71) --
	(179.81, 65.71) --
	(179.81, 65.71) --
	(179.81, 65.71) --
	(179.81, 65.71) --
	(179.81, 65.71) --
	(179.73, 65.71) --
	(179.73, 65.71) --
	(179.73, 65.71) --
	(179.71, 65.71) --
	(179.71, 65.71) --
	(179.71, 65.71) --
	(179.66, 65.71) --
	(179.66, 65.71) --
	(179.66, 65.71) --
	(179.59, 65.71) --
	(179.59, 65.71) --
	(179.59, 65.71) --
	(179.51, 65.71) --
	(179.51, 65.71) --
	(179.51, 65.71) --
	(179.44, 65.71) --
	(179.44, 65.71) --
	(179.44, 65.71) --
	(179.37, 65.71) --
	(179.37, 65.71) --
	(179.37, 65.71) --
	(179.33, 65.71) --
	(179.33, 65.71) --
	(179.33, 65.71) --
	(179.29, 65.71) --
	(179.29, 65.71) --
	(179.29, 65.71) --
	(179.22, 65.71) --
	(179.22, 65.71) --
	(179.22, 65.71) --
	(179.14, 65.71) --
	(179.14, 65.71) --
	(179.14, 65.71) --
	(179.07, 65.71) --
	(179.07, 65.71) --
	(179.07, 65.71) --
	(179.00, 65.71) --
	(179.00, 65.71) --
	(179.00, 65.71) --
	(178.94, 65.71) --
	(178.94, 65.71) --
	(178.94, 65.71) --
	(178.92, 65.71) --
	(178.92, 65.71) --
	(178.92, 65.71) --
	(178.85, 65.71) --
	(178.85, 65.71) --
	(178.85, 65.71) --
	(178.78, 65.71) --
	(178.78, 65.71) --
	(178.78, 65.71) --
	(178.75, 65.71) --
	(178.75, 65.71) --
	(178.75, 65.71) --
	(178.70, 65.71) --
	(178.70, 65.71) --
	(178.70, 65.71) --
	(178.63, 65.71) --
	(178.63, 65.71) --
	(178.63, 65.71) --
	(178.56, 65.71) --
	(178.56, 65.71) --
	(178.56, 65.71) --
	(178.55, 65.71) --
	(178.55, 65.71) --
	(178.55, 65.71) --
	(178.48, 65.71) --
	(178.48, 65.71) --
	(178.48, 65.71) --
	(178.41, 65.71) --
	(178.41, 65.71) --
	(178.41, 65.71) --
	(178.37, 65.71) --
	(178.37, 65.71) --
	(178.37, 65.71) --
	(178.33, 65.71) --
	(178.33, 65.71) --
	(178.33, 65.71) --
	(178.26, 65.71) --
	(178.26, 65.71) --
	(178.26, 65.71) --
	(178.18, 65.71) --
	(178.18, 65.71) --
	(178.18, 65.71) --
	(178.18, 65.71) --
	(178.18, 65.71) --
	(178.18, 65.71) --
	(178.11, 65.71) --
	(178.11, 65.71) --
	(178.11, 65.71) --
	(178.04, 65.71) --
	(178.04, 65.71) --
	(178.04, 65.71) --
	(177.96, 65.71) --
	(177.96, 65.71) --
	(177.96, 65.71) --
	(177.89, 65.71) --
	(177.89, 65.71) --
	(177.89, 65.71) --
	(177.89, 65.71) --
	(177.89, 65.71) --
	(177.89, 65.71) --
	(177.81, 65.71) --
	(177.81, 65.71) --
	(177.81, 65.71) --
	(177.79, 65.71) --
	(177.79, 65.71) --
	(177.79, 65.71) --
	(177.74, 65.71) --
	(177.74, 65.71) --
	(177.74, 65.71) --
	(177.67, 65.71) --
	(177.67, 65.71) --
	(177.67, 65.71) --
	(177.60, 65.71) --
	(177.60, 65.71) --
	(177.60, 65.71) --
	(177.59, 65.71) --
	(177.59, 65.71) --
	(177.59, 65.71) --
	(177.52, 65.71) --
	(177.52, 65.71) --
	(177.52, 65.71) --
	(177.51, 65.71) --
	(177.51, 65.71) --
	(177.51, 65.71) --
	(177.44, 65.71) --
	(177.44, 65.71) --
	(177.44, 65.71) --
	(177.37, 65.71) --
	(177.37, 65.71) --
	(177.37, 65.71) --
	(177.31, 65.71) --
	(177.31, 65.71) --
	(177.31, 65.71) --
	(177.30, 65.71) --
	(177.30, 65.71) --
	(177.30, 65.71) --
	(177.22, 65.71) --
	(177.22, 65.71) --
	(177.22, 65.71) --
	(177.15, 65.71) --
	(177.15, 65.71) --
	(177.15, 65.71) --
	(177.12, 65.71) --
	(177.12, 65.71) --
	(177.12, 65.71) --
	(177.07, 65.71) --
	(177.07, 65.71) --
	(177.07, 65.71) --
	(177.03, 65.71) --
	(177.03, 65.71) --
	(177.03, 65.71) --
	(177.00, 65.71) --
	(177.00, 65.71) --
	(177.00, 65.71) --
	(176.93, 65.71) --
	(176.93, 65.71) --
	(176.93, 65.71) --
	(176.85, 65.71) --
	(176.85, 65.71) --
	(176.85, 65.71) --
	(176.83, 65.71) --
	(176.83, 65.71) --
	(176.83, 65.71) --
	(176.78, 65.71) --
	(176.78, 65.71) --
	(176.78, 65.71) --
	(176.74, 65.71) --
	(176.74, 65.71) --
	(176.74, 65.71) --
	(176.70, 65.71) --
	(176.70, 65.71) --
	(176.70, 65.71) --
	(176.64, 65.71) --
	(176.64, 65.71) --
	(176.64, 65.71) --
	(176.63, 65.71) --
	(176.63, 65.71) --
	(176.63, 65.71) --
	(176.56, 65.71) --
	(176.56, 65.71) --
	(176.56, 65.71) --
	(176.48, 65.71) --
	(176.48, 65.71) --
	(176.48, 65.71) --
	(176.45, 65.71) --
	(176.45, 65.71) --
	(176.45, 65.71) --
	(176.41, 65.71) --
	(176.41, 65.71) --
	(176.41, 65.71) --
	(176.36, 65.71) --
	(176.36, 65.71) --
	(176.36, 65.71) --
	(176.33, 65.71) --
	(176.33, 65.71) --
	(176.33, 65.71) --
	(176.26, 65.71) --
	(176.26, 65.71) --
	(176.26, 65.71) --
	(176.19, 65.71) --
	(176.19, 65.71) --
	(176.19, 65.71) --
	(176.16, 65.71) --
	(176.16, 65.71) --
	(176.16, 65.71) --
	(176.11, 65.71) --
	(176.11, 65.71) --
	(176.11, 65.71) --
	(176.07, 65.71) --
	(176.07, 65.71) --
	(176.07, 65.71) --
	(176.04, 65.71) --
	(176.04, 65.71) --
	(176.04, 65.71) --
	(175.97, 65.71) --
	(175.97, 65.71) --
	(175.97, 65.71) --
	(175.96, 65.71) --
	(175.96, 65.71) --
	(175.96, 65.71) --
	(175.89, 65.71) --
	(175.89, 65.71) --
	(175.89, 65.71) --
	(175.82, 65.71) --
	(175.82, 65.71) --
	(175.82, 65.71) --
	(175.78, 65.71) --
	(175.78, 65.71) --
	(175.78, 65.71) --
	(175.74, 65.71) --
	(175.74, 65.71) --
	(175.74, 65.71) --
	(175.68, 65.71) --
	(175.68, 65.71) --
	(175.68, 65.71) --
	(175.67, 65.71) --
	(175.67, 65.71) --
	(175.67, 65.71) --
	(175.59, 65.71) --
	(175.59, 65.71) --
	(175.59, 65.71) --
	(175.59, 65.71) --
	(175.59, 65.71) --
	(175.59, 65.71) --
	(175.52, 65.71) --
	(175.52, 65.71) --
	(175.52, 65.71) --
	(175.45, 65.71) --
	(175.45, 65.71) --
	(175.45, 65.71) --
	(175.37, 65.71) --
	(175.37, 65.71) --
	(175.37, 65.71) --
	(175.30, 65.71) --
	(175.30, 65.71) --
	(175.30, 65.71) --
	(175.30, 65.71) --
	(175.30, 65.71) --
	(175.30, 65.71) --
	(175.22, 65.71) --
	(175.22, 65.71) --
	(175.22, 65.71) --
	(175.15, 65.71) --
	(175.15, 65.71) --
	(175.15, 65.71) --
	(175.07, 65.71) --
	(175.07, 65.71) --
	(175.07, 65.71) --
	(175.01, 65.71) --
	(175.01, 65.71) --
	(175.01, 65.71) --
	(175.00, 65.71) --
	(175.00, 65.71) --
	(175.00, 65.71) --
	(174.93, 65.71) --
	(174.93, 65.71) --
	(174.93, 65.71) --
	(174.92, 65.71) --
	(174.92, 65.71) --
	(174.92, 65.71) --
	(174.85, 65.71) --
	(174.85, 65.71) --
	(174.85, 65.71) --
	(174.82, 65.71) --
	(174.82, 65.71) --
	(174.82, 65.71) --
	(174.78, 65.71) --
	(174.78, 65.71) --
	(174.78, 65.71) --
	(174.70, 65.71) --
	(174.70, 65.71) --
	(174.70, 65.71) --
	(174.63, 65.71) --
	(174.63, 65.71) --
	(174.63, 65.71) --
	(174.63, 65.71) --
	(174.63, 65.71) --
	(174.63, 65.71) --
	(174.56, 65.71) --
	(174.56, 65.71) --
	(174.56, 65.71) --
	(174.53, 65.71) --
	(174.53, 65.71) --
	(174.53, 65.71) --
	(174.48, 65.71) --
	(174.48, 65.71) --
	(174.48, 65.71) --
	(174.44, 65.71) --
	(174.44, 65.71) --
	(174.44, 65.71) --
	(174.41, 65.71) --
	(174.41, 65.71) --
	(174.41, 65.71) --
	(174.34, 65.71) --
	(174.34, 65.71) --
	(174.34, 65.71) --
	(174.33, 65.71) --
	(174.33, 65.71) --
	(174.33, 65.71) --
	(174.26, 65.71) --
	(174.26, 65.71) --
	(174.26, 65.71) --
	(174.25, 65.71) --
	(174.25, 65.71) --
	(174.25, 65.71) --
	(174.18, 65.71) --
	(174.18, 65.71) --
	(174.18, 65.71) --
	(174.15, 65.71) --
	(174.15, 65.71) --
	(174.15, 65.71) --
	(174.11, 65.71) --
	(174.11, 65.71) --
	(174.11, 65.71) --
	(174.04, 65.71) --
	(174.04, 65.71) --
	(174.04, 65.71) --
	(173.96, 65.71) --
	(173.96, 65.71) --
	(173.96, 65.71) --
	(173.89, 65.71) --
	(173.89, 65.71) --
	(173.89, 65.71) --
	(173.86, 65.71) --
	(173.86, 65.71) --
	(173.86, 65.71) --
	(173.81, 65.71) --
	(173.81, 65.71) --
	(173.81, 65.71) --
	(173.77, 65.71) --
	(173.77, 65.71) --
	(173.77, 65.71) --
	(173.74, 65.71) --
	(173.74, 65.71) --
	(173.74, 65.71) --
	(173.67, 65.71) --
	(173.67, 65.71) --
	(173.67, 65.71) --
	(173.59, 65.71) --
	(173.59, 65.71) --
	(173.59, 65.71) --
	(173.58, 65.71) --
	(173.58, 65.71) --
	(173.58, 65.71) --
	(173.52, 65.71) --
	(173.52, 65.71) --
	(173.52, 65.71) --
	(173.48, 65.71) --
	(173.48, 65.71) --
	(173.48, 65.71) --
	(173.44, 65.71) --
	(173.44, 65.71) --
	(173.44, 65.71) --
	(173.37, 65.71) --
	(173.37, 65.71) --
	(173.37, 65.71) --
	(173.29, 65.71) --
	(173.29, 65.71) --
	(173.29, 65.71) --
	(173.29, 65.71) --
	(173.29, 65.71) --
	(173.29, 65.71) --
	(173.22, 65.71) --
	(173.22, 65.71) --
	(173.22, 65.71) --
	(173.15, 65.71) --
	(173.15, 65.71) --
	(173.15, 65.71) --
	(173.07, 65.71) --
	(173.07, 65.71) --
	(173.07, 65.71) --
	(173.00, 65.71) --
	(173.00, 65.71) --
	(173.00, 65.71) --
	(173.00, 65.71) --
	(173.00, 65.71) --
	(173.00, 65.71) --
	(172.92, 65.71) --
	(172.92, 65.71) --
	(172.92, 65.71) --
	(172.85, 65.71) --
	(172.85, 65.71) --
	(172.85, 65.71) --
	(172.81, 65.71) --
	(172.81, 65.71) --
	(172.81, 65.71) --
	(172.78, 65.71) --
	(172.78, 65.71) --
	(172.78, 65.71) --
	(172.70, 65.71) --
	(172.70, 65.71) --
	(172.70, 65.71) --
	(172.63, 65.71) --
	(172.63, 65.71) --
	(172.63, 65.71) --
	(172.55, 65.71) --
	(172.55, 65.71) --
	(172.55, 65.71) --
	(172.48, 65.71) --
	(172.48, 65.71) --
	(172.48, 65.71) --
	(172.43, 65.71) --
	(172.43, 65.71) --
	(172.43, 65.71) --
	(172.40, 65.71) --
	(172.40, 65.71) --
	(172.40, 65.71) --
	(172.33, 65.71) --
	(172.33, 65.71) --
	(172.33, 65.71) --
	(172.33, 65.71) --
	(172.33, 65.71) --
	(172.33, 65.71) --
	(172.26, 65.71) --
	(172.26, 65.71) --
	(172.26, 65.71) --
	(172.18, 65.71) --
	(172.18, 65.71) --
	(172.18, 65.71) --
	(172.14, 65.71) --
	(172.14, 65.71) --
	(172.14, 65.71) --
	(172.11, 65.71) --
	(172.11, 65.71) --
	(172.11, 65.71) --
	(172.03, 65.71) --
	(172.03, 65.71) --
	(172.03, 65.71) --
	(171.96, 65.71) --
	(171.96, 65.71) --
	(171.96, 65.71) --
	(171.88, 65.71) --
	(171.88, 65.71) --
	(171.88, 65.71) --
	(171.85, 65.71) --
	(171.85, 65.71) --
	(171.85, 65.71) --
	(171.81, 65.71) --
	(171.81, 65.71) --
	(171.81, 65.71) --
	(171.74, 65.71) --
	(171.74, 65.71) --
	(171.74, 65.71) --
	(171.66, 65.71) --
	(171.66, 65.71) --
	(171.66, 65.71) --
	(171.66, 65.71) --
	(171.66, 65.71) --
	(171.66, 65.71) --
	(171.59, 65.71) --
	(171.59, 65.71) --
	(171.59, 65.71) --
	(171.51, 65.71) --
	(171.51, 65.71) --
	(171.51, 65.71) --
	(171.44, 65.71) --
	(171.44, 65.71) --
	(171.44, 65.71) --
	(171.37, 65.71) --
	(171.37, 65.71) --
	(171.37, 65.71) --
	(171.36, 65.71) --
	(171.36, 65.71) --
	(171.36, 65.71) --
	(171.29, 65.71) --
	(171.29, 65.71) --
	(171.29, 65.71) --
	(171.22, 65.71) --
	(171.22, 65.71) --
	(171.22, 65.71) --
	(171.14, 65.71) --
	(171.14, 65.71) --
	(171.14, 65.71) --
	(171.07, 65.71) --
	(171.07, 65.71) --
	(171.07, 65.71) --
	(171.05, 65.71) --
	(171.05, 65.71) --
	(171.05, 65.71) --
	(170.99, 65.71) --
	(170.99, 65.71) --
	(170.99, 65.71) --
	(170.92, 65.71) --
	(170.92, 65.71) --
	(170.92, 65.71) --
	(170.89, 65.71) --
	(170.89, 65.71) --
	(170.89, 65.71) --
	(170.84, 65.71) --
	(170.84, 65.71) --
	(170.84, 65.71) --
	(170.77, 65.71) --
	(170.77, 65.71) --
	(170.77, 65.71) --
	(170.70, 65.71) --
	(170.70, 65.71) --
	(170.70, 65.71) --
	(170.62, 65.71) --
	(170.62, 65.71) --
	(170.62, 65.71) --
	(170.55, 65.71) --
	(170.55, 65.71) --
	(170.55, 65.71) --
	(170.51, 65.71) --
	(170.51, 65.71) --
	(170.51, 65.71) --
	(170.47, 65.71) --
	(170.47, 65.71) --
	(170.47, 65.71) --
	(170.40, 65.71) --
	(170.40, 65.71) --
	(170.40, 65.71) --
	(170.32, 65.71) --
	(170.32, 65.71) --
	(170.32, 65.71) --
	(170.25, 65.71) --
	(170.25, 65.71) --
	(170.25, 65.71) --
	(170.17, 65.71) --
	(170.17, 65.71) --
	(170.17, 65.71) --
	(170.13, 65.71) --
	(170.13, 65.71) --
	(170.13, 65.71) --
	(170.10, 65.71) --
	(170.10, 65.71) --
	(170.10, 65.71) --
	(170.03, 65.71) --
	(170.03, 65.71) --
	(170.03, 65.71) --
	(169.95, 65.71) --
	(169.95, 65.71) --
	(169.95, 65.71) --
	(169.88, 65.71) --
	(169.88, 65.71) --
	(169.88, 65.71) --
	(169.80, 65.71) --
	(169.80, 65.71) --
	(169.80, 65.71) --
	(169.76, 65.71) --
	(169.76, 65.71) --
	(169.76, 65.71) --
	(169.73, 65.71) --
	(169.73, 65.71) --
	(169.73, 65.71) --
	(169.65, 65.71) --
	(169.65, 65.71) --
	(169.65, 65.71) --
	(169.58, 65.71) --
	(169.58, 65.71) --
	(169.58, 65.71) --
	(169.51, 65.71) --
	(169.51, 65.71) --
	(169.51, 65.71) --
	(169.43, 65.71) --
	(169.43, 65.71) --
	(169.43, 65.71) --
	(169.36, 65.71) --
	(169.36, 65.71) --
	(169.36, 65.71) --
	(169.36, 65.71) --
	(169.36, 65.71) --
	(169.36, 65.71) --
	(169.28, 65.71) --
	(169.28, 65.71) --
	(169.28, 65.71) --
	(169.21, 65.71) --
	(169.21, 65.71) --
	(169.21, 65.71) --
	(169.19, 65.71) --
	(169.19, 65.71) --
	(169.19, 65.71) --
	(169.13, 65.71) --
	(169.13, 65.71) --
	(169.13, 65.71) --
	(169.06, 65.71) --
	(169.06, 65.71) --
	(169.06, 65.71) --
	(168.98, 65.71) --
	(168.98, 65.71) --
	(168.98, 65.71) --
	(168.91, 65.71) --
	(168.91, 65.71) --
	(168.91, 65.71) --
	(168.87, 65.71) --
	(168.87, 65.71) --
	(168.87, 65.71) --
	(168.83, 65.71) --
	(168.83, 65.71) --
	(168.83, 65.71) --
	(168.76, 65.71) --
	(168.76, 65.71) --
	(168.76, 65.71) --
	(168.75, 65.71) --
	(168.75, 65.71) --
	(168.75, 65.71) --
	(168.71, 65.71) --
	(168.71, 65.71) --
	(168.71, 65.71) --
	(168.69, 65.71) --
	(168.69, 65.71) --
	(168.69, 65.71) --
	(168.67, 65.71) --
	(168.67, 65.71) --
	(168.67, 65.71) --
	(168.63, 65.71) --
	(168.63, 65.71) --
	(168.63, 65.71) --
	(168.61, 65.71) --
	(168.61, 65.71) --
	(168.61, 65.71) --
	(168.55, 65.71) --
	(168.55, 65.71) --
	(168.55, 65.71) --
	(168.54, 65.71) --
	(168.54, 65.71) --
	(168.54, 65.71) --
	(168.51, 65.71) --
	(168.51, 65.71) --
	(168.51, 65.71) --
	(168.46, 65.71) --
	(168.46, 65.71) --
	(168.46, 65.71) --
	(168.39, 65.71) --
	(168.39, 65.71) --
	(168.39, 65.71) --
	(168.34, 65.71) --
	(168.34, 65.71) --
	(168.34, 65.71) --
	(168.31, 65.71) --
	(168.31, 65.71) --
	(168.31, 65.71) --
	(168.30, 65.71) --
	(168.30, 65.71) --
	(168.30, 65.71) --
	(168.24, 65.71) --
	(168.24, 65.71) --
	(168.24, 65.71) --
	(168.16, 65.71) --
	(168.16, 65.71) --
	(168.16, 65.71) --
	(168.09, 65.71) --
	(168.09, 65.71) --
	(168.09, 65.71) --
	(168.01, 65.71) --
	(168.01, 65.71) --
	(168.01, 65.71) --
	(167.98, 65.71) --
	(167.98, 65.71) --
	(167.98, 65.71) --
	(167.94, 65.71) --
	(167.94, 65.71) --
	(167.94, 65.71) --
	(167.87, 65.71) --
	(167.87, 65.71) --
	(167.87, 65.71) --
	(167.83, 65.71) --
	(167.83, 65.71) --
	(167.83, 65.71) --
	(167.79, 65.71) --
	(167.79, 65.71) --
	(167.79, 65.71) --
	(167.72, 65.71) --
	(167.72, 65.71) --
	(167.72, 65.71) --
	(167.66, 65.71) --
	(167.66, 65.71) --
	(167.66, 65.71) --
	(167.64, 65.71) --
	(167.64, 65.71) --
	(167.64, 65.71) --
	(167.57, 65.71) --
	(167.57, 65.71) --
	(167.57, 65.71) --
	(167.49, 65.71) --
	(167.49, 65.71) --
	(167.49, 65.71) --
	(167.42, 65.71) --
	(167.42, 65.71) --
	(167.42, 65.71) --
	(167.34, 65.71) --
	(167.34, 65.71) --
	(167.34, 65.71) --
	(167.27, 65.71) --
	(167.27, 65.71) --
	(167.27, 65.71) --
	(167.20, 65.71) --
	(167.20, 65.71) --
	(167.20, 65.71) --
	(167.17, 65.71) --
	(167.17, 65.71) --
	(167.17, 65.71) --
	(167.12, 65.71) --
	(167.12, 65.71) --
	(167.12, 65.71) --
	(167.05, 65.71) --
	(167.05, 65.71) --
	(167.05, 65.71) --
	(166.97, 65.71) --
	(166.97, 65.71) --
	(166.97, 65.71) --
	(166.90, 65.71) --
	(166.90, 65.71) --
	(166.90, 65.71) --
	(166.82, 65.71) --
	(166.82, 65.71) --
	(166.82, 65.71) --
	(166.81, 65.71) --
	(166.81, 65.71) --
	(166.81, 65.71) --
	(166.75, 65.71) --
	(166.75, 65.71) --
	(166.75, 65.71) --
	(166.67, 65.71) --
	(166.67, 65.71) --
	(166.67, 65.71) --
	(166.60, 65.71) --
	(166.60, 65.71) --
	(166.60, 65.71) --
	(166.52, 65.71) --
	(166.52, 65.71) --
	(166.52, 65.71) --
	(166.48, 65.71) --
	(166.48, 65.71) --
	(166.48, 65.71) --
	(166.45, 65.71) --
	(166.45, 65.71) --
	(166.45, 65.71) --
	(166.44, 65.71) --
	(166.44, 65.71) --
	(166.44, 65.71) --
	(166.38, 65.71) --
	(166.38, 65.71) --
	(166.38, 65.71) --
	(166.30, 65.71) --
	(166.30, 65.71) --
	(166.30, 65.71) --
	(166.23, 65.71) --
	(166.23, 65.71) --
	(166.23, 65.71) --
	(166.16, 65.71) --
	(166.16, 65.71) --
	(166.16, 65.71) --
	(166.15, 65.71) --
	(166.15, 65.71) --
	(166.15, 65.71) --
	(166.08, 65.71) --
	(166.08, 65.71) --
	(166.08, 65.71) --
	(166.00, 65.71) --
	(166.00, 65.71) --
	(166.00, 65.71) --
	(166.00, 65.71) --
	(166.00, 65.71) --
	(166.00, 65.71) --
	(165.93, 65.71) --
	(165.93, 65.71) --
	(165.93, 65.71) --
	(165.85, 65.71) --
	(165.85, 65.71) --
	(165.85, 65.71) --
	(165.78, 65.71) --
	(165.78, 65.71) --
	(165.78, 65.71) --
	(165.76, 65.71) --
	(165.76, 65.71) --
	(165.76, 65.71) --
	(165.70, 65.71) --
	(165.70, 65.71) --
	(165.70, 65.71) --
	(165.63, 65.71) --
	(165.63, 65.71) --
	(165.63, 65.71) --
	(165.60, 65.71) --
	(165.60, 65.71) --
	(165.60, 65.71) --
	(165.55, 65.71) --
	(165.55, 65.71) --
	(165.55, 65.71) --
	(165.48, 65.71) --
	(165.48, 65.71) --
	(165.48, 65.71) --
	(165.41, 65.71) --
	(165.41, 65.71) --
	(165.41, 65.71) --
	(165.33, 65.71) --
	(165.33, 65.71) --
	(165.33, 65.71) --
	(165.27, 65.71) --
	(165.27, 65.71) --
	(165.27, 65.71) --
	(165.25, 65.71) --
	(165.25, 65.71) --
	(165.25, 65.71) --
	(165.18, 65.71) --
	(165.18, 65.71) --
	(165.18, 65.71) --
	(165.11, 65.71) --
	(165.11, 65.71) --
	(165.11, 65.71) --
	(165.07, 65.71) --
	(165.07, 65.71) --
	(165.07, 65.71) --
	(165.03, 65.71) --
	(165.03, 65.71) --
	(165.03, 65.71) --
	(164.99, 65.71) --
	(164.99, 65.71) --
	(164.99, 65.71) --
	(164.96, 65.71) --
	(164.96, 65.71) --
	(164.96, 65.71) --
	(164.95, 65.71) --
	(164.95, 65.71) --
	(164.95, 65.71) --
	(164.88, 65.71) --
	(164.88, 65.71) --
	(164.88, 65.71) --
	(164.81, 65.71) --
	(164.81, 65.71) --
	(164.81, 65.71) --
	(164.73, 65.71) --
	(164.73, 65.71) --
	(164.73, 65.71) --
	(164.71, 65.71) --
	(164.71, 65.71) --
	(164.71, 65.71) --
	(164.67, 65.71) --
	(164.67, 65.71) --
	(164.67, 65.71) --
	(164.66, 65.71) --
	(164.66, 65.71) --
	(164.66, 65.71) --
	(164.66, 65.71) --
	(164.66, 65.71) --
	(164.66, 65.71) --
	(164.58, 65.71) --
	(164.58, 65.71) --
	(164.58, 65.71) --
	(164.55, 65.71) --
	(164.55, 65.71) --
	(164.55, 65.71) --
	(164.51, 65.71) --
	(164.51, 65.71) --
	(164.51, 65.71) --
	(164.43, 65.71) --
	(164.43, 65.71) --
	(164.43, 65.71) --
	(164.42, 65.71) --
	(164.42, 65.71) --
	(164.42, 65.71) --
	(164.36, 65.71) --
	(164.36, 65.71) --
	(164.36, 65.71) --
	(164.34, 65.71) --
	(164.34, 65.71) --
	(164.34, 65.71) --
	(164.28, 65.71) --
	(164.28, 65.71) --
	(164.28, 65.71) --
	(164.22, 65.71) --
	(164.22, 65.71) --
	(164.22, 65.71) --
	(164.21, 65.71) --
	(164.21, 65.71) --
	(164.21, 65.71) --
	(164.18, 65.71) --
	(164.18, 65.71) --
	(164.18, 65.71) --
	(164.14, 65.71) --
	(164.14, 65.71) --
	(164.14, 65.71) --
	(164.06, 65.71) --
	(164.06, 65.71) --
	(164.06, 65.71) --
	(164.02, 65.71) --
	(164.02, 65.71) --
	(164.02, 65.71) --
	(163.99, 65.71) --
	(163.99, 65.71) --
	(163.99, 65.71) --
	(163.91, 65.71) --
	(163.91, 65.71) --
	(163.91, 65.71) --
	(163.90, 65.71) --
	(163.90, 65.71) --
	(163.90, 65.71) --
	(163.84, 65.71) --
	(163.84, 65.71) --
	(163.84, 65.71) --
	(163.82, 65.71) --
	(163.82, 65.71) --
	(163.82, 65.71) --
	(163.76, 65.71) --
	(163.76, 65.71) --
	(163.76, 65.71) --
	(163.71, 65.71) --
	(163.71, 65.71) --
	(163.71, 65.71) --
	(163.70, 65.71) --
	(163.70, 65.71) --
	(163.70, 65.71) --
	(163.69, 65.71) --
	(163.69, 65.71) --
	(163.69, 65.71) --
	(163.66, 65.71) --
	(163.66, 65.71) --
	(163.66, 65.71) --
	(163.61, 65.71) --
	(163.61, 65.71) --
	(163.61, 65.71) --
	(163.57, 65.71) --
	(163.57, 65.71) --
	(163.57, 65.71) --
	(163.54, 65.71) --
	(163.54, 65.71) --
	(163.54, 65.71) --
	(163.53, 65.71) --
	(163.53, 65.71) --
	(163.53, 65.71) --
	(163.46, 65.71) --
	(163.46, 65.71) --
	(163.46, 65.71) --
	(163.45, 65.71) --
	(163.45, 65.71) --
	(163.45, 65.71) --
	(163.39, 65.71) --
	(163.39, 65.71) --
	(163.39, 65.71) --
	(163.37, 65.71) --
	(163.37, 65.71) --
	(163.37, 65.71) --
	(163.31, 65.71) --
	(163.31, 65.71) --
	(163.31, 65.71) --
	(163.25, 65.71) --
	(163.25, 65.71) --
	(163.25, 65.71) --
	(163.24, 65.71) --
	(163.24, 65.71) --
	(163.24, 65.71) --
	(163.16, 65.71) --
	(163.16, 65.71) --
	(163.16, 65.71) --
	(163.09, 65.71) --
	(163.09, 65.71) --
	(163.09, 65.71) --
	(163.09, 65.71) --
	(163.09, 65.71) --
	(163.09, 65.71) --
	(163.05, 65.71) --
	(163.05, 65.71) --
	(163.05, 65.71) --
	(163.01, 65.71) --
	(163.01, 65.71) --
	(163.01, 65.71) --
	(162.94, 65.71) --
	(162.94, 65.71) --
	(162.94, 65.71) --
	(162.93, 65.71) --
	(162.93, 65.71) --
	(162.93, 65.71) --
	(162.89, 65.71) --
	(162.89, 65.71) --
	(162.89, 65.71) --
	(162.87, 65.71) --
	(162.87, 65.71) --
	(162.87, 65.71) --
	(162.79, 65.71) --
	(162.79, 65.71) --
	(162.79, 65.71) --
	(162.73, 65.71) --
	(162.73, 65.71) --
	(162.73, 65.71) --
	(162.71, 65.71) --
	(162.71, 65.71) --
	(162.71, 65.71) --
	(162.69, 65.71) --
	(162.69, 65.71) --
	(162.69, 65.71) --
	(162.64, 65.71) --
	(162.64, 65.71) --
	(162.64, 65.71) --
	(162.56, 65.71) --
	(162.56, 65.71) --
	(162.56, 65.71) --
	(162.49, 65.71) --
	(162.49, 65.71) --
	(162.49, 65.71) --
	(162.48, 65.71) --
	(162.48, 65.71) --
	(162.48, 65.71) --
	(162.42, 65.71) --
	(162.42, 65.71) --
	(162.42, 65.71) --
	(162.34, 65.71) --
	(162.34, 65.71) --
	(162.34, 65.71) --
	(162.28, 65.71) --
	(162.28, 65.71) --
	(162.28, 65.71) --
	(162.27, 65.71) --
	(162.27, 65.71) --
	(162.27, 65.71) --
	(162.19, 65.71) --
	(162.19, 65.71) --
	(162.19, 65.71) --
	(162.12, 65.71) --
	(162.12, 65.71) --
	(162.12, 65.71) --
	(162.12, 65.71) --
	(162.12, 65.71) --
	(162.12, 65.71) --
	(162.04, 65.71) --
	(162.04, 65.71) --
	(162.04, 65.71) --
	(162.04, 65.71) --
	(162.04, 65.71) --
	(162.04, 65.71) --
	(162.00, 65.71) --
	(162.00, 65.71) --
	(162.00, 65.71) --
	(161.97, 65.71) --
	(161.97, 65.71) --
	(161.97, 65.71) --
	(161.96, 65.71) --
	(161.96, 65.71) --
	(161.96, 65.71) --
	(161.92, 65.71) --
	(161.92, 65.71) --
	(161.92, 65.71) --
	(161.89, 65.71) --
	(161.89, 65.71) --
	(161.89, 65.71) --
	(161.84, 65.71) --
	(161.84, 65.71) --
	(161.84, 65.71) --
	(161.82, 65.71) --
	(161.82, 65.71) --
	(161.82, 65.71) --
	(161.80, 65.71) --
	(161.80, 65.71) --
	(161.80, 65.71) --
	(161.74, 65.71) --
	(161.74, 65.71) --
	(161.74, 65.71) --
	(161.67, 65.71) --
	(161.67, 65.71) --
	(161.67, 65.71) --
	(161.59, 65.71) --
	(161.59, 65.71) --
	(161.59, 65.71) --
	(161.55, 65.71) --
	(161.55, 65.71) --
	(161.55, 65.71) --
	(161.52, 65.71) --
	(161.52, 65.71) --
	(161.52, 65.71) --
	(161.51, 65.71) --
	(161.51, 65.71) --
	(161.51, 65.71) --
	(161.44, 65.71) --
	(161.44, 65.71) --
	(161.44, 65.71) --
	(161.43, 65.71) --
	(161.43, 65.71) --
	(161.43, 65.71) --
	(161.39, 65.71) --
	(161.39, 65.71) --
	(161.39, 65.71) --
	(161.37, 65.71) --
	(161.37, 65.71) --
	(161.37, 65.71) --
	(161.31, 65.71) --
	(161.31, 65.71) --
	(161.31, 65.71) --
	(161.31, 65.71) --
	(161.31, 65.71) --
	(161.31, 65.71) --
	(161.29, 65.71) --
	(161.29, 65.71) --
	(161.29, 65.71) --
	(161.27, 65.71) --
	(161.27, 65.71) --
	(161.27, 65.71) --
	(161.23, 65.71) --
	(161.23, 65.71) --
	(161.23, 65.71) --
	(161.22, 65.71) --
	(161.22, 65.71) --
	(161.22, 65.71) --
	(161.14, 65.71) --
	(161.14, 65.71) --
	(161.14, 65.71) --
	(161.11, 65.71) --
	(161.11, 65.71) --
	(161.11, 65.71) --
	(161.07, 65.71) --
	(161.07, 65.71) --
	(161.07, 65.71) --
	(161.07, 65.71) --
	(161.07, 65.71) --
	(161.07, 65.71) --
	(160.99, 65.71) --
	(160.99, 65.71) --
	(160.99, 65.71) --
	(160.95, 65.71) --
	(160.95, 65.71) --
	(160.95, 65.71) --
	(160.92, 65.71) --
	(160.92, 65.71) --
	(160.92, 65.71) --
	(160.84, 65.71) --
	(160.84, 65.71) --
	(160.84, 65.71) --
	(160.79, 65.71) --
	(160.79, 65.71) --
	(160.79, 65.71) --
	(160.77, 65.71) --
	(160.77, 65.71) --
	(160.77, 65.71) --
	(160.71, 65.71) --
	(160.71, 65.71) --
	(160.71, 65.71) --
	(160.69, 65.71) --
	(160.69, 65.71) --
	(160.69, 65.71) --
	(160.62, 65.71) --
	(160.62, 65.71) --
	(160.62, 65.71) --
	(160.58, 65.71) --
	(160.58, 65.71) --
	(160.58, 65.71) --
	(160.54, 65.71) --
	(160.54, 65.71) --
	(160.54, 65.71) --
	(160.54, 65.71) --
	(160.54, 65.71) --
	(160.54, 65.71) --
	(160.47, 65.71) --
	(160.47, 65.71) --
	(160.47, 65.71) --
	(160.42, 65.71) --
	(160.42, 65.71) --
	(160.42, 65.71) --
	(160.39, 65.71) --
	(160.39, 65.71) --
	(160.39, 65.71) --
	(160.32, 65.71) --
	(160.32, 65.71) --
	(160.32, 65.71) --
	(160.30, 65.71) --
	(160.30, 65.71) --
	(160.30, 65.71) --
	(160.26, 65.71) --
	(160.26, 65.71) --
	(160.26, 65.71) --
	(160.24, 65.71) --
	(160.24, 65.71) --
	(160.24, 65.71) --
	(160.17, 65.71) --
	(160.17, 65.71) --
	(160.17, 65.71) --
	(160.14, 65.71) --
	(160.14, 65.71) --
	(160.14, 65.71) --
	(160.09, 65.71) --
	(160.09, 65.71) --
	(160.09, 65.71) --
	(160.02, 65.71) --
	(160.02, 65.71) --
	(160.02, 65.71) --
	(159.98, 65.71) --
	(159.98, 65.71) --
	(159.98, 65.71) --
	(159.94, 65.71) --
	(159.94, 65.71) --
	(159.94, 65.71) --
	(159.87, 65.71) --
	(159.87, 65.71) --
	(159.87, 65.71) --
	(159.80, 65.71) --
	(159.80, 65.71) --
	(159.80, 65.71) --
	(159.78, 65.71) --
	(159.78, 65.71) --
	(159.78, 65.71) --
	(159.72, 65.71) --
	(159.72, 65.71) --
	(159.72, 65.71) --
	(159.65, 65.71) --
	(159.65, 65.71) --
	(159.65, 65.71) --
	(159.57, 65.71) --
	(159.57, 65.71) --
	(159.57, 65.71) --
	(159.57, 65.71) --
	(159.57, 65.71) --
	(159.57, 65.71) --
	(159.49, 65.71) --
	(159.49, 65.71) --
	(159.49, 65.71) --
	(159.42, 65.71) --
	(159.42, 65.71) --
	(159.42, 65.71) --
	(159.34, 65.71) --
	(159.34, 65.71) --
	(159.34, 65.71) --
	(159.27, 65.71) --
	(159.27, 65.71) --
	(159.27, 65.71) --
	(159.25, 65.71) --
	(159.25, 65.71) --
	(159.25, 65.71) --
	(159.19, 65.71) --
	(159.19, 65.71) --
	(159.19, 65.71) --
	(159.12, 65.71) --
	(159.12, 65.71) --
	(159.12, 65.71) --
	(159.04, 65.71) --
	(159.04, 65.71) --
	(159.04, 65.71) --
	(158.97, 65.71) --
	(158.97, 65.71) --
	(158.97, 65.71) --
	(158.89, 65.71) --
	(158.89, 65.71) --
	(158.89, 65.71) --
	(158.82, 65.71) --
	(158.82, 65.71) --
	(158.82, 65.71) --
	(158.76, 65.71) --
	(158.76, 65.71) --
	(158.76, 65.71) --
	(158.75, 65.71) --
	(158.75, 65.71) --
	(158.75, 65.71) --
	(158.67, 65.71) --
	(158.67, 65.71) --
	(158.67, 65.71) --
	(158.60, 65.71) --
	(158.60, 65.71) --
	(158.60, 65.71) --
	(158.52, 65.71) --
	(158.52, 65.71) --
	(158.52, 65.71) --
	(158.45, 65.71) --
	(158.45, 65.71) --
	(158.45, 65.71) --
	(158.37, 65.71) --
	(158.37, 65.71) --
	(158.37, 65.71) --
	(158.30, 65.71) --
	(158.30, 65.71) --
	(158.30, 65.71) --
	(158.24, 65.71) --
	(158.24, 65.71) --
	(158.24, 65.71) --
	(158.22, 65.71) --
	(158.22, 65.71) --
	(158.22, 65.71) --
	(158.15, 65.71) --
	(158.15, 65.71) --
	(158.14, 65.71) --
	(158.07, 65.71) --
	(158.07, 65.71) --
	(158.07, 65.71) --
	(158.04, 65.71) --
	(158.04, 65.71) --
	(158.04, 65.71) --
	(157.99, 65.71) --
	(157.99, 65.71) --
	(157.99, 65.71) --
	(157.92, 65.71) --
	(157.92, 65.71) --
	(157.92, 65.71) --
	(157.84, 65.71) --
	(157.84, 65.71) --
	(157.84, 65.71) --
	(157.77, 65.71) --
	(157.77, 65.71) --
	(157.77, 65.71) --
	(157.69, 65.71) --
	(157.69, 65.71) --
	(157.69, 65.71) --
	(157.62, 65.71) --
	(157.62, 65.71) --
	(157.62, 65.71) --
	(157.54, 65.71) --
	(157.54, 65.71) --
	(157.54, 65.71) --
	(157.47, 65.71) --
	(157.47, 65.71) --
	(157.47, 65.71) --
	(157.39, 65.71) --
	(157.39, 65.71) --
	(157.39, 65.71) --
	(157.32, 65.71) --
	(157.32, 65.71) --
	(157.32, 65.71) --
	(157.24, 65.71) --
	(157.24, 65.71) --
	(157.24, 65.71) --
	(157.23, 65.71) --
	(157.23, 65.71) --
	(157.23, 65.71) --
	(157.17, 65.71) --
	(157.17, 65.71) --
	(157.17, 65.71) --
	(157.09, 65.71) --
	(157.09, 65.71) --
	(157.09, 65.71) --
	(157.03, 65.71) --
	(157.03, 65.71) --
	(157.03, 65.71) --
	(157.02, 65.71) --
	(157.02, 65.71) --
	(157.02, 65.71) --
	(156.94, 65.71) --
	(156.94, 65.71) --
	(156.94, 65.71) --
	(156.87, 65.71) --
	(156.87, 65.71) --
	(156.87, 65.71) --
	(156.87, 65.71) --
	(156.87, 65.71) --
	(156.87, 65.71) --
	(156.79, 65.71) --
	(156.79, 65.71) --
	(156.79, 65.71) --
	(156.72, 65.71) --
	(156.72, 65.71) --
	(156.72, 65.71) --
	(156.64, 65.71) --
	(156.64, 65.71) --
	(156.64, 65.71) --
	(156.57, 65.71) --
	(156.57, 65.71) --
	(156.57, 65.71) --
	(156.50, 65.71) --
	(156.50, 65.71) --
	(156.50, 65.71) --
	(156.49, 65.71) --
	(156.49, 65.71) --
	(156.49, 65.71) --
	(156.42, 65.71) --
	(156.42, 65.71) --
	(156.42, 65.71) --
	(156.34, 65.71) --
	(156.34, 65.71) --
	(156.34, 65.71) --
	(156.27, 65.71) --
	(156.27, 65.71) --
	(156.27, 65.71) --
	(156.23, 65.71) --
	(156.23, 65.71) --
	(156.23, 65.71) --
	(156.19, 65.71) --
	(156.19, 65.71) --
	(156.19, 65.71) --
	(156.12, 65.71) --
	(156.12, 65.71) --
	(156.12, 65.71) --
	(156.04, 65.71) --
	(156.04, 65.71) --
	(156.04, 65.71) --
	(155.97, 65.71) --
	(155.97, 65.71) --
	(155.97, 65.71) --
	(155.90, 65.71) --
	(155.90, 65.71) --
	(155.90, 65.71) --
	(155.89, 65.71) --
	(155.89, 65.71) --
	(155.89, 65.71) --
	(155.82, 65.71) --
	(155.82, 65.71) --
	(155.82, 65.71) --
	(155.74, 65.71) --
	(155.74, 65.71) --
	(155.74, 65.71) --
	(155.67, 65.71) --
	(155.67, 65.71) --
	(155.67, 65.71) --
	(155.59, 65.71) --
	(155.59, 65.71) --
	(155.59, 65.71) --
	(155.52, 65.71) --
	(155.52, 65.71) --
	(155.52, 65.71) --
	(155.44, 65.71) --
	(155.44, 65.71) --
	(155.44, 65.71) --
	(155.37, 65.71) --
	(155.37, 65.71) --
	(155.37, 65.71) --
	(155.29, 65.71) --
	(155.29, 65.71) --
	(155.29, 65.71) --
	(155.22, 65.71) --
	(155.22, 65.71) --
	(155.22, 65.71) --
	(155.14, 65.71) --
	(155.14, 65.71) --
	(155.14, 65.71) --
	(155.06, 65.71) --
	(155.06, 65.71) --
	(155.06, 65.71) --
	(154.99, 65.71) --
	(154.99, 65.71) --
	(154.99, 65.71) --
	(154.91, 65.71) --
	(154.91, 65.71) --
	(154.91, 65.71) --
	(154.84, 65.71) --
	(154.84, 65.71) --
	(154.84, 65.71) --
	(154.76, 65.71) --
	(154.76, 65.71) --
	(154.76, 65.71) --
	(154.69, 65.71) --
	(154.69, 65.71) --
	(154.69, 65.71) --
	(154.64, 65.71) --
	(154.64, 65.71) --
	(154.64, 65.71) --
	(154.61, 65.71) --
	(154.61, 65.71) --
	(154.61, 65.71) --
	(154.54, 65.71) --
	(154.54, 65.71) --
	(154.54, 65.71) --
	(154.46, 65.71) --
	(154.46, 65.71) --
	(154.46, 65.71) --
	(154.39, 65.71) --
	(154.39, 65.71) --
	(154.39, 65.71) --
	(154.31, 65.71) --
	(154.31, 65.71) --
	(154.31, 65.71) --
	(154.24, 65.71) --
	(154.24, 65.71) --
	(154.24, 65.71) --
	(154.16, 65.71) --
	(154.16, 65.71) --
	(154.16, 65.71) --
	(154.09, 65.71) --
	(154.09, 65.71) --
	(154.09, 65.71) --
	(154.01, 65.71) --
	(154.01, 65.71) --
	(154.01, 65.71) --
	(153.95, 65.71) --
	(153.95, 65.71) --
	(153.95, 65.71) --
	(153.94, 65.71) --
	(153.94, 65.71) --
	(153.94, 65.71) --
	(153.86, 65.71) --
	(153.86, 65.71) --
	(153.86, 65.71) --
	(153.79, 65.71) --
	(153.79, 65.71) --
	(153.79, 65.71) --
	(153.71, 65.71) --
	(153.71, 65.71) --
	(153.71, 65.71) --
	(153.67, 65.71) --
	(153.67, 65.71) --
	(153.67, 65.71) --
	(153.63, 65.71) --
	(153.63, 65.71) --
	(153.63, 65.71) --
	(153.59, 65.71) --
	(153.59, 65.71) --
	(153.59, 65.71) --
	(153.56, 65.71) --
	(153.56, 65.71) --
	(153.56, 65.71) --
	(153.48, 65.71) --
	(153.48, 65.71) --
	(153.48, 65.71) --
	(153.47, 65.71) --
	(153.47, 65.71) --
	(153.47, 65.71) --
	(153.41, 65.71) --
	(153.41, 65.71) --
	(153.41, 65.71) --
	(153.39, 65.71) --
	(153.39, 65.71) --
	(153.39, 65.71) --
	(153.35, 65.71) --
	(153.35, 65.71) --
	(153.35, 65.71) --
	(153.33, 65.71) --
	(153.33, 65.71) --
	(153.33, 65.71) --
	(153.26, 65.71) --
	(153.26, 65.71) --
	(153.26, 65.71) --
	(153.26, 65.71) --
	(153.26, 65.71) --
	(153.26, 65.71) --
	(153.18, 65.71) --
	(153.18, 65.71) --
	(153.18, 65.71) --
	(153.11, 65.71) --
	(153.11, 65.71) --
	(153.11, 65.71) --
	(153.11, 65.71) --
	(153.11, 65.71) --
	(153.11, 65.71) --
	(153.03, 65.71) --
	(153.03, 65.71) --
	(153.03, 65.71) --
	(152.96, 65.71) --
	(152.96, 65.71) --
	(152.96, 65.71) --
	(152.88, 65.71) --
	(152.88, 65.71) --
	(152.88, 65.71) --
	(152.81, 65.71) --
	(152.81, 65.71) --
	(152.81, 65.71) --
	(152.73, 65.71) --
	(152.73, 65.71) --
	(152.73, 65.71) --
	(152.66, 65.71) --
	(152.66, 65.71) --
	(152.66, 65.71) --
	(152.58, 65.71) --
	(152.58, 65.71) --
	(152.58, 65.71) --
	(152.51, 65.71) --
	(152.51, 65.71) --
	(152.51, 65.71) --
	(152.43, 65.71) --
	(152.43, 65.71) --
	(152.43, 65.71) --
	(152.35, 65.71) --
	(152.35, 65.71) --
	(152.35, 65.71) --
	(152.28, 65.71) --
	(152.28, 65.71) --
	(152.28, 65.71) --
	(152.20, 65.71) --
	(152.20, 65.71) --
	(152.20, 65.71) --
	(152.13, 65.71) --
	(152.13, 65.71) --
	(152.13, 65.71) --
	(152.05, 65.71) --
	(152.05, 65.71) --
	(152.05, 65.71) --
	(151.98, 65.71) --
	(151.98, 65.71) --
	(151.98, 65.71) --
	(151.90, 65.71) --
	(151.90, 65.71) --
	(151.90, 65.71) --
	(151.83, 65.71) --
	(151.83, 65.71) --
	(151.83, 65.71) --
	(151.75, 65.71) --
	(151.75, 65.71) --
	(151.75, 65.71) --
	(151.68, 65.71) --
	(151.68, 65.71) --
	(151.68, 65.71) --
	(151.60, 65.71) --
	(151.60, 65.71) --
	(151.60, 65.71) --
	(151.53, 65.71) --
	(151.53, 65.71) --
	(151.53, 65.71) --
	(151.45, 65.71) --
	(151.45, 65.71) --
	(151.45, 65.71) --
	(151.37, 65.71) --
	(151.37, 65.71) --
	(151.37, 65.71) --
	(151.30, 65.71) --
	(151.30, 65.71) --
	(151.30, 65.71) --
	(151.22, 65.71) --
	(151.22, 65.71) --
	(151.22, 65.71) --
	(151.15, 65.71) --
	(151.15, 65.71) --
	(151.15, 65.71) --
	(151.07, 65.71) --
	(151.07, 65.71) --
	(151.07, 65.71) --
	(151.00, 65.71) --
	(151.00, 65.71) --
	(151.00, 65.71) --
	(150.92, 65.71) --
	(150.92, 65.71) --
	(150.92, 65.71) --
	(150.85, 65.71) --
	(150.85, 65.71) --
	(150.85, 65.71) --
	(150.77, 65.71) --
	(150.77, 65.71) --
	(150.77, 65.71) --
	(150.70, 65.71) --
	(150.70, 65.71) --
	(150.70, 65.71) --
	(150.62, 65.71) --
	(150.62, 65.71) --
	(150.62, 65.71) --
	(150.55, 65.71) --
	(150.55, 65.71) --
	(150.55, 65.71) --
	(150.47, 65.71) --
	(150.47, 65.71) --
	(150.47, 65.71) --
	(150.39, 65.71) --
	(150.39, 65.71) --
	(150.39, 65.71) --
	(150.32, 65.71) --
	(150.32, 65.71) --
	(150.32, 65.71) --
	(150.24, 65.71) --
	(150.24, 65.71) --
	(150.24, 65.71) --
	(150.17, 65.71) --
	(150.17, 65.71) --
	(150.17, 65.71) --
	(150.09, 65.71) --
	(150.09, 65.71) --
	(150.09, 65.71) --
	(150.02, 65.71) --
	(150.02, 65.71) --
	(150.02, 65.71) --
	(149.94, 65.71) --
	(149.94, 65.71) --
	(149.94, 65.71) --
	(149.87, 65.71) --
	(149.87, 65.71) --
	(149.87, 65.71) --
	(149.79, 65.71) --
	(149.79, 65.71) --
	(149.79, 65.71) --
	(149.72, 65.71) --
	(149.72, 65.71) --
	(149.72, 65.71) --
	(149.64, 65.71) --
	(149.64, 65.71) --
	(149.64, 65.71) --
	(149.56, 65.71) --
	(149.56, 65.71) --
	(149.56, 65.71) --
	(149.49, 65.71) --
	(149.49, 65.71) --
	(149.49, 65.71) --
	(149.41, 65.71) --
	(149.41, 65.71) --
	(149.41, 65.71) --
	(149.34, 65.71) --
	(149.34, 65.71) --
	(149.34, 65.71) --
	(149.26, 65.71) --
	(149.26, 65.71) --
	(149.26, 65.71) --
	(149.19, 65.71) --
	(149.19, 65.71) --
	(149.19, 65.71) --
	(149.11, 65.71) --
	(149.11, 65.71) --
	(149.11, 65.71) --
	(149.04, 65.71) --
	(149.04, 65.71) --
	(149.04, 65.71) --
	(148.96, 65.71) --
	(148.96, 65.71) --
	(148.96, 65.71) --
	(148.88, 65.71) --
	(148.88, 65.71) --
	(148.88, 65.71) --
	(148.81, 65.71) --
	(148.81, 65.71) --
	(148.81, 65.71) --
	(148.73, 65.71) --
	(148.73, 65.71) --
	(148.73, 65.71) --
	(148.66, 65.71) --
	(148.66, 65.71) --
	(148.66, 65.71) --
	(148.58, 65.71) --
	(148.58, 65.71) --
	(148.58, 65.71) --
	(148.51, 65.71) --
	(148.51, 65.71) --
	(148.51, 65.71) --
	(148.43, 65.71) --
	(148.43, 65.71) --
	(148.43, 65.71) --
	(148.36, 65.71) --
	(148.36, 65.71) --
	(148.36, 65.71) --
	(148.28, 65.71) --
	(148.28, 65.71) --
	(148.28, 65.71) --
	(148.20, 65.71) --
	(148.20, 65.71) --
	(148.20, 65.71) --
	(148.13, 65.71) --
	(148.13, 65.71) --
	(148.13, 65.71) --
	(148.05, 65.71) --
	(148.05, 65.71) --
	(148.05, 65.71) --
	(147.98, 65.71) --
	(147.98, 65.71) --
	(147.98, 65.71) --
	(147.90, 65.71) --
	(147.90, 65.71) --
	(147.90, 65.71) --
	(147.83, 65.71) --
	(147.83, 65.71) --
	(147.83, 65.71) --
	(147.75, 65.71) --
	(147.75, 65.71) --
	(147.75, 65.71) --
	(147.68, 65.71) --
	(147.68, 65.71) --
	(147.68, 65.71) --
	(147.60, 65.71) --
	(147.60, 65.71) --
	(147.60, 65.71) --
	(147.52, 65.71) --
	(147.52, 65.71) --
	(147.52, 65.71) --
	(147.45, 65.71) --
	(147.45, 65.71) --
	(147.45, 65.71) --
	(147.37, 65.71) --
	(147.37, 65.71) --
	(147.37, 65.71) --
	(147.30, 65.71) --
	(147.30, 65.71) --
	(147.30, 65.71) --
	(147.22, 65.71) --
	(147.22, 65.71) --
	(147.22, 65.71) --
	(147.15, 65.71) --
	(147.15, 65.71) --
	(147.15, 65.71) --
	(147.07, 65.71) --
	(147.07, 65.71) --
	(147.07, 65.71) --
	(146.99, 65.71) --
	(146.99, 65.71) --
	(146.99, 65.71) --
	(146.92, 65.71) --
	(146.92, 65.71) --
	(146.92, 65.71) --
	(146.84, 65.71) --
	(146.84, 65.71) --
	(146.84, 65.71) --
	(146.77, 65.71) --
	(146.77, 65.71) --
	(146.77, 65.71) --
	(146.69, 65.71) --
	(146.69, 65.71) --
	(146.69, 65.71) --
	(146.62, 65.71) --
	(146.62, 65.71) --
	(146.62, 65.71) --
	(146.54, 65.71) --
	(146.54, 65.71) --
	(146.54, 65.71) --
	(146.47, 65.71) --
	(146.47, 65.71) --
	(146.47, 65.71) --
	(146.39, 65.71) --
	(146.39, 65.71) --
	(146.39, 65.71) --
	(146.32, 65.71) --
	(146.32, 65.71) --
	(146.32, 65.71) --
	(146.24, 65.71) --
	(146.24, 65.71) --
	(146.24, 65.71) --
	(146.16, 65.71) --
	(146.16, 65.71) --
	(146.16, 65.71) --
	(146.09, 65.71) --
	(146.09, 65.71) --
	(146.09, 65.71) --
	(146.01, 65.71) --
	(146.01, 65.71) --
	(146.01, 65.71) --
	(145.94, 65.71) --
	(145.94, 65.71) --
	(145.94, 65.71) --
	(145.86, 65.71) --
	(145.86, 65.71) --
	(145.86, 65.71) --
	(145.79, 65.71) --
	(145.79, 65.71) --
	(145.79, 65.71) --
	(145.71, 65.71) --
	(145.71, 65.71) --
	(145.71, 65.71) --
	(145.63, 65.71) --
	(145.63, 65.71) --
	(145.63, 65.71) --
	(145.56, 65.71) --
	(145.56, 65.71) --
	(145.56, 65.71) --
	(145.48, 65.71) --
	(145.48, 65.71) --
	(145.48, 65.71) --
	(145.41, 65.71) --
	(145.41, 65.71) --
	(145.41, 65.71) --
	(145.33, 65.71) --
	(145.33, 65.71) --
	(145.33, 65.71) --
	(145.26, 65.71) --
	(145.26, 65.71) --
	(145.26, 65.71) --
	(145.18, 65.71) --
	(145.18, 65.71) --
	(145.18, 65.71) --
	(145.10, 65.71) --
	(145.10, 65.71) --
	(145.10, 65.71) --
	(145.03, 65.71) --
	(145.03, 65.71) --
	(145.03, 65.71) --
	(144.95, 65.71) --
	(144.95, 65.71) --
	(144.95, 65.71) --
	(144.88, 65.71) --
	(144.88, 65.71) --
	(144.88, 65.71) --
	(144.80, 65.71) --
	(144.80, 65.71) --
	(144.80, 65.71) --
	(144.72, 65.71) --
	(144.72, 65.71) --
	(144.72, 65.71) --
	(144.65, 65.71) --
	(144.65, 65.71) --
	(144.65, 65.71) --
	(144.57, 65.71) --
	(144.57, 65.71) --
	(144.57, 65.71) --
	(144.50, 65.71) --
	(144.50, 65.71) --
	(144.50, 65.71) --
	(144.42, 65.71) --
	(144.42, 65.71) --
	(144.42, 65.71) --
	(144.35, 65.71) --
	(144.35, 65.71) --
	(144.35, 65.71) --
	(144.27, 65.71) --
	(144.27, 65.71) --
	(144.27, 65.71) --
	(144.19, 65.71) --
	(144.19, 65.71) --
	(144.19, 65.71) --
	(144.12, 65.71) --
	(144.12, 65.71) --
	(144.12, 65.71) --
	(144.04, 65.71) --
	(144.04, 65.71) --
	(144.04, 65.71) --
	(143.97, 65.71) --
	(143.97, 65.71) --
	(143.97, 65.71) --
	(143.89, 65.71) --
	(143.89, 65.71) --
	(143.89, 65.71) --
	(143.82, 65.71) --
	(143.82, 65.71) --
	(143.82, 65.71) --
	(143.74, 65.71) --
	(143.74, 65.71) --
	(143.74, 65.71) --
	(143.66, 65.71) --
	(143.66, 65.71) --
	(143.66, 65.71) --
	(143.59, 65.71) --
	(143.59, 65.71) --
	(143.59, 65.71) --
	(143.51, 65.71) --
	(143.51, 65.71) --
	(143.51, 65.71) --
	(143.44, 65.71) --
	(143.44, 65.71) --
	(143.44, 65.71) --
	(143.36, 65.71) --
	(143.36, 65.71) --
	(143.36, 65.71) --
	(143.29, 65.71) --
	(143.29, 65.71) --
	(143.29, 65.71) --
	(143.21, 65.71) --
	(143.21, 65.71) --
	(143.21, 65.71) --
	(143.13, 65.71) --
	(143.13, 65.71) --
	(143.13, 65.71) --
	(143.06, 65.71) --
	(143.06, 65.71) --
	(143.06, 65.71) --
	(142.98, 65.71) --
	(142.98, 65.71) --
	(142.98, 65.71) --
	(142.91, 65.71) --
	(142.91, 65.71) --
	(142.91, 65.71) --
	(142.83, 65.71) --
	(142.83, 65.71) --
	(142.83, 65.71) --
	(142.75, 65.71) --
	(142.75, 65.71) --
	(142.75, 65.71) --
	(142.68, 65.71) --
	(142.68, 65.71) --
	(142.68, 65.71) --
	(142.60, 65.71) --
	(142.60, 65.71) --
	(142.60, 65.71) --
	(142.53, 65.71) --
	(142.53, 65.71) --
	(142.53, 65.71) --
	(142.45, 65.71) --
	(142.45, 65.71) --
	(142.45, 65.71) --
	(142.37, 65.71) --
	(142.37, 65.71) --
	(142.37, 65.71) --
	(142.30, 65.71) --
	(142.30, 65.71) --
	(142.30, 65.71) --
	(142.22, 65.71) --
	(142.22, 65.71) --
	(142.22, 65.71) --
	(142.15, 65.71) --
	(142.15, 65.71) --
	(142.15, 65.71) --
	(142.07, 65.71) --
	(142.07, 65.71) --
	(142.07, 65.71) --
	(142.00, 65.71) --
	(142.00, 65.71) --
	(142.00, 65.71) --
	(141.92, 65.71) --
	(141.92, 65.71) --
	(141.92, 65.71) --
	(141.84, 65.71) --
	(141.84, 65.71) --
	(141.84, 65.71) --
	(141.77, 65.71) --
	(141.77, 65.71) --
	(141.77, 65.71) --
	(141.69, 65.71) --
	(141.69, 65.71) --
	(141.69, 65.71) --
	(141.62, 65.71) --
	(141.62, 65.71) --
	(141.62, 65.71) --
	(141.54, 65.71) --
	(141.54, 65.71) --
	(141.54, 65.71) --
	(141.46, 65.71) --
	(141.46, 65.71) --
	(141.46, 65.71) --
	(141.39, 65.71) --
	(141.39, 65.71) --
	(141.39, 65.71) --
	(141.31, 65.71) --
	(141.31, 65.71) --
	(141.31, 65.71) --
	(141.24, 65.71) --
	(141.24, 65.71) --
	(141.24, 65.71) --
	(141.16, 65.71) --
	(141.16, 65.71) --
	(141.16, 65.71) --
	(141.08, 65.71) --
	(141.08, 65.71) --
	(141.08, 65.71) --
	(141.01, 65.71) --
	(141.01, 65.71) --
	(141.01, 65.71) --
	(140.93, 65.71) --
	(140.93, 65.71) --
	(140.93, 65.71) --
	(140.86, 65.71) --
	(140.86, 65.71) --
	(140.86, 65.71) --
	(140.78, 65.71) --
	(140.78, 65.71) --
	(140.78, 65.71) --
	(140.71, 65.71) --
	(140.71, 65.71) --
	(140.71, 65.71) --
	(140.63, 65.71) --
	(140.63, 65.71) --
	(140.63, 65.71) --
	(140.55, 65.71) --
	(140.55, 65.71) --
	(140.55, 65.71) --
	(140.48, 65.71) --
	(140.48, 65.71) --
	(140.48, 65.71) --
	(140.40, 65.71) --
	(140.40, 65.71) --
	(140.40, 65.71) --
	(140.33, 65.71) --
	(140.33, 65.71) --
	(140.33, 65.71) --
	(140.25, 65.71) --
	(140.25, 65.71) --
	(140.25, 65.71) --
	(140.17, 65.71) --
	(140.17, 65.71) --
	(140.17, 65.71) --
	(140.10, 65.71) --
	(140.10, 65.71) --
	(140.10, 65.71) --
	(140.02, 65.71) --
	(140.02, 65.71) --
	(140.02, 65.71) --
	(139.95, 65.71) --
	(139.95, 65.71) --
	(139.95, 65.71) --
	(139.87, 65.71) --
	(139.87, 65.71) --
	(139.87, 65.71) --
	(139.79, 65.71) --
	(139.79, 65.71) --
	(139.79, 65.71) --
	(139.72, 65.71) --
	(139.72, 65.71) --
	(139.72, 65.71) --
	(139.64, 65.71) --
	(139.64, 65.71) --
	(139.64, 65.71) --
	(139.57, 65.71) --
	(139.57, 65.71) --
	(139.57, 65.71) --
	(139.49, 65.71) --
	(139.49, 65.71) --
	(139.49, 65.71) --
	(139.42, 65.71) --
	(139.42, 65.71) --
	(139.42, 65.71) --
	(139.34, 65.71) --
	(139.34, 65.71) --
	(139.34, 65.71) --
	(139.26, 65.71) --
	(139.26, 65.71) --
	(139.26, 65.71) --
	(139.19, 65.71) --
	(139.19, 65.71) --
	(139.19, 65.71) --
	(139.11, 65.71) --
	(139.11, 65.71) --
	(139.11, 65.71) --
	(139.03, 65.71) --
	(139.03, 65.71) --
	(139.03, 65.71) --
	(138.96, 65.71) --
	(138.96, 65.71) --
	(138.96, 65.71) --
	(138.88, 65.71) --
	(138.88, 65.71) --
	(138.88, 65.71) --
	(138.81, 65.71) --
	(138.81, 65.71) --
	(138.81, 65.71) --
	(138.73, 65.71) --
	(138.73, 65.71) --
	(138.73, 65.71) --
	(138.65, 65.71) --
	(138.65, 65.71) --
	(138.65, 65.71) --
	(138.58, 65.71) --
	(138.58, 65.71) --
	(138.58, 65.71) --
	(138.50, 65.71) --
	(138.50, 65.71) --
	(138.50, 65.71) --
	(138.43, 65.71) --
	(138.43, 65.71) --
	(138.43, 65.71) --
	(138.35, 65.71) --
	(138.35, 65.71) --
	(138.35, 65.71) --
	(138.27, 65.71) --
	(138.27, 65.71) --
	(138.27, 65.71) --
	(138.20, 65.71) --
	(138.20, 65.71) --
	(138.20, 65.71) --
	(138.12, 65.71) --
	(138.12, 65.71) --
	(138.12, 65.71) --
	(138.05, 65.71) --
	(138.05, 65.71) --
	(138.05, 65.71) --
	(137.97, 65.71) --
	(137.97, 65.71) --
	(137.97, 65.71) --
	(137.89, 65.71) --
	(137.89, 65.71) --
	(137.89, 65.71) --
	(137.82, 65.71) --
	(137.82, 65.71) --
	(137.82, 65.71) --
	(137.74, 65.71) --
	(137.74, 65.71) --
	(137.74, 65.71) --
	(137.66, 65.71) --
	(137.66, 65.71) --
	(137.66, 65.71) --
	(137.59, 65.71) --
	(137.59, 65.71) --
	(137.59, 65.71) --
	(137.51, 65.71) --
	(137.51, 65.71) --
	(137.51, 65.71) --
	(137.44, 65.71) --
	(137.44, 65.71) --
	(137.44, 65.71) --
	(137.36, 65.71) --
	(137.36, 65.71) --
	(137.36, 65.71) --
	(137.29, 65.71) --
	(137.29, 65.71) --
	(137.29, 65.71) --
	(137.21, 65.71) --
	(137.21, 65.71) --
	(137.21, 65.71) --
	(137.13, 65.71) --
	(137.13, 65.71) --
	(137.13, 65.71) --
	(137.06, 65.71) --
	(137.06, 65.71) --
	(137.06, 65.71) --
	(136.98, 65.71) --
	(136.98, 65.71) --
	(136.98, 65.71) --
	(136.90, 65.71) --
	(136.90, 65.71) --
	(136.90, 65.71) --
	(136.83, 65.71) --
	(136.83, 65.71) --
	(136.83, 65.71) --
	(136.75, 65.71) --
	(136.75, 65.71) --
	(136.75, 65.71) --
	(136.68, 65.71) --
	(136.68, 65.71) --
	(136.68, 65.71) --
	(136.60, 65.71) --
	(136.60, 65.71) --
	(136.60, 65.71) --
	(136.52, 65.71) --
	(136.52, 65.71) --
	(136.52, 65.71) --
	(136.45, 65.71) --
	(136.45, 65.71) --
	(136.45, 65.71) --
	(136.37, 65.71) --
	(136.37, 65.71) --
	(136.37, 65.71) --
	(136.30, 65.71) --
	(136.30, 65.71) --
	(136.30, 65.71) --
	(136.22, 65.71) --
	(136.22, 65.71) --
	(136.22, 65.71) --
	(136.14, 65.71) --
	(136.14, 65.71) --
	(136.14, 65.71) --
	(136.07, 65.71) --
	(136.07, 65.71) --
	(136.07, 65.71) --
	(135.99, 65.71) --
	(135.99, 65.71) --
	(135.99, 65.71) --
	(135.91, 65.71) --
	(135.91, 65.71) --
	(135.91, 65.71) --
	(135.84, 65.71) --
	(135.84, 65.71) --
	(135.84, 65.71) --
	(135.76, 65.71) --
	(135.76, 65.71) --
	(135.76, 65.71) --
	(135.69, 65.71) --
	(135.69, 65.71) --
	(135.69, 65.71) --
	(135.61, 65.71) --
	(135.61, 65.71) --
	(135.61, 65.71) --
	(135.53, 65.71) --
	(135.53, 65.71) --
	(135.53, 65.71) --
	(135.46, 65.71) --
	(135.46, 65.71) --
	(135.46, 65.71) --
	(135.38, 65.71) --
	(135.38, 65.71) --
	(135.38, 65.71) --
	(135.31, 65.71) --
	(135.31, 65.71) --
	(135.31, 65.71) --
	(135.23, 65.71) --
	(135.23, 65.71) --
	(135.23, 65.71) --
	(135.15, 65.71) --
	(135.15, 65.71) --
	(135.15, 65.71) --
	(135.08, 65.71) --
	(135.08, 65.71) --
	(135.08, 65.71) --
	(135.00, 65.71) --
	(135.00, 65.71) --
	(135.00, 65.71) --
	(134.92, 65.71) --
	(134.92, 65.71) --
	(134.92, 65.71) --
	(134.85, 65.71) --
	(134.85, 65.71) --
	(134.85, 65.71) --
	(134.77, 65.71) --
	(134.77, 65.71) --
	(134.77, 65.71) --
	(134.70, 65.71) --
	(134.69, 65.71) --
	(134.69, 65.71) --
	(134.62, 65.71) --
	(134.62, 65.71) --
	(134.62, 65.71) --
	(134.54, 65.71) --
	(134.54, 65.71) --
	(134.54, 65.71) --
	(134.47, 65.71) --
	(134.47, 65.71) --
	(134.47, 65.71) --
	(134.39, 65.71) --
	(134.39, 65.71) --
	(134.39, 65.71) --
	(134.31, 65.71) --
	(134.31, 65.71) --
	(134.31, 65.71) --
	(134.24, 65.71) --
	(134.24, 65.71) --
	(134.24, 65.71) --
	(134.16, 65.71) --
	(134.16, 65.71) --
	(134.16, 65.71) --
	(134.08, 65.71) --
	(134.08, 65.71) --
	(134.08, 65.71) --
	(134.01, 65.71) --
	(134.01, 65.71) --
	(134.01, 65.71) --
	(133.93, 65.71) --
	(133.93, 65.71) --
	(133.93, 65.71) --
	(133.86, 65.71) --
	(133.86, 65.71) --
	(133.86, 65.71) --
	(133.78, 65.71) --
	(133.78, 65.71) --
	(133.78, 65.71) --
	(133.70, 65.71) --
	(133.70, 65.71) --
	(133.70, 65.71) --
	(133.63, 65.71) --
	(133.63, 65.71) --
	(133.63, 65.71) --
	(133.55, 65.71) --
	(133.55, 65.71) --
	(133.55, 65.71) --
	(133.48, 65.71) --
	(133.48, 65.71) --
	(133.48, 65.71) --
	cycle;
\definecolor{drawColor}{RGB}{231,107,243}

\path[draw=drawColor,line width= 0.6pt,line join=round] (133.48, 65.72) --
	(133.48, 65.72) --
	(133.55, 65.72) --
	(133.55, 65.72) --
	(133.55, 65.72) --
	(133.63, 65.71) --
	(133.63, 65.71) --
	(133.63, 65.71) --
	(133.70, 65.72) --
	(133.70, 65.72) --
	(133.70, 65.72) --
	(133.78, 65.71) --
	(133.78, 65.71) --
	(133.78, 65.71) --
	(133.86, 65.71) --
	(133.86, 65.71) --
	(133.86, 65.71) --
	(133.93, 65.72) --
	(133.93, 65.72) --
	(133.93, 65.72) --
	(134.01, 65.71) --
	(134.01, 65.71) --
	(134.01, 65.71) --
	(134.08, 65.72) --
	(134.08, 65.72) --
	(134.08, 65.72) --
	(134.16, 65.71) --
	(134.16, 65.71) --
	(134.16, 65.71) --
	(134.24, 65.72) --
	(134.24, 65.72) --
	(134.24, 65.72) --
	(134.31, 65.72) --
	(134.31, 65.72) --
	(134.31, 65.72) --
	(134.39, 65.72) --
	(134.39, 65.72) --
	(134.39, 65.72) --
	(134.47, 65.72) --
	(134.47, 65.72) --
	(134.47, 65.72) --
	(134.54, 65.72) --
	(134.54, 65.72) --
	(134.54, 65.72) --
	(134.62, 65.72) --
	(134.62, 65.72) --
	(134.62, 65.72) --
	(134.69, 65.72) --
	(134.69, 65.72) --
	(134.70, 65.72) --
	(134.77, 65.71) --
	(134.77, 65.71) --
	(134.77, 65.71) --
	(134.85, 65.71) --
	(134.85, 65.71) --
	(134.85, 65.71) --
	(134.92, 65.72) --
	(134.92, 65.72) --
	(134.92, 65.72) --
	(135.00, 65.72) --
	(135.00, 65.72) --
	(135.00, 65.72) --
	(135.08, 65.72) --
	(135.08, 65.72) --
	(135.08, 65.72) --
	(135.15, 65.72) --
	(135.15, 65.72) --
	(135.15, 65.72) --
	(135.23, 65.71) --
	(135.23, 65.71) --
	(135.23, 65.71) --
	(135.31, 65.72) --
	(135.31, 65.72) --
	(135.31, 65.72) --
	(135.38, 65.72) --
	(135.38, 65.72) --
	(135.38, 65.72) --
	(135.46, 65.72) --
	(135.46, 65.72) --
	(135.46, 65.72) --
	(135.53, 65.73) --
	(135.53, 65.73) --
	(135.53, 65.73) --
	(135.61, 65.73) --
	(135.61, 65.73) --
	(135.61, 65.73) --
	(135.69, 65.72) --
	(135.69, 65.72) --
	(135.69, 65.72) --
	(135.76, 65.73) --
	(135.76, 65.73) --
	(135.76, 65.73) --
	(135.84, 65.73) --
	(135.84, 65.73) --
	(135.84, 65.73) --
	(135.91, 65.73) --
	(135.91, 65.73) --
	(135.91, 65.73) --
	(135.99, 65.72) --
	(135.99, 65.72) --
	(135.99, 65.72) --
	(136.07, 65.72) --
	(136.07, 65.72) --
	(136.07, 65.72) --
	(136.14, 65.72) --
	(136.14, 65.72) --
	(136.14, 65.72) --
	(136.22, 65.72) --
	(136.22, 65.72) --
	(136.22, 65.72) --
	(136.30, 65.72) --
	(136.30, 65.72) --
	(136.30, 65.72) --
	(136.37, 65.72) --
	(136.37, 65.72) --
	(136.37, 65.72) --
	(136.45, 65.73) --
	(136.45, 65.73) --
	(136.45, 65.73) --
	(136.52, 65.72) --
	(136.52, 65.72) --
	(136.52, 65.72) --
	(136.60, 65.72) --
	(136.60, 65.72) --
	(136.60, 65.72) --
	(136.68, 65.72) --
	(136.68, 65.72) --
	(136.68, 65.72) --
	(136.75, 65.72) --
	(136.75, 65.72) --
	(136.75, 65.72) --
	(136.83, 65.73) --
	(136.83, 65.73) --
	(136.83, 65.73) --
	(136.90, 65.73) --
	(136.90, 65.73) --
	(136.90, 65.73) --
	(136.98, 65.72) --
	(136.98, 65.72) --
	(136.98, 65.72) --
	(137.06, 65.72) --
	(137.06, 65.72) --
	(137.06, 65.72) --
	(137.13, 65.72) --
	(137.13, 65.72) --
	(137.13, 65.72) --
	(137.21, 65.73) --
	(137.21, 65.73) --
	(137.21, 65.73) --
	(137.29, 65.72) --
	(137.29, 65.72) --
	(137.29, 65.72) --
	(137.36, 65.72) --
	(137.36, 65.72) --
	(137.36, 65.72) --
	(137.44, 65.72) --
	(137.44, 65.72) --
	(137.44, 65.72) --
	(137.51, 65.72) --
	(137.51, 65.72) --
	(137.51, 65.72) --
	(137.59, 65.72) --
	(137.59, 65.72) --
	(137.59, 65.72) --
	(137.66, 65.72) --
	(137.66, 65.72) --
	(137.66, 65.72) --
	(137.74, 65.73) --
	(137.74, 65.73) --
	(137.74, 65.73) --
	(137.82, 65.72) --
	(137.82, 65.72) --
	(137.82, 65.72) --
	(137.89, 65.72) --
	(137.89, 65.72) --
	(137.89, 65.72) --
	(137.97, 65.71) --
	(137.97, 65.71) --
	(137.97, 65.71) --
	(138.05, 65.73) --
	(138.05, 65.73) --
	(138.05, 65.73) --
	(138.12, 65.73) --
	(138.12, 65.73) --
	(138.12, 65.73) --
	(138.20, 65.73) --
	(138.20, 65.73) --
	(138.20, 65.73) --
	(138.27, 65.72) --
	(138.27, 65.72) --
	(138.27, 65.72) --
	(138.35, 65.72) --
	(138.35, 65.72) --
	(138.35, 65.72) --
	(138.43, 65.73) --
	(138.43, 65.73) --
	(138.43, 65.73) --
	(138.50, 65.73) --
	(138.50, 65.73) --
	(138.50, 65.73) --
	(138.58, 65.72) --
	(138.58, 65.72) --
	(138.58, 65.72) --
	(138.65, 65.72) --
	(138.65, 65.72) --
	(138.65, 65.72) --
	(138.73, 65.72) --
	(138.73, 65.72) --
	(138.73, 65.72) --
	(138.81, 65.72) --
	(138.81, 65.72) --
	(138.81, 65.72) --
	(138.88, 65.73) --
	(138.88, 65.73) --
	(138.88, 65.73) --
	(138.96, 65.73) --
	(138.96, 65.73) --
	(138.96, 65.73) --
	(139.03, 65.73) --
	(139.03, 65.73) --
	(139.03, 65.73) --
	(139.11, 65.72) --
	(139.11, 65.72) --
	(139.11, 65.72) --
	(139.19, 65.73) --
	(139.19, 65.73) --
	(139.19, 65.73) --
	(139.26, 65.72) --
	(139.26, 65.72) --
	(139.26, 65.72) --
	(139.34, 65.72) --
	(139.34, 65.72) --
	(139.34, 65.72) --
	(139.42, 65.73) --
	(139.42, 65.73) --
	(139.42, 65.73) --
	(139.49, 65.72) --
	(139.49, 65.72) --
	(139.49, 65.72) --
	(139.57, 65.72) --
	(139.57, 65.72) --
	(139.57, 65.72) --
	(139.64, 65.73) --
	(139.64, 65.73) --
	(139.64, 65.73) --
	(139.72, 65.73) --
	(139.72, 65.73) --
	(139.72, 65.73) --
	(139.79, 65.72) --
	(139.79, 65.72) --
	(139.79, 65.72) --
	(139.87, 65.72) --
	(139.87, 65.72) --
	(139.87, 65.72) --
	(139.95, 65.72) --
	(139.95, 65.72) --
	(139.95, 65.72) --
	(140.02, 65.73) --
	(140.02, 65.73) --
	(140.02, 65.73) --
	(140.10, 65.72) --
	(140.10, 65.72) --
	(140.10, 65.72) --
	(140.17, 65.72) --
	(140.17, 65.72) --
	(140.17, 65.72) --
	(140.25, 65.73) --
	(140.25, 65.73) --
	(140.25, 65.73) --
	(140.33, 65.73) --
	(140.33, 65.73) --
	(140.33, 65.73) --
	(140.40, 65.72) --
	(140.40, 65.72) --
	(140.40, 65.72) --
	(140.48, 65.72) --
	(140.48, 65.72) --
	(140.48, 65.72) --
	(140.55, 65.72) --
	(140.55, 65.72) --
	(140.55, 65.72) --
	(140.63, 65.72) --
	(140.63, 65.72) --
	(140.63, 65.72) --
	(140.71, 65.72) --
	(140.71, 65.72) --
	(140.71, 65.72) --
	(140.78, 65.73) --
	(140.78, 65.73) --
	(140.78, 65.73) --
	(140.86, 65.73) --
	(140.86, 65.73) --
	(140.86, 65.73) --
	(140.93, 65.72) --
	(140.93, 65.72) --
	(140.93, 65.72) --
	(141.01, 65.72) --
	(141.01, 65.72) --
	(141.01, 65.72) --
	(141.08, 65.72) --
	(141.08, 65.72) --
	(141.08, 65.72) --
	(141.16, 65.72) --
	(141.16, 65.72) --
	(141.16, 65.72) --
	(141.24, 65.73) --
	(141.24, 65.73) --
	(141.24, 65.73) --
	(141.31, 65.73) --
	(141.31, 65.73) --
	(141.31, 65.73) --
	(141.39, 65.72) --
	(141.39, 65.72) --
	(141.39, 65.72) --
	(141.46, 65.72) --
	(141.46, 65.72) --
	(141.46, 65.72) --
	(141.54, 65.72) --
	(141.54, 65.72) --
	(141.54, 65.72) --
	(141.62, 65.72) --
	(141.62, 65.72) --
	(141.62, 65.72) --
	(141.69, 65.73) --
	(141.69, 65.73) --
	(141.69, 65.73) --
	(141.77, 65.72) --
	(141.77, 65.72) --
	(141.77, 65.72) --
	(141.84, 65.73) --
	(141.84, 65.73) --
	(141.84, 65.73) --
	(141.92, 65.73) --
	(141.92, 65.73) --
	(141.92, 65.73) --
	(142.00, 65.73) --
	(142.00, 65.73) --
	(142.00, 65.73) --
	(142.07, 65.73) --
	(142.07, 65.73) --
	(142.07, 65.73) --
	(142.15, 65.73) --
	(142.15, 65.73) --
	(142.15, 65.73) --
	(142.22, 65.73) --
	(142.22, 65.73) --
	(142.22, 65.73) --
	(142.30, 65.73) --
	(142.30, 65.73) --
	(142.30, 65.73) --
	(142.37, 65.72) --
	(142.37, 65.72) --
	(142.37, 65.72) --
	(142.45, 65.72) --
	(142.45, 65.72) --
	(142.45, 65.72) --
	(142.53, 65.73) --
	(142.53, 65.73) --
	(142.53, 65.73) --
	(142.60, 65.73) --
	(142.60, 65.73) --
	(142.60, 65.73) --
	(142.68, 65.73) --
	(142.68, 65.73) --
	(142.68, 65.73) --
	(142.75, 65.72) --
	(142.75, 65.72) --
	(142.75, 65.72) --
	(142.83, 65.72) --
	(142.83, 65.72) --
	(142.83, 65.72) --
	(142.91, 65.72) --
	(142.91, 65.72) --
	(142.91, 65.72) --
	(142.98, 65.72) --
	(142.98, 65.72) --
	(142.98, 65.72) --
	(143.06, 65.73) --
	(143.06, 65.73) --
	(143.06, 65.73) --
	(143.13, 65.72) --
	(143.13, 65.72) --
	(143.13, 65.72) --
	(143.21, 65.72) --
	(143.21, 65.72) --
	(143.21, 65.72) --
	(143.29, 65.72) --
	(143.29, 65.72) --
	(143.29, 65.72) --
	(143.36, 65.73) --
	(143.36, 65.73) --
	(143.36, 65.73) --
	(143.44, 65.72) --
	(143.44, 65.72) --
	(143.44, 65.72) --
	(143.51, 65.72) --
	(143.51, 65.72) --
	(143.51, 65.72) --
	(143.59, 65.73) --
	(143.59, 65.73) --
	(143.59, 65.73) --
	(143.66, 65.73) --
	(143.66, 65.73) --
	(143.66, 65.73) --
	(143.74, 65.72) --
	(143.74, 65.72) --
	(143.74, 65.72) --
	(143.82, 65.72) --
	(143.82, 65.72) --
	(143.82, 65.72) --
	(143.89, 65.73) --
	(143.89, 65.73) --
	(143.89, 65.73) --
	(143.97, 65.73) --
	(143.97, 65.73) --
	(143.97, 65.73) --
	(144.04, 65.72) --
	(144.04, 65.72) --
	(144.04, 65.72) --
	(144.12, 65.72) --
	(144.12, 65.72) --
	(144.12, 65.72) --
	(144.19, 65.72) --
	(144.19, 65.72) --
	(144.19, 65.72) --
	(144.27, 65.72) --
	(144.27, 65.72) --
	(144.27, 65.72) --
	(144.35, 65.72) --
	(144.35, 65.72) --
	(144.35, 65.72) --
	(144.42, 65.73) --
	(144.42, 65.73) --
	(144.42, 65.73) --
	(144.50, 65.72) --
	(144.50, 65.72) --
	(144.50, 65.72) --
	(144.57, 65.72) --
	(144.57, 65.72) --
	(144.57, 65.72) --
	(144.65, 65.73) --
	(144.65, 65.73) --
	(144.65, 65.73) --
	(144.72, 65.72) --
	(144.72, 65.72) --
	(144.72, 65.72) --
	(144.80, 65.72) --
	(144.80, 65.72) --
	(144.80, 65.72) --
	(144.88, 65.73) --
	(144.88, 65.73) --
	(144.88, 65.73) --
	(144.95, 65.73) --
	(144.95, 65.73) --
	(144.95, 65.73) --
	(145.03, 65.72) --
	(145.03, 65.72) --
	(145.03, 65.72) --
	(145.10, 65.73) --
	(145.10, 65.73) --
	(145.10, 65.73) --
	(145.18, 65.73) --
	(145.18, 65.73) --
	(145.18, 65.73) --
	(145.26, 65.73) --
	(145.26, 65.73) --
	(145.26, 65.73) --
	(145.33, 65.73) --
	(145.33, 65.73) --
	(145.33, 65.73) --
	(145.41, 65.72) --
	(145.41, 65.72) --
	(145.41, 65.72) --
	(145.48, 65.73) --
	(145.48, 65.73) --
	(145.48, 65.73) --
	(145.56, 65.73) --
	(145.56, 65.73) --
	(145.56, 65.73) --
	(145.63, 65.73) --
	(145.63, 65.73) --
	(145.63, 65.73) --
	(145.71, 65.73) --
	(145.71, 65.73) --
	(145.71, 65.73) --
	(145.79, 65.73) --
	(145.79, 65.73) --
	(145.79, 65.73) --
	(145.86, 65.73) --
	(145.86, 65.73) --
	(145.86, 65.73) --
	(145.94, 65.72) --
	(145.94, 65.72) --
	(145.94, 65.72) --
	(146.01, 65.73) --
	(146.01, 65.73) --
	(146.01, 65.73) --
	(146.09, 65.72) --
	(146.09, 65.72) --
	(146.09, 65.72) --
	(146.16, 65.72) --
	(146.16, 65.72) --
	(146.16, 65.72) --
	(146.24, 65.73) --
	(146.24, 65.73) --
	(146.24, 65.73) --
	(146.32, 65.73) --
	(146.32, 65.73) --
	(146.32, 65.73) --
	(146.39, 65.72) --
	(146.39, 65.72) --
	(146.39, 65.72) --
	(146.47, 65.72) --
	(146.47, 65.72) --
	(146.47, 65.72) --
	(146.54, 65.73) --
	(146.54, 65.73) --
	(146.54, 65.73) --
	(146.62, 65.73) --
	(146.62, 65.73) --
	(146.62, 65.73) --
	(146.69, 65.72) --
	(146.69, 65.72) --
	(146.69, 65.72) --
	(146.77, 65.72) --
	(146.77, 65.72) --
	(146.77, 65.72) --
	(146.84, 65.72) --
	(146.84, 65.72) --
	(146.84, 65.72) --
	(146.92, 65.73) --
	(146.92, 65.73) --
	(146.92, 65.73) --
	(146.99, 65.72) --
	(146.99, 65.72) --
	(146.99, 65.72) --
	(147.07, 65.72) --
	(147.07, 65.72) --
	(147.07, 65.72) --
	(147.15, 65.73) --
	(147.15, 65.73) --
	(147.15, 65.73) --
	(147.22, 65.73) --
	(147.22, 65.73) --
	(147.22, 65.73) --
	(147.30, 65.72) --
	(147.30, 65.72) --
	(147.30, 65.72) --
	(147.37, 65.73) --
	(147.37, 65.73) --
	(147.37, 65.73) --
	(147.45, 65.73) --
	(147.45, 65.73) --
	(147.45, 65.73) --
	(147.52, 65.73) --
	(147.52, 65.73) --
	(147.52, 65.73) --
	(147.60, 65.73) --
	(147.60, 65.73) --
	(147.60, 65.73) --
	(147.68, 65.73) --
	(147.68, 65.73) --
	(147.68, 65.73) --
	(147.75, 65.73) --
	(147.75, 65.73) --
	(147.75, 65.73) --
	(147.83, 65.72) --
	(147.83, 65.72) --
	(147.83, 65.72) --
	(147.90, 65.72) --
	(147.90, 65.72) --
	(147.90, 65.72) --
	(147.98, 65.73) --
	(147.98, 65.73) --
	(147.98, 65.73) --
	(148.05, 65.73) --
	(148.05, 65.73) --
	(148.05, 65.73) --
	(148.13, 65.73) --
	(148.13, 65.73) --
	(148.13, 65.73) --
	(148.20, 65.73) --
	(148.20, 65.73) --
	(148.20, 65.73) --
	(148.28, 65.73) --
	(148.28, 65.73) --
	(148.28, 65.73) --
	(148.36, 65.73) --
	(148.36, 65.73) --
	(148.36, 65.73) --
	(148.43, 65.72) --
	(148.43, 65.72) --
	(148.43, 65.72) --
	(148.51, 65.73) --
	(148.51, 65.73) --
	(148.51, 65.73) --
	(148.58, 65.72) --
	(148.58, 65.72) --
	(148.58, 65.72) --
	(148.66, 65.72) --
	(148.66, 65.72) --
	(148.66, 65.72) --
	(148.73, 65.73) --
	(148.73, 65.73) --
	(148.73, 65.73) --
	(148.81, 65.73) --
	(148.81, 65.73) --
	(148.81, 65.73) --
	(148.88, 65.73) --
	(148.88, 65.73) --
	(148.88, 65.73) --
	(148.96, 65.72) --
	(148.96, 65.72) --
	(148.96, 65.72) --
	(149.04, 65.73) --
	(149.04, 65.73) --
	(149.04, 65.73) --
	(149.11, 65.73) --
	(149.11, 65.73) --
	(149.11, 65.73) --
	(149.19, 65.73) --
	(149.19, 65.73) --
	(149.19, 65.73) --
	(149.26, 65.73) --
	(149.26, 65.73) --
	(149.26, 65.73) --
	(149.34, 65.72) --
	(149.34, 65.72) --
	(149.34, 65.72) --
	(149.41, 65.72) --
	(149.41, 65.72) --
	(149.41, 65.72) --
	(149.49, 65.73) --
	(149.49, 65.73) --
	(149.49, 65.73) --
	(149.56, 65.72) --
	(149.56, 65.72) --
	(149.56, 65.72) --
	(149.64, 65.73) --
	(149.64, 65.73) --
	(149.64, 65.73) --
	(149.72, 65.73) --
	(149.72, 65.73) --
	(149.72, 65.73) --
	(149.79, 65.73) --
	(149.79, 65.73) --
	(149.79, 65.73) --
	(149.87, 65.72) --
	(149.87, 65.72) --
	(149.87, 65.72) --
	(149.94, 65.73) --
	(149.94, 65.73) --
	(149.94, 65.73) --
	(150.02, 65.72) --
	(150.02, 65.72) --
	(150.02, 65.72) --
	(150.09, 65.73) --
	(150.09, 65.73) --
	(150.09, 65.73) --
	(150.17, 65.73) --
	(150.17, 65.73) --
	(150.17, 65.73) --
	(150.24, 65.73) --
	(150.24, 65.73) --
	(150.24, 65.73) --
	(150.32, 65.73) --
	(150.32, 65.73) --
	(150.32, 65.73) --
	(150.39, 65.73) --
	(150.39, 65.73) --
	(150.39, 65.73) --
	(150.47, 65.73) --
	(150.47, 65.73) --
	(150.47, 65.73) --
	(150.55, 65.73) --
	(150.55, 65.73) --
	(150.55, 65.73) --
	(150.62, 65.73) --
	(150.62, 65.73) --
	(150.62, 65.73) --
	(150.70, 65.72) --
	(150.70, 65.72) --
	(150.70, 65.72) --
	(150.77, 65.73) --
	(150.77, 65.73) --
	(150.77, 65.73) --
	(150.85, 65.72) --
	(150.85, 65.72) --
	(150.85, 65.72) --
	(150.92, 65.73) --
	(150.92, 65.73) --
	(150.92, 65.73) --
	(151.00, 65.73) --
	(151.00, 65.73) --
	(151.00, 65.73) --
	(151.07, 65.72) --
	(151.07, 65.72) --
	(151.07, 65.72) --
	(151.15, 65.73) --
	(151.15, 65.73) --
	(151.15, 65.73) --
	(151.22, 65.73) --
	(151.22, 65.73) --
	(151.22, 65.73) --
	(151.30, 65.73) --
	(151.30, 65.73) --
	(151.30, 65.73) --
	(151.37, 65.72) --
	(151.37, 65.72) --
	(151.37, 65.72) --
	(151.45, 65.72) --
	(151.45, 65.72) --
	(151.45, 65.72) --
	(151.53, 65.73) --
	(151.53, 65.73) --
	(151.53, 65.73) --
	(151.60, 65.73) --
	(151.60, 65.73) --
	(151.60, 65.73) --
	(151.68, 65.73) --
	(151.68, 65.73) --
	(151.68, 65.73) --
	(151.75, 65.72) --
	(151.75, 65.72) --
	(151.75, 65.72) --
	(151.83, 65.73) --
	(151.83, 65.73) --
	(151.83, 65.73) --
	(151.90, 65.73) --
	(151.90, 65.73) --
	(151.90, 65.73) --
	(151.98, 65.73) --
	(151.98, 65.73) --
	(151.98, 65.73) --
	(152.05, 65.73) --
	(152.05, 65.73) --
	(152.05, 65.73) --
	(152.13, 65.73) --
	(152.13, 65.73) --
	(152.13, 65.73) --
	(152.20, 65.73) --
	(152.20, 65.73) --
	(152.20, 65.73) --
	(152.28, 65.73) --
	(152.28, 65.73) --
	(152.28, 65.73) --
	(152.35, 65.73) --
	(152.35, 65.73) --
	(152.35, 65.73) --
	(152.43, 65.73) --
	(152.43, 65.73) --
	(152.43, 65.73) --
	(152.51, 65.72) --
	(152.51, 65.72) --
	(152.51, 65.72) --
	(152.58, 65.73) --
	(152.58, 65.73) --
	(152.58, 65.73) --
	(152.66, 65.73) --
	(152.66, 65.73) --
	(152.66, 65.73) --
	(152.73, 65.73) --
	(152.73, 65.73) --
	(152.73, 65.73) --
	(152.81, 65.72) --
	(152.81, 65.72) --
	(152.81, 65.72) --
	(152.88, 65.73) --
	(152.88, 65.73) --
	(152.88, 65.73) --
	(152.96, 65.73) --
	(152.96, 65.73) --
	(152.96, 65.73) --
	(153.03, 65.73) --
	(153.03, 65.73) --
	(153.03, 65.73) --
	(153.11, 65.73) --
	(153.11, 65.73) --
	(153.11, 65.73) --
	(153.11, 65.73) --
	(153.11, 65.73) --
	(153.11, 65.73) --
	(153.18, 65.73) --
	(153.18, 65.73) --
	(153.18, 65.73) --
	(153.26, 65.72) --
	(153.26, 65.72) --
	(153.26, 65.72) --
	(153.26, 65.72) --
	(153.26, 65.72) --
	(153.26, 65.72) --
	(153.33, 65.73) --
	(153.33, 65.73) --
	(153.33, 65.73) --
	(153.35, 65.73) --
	(153.35, 65.73) --
	(153.35, 65.73) --
	(153.39, 65.73) --
	(153.39, 65.73) --
	(153.39, 65.73) --
	(153.41, 65.73) --
	(153.41, 65.73) --
	(153.41, 65.73) --
	(153.47, 65.73) --
	(153.47, 65.73) --
	(153.47, 65.73) --
	(153.48, 65.73) --
	(153.48, 65.73) --
	(153.48, 65.73) --
	(153.56, 65.72) --
	(153.56, 65.72) --
	(153.56, 65.72) --
	(153.59, 65.72) --
	(153.59, 65.72) --
	(153.59, 65.72) --
	(153.63, 65.73) --
	(153.63, 65.73) --
	(153.63, 65.73) --
	(153.67, 65.73) --
	(153.67, 65.73) --
	(153.67, 65.73) --
	(153.71, 65.73) --
	(153.71, 65.73) --
	(153.71, 65.73) --
	(153.79, 65.73) --
	(153.79, 65.73) --
	(153.79, 65.73) --
	(153.86, 65.72) --
	(153.86, 65.72) --
	(153.86, 65.72) --
	(153.94, 65.73) --
	(153.94, 65.73) --
	(153.94, 65.73) --
	(153.95, 65.73) --
	(153.95, 65.73) --
	(153.95, 65.73) --
	(154.01, 65.73) --
	(154.01, 65.73) --
	(154.01, 65.73) --
	(154.09, 65.73) --
	(154.09, 65.73) --
	(154.09, 65.73) --
	(154.16, 65.72) --
	(154.16, 65.72) --
	(154.16, 65.72) --
	(154.24, 65.73) --
	(154.24, 65.73) --
	(154.24, 65.73) --
	(154.31, 65.73) --
	(154.31, 65.73) --
	(154.31, 65.73) --
	(154.39, 65.73) --
	(154.39, 65.73) --
	(154.39, 65.73) --
	(154.46, 65.73) --
	(154.46, 65.73) --
	(154.46, 65.73) --
	(154.54, 65.73) --
	(154.54, 65.73) --
	(154.54, 65.73) --
	(154.61, 65.73) --
	(154.61, 65.73) --
	(154.61, 65.73) --
	(154.64, 65.73) --
	(154.64, 65.73) --
	(154.64, 65.73) --
	(154.69, 65.73) --
	(154.69, 65.73) --
	(154.69, 65.73) --
	(154.76, 65.73) --
	(154.76, 65.73) --
	(154.76, 65.73) --
	(154.84, 65.73) --
	(154.84, 65.73) --
	(154.84, 65.73) --
	(154.91, 65.73) --
	(154.91, 65.73) --
	(154.91, 65.73) --
	(154.99, 65.72) --
	(154.99, 65.72) --
	(154.99, 65.72) --
	(155.06, 65.73) --
	(155.06, 65.73) --
	(155.06, 65.73) --
	(155.14, 65.73) --
	(155.14, 65.73) --
	(155.14, 65.73) --
	(155.22, 65.73) --
	(155.22, 65.73) --
	(155.22, 65.73) --
	(155.29, 65.73) --
	(155.29, 65.73) --
	(155.29, 65.73) --
	(155.37, 65.73) --
	(155.37, 65.73) --
	(155.37, 65.73) --
	(155.44, 65.73) --
	(155.44, 65.73) --
	(155.44, 65.73) --
	(155.52, 65.73) --
	(155.52, 65.73) --
	(155.52, 65.73) --
	(155.59, 65.73) --
	(155.59, 65.73) --
	(155.59, 65.73) --
	(155.67, 65.73) --
	(155.67, 65.73) --
	(155.67, 65.73) --
	(155.74, 65.73) --
	(155.74, 65.73) --
	(155.74, 65.73) --
	(155.82, 65.73) --
	(155.82, 65.73) --
	(155.82, 65.73) --
	(155.89, 65.73) --
	(155.89, 65.73) --
	(155.89, 65.73) --
	(155.90, 65.73) --
	(155.90, 65.73) --
	(155.90, 65.73) --
	(155.97, 65.73) --
	(155.97, 65.73) --
	(155.97, 65.73) --
	(156.04, 65.73) --
	(156.04, 65.73) --
	(156.04, 65.73) --
	(156.12, 65.73) --
	(156.12, 65.73) --
	(156.12, 65.73) --
	(156.19, 65.73) --
	(156.19, 65.73) --
	(156.19, 65.73) --
	(156.23, 65.73) --
	(156.23, 65.73) --
	(156.23, 65.73) --
	(156.27, 65.74) --
	(156.27, 65.74) --
	(156.27, 65.74) --
	(156.34, 65.73) --
	(156.34, 65.73) --
	(156.34, 65.73) --
	(156.42, 65.73) --
	(156.42, 65.73) --
	(156.42, 65.73) --
	(156.49, 65.73) --
	(156.49, 65.73) --
	(156.49, 65.73) --
	(156.50, 65.73) --
	(156.50, 65.73) --
	(156.50, 65.73) --
	(156.57, 65.73) --
	(156.57, 65.73) --
	(156.57, 65.73) --
	(156.64, 65.73) --
	(156.64, 65.73) --
	(156.64, 65.73) --
	(156.72, 65.73) --
	(156.72, 65.73) --
	(156.72, 65.73) --
	(156.79, 65.74) --
	(156.79, 65.74) --
	(156.79, 65.74) --
	(156.87, 65.73) --
	(156.87, 65.73) --
	(156.87, 65.73) --
	(156.87, 65.73) --
	(156.87, 65.73) --
	(156.87, 65.73) --
	(156.94, 65.73) --
	(156.94, 65.73) --
	(156.94, 65.73) --
	(157.02, 65.73) --
	(157.02, 65.73) --
	(157.02, 65.73) --
	(157.03, 65.73) --
	(157.03, 65.73) --
	(157.03, 65.73) --
	(157.09, 65.73) --
	(157.09, 65.73) --
	(157.09, 65.73) --
	(157.17, 65.73) --
	(157.17, 65.73) --
	(157.17, 65.73) --
	(157.23, 65.73) --
	(157.23, 65.73) --
	(157.23, 65.73) --
	(157.24, 65.73) --
	(157.24, 65.73) --
	(157.24, 65.73) --
	(157.32, 65.73) --
	(157.32, 65.73) --
	(157.32, 65.73) --
	(157.39, 65.73) --
	(157.39, 65.73) --
	(157.39, 65.73) --
	(157.47, 65.73) --
	(157.47, 65.73) --
	(157.47, 65.73) --
	(157.54, 65.73) --
	(157.54, 65.73) --
	(157.54, 65.73) --
	(157.62, 65.73) --
	(157.62, 65.73) --
	(157.62, 65.73) --
	(157.69, 65.73) --
	(157.69, 65.73) --
	(157.69, 65.73) --
	(157.77, 65.73) --
	(157.77, 65.73) --
	(157.77, 65.73) --
	(157.84, 65.73) --
	(157.84, 65.73) --
	(157.84, 65.73) --
	(157.92, 65.73) --
	(157.92, 65.73) --
	(157.92, 65.73) --
	(157.99, 65.73) --
	(157.99, 65.73) --
	(157.99, 65.73) --
	(158.04, 65.73) --
	(158.04, 65.73) --
	(158.04, 65.73) --
	(158.07, 65.73) --
	(158.07, 65.73) --
	(158.07, 65.73) --
	(158.14, 65.73) --
	(158.15, 65.73) --
	(158.15, 65.73) --
	(158.22, 65.73) --
	(158.22, 65.73) --
	(158.22, 65.73) --
	(158.24, 65.73) --
	(158.24, 65.73) --
	(158.24, 65.73) --
	(158.30, 65.73) --
	(158.30, 65.73) --
	(158.30, 65.73) --
	(158.37, 65.73) --
	(158.37, 65.73) --
	(158.37, 65.73) --
	(158.45, 65.73) --
	(158.45, 65.73) --
	(158.45, 65.73) --
	(158.52, 65.72) --
	(158.52, 65.72) --
	(158.52, 65.72) --
	(158.60, 65.73) --
	(158.60, 65.73) --
	(158.60, 65.73) --
	(158.67, 65.74) --
	(158.67, 65.74) --
	(158.67, 65.74) --
	(158.75, 65.73) --
	(158.75, 65.73) --
	(158.75, 65.73) --
	(158.76, 65.73) --
	(158.76, 65.73) --
	(158.76, 65.73) --
	(158.82, 65.74) --
	(158.82, 65.74) --
	(158.82, 65.74) --
	(158.89, 65.73) --
	(158.89, 65.73) --
	(158.89, 65.73) --
	(158.97, 65.74) --
	(158.97, 65.74) --
	(158.97, 65.74) --
	(159.04, 65.73) --
	(159.04, 65.73) --
	(159.04, 65.73) --
	(159.12, 65.73) --
	(159.12, 65.73) --
	(159.12, 65.73) --
	(159.19, 65.73) --
	(159.19, 65.73) --
	(159.19, 65.73) --
	(159.25, 65.72) --
	(159.25, 65.72) --
	(159.25, 65.72) --
	(159.27, 65.72) --
	(159.27, 65.72) --
	(159.27, 65.72) --
	(159.34, 65.73) --
	(159.34, 65.73) --
	(159.34, 65.73) --
	(159.42, 65.73) --
	(159.42, 65.73) --
	(159.42, 65.73) --
	(159.49, 65.74) --
	(159.49, 65.74) --
	(159.49, 65.74) --
	(159.57, 65.73) --
	(159.57, 65.73) --
	(159.57, 65.73) --
	(159.57, 65.73) --
	(159.57, 65.73) --
	(159.57, 65.73) --
	(159.65, 65.73) --
	(159.65, 65.73) --
	(159.65, 65.73) --
	(159.72, 65.73) --
	(159.72, 65.73) --
	(159.72, 65.73) --
	(159.78, 65.73) --
	(159.78, 65.73) --
	(159.78, 65.73) --
	(159.80, 65.73) --
	(159.80, 65.73) --
	(159.80, 65.73) --
	(159.87, 65.73) --
	(159.87, 65.73) --
	(159.87, 65.73) --
	(159.94, 65.73) --
	(159.94, 65.73) --
	(159.94, 65.73) --
	(159.98, 65.73) --
	(159.98, 65.73) --
	(159.98, 65.73) --
	(160.02, 65.73) --
	(160.02, 65.73) --
	(160.02, 65.73) --
	(160.09, 65.74) --
	(160.09, 65.74) --
	(160.09, 65.74) --
	(160.14, 65.74) --
	(160.14, 65.74) --
	(160.14, 65.74) --
	(160.17, 65.74) --
	(160.17, 65.74) --
	(160.17, 65.74) --
	(160.24, 65.74) --
	(160.24, 65.74) --
	(160.24, 65.74) --
	(160.26, 65.74) --
	(160.26, 65.74) --
	(160.26, 65.74) --
	(160.30, 65.73) --
	(160.30, 65.73) --
	(160.30, 65.73) --
	(160.32, 65.73) --
	(160.32, 65.73) --
	(160.32, 65.73) --
	(160.39, 65.73) --
	(160.39, 65.73) --
	(160.39, 65.73) --
	(160.42, 65.73) --
	(160.42, 65.73) --
	(160.42, 65.73) --
	(160.47, 65.73) --
	(160.47, 65.73) --
	(160.47, 65.73) --
	(160.54, 65.73) --
	(160.54, 65.73) --
	(160.54, 65.73) --
	(160.54, 65.73) --
	(160.54, 65.73) --
	(160.54, 65.73) --
	(160.58, 65.73) --
	(160.58, 65.73) --
	(160.58, 65.73) --
	(160.62, 65.73) --
	(160.62, 65.73) --
	(160.62, 65.73) --
	(160.69, 65.74) --
	(160.69, 65.74) --
	(160.69, 65.74) --
	(160.71, 65.73) --
	(160.71, 65.73) --
	(160.71, 65.73) --
	(160.77, 65.73) --
	(160.77, 65.73) --
	(160.77, 65.73) --
	(160.79, 65.73) --
	(160.79, 65.73) --
	(160.79, 65.73) --
	(160.84, 65.73) --
	(160.84, 65.73) --
	(160.84, 65.73) --
	(160.92, 65.73) --
	(160.92, 65.73) --
	(160.92, 65.73) --
	(160.95, 65.73) --
	(160.95, 65.73) --
	(160.95, 65.73) --
	(160.99, 65.74) --
	(160.99, 65.74) --
	(160.99, 65.74) --
	(161.07, 65.74) --
	(161.07, 65.74) --
	(161.07, 65.74) --
	(161.07, 65.74) --
	(161.07, 65.74) --
	(161.07, 65.74) --
	(161.11, 65.74) --
	(161.11, 65.74) --
	(161.11, 65.74) --
	(161.14, 65.74) --
	(161.14, 65.74) --
	(161.14, 65.74) --
	(161.22, 65.74) --
	(161.22, 65.74) --
	(161.22, 65.74) --
	(161.23, 65.74) --
	(161.23, 65.74) --
	(161.23, 65.74) --
	(161.27, 65.74) --
	(161.27, 65.74) --
	(161.27, 65.74) --
	(161.29, 65.74) --
	(161.29, 65.74) --
	(161.29, 65.74) --
	(161.31, 65.74) --
	(161.31, 65.74) --
	(161.31, 65.74) --
	(161.31, 65.73) --
	(161.31, 65.73) --
	(161.31, 65.73) --
	(161.37, 65.73) --
	(161.37, 65.73) --
	(161.37, 65.73) --
	(161.39, 65.73) --
	(161.39, 65.73) --
	(161.39, 65.73) --
	(161.43, 65.73) --
	(161.43, 65.73) --
	(161.43, 65.73) --
	(161.44, 65.73) --
	(161.44, 65.73) --
	(161.44, 65.73) --
	(161.51, 65.74) --
	(161.51, 65.74) --
	(161.51, 65.74) --
	(161.52, 65.74) --
	(161.52, 65.74) --
	(161.52, 65.74) --
	(161.55, 65.74) --
	(161.55, 65.74) --
	(161.55, 65.74) --
	(161.59, 65.74) --
	(161.59, 65.74) --
	(161.59, 65.74) --
	(161.67, 65.74) --
	(161.67, 65.74) --
	(161.67, 65.74) --
	(161.74, 65.74) --
	(161.74, 65.74) --
	(161.74, 65.74) --
	(161.80, 65.74) --
	(161.80, 65.74) --
	(161.80, 65.74) --
	(161.82, 65.74) --
	(161.82, 65.74) --
	(161.82, 65.74) --
	(161.84, 65.74) --
	(161.84, 65.74) --
	(161.84, 65.74) --
	(161.89, 65.74) --
	(161.89, 65.74) --
	(161.89, 65.74) --
	(161.92, 65.74) --
	(161.92, 65.74) --
	(161.92, 65.74) --
	(161.96, 65.74) --
	(161.96, 65.74) --
	(161.96, 65.74) --
	(161.97, 65.74) --
	(161.97, 65.74) --
	(161.97, 65.74) --
	(162.00, 65.74) --
	(162.00, 65.74) --
	(162.00, 65.74) --
	(162.04, 65.74) --
	(162.04, 65.74) --
	(162.04, 65.74) --
	(162.04, 65.74) --
	(162.04, 65.74) --
	(162.04, 65.74) --
	(162.12, 65.75) --
	(162.12, 65.75) --
	(162.12, 65.75) --
	(162.12, 65.75) --
	(162.12, 65.75) --
	(162.12, 65.75) --
	(162.19, 65.75) --
	(162.19, 65.75) --
	(162.19, 65.75) --
	(162.27, 65.74) --
	(162.27, 65.74) --
	(162.27, 65.74) --
	(162.28, 65.74) --
	(162.28, 65.74) --
	(162.28, 65.74) --
	(162.34, 65.75) --
	(162.34, 65.75) --
	(162.34, 65.75) --
	(162.42, 65.76) --
	(162.42, 65.76) --
	(162.42, 65.76) --
	(162.48, 65.75) --
	(162.48, 65.75) --
	(162.48, 65.75) --
	(162.49, 65.75) --
	(162.49, 65.75) --
	(162.49, 65.75) --
	(162.56, 65.75) --
	(162.56, 65.75) --
	(162.56, 65.75) --
	(162.64, 65.75) --
	(162.64, 65.75) --
	(162.64, 65.75) --
	(162.69, 65.76) --
	(162.69, 65.76) --
	(162.69, 65.76) --
	(162.71, 65.76) --
	(162.71, 65.76) --
	(162.71, 65.76) --
	(162.73, 65.76) --
	(162.73, 65.76) --
	(162.73, 65.76) --
	(162.79, 65.76) --
	(162.79, 65.76) --
	(162.79, 65.76) --
	(162.87, 65.76) --
	(162.87, 65.76) --
	(162.87, 65.76) --
	(162.89, 65.76) --
	(162.89, 65.76) --
	(162.89, 65.76) --
	(162.93, 65.76) --
	(162.93, 65.76) --
	(162.93, 65.76) --
	(162.94, 65.76) --
	(162.94, 65.76) --
	(162.94, 65.76) --
	(163.01, 65.76) --
	(163.01, 65.76) --
	(163.01, 65.76) --
	(163.05, 65.76) --
	(163.05, 65.76) --
	(163.05, 65.76) --
	(163.09, 65.76) --
	(163.09, 65.76) --
	(163.09, 65.76) --
	(163.09, 65.76) --
	(163.09, 65.76) --
	(163.09, 65.76) --
	(163.16, 65.77) --
	(163.16, 65.77) --
	(163.16, 65.77) --
	(163.24, 65.77) --
	(163.24, 65.77) --
	(163.24, 65.77) --
	(163.25, 65.76) --
	(163.25, 65.76) --
	(163.25, 65.76) --
	(163.31, 65.76) --
	(163.31, 65.76) --
	(163.31, 65.76) --
	(163.37, 65.76) --
	(163.37, 65.76) --
	(163.37, 65.76) --
	(163.39, 65.76) --
	(163.39, 65.76) --
	(163.39, 65.76) --
	(163.45, 65.76) --
	(163.45, 65.76) --
	(163.45, 65.76) --
	(163.46, 65.76) --
	(163.46, 65.76) --
	(163.46, 65.76) --
	(163.53, 65.77) --
	(163.53, 65.77) --
	(163.53, 65.77) --
	(163.54, 65.77) --
	(163.54, 65.77) --
	(163.54, 65.77) --
	(163.57, 65.77) --
	(163.57, 65.77) --
	(163.57, 65.77) --
	(163.61, 65.76) --
	(163.61, 65.76) --
	(163.61, 65.76) --
	(163.66, 65.76) --
	(163.66, 65.76) --
	(163.66, 65.76) --
	(163.69, 65.77) --
	(163.69, 65.77) --
	(163.69, 65.77) --
	(163.70, 65.77) --
	(163.70, 65.77) --
	(163.70, 65.77) --
	(163.71, 65.77) --
	(163.71, 65.77) --
	(163.71, 65.77) --
	(163.76, 65.76) --
	(163.76, 65.76) --
	(163.76, 65.76) --
	(163.82, 65.76) --
	(163.82, 65.76) --
	(163.82, 65.76) --
	(163.84, 65.76) --
	(163.84, 65.76) --
	(163.84, 65.76) --
	(163.90, 65.77) --
	(163.90, 65.77) --
	(163.90, 65.77) --
	(163.91, 65.77) --
	(163.91, 65.77) --
	(163.91, 65.77) --
	(163.99, 65.77) --
	(163.99, 65.77) --
	(163.99, 65.77) --
	(164.02, 65.77) --
	(164.02, 65.77) --
	(164.02, 65.77) --
	(164.06, 65.77) --
	(164.06, 65.77) --
	(164.06, 65.77) --
	(164.14, 65.78) --
	(164.14, 65.78) --
	(164.14, 65.78) --
	(164.18, 65.77) --
	(164.18, 65.77) --
	(164.18, 65.77) --
	(164.21, 65.77) --
	(164.21, 65.77) --
	(164.21, 65.77) --
	(164.22, 65.77) --
	(164.22, 65.77) --
	(164.22, 65.77) --
	(164.28, 65.77) --
	(164.28, 65.77) --
	(164.28, 65.77) --
	(164.34, 65.78) --
	(164.34, 65.78) --
	(164.34, 65.78) --
	(164.36, 65.78) --
	(164.36, 65.78) --
	(164.36, 65.78) --
	(164.42, 65.78) --
	(164.42, 65.78) --
	(164.42, 65.78) --
	(164.43, 65.78) --
	(164.43, 65.78) --
	(164.43, 65.78) --
	(164.51, 65.79) --
	(164.51, 65.79) --
	(164.51, 65.79) --
	(164.55, 65.79) --
	(164.55, 65.79) --
	(164.55, 65.79) --
	(164.58, 65.80) --
	(164.58, 65.80) --
	(164.58, 65.80) --
	(164.66, 65.80) --
	(164.66, 65.80) --
	(164.66, 65.80) --
	(164.66, 65.80) --
	(164.66, 65.80) --
	(164.66, 65.80) --
	(164.67, 65.80) --
	(164.67, 65.80) --
	(164.67, 65.80) --
	(164.71, 65.80) --
	(164.71, 65.80) --
	(164.71, 65.80) --
	(164.73, 65.80) --
	(164.73, 65.80) --
	(164.73, 65.80) --
	(164.81, 65.82) --
	(164.81, 65.82) --
	(164.81, 65.82) --
	(164.88, 65.82) --
	(164.88, 65.82) --
	(164.88, 65.82) --
	(164.95, 65.82) --
	(164.95, 65.82) --
	(164.95, 65.82) --
	(164.96, 65.83) --
	(164.96, 65.83) --
	(164.96, 65.83) --
	(164.99, 65.83) --
	(164.99, 65.83) --
	(164.99, 65.83) --
	(165.03, 65.84) --
	(165.03, 65.84) --
	(165.03, 65.84) --
	(165.07, 65.85) --
	(165.07, 65.85) --
	(165.07, 65.85) --
	(165.11, 65.85) --
	(165.11, 65.85) --
	(165.11, 65.85) --
	(165.18, 65.86) --
	(165.18, 65.86) --
	(165.18, 65.86) --
	(165.25, 65.87) --
	(165.25, 65.87) --
	(165.25, 65.87) --
	(165.27, 65.87) --
	(165.27, 65.87) --
	(165.27, 65.87) --
	(165.33, 65.89) --
	(165.33, 65.89) --
	(165.33, 65.89) --
	(165.41, 65.89) --
	(165.41, 65.89) --
	(165.41, 65.89) --
	(165.48, 65.90) --
	(165.48, 65.90) --
	(165.48, 65.90) --
	(165.55, 65.92) --
	(165.55, 65.92) --
	(165.55, 65.92) --
	(165.60, 65.93) --
	(165.60, 65.93) --
	(165.60, 65.93) --
	(165.63, 65.93) --
	(165.63, 65.93) --
	(165.63, 65.93) --
	(165.70, 65.95) --
	(165.70, 65.95) --
	(165.70, 65.95) --
	(165.76, 65.96) --
	(165.76, 65.96) --
	(165.76, 65.96) --
	(165.78, 65.96) --
	(165.78, 65.96) --
	(165.78, 65.96) --
	(165.85, 65.98) --
	(165.85, 65.98) --
	(165.85, 65.98) --
	(165.93, 66.00) --
	(165.93, 66.00) --
	(165.93, 66.00) --
	(166.00, 66.01) --
	(166.00, 66.01) --
	(166.00, 66.01) --
	(166.00, 66.01) --
	(166.00, 66.01) --
	(166.00, 66.01) --
	(166.08, 66.02) --
	(166.08, 66.02) --
	(166.08, 66.02) --
	(166.15, 66.05) --
	(166.15, 66.05) --
	(166.15, 66.05) --
	(166.16, 66.05) --
	(166.16, 66.05) --
	(166.16, 66.05) --
	(166.23, 66.07) --
	(166.23, 66.07) --
	(166.23, 66.07) --
	(166.30, 66.09) --
	(166.30, 66.09) --
	(166.30, 66.09) --
	(166.38, 66.11) --
	(166.38, 66.11) --
	(166.38, 66.11) --
	(166.44, 66.13) --
	(166.44, 66.13) --
	(166.44, 66.13) --
	(166.45, 66.13) --
	(166.45, 66.13) --
	(166.45, 66.13) --
	(166.48, 66.14) --
	(166.48, 66.14) --
	(166.48, 66.14) --
	(166.52, 66.15) --
	(166.52, 66.15) --
	(166.52, 66.15) --
	(166.60, 66.18) --
	(166.60, 66.18) --
	(166.60, 66.18) --
	(166.67, 66.19) --
	(166.67, 66.19) --
	(166.67, 66.19) --
	(166.75, 66.21) --
	(166.75, 66.21) --
	(166.75, 66.21) --
	(166.81, 66.24) --
	(166.81, 66.24) --
	(166.81, 66.24) --
	(166.82, 66.24) --
	(166.82, 66.24) --
	(166.82, 66.24) --
	(166.90, 66.26) --
	(166.90, 66.26) --
	(166.90, 66.26) --
	(166.97, 66.29) --
	(166.97, 66.29) --
	(166.97, 66.29) --
	(167.05, 66.32) --
	(167.05, 66.32) --
	(167.05, 66.32) --
	(167.12, 66.32) --
	(167.12, 66.32) --
	(167.12, 66.32) --
	(167.17, 66.34) --
	(167.17, 66.34) --
	(167.17, 66.34) --
	(167.20, 66.35) --
	(167.20, 66.35) --
	(167.20, 66.35) --
	(167.27, 66.36) --
	(167.27, 66.36) --
	(167.27, 66.36) --
	(167.34, 66.38) --
	(167.34, 66.38) --
	(167.34, 66.38) --
	(167.42, 66.41) --
	(167.42, 66.41) --
	(167.42, 66.41) --
	(167.49, 66.42) --
	(167.49, 66.42) --
	(167.49, 66.42) --
	(167.57, 66.44) --
	(167.57, 66.44) --
	(167.57, 66.44) --
	(167.64, 66.43) --
	(167.64, 66.43) --
	(167.64, 66.43) --
	(167.66, 66.44) --
	(167.66, 66.44) --
	(167.66, 66.44) --
	(167.72, 66.46) --
	(167.72, 66.46) --
	(167.72, 66.46) --
	(167.79, 66.47) --
	(167.79, 66.47) --
	(167.79, 66.47) --
	(167.83, 66.47) --
	(167.83, 66.47) --
	(167.83, 66.47) --
	(167.87, 66.48) --
	(167.87, 66.48) --
	(167.87, 66.48) --
	(167.94, 66.49) --
	(167.94, 66.49) --
	(167.94, 66.49) --
	(167.98, 66.51) --
	(167.98, 66.51) --
	(167.98, 66.51) --
	(168.01, 66.52) --
	(168.01, 66.52) --
	(168.01, 66.52) --
	(168.09, 66.53) --
	(168.09, 66.53) --
	(168.09, 66.53) --
	(168.16, 66.55) --
	(168.16, 66.55) --
	(168.16, 66.55) --
	(168.24, 66.54) --
	(168.24, 66.54) --
	(168.24, 66.54) --
	(168.30, 66.57) --
	(168.30, 66.57) --
	(168.30, 66.57) --
	(168.31, 66.57) --
	(168.31, 66.57) --
	(168.31, 66.57) --
	(168.34, 66.57) --
	(168.34, 66.57) --
	(168.34, 66.57) --
	(168.39, 66.57) --
	(168.39, 66.57) --
	(168.39, 66.57) --
	(168.46, 66.57) --
	(168.46, 66.57) --
	(168.46, 66.57) --
	(168.51, 66.59) --
	(168.51, 66.59) --
	(168.51, 66.59) --
	(168.54, 66.61) --
	(168.54, 66.61) --
	(168.54, 66.61) --
	(168.55, 66.61) --
	(168.55, 66.61) --
	(168.55, 66.61) --
	(168.61, 66.63) --
	(168.61, 66.63) --
	(168.61, 66.63) --
	(168.63, 66.63) --
	(168.63, 66.63) --
	(168.63, 66.63) --
	(168.67, 66.64) --
	(168.67, 66.64) --
	(168.67, 66.64) --
	(168.69, 66.64) --
	(168.69, 66.64) --
	(168.69, 66.64) --
	(168.71, 66.64) --
	(168.71, 66.64) --
	(168.71, 66.64) --
	(168.75, 66.65) --
	(168.75, 66.65) --
	(168.75, 66.65) --
	(168.76, 66.65) --
	(168.76, 66.65) --
	(168.76, 66.65) --
	(168.83, 66.65) --
	(168.83, 66.65) --
	(168.83, 66.65) --
	(168.87, 66.67) --
	(168.87, 66.67) --
	(168.87, 66.67) --
	(168.91, 66.69) --
	(168.91, 66.69) --
	(168.91, 66.69) --
	(168.98, 66.69) --
	(168.98, 66.69) --
	(168.98, 66.69) --
	(169.06, 66.71) --
	(169.06, 66.71) --
	(169.06, 66.71) --
	(169.13, 66.73) --
	(169.13, 66.73) --
	(169.13, 66.73) --
	(169.19, 66.73) --
	(169.19, 66.73) --
	(169.19, 66.73) --
	(169.21, 66.73) --
	(169.21, 66.73) --
	(169.21, 66.73) --
	(169.28, 66.73) --
	(169.28, 66.73) --
	(169.28, 66.73) --
	(169.36, 66.75) --
	(169.36, 66.75) --
	(169.36, 66.75) --
	(169.36, 66.75) --
	(169.36, 66.75) --
	(169.36, 66.75) --
	(169.43, 66.75) --
	(169.43, 66.75) --
	(169.43, 66.75) --
	(169.51, 66.78) --
	(169.51, 66.78) --
	(169.51, 66.78) --
	(169.58, 66.80) --
	(169.58, 66.80) --
	(169.58, 66.80) --
	(169.65, 66.85) --
	(169.65, 66.85) --
	(169.65, 66.85) --
	(169.73, 66.81) --
	(169.73, 66.81) --
	(169.73, 66.81) --
	(169.76, 66.82) --
	(169.76, 66.82) --
	(169.76, 66.82) --
	(169.80, 66.83) --
	(169.80, 66.83) --
	(169.80, 66.83) --
	(169.88, 66.83) --
	(169.88, 66.83) --
	(169.88, 66.83) --
	(169.95, 66.83) --
	(169.95, 66.83) --
	(169.95, 66.83) --
	(170.03, 66.83) --
	(170.03, 66.83) --
	(170.03, 66.83) --
	(170.10, 66.85) --
	(170.10, 66.85) --
	(170.10, 66.85) --
	(170.13, 66.86) --
	(170.13, 66.86) --
	(170.13, 66.86) --
	(170.17, 66.87) --
	(170.17, 66.87) --
	(170.17, 66.87) --
	(170.25, 66.88) --
	(170.25, 66.88) --
	(170.25, 66.88) --
	(170.32, 66.89) --
	(170.32, 66.89) --
	(170.32, 66.89) --
	(170.40, 66.91) --
	(170.40, 66.91) --
	(170.40, 66.91) --
	(170.47, 66.94) --
	(170.47, 66.94) --
	(170.47, 66.94) --
	(170.51, 66.95) --
	(170.51, 66.95) --
	(170.51, 66.95) --
	(170.55, 66.95) --
	(170.55, 66.95) --
	(170.55, 66.95) --
	(170.62, 66.97) --
	(170.62, 66.97) --
	(170.62, 66.97) --
	(170.70, 67.02) --
	(170.70, 67.02) --
	(170.70, 67.02) --
	(170.77, 67.04) --
	(170.77, 67.04) --
	(170.77, 67.04) --
	(170.84, 67.08) --
	(170.84, 67.08) --
	(170.84, 67.08) --
	(170.89, 67.06) --
	(170.89, 67.06) --
	(170.89, 67.06) --
	(170.92, 67.06) --
	(170.92, 67.06) --
	(170.92, 67.06) --
	(170.99, 67.09) --
	(170.99, 67.09) --
	(170.99, 67.09) --
	(171.05, 67.11) --
	(171.05, 67.11) --
	(171.05, 67.11) --
	(171.07, 67.11) --
	(171.07, 67.11) --
	(171.07, 67.11) --
	(171.14, 67.07) --
	(171.14, 67.07) --
	(171.14, 67.07) --
	(171.22, 67.08) --
	(171.22, 67.08) --
	(171.22, 67.08) --
	(171.29, 67.08) --
	(171.29, 67.08) --
	(171.29, 67.08) --
	(171.36, 67.12) --
	(171.36, 67.12) --
	(171.36, 67.12) --
	(171.37, 67.12) --
	(171.37, 67.12) --
	(171.37, 67.12) --
	(171.44, 67.15) --
	(171.44, 67.15) --
	(171.44, 67.15) --
	(171.51, 67.13) --
	(171.51, 67.13) --
	(171.51, 67.13) --
	(171.59, 67.19) --
	(171.59, 67.19) --
	(171.59, 67.19) --
	(171.66, 67.16) --
	(171.66, 67.16) --
	(171.66, 67.16) --
	(171.66, 67.16) --
	(171.66, 67.16) --
	(171.66, 67.16) --
	(171.74, 67.18) --
	(171.74, 67.18) --
	(171.74, 67.18) --
	(171.81, 67.22) --
	(171.81, 67.22) --
	(171.81, 67.22) --
	(171.85, 67.25) --
	(171.85, 67.25) --
	(171.85, 67.25) --
	(171.88, 67.28) --
	(171.88, 67.28) --
	(171.88, 67.28) --
	(171.96, 67.29) --
	(171.96, 67.29) --
	(171.96, 67.29) --
	(172.03, 67.28) --
	(172.03, 67.28) --
	(172.03, 67.28) --
	(172.11, 67.34) --
	(172.11, 67.34) --
	(172.11, 67.34) --
	(172.14, 67.34) --
	(172.14, 67.34) --
	(172.14, 67.34) --
	(172.18, 67.33) --
	(172.18, 67.33) --
	(172.18, 67.33) --
	(172.26, 67.33) --
	(172.26, 67.33) --
	(172.26, 67.33) --
	(172.33, 67.36) --
	(172.33, 67.36) --
	(172.33, 67.36) --
	(172.33, 67.36) --
	(172.33, 67.36) --
	(172.33, 67.36) --
	(172.40, 67.40) --
	(172.40, 67.40) --
	(172.40, 67.40) --
	(172.43, 67.41) --
	(172.43, 67.41) --
	(172.43, 67.41) --
	(172.48, 67.42) --
	(172.48, 67.42) --
	(172.48, 67.42) --
	(172.55, 67.42) --
	(172.55, 67.42) --
	(172.55, 67.42) --
	(172.63, 67.49) --
	(172.63, 67.49) --
	(172.63, 67.49) --
	(172.70, 67.42) --
	(172.70, 67.42) --
	(172.70, 67.42) --
	(172.78, 67.46) --
	(172.78, 67.46) --
	(172.78, 67.46) --
	(172.81, 67.45) --
	(172.81, 67.45) --
	(172.81, 67.45) --
	(172.85, 67.44) --
	(172.85, 67.44) --
	(172.85, 67.44) --
	(172.92, 67.53) --
	(172.92, 67.53) --
	(172.92, 67.53) --
	(173.00, 67.51) --
	(173.00, 67.51) --
	(173.00, 67.51) --
	(173.00, 67.51) --
	(173.00, 67.51) --
	(173.00, 67.51) --
	(173.07, 67.51) --
	(173.07, 67.51) --
	(173.07, 67.51) --
	(173.15, 67.53) --
	(173.15, 67.53) --
	(173.15, 67.53) --
	(173.22, 67.57) --
	(173.22, 67.57) --
	(173.22, 67.57) --
	(173.29, 67.59) --
	(173.29, 67.59) --
	(173.29, 67.59) --
	(173.29, 67.59) --
	(173.29, 67.59) --
	(173.29, 67.59) --
	(173.37, 67.60) --
	(173.37, 67.60) --
	(173.37, 67.60) --
	(173.44, 67.63) --
	(173.44, 67.63) --
	(173.44, 67.63) --
	(173.48, 67.60) --
	(173.48, 67.60) --
	(173.48, 67.60) --
	(173.52, 67.57) --
	(173.52, 67.57) --
	(173.52, 67.57) --
	(173.58, 67.60) --
	(173.58, 67.60) --
	(173.58, 67.60) --
	(173.59, 67.60) --
	(173.59, 67.60) --
	(173.59, 67.60) --
	(173.67, 67.69) --
	(173.67, 67.69) --
	(173.67, 67.69) --
	(173.74, 67.68) --
	(173.74, 67.68) --
	(173.74, 67.68) --
	(173.77, 67.69) --
	(173.77, 67.69) --
	(173.77, 67.69) --
	(173.81, 67.70) --
	(173.81, 67.70) --
	(173.81, 67.70) --
	(173.86, 67.77) --
	(173.86, 67.77) --
	(173.86, 67.77) --
	(173.89, 67.80) --
	(173.89, 67.80) --
	(173.89, 67.80) --
	(173.96, 67.69) --
	(173.96, 67.69) --
	(173.96, 67.69) --
	(174.04, 67.67) --
	(174.04, 67.67) --
	(174.04, 67.67) --
	(174.11, 67.67) --
	(174.11, 67.67) --
	(174.11, 67.67) --
	(174.15, 67.68) --
	(174.15, 67.68) --
	(174.15, 67.68) --
	(174.18, 67.70) --
	(174.18, 67.70) --
	(174.18, 67.70) --
	(174.25, 67.69) --
	(174.25, 67.69) --
	(174.25, 67.69) --
	(174.26, 67.69) --
	(174.26, 67.69) --
	(174.26, 67.69) --
	(174.33, 67.72) --
	(174.33, 67.72) --
	(174.33, 67.72) --
	(174.34, 67.72) --
	(174.34, 67.72) --
	(174.34, 67.72) --
	(174.41, 67.71) --
	(174.41, 67.71) --
	(174.41, 67.71) --
	(174.44, 67.72) --
	(174.44, 67.72) --
	(174.44, 67.72) --
	(174.48, 67.74) --
	(174.48, 67.74) --
	(174.48, 67.74) --
	(174.53, 67.77) --
	(174.53, 67.77) --
	(174.53, 67.77) --
	(174.56, 67.78) --
	(174.56, 67.78) --
	(174.56, 67.78) --
	(174.63, 67.84) --
	(174.63, 67.84) --
	(174.63, 67.84) --
	(174.63, 67.84) --
	(174.63, 67.84) --
	(174.63, 67.84) --
	(174.70, 67.79) --
	(174.70, 67.79) --
	(174.70, 67.79) --
	(174.78, 67.79) --
	(174.78, 67.79) --
	(174.78, 67.79) --
	(174.82, 67.82) --
	(174.82, 67.82) --
	(174.82, 67.82) --
	(174.85, 67.83) --
	(174.85, 67.83) --
	(174.85, 67.83) --
	(174.92, 67.86) --
	(174.92, 67.86) --
	(174.92, 67.86) --
	(174.93, 67.86) --
	(174.93, 67.86) --
	(174.93, 67.86) --
	(175.00, 67.84) --
	(175.00, 67.84) --
	(175.00, 67.84) --
	(175.01, 67.86) --
	(175.01, 67.86) --
	(175.01, 67.86) --
	(175.07, 67.92) --
	(175.07, 67.92) --
	(175.07, 67.92) --
	(175.15, 67.91) --
	(175.15, 67.91) --
	(175.15, 67.91) --
	(175.22, 67.93) --
	(175.22, 67.93) --
	(175.22, 67.93) --
	(175.30, 67.95) --
	(175.30, 67.95) --
	(175.30, 67.95) --
	(175.30, 67.96) --
	(175.30, 67.96) --
	(175.30, 67.96) --
	(175.37, 68.06) --
	(175.37, 68.06) --
	(175.37, 68.06) --
	(175.45, 68.08) --
	(175.45, 68.08) --
	(175.45, 68.08) --
	(175.52, 68.15) --
	(175.52, 68.15) --
	(175.52, 68.15) --
	(175.59, 68.34) --
	(175.59, 68.34) --
	(175.59, 68.34) --
	(175.59, 68.35) --
	(175.59, 68.35) --
	(175.59, 68.35) --
	(175.67, 68.50) --
	(175.67, 68.50) --
	(175.67, 68.50) --
	(175.68, 68.55) --
	(175.68, 68.55) --
	(175.68, 68.55) --
	(175.74, 68.71) --
	(175.74, 68.71) --
	(175.74, 68.71) --
	(175.78, 68.70) --
	(175.78, 68.70) --
	(175.78, 68.70) --
	(175.82, 68.68) --
	(175.82, 68.68) --
	(175.82, 68.68) --
	(175.89, 68.50) --
	(175.89, 68.50) --
	(175.89, 68.50) --
	(175.96, 68.41) --
	(175.96, 68.41) --
	(175.96, 68.41) --
	(175.97, 68.43) --
	(175.97, 68.43) --
	(175.97, 68.43) --
	(176.04, 68.52) --
	(176.04, 68.52) --
	(176.04, 68.52) --
	(176.07, 68.55) --
	(176.07, 68.55) --
	(176.07, 68.55) --
	(176.11, 68.60) --
	(176.11, 68.60) --
	(176.11, 68.60) --
	(176.16, 68.62) --
	(176.16, 68.62) --
	(176.16, 68.62) --
	(176.19, 68.63) --
	(176.19, 68.63) --
	(176.19, 68.63) --
	(176.26, 68.68) --
	(176.26, 68.68) --
	(176.26, 68.68) --
	(176.33, 68.50) --
	(176.33, 68.50) --
	(176.33, 68.50) --
	(176.36, 68.51) --
	(176.36, 68.51) --
	(176.36, 68.51) --
	(176.41, 68.55) --
	(176.41, 68.55) --
	(176.41, 68.55) --
	(176.45, 68.56) --
	(176.45, 68.56) --
	(176.45, 68.56) --
	(176.48, 68.56) --
	(176.48, 68.56) --
	(176.48, 68.56) --
	(176.56, 68.66) --
	(176.56, 68.66) --
	(176.56, 68.66) --
	(176.63, 68.74) --
	(176.63, 68.74) --
	(176.63, 68.74) --
	(176.64, 68.77) --
	(176.64, 68.77) --
	(176.64, 68.77) --
	(176.70, 68.90) --
	(176.70, 68.90) --
	(176.70, 68.90) --
	(176.74, 68.89) --
	(176.74, 68.89) --
	(176.74, 68.89) --
	(176.78, 68.88) --
	(176.78, 68.88) --
	(176.78, 68.88) --
	(176.83, 68.93) --
	(176.83, 68.93) --
	(176.83, 68.93) --
	(176.85, 68.95) --
	(176.85, 68.95) --
	(176.85, 68.95) --
	(176.93, 69.07) --
	(176.93, 69.07) --
	(176.93, 69.07) --
	(177.00, 69.24) --
	(177.00, 69.24) --
	(177.00, 69.24) --
	(177.03, 69.25) --
	(177.03, 69.25) --
	(177.03, 69.25) --
	(177.07, 69.28) --
	(177.07, 69.28) --
	(177.07, 69.28) --
	(177.12, 69.20) --
	(177.12, 69.20) --
	(177.12, 69.20) --
	(177.15, 69.16) --
	(177.15, 69.16) --
	(177.15, 69.16) --
	(177.22, 69.20) --
	(177.22, 69.20) --
	(177.22, 69.20) --
	(177.30, 69.24) --
	(177.30, 69.24) --
	(177.30, 69.24) --
	(177.31, 69.27) --
	(177.31, 69.27) --
	(177.31, 69.27) --
	(177.37, 69.38) --
	(177.37, 69.38) --
	(177.37, 69.38) --
	(177.44, 69.59) --
	(177.44, 69.59) --
	(177.44, 69.59) --
	(177.51, 69.53) --
	(177.51, 69.53) --
	(177.51, 69.53) --
	(177.52, 69.51) --
	(177.52, 69.51) --
	(177.52, 69.51) --
	(177.59, 69.68) --
	(177.59, 69.68) --
	(177.59, 69.68) --
	(177.60, 69.70) --
	(177.60, 69.70) --
	(177.60, 69.70) --
	(177.67, 69.86) --
	(177.67, 69.86) --
	(177.67, 69.86) --
	(177.74, 70.15) --
	(177.74, 70.15) --
	(177.74, 70.15) --
	(177.79, 70.22) --
	(177.79, 70.22) --
	(177.79, 70.22) --
	(177.81, 70.24) --
	(177.81, 70.24) --
	(177.81, 70.24) --
	(177.89, 70.27) --
	(177.89, 70.27) --
	(177.89, 70.27) --
	(177.89, 70.27) --
	(177.89, 70.27) --
	(177.89, 70.27) --
	(177.96, 70.19) --
	(177.96, 70.19) --
	(177.96, 70.19) --
	(178.04, 69.94) --
	(178.04, 69.94) --
	(178.04, 69.94) --
	(178.11, 70.13) --
	(178.11, 70.13) --
	(178.11, 70.13) --
	(178.18, 70.14) --
	(178.18, 70.14) --
	(178.18, 70.14) --
	(178.18, 70.15) --
	(178.18, 70.15) --
	(178.18, 70.15) --
	(178.26, 70.31) --
	(178.26, 70.31) --
	(178.26, 70.31) --
	(178.33, 70.54) --
	(178.33, 70.54) --
	(178.33, 70.54) --
	(178.37, 70.76) --
	(178.37, 70.76) --
	(178.37, 70.76) --
	(178.41, 71.00) --
	(178.41, 71.00) --
	(178.41, 71.00) --
	(178.48, 71.66) --
	(178.48, 71.66) --
	(178.48, 71.66) --
	(178.55, 72.00) --
	(178.55, 72.00) --
	(178.55, 72.00) --
	(178.56, 71.99) --
	(178.56, 71.99) --
	(178.56, 71.99) --
	(178.63, 71.85) --
	(178.63, 71.85) --
	(178.63, 71.85) --
	(178.70, 71.60) --
	(178.70, 71.60) --
	(178.70, 71.60) --
	(178.75, 71.57) --
	(178.75, 71.57) --
	(178.75, 71.57) --
	(178.78, 71.56) --
	(178.78, 71.56) --
	(178.78, 71.56) --
	(178.85, 71.30) --
	(178.85, 71.30) --
	(178.85, 71.30) --
	(178.92, 70.99) --
	(178.92, 70.99) --
	(178.92, 70.99) --
	(178.94, 70.98) --
	(178.94, 70.98) --
	(178.94, 70.98) --
	(179.00, 70.95) --
	(179.00, 70.95) --
	(179.00, 70.95) --
	(179.07, 70.99) --
	(179.07, 70.99) --
	(179.07, 70.99) --
	(179.14, 70.81) --
	(179.14, 70.81) --
	(179.14, 70.81) --
	(179.22, 70.85) --
	(179.22, 70.85) --
	(179.22, 70.85) --
	(179.29, 70.77) --
	(179.29, 70.77) --
	(179.29, 70.77) --
	(179.33, 70.92) --
	(179.33, 70.92) --
	(179.33, 70.92) --
	(179.37, 71.11) --
	(179.37, 71.11) --
	(179.37, 71.11) --
	(179.44, 71.43) --
	(179.44, 71.43) --
	(179.44, 71.43) --
	(179.51, 71.76) --
	(179.51, 71.76) --
	(179.51, 71.76) --
	(179.59, 71.51) --
	(179.59, 71.51) --
	(179.59, 71.51) --
	(179.66, 71.11) --
	(179.66, 71.11) --
	(179.66, 71.11) --
	(179.71, 70.94) --
	(179.71, 70.94) --
	(179.71, 70.94) --
	(179.73, 70.84) --
	(179.73, 70.84) --
	(179.73, 70.84) --
	(179.81, 70.74) --
	(179.81, 70.74) --
	(179.81, 70.74) --
	(179.81, 70.73) --
	(179.81, 70.73) --
	(179.81, 70.73) --
	(179.88, 70.79) --
	(179.88, 70.79) --
	(179.88, 70.79) --
	(179.96, 70.68) --
	(179.96, 70.68) --
	(179.96, 70.68) --
	(180.03, 70.70) --
	(180.03, 70.70) --
	(180.03, 70.70) --
	(180.10, 70.75) --
	(180.10, 70.75) --
	(180.10, 70.75) --
	(180.18, 70.75) --
	(180.18, 70.75) --
	(180.18, 70.75) --
	(180.19, 70.78) --
	(180.19, 70.78) --
	(180.19, 70.78) --
	(180.25, 70.99) --
	(180.25, 70.99) --
	(180.25, 70.99) --
	(180.33, 71.03) --
	(180.33, 71.03) --
	(180.33, 71.03) --
	(180.40, 71.16) --
	(180.40, 71.16) --
	(180.40, 71.16) --
	(180.47, 71.22) --
	(180.47, 71.22) --
	(180.47, 71.22) --
	(180.55, 71.48) --
	(180.55, 71.48) --
	(180.55, 71.48) --
	(180.57, 71.56) --
	(180.57, 71.56) --
	(180.57, 71.56) --
	(180.62, 71.71) --
	(180.62, 71.71) --
	(180.62, 71.71) --
	(180.70, 72.11) --
	(180.70, 72.11) --
	(180.70, 72.11) --
	(180.77, 72.27) --
	(180.77, 72.27) --
	(180.77, 72.27) --
	(180.84, 72.20) --
	(180.84, 72.20) --
	(180.84, 72.20) --
	(180.92, 72.13) --
	(180.92, 72.13) --
	(180.92, 72.13) --
	(180.99, 72.14) --
	(180.99, 72.14) --
	(180.99, 72.14) --
	(181.05, 72.26) --
	(181.05, 72.26) --
	(181.05, 72.26) --
	(181.06, 72.29) --
	(181.06, 72.29) --
	(181.06, 72.29) --
	(181.14, 72.20) --
	(181.14, 72.20) --
	(181.14, 72.20) --
	(181.21, 72.14) --
	(181.21, 72.14) --
	(181.21, 72.14) --
	(181.28, 72.07) --
	(181.28, 72.07) --
	(181.28, 72.07) --
	(181.36, 71.95) --
	(181.36, 71.95) --
	(181.36, 71.95) --
	(181.43, 71.95) --
	(181.43, 71.95) --
	(181.43, 71.95) --
	(181.51, 71.83) --
	(181.51, 71.83) --
	(181.51, 71.83) --
	(181.53, 71.82) --
	(181.53, 71.82) --
	(181.53, 71.82) --
	(181.58, 71.79) --
	(181.58, 71.79) --
	(181.58, 71.79) --
	(181.65, 71.79) --
	(181.65, 71.79) --
	(181.65, 71.79) --
	(181.73, 71.75) --
	(181.73, 71.75) --
	(181.73, 71.75) --
	(181.80, 71.77) --
	(181.80, 71.77) --
	(181.80, 71.77) --
	(181.87, 71.82) --
	(181.87, 71.82) --
	(181.87, 71.82) --
	(181.95, 71.91) --
	(181.95, 71.91) --
	(181.95, 71.91) --
	(182.01, 71.95) --
	(182.01, 71.95) --
	(182.01, 71.95) --
	(182.02, 71.95) --
	(182.02, 71.95) --
	(182.02, 71.95) --
	(182.10, 71.98) --
	(182.10, 71.98) --
	(182.10, 71.98) --
	(182.17, 72.11) --
	(182.17, 72.11) --
	(182.17, 72.11) --
	(182.24, 72.21) --
	(182.24, 72.21) --
	(182.24, 72.21) --
	(182.32, 72.37) --
	(182.32, 72.37) --
	(182.32, 72.37) --
	(182.39, 72.63) --
	(182.39, 72.63) --
	(182.39, 72.63) --
	(182.46, 73.08) --
	(182.46, 73.08) --
	(182.46, 73.08) --
	(182.54, 73.22) --
	(182.54, 73.22) --
	(182.54, 73.22) --
	(182.58, 73.18) --
	(182.58, 73.18) --
	(182.58, 73.18) --
	(182.61, 73.15) --
	(182.61, 73.15) --
	(182.61, 73.15) --
	(182.68, 73.04) --
	(182.69, 73.04) --
	(182.69, 73.04) --
	(182.76, 72.78) --
	(182.76, 72.78) --
	(182.76, 72.78) --
	(182.83, 72.59) --
	(182.83, 72.59) --
	(182.83, 72.59) --
	(182.91, 72.43) --
	(182.91, 72.43) --
	(182.91, 72.43) --
	(182.98, 72.32) --
	(182.98, 72.32) --
	(182.98, 72.32) --
	(183.05, 72.30) --
	(183.05, 72.30) --
	(183.05, 72.30) --
	(183.13, 72.18) --
	(183.13, 72.18) --
	(183.13, 72.18) --
	(183.20, 72.21) --
	(183.20, 72.21) --
	(183.20, 72.21) --
	(183.27, 72.14) --
	(183.27, 72.14) --
	(183.27, 72.14) --
	(183.35, 72.12) --
	(183.35, 72.12) --
	(183.35, 72.12) --
	(183.35, 72.12) --
	(183.35, 72.12) --
	(183.35, 72.12) --
	(183.42, 72.11) --
	(183.42, 72.11) --
	(183.42, 72.11) --
	(183.50, 72.10) --
	(183.50, 72.10) --
	(183.50, 72.10) --
	(183.57, 72.07) --
	(183.57, 72.07) --
	(183.57, 72.07) --
	(183.64, 72.11) --
	(183.64, 72.11) --
	(183.64, 72.11) --
	(183.72, 72.06) --
	(183.72, 72.06) --
	(183.72, 72.06) --
	(183.79, 72.10) --
	(183.79, 72.10) --
	(183.79, 72.10) --
	(183.86, 72.07) --
	(183.86, 72.07) --
	(183.86, 72.07) --
	(183.94, 72.07) --
	(183.94, 72.07) --
	(183.94, 72.07) --
	(184.01, 72.08) --
	(184.01, 72.08) --
	(184.01, 72.08) --
	(184.09, 72.06) --
	(184.09, 72.06) --
	(184.09, 72.06) --
	(184.12, 72.07) --
	(184.12, 72.07) --
	(184.12, 72.07) --
	(184.16, 72.09) --
	(184.16, 72.09) --
	(184.16, 72.09) --
	(184.23, 72.13) --
	(184.23, 72.13) --
	(184.23, 72.13) --
	(184.30, 72.14) --
	(184.30, 72.14) --
	(184.30, 72.14) --
	(184.38, 72.16) --
	(184.38, 72.16) --
	(184.38, 72.16) --
	(184.41, 72.13) --
	(184.41, 72.13) --
	(184.41, 72.13) --
	(184.45, 72.08) --
	(184.45, 72.08) --
	(184.45, 72.08) --
	(184.53, 72.08) --
	(184.53, 72.08) --
	(184.53, 72.08) --
	(184.60, 72.06) --
	(184.60, 72.06) --
	(184.60, 72.06) --
	(184.67, 72.04) --
	(184.67, 72.04) --
	(184.67, 72.04) --
	(184.75, 72.05) --
	(184.75, 72.05) --
	(184.75, 72.05) --
	(184.82, 72.05) --
	(184.82, 72.05) --
	(184.82, 72.05) --
	(184.89, 72.06) --
	(184.89, 72.06) --
	(184.89, 72.06) --
	(184.97, 72.01) --
	(184.97, 72.01) --
	(184.97, 72.01) --
	(184.98, 72.01) --
	(184.98, 72.01) --
	(184.98, 72.01) --
	(185.04, 72.03) --
	(185.04, 72.03) --
	(185.04, 72.03) --
	(185.11, 72.09) --
	(185.11, 72.09) --
	(185.11, 72.09) --
	(185.19, 72.05) --
	(185.19, 72.05) --
	(185.19, 72.05) --
	(185.26, 72.05) --
	(185.26, 72.05) --
	(185.26, 72.05) --
	(185.27, 72.05) --
	(185.27, 72.05) --
	(185.27, 72.05) --
	(185.33, 72.08) --
	(185.33, 72.08) --
	(185.33, 72.08) --
	(185.41, 72.09) --
	(185.41, 72.09) --
	(185.41, 72.09) --
	(185.48, 72.08) --
	(185.48, 72.08) --
	(185.48, 72.08) --
	(185.55, 72.06) --
	(185.55, 72.06) --
	(185.55, 72.06) --
	(185.63, 72.10) --
	(185.63, 72.10) --
	(185.63, 72.10) --
	(185.70, 72.17) --
	(185.70, 72.17) --
	(185.70, 72.17) --
	(185.75, 72.18) --
	(185.75, 72.18) --
	(185.75, 72.18) --
	(185.78, 72.19) --
	(185.78, 72.19) --
	(185.78, 72.19) --
	(185.85, 72.20) --
	(185.85, 72.20) --
	(185.85, 72.20) --
	(185.92, 72.25) --
	(185.92, 72.25) --
	(185.92, 72.25) --
	(186.00, 72.27) --
	(186.00, 72.27) --
	(186.00, 72.27) --
	(186.03, 72.25) --
	(186.03, 72.25) --
	(186.03, 72.25) --
	(186.07, 72.23) --
	(186.07, 72.23) --
	(186.07, 72.23) --
	(186.14, 72.25) --
	(186.14, 72.25) --
	(186.14, 72.25) --
	(186.22, 72.28) --
	(186.22, 72.28) --
	(186.22, 72.28) --
	(186.29, 72.30) --
	(186.29, 72.30) --
	(186.29, 72.30) --
	(186.36, 72.32) --
	(186.36, 72.32) --
	(186.36, 72.32) --
	(186.44, 72.32) --
	(186.44, 72.32) --
	(186.44, 72.32) --
	(186.51, 72.34) --
	(186.51, 72.34) --
	(186.51, 72.34) --
	(186.58, 72.35) --
	(186.58, 72.35) --
	(186.58, 72.35) --
	(186.66, 72.39) --
	(186.66, 72.39) --
	(186.66, 72.39) --
	(186.71, 72.41) --
	(186.71, 72.41) --
	(186.71, 72.41) --
	(186.73, 72.42) --
	(186.73, 72.42) --
	(186.73, 72.42) --
	(186.80, 72.49) --
	(186.80, 72.49) --
	(186.80, 72.49) --
	(186.88, 72.48) --
	(186.88, 72.48) --
	(186.88, 72.48) --
	(186.95, 72.46) --
	(186.95, 72.46) --
	(186.95, 72.46) --
	(187.02, 72.49) --
	(187.02, 72.49) --
	(187.02, 72.49) --
	(187.10, 72.57) --
	(187.10, 72.57) --
	(187.10, 72.57) --
	(187.17, 72.55) --
	(187.17, 72.55) --
	(187.17, 72.55) --
	(187.18, 72.56) --
	(187.18, 72.56) --
	(187.18, 72.56) --
	(187.25, 72.58) --
	(187.25, 72.58) --
	(187.25, 72.58) --
	(187.32, 72.58) --
	(187.32, 72.58) --
	(187.32, 72.58) --
	(187.39, 72.62) --
	(187.39, 72.62) --
	(187.39, 72.62) --
	(187.46, 72.64) --
	(187.46, 72.64) --
	(187.46, 72.64) --
	(187.54, 72.73) --
	(187.54, 72.73) --
	(187.54, 72.73) --
	(187.61, 72.75) --
	(187.61, 72.75) --
	(187.61, 72.75) --
	(187.68, 72.77) --
	(187.68, 72.77) --
	(187.68, 72.77) --
	(187.76, 72.76) --
	(187.76, 72.76) --
	(187.76, 72.76) --
	(187.76, 72.76) --
	(187.76, 72.76) --
	(187.76, 72.76) --
	(187.83, 72.79) --
	(187.83, 72.79) --
	(187.83, 72.79) --
	(187.91, 72.78) --
	(187.91, 72.78) --
	(187.91, 72.78) --
	(187.98, 72.86) --
	(187.98, 72.86) --
	(187.98, 72.86) --
	(188.05, 72.88) --
	(188.05, 72.88) --
	(188.05, 72.88) --
	(188.13, 72.91) --
	(188.13, 72.91) --
	(188.13, 72.91) --
	(188.20, 72.93) --
	(188.20, 72.93) --
	(188.20, 72.93) --
	(188.27, 72.96) --
	(188.27, 72.96) --
	(188.27, 72.96) --
	(188.33, 73.02) --
	(188.33, 73.02) --
	(188.33, 73.02) --
	(188.34, 73.02) --
	(188.34, 73.02) --
	(188.34, 73.02) --
	(188.42, 73.02) --
	(188.42, 73.02) --
	(188.42, 73.02) --
	(188.49, 73.09) --
	(188.49, 73.09) --
	(188.49, 73.09) --
	(188.56, 73.14) --
	(188.56, 73.14) --
	(188.56, 73.14) --
	(188.64, 73.13) --
	(188.64, 73.13) --
	(188.64, 73.13) --
	(188.71, 73.12) --
	(188.71, 73.12) --
	(188.71, 73.12) --
	(188.79, 73.15) --
	(188.79, 73.15) --
	(188.79, 73.15) --
	(188.86, 73.15) --
	(188.86, 73.15) --
	(188.86, 73.15) --
	(188.93, 73.16) --
	(188.93, 73.16) --
	(188.93, 73.16) --
	(189.01, 73.19) --
	(189.01, 73.19) --
	(189.01, 73.19) --
	(189.08, 73.21) --
	(189.08, 73.21) --
	(189.08, 73.21) --
	(189.15, 73.29) --
	(189.15, 73.29) --
	(189.15, 73.29) --
	(189.22, 73.25) --
	(189.22, 73.25) --
	(189.22, 73.25) --
	(189.29, 73.35) --
	(189.29, 73.35) --
	(189.29, 73.35) --
	(189.30, 73.36) --
	(189.30, 73.36) --
	(189.30, 73.36) --
	(189.37, 73.37) --
	(189.37, 73.37) --
	(189.37, 73.37) --
	(189.44, 73.35) --
	(189.44, 73.35) --
	(189.44, 73.35) --
	(189.52, 73.39) --
	(189.52, 73.39) --
	(189.52, 73.39) --
	(189.59, 73.38) --
	(189.59, 73.38) --
	(189.59, 73.38) --
	(189.66, 73.45) --
	(189.66, 73.45) --
	(189.66, 73.45) --
	(189.74, 73.41) --
	(189.74, 73.41) --
	(189.74, 73.41) --
	(189.81, 73.45) --
	(189.81, 73.45) --
	(189.81, 73.45) --
	(189.88, 73.48) --
	(189.88, 73.48) --
	(189.88, 73.48) --
	(189.96, 73.46) --
	(189.96, 73.46) --
	(189.96, 73.46) --
	(190.03, 73.47) --
	(190.03, 73.47) --
	(190.03, 73.47) --
	(190.10, 73.49) --
	(190.10, 73.49) --
	(190.10, 73.49) --
	(190.18, 73.49) --
	(190.18, 73.49) --
	(190.18, 73.49) --
	(190.25, 73.48) --
	(190.25, 73.48) --
	(190.25, 73.48) --
	(190.25, 73.48) --
	(190.25, 73.48) --
	(190.25, 73.48) --
	(190.32, 73.56) --
	(190.32, 73.56) --
	(190.32, 73.56) --
	(190.40, 73.51) --
	(190.40, 73.51) --
	(190.40, 73.51) --
	(190.47, 73.57) --
	(190.47, 73.57) --
	(190.47, 73.57) --
	(190.54, 73.53) --
	(190.54, 73.53) --
	(190.54, 73.53) --
	(190.62, 73.59) --
	(190.62, 73.59) --
	(190.62, 73.59) --
	(190.69, 73.53) --
	(190.69, 73.53) --
	(190.69, 73.53) --
	(190.76, 73.54) --
	(190.76, 73.54) --
	(190.76, 73.54) --
	(190.84, 73.54) --
	(190.84, 73.54) --
	(190.84, 73.54) --
	(190.91, 73.52) --
	(190.91, 73.52) --
	(190.91, 73.52) --
	(190.98, 73.51) --
	(190.98, 73.51) --
	(190.98, 73.51) --
	(191.06, 73.52) --
	(191.06, 73.52) --
	(191.06, 73.52) --
	(191.13, 73.46) --
	(191.13, 73.46) --
	(191.13, 73.46) --
	(191.20, 73.50) --
	(191.20, 73.50) --
	(191.20, 73.50) --
	(191.27, 73.48) --
	(191.27, 73.48) --
	(191.27, 73.48) --
	(191.35, 73.45) --
	(191.35, 73.45) --
	(191.35, 73.45) --
	(191.42, 73.49) --
	(191.42, 73.49) --
	(191.42, 73.49) --
	(191.49, 73.47) --
	(191.49, 73.47) --
	(191.49, 73.47) --
	(191.57, 73.46) --
	(191.57, 73.46) --
	(191.57, 73.46) --
	(191.64, 73.41) --
	(191.64, 73.41) --
	(191.64, 73.41) --
	(191.71, 73.43) --
	(191.71, 73.43) --
	(191.71, 73.43) --
	(191.79, 73.38) --
	(191.79, 73.38) --
	(191.79, 73.38) --
	(191.86, 73.44) --
	(191.86, 73.44) --
	(191.86, 73.44) --
	(191.93, 73.42) --
	(191.93, 73.42) --
	(191.93, 73.42) --
	(192.01, 73.38) --
	(192.01, 73.38) --
	(192.01, 73.38) --
	(192.07, 73.38) --
	(192.07, 73.38) --
	(192.07, 73.38) --
	(192.08, 73.38) --
	(192.08, 73.38) --
	(192.08, 73.38) --
	(192.15, 73.33) --
	(192.15, 73.33) --
	(192.15, 73.33) --
	(192.23, 73.33) --
	(192.23, 73.33) --
	(192.23, 73.33) --
	(192.30, 73.35) --
	(192.30, 73.35) --
	(192.30, 73.35) --
	(192.37, 73.32) --
	(192.37, 73.32) --
	(192.37, 73.32) --
	(192.44, 73.31) --
	(192.44, 73.31) --
	(192.44, 73.31) --
	(192.52, 73.27) --
	(192.52, 73.27) --
	(192.52, 73.27) --
	(192.59, 73.21) --
	(192.59, 73.21) --
	(192.59, 73.21) --
	(192.66, 73.25) --
	(192.66, 73.25) --
	(192.66, 73.25) --
	(192.74, 73.22) --
	(192.74, 73.22) --
	(192.74, 73.22) --
	(192.81, 73.19) --
	(192.81, 73.19) --
	(192.81, 73.19) --
	(192.88, 73.15) --
	(192.88, 73.15) --
	(192.88, 73.15) --
	(192.96, 73.12) --
	(192.96, 73.12) --
	(192.96, 73.12) --
	(193.03, 73.09) --
	(193.03, 73.09) --
	(193.03, 73.09) --
	(193.10, 73.08) --
	(193.10, 73.08) --
	(193.10, 73.08) --
	(193.17, 73.08) --
	(193.17, 73.08) --
	(193.17, 73.08) --
	(193.25, 73.04) --
	(193.25, 73.04) --
	(193.25, 73.04) --
	(193.32, 73.03) --
	(193.32, 73.03) --
	(193.32, 73.03) --
	(193.39, 73.01) --
	(193.39, 73.01) --
	(193.39, 73.01) --
	(193.41, 72.99) --
	(193.41, 72.99) --
	(193.41, 72.99) --
	(193.47, 72.94) --
	(193.47, 72.94) --
	(193.47, 72.94) --
	(193.54, 72.89) --
	(193.54, 72.89) --
	(193.54, 72.89) --
	(193.61, 72.87) --
	(193.61, 72.87) --
	(193.61, 72.87) --
	(193.69, 72.92) --
	(193.69, 72.92) --
	(193.69, 72.92) --
	(193.76, 72.88) --
	(193.76, 72.88) --
	(193.76, 72.88) --
	(193.83, 72.86) --
	(193.83, 72.86) --
	(193.83, 72.86) --
	(193.91, 72.78) --
	(193.91, 72.78) --
	(193.91, 72.78) --
	(193.98, 72.78) --
	(193.98, 72.78) --
	(193.98, 72.78) --
	(194.05, 72.73) --
	(194.05, 72.73) --
	(194.05, 72.73) --
	(194.12, 72.70) --
	(194.12, 72.70) --
	(194.12, 72.70) --
	(194.20, 72.63) --
	(194.20, 72.63) --
	(194.20, 72.63) --
	(194.27, 72.70) --
	(194.27, 72.70) --
	(194.27, 72.70) --
	(194.34, 72.59) --
	(194.34, 72.59) --
	(194.34, 72.59) --
	(194.42, 72.61) --
	(194.42, 72.61) --
	(194.42, 72.61) --
	(194.49, 72.57) --
	(194.49, 72.57) --
	(194.49, 72.57) --
	(194.56, 72.57) --
	(194.56, 72.57) --
	(194.56, 72.57) --
	(194.63, 72.55) --
	(194.63, 72.55) --
	(194.63, 72.55) --
	(194.71, 72.48) --
	(194.71, 72.48) --
	(194.71, 72.48) --
	(194.78, 72.47) --
	(194.78, 72.47) --
	(194.78, 72.47) --
	(194.85, 72.45) --
	(194.85, 72.45) --
	(194.85, 72.45) --
	(194.93, 72.47) --
	(194.93, 72.47) --
	(194.93, 72.47) --
	(195.00, 72.42) --
	(195.00, 72.42) --
	(195.00, 72.42) --
	(195.07, 72.39) --
	(195.07, 72.39) --
	(195.07, 72.39) --
	(195.15, 72.39) --
	(195.15, 72.39) --
	(195.15, 72.39) --
	(195.22, 72.37) --
	(195.22, 72.37) --
	(195.22, 72.37) --
	(195.23, 72.37) --
	(195.23, 72.37) --
	(195.23, 72.37) --
	(195.29, 72.33) --
	(195.29, 72.33) --
	(195.29, 72.33) --
	(195.36, 72.30) --
	(195.36, 72.30) --
	(195.36, 72.30) --
	(195.44, 72.25) --
	(195.44, 72.25) --
	(195.44, 72.25) --
	(195.51, 72.23) --
	(195.51, 72.23) --
	(195.51, 72.23) --
	(195.58, 72.20) --
	(195.58, 72.20) --
	(195.58, 72.20) --
	(195.66, 72.20) --
	(195.66, 72.20) --
	(195.66, 72.20) --
	(195.73, 72.16) --
	(195.73, 72.16) --
	(195.73, 72.16) --
	(195.80, 72.18) --
	(195.80, 72.18) --
	(195.80, 72.18) --
	(195.87, 72.11) --
	(195.87, 72.11) --
	(195.87, 72.11) --
	(195.95, 72.08) --
	(195.95, 72.08) --
	(195.95, 72.08) --
	(196.02, 72.04) --
	(196.02, 72.04) --
	(196.02, 72.04) --
	(196.09, 72.03) --
	(196.09, 72.03) --
	(196.09, 72.03) --
	(196.17, 72.04) --
	(196.17, 72.04) --
	(196.17, 72.04) --
	(196.24, 71.98) --
	(196.24, 71.98) --
	(196.24, 71.98) --
	(196.31, 71.98) --
	(196.31, 71.98) --
	(196.31, 71.98) --
	(196.38, 71.93) --
	(196.38, 71.93) --
	(196.38, 71.93) --
	(196.46, 71.88) --
	(196.46, 71.88) --
	(196.46, 71.88) --
	(196.53, 71.86) --
	(196.53, 71.86) --
	(196.53, 71.86) --
	(196.60, 71.85) --
	(196.60, 71.85) --
	(196.60, 71.85) --
	(196.68, 71.82) --
	(196.68, 71.82) --
	(196.68, 71.82) --
	(196.75, 71.78) --
	(196.75, 71.78) --
	(196.75, 71.78) --
	(196.82, 71.80) --
	(196.82, 71.80) --
	(196.82, 71.80) --
	(196.89, 71.74) --
	(196.89, 71.74) --
	(196.89, 71.74) --
	(196.97, 71.72) --
	(196.97, 71.72) --
	(196.97, 71.72) --
	(197.04, 71.71) --
	(197.04, 71.71) --
	(197.04, 71.71) --
	(197.06, 71.71) --
	(197.06, 71.71) --
	(197.06, 71.71) --
	(197.11, 71.72) --
	(197.11, 71.72) --
	(197.11, 71.72) --
	(197.19, 71.70) --
	(197.19, 71.70) --
	(197.19, 71.70) --
	(197.26, 71.68) --
	(197.26, 71.68) --
	(197.26, 71.68) --
	(197.33, 71.67) --
	(197.33, 71.67) --
	(197.33, 71.67) --
	(197.40, 71.67) --
	(197.40, 71.67) --
	(197.40, 71.67) --
	(197.48, 71.64) --
	(197.48, 71.64) --
	(197.48, 71.64) --
	(197.55, 71.61) --
	(197.55, 71.61) --
	(197.55, 71.61) --
	(197.62, 71.61) --
	(197.62, 71.61) --
	(197.62, 71.61) --
	(197.69, 71.54) --
	(197.69, 71.54) --
	(197.69, 71.54) --
	(197.77, 71.54) --
	(197.77, 71.54) --
	(197.77, 71.54) --
	(197.84, 71.52) --
	(197.84, 71.52) --
	(197.84, 71.52) --
	(197.91, 71.53) --
	(197.91, 71.53) --
	(197.91, 71.53) --
	(197.99, 71.50) --
	(197.99, 71.50) --
	(197.99, 71.50) --
	(198.06, 71.47) --
	(198.06, 71.47) --
	(198.06, 71.47) --
	(198.11, 71.45) --
	(198.11, 71.45) --
	(198.11, 71.45) --
	(198.13, 71.44) --
	(198.13, 71.44) --
	(198.13, 71.44) --
	(198.20, 71.42) --
	(198.20, 71.42) --
	(198.20, 71.42) --
	(198.28, 71.42) --
	(198.28, 71.42) --
	(198.28, 71.42) --
	(198.35, 71.37) --
	(198.35, 71.37) --
	(198.35, 71.37) --
	(198.42, 71.36) --
	(198.42, 71.36) --
	(198.42, 71.36) --
	(198.50, 71.30) --
	(198.50, 71.30) --
	(198.50, 71.30) --
	(198.57, 71.29) --
	(198.57, 71.29) --
	(198.57, 71.29) --
	(198.64, 71.29) --
	(198.64, 71.29) --
	(198.64, 71.29) --
	(198.71, 71.21) --
	(198.71, 71.21) --
	(198.71, 71.21) --
	(198.79, 71.19) --
	(198.79, 71.19) --
	(198.79, 71.19) --
	(198.86, 71.21) --
	(198.86, 71.21) --
	(198.86, 71.21) --
	(198.93, 71.17) --
	(198.93, 71.17) --
	(198.93, 71.17) --
	(199.00, 71.16) --
	(199.00, 71.16) --
	(199.00, 71.16) --
	(199.08, 71.15) --
	(199.08, 71.15) --
	(199.08, 71.15) --
	(199.15, 71.14) --
	(199.15, 71.14) --
	(199.15, 71.14) --
	(199.22, 71.08) --
	(199.22, 71.08) --
	(199.22, 71.08) --
	(199.26, 71.09) --
	(199.26, 71.09) --
	(199.26, 71.09) --
	(199.29, 71.09) --
	(199.29, 71.09) --
	(199.29, 71.09) --
	(199.37, 71.06) --
	(199.37, 71.06) --
	(199.37, 71.06) --
	(199.44, 71.00) --
	(199.44, 71.00) --
	(199.44, 71.00) --
	(199.51, 71.02) --
	(199.51, 71.02) --
	(199.51, 71.02) --
	(199.59, 71.03) --
	(199.59, 71.03) --
	(199.59, 71.03) --
	(199.66, 71.03) --
	(199.66, 71.03) --
	(199.66, 71.03) --
	(199.73, 71.02) --
	(199.73, 71.02) --
	(199.73, 71.02) --
	(199.80, 70.98) --
	(199.80, 70.98) --
	(199.80, 70.98) --
	(199.88, 70.99) --
	(199.88, 70.99) --
	(199.88, 70.99) --
	(199.95, 70.95) --
	(199.95, 70.95) --
	(199.95, 70.95) --
	(200.02, 70.97) --
	(200.02, 70.97) --
	(200.02, 70.97) --
	(200.03, 70.97) --
	(200.03, 70.97) --
	(200.03, 70.97) --
	(200.09, 70.99) --
	(200.09, 70.99) --
	(200.09, 70.99) --
	(200.17, 70.96) --
	(200.17, 70.96) --
	(200.17, 70.96) --
	(200.24, 71.00) --
	(200.24, 71.00) --
	(200.24, 71.00) --
	(200.31, 70.99) --
	(200.31, 70.99) --
	(200.31, 70.99) --
	(200.38, 70.99) --
	(200.38, 70.99) --
	(200.38, 70.99) --
	(200.46, 71.00) --
	(200.46, 71.00) --
	(200.46, 71.00) --
	(200.53, 71.02) --
	(200.53, 71.02) --
	(200.53, 71.02) --
	(200.60, 70.98) --
	(200.60, 70.98) --
	(200.60, 70.98) --
	(200.67, 71.00) --
	(200.67, 71.00) --
	(200.67, 71.00) --
	(200.75, 71.00) --
	(200.75, 71.00) --
	(200.75, 71.00) --
	(200.79, 71.01) --
	(200.79, 71.01) --
	(200.79, 71.01) --
	(200.82, 71.01) --
	(200.82, 71.01) --
	(200.82, 71.01) --
	(200.89, 71.00) --
	(200.89, 71.00) --
	(200.89, 71.00) --
	(200.96, 71.02) --
	(200.96, 71.02) --
	(200.96, 71.02) --
	(201.04, 71.04) --
	(201.04, 71.04) --
	(201.04, 71.04) --
	(201.11, 71.02) --
	(201.11, 71.02) --
	(201.11, 71.02) --
	(201.18, 71.02) --
	(201.18, 71.02) --
	(201.18, 71.02) --
	(201.26, 71.05) --
	(201.26, 71.05) --
	(201.26, 71.05) --
	(201.33, 71.06) --
	(201.33, 71.06) --
	(201.33, 71.06) --
	(201.40, 71.06) --
	(201.40, 71.06) --
	(201.40, 71.06) --
	(201.47, 71.08) --
	(201.47, 71.08) --
	(201.47, 71.08) --
	(201.54, 71.08) --
	(201.54, 71.08) --
	(201.54, 71.08) --
	(201.62, 71.11) --
	(201.62, 71.11) --
	(201.62, 71.11) --
	(201.69, 71.09) --
	(201.69, 71.09) --
	(201.69, 71.09) --
	(201.76, 71.15) --
	(201.76, 71.15) --
	(201.76, 71.15) --
	(201.83, 71.15) --
	(201.83, 71.15) --
	(201.83, 71.15) --
	(201.91, 71.17) --
	(201.91, 71.17) --
	(201.91, 71.17) --
	(201.98, 71.20) --
	(201.98, 71.20) --
	(201.98, 71.20) --
	(202.04, 71.21) --
	(202.04, 71.21) --
	(202.04, 71.21) --
	(202.05, 71.22) --
	(202.05, 71.22) --
	(202.05, 71.22) --
	(202.12, 71.24) --
	(202.12, 71.24) --
	(202.12, 71.24) --
	(202.20, 71.25) --
	(202.20, 71.25) --
	(202.20, 71.25) --
	(202.27, 71.29) --
	(202.27, 71.29) --
	(202.27, 71.29) --
	(202.34, 71.31) --
	(202.34, 71.31) --
	(202.34, 71.31) --
	(202.41, 71.32) --
	(202.41, 71.32) --
	(202.41, 71.32) --
	(202.49, 71.32) --
	(202.49, 71.32) --
	(202.49, 71.32) --
	(202.56, 71.33) --
	(202.56, 71.33) --
	(202.56, 71.33) --
	(202.63, 71.36) --
	(202.63, 71.36) --
	(202.63, 71.36) --
	(202.70, 71.36) --
	(202.70, 71.36) --
	(202.70, 71.36) --
	(202.78, 71.41) --
	(202.78, 71.41) --
	(202.78, 71.41) --
	(202.81, 71.43) --
	(202.81, 71.43) --
	(202.81, 71.43) --
	(202.85, 71.45) --
	(202.85, 71.45) --
	(202.85, 71.45) --
	(202.92, 71.49) --
	(202.92, 71.49) --
	(202.92, 71.49) --
	(202.99, 71.53) --
	(202.99, 71.53) --
	(202.99, 71.53) --
	(203.07, 71.58) --
	(203.07, 71.58) --
	(203.07, 71.58) --
	(203.14, 71.57) --
	(203.14, 71.57) --
	(203.14, 71.57) --
	(203.21, 71.55) --
	(203.21, 71.55) --
	(203.21, 71.55) --
	(203.28, 71.59) --
	(203.28, 71.59) --
	(203.28, 71.59) --
	(203.36, 71.56) --
	(203.36, 71.56) --
	(203.36, 71.56) --
	(203.43, 71.57) --
	(203.43, 71.57) --
	(203.43, 71.57) --
	(203.50, 71.57) --
	(203.50, 71.57) --
	(203.50, 71.57) --
	(203.57, 71.48) --
	(203.57, 71.48) --
	(203.57, 71.48) --
	(203.64, 71.50) --
	(203.65, 71.50) --
	(203.65, 71.50) --
	(203.72, 71.47) --
	(203.72, 71.47) --
	(203.72, 71.47) --
	(203.76, 71.47) --
	(203.76, 71.47) --
	(203.76, 71.47) --
	(203.79, 71.46) --
	(203.79, 71.46) --
	(203.79, 71.46) --
	(203.86, 71.44) --
	(203.86, 71.44) --
	(203.86, 71.44) --
	(203.94, 71.42) --
	(203.94, 71.42) --
	(203.94, 71.42) --
	(204.01, 71.35) --
	(204.01, 71.35) --
	(204.01, 71.35) --
	(204.08, 71.34) --
	(204.08, 71.34) --
	(204.08, 71.34) --
	(204.15, 71.30) --
	(204.15, 71.30) --
	(204.15, 71.30) --
	(204.22, 71.30) --
	(204.22, 71.30) --
	(204.22, 71.30) --
	(204.30, 71.30) --
	(204.30, 71.30) --
	(204.30, 71.30) --
	(204.37, 71.29) --
	(204.37, 71.29) --
	(204.37, 71.29) --
	(204.44, 71.25) --
	(204.44, 71.25) --
	(204.44, 71.25) --
	(204.51, 71.24) --
	(204.51, 71.24) --
	(204.51, 71.24) --
	(204.59, 71.24) --
	(204.59, 71.24) --
	(204.59, 71.24) --
	(204.63, 71.26) --
	(204.63, 71.26) --
	(204.63, 71.26) --
	(204.66, 71.27) --
	(204.66, 71.27) --
	(204.66, 71.27) --
	(204.73, 71.23) --
	(204.73, 71.23) --
	(204.73, 71.23) --
	(204.80, 71.23) --
	(204.80, 71.23) --
	(204.80, 71.23) --
	(204.88, 71.20) --
	(204.88, 71.20) --
	(204.88, 71.20) --
	(204.95, 71.22) --
	(204.95, 71.22) --
	(204.95, 71.22) --
	(205.02, 71.22) --
	(205.02, 71.22) --
	(205.02, 71.22) --
	(205.09, 71.22) --
	(205.09, 71.22) --
	(205.09, 71.22) --
	(205.16, 71.18) --
	(205.16, 71.18) --
	(205.16, 71.18) --
	(205.24, 71.18) --
	(205.24, 71.18) --
	(205.24, 71.18) --
	(205.31, 71.19) --
	(205.31, 71.19) --
	(205.31, 71.19) --
	(205.38, 71.13) --
	(205.38, 71.13) --
	(205.38, 71.13) --
	(205.45, 71.12) --
	(205.45, 71.12) --
	(205.45, 71.12) --
	(205.53, 71.07) --
	(205.53, 71.07) --
	(205.53, 71.07) --
	(205.58, 71.05) --
	(205.58, 71.05) --
	(205.58, 71.05) --
	(205.60, 71.05) --
	(205.60, 71.05) --
	(205.60, 71.05) --
	(205.67, 71.01) --
	(205.67, 71.01) --
	(205.67, 71.01) --
	(205.74, 70.98) --
	(205.74, 70.98) --
	(205.74, 70.98) --
	(205.82, 70.95) --
	(205.82, 70.95) --
	(205.82, 70.95) --
	(205.89, 70.91) --
	(205.89, 70.91) --
	(205.89, 70.91) --
	(205.96, 70.84) --
	(205.96, 70.84) --
	(205.96, 70.84) --
	(206.03, 70.84) --
	(206.03, 70.84) --
	(206.03, 70.84) --
	(206.10, 70.82) --
	(206.10, 70.82) --
	(206.10, 70.82) --
	(206.18, 70.80) --
	(206.18, 70.80) --
	(206.18, 70.80) --
	(206.25, 70.77) --
	(206.25, 70.77) --
	(206.25, 70.77) --
	(206.32, 70.75) --
	(206.32, 70.75) --
	(206.32, 70.75) --
	(206.39, 70.73) --
	(206.39, 70.73) --
	(206.39, 70.73) --
	(206.46, 70.71) --
	(206.46, 70.71) --
	(206.46, 70.71) --
	(206.54, 70.69) --
	(206.54, 70.69) --
	(206.54, 70.69) --
	(206.54, 70.68) --
	(206.54, 70.68) --
	(206.54, 70.68) --
	(206.61, 70.67) --
	(206.61, 70.67) --
	(206.61, 70.67) --
	(206.68, 70.67) --
	(206.68, 70.67) --
	(206.68, 70.67) --
	(206.75, 70.65) --
	(206.75, 70.65) --
	(206.75, 70.65) --
	(206.83, 70.66) --
	(206.83, 70.66) --
	(206.83, 70.66) --
	(206.90, 70.60) --
	(206.90, 70.60) --
	(206.90, 70.60) --
	(206.97, 70.63) --
	(206.97, 70.63) --
	(206.97, 70.63) --
	(207.04, 70.61) --
	(207.04, 70.61) --
	(207.04, 70.61) --
	(207.12, 70.58) --
	(207.12, 70.58) --
	(207.12, 70.58) --
	(207.19, 70.61) --
	(207.19, 70.61) --
	(207.19, 70.61) --
	(207.26, 70.62) --
	(207.26, 70.62) --
	(207.26, 70.62) --
	(207.33, 70.62) --
	(207.33, 70.62) --
	(207.33, 70.62) --
	(207.40, 70.68) --
	(207.40, 70.68) --
	(207.40, 70.68) --
	(207.48, 70.72) --
	(207.48, 70.72) --
	(207.48, 70.72) --
	(207.55, 70.77) --
	(207.55, 70.77) --
	(207.55, 70.77) --
	(207.60, 70.82) --
	(207.60, 70.82) --
	(207.60, 70.82) --
	(207.62, 70.84) --
	(207.62, 70.84) --
	(207.62, 70.84) --
	(207.69, 70.86) --
	(207.69, 70.86) --
	(207.69, 70.86) --
	(207.76, 70.88) --
	(207.76, 70.88) --
	(207.76, 70.88) --
	(207.84, 70.90) --
	(207.84, 70.90) --
	(207.84, 70.90) --
	(207.91, 70.89) --
	(207.91, 70.89) --
	(207.91, 70.89) --
	(207.98, 70.84) --
	(207.98, 70.84) --
	(207.98, 70.84) --
	(208.05, 70.82) --
	(208.05, 70.82) --
	(208.05, 70.82) --
	(208.12, 70.75) --
	(208.12, 70.75) --
	(208.12, 70.75) --
	(208.20, 70.68) --
	(208.20, 70.68) --
	(208.20, 70.68) --
	(208.27, 70.61) --
	(208.27, 70.61) --
	(208.27, 70.61) --
	(208.34, 70.60) --
	(208.34, 70.60) --
	(208.34, 70.60) --
	(208.41, 70.56) --
	(208.41, 70.56) --
	(208.41, 70.56) --
	(208.46, 70.53) --
	(208.46, 70.53) --
	(208.46, 70.53) --
	(208.48, 70.51) --
	(208.48, 70.51) --
	(208.48, 70.51) --
	(208.56, 70.51) --
	(208.56, 70.51) --
	(208.56, 70.51) --
	(208.63, 70.49) --
	(208.63, 70.49) --
	(208.63, 70.49) --
	(208.70, 70.46) --
	(208.70, 70.46) --
	(208.70, 70.46) --
	(208.77, 70.45) --
	(208.77, 70.45) --
	(208.77, 70.45) --
	(208.84, 70.46) --
	(208.84, 70.46) --
	(208.84, 70.46) --
	(208.92, 70.44) --
	(208.92, 70.44) --
	(208.92, 70.44) --
	(208.99, 70.44) --
	(208.99, 70.44) --
	(208.99, 70.44) --
	(209.06, 70.40) --
	(209.06, 70.40) --
	(209.06, 70.40) --
	(209.13, 70.43) --
	(209.13, 70.43) --
	(209.13, 70.43) --
	(209.20, 70.43) --
	(209.20, 70.43) --
	(209.20, 70.43) --
	(209.28, 70.45) --
	(209.28, 70.45) --
	(209.28, 70.45) --
	(209.32, 70.45) --
	(209.32, 70.45) --
	(209.32, 70.45) --
	(209.35, 70.45) --
	(209.35, 70.45) --
	(209.35, 70.45) --
	(209.42, 70.50) --
	(209.42, 70.50) --
	(209.42, 70.50) --
	(209.49, 70.48) --
	(209.49, 70.48) --
	(209.49, 70.48) --
	(209.56, 70.47) --
	(209.56, 70.47) --
	(209.56, 70.47) --
	(209.64, 70.48) --
	(209.64, 70.48) --
	(209.64, 70.48) --
	(209.71, 70.51) --
	(209.71, 70.51) --
	(209.71, 70.51) --
	(209.78, 70.55) --
	(209.78, 70.55) --
	(209.78, 70.55) --
	(209.85, 70.54) --
	(209.85, 70.54) --
	(209.85, 70.54) --
	(209.92, 70.57) --
	(209.92, 70.57) --
	(209.92, 70.57) --
	(210.00, 70.57) --
	(210.00, 70.57) --
	(210.00, 70.57) --
	(210.07, 70.61) --
	(210.07, 70.61) --
	(210.07, 70.61) --
	(210.14, 70.59) --
	(210.14, 70.59) --
	(210.14, 70.59) --
	(210.21, 70.66) --
	(210.21, 70.66) --
	(210.21, 70.66) --
	(210.28, 70.66) --
	(210.28, 70.66) --
	(210.28, 70.66) --
	(210.29, 70.66) --
	(210.29, 70.66) --
	(210.29, 70.66) --
	(210.36, 70.72) --
	(210.36, 70.72) --
	(210.36, 70.72) --
	(210.43, 70.75) --
	(210.43, 70.75) --
	(210.43, 70.75) --
	(210.50, 70.81) --
	(210.50, 70.81) --
	(210.50, 70.81) --
	(210.57, 70.82) --
	(210.57, 70.82) --
	(210.57, 70.82) --
	(210.65, 70.88) --
	(210.65, 70.88) --
	(210.65, 70.88) --
	(210.72, 70.88) --
	(210.72, 70.88) --
	(210.72, 70.88) --
	(210.79, 70.93) --
	(210.79, 70.93) --
	(210.79, 70.93) --
	(210.86, 71.00) --
	(210.86, 71.00) --
	(210.86, 71.00) --
	(210.93, 71.05) --
	(210.93, 71.05) --
	(210.93, 71.05) --
	(211.00, 71.10) --
	(211.00, 71.10) --
	(211.00, 71.10) --
	(211.08, 71.05) --
	(211.08, 71.05) --
	(211.08, 71.05) --
	(211.15, 71.07) --
	(211.15, 71.07) --
	(211.15, 71.07) --
	(211.22, 71.13) --
	(211.22, 71.13) --
	(211.22, 71.13) --
	(211.29, 71.14) --
	(211.29, 71.14) --
	(211.29, 71.14) --
	(211.36, 71.14) --
	(211.36, 71.14) --
	(211.36, 71.14) --
	(211.43, 71.16) --
	(211.43, 71.16) --
	(211.43, 71.16) --
	(211.43, 71.16) --
	(211.43, 71.16) --
	(211.43, 71.16) --
	(211.51, 71.16) --
	(211.51, 71.16) --
	(211.51, 71.16) --
	(211.58, 71.22) --
	(211.58, 71.22) --
	(211.58, 71.22) --
	(211.65, 71.30) --
	(211.65, 71.30) --
	(211.65, 71.30) --
	(211.72, 71.33) --
	(211.72, 71.33) --
	(211.72, 71.33) --
	(211.79, 71.38) --
	(211.79, 71.38) --
	(211.79, 71.38) --
	(211.87, 71.47) --
	(211.87, 71.47) --
	(211.87, 71.47) --
	(211.94, 71.51) --
	(211.94, 71.51) --
	(211.94, 71.51) --
	(212.01, 71.59) --
	(212.01, 71.59) --
	(212.01, 71.59) --
	(212.08, 71.60) --
	(212.08, 71.60) --
	(212.08, 71.60) --
	(212.15, 71.64) --
	(212.15, 71.64) --
	(212.15, 71.64) --
	(212.23, 71.66) --
	(212.23, 71.66) --
	(212.23, 71.66) --
	(212.30, 71.67) --
	(212.30, 71.67) --
	(212.30, 71.67) --
	(212.37, 71.65) --
	(212.37, 71.65) --
	(212.37, 71.65) --
	(212.44, 71.70) --
	(212.44, 71.70) --
	(212.44, 71.70) --
	(212.51, 71.68) --
	(212.51, 71.68) --
	(212.51, 71.68) --
	(212.58, 71.73) --
	(212.58, 71.73) --
	(212.58, 71.73) --
	(212.66, 71.78) --
	(212.66, 71.78) --
	(212.66, 71.78) --
	(212.73, 71.80) --
	(212.73, 71.80) --
	(212.73, 71.80) --
	(212.80, 71.82) --
	(212.80, 71.82) --
	(212.80, 71.82) --
	(212.87, 71.88) --
	(212.87, 71.88) --
	(212.87, 71.88) --
	(212.94, 71.94) --
	(212.94, 71.94) --
	(212.94, 71.94) --
	(212.96, 71.96) --
	(212.96, 71.96) --
	(212.96, 71.96) --
	(213.02, 72.02) --
	(213.02, 72.02) --
	(213.02, 72.02) --
	(213.09, 72.15) --
	(213.09, 72.15) --
	(213.09, 72.15) --
	(213.16, 72.32) --
	(213.16, 72.32) --
	(213.16, 72.32) --
	(213.23, 72.58) --
	(213.23, 72.58) --
	(213.23, 72.58) --
	(213.30, 72.90) --
	(213.30, 72.90) --
	(213.30, 72.90) --
	(213.37, 73.24) --
	(213.37, 73.24) --
	(213.37, 73.24) --
	(213.45, 73.43) --
	(213.45, 73.43) --
	(213.45, 73.43) --
	(213.52, 73.36) --
	(213.52, 73.36) --
	(213.52, 73.36) --
	(213.59, 73.32) --
	(213.59, 73.32) --
	(213.59, 73.32) --
	(213.66, 73.30) --
	(213.66, 73.30) --
	(213.66, 73.30) --
	(213.73, 73.30) --
	(213.73, 73.30) --
	(213.73, 73.30) --
	(213.80, 73.30) --
	(213.80, 73.30) --
	(213.80, 73.30) --
	(213.88, 73.43) --
	(213.88, 73.43) --
	(213.88, 73.43) --
	(213.95, 73.49) --
	(213.95, 73.49) --
	(213.95, 73.49) --
	(214.02, 73.58) --
	(214.02, 73.58) --
	(214.02, 73.58) --
	(214.09, 73.71) --
	(214.09, 73.71) --
	(214.09, 73.71) --
	(214.16, 73.87) --
	(214.16, 73.87) --
	(214.16, 73.87) --
	(214.23, 74.09) --
	(214.23, 74.09) --
	(214.23, 74.09) --
	(214.31, 74.22) --
	(214.31, 74.22) --
	(214.31, 74.22) --
	(214.31, 74.22) --
	(214.31, 74.22) --
	(214.31, 74.22) --
	(214.38, 74.33) --
	(214.38, 74.33) --
	(214.38, 74.33) --
	(214.45, 74.35) --
	(214.45, 74.35) --
	(214.45, 74.35) --
	(214.52, 74.32) --
	(214.52, 74.32) --
	(214.52, 74.32) --
	(214.59, 74.16) --
	(214.59, 74.16) --
	(214.59, 74.16) --
	(214.66, 74.10) --
	(214.66, 74.10) --
	(214.66, 74.10) --
	(214.74, 73.87) --
	(214.74, 73.87) --
	(214.74, 73.87) --
	(214.81, 73.73) --
	(214.81, 73.73) --
	(214.81, 73.73) --
	(214.88, 73.62) --
	(214.88, 73.62) --
	(214.88, 73.62) --
	(214.95, 73.51) --
	(214.95, 73.51) --
	(214.95, 73.51) --
	(215.02, 73.38) --
	(215.02, 73.38) --
	(215.02, 73.38) --
	(215.09, 73.31) --
	(215.09, 73.31) --
	(215.09, 73.31) --
	(215.17, 73.28) --
	(215.17, 73.28) --
	(215.17, 73.28) --
	(215.24, 73.26) --
	(215.24, 73.26) --
	(215.24, 73.26) --
	(215.31, 73.18) --
	(215.31, 73.18) --
	(215.31, 73.18) --
	(215.38, 73.12) --
	(215.38, 73.12) --
	(215.38, 73.12) --
	(215.45, 73.11) --
	(215.45, 73.11) --
	(215.45, 73.11) --
	(215.52, 73.04) --
	(215.52, 73.04) --
	(215.52, 73.04) --
	(215.60, 73.10) --
	(215.60, 73.10) --
	(215.60, 73.10) --
	(215.67, 73.11) --
	(215.67, 73.11) --
	(215.67, 73.11) --
	(215.74, 73.12) --
	(215.74, 73.12) --
	(215.74, 73.12) --
	(215.74, 73.12) --
	(215.74, 73.12) --
	(215.74, 73.12) --
	(215.81, 73.15) --
	(215.81, 73.15) --
	(215.81, 73.15) --
	(215.88, 73.17) --
	(215.88, 73.17) --
	(215.88, 73.17) --
	(215.95, 73.22) --
	(215.95, 73.22) --
	(215.95, 73.22) --
	(216.03, 73.16) --
	(216.03, 73.16) --
	(216.03, 73.16) --
	(216.10, 73.15) --
	(216.10, 73.15) --
	(216.10, 73.15) --
	(216.17, 73.12) --
	(216.17, 73.12) --
	(216.17, 73.12) --
	(216.24, 73.10) --
	(216.24, 73.10) --
	(216.24, 73.10) --
	(216.31, 73.05) --
	(216.31, 73.05) --
	(216.31, 73.05) --
	(216.38, 73.02) --
	(216.38, 73.02) --
	(216.38, 73.02) --
	(216.46, 72.95) --
	(216.46, 72.95) --
	(216.46, 72.95) --
	(216.53, 72.93) --
	(216.53, 72.93) --
	(216.53, 72.93) --
	(216.60, 72.94) --
	(216.60, 72.94) --
	(216.60, 72.94) --
	(216.67, 72.96) --
	(216.67, 72.96) --
	(216.67, 72.96) --
	(216.74, 72.92) --
	(216.74, 72.92) --
	(216.74, 72.92) --
	(216.81, 72.90) --
	(216.81, 72.90) --
	(216.81, 72.90) --
	(216.88, 72.87) --
	(216.88, 72.87) --
	(216.88, 72.87) --
	(216.96, 72.88) --
	(216.96, 72.88) --
	(216.96, 72.88) --
	(217.03, 72.86) --
	(217.03, 72.86) --
	(217.03, 72.86) --
	(217.10, 72.88) --
	(217.10, 72.88) --
	(217.10, 72.88) --
	(217.17, 72.93) --
	(217.17, 72.93) --
	(217.17, 72.93) --
	(217.24, 73.03) --
	(217.24, 73.03) --
	(217.24, 73.03) --
	(217.31, 73.10) --
	(217.31, 73.10) --
	(217.31, 73.10) --
	(217.38, 73.17) --
	(217.38, 73.17) --
	(217.38, 73.17) --
	(217.46, 73.25) --
	(217.46, 73.25) --
	(217.46, 73.25) --
	(217.53, 73.38) --
	(217.53, 73.38) --
	(217.53, 73.38) --
	(217.60, 73.53) --
	(217.60, 73.53) --
	(217.60, 73.53) --
	(217.67, 73.60) --
	(217.67, 73.60) --
	(217.67, 73.60) --
	(217.74, 73.79) --
	(217.74, 73.79) --
	(217.74, 73.79) --
	(217.76, 73.83) --
	(217.76, 73.83) --
	(217.76, 73.83) --
	(217.81, 74.01) --
	(217.81, 74.01) --
	(217.81, 74.01) --
	(217.88, 74.16) --
	(217.88, 74.16) --
	(217.88, 74.16) --
	(217.96, 74.39) --
	(217.96, 74.39) --
	(217.96, 74.39) --
	(218.03, 74.64) --
	(218.03, 74.64) --
	(218.03, 74.64) --
	(218.10, 74.81) --
	(218.10, 74.81) --
	(218.10, 74.81) --
	(218.17, 74.94) --
	(218.17, 74.94) --
	(218.17, 74.94) --
	(218.24, 74.93) --
	(218.24, 74.93) --
	(218.24, 74.93) --
	(218.31, 74.85) --
	(218.31, 74.85) --
	(218.31, 74.85) --
	(218.38, 74.70) --
	(218.38, 74.70) --
	(218.38, 74.70) --
	(218.46, 74.47) --
	(218.46, 74.47) --
	(218.46, 74.47) --
	(218.53, 74.32) --
	(218.53, 74.32) --
	(218.53, 74.32) --
	(218.60, 74.11) --
	(218.60, 74.11) --
	(218.60, 74.11) --
	(218.67, 73.91) --
	(218.67, 73.91) --
	(218.67, 73.91) --
	(218.74, 73.84) --
	(218.74, 73.84) --
	(218.74, 73.84) --
	(218.81, 73.66) --
	(218.81, 73.66) --
	(218.81, 73.66) --
	(218.88, 73.48) --
	(218.88, 73.48) --
	(218.88, 73.48) --
	(218.95, 73.32) --
	(218.95, 73.32) --
	(218.95, 73.32) --
	(219.03, 73.20) --
	(219.03, 73.20) --
	(219.03, 73.20) --
	(219.10, 73.16) --
	(219.10, 73.16) --
	(219.10, 73.16) --
	(219.17, 73.02) --
	(219.17, 73.02) --
	(219.17, 73.02) --
	(219.24, 72.96) --
	(219.24, 72.96) --
	(219.24, 72.96) --
	(219.31, 72.89) --
	(219.31, 72.89) --
	(219.31, 72.89) --
	(219.38, 72.81) --
	(219.38, 72.81) --
	(219.38, 72.81) --
	(219.46, 72.67) --
	(219.46, 72.67) --
	(219.46, 72.67) --
	(219.48, 72.63) --
	(219.48, 72.63) --
	(219.48, 72.63) --
	(219.53, 72.57) --
	(219.53, 72.57) --
	(219.53, 72.57) --
	(219.60, 72.49) --
	(219.60, 72.49) --
	(219.60, 72.49) --
	(219.67, 72.50) --
	(219.67, 72.50) --
	(219.67, 72.50) --
	(219.74, 72.41) --
	(219.74, 72.41) --
	(219.74, 72.41) --
	(219.81, 72.37) --
	(219.81, 72.37) --
	(219.81, 72.37) --
	(219.88, 72.35) --
	(219.88, 72.35) --
	(219.88, 72.35) --
	(219.95, 72.36) --
	(219.95, 72.36) --
	(219.95, 72.36) --
	(220.03, 72.35) --
	(220.03, 72.35) --
	(220.03, 72.35) --
	(220.10, 72.39) --
	(220.10, 72.39) --
	(220.10, 72.39) --
	(220.17, 72.41) --
	(220.17, 72.41) --
	(220.17, 72.41) --
	(220.24, 72.43) --
	(220.24, 72.43) --
	(220.24, 72.43) --
	(220.31, 72.46) --
	(220.31, 72.46) --
	(220.31, 72.46) --
	(220.38, 72.51) --
	(220.38, 72.51) --
	(220.38, 72.51) --
	(220.45, 72.60) --
	(220.45, 72.60) --
	(220.45, 72.60) --
	(220.52, 72.76) --
	(220.52, 72.76) --
	(220.52, 72.76) --
	(220.60, 72.90) --
	(220.60, 72.90) --
	(220.60, 72.90) --
	(220.67, 73.08) --
	(220.67, 73.08) --
	(220.67, 73.08) --
	(220.74, 73.22) --
	(220.74, 73.22) --
	(220.74, 73.22) --
	(220.81, 73.29) --
	(220.81, 73.29) --
	(220.81, 73.29) --
	(220.88, 73.27) --
	(220.88, 73.27) --
	(220.88, 73.27) --
	(220.95, 73.24) --
	(220.95, 73.24) --
	(220.95, 73.24) --
	(221.02, 73.23) --
	(221.02, 73.23) --
	(221.02, 73.23) --
	(221.09, 73.25) --
	(221.09, 73.25) --
	(221.09, 73.25) --
	(221.17, 73.38) --
	(221.17, 73.38) --
	(221.17, 73.38) --
	(221.24, 73.60) --
	(221.24, 73.60) --
	(221.24, 73.60) --
	(221.31, 73.82) --
	(221.31, 73.82) --
	(221.31, 73.82) --
	(221.38, 73.88) --
	(221.38, 73.88) --
	(221.38, 73.88) --
	(221.45, 73.91) --
	(221.45, 73.91) --
	(221.45, 73.91) --
	(221.49, 73.97) --
	(221.49, 73.97) --
	(221.49, 73.97) --
	(221.52, 74.00) --
	(221.52, 74.00) --
	(221.52, 74.00) --
	(221.59, 73.85) --
	(221.59, 73.85) --
	(221.59, 73.85) --
	(221.66, 73.65) --
	(221.66, 73.65) --
	(221.66, 73.65) --
	(221.74, 73.37) --
	(221.74, 73.37) --
	(221.74, 73.37) --
	(221.81, 73.02) --
	(221.81, 73.02) --
	(221.81, 73.02) --
	(221.88, 72.87) --
	(221.88, 72.87) --
	(221.88, 72.87) --
	(221.95, 72.65) --
	(221.95, 72.65) --
	(221.95, 72.65) --
	(222.02, 72.51) --
	(222.02, 72.51) --
	(222.02, 72.51) --
	(222.09, 72.49) --
	(222.09, 72.49) --
	(222.09, 72.49) --
	(222.16, 72.45) --
	(222.16, 72.45) --
	(222.16, 72.45) --
	(222.23, 72.45) --
	(222.23, 72.45) --
	(222.23, 72.45) --
	(222.30, 72.53) --
	(222.30, 72.53) --
	(222.30, 72.53) --
	(222.37, 72.65) --
	(222.37, 72.65) --
	(222.37, 72.65) --
	(222.45, 72.71) --
	(222.45, 72.71) --
	(222.45, 72.71) --
	(222.52, 72.88) --
	(222.52, 72.88) --
	(222.52, 72.88) --
	(222.59, 73.05) --
	(222.59, 73.05) --
	(222.59, 73.05) --
	(222.66, 73.13) --
	(222.66, 73.13) --
	(222.66, 73.13) --
	(222.73, 73.19) --
	(222.73, 73.19) --
	(222.73, 73.19) --
	(222.80, 73.16) --
	(222.80, 73.16) --
	(222.80, 73.16) --
	(222.87, 73.06) --
	(222.87, 73.06) --
	(222.87, 73.06) --
	(222.94, 72.95) --
	(222.94, 72.95) --
	(222.94, 72.95) --
	(223.02, 72.89) --
	(223.02, 72.89) --
	(223.02, 72.89) --
	(223.09, 72.72) --
	(223.09, 72.72) --
	(223.09, 72.72) --
	(223.16, 72.60) --
	(223.16, 72.60) --
	(223.16, 72.60) --
	(223.23, 72.50) --
	(223.23, 72.50) --
	(223.23, 72.50) --
	(223.30, 72.36) --
	(223.30, 72.36) --
	(223.30, 72.36) --
	(223.37, 72.26) --
	(223.37, 72.26) --
	(223.37, 72.26) --
	(223.44, 72.19) --
	(223.44, 72.19) --
	(223.44, 72.19) --
	(223.51, 72.11) --
	(223.51, 72.11) --
	(223.51, 72.11) --
	(223.58, 72.02) --
	(223.58, 72.02) --
	(223.58, 72.02) --
	(223.66, 72.01) --
	(223.66, 72.01) --
	(223.66, 72.01) --
	(223.73, 71.96) --
	(223.73, 71.96) --
	(223.73, 71.96) --
	(223.79, 71.95) --
	(223.79, 71.95) --
	(223.79, 71.95) --
	(223.80, 71.95) --
	(223.80, 71.95) --
	(223.80, 71.95) --
	(223.87, 71.92) --
	(223.87, 71.92) --
	(223.87, 71.92) --
	(223.94, 71.97) --
	(223.94, 71.97) --
	(223.94, 71.97) --
	(224.01, 72.07) --
	(224.01, 72.07) --
	(224.01, 72.07) --
	(224.08, 72.18) --
	(224.08, 72.18) --
	(224.08, 72.18) --
	(224.15, 72.17) --
	(224.15, 72.17) --
	(224.15, 72.17) --
	(224.22, 72.17) --
	(224.22, 72.17) --
	(224.22, 72.17) --
	(224.29, 72.17) --
	(224.29, 72.17) --
	(224.29, 72.17) --
	(224.36, 72.07) --
	(224.36, 72.07) --
	(224.36, 72.07) --
	(224.44, 72.03) --
	(224.44, 72.03) --
	(224.44, 72.03) --
	(224.51, 71.94) --
	(224.51, 71.94) --
	(224.51, 71.94) --
	(224.58, 71.87) --
	(224.58, 71.87) --
	(224.58, 71.87) --
	(224.65, 71.90) --
	(224.65, 71.90) --
	(224.65, 71.90) --
	(224.72, 71.92) --
	(224.72, 71.92) --
	(224.72, 71.92) --
	(224.79, 71.90) --
	(224.79, 71.90) --
	(224.79, 71.90) --
	(224.86, 71.83) --
	(224.86, 71.83) --
	(224.86, 71.83) --
	(224.93, 71.81) --
	(224.93, 71.81) --
	(224.93, 71.81) --
	(225.00, 71.76) --
	(225.00, 71.76) --
	(225.00, 71.76) --
	(225.07, 71.72) --
	(225.07, 71.72) --
	(225.07, 71.72) --
	(225.15, 71.67) --
	(225.15, 71.67) --
	(225.15, 71.67) --
	(225.22, 71.66) --
	(225.22, 71.66) --
	(225.22, 71.66) --
	(225.29, 71.67) --
	(225.29, 71.67) --
	(225.29, 71.67) --
	(225.36, 71.68) --
	(225.36, 71.68) --
	(225.36, 71.68) --
	(225.43, 71.66) --
	(225.43, 71.66) --
	(225.43, 71.66) --
	(225.50, 71.73) --
	(225.50, 71.73) --
	(225.50, 71.73) --
	(225.57, 71.75) --
	(225.57, 71.75) --
	(225.57, 71.75) --
	(225.61, 71.78) --
	(225.61, 71.78) --
	(225.61, 71.78) --
	(225.64, 71.80) --
	(225.64, 71.80) --
	(225.64, 71.80) --
	(225.71, 71.90) --
	(225.71, 71.90) --
	(225.71, 71.90) --
	(225.78, 71.94) --
	(225.78, 71.94) --
	(225.78, 71.94) --
	(225.85, 72.05) --
	(225.85, 72.05) --
	(225.85, 72.05) --
	(225.93, 72.18) --
	(225.93, 72.18) --
	(225.93, 72.18) --
	(226.00, 72.25) --
	(226.00, 72.25) --
	(226.00, 72.25) --
	(226.07, 72.31) --
	(226.07, 72.31) --
	(226.07, 72.31) --
	(226.14, 72.35) --
	(226.14, 72.35) --
	(226.14, 72.35) --
	(226.21, 72.54) --
	(226.21, 72.54) --
	(226.21, 72.54) --
	(226.28, 72.91) --
	(226.28, 72.91) --
	(226.28, 72.91) --
	(226.35, 73.85) --
	(226.35, 73.85) --
	(226.35, 73.85) --
	(226.42, 75.69) --
	(226.42, 75.69) --
	(226.42, 75.69) --
	(226.49, 77.82) --
	(226.49, 77.82) --
	(226.49, 77.82) --
	(226.56, 78.62) --
	(226.56, 78.62) --
	(226.56, 78.62) --
	(226.63, 77.77) --
	(226.63, 77.77) --
	(226.63, 77.77) --
	(226.71, 76.26) --
	(226.71, 76.26) --
	(226.71, 76.26) --
	(226.78, 74.90) --
	(226.78, 74.90) --
	(226.78, 74.90) --
	(226.85, 73.79) --
	(226.85, 73.79) --
	(226.85, 73.79) --
	(226.92, 72.98) --
	(226.92, 72.98) --
	(226.92, 72.98) --
	(226.99, 72.36) --
	(226.99, 72.36) --
	(226.99, 72.36) --
	(227.06, 71.88) --
	(227.06, 71.88) --
	(227.06, 71.88) --
	(227.13, 71.59) --
	(227.13, 71.59) --
	(227.13, 71.59) --
	(227.20, 71.30) --
	(227.20, 71.30) --
	(227.20, 71.30) --
	(227.27, 71.08) --
	(227.27, 71.08) --
	(227.27, 71.08) --
	(227.34, 70.94) --
	(227.34, 70.94) --
	(227.34, 70.94) --
	(227.41, 70.76) --
	(227.41, 70.76) --
	(227.41, 70.76) --
	(227.48, 70.66) --
	(227.48, 70.66) --
	(227.48, 70.66) --
	(227.56, 70.56) --
	(227.56, 70.56) --
	(227.56, 70.56) --
	(227.62, 70.46) --
	(227.62, 70.46) --
	(227.62, 70.46) --
	(227.70, 70.38) --
	(227.70, 70.38) --
	(227.70, 70.38) --
	(227.77, 70.32) --
	(227.77, 70.32) --
	(227.77, 70.32) --
	(227.84, 70.23) --
	(227.84, 70.23) --
	(227.84, 70.23) --
	(227.91, 70.13) --
	(227.91, 70.13) --
	(227.91, 70.13) --
	(227.91, 70.12) --
	(227.91, 70.12) --
	(227.91, 70.12) --
	(227.98, 70.03) --
	(227.98, 70.03) --
	(227.98, 70.03) --
	(228.05, 69.97) --
	(228.05, 69.97) --
	(228.05, 69.97) --
	(228.12, 69.90) --
	(228.12, 69.90) --
	(228.12, 69.90) --
	(228.19, 69.83) --
	(228.19, 69.83) --
	(228.19, 69.83) --
	(228.26, 69.76) --
	(228.26, 69.76) --
	(228.26, 69.76) --
	(228.33, 69.71) --
	(228.33, 69.71) --
	(228.33, 69.71) --
	(228.40, 69.65) --
	(228.40, 69.65) --
	(228.40, 69.65) --
	(228.47, 69.58) --
	(228.47, 69.58) --
	(228.47, 69.58) --
	(228.55, 69.58) --
	(228.55, 69.58) --
	(228.55, 69.58) --
	(228.62, 69.49) --
	(228.62, 69.49) --
	(228.62, 69.49) --
	(228.69, 69.44) --
	(228.69, 69.44) --
	(228.69, 69.44) --
	(228.76, 69.41) --
	(228.76, 69.41) --
	(228.76, 69.41) --
	(228.83, 69.39) --
	(228.83, 69.39) --
	(228.83, 69.39) --
	(228.90, 69.36) --
	(228.90, 69.36) --
	(228.90, 69.36) --
	(228.97, 69.30) --
	(228.97, 69.30) --
	(228.97, 69.30) --
	(229.04, 69.29) --
	(229.04, 69.29) --
	(229.04, 69.29) --
	(229.11, 69.29) --
	(229.11, 69.29) --
	(229.11, 69.29) --
	(229.18, 69.28) --
	(229.18, 69.28) --
	(229.18, 69.28) --
	(229.25, 69.31) --
	(229.25, 69.31) --
	(229.25, 69.31) --
	(229.32, 69.33) --
	(229.32, 69.33) --
	(229.32, 69.33) --
	(229.39, 69.31) --
	(229.39, 69.31) --
	(229.39, 69.31) --
	(229.46, 69.28) --
	(229.46, 69.28) --
	(229.46, 69.28) --
	(229.53, 69.27) --
	(229.53, 69.27) --
	(229.53, 69.27) --
	(229.60, 69.22) --
	(229.60, 69.22) --
	(229.60, 69.22) --
	(229.68, 69.18) --
	(229.68, 69.18) --
	(229.68, 69.18) --
	(229.75, 69.13) --
	(229.75, 69.13) --
	(229.75, 69.13) --
	(229.82, 69.09) --
	(229.82, 69.09) --
	(229.82, 69.09) --
	(229.89, 69.06) --
	(229.89, 69.06) --
	(229.89, 69.06) --
	(229.96, 69.06) --
	(229.96, 69.06) --
	(229.96, 69.06) --
	(230.03, 69.07) --
	(230.03, 69.07) --
	(230.03, 69.07) --
	(230.10, 69.03) --
	(230.10, 69.03) --
	(230.10, 69.03) --
	(230.17, 69.06) --
	(230.17, 69.06) --
	(230.17, 69.06) --
	(230.24, 69.08) --
	(230.24, 69.08) --
	(230.24, 69.08) --
	(230.31, 69.22) --
	(230.31, 69.22) --
	(230.31, 69.22) --
	(230.38, 69.42) --
	(230.38, 69.42) --
	(230.38, 69.42) --
	(230.45, 69.69) --
	(230.45, 69.69) --
	(230.45, 69.69) --
	(230.50, 69.86) --
	(230.50, 69.86) --
	(230.50, 69.86) --
	(230.52, 69.94) --
	(230.52, 69.94) --
	(230.52, 69.94) --
	(230.59, 70.07) --
	(230.59, 70.07) --
	(230.59, 70.07) --
	(230.66, 70.00) --
	(230.66, 70.00) --
	(230.66, 70.00) --
	(230.73, 69.87) --
	(230.73, 69.87) --
	(230.73, 69.87) --
	(230.80, 69.67) --
	(230.80, 69.67) --
	(230.80, 69.67) --
	(230.88, 69.50) --
	(230.88, 69.50) --
	(230.88, 69.50) --
	(230.95, 69.30) --
	(230.95, 69.30) --
	(230.95, 69.30) --
	(231.02, 69.16) --
	(231.02, 69.16) --
	(231.02, 69.16) --
	(231.09, 69.09) --
	(231.09, 69.09) --
	(231.09, 69.09) --
	(231.16, 68.96) --
	(231.16, 68.96) --
	(231.16, 68.96) --
	(231.23, 68.91) --
	(231.23, 68.91) --
	(231.23, 68.91) --
	(231.30, 68.86) --
	(231.30, 68.86) --
	(231.30, 68.86) --
	(231.37, 68.81) --
	(231.37, 68.81) --
	(231.37, 68.81) --
	(231.44, 68.81) --
	(231.44, 68.81) --
	(231.44, 68.81) --
	(231.51, 68.86) --
	(231.51, 68.86) --
	(231.51, 68.86) --
	(231.58, 68.91) --
	(231.58, 68.91) --
	(231.58, 68.91) --
	(231.65, 69.02) --
	(231.65, 69.02) --
	(231.65, 69.02) --
	(231.72, 69.13) --
	(231.72, 69.13) --
	(231.72, 69.13) --
	(231.79, 69.28) --
	(231.79, 69.28) --
	(231.79, 69.28) --
	(231.86, 69.36) --
	(231.86, 69.36) --
	(231.86, 69.36) --
	(231.93, 69.33) --
	(231.93, 69.33) --
	(231.93, 69.33) --
	(232.00, 69.28) --
	(232.00, 69.28) --
	(232.00, 69.28) --
	(232.07, 69.17) --
	(232.07, 69.17) --
	(232.07, 69.17) --
	(232.14, 69.12) --
	(232.14, 69.12) --
	(232.14, 69.12) --
	(232.21, 69.03) --
	(232.21, 69.03) --
	(232.21, 69.03) --
	(232.28, 68.94) --
	(232.28, 68.94) --
	(232.28, 68.94) --
	(232.36, 68.88) --
	(232.36, 68.88) --
	(232.36, 68.88) --
	(232.43, 68.81) --
	(232.43, 68.81) --
	(232.43, 68.81) --
	(232.50, 68.76) --
	(232.50, 68.76) --
	(232.50, 68.76) --
	(232.57, 68.73) --
	(232.57, 68.73) --
	(232.57, 68.73) --
	(232.64, 68.67) --
	(232.64, 68.67) --
	(232.64, 68.67) --
	(232.71, 68.62) --
	(232.71, 68.62) --
	(232.71, 68.62) --
	(232.78, 68.57) --
	(232.78, 68.57) --
	(232.78, 68.57) --
	(232.85, 68.55) --
	(232.85, 68.55) --
	(232.85, 68.55) --
	(232.92, 68.50) --
	(232.92, 68.50) --
	(232.92, 68.50) --
	(232.99, 68.47) --
	(232.99, 68.47) --
	(232.99, 68.47) --
	(233.06, 68.46) --
	(233.06, 68.46) --
	(233.06, 68.46) --
	(233.13, 68.47) --
	(233.13, 68.47) --
	(233.13, 68.47) --
	(233.20, 68.49) --
	(233.20, 68.49) --
	(233.20, 68.49) --
	(233.27, 68.52) --
	(233.27, 68.52) --
	(233.27, 68.52) --
	(233.28, 68.53) --
	(233.28, 68.53) --
	(233.28, 68.53) --
	(233.34, 68.56) --
	(233.34, 68.56) --
	(233.34, 68.56) --
	(233.41, 68.57) --
	(233.41, 68.57) --
	(233.41, 68.57) --
	(233.48, 68.55) --
	(233.48, 68.55) --
	(233.48, 68.55) --
	(233.55, 68.61) --
	(233.55, 68.61) --
	(233.55, 68.61) --
	(233.62, 68.60) --
	(233.62, 68.60) --
	(233.62, 68.60) --
	(233.69, 68.58) --
	(233.69, 68.58) --
	(233.69, 68.58) --
	(233.76, 68.56) --
	(233.76, 68.56) --
	(233.76, 68.56) --
	(233.83, 68.56) --
	(233.83, 68.56) --
	(233.83, 68.56) --
	(233.90, 68.55) --
	(233.90, 68.55) --
	(233.90, 68.55) --
	(233.97, 68.52) --
	(233.97, 68.52) --
	(233.97, 68.52) --
	(234.04, 68.49) --
	(234.04, 68.49) --
	(234.04, 68.49) --
	(234.11, 68.49) --
	(234.11, 68.49) --
	(234.11, 68.49) --
	(234.18, 68.44) --
	(234.18, 68.44) --
	(234.18, 68.44) --
	(234.25, 68.42) --
	(234.26, 68.42) --
	(234.26, 68.42) --
	(234.32, 68.42) --
	(234.32, 68.42) --
	(234.32, 68.42) --
	(234.40, 68.45) --
	(234.40, 68.45) --
	(234.40, 68.45) --
	(234.47, 68.40) --
	(234.47, 68.40) --
	(234.47, 68.40) --
	(234.54, 68.41) --
	(234.54, 68.41) --
	(234.54, 68.41) --
	(234.61, 68.40) --
	(234.61, 68.40) --
	(234.61, 68.40) --
	(234.68, 68.44) --
	(234.68, 68.44) --
	(234.68, 68.44) --
	(234.75, 68.44) --
	(234.75, 68.44) --
	(234.75, 68.44) --
	(234.82, 68.48) --
	(234.82, 68.48) --
	(234.82, 68.48) --
	(234.89, 68.51) --
	(234.89, 68.51) --
	(234.89, 68.51) --
	(234.96, 68.60) --
	(234.96, 68.60) --
	(234.96, 68.60) --
	(235.03, 68.70) --
	(235.03, 68.70) --
	(235.03, 68.70) --
	(235.10, 68.79) --
	(235.10, 68.79) --
	(235.10, 68.79) --
	(235.17, 68.91) --
	(235.17, 68.91) --
	(235.17, 68.91) --
	(235.24, 69.07) --
	(235.24, 69.07) --
	(235.24, 69.07) --
	(235.31, 69.21) --
	(235.31, 69.21) --
	(235.31, 69.21) --
	(235.38, 69.45) --
	(235.38, 69.45) --
	(235.38, 69.45) --
	(235.45, 69.81) --
	(235.45, 69.81) --
	(235.45, 69.81) --
	(235.52, 70.29) --
	(235.52, 70.29) --
	(235.52, 70.29) --
	(235.59, 70.98) --
	(235.59, 70.98) --
	(235.59, 70.98) --
	(235.66, 71.90) --
	(235.66, 71.90) --
	(235.66, 71.90) --
	(235.73, 73.16) --
	(235.73, 73.16) --
	(235.73, 73.16) --
	(235.80, 75.40) --
	(235.80, 75.40) --
	(235.80, 75.40) --
	(235.87, 78.92) --
	(235.87, 78.92) --
	(235.87, 78.92) --
	(235.94, 82.69) --
	(235.94, 82.69) --
	(235.94, 82.69) --
	(236.01, 82.69) --
	(236.01, 82.69) --
	(236.01, 82.69) --
	(236.08, 82.69) --
	(236.08, 82.69) --
	(236.08, 82.69) --
	(236.15, 82.69) --
	(236.15, 82.69) --
	(236.15, 82.69) --
	(236.22, 82.69) --
	(236.22, 82.69) --
	(236.22, 82.69) --
	(236.25, 82.69) --
	(236.25, 82.69) --
	(236.25, 82.69) --
	(236.29, 82.69) --
	(236.29, 82.69) --
	(236.29, 82.69) --
	(236.36, 80.81) --
	(236.36, 80.81) --
	(236.36, 80.81) --
	(236.43, 78.39) --
	(236.43, 78.39) --
	(236.43, 78.39) --
	(236.50, 76.91) --
	(236.50, 76.90) --
	(236.50, 76.90) --
	(236.57, 76.42) --
	(236.57, 76.42) --
	(236.57, 76.42) --
	(236.64, 76.85) --
	(236.64, 76.85) --
	(236.64, 76.85) --
	(236.71, 77.89) --
	(236.71, 77.89) --
	(236.71, 77.89) --
	(236.78, 79.20) --
	(236.78, 79.20) --
	(236.78, 79.20) --
	(236.85, 79.49) --
	(236.85, 79.49) --
	(236.85, 79.49) --
	(236.92, 78.40) --
	(236.92, 78.40) --
	(236.92, 78.40) --
	(236.99, 76.41) --
	(236.99, 76.41) --
	(236.99, 76.41) --
	(237.06, 74.44) --
	(237.06, 74.44) --
	(237.06, 74.44) --
	(237.13, 72.66) --
	(237.13, 72.66) --
	(237.13, 72.66) --
	(237.20, 71.41) --
	(237.20, 71.41) --
	(237.20, 71.41) --
	(237.27, 70.59) --
	(237.27, 70.59) --
	(237.27, 70.59) --
	(237.34, 70.08) --
	(237.34, 70.08) --
	(237.34, 70.08) --
	(237.41, 69.77) --
	(237.41, 69.77) --
	(237.41, 69.77) --
	(237.48, 69.61) --
	(237.48, 69.61) --
	(237.48, 69.61) --
	(237.55, 69.68) --
	(237.55, 69.68) --
	(237.55, 69.68) --
	(237.62, 69.98) --
	(237.62, 69.98) --
	(237.62, 69.98) --
	(237.69, 70.55) --
	(237.69, 70.55) --
	(237.69, 70.55) --
	(237.76, 71.47) --
	(237.76, 71.47) --
	(237.76, 71.47) --
	(237.83, 72.68) --
	(237.83, 72.68) --
	(237.83, 72.68) --
	(237.90, 73.50) --
	(237.90, 73.50) --
	(237.90, 73.50) --
	(237.97, 73.43) --
	(237.97, 73.43) --
	(237.97, 73.43) --
	(238.04, 72.65) --
	(238.04, 72.65) --
	(238.04, 72.65) --
	(238.11, 71.63) --
	(238.11, 71.63) --
	(238.11, 71.63) --
	(238.18, 70.62) --
	(238.18, 70.62) --
	(238.18, 70.62) --
	(238.25, 69.82) --
	(238.25, 69.82) --
	(238.25, 69.82) --
	(238.32, 69.24) --
	(238.32, 69.24) --
	(238.32, 69.24) --
	(238.39, 68.78) --
	(238.39, 68.78) --
	(238.39, 68.78) --
	(238.46, 68.53) --
	(238.46, 68.53) --
	(238.46, 68.53) --
	(238.53, 68.32) --
	(238.53, 68.32) --
	(238.53, 68.32) --
	(238.60, 68.23) --
	(238.60, 68.23) --
	(238.60, 68.23) --
	(238.67, 68.25) --
	(238.67, 68.25) --
	(238.67, 68.25) --
	(238.74, 68.38) --
	(238.74, 68.38) --
	(238.74, 68.38) --
	(238.74, 68.38) --
	(238.74, 68.38) --
	(238.74, 68.38) --
	(238.81, 68.72) --
	(238.81, 68.72) --
	(238.81, 68.72) --
	(238.88, 69.26) --
	(238.88, 69.26) --
	(238.88, 69.26) --
	(238.95, 69.46) --
	(238.95, 69.46) --
	(238.95, 69.46) --
	(239.02, 69.21) --
	(239.02, 69.21) --
	(239.02, 69.21) --
	(239.09, 68.82) --
	(239.09, 68.82) --
	(239.09, 68.82) --
	(239.16, 68.38) --
	(239.16, 68.38) --
	(239.16, 68.38) --
	(239.23, 67.99) --
	(239.23, 67.99) --
	(239.23, 67.99) --
	(239.30, 67.70) --
	(239.30, 67.70) --
	(239.30, 67.70) --
	(239.37, 67.50) --
	(239.37, 67.50) --
	(239.37, 67.50) --
	(239.44, 67.34) --
	(239.44, 67.34) --
	(239.44, 67.34) --
	(239.51, 67.24) --
	(239.51, 67.24) --
	(239.51, 67.24) --
	(239.58, 67.15) --
	(239.58, 67.15) --
	(239.58, 67.15) --
	(239.65, 67.09) --
	(239.65, 67.09) --
	(239.65, 67.09) --
	(239.72, 67.03) --
	(239.72, 67.03) --
	(239.72, 67.03) --
	(239.79, 66.98) --
	(239.79, 66.98) --
	(239.79, 66.98) --
	(239.86, 66.97) --
	(239.86, 66.97) --
	(239.86, 66.97) --
	(239.93, 66.92) --
	(239.93, 66.92) --
	(239.93, 66.92) --
	(240.00, 66.89) --
	(240.00, 66.89) --
	(240.00, 66.89) --
	(240.07, 66.86) --
	(240.07, 66.86) --
	(240.07, 66.86) --
	(240.14, 66.83) --
	(240.14, 66.83) --
	(240.14, 66.83) --
	(240.21, 66.81) --
	(240.21, 66.81) --
	(240.21, 66.81) --
	(240.28, 66.78) --
	(240.28, 66.78) --
	(240.28, 66.78) --
	(240.35, 66.76) --
	(240.35, 66.76) --
	(240.35, 66.76) --
	(240.42, 66.76) --
	(240.42, 66.76) --
	(240.42, 66.76) --
	(240.49, 66.74) --
	(240.49, 66.74) --
	(240.49, 66.74) --
	(240.56, 66.73) --
	(240.56, 66.73) --
	(240.56, 66.73) --
	(240.56, 66.73) --
	(240.56, 66.73) --
	(240.56, 66.73) --
	(240.63, 66.72) --
	(240.63, 66.72) --
	(240.63, 66.72) --
	(240.70, 66.73) --
	(240.70, 66.73) --
	(240.70, 66.73) --
	(240.77, 66.73) --
	(240.77, 66.73) --
	(240.77, 66.73) --
	(240.84, 66.71) --
	(240.84, 66.71) --
	(240.84, 66.71) --
	(240.91, 66.70) --
	(240.91, 66.70) --
	(240.91, 66.70) --
	(240.98, 66.71) --
	(240.98, 66.71) --
	(240.98, 66.71) --
	(241.05, 66.68) --
	(241.05, 66.68) --
	(241.05, 66.68) --
	(241.12, 66.69) --
	(241.12, 66.69) --
	(241.12, 66.69) --
	(241.19, 66.68) --
	(241.19, 66.68) --
	(241.19, 66.68) --
	(241.26, 66.69) --
	(241.26, 66.69) --
	(241.26, 66.69) --
	(241.33, 66.68) --
	(241.33, 66.68) --
	(241.33, 66.68) --
	(241.40, 66.68) --
	(241.40, 66.68) --
	(241.40, 66.68) --
	(241.47, 66.68) --
	(241.47, 66.68) --
	(241.47, 66.68) --
	(241.54, 66.68) --
	(241.54, 66.68) --
	(241.54, 66.68) --
	(241.61, 66.69) --
	(241.61, 66.69) --
	(241.61, 66.69) --
	(241.68, 66.70) --
	(241.68, 66.70) --
	(241.68, 66.70) --
	(241.75, 66.69) --
	(241.75, 66.69) --
	(241.75, 66.69) --
	(241.82, 66.69) --
	(241.82, 66.69) --
	(241.82, 66.69) --
	(241.89, 66.68) --
	(241.89, 66.68) --
	(241.89, 66.68) --
	(241.96, 66.69) --
	(241.96, 66.69) --
	(241.96, 66.69) --
	(242.03, 66.68) --
	(242.03, 66.68) --
	(242.03, 66.68) --
	(242.09, 66.68) --
	(242.09, 66.68) --
	(242.09, 66.68) --
	(242.16, 66.66) --
	(242.16, 66.66) --
	(242.16, 66.66) --
	(242.23, 66.68) --
	(242.23, 66.68) --
	(242.23, 66.68) --
	(242.30, 66.68) --
	(242.30, 66.68) --
	(242.30, 66.68) --
	(242.37, 66.67) --
	(242.37, 66.67) --
	(242.37, 66.67) --
	(242.44, 66.67) --
	(242.44, 66.67) --
	(242.44, 66.67) --
	(242.51, 66.67) --
	(242.51, 66.67) --
	(242.51, 66.67) --
	(242.58, 66.69) --
	(242.58, 66.69) --
	(242.58, 66.69) --
	(242.65, 66.72) --
	(242.65, 66.72) --
	(242.65, 66.72) --
	(242.72, 66.74) --
	(242.72, 66.74) --
	(242.72, 66.74) --
	(242.79, 66.77) --
	(242.79, 66.77) --
	(242.79, 66.77) --
	(242.86, 66.80) --
	(242.86, 66.80) --
	(242.86, 66.80) --
	(242.93, 66.80) --
	(242.93, 66.80) --
	(242.93, 66.80) --
	(243.00, 66.81) --
	(243.00, 66.81) --
	(243.00, 66.81) --
	(243.07, 66.82) --
	(243.07, 66.82) --
	(243.07, 66.82) --
	(243.14, 66.82) --
	(243.14, 66.82) --
	(243.14, 66.82) --
	(243.21, 66.83) --
	(243.21, 66.83) --
	(243.21, 66.83) --
	(243.28, 66.83) --
	(243.28, 66.83) --
	(243.28, 66.83) --
	(243.35, 66.84) --
	(243.35, 66.84) --
	(243.35, 66.84) --
	(243.42, 66.84) --
	(243.42, 66.84) --
	(243.42, 66.84) --
	(243.49, 66.86) --
	(243.49, 66.86) --
	(243.49, 66.86) --
	(243.56, 66.87) --
	(243.56, 66.87) --
	(243.56, 66.87) --
	(243.63, 66.89) --
	(243.63, 66.89) --
	(243.63, 66.89) --
	(243.70, 66.91) --
	(243.70, 66.91) --
	(243.70, 66.91) --
	(243.77, 66.92) --
	(243.77, 66.92) --
	(243.77, 66.92) --
	(243.84, 66.94) --
	(243.84, 66.94) --
	(243.84, 66.94) --
	(243.90, 66.96) --
	(243.91, 66.96) --
	(243.91, 66.96) --
	(243.98, 66.98) --
	(243.98, 66.98) --
	(243.98, 66.98) --
	(244.05, 67.00) --
	(244.05, 67.00) --
	(244.05, 67.00) --
	(244.11, 67.00) --
	(244.11, 67.00) --
	(244.11, 67.00) --
	(244.18, 67.04) --
	(244.18, 67.04) --
	(244.18, 67.04) --
	(244.25, 67.08) --
	(244.25, 67.08) --
	(244.25, 67.08) --
	(244.32, 67.11) --
	(244.32, 67.11) --
	(244.32, 67.11) --
	(244.39, 67.13) --
	(244.39, 67.13) --
	(244.39, 67.13) --
	(244.46, 67.16) --
	(244.46, 67.16) --
	(244.46, 67.16) --
	(244.53, 67.18) --
	(244.53, 67.18) --
	(244.53, 67.18) --
	(244.60, 67.20) --
	(244.60, 67.20) --
	(244.60, 67.20) --
	(244.67, 67.22) --
	(244.67, 67.22) --
	(244.67, 67.22) --
	(244.74, 67.25) --
	(244.74, 67.25) --
	(244.74, 67.25) --
	(244.81, 67.30) --
	(244.81, 67.30) --
	(244.81, 67.30) --
	(244.88, 67.37) --
	(244.88, 67.37) --
	(244.88, 67.37) --
	(244.95, 67.48) --
	(244.95, 67.48) --
	(244.95, 67.48) --
	(245.02, 67.62) --
	(245.02, 67.62) --
	(245.02, 67.62) --
	(245.09, 67.84) --
	(245.09, 67.84) --
	(245.09, 67.84) --
	(245.16, 68.20) --
	(245.16, 68.20) --
	(245.16, 68.20) --
	(245.23, 68.74) --
	(245.23, 68.74) --
	(245.23, 68.74) --
	(245.30, 69.71) --
	(245.30, 69.71) --
	(245.30, 69.71) --
	(245.37, 71.46) --
	(245.37, 71.46) --
	(245.37, 71.46) --
	(245.44, 73.83) --
	(245.44, 73.83) --
	(245.44, 73.83) --
	(245.51, 75.56) --
	(245.51, 75.56) --
	(245.51, 75.56) --
	(245.57, 76.33) --
	(245.57, 76.33) --
	(245.57, 76.33) --
	(245.64, 76.13) --
	(245.64, 76.13) --
	(245.64, 76.13) --
	(245.71, 75.37) --
	(245.71, 75.37) --
	(245.71, 75.37) --
	(245.78, 74.34) --
	(245.78, 74.34) --
	(245.78, 74.34) --
	(245.85, 73.11) --
	(245.85, 73.11) --
	(245.85, 73.11) --
	(245.92, 71.87) --
	(245.92, 71.87) --
	(245.92, 71.87) --
	(245.99, 70.85) --
	(245.99, 70.85) --
	(245.99, 70.85) --
	(246.06, 70.08) --
	(246.06, 70.08) --
	(246.06, 70.08) --
	(246.13, 69.51) --
	(246.13, 69.51) --
	(246.13, 69.51) --
	(246.20, 69.07) --
	(246.20, 69.07) --
	(246.20, 69.07) --
	(246.27, 68.73) --
	(246.27, 68.73) --
	(246.27, 68.73) --
	(246.34, 68.47) --
	(246.34, 68.47) --
	(246.34, 68.47) --
	(246.41, 68.20) --
	(246.41, 68.20) --
	(246.41, 68.20) --
	(246.48, 67.99) --
	(246.48, 67.99) --
	(246.48, 67.99) --
	(246.55, 67.83) --
	(246.55, 67.83) --
	(246.55, 67.83) --
	(246.61, 67.67) --
	(246.61, 67.67) --
	(246.61, 67.67) --
	(246.68, 67.51) --
	(246.68, 67.51) --
	(246.68, 67.51) --
	(246.75, 67.43) --
	(246.75, 67.43) --
	(246.75, 67.43) --
	(246.82, 67.36) --
	(246.82, 67.36) --
	(246.82, 67.36) --
	(246.89, 67.34) --
	(246.89, 67.34) --
	(246.89, 67.34) --
	(246.96, 67.32) --
	(246.96, 67.32) --
	(246.96, 67.32) --
	(247.03, 67.33) --
	(247.03, 67.33) --
	(247.03, 67.33) --
	(247.10, 67.31) --
	(247.10, 67.31) --
	(247.10, 67.31) --
	(247.17, 67.34) --
	(247.17, 67.34) --
	(247.17, 67.34) --
	(247.24, 67.41) --
	(247.24, 67.41) --
	(247.24, 67.41) --
	(247.31, 67.54) --
	(247.31, 67.54) --
	(247.31, 67.54) --
	(247.38, 67.72) --
	(247.38, 67.72) --
	(247.38, 67.72) --
	(247.45, 68.03) --
	(247.45, 68.03) --
	(247.45, 68.03) --
	(247.52, 68.64) --
	(247.52, 68.64) --
	(247.52, 68.64) --
	(247.59, 69.43) --
	(247.59, 69.43) --
	(247.59, 69.43) --
	(247.65, 70.01) --
	(247.65, 70.01) --
	(247.65, 70.01) --
	(247.72, 70.01) --
	(247.72, 70.01) --
	(247.72, 70.01) --
	(247.79, 69.78) --
	(247.79, 69.78) --
	(247.79, 69.78) --
	(247.86, 69.39) --
	(247.86, 69.39) --
	(247.86, 69.39) --
	(247.93, 68.97) --
	(247.93, 68.97) --
	(247.93, 68.97) --
	(248.00, 68.51) --
	(248.00, 68.51) --
	(248.00, 68.51) --
	(248.07, 68.12) --
	(248.07, 68.12) --
	(248.07, 68.12) --
	(248.14, 67.83) --
	(248.14, 67.83) --
	(248.14, 67.83) --
	(248.21, 67.61) --
	(248.21, 67.61) --
	(248.21, 67.61) --
	(248.28, 67.46) --
	(248.28, 67.46) --
	(248.28, 67.46) --
	(248.35, 67.34) --
	(248.35, 67.34) --
	(248.35, 67.34) --
	(248.42, 67.24) --
	(248.42, 67.24) --
	(248.42, 67.24) --
	(248.49, 67.13) --
	(248.49, 67.13) --
	(248.49, 67.13) --
	(248.56, 67.04) --
	(248.56, 67.04) --
	(248.56, 67.04) --
	(248.63, 66.97) --
	(248.63, 66.97) --
	(248.63, 66.97) --
	(248.69, 66.93) --
	(248.69, 66.93) --
	(248.69, 66.93) --
	(248.76, 66.94) --
	(248.76, 66.94) --
	(248.76, 66.94) --
	(248.83, 66.97) --
	(248.83, 66.97) --
	(248.83, 66.97) --
	(248.90, 67.09) --
	(248.90, 67.09) --
	(248.90, 67.09) --
	(248.97, 67.38) --
	(248.97, 67.38) --
	(248.97, 67.38) --
	(249.04, 68.13) --
	(249.04, 68.13) --
	(249.04, 68.13) --
	(249.11, 69.21) --
	(249.11, 69.21) --
	(249.11, 69.21) --
	(249.18, 69.88) --
	(249.18, 69.88) --
	(249.18, 69.88) --
	(249.25, 69.78) --
	(249.25, 69.78) --
	(249.25, 69.78) --
	(249.32, 69.46) --
	(249.32, 69.46) --
	(249.32, 69.46) --
	(249.39, 69.04) --
	(249.39, 69.04) --
	(249.39, 69.04) --
	(249.46, 68.55) --
	(249.46, 68.55) --
	(249.46, 68.55) --
	(249.52, 68.01) --
	(249.52, 68.01) --
	(249.52, 68.01) --
	(249.59, 67.58) --
	(249.59, 67.58) --
	(249.59, 67.58) --
	(249.66, 67.25) --
	(249.66, 67.25) --
	(249.66, 67.25) --
	(249.73, 67.01) --
	(249.73, 67.01) --
	(249.73, 67.01) --
	(249.80, 66.84) --
	(249.80, 66.84) --
	(249.80, 66.84) --
	(249.87, 66.73) --
	(249.87, 66.73) --
	(249.87, 66.73) --
	(249.94, 66.65) --
	(249.94, 66.65) --
	(249.94, 66.65) --
	(250.01, 66.59) --
	(250.01, 66.59) --
	(250.01, 66.59) --
	(250.08, 66.55) --
	(250.08, 66.55) --
	(250.08, 66.55) --
	(250.15, 66.53) --
	(250.15, 66.53) --
	(250.15, 66.53) --
	(250.22, 66.52) --
	(250.22, 66.52) --
	(250.22, 66.52) --
	(250.28, 66.53) --
	(250.28, 66.53) --
	(250.28, 66.53) --
	(250.35, 66.54) --
	(250.35, 66.54) --
	(250.35, 66.54) --
	(250.42, 66.56) --
	(250.42, 66.56) --
	(250.42, 66.56) --
	(250.49, 66.63) --
	(250.49, 66.63) --
	(250.49, 66.63) --
	(250.56, 66.69) --
	(250.56, 66.69) --
	(250.56, 66.69) --
	(250.63, 66.79) --
	(250.63, 66.79) --
	(250.63, 66.79) --
	(250.70, 66.99) --
	(250.70, 66.99) --
	(250.70, 66.99) --
	(250.77, 67.30) --
	(250.77, 67.30) --
	(250.77, 67.30) --
	(250.84, 67.77) --
	(250.84, 67.77) --
	(250.84, 67.77) --
	(250.91, 68.49) --
	(250.91, 68.49) --
	(250.91, 68.49) --
	(250.97, 69.35) --
	(250.97, 69.35) --
	(250.97, 69.35) --
	(251.04, 69.90) --
	(251.04, 69.90) --
	(251.04, 69.90) --
	(251.11, 70.03) --
	(251.11, 70.03) --
	(251.11, 70.03) --
	(251.18, 69.84) --
	(251.18, 69.84) --
	(251.18, 69.84) --
	(251.25, 69.56) --
	(251.25, 69.56) --
	(251.25, 69.56) --
	(251.32, 69.11) --
	(251.32, 69.11) --
	(251.32, 69.11) --
	(251.39, 68.61) --
	(251.39, 68.61) --
	(251.39, 68.61) --
	(251.46, 68.14) --
	(251.46, 68.14) --
	(251.46, 68.14) --
	(251.53, 67.75) --
	(251.53, 67.75) --
	(251.53, 67.75) --
	(251.60, 67.45) --
	(251.60, 67.45) --
	(251.60, 67.45) --
	(251.66, 67.25) --
	(251.66, 67.25) --
	(251.66, 67.25) --
	(251.73, 67.10) --
	(251.73, 67.10) --
	(251.73, 67.10) --
	(251.80, 66.99) --
	(251.80, 66.99) --
	(251.80, 66.99) --
	(251.87, 66.90) --
	(251.87, 66.90) --
	(251.87, 66.90) --
	(251.94, 66.80) --
	(251.94, 66.80) --
	(251.94, 66.80) --
	(252.01, 66.73) --
	(252.01, 66.73) --
	(252.01, 66.73) --
	(252.08, 66.66) --
	(252.08, 66.66) --
	(252.08, 66.66) --
	(252.15, 66.58) --
	(252.15, 66.58) --
	(252.15, 66.58) --
	(252.22, 66.51) --
	(252.22, 66.51) --
	(252.22, 66.51) --
	(252.29, 66.47) --
	(252.29, 66.47) --
	(252.29, 66.47) --
	(252.35, 66.41) --
	(252.35, 66.41) --
	(252.35, 66.41) --
	(252.42, 66.36) --
	(252.42, 66.36) --
	(252.42, 66.36) --
	(252.49, 66.33) --
	(252.49, 66.33) --
	(252.49, 66.33) --
	(252.56, 66.30) --
	(252.56, 66.30) --
	(252.56, 66.30) --
	(252.63, 66.27) --
	(252.63, 66.27) --
	(252.63, 66.27) --
	(252.70, 66.24) --
	(252.70, 66.24) --
	(252.70, 66.24) --
	(252.77, 66.21) --
	(252.77, 66.21) --
	(252.77, 66.21) --
	(252.84, 66.20) --
	(252.84, 66.20) --
	(252.84, 66.20) --
	(252.91, 66.19) --
	(252.91, 66.19) --
	(252.91, 66.19) --
	(252.97, 66.17) --
	(252.97, 66.17) --
	(252.97, 66.17) --
	(253.04, 66.16) --
	(253.04, 66.16) --
	(253.04, 66.16) --
	(253.11, 66.16) --
	(253.11, 66.16) --
	(253.11, 66.16) --
	(253.18, 66.16) --
	(253.18, 66.16) --
	(253.18, 66.16) --
	(253.25, 66.16) --
	(253.25, 66.16) --
	(253.25, 66.16) --
	(253.32, 66.17) --
	(253.32, 66.17) --
	(253.32, 66.17) --
	(253.39, 66.17) --
	(253.39, 66.17) --
	(253.39, 66.17) --
	(253.46, 66.17) --
	(253.46, 66.17) --
	(253.46, 66.17) --
	(253.53, 66.18) --
	(253.53, 66.18) --
	(253.53, 66.18) --
	(253.59, 66.17) --
	(253.59, 66.17) --
	(253.59, 66.17) --
	(253.66, 66.17) --
	(253.66, 66.17) --
	(253.66, 66.17) --
	(253.73, 66.15) --
	(253.73, 66.15) --
	(253.73, 66.15) --
	(253.80, 66.14) --
	(253.80, 66.14) --
	(253.80, 66.14) --
	(253.87, 66.13) --
	(253.87, 66.13) --
	(253.87, 66.13) --
	(253.94, 66.13) --
	(253.94, 66.13) --
	(253.94, 66.13) --
	(254.01, 66.12) --
	(254.01, 66.12) --
	(254.01, 66.12) --
	(254.08, 66.13) --
	(254.08, 66.13) --
	(254.08, 66.13) --
	(254.14, 66.12) --
	(254.14, 66.12) --
	(254.14, 66.12) --
	(254.21, 66.14) --
	(254.21, 66.14) --
	(254.21, 66.14) --
	(254.28, 66.15) --
	(254.28, 66.15) --
	(254.28, 66.15) --
	(254.35, 66.19) --
	(254.35, 66.19) --
	(254.35, 66.19) --
	(254.42, 66.24) --
	(254.42, 66.24) --
	(254.42, 66.24) --
	(254.49, 66.30) --
	(254.49, 66.30) --
	(254.49, 66.30) --
	(254.56, 66.35) --
	(254.56, 66.35) --
	(254.56, 66.35) --
	(254.63, 66.36) --
	(254.63, 66.36) --
	(254.63, 66.36) --
	(254.70, 66.36) --
	(254.70, 66.36) --
	(254.70, 66.36) --
	(254.76, 66.34) --
	(254.76, 66.34) --
	(254.76, 66.34) --
	(254.83, 66.31) --
	(254.83, 66.31) --
	(254.83, 66.31) --
	(254.90, 66.28) --
	(254.90, 66.28) --
	(254.90, 66.28) --
	(254.97, 66.25) --
	(254.97, 66.25) --
	(254.97, 66.25) --
	(255.04, 66.22) --
	(255.04, 66.22) --
	(255.04, 66.22) --
	(255.11, 66.19) --
	(255.11, 66.19) --
	(255.11, 66.19) --
	(255.18, 66.16) --
	(255.18, 66.16) --
	(255.18, 66.16) --
	(255.24, 66.15) --
	(255.24, 66.15) --
	(255.24, 66.15) --
	(255.31, 66.12) --
	(255.31, 66.12) --
	(255.31, 66.12) --
	(255.38, 66.10) --
	(255.38, 66.10) --
	(255.38, 66.10) --
	(255.45, 66.09) --
	(255.45, 66.09) --
	(255.45, 66.09) --
	(255.52, 66.10) --
	(255.52, 66.10) --
	(255.52, 66.10) --
	(255.59, 66.09) --
	(255.59, 66.09) --
	(255.59, 66.09) --
	(255.66, 66.11) --
	(255.66, 66.11) --
	(255.66, 66.11) --
	(255.73, 66.11) --
	(255.73, 66.11) --
	(255.73, 66.11) --
	(255.79, 66.11) --
	(255.79, 66.11) --
	(255.79, 66.11) --
	(255.86, 66.11) --
	(255.86, 66.11) --
	(255.86, 66.11) --
	(255.93, 66.11) --
	(255.93, 66.11) --
	(255.93, 66.11) --
	(256.00, 66.11) --
	(256.00, 66.11) --
	(256.00, 66.11) --
	(256.07, 66.11) --
	(256.07, 66.11) --
	(256.07, 66.11) --
	(256.14, 66.10) --
	(256.14, 66.10) --
	(256.14, 66.10) --
	(256.20, 66.11) --
	(256.20, 66.11) --
	(256.20, 66.11) --
	(256.27, 66.12) --
	(256.27, 66.12) --
	(256.27, 66.12) --
	(256.34, 66.13) --
	(256.34, 66.13) --
	(256.34, 66.13) --
	(256.41, 66.16) --
	(256.41, 66.16) --
	(256.41, 66.16) --
	(256.48, 66.20) --
	(256.48, 66.20) --
	(256.48, 66.20) --
	(256.55, 66.23) --
	(256.55, 66.23) --
	(256.55, 66.23) --
	(256.62, 66.26) --
	(256.62, 66.26) --
	(256.62, 66.26) --
	(256.69, 66.29) --
	(256.69, 66.29) --
	(256.69, 66.29) --
	(256.75, 66.30) --
	(256.75, 66.30) --
	(256.75, 66.30) --
	(256.82, 66.32) --
	(256.82, 66.32) --
	(256.82, 66.32) --
	(256.89, 66.28) --
	(256.89, 66.28) --
	(256.89, 66.28) --
	(256.96, 66.25) --
	(256.96, 66.25) --
	(256.96, 66.25) --
	(257.03, 66.23) --
	(257.03, 66.23) --
	(257.03, 66.23) --
	(257.10, 66.19) --
	(257.10, 66.19) --
	(257.10, 66.19) --
	(257.17, 66.16) --
	(257.17, 66.16) --
	(257.17, 66.16) --
	(257.24, 66.12) --
	(257.24, 66.12) --
	(257.24, 66.12) --
	(257.30, 66.09) --
	(257.30, 66.09) --
	(257.30, 66.09) --
	(257.37, 66.06) --
	(257.37, 66.06) --
	(257.37, 66.06) --
	(257.44, 66.04) --
	(257.44, 66.04) --
	(257.44, 66.04) --
	(257.51, 66.03) --
	(257.51, 66.03) --
	(257.51, 66.03) --
	(257.58, 66.00) --
	(257.58, 66.00) --
	(257.58, 66.00) --
	(257.65, 65.98) --
	(257.65, 65.98) --
	(257.65, 65.98) --
	(257.71, 65.98) --
	(257.71, 65.98) --
	(257.71, 65.98) --
	(257.78, 65.96) --
	(257.78, 65.96) --
	(257.78, 65.96) --
	(257.85, 65.96) --
	(257.85, 65.96) --
	(257.85, 65.96) --
	(257.92, 65.95) --
	(257.92, 65.95) --
	(257.92, 65.95) --
	(257.99, 65.94) --
	(257.99, 65.94) --
	(257.99, 65.94) --
	(258.06, 65.93) --
	(258.06, 65.93) --
	(258.06, 65.93) --
	(258.12, 65.92) --
	(258.12, 65.92) --
	(258.12, 65.92) --
	(258.19, 65.93) --
	(258.19, 65.93) --
	(258.19, 65.93) --
	(258.26, 65.92) --
	(258.26, 65.92) --
	(258.26, 65.92) --
	(258.33, 65.91) --
	(258.33, 65.91) --
	(258.33, 65.91) --
	(258.40, 65.91) --
	(258.40, 65.91) --
	(258.40, 65.91) --
	(258.47, 65.91) --
	(258.47, 65.91) --
	(258.47, 65.91) --
	(258.54, 65.90) --
	(258.54, 65.90) --
	(258.54, 65.90) --
	(258.60, 65.90) --
	(258.60, 65.90) --
	(258.60, 65.90) --
	(258.67, 65.90) --
	(258.67, 65.90) --
	(258.67, 65.90) --
	(258.74, 65.91) --
	(258.74, 65.91) --
	(258.74, 65.91) --
	(258.81, 65.90) --
	(258.81, 65.90) --
	(258.81, 65.90) --
	(258.88, 65.90) --
	(258.88, 65.90) --
	(258.88, 65.90) --
	(258.95, 65.91) --
	(258.95, 65.91) --
	(258.95, 65.91) --
	(259.01, 65.91) --
	(259.01, 65.91) --
	(259.01, 65.91) --
	(259.08, 65.91) --
	(259.08, 65.91) --
	(259.08, 65.91) --
	(259.15, 65.91) --
	(259.15, 65.91) --
	(259.15, 65.91) --
	(259.22, 65.93) --
	(259.22, 65.93) --
	(259.22, 65.93) --
	(259.29, 65.95) --
	(259.29, 65.95) --
	(259.29, 65.95) --
	(259.36, 65.95) --
	(259.36, 65.95) --
	(259.36, 65.95) --
	(259.42, 65.97) --
	(259.42, 65.97) --
	(259.42, 65.97) --
	(259.49, 65.98) --
	(259.49, 65.98) --
	(259.49, 65.98) --
	(259.56, 65.99) --
	(259.56, 65.99) --
	(259.56, 65.99) --
	(259.63, 66.00) --
	(259.63, 66.00) --
	(259.63, 66.00) --
	(259.70, 66.01) --
	(259.70, 66.01) --
	(259.70, 66.01) --
	(259.77, 66.01) --
	(259.77, 66.01) --
	(259.77, 66.01) --
	(259.83, 66.02) --
	(259.83, 66.02) --
	(259.83, 66.02) --
	(259.90, 66.01) --
	(259.90, 66.01) --
	(259.90, 66.01) --
	(259.97, 66.01) --
	(259.97, 66.01) --
	(259.97, 66.01) --
	(260.04, 66.02) --
	(260.04, 66.02) --
	(260.04, 66.02) --
	(260.11, 66.01) --
	(260.11, 66.01) --
	(260.11, 66.01) --
	(260.18, 66.01) --
	(260.18, 66.01) --
	(260.18, 66.01) --
	(260.25, 66.01) --
	(260.25, 66.01) --
	(260.25, 66.01) --
	(260.31, 65.99) --
	(260.31, 65.99) --
	(260.31, 65.99) --
	(260.38, 65.99) --
	(260.38, 65.99) --
	(260.38, 65.99) --
	(260.45, 65.99) --
	(260.45, 65.99) --
	(260.45, 65.99) --
	(260.52, 65.98) --
	(260.52, 65.98) --
	(260.52, 65.98) --
	(260.59, 65.97) --
	(260.59, 65.97) --
	(260.59, 65.97) --
	(260.66, 65.97) --
	(260.66, 65.97) --
	(260.66, 65.97) --
	(260.72, 65.98) --
	(260.72, 65.98) --
	(260.72, 65.98) --
	(260.79, 65.99) --
	(260.79, 65.99) --
	(260.79, 65.99) --
	(260.86, 66.01) --
	(260.86, 66.01) --
	(260.86, 66.01) --
	(260.93, 66.04) --
	(260.93, 66.04) --
	(260.93, 66.04) --
	(261.00, 66.09) --
	(261.00, 66.09) --
	(261.00, 66.09) --
	(261.06, 66.17) --
	(261.06, 66.17) --
	(261.06, 66.17) --
	(261.13, 66.30) --
	(261.13, 66.30) --
	(261.13, 66.30) --
	(261.20, 66.51) --
	(261.20, 66.51) --
	(261.20, 66.51) --
	(261.27, 66.70) --
	(261.27, 66.70) --
	(261.27, 66.70) --
	(261.34, 66.82) --
	(261.34, 66.82) --
	(261.34, 66.82) --
	(261.40, 66.80) --
	(261.40, 66.80) --
	(261.40, 66.80) --
	(261.47, 66.74) --
	(261.47, 66.74) --
	(261.47, 66.74) --
	(261.54, 66.65) --
	(261.54, 66.65) --
	(261.54, 66.65) --
	(261.61, 66.53) --
	(261.61, 66.53) --
	(261.61, 66.53) --
	(261.68, 66.40) --
	(261.68, 66.40) --
	(261.68, 66.40) --
	(261.75, 66.27) --
	(261.75, 66.27) --
	(261.75, 66.27) --
	(261.81, 66.16) --
	(261.81, 66.16) --
	(261.81, 66.16) --
	(261.88, 66.07) --
	(261.88, 66.07) --
	(261.88, 66.07) --
	(261.95, 66.01) --
	(261.95, 66.01) --
	(261.95, 66.01) --
	(262.02, 65.97) --
	(262.02, 65.97) --
	(262.02, 65.97) --
	(262.09, 65.94) --
	(262.09, 65.94) --
	(262.09, 65.94) --
	(262.16, 65.91) --
	(262.16, 65.91) --
	(262.16, 65.91) --
	(262.22, 65.91) --
	(262.22, 65.91) --
	(262.22, 65.91) --
	(262.29, 65.91) --
	(262.29, 65.91) --
	(262.29, 65.91) --
	(262.36, 65.90) --
	(262.36, 65.90) --
	(262.36, 65.90) --
	(262.43, 65.89) --
	(262.43, 65.89) --
	(262.43, 65.89) --
	(262.50, 65.89) --
	(262.50, 65.89) --
	(262.50, 65.89) --
	(262.56, 65.91) --
	(262.56, 65.91) --
	(262.56, 65.91) --
	(262.63, 65.90) --
	(262.63, 65.90) --
	(262.63, 65.90) --
	(262.70, 65.92) --
	(262.70, 65.92) --
	(262.70, 65.92) --
	(262.77, 65.94) --
	(262.77, 65.94) --
	(262.77, 65.94) --
	(262.84, 65.96) --
	(262.84, 65.96) --
	(262.84, 65.96) --
	(262.90, 66.00) --
	(262.90, 66.00) --
	(262.90, 66.00) --
	(262.97, 66.03) --
	(262.97, 66.03) --
	(262.97, 66.03) --
	(263.04, 66.11) --
	(263.04, 66.11) --
	(263.04, 66.11) --
	(263.11, 66.24) --
	(263.11, 66.24) --
	(263.11, 66.24) --
	(263.18, 66.40) --
	(263.18, 66.40) --
	(263.18, 66.40) --
	(263.25, 66.56) --
	(263.25, 66.56) --
	(263.25, 66.56) --
	(263.31, 66.69) --
	(263.31, 66.69) --
	(263.31, 66.69) --
	(263.38, 66.73) --
	(263.38, 66.73) --
	(263.38, 66.73) --
	(263.45, 66.72) --
	(263.45, 66.72) --
	(263.45, 66.72) --
	(263.52, 66.67) --
	(263.52, 66.67) --
	(263.52, 66.67) --
	(263.58, 66.59) --
	(263.58, 66.59) --
	(263.58, 66.59) --
	(263.65, 66.49) --
	(263.65, 66.49) --
	(263.65, 66.49) --
	(263.72, 66.37) --
	(263.72, 66.37) --
	(263.72, 66.37) --
	(263.79, 66.27) --
	(263.79, 66.27) --
	(263.79, 66.27) --
	(263.86, 66.17) --
	(263.86, 66.17) --
	(263.86, 66.17) --
	(263.93, 66.11) --
	(263.93, 66.11) --
	(263.93, 66.11) --
	(263.99, 66.05) --
	(263.99, 66.05) --
	(263.99, 66.05) --
	(264.06, 66.00) --
	(264.06, 66.00) --
	(264.06, 66.00) --
	(264.13, 65.97) --
	(264.13, 65.97) --
	(264.13, 65.97) --
	(264.20, 65.95) --
	(264.20, 65.95) --
	(264.20, 65.95) --
	(264.26, 65.92) --
	(264.26, 65.92) --
	(264.26, 65.92) --
	(264.33, 65.91) --
	(264.33, 65.91) --
	(264.33, 65.91) --
	(264.40, 65.89) --
	(264.40, 65.89) --
	(264.40, 65.89) --
	(264.47, 65.87) --
	(264.47, 65.87) --
	(264.47, 65.87) --
	(264.54, 65.87) --
	(264.54, 65.87) --
	(264.54, 65.87) --
	(264.61, 65.86) --
	(264.61, 65.86) --
	(264.61, 65.86) --
	(264.67, 65.86) --
	(264.67, 65.86) --
	(264.67, 65.86) --
	(264.74, 65.85) --
	(264.74, 65.85) --
	(264.74, 65.85) --
	(264.81, 65.85) --
	(264.81, 65.85) --
	(264.81, 65.85) --
	(264.88, 65.85) --
	(264.88, 65.85) --
	(264.88, 65.85) --
	(264.94, 65.84) --
	(264.94, 65.84) --
	(264.94, 65.84) --
	(265.01, 65.84) --
	(265.01, 65.84) --
	(265.01, 65.84) --
	(265.08, 65.84) --
	(265.08, 65.84) --
	(265.08, 65.84) --
	(265.15, 65.84) --
	(265.15, 65.84) --
	(265.15, 65.84) --
	(265.22, 65.83) --
	(265.22, 65.83) --
	(265.22, 65.83) --
	(265.28, 65.83) --
	(265.28, 65.83) --
	(265.28, 65.83) --
	(265.35, 65.83) --
	(265.35, 65.83) --
	(265.35, 65.83) --
	(265.42, 65.82) --
	(265.42, 65.82) --
	(265.42, 65.82) --
	(265.49, 65.82) --
	(265.49, 65.82) --
	(265.49, 65.82) --
	(265.56, 65.82) --
	(265.56, 65.82) --
	(265.56, 65.82) --
	(265.62, 65.81) --
	(265.62, 65.81) --
	(265.62, 65.81) --
	(265.69, 65.82) --
	(265.69, 65.82) --
	(265.69, 65.82) --
	(265.76, 65.81) --
	(265.76, 65.81) --
	(265.76, 65.81) --
	(265.83, 65.81) --
	(265.83, 65.81) --
	(265.83, 65.81) --
	(265.90, 65.81) --
	(265.90, 65.81) --
	(265.90, 65.81) --
	(265.96, 65.81) --
	(265.96, 65.81) --
	(265.96, 65.81) --
	(266.03, 65.81) --
	(266.03, 65.81) --
	(266.03, 65.81) --
	(266.10, 65.83) --
	(266.10, 65.83) --
	(266.10, 65.83) --
	(266.17, 65.83) --
	(266.17, 65.83) --
	(266.17, 65.83) --
	(266.23, 65.83) --
	(266.23, 65.83) --
	(266.23, 65.83) --
	(266.30, 65.84) --
	(266.30, 65.84) --
	(266.30, 65.84) --
	(266.37, 65.85) --
	(266.37, 65.85) --
	(266.37, 65.85) --
	(266.44, 65.85) --
	(266.44, 65.85) --
	(266.44, 65.85) --
	(266.51, 65.85) --
	(266.51, 65.85) --
	(266.51, 65.85) --
	(266.57, 65.85) --
	(266.57, 65.85) --
	(266.57, 65.85) --
	(266.64, 65.86) --
	(266.64, 65.86) --
	(266.64, 65.86) --
	(266.71, 65.85) --
	(266.71, 65.85) --
	(266.71, 65.85) --
	(266.78, 65.84) --
	(266.78, 65.84) --
	(266.78, 65.84) --
	(266.84, 65.84) --
	(266.84, 65.84) --
	(266.84, 65.84) --
	(266.91, 65.82) --
	(266.91, 65.82) --
	(266.91, 65.82) --
	(266.98, 65.82) --
	(266.98, 65.82) --
	(266.98, 65.82) --
	(267.05, 65.81) --
	(267.05, 65.81) --
	(267.05, 65.81) --
	(267.11, 65.81) --
	(267.11, 65.81) --
	(267.11, 65.81) --
	(267.18, 65.80) --
	(267.18, 65.80) --
	(267.18, 65.80) --
	(267.25, 65.79) --
	(267.25, 65.79) --
	(267.25, 65.79) --
	(267.32, 65.79) --
	(267.32, 65.79) --
	(267.32, 65.79) --
	(267.39, 65.79) --
	(267.39, 65.79) --
	(267.39, 65.79) --
	(267.45, 65.79) --
	(267.45, 65.79) --
	(267.45, 65.79) --
	(267.52, 65.78) --
	(267.52, 65.78) --
	(267.52, 65.78) --
	(267.59, 65.79) --
	(267.59, 65.79) --
	(267.59, 65.79) --
	(267.66, 65.78) --
	(267.66, 65.78) --
	(267.66, 65.78) --
	(267.72, 65.78) --
	(267.72, 65.78) --
	(267.72, 65.78) --
	(267.79, 65.78) --
	(267.79, 65.78) --
	(267.79, 65.78) --
	(267.86, 65.77) --
	(267.86, 65.77) --
	(267.86, 65.77) --
	(267.93, 65.77) --
	(267.93, 65.77) --
	(267.93, 65.77) --
	(267.99, 65.76) --
	(267.99, 65.76) --
	(267.99, 65.76) --
	(268.06, 65.77) --
	(268.06, 65.77) --
	(268.06, 65.77) --
	(268.13, 65.76) --
	(268.13, 65.76) --
	(268.13, 65.76) --
	(268.20, 65.77) --
	(268.20, 65.77) --
	(268.20, 65.77) --
	(268.27, 65.76) --
	(268.27, 65.76) --
	(268.27, 65.76) --
	(268.33, 65.76) --
	(268.33, 65.76) --
	(268.33, 65.76) --
	(268.40, 65.76) --
	(268.40, 65.76) --
	(268.40, 65.76) --
	(268.47, 65.76) --
	(268.47, 65.76) --
	(268.47, 65.76) --
	(268.54, 65.76) --
	(268.54, 65.76) --
	(268.54, 65.76) --
	(268.60, 65.76) --
	(268.60, 65.76) --
	(268.60, 65.76) --
	(268.67, 65.76) --
	(268.67, 65.76) --
	(268.67, 65.76) --
	(268.74, 65.76) --
	(268.74, 65.76) --
	(268.74, 65.76) --
	(268.81, 65.76) --
	(268.81, 65.76) --
	(268.81, 65.76) --
	(268.87, 65.75) --
	(268.87, 65.75) --
	(268.87, 65.75) --
	(268.94, 65.76) --
	(268.94, 65.76) --
	(268.94, 65.76) --
	(269.01, 65.75) --
	(269.01, 65.75) --
	(269.01, 65.75) --
	(269.08, 65.76) --
	(269.08, 65.76) --
	(269.08, 65.76) --
	(269.14, 65.75) --
	(269.14, 65.75) --
	(269.14, 65.75) --
	(269.21, 65.76) --
	(269.21, 65.76) --
	(269.21, 65.76) --
	(269.28, 65.75) --
	(269.28, 65.75) --
	(269.28, 65.75) --
	(269.35, 65.75) --
	(269.35, 65.75) --
	(269.35, 65.75) --
	(269.42, 65.75) --
	(269.42, 65.75) --
	(269.42, 65.75) --
	(269.48, 65.75) --
	(269.48, 65.75) --
	(269.48, 65.75) --
	(269.55, 65.75) --
	(269.55, 65.75) --
	(269.55, 65.75) --
	(269.62, 65.75) --
	(269.62, 65.75) --
	(269.62, 65.75) --
	(269.69, 65.74) --
	(269.69, 65.74) --
	(269.69, 65.74) --
	(269.75, 65.75) --
	(269.75, 65.75) --
	(269.75, 65.75) --
	(269.82, 65.75) --
	(269.82, 65.75) --
	(269.82, 65.75) --
	(269.89, 65.75) --
	(269.89, 65.75) --
	(269.89, 65.75) --
	(269.95, 65.74) --
	(269.95, 65.74) --
	(269.95, 65.74) --
	(270.02, 65.75) --
	(270.02, 65.75) --
	(270.02, 65.75) --
	(270.09, 65.75) --
	(270.09, 65.75) --
	(270.09, 65.75) --
	(270.16, 65.74) --
	(270.16, 65.74) --
	(270.16, 65.74) --
	(270.22, 65.74) --
	(270.22, 65.74) --
	(270.22, 65.74) --
	(270.29, 65.75) --
	(270.29, 65.75) --
	(270.29, 65.75) --
	(270.36, 65.75) --
	(270.36, 65.75) --
	(270.36, 65.75) --
	(270.43, 65.75) --
	(270.43, 65.75) --
	(270.43, 65.75) --
	(270.49, 65.74) --
	(270.49, 65.74) --
	(270.49, 65.74) --
	(270.56, 65.74) --
	(270.56, 65.74) --
	(270.56, 65.74) --
	(270.63, 65.74) --
	(270.63, 65.74) --
	(270.63, 65.74) --
	(270.70, 65.74) --
	(270.70, 65.74) --
	(270.70, 65.74) --
	(270.76, 65.74) --
	(270.76, 65.74) --
	(270.76, 65.74) --
	(270.83, 65.74) --
	(270.83, 65.74) --
	(270.83, 65.74) --
	(270.90, 65.75) --
	(270.90, 65.75) --
	(270.90, 65.75) --
	(270.97, 65.74) --
	(270.97, 65.74) --
	(270.97, 65.74) --
	(271.03, 65.74) --
	(271.03, 65.74) --
	(271.03, 65.74) --
	(271.10, 65.74) --
	(271.10, 65.74) --
	(271.10, 65.74) --
	(271.17, 65.74) --
	(271.17, 65.74) --
	(271.17, 65.74) --
	(271.24, 65.74) --
	(271.24, 65.74) --
	(271.24, 65.74) --
	(271.30, 65.75) --
	(271.30, 65.75) --
	(271.30, 65.75) --
	(271.37, 65.73) --
	(271.37, 65.73) --
	(271.37, 65.73) --
	(271.44, 65.74) --
	(271.44, 65.74) --
	(271.44, 65.74) --
	(271.51, 65.74) --
	(271.51, 65.74) --
	(271.51, 65.74) --
	(271.57, 65.74) --
	(271.57, 65.74) --
	(271.57, 65.74) --
	(271.64, 65.74) --
	(271.64, 65.74) --
	(271.64, 65.74) --
	(271.71, 65.74) --
	(271.71, 65.74) --
	(271.71, 65.74) --
	(271.77, 65.74) --
	(271.77, 65.74) --
	(271.77, 65.74) --
	(271.84, 65.74) --
	(271.84, 65.74) --
	(271.84, 65.74) --
	(271.91, 65.73) --
	(271.91, 65.73) --
	(271.91, 65.73) --
	(271.98, 65.74) --
	(271.98, 65.74) --
	(271.98, 65.74) --
	(272.04, 65.74) --
	(272.04, 65.74) --
	(272.04, 65.74) --
	(272.11, 65.74) --
	(272.11, 65.74) --
	(272.11, 65.74) --
	(272.18, 65.74) --
	(272.18, 65.74) --
	(272.18, 65.74) --
	(272.25, 65.74) --
	(272.25, 65.74) --
	(272.25, 65.74) --
	(272.31, 65.74) --
	(272.31, 65.74) --
	(272.31, 65.74) --
	(272.38, 65.73) --
	(272.38, 65.73) --
	(272.38, 65.73) --
	(272.45, 65.74) --
	(272.45, 65.74) --
	(272.45, 65.74) --
	(272.52, 65.74) --
	(272.52, 65.74) --
	(272.52, 65.74) --
	(272.58, 65.73) --
	(272.58, 65.73) --
	(272.58, 65.73) --
	(272.65, 65.74) --
	(272.65, 65.74) --
	(272.65, 65.74) --
	(272.72, 65.74) --
	(272.72, 65.74) --
	(272.72, 65.74) --
	(272.79, 65.73) --
	(272.79, 65.73) --
	(272.79, 65.73) --
	(272.85, 65.74) --
	(272.85, 65.74) --
	(272.85, 65.74) --
	(272.92, 65.73) --
	(272.92, 65.73) --
	(272.92, 65.73) --
	(272.99, 65.74) --
	(272.99, 65.74) --
	(272.99, 65.74) --
	(273.05, 65.73) --
	(273.05, 65.73) --
	(273.05, 65.73) --
	(273.12, 65.74) --
	(273.12, 65.74) --
	(273.12, 65.74) --
	(273.19, 65.74) --
	(273.19, 65.74) --
	(273.19, 65.74) --
	(273.26, 65.74) --
	(273.26, 65.74) --
	(273.26, 65.74) --
	(273.32, 65.74) --
	(273.32, 65.74) --
	(273.32, 65.74) --
	(273.39, 65.74) --
	(273.39, 65.74) --
	(273.39, 65.74) --
	(273.46, 65.73) --
	(273.46, 65.73) --
	(273.46, 65.73) --
	(273.53, 65.74) --
	(273.53, 65.74) --
	(273.53, 65.74) --
	(273.59, 65.73) --
	(273.59, 65.73) --
	(273.59, 65.73) --
	(273.66, 65.74) --
	(273.66, 65.74) --
	(273.66, 65.74) --
	(273.73, 65.74) --
	(273.73, 65.74) --
	(273.73, 65.74) --
	(273.79, 65.73) --
	(273.79, 65.73) --
	(273.79, 65.73) --
	(273.86, 65.73) --
	(273.86, 65.73) --
	(273.86, 65.73) --
	(273.93, 65.73) --
	(273.93, 65.73) --
	(273.93, 65.73) --
	(274.00, 65.73) --
	(274.00, 65.73) --
	(274.00, 65.73) --
	(274.06, 65.73) --
	(274.06, 65.73) --
	(274.06, 65.73) --
	(274.13, 65.73) --
	(274.13, 65.73) --
	(274.13, 65.73) --
	(274.20, 65.73) --
	(274.20, 65.73) --
	(274.20, 65.73) --
	(274.26, 65.73) --
	(274.26, 65.73) --
	(274.26, 65.73) --
	(274.33, 65.72) --
	(274.33, 65.72) --
	(274.33, 65.72) --
	(274.40, 65.73) --
	(274.40, 65.73) --
	(274.40, 65.73) --
	(274.46, 65.73) --
	(274.46, 65.73) --
	(274.46, 65.73) --
	(274.53, 65.73) --
	(274.53, 65.73) --
	(274.53, 65.73) --
	(274.60, 65.72) --
	(274.60, 65.72) --
	(274.60, 65.72) --
	(274.67, 65.73) --
	(274.67, 65.73) --
	(274.67, 65.73) --
	(274.73, 65.72) --
	(274.73, 65.72) --
	(274.73, 65.72) --
	(274.80, 65.73) --
	(274.80, 65.73) --
	(274.80, 65.73) --
	(274.87, 65.73) --
	(274.87, 65.73) --
	(274.87, 65.73) --
	(274.93, 65.73) --
	(274.93, 65.73) --
	(274.93, 65.73) --
	(275.00, 65.73) --
	(275.00, 65.73) --
	(275.00, 65.73) --
	(275.07, 65.73) --
	(275.07, 65.73) --
	(275.07, 65.73) --
	(275.14, 65.73) --
	(275.14, 65.73) --
	(275.14, 65.73) --
	(275.20, 65.72) --
	(275.20, 65.72) --
	(275.20, 65.72) --
	(275.27, 65.72) --
	(275.27, 65.72) --
	(275.27, 65.72) --
	(275.34, 65.72) --
	(275.34, 65.72) --
	(275.34, 65.72) --
	(275.40, 65.72) --
	(275.40, 65.72) --
	(275.40, 65.72) --
	(275.47, 65.73) --
	(275.47, 65.73) --
	(275.47, 65.73) --
	(275.54, 65.72) --
	(275.54, 65.72) --
	(275.54, 65.72) --
	(275.60, 65.72) --
	(275.60, 65.72) --
	(275.61, 65.72) --
	(275.67, 65.72) --
	(275.67, 65.72) --
	(275.67, 65.72) --
	(275.74, 65.73) --
	(275.74, 65.73) --
	(275.74, 65.73) --
	(275.81, 65.73) --
	(275.81, 65.73) --
	(275.81, 65.73) --
	(275.87, 65.72) --
	(275.87, 65.72) --
	(275.87, 65.72) --
	(275.94, 65.72) --
	(275.94, 65.72) --
	(275.94, 65.72) --
	(276.01, 65.73) --
	(276.01, 65.73) --
	(276.01, 65.73) --
	(276.07, 65.72) --
	(276.07, 65.72) --
	(276.07, 65.72) --
	(276.14, 65.73) --
	(276.14, 65.73) --
	(276.14, 65.73) --
	(276.21, 65.72) --
	(276.21, 65.72) --
	(276.21, 65.72) --
	(276.28, 65.72) --
	(276.28, 65.72) --
	(276.28, 65.72) --
	(276.34, 65.73) --
	(276.34, 65.73) --
	(276.34, 65.73) --
	(276.41, 65.73) --
	(276.41, 65.73) --
	(276.41, 65.73) --
	(276.48, 65.72) --
	(276.48, 65.72) --
	(276.48, 65.72) --
	(276.54, 65.72) --
	(276.54, 65.72) --
	(276.54, 65.72) --
	(276.61, 65.72) --
	(276.61, 65.72) --
	(276.61, 65.72) --
	(276.68, 65.73) --
	(276.68, 65.73) --
	(276.68, 65.73) --
	(276.74, 65.72) --
	(276.74, 65.72) --
	(276.74, 65.72) --
	(276.81, 65.72) --
	(276.81, 65.72) --
	(276.81, 65.72) --
	(276.88, 65.72) --
	(276.88, 65.72) --
	(276.88, 65.72) --
	(276.94, 65.72) --
	(276.94, 65.72) --
	(276.94, 65.72) --
	(277.01, 65.72) --
	(277.01, 65.72) --
	(277.01, 65.72) --
	(277.08, 65.73) --
	(277.08, 65.73) --
	(277.08, 65.73) --
	(277.15, 65.73) --
	(277.15, 65.73) --
	(277.15, 65.73) --
	(277.21, 65.72) --
	(277.21, 65.72) --
	(277.21, 65.72) --
	(277.28, 65.72) --
	(277.28, 65.72) --
	(277.28, 65.72) --
	(277.35, 65.73) --
	(277.35, 65.73) --
	(277.35, 65.73) --
	(277.41, 65.73) --
	(277.41, 65.73) --
	(277.41, 65.73) --
	(277.48, 65.73) --
	(277.48, 65.73) --
	(277.48, 65.73) --
	(277.55, 65.72) --
	(277.55, 65.72) --
	(277.55, 65.72) --
	(277.61, 65.73) --
	(277.61, 65.73) --
	(277.61, 65.73) --
	(277.68, 65.72) --
	(277.68, 65.72) --
	(277.68, 65.72) --
	(277.75, 65.72) --
	(277.75, 65.72) --
	(277.75, 65.72) --
	(277.81, 65.72) --
	(277.81, 65.72) --
	(277.81, 65.72) --
	(277.88, 65.72) --
	(277.88, 65.72) --
	(277.88, 65.72) --
	(277.95, 65.72) --
	(277.95, 65.72) --
	(277.95, 65.72) --
	(278.01, 65.73) --
	(278.01, 65.73) --
	(278.01, 65.73) --
	(278.08, 65.72) --
	(278.08, 65.72) --
	(278.08, 65.72) --
	(278.15, 65.72) --
	(278.15, 65.72) --
	(278.15, 65.72) --
	(278.21, 65.72) --
	(278.21, 65.72) --
	(278.21, 65.72) --
	(278.28, 65.72) --
	(278.28, 65.72) --
	(278.28, 65.72) --
	(278.35, 65.73) --
	(278.35, 65.73) --
	(278.35, 65.73) --
	(278.41, 65.72) --
	(278.41, 65.72) --
	(278.41, 65.72) --
	(278.48, 65.72) --
	(278.48, 65.72) --
	(278.48, 65.72) --
	(278.55, 65.72) --
	(278.55, 65.72) --
	(278.55, 65.72) --
	(278.61, 65.72) --
	(278.61, 65.72) --
	(278.61, 65.72) --
	(278.68, 65.72) --
	(278.68, 65.72) --
	(278.68, 65.72) --
	(278.75, 65.72) --
	(278.75, 65.72) --
	(278.75, 65.72) --
	(278.82, 65.73) --
	(278.82, 65.73) --
	(278.82, 65.73) --
	(278.88, 65.72) --
	(278.88, 65.72) --
	(278.88, 65.72) --
	(278.95, 65.72) --
	(278.95, 65.72) --
	(278.95, 65.72) --
	(279.02, 65.72) --
	(279.02, 65.72) --
	(279.02, 65.72) --
	(279.08, 65.73) --
	(279.08, 65.73) --
	(279.08, 65.73) --
	(279.15, 65.72) --
	(279.15, 65.72) --
	(279.15, 65.72) --
	(279.22, 65.72) --
	(279.22, 65.72) --
	(279.22, 65.72) --
	(279.28, 65.72) --
	(279.28, 65.72) --
	(279.28, 65.72) --
	(279.35, 65.72) --
	(279.35, 65.72) --
	(279.35, 65.72) --
	(279.42, 65.72) --
	(279.42, 65.72) --
	(279.42, 65.72) --
	(279.48, 65.73) --
	(279.48, 65.73) --
	(279.48, 65.73) --
	(279.55, 65.73) --
	(279.55, 65.73) --
	(279.55, 65.73) --
	(279.62, 65.72) --
	(279.62, 65.72) --
	(279.62, 65.72) --
	(279.68, 65.72) --
	(279.68, 65.72) --
	(279.68, 65.72) --
	(279.75, 65.72) --
	(279.75, 65.72) --
	(279.75, 65.72) --
	(279.82, 65.73) --
	(279.82, 65.73) --
	(279.82, 65.73) --
	(279.88, 65.72) --
	(279.88, 65.72) --
	(279.88, 65.72) --
	(279.95, 65.72) --
	(279.95, 65.72) --
	(279.95, 65.72) --
	(280.02, 65.73) --
	(280.02, 65.73) --
	(280.02, 65.73) --
	(280.08, 65.72) --
	(280.08, 65.72) --
	(280.08, 65.72) --
	(280.15, 65.72) --
	(280.15, 65.72) --
	(280.15, 65.72) --
	(280.22, 65.73) --
	(280.22, 65.73) --
	(280.22, 65.73) --
	(280.28, 65.73) --
	(280.28, 65.73) --
	(280.28, 65.73) --
	(280.35, 65.72) --
	(280.35, 65.72) --
	(280.35, 65.72) --
	(280.42, 65.72) --
	(280.42, 65.72) --
	(280.42, 65.72) --
	(280.48, 65.72) --
	(280.48, 65.72);

\node[text=drawColor,anchor=base west,inner sep=0pt, outer sep=0pt, scale=  0.57] at (191.81, 86.89) {Xenon arc};
\definecolor{drawColor}{gray}{0.20}

\path[draw=drawColor,line width= 0.5pt,line join=round,line cap=round] (145.79, 65.71) rectangle (270.13,101.01);
\end{scope}
\begin{scope}
\path[clip] (145.79, 25.90) rectangle (270.13, 61.21);
\definecolor{fillColor}{RGB}{255,255,255}

\path[fill=fillColor] (145.79, 25.90) rectangle (270.13, 61.21);
\definecolor{fillColor}{RGB}{0,191,125}

\path[fill=fillColor] (133.48, 25.92) --
	(133.48, 25.92) --
	(133.55, 26.00) --
	(133.55, 26.00) --
	(133.55, 26.00) --
	(133.63, 26.00) --
	(133.63, 26.00) --
	(133.63, 26.00) --
	(133.70, 26.00) --
	(133.70, 26.00) --
	(133.70, 26.00) --
	(133.78, 26.00) --
	(133.78, 26.00) --
	(133.78, 26.00) --
	(133.86, 26.00) --
	(133.86, 26.00) --
	(133.86, 26.00) --
	(133.93, 26.01) --
	(133.93, 26.01) --
	(133.93, 26.01) --
	(134.01, 26.00) --
	(134.01, 26.00) --
	(134.01, 26.00) --
	(134.08, 26.01) --
	(134.08, 26.01) --
	(134.08, 26.01) --
	(134.16, 26.01) --
	(134.16, 26.01) --
	(134.16, 26.01) --
	(134.24, 26.00) --
	(134.24, 26.00) --
	(134.24, 26.00) --
	(134.31, 26.00) --
	(134.31, 26.00) --
	(134.31, 26.00) --
	(134.39, 26.00) --
	(134.39, 26.00) --
	(134.39, 26.00) --
	(134.47, 26.00) --
	(134.47, 26.00) --
	(134.47, 26.00) --
	(134.54, 26.01) --
	(134.54, 26.01) --
	(134.54, 26.01) --
	(134.62, 26.00) --
	(134.62, 26.00) --
	(134.62, 26.00) --
	(134.69, 26.01) --
	(134.69, 26.01) --
	(134.70, 26.01) --
	(134.77, 26.01) --
	(134.77, 26.01) --
	(134.77, 26.01) --
	(134.85, 26.00) --
	(134.85, 26.00) --
	(134.85, 26.00) --
	(134.92, 26.00) --
	(134.92, 26.00) --
	(134.92, 26.00) --
	(135.00, 26.02) --
	(135.00, 26.02) --
	(135.00, 26.02) --
	(135.08, 26.00) --
	(135.08, 26.00) --
	(135.08, 26.00) --
	(135.15, 26.00) --
	(135.15, 26.00) --
	(135.15, 26.00) --
	(135.23, 26.01) --
	(135.23, 26.01) --
	(135.23, 26.01) --
	(135.31, 26.00) --
	(135.31, 26.00) --
	(135.31, 26.00) --
	(135.38, 26.03) --
	(135.38, 26.03) --
	(135.38, 26.03) --
	(135.46, 26.04) --
	(135.46, 26.04) --
	(135.46, 26.04) --
	(135.53, 26.03) --
	(135.53, 26.03) --
	(135.53, 26.03) --
	(135.61, 26.04) --
	(135.61, 26.04) --
	(135.61, 26.04) --
	(135.69, 26.03) --
	(135.69, 26.03) --
	(135.69, 26.03) --
	(135.76, 26.03) --
	(135.76, 26.03) --
	(135.76, 26.03) --
	(135.84, 26.04) --
	(135.84, 26.04) --
	(135.84, 26.04) --
	(135.91, 26.03) --
	(135.91, 26.03) --
	(135.91, 26.03) --
	(135.99, 26.03) --
	(135.99, 26.03) --
	(135.99, 26.03) --
	(136.07, 26.03) --
	(136.07, 26.03) --
	(136.07, 26.03) --
	(136.14, 26.03) --
	(136.14, 26.03) --
	(136.14, 26.03) --
	(136.22, 26.03) --
	(136.22, 26.03) --
	(136.22, 26.03) --
	(136.30, 26.04) --
	(136.30, 26.04) --
	(136.30, 26.04) --
	(136.37, 26.04) --
	(136.37, 26.04) --
	(136.37, 26.04) --
	(136.45, 26.04) --
	(136.45, 26.04) --
	(136.45, 26.04) --
	(136.52, 26.03) --
	(136.52, 26.03) --
	(136.52, 26.03) --
	(136.60, 26.04) --
	(136.60, 26.04) --
	(136.60, 26.04) --
	(136.68, 26.05) --
	(136.68, 26.05) --
	(136.68, 26.05) --
	(136.75, 26.02) --
	(136.75, 26.02) --
	(136.75, 26.02) --
	(136.83, 26.04) --
	(136.83, 26.04) --
	(136.83, 26.04) --
	(136.90, 26.04) --
	(136.90, 26.04) --
	(136.90, 26.04) --
	(136.98, 26.03) --
	(136.98, 26.03) --
	(136.98, 26.03) --
	(137.06, 26.03) --
	(137.06, 26.03) --
	(137.06, 26.03) --
	(137.13, 26.03) --
	(137.13, 26.03) --
	(137.13, 26.03) --
	(137.21, 26.03) --
	(137.21, 26.03) --
	(137.21, 26.03) --
	(137.29, 26.04) --
	(137.29, 26.04) --
	(137.29, 26.04) --
	(137.36, 26.03) --
	(137.36, 26.03) --
	(137.36, 26.03) --
	(137.44, 26.04) --
	(137.44, 26.04) --
	(137.44, 26.04) --
	(137.51, 26.05) --
	(137.51, 26.05) --
	(137.51, 26.05) --
	(137.59, 26.04) --
	(137.59, 26.04) --
	(137.59, 26.04) --
	(137.66, 26.03) --
	(137.66, 26.03) --
	(137.66, 26.03) --
	(137.74, 26.03) --
	(137.74, 26.03) --
	(137.74, 26.03) --
	(137.82, 26.04) --
	(137.82, 26.04) --
	(137.82, 26.04) --
	(137.89, 26.04) --
	(137.89, 26.04) --
	(137.89, 26.04) --
	(137.97, 26.04) --
	(137.97, 26.04) --
	(137.97, 26.04) --
	(138.05, 26.04) --
	(138.05, 26.04) --
	(138.05, 26.04) --
	(138.12, 26.04) --
	(138.12, 26.04) --
	(138.12, 26.04) --
	(138.20, 26.03) --
	(138.20, 26.03) --
	(138.20, 26.03) --
	(138.27, 26.04) --
	(138.27, 26.04) --
	(138.27, 26.04) --
	(138.35, 26.05) --
	(138.35, 26.05) --
	(138.35, 26.05) --
	(138.43, 26.03) --
	(138.43, 26.03) --
	(138.43, 26.03) --
	(138.50, 26.04) --
	(138.50, 26.04) --
	(138.50, 26.04) --
	(138.58, 26.05) --
	(138.58, 26.05) --
	(138.58, 26.05) --
	(138.65, 26.04) --
	(138.65, 26.04) --
	(138.65, 26.04) --
	(138.73, 26.04) --
	(138.73, 26.04) --
	(138.73, 26.04) --
	(138.81, 26.04) --
	(138.81, 26.04) --
	(138.81, 26.04) --
	(138.88, 26.04) --
	(138.88, 26.04) --
	(138.88, 26.04) --
	(138.96, 26.04) --
	(138.96, 26.04) --
	(138.96, 26.04) --
	(139.03, 26.04) --
	(139.03, 26.04) --
	(139.03, 26.04) --
	(139.11, 26.04) --
	(139.11, 26.04) --
	(139.11, 26.04) --
	(139.19, 26.06) --
	(139.19, 26.06) --
	(139.19, 26.06) --
	(139.26, 26.05) --
	(139.26, 26.05) --
	(139.26, 26.05) --
	(139.34, 26.04) --
	(139.34, 26.04) --
	(139.34, 26.04) --
	(139.42, 26.05) --
	(139.42, 26.05) --
	(139.42, 26.05) --
	(139.49, 26.04) --
	(139.49, 26.04) --
	(139.49, 26.04) --
	(139.57, 26.05) --
	(139.57, 26.05) --
	(139.57, 26.05) --
	(139.64, 26.04) --
	(139.64, 26.04) --
	(139.64, 26.04) --
	(139.72, 26.04) --
	(139.72, 26.04) --
	(139.72, 26.04) --
	(139.79, 26.05) --
	(139.79, 26.05) --
	(139.79, 26.05) --
	(139.87, 26.04) --
	(139.87, 26.04) --
	(139.87, 26.04) --
	(139.95, 26.05) --
	(139.95, 26.05) --
	(139.95, 26.05) --
	(140.02, 26.06) --
	(140.02, 26.06) --
	(140.02, 26.06) --
	(140.10, 26.04) --
	(140.10, 26.04) --
	(140.10, 26.04) --
	(140.17, 26.04) --
	(140.17, 26.04) --
	(140.17, 26.04) --
	(140.25, 26.05) --
	(140.25, 26.05) --
	(140.25, 26.05) --
	(140.33, 26.04) --
	(140.33, 26.04) --
	(140.33, 26.04) --
	(140.40, 26.05) --
	(140.40, 26.05) --
	(140.40, 26.05) --
	(140.48, 26.05) --
	(140.48, 26.05) --
	(140.48, 26.05) --
	(140.55, 26.04) --
	(140.55, 26.04) --
	(140.55, 26.04) --
	(140.63, 26.04) --
	(140.63, 26.04) --
	(140.63, 26.04) --
	(140.71, 26.04) --
	(140.71, 26.04) --
	(140.71, 26.04) --
	(140.78, 26.05) --
	(140.78, 26.05) --
	(140.78, 26.05) --
	(140.86, 26.05) --
	(140.86, 26.05) --
	(140.86, 26.05) --
	(140.93, 26.04) --
	(140.93, 26.04) --
	(140.93, 26.04) --
	(141.01, 26.05) --
	(141.01, 26.05) --
	(141.01, 26.05) --
	(141.08, 26.05) --
	(141.08, 26.05) --
	(141.08, 26.05) --
	(141.16, 26.04) --
	(141.16, 26.04) --
	(141.16, 26.04) --
	(141.24, 26.04) --
	(141.24, 26.04) --
	(141.24, 26.04) --
	(141.31, 26.06) --
	(141.31, 26.06) --
	(141.31, 26.06) --
	(141.39, 26.04) --
	(141.39, 26.04) --
	(141.39, 26.04) --
	(141.46, 26.06) --
	(141.46, 26.06) --
	(141.46, 26.06) --
	(141.54, 26.05) --
	(141.54, 26.05) --
	(141.54, 26.05) --
	(141.62, 26.04) --
	(141.62, 26.04) --
	(141.62, 26.04) --
	(141.69, 26.06) --
	(141.69, 26.06) --
	(141.69, 26.06) --
	(141.77, 26.04) --
	(141.77, 26.04) --
	(141.77, 26.04) --
	(141.84, 26.05) --
	(141.84, 26.05) --
	(141.84, 26.05) --
	(141.92, 26.05) --
	(141.92, 26.05) --
	(141.92, 26.05) --
	(142.00, 26.05) --
	(142.00, 26.05) --
	(142.00, 26.05) --
	(142.07, 26.05) --
	(142.07, 26.05) --
	(142.07, 26.05) --
	(142.15, 26.06) --
	(142.15, 26.06) --
	(142.15, 26.06) --
	(142.22, 26.05) --
	(142.22, 26.05) --
	(142.22, 26.05) --
	(142.30, 26.05) --
	(142.30, 26.05) --
	(142.30, 26.05) --
	(142.37, 26.05) --
	(142.37, 26.05) --
	(142.37, 26.05) --
	(142.45, 26.04) --
	(142.45, 26.04) --
	(142.45, 26.04) --
	(142.53, 26.06) --
	(142.53, 26.06) --
	(142.53, 26.06) --
	(142.60, 26.04) --
	(142.60, 26.04) --
	(142.60, 26.04) --
	(142.68, 26.05) --
	(142.68, 26.05) --
	(142.68, 26.05) --
	(142.75, 26.06) --
	(142.75, 26.06) --
	(142.75, 26.06) --
	(142.83, 26.05) --
	(142.83, 26.05) --
	(142.83, 26.05) --
	(142.91, 26.05) --
	(142.91, 26.05) --
	(142.91, 26.05) --
	(142.98, 26.05) --
	(142.98, 26.05) --
	(142.98, 26.05) --
	(143.06, 26.06) --
	(143.06, 26.06) --
	(143.06, 26.06) --
	(143.13, 26.06) --
	(143.13, 26.06) --
	(143.13, 26.06) --
	(143.21, 26.06) --
	(143.21, 26.06) --
	(143.21, 26.06) --
	(143.29, 26.06) --
	(143.29, 26.06) --
	(143.29, 26.06) --
	(143.36, 26.07) --
	(143.36, 26.07) --
	(143.36, 26.07) --
	(143.44, 26.06) --
	(143.44, 26.06) --
	(143.44, 26.06) --
	(143.51, 26.06) --
	(143.51, 26.06) --
	(143.51, 26.06) --
	(143.59, 26.06) --
	(143.59, 26.06) --
	(143.59, 26.06) --
	(143.66, 26.06) --
	(143.66, 26.06) --
	(143.66, 26.06) --
	(143.74, 26.07) --
	(143.74, 26.07) --
	(143.74, 26.07) --
	(143.82, 26.07) --
	(143.82, 26.07) --
	(143.82, 26.07) --
	(143.89, 26.06) --
	(143.89, 26.06) --
	(143.89, 26.06) --
	(143.97, 26.06) --
	(143.97, 26.06) --
	(143.97, 26.06) --
	(144.04, 26.07) --
	(144.04, 26.07) --
	(144.04, 26.07) --
	(144.12, 26.07) --
	(144.12, 26.07) --
	(144.12, 26.07) --
	(144.19, 26.07) --
	(144.19, 26.07) --
	(144.19, 26.07) --
	(144.27, 26.07) --
	(144.27, 26.07) --
	(144.27, 26.07) --
	(144.35, 26.07) --
	(144.35, 26.07) --
	(144.35, 26.07) --
	(144.42, 26.08) --
	(144.42, 26.08) --
	(144.42, 26.08) --
	(144.50, 26.08) --
	(144.50, 26.08) --
	(144.50, 26.08) --
	(144.57, 26.08) --
	(144.57, 26.08) --
	(144.57, 26.08) --
	(144.65, 26.09) --
	(144.65, 26.09) --
	(144.65, 26.09) --
	(144.72, 26.08) --
	(144.72, 26.08) --
	(144.72, 26.08) --
	(144.80, 26.09) --
	(144.80, 26.09) --
	(144.80, 26.09) --
	(144.88, 26.09) --
	(144.88, 26.09) --
	(144.88, 26.09) --
	(144.95, 26.10) --
	(144.95, 26.10) --
	(144.95, 26.10) --
	(145.03, 26.10) --
	(145.03, 26.10) --
	(145.03, 26.10) --
	(145.10, 26.10) --
	(145.10, 26.10) --
	(145.10, 26.10) --
	(145.18, 26.11) --
	(145.18, 26.11) --
	(145.18, 26.11) --
	(145.26, 26.12) --
	(145.26, 26.12) --
	(145.26, 26.12) --
	(145.33, 26.11) --
	(145.33, 26.11) --
	(145.33, 26.11) --
	(145.41, 26.12) --
	(145.41, 26.12) --
	(145.41, 26.12) --
	(145.48, 26.14) --
	(145.48, 26.14) --
	(145.48, 26.14) --
	(145.56, 26.14) --
	(145.56, 26.14) --
	(145.56, 26.14) --
	(145.63, 26.17) --
	(145.63, 26.17) --
	(145.63, 26.17) --
	(145.71, 26.24) --
	(145.71, 26.24) --
	(145.71, 26.24) --
	(145.79, 26.36) --
	(145.79, 26.36) --
	(145.79, 26.36) --
	(145.86, 26.47) --
	(145.86, 26.47) --
	(145.86, 26.47) --
	(145.94, 26.45) --
	(145.94, 26.45) --
	(145.94, 26.45) --
	(146.01, 26.30) --
	(146.01, 26.30) --
	(146.01, 26.30) --
	(146.09, 26.21) --
	(146.09, 26.21) --
	(146.09, 26.21) --
	(146.16, 26.18) --
	(146.16, 26.18) --
	(146.16, 26.18) --
	(146.24, 26.17) --
	(146.24, 26.17) --
	(146.24, 26.17) --
	(146.32, 26.18) --
	(146.32, 26.18) --
	(146.32, 26.18) --
	(146.39, 26.21) --
	(146.39, 26.21) --
	(146.39, 26.21) --
	(146.47, 26.26) --
	(146.47, 26.26) --
	(146.47, 26.26) --
	(146.54, 26.33) --
	(146.54, 26.33) --
	(146.54, 26.33) --
	(146.62, 26.39) --
	(146.62, 26.39) --
	(146.62, 26.39) --
	(146.69, 26.43) --
	(146.69, 26.43) --
	(146.69, 26.43) --
	(146.77, 26.44) --
	(146.77, 26.44) --
	(146.77, 26.44) --
	(146.84, 26.47) --
	(146.84, 26.47) --
	(146.84, 26.47) --
	(146.92, 26.51) --
	(146.92, 26.51) --
	(146.92, 26.51) --
	(146.99, 26.51) --
	(146.99, 26.51) --
	(146.99, 26.51) --
	(147.07, 26.53) --
	(147.07, 26.53) --
	(147.07, 26.53) --
	(147.15, 26.57) --
	(147.15, 26.57) --
	(147.15, 26.57) --
	(147.22, 26.68) --
	(147.22, 26.68) --
	(147.22, 26.68) --
	(147.30, 26.79) --
	(147.30, 26.79) --
	(147.30, 26.79) --
	(147.37, 26.90) --
	(147.37, 26.90) --
	(147.37, 26.90) --
	(147.45, 26.97) --
	(147.45, 26.97) --
	(147.45, 26.97) --
	(147.52, 26.96) --
	(147.52, 26.96) --
	(147.52, 26.96) --
	(147.60, 26.91) --
	(147.60, 26.91) --
	(147.60, 26.91) --
	(147.68, 26.91) --
	(147.68, 26.91) --
	(147.68, 26.91) --
	(147.75, 26.89) --
	(147.75, 26.89) --
	(147.75, 26.89) --
	(147.83, 26.78) --
	(147.83, 26.78) --
	(147.83, 26.78) --
	(147.90, 26.66) --
	(147.90, 26.66) --
	(147.90, 26.66) --
	(147.98, 26.61) --
	(147.98, 26.61) --
	(147.98, 26.61) --
	(148.05, 26.67) --
	(148.05, 26.67) --
	(148.05, 26.67) --
	(148.13, 26.93) --
	(148.13, 26.93) --
	(148.13, 26.93) --
	(148.20, 27.33) --
	(148.20, 27.33) --
	(148.20, 27.33) --
	(148.28, 27.67) --
	(148.28, 27.67) --
	(148.28, 27.67) --
	(148.36, 27.61) --
	(148.36, 27.61) --
	(148.36, 27.61) --
	(148.43, 27.17) --
	(148.43, 27.17) --
	(148.43, 27.17) --
	(148.51, 26.85) --
	(148.51, 26.85) --
	(148.51, 26.85) --
	(148.58, 26.80) --
	(148.58, 26.80) --
	(148.58, 26.80) --
	(148.66, 26.80) --
	(148.66, 26.80) --
	(148.66, 26.80) --
	(148.73, 26.80) --
	(148.73, 26.80) --
	(148.73, 26.80) --
	(148.81, 26.77) --
	(148.81, 26.77) --
	(148.81, 26.77) --
	(148.88, 26.77) --
	(148.88, 26.77) --
	(148.88, 26.77) --
	(148.96, 26.78) --
	(148.96, 26.78) --
	(148.96, 26.78) --
	(149.04, 26.74) --
	(149.04, 26.74) --
	(149.04, 26.74) --
	(149.11, 26.65) --
	(149.11, 26.65) --
	(149.11, 26.65) --
	(149.19, 26.63) --
	(149.19, 26.63) --
	(149.19, 26.63) --
	(149.26, 26.66) --
	(149.26, 26.66) --
	(149.26, 26.66) --
	(149.34, 26.70) --
	(149.34, 26.70) --
	(149.34, 26.70) --
	(149.41, 26.72) --
	(149.41, 26.72) --
	(149.41, 26.72) --
	(149.49, 26.69) --
	(149.49, 26.69) --
	(149.49, 26.69) --
	(149.56, 26.86) --
	(149.56, 26.86) --
	(149.56, 26.86) --
	(149.64, 27.26) --
	(149.64, 27.26) --
	(149.64, 27.26) --
	(149.72, 27.68) --
	(149.72, 27.68) --
	(149.72, 27.68) --
	(149.79, 27.76) --
	(149.79, 27.76) --
	(149.79, 27.76) --
	(149.87, 27.34) --
	(149.87, 27.34) --
	(149.87, 27.34) --
	(149.94, 27.06) --
	(149.94, 27.06) --
	(149.94, 27.06) --
	(150.02, 27.43) --
	(150.02, 27.43) --
	(150.02, 27.43) --
	(150.09, 28.09) --
	(150.09, 28.09) --
	(150.09, 28.09) --
	(150.17, 28.57) --
	(150.17, 28.57) --
	(150.17, 28.57) --
	(150.24, 28.46) --
	(150.24, 28.46) --
	(150.24, 28.46) --
	(150.32, 27.91) --
	(150.32, 27.91) --
	(150.32, 27.91) --
	(150.39, 27.56) --
	(150.39, 27.56) --
	(150.39, 27.56) --
	(150.47, 27.51) --
	(150.47, 27.51) --
	(150.47, 27.51) --
	(150.55, 27.57) --
	(150.55, 27.57) --
	(150.55, 27.57) --
	(150.62, 27.74) --
	(150.62, 27.74) --
	(150.62, 27.74) --
	(150.70, 28.10) --
	(150.70, 28.10) --
	(150.70, 28.10) --
	(150.77, 29.17) --
	(150.77, 29.17) --
	(150.77, 29.17) --
	(150.85, 30.87) --
	(150.85, 30.87) --
	(150.85, 30.87) --
	(150.92, 32.27) --
	(150.92, 32.27) --
	(150.92, 32.27) --
	(151.00, 32.13) --
	(151.00, 32.13) --
	(151.00, 32.13) --
	(151.07, 30.11) --
	(151.07, 30.11) --
	(151.07, 30.11) --
	(151.15, 28.76) --
	(151.15, 28.76) --
	(151.15, 28.76) --
	(151.22, 28.50) --
	(151.22, 28.50) --
	(151.22, 28.50) --
	(151.30, 28.58) --
	(151.30, 28.58) --
	(151.30, 28.58) --
	(151.37, 28.84) --
	(151.37, 28.84) --
	(151.37, 28.84) --
	(151.45, 29.31) --
	(151.45, 29.31) --
	(151.45, 29.31) --
	(151.53, 29.72) --
	(151.53, 29.72) --
	(151.53, 29.72) --
	(151.60, 30.66) --
	(151.60, 30.66) --
	(151.60, 30.66) --
	(151.68, 33.12) --
	(151.68, 33.12) --
	(151.68, 33.12) --
	(151.75, 35.81) --
	(151.75, 35.81) --
	(151.75, 35.81) --
	(151.83, 36.67) --
	(151.83, 36.67) --
	(151.83, 36.67) --
	(151.90, 35.78) --
	(151.90, 35.78) --
	(151.90, 35.78) --
	(151.98, 31.87) --
	(151.98, 31.87) --
	(151.98, 31.87) --
	(152.05, 30.01) --
	(152.05, 30.01) --
	(152.05, 30.01) --
	(152.13, 29.62) --
	(152.13, 29.62) --
	(152.13, 29.62) --
	(152.20, 29.44) --
	(152.20, 29.44) --
	(152.20, 29.44) --
	(152.28, 29.25) --
	(152.28, 29.25) --
	(152.28, 29.25) --
	(152.35, 29.19) --
	(152.35, 29.18) --
	(152.35, 29.19) --
	(152.43, 29.31) --
	(152.43, 29.31) --
	(152.43, 29.31) --
	(152.51, 29.00) --
	(152.51, 29.00) --
	(152.51, 29.00) --
	(152.58, 28.58) --
	(152.58, 28.58) --
	(152.58, 28.58) --
	(152.66, 28.36) --
	(152.66, 28.36) --
	(152.66, 28.36) --
	(152.73, 28.21) --
	(152.73, 28.21) --
	(152.73, 28.21) --
	(152.81, 28.11) --
	(152.81, 28.11) --
	(152.81, 28.11) --
	(152.88, 28.19) --
	(152.88, 28.19) --
	(152.88, 28.19) --
	(152.96, 28.48) --
	(152.96, 28.48) --
	(152.96, 28.48) --
	(153.03, 28.77) --
	(153.03, 28.77) --
	(153.03, 28.77) --
	(153.11, 28.82) --
	(153.11, 28.82) --
	(153.11, 28.82) --
	(153.11, 28.82) --
	(153.11, 28.82) --
	(153.11, 28.82) --
	(153.18, 28.54) --
	(153.18, 28.54) --
	(153.18, 28.54) --
	(153.26, 29.07) --
	(153.26, 29.07) --
	(153.26, 29.07) --
	(153.26, 29.11) --
	(153.26, 29.11) --
	(153.26, 29.12) --
	(153.33, 32.91) --
	(153.33, 32.91) --
	(153.33, 32.91) --
	(153.35, 33.69) --
	(153.35, 33.69) --
	(153.35, 33.69) --
	(153.39, 35.92) --
	(153.39, 35.92) --
	(153.39, 35.92) --
	(153.41, 37.01) --
	(153.41, 37.01) --
	(153.41, 37.01) --
	(153.47, 38.45) --
	(153.47, 38.45) --
	(153.47, 38.45) --
	(153.48, 38.80) --
	(153.48, 38.81) --
	(153.48, 38.81) --
	(153.56, 39.16) --
	(153.56, 39.16) --
	(153.56, 39.16) --
	(153.59, 38.89) --
	(153.59, 38.89) --
	(153.59, 38.89) --
	(153.63, 38.52) --
	(153.63, 38.52) --
	(153.63, 38.52) --
	(153.67, 37.16) --
	(153.67, 37.15) --
	(153.67, 37.15) --
	(153.71, 35.73) --
	(153.71, 35.73) --
	(153.71, 35.73) --
	(153.79, 30.24) --
	(153.79, 30.24) --
	(153.79, 30.24) --
	(153.86, 28.70) --
	(153.86, 28.70) --
	(153.86, 28.70) --
	(153.94, 28.23) --
	(153.94, 28.23) --
	(153.94, 28.23) --
	(153.95, 28.18) --
	(153.95, 28.18) --
	(153.95, 28.18) --
	(154.01, 28.01) --
	(154.01, 28.01) --
	(154.01, 28.01) --
	(154.09, 27.80) --
	(154.09, 27.80) --
	(154.09, 27.80) --
	(154.16, 27.68) --
	(154.16, 27.68) --
	(154.16, 27.68) --
	(154.24, 27.67) --
	(154.24, 27.67) --
	(154.24, 27.67) --
	(154.31, 27.86) --
	(154.31, 27.86) --
	(154.31, 27.86) --
	(154.39, 28.10) --
	(154.39, 28.10) --
	(154.39, 28.10) --
	(154.46, 28.35) --
	(154.46, 28.35) --
	(154.46, 28.35) --
	(154.54, 28.69) --
	(154.54, 28.69) --
	(154.54, 28.69) --
	(154.61, 29.00) --
	(154.61, 29.00) --
	(154.61, 29.00) --
	(154.64, 29.05) --
	(154.64, 29.05) --
	(154.64, 29.05) --
	(154.69, 29.15) --
	(154.69, 29.15) --
	(154.69, 29.15) --
	(154.76, 29.30) --
	(154.76, 29.30) --
	(154.76, 29.30) --
	(154.84, 29.91) --
	(154.84, 29.91) --
	(154.84, 29.91) --
	(154.91, 30.77) --
	(154.91, 30.77) --
	(154.91, 30.77) --
	(154.99, 31.39) --
	(154.99, 31.39) --
	(154.99, 31.39) --
	(155.06, 31.69) --
	(155.06, 31.69) --
	(155.06, 31.69) --
	(155.14, 31.70) --
	(155.14, 31.70) --
	(155.14, 31.70) --
	(155.22, 30.90) --
	(155.22, 30.90) --
	(155.22, 30.90) --
	(155.29, 29.68) --
	(155.29, 29.68) --
	(155.29, 29.68) --
	(155.37, 28.93) --
	(155.37, 28.92) --
	(155.37, 28.92) --
	(155.44, 28.57) --
	(155.44, 28.57) --
	(155.44, 28.57) --
	(155.52, 28.30) --
	(155.52, 28.30) --
	(155.52, 28.30) --
	(155.59, 28.00) --
	(155.59, 28.00) --
	(155.59, 28.00) --
	(155.67, 27.93) --
	(155.67, 27.93) --
	(155.67, 27.93) --
	(155.74, 27.97) --
	(155.74, 27.97) --
	(155.74, 27.97) --
	(155.82, 27.98) --
	(155.82, 27.98) --
	(155.82, 27.98) --
	(155.89, 27.98) --
	(155.89, 27.98) --
	(155.89, 27.98) --
	(155.90, 27.99) --
	(155.90, 27.99) --
	(155.90, 27.99) --
	(155.97, 28.02) --
	(155.97, 28.02) --
	(155.97, 28.02) --
	(156.04, 28.21) --
	(156.04, 28.21) --
	(156.04, 28.21) --
	(156.12, 28.34) --
	(156.12, 28.34) --
	(156.12, 28.34) --
	(156.19, 28.22) --
	(156.19, 28.22) --
	(156.19, 28.22) --
	(156.23, 28.19) --
	(156.23, 28.19) --
	(156.23, 28.19) --
	(156.27, 28.17) --
	(156.27, 28.17) --
	(156.27, 28.17) --
	(156.34, 28.33) --
	(156.34, 28.33) --
	(156.34, 28.33) --
	(156.42, 28.55) --
	(156.42, 28.55) --
	(156.42, 28.55) --
	(156.49, 28.30) --
	(156.49, 28.30) --
	(156.49, 28.30) --
	(156.50, 28.26) --
	(156.50, 28.26) --
	(156.50, 28.26) --
	(156.57, 27.91) --
	(156.57, 27.91) --
	(156.57, 27.91) --
	(156.64, 27.86) --
	(156.64, 27.86) --
	(156.64, 27.86) --
	(156.72, 28.13) --
	(156.72, 28.13) --
	(156.72, 28.13) --
	(156.79, 29.12) --
	(156.79, 29.12) --
	(156.79, 29.12) --
	(156.87, 30.60) --
	(156.87, 30.60) --
	(156.87, 30.60) --
	(156.87, 30.66) --
	(156.87, 30.66) --
	(156.87, 30.66) --
	(156.94, 32.00) --
	(156.94, 32.00) --
	(156.94, 32.00) --
	(157.02, 31.98) --
	(157.02, 31.98) --
	(157.02, 31.98) --
	(157.03, 31.76) --
	(157.03, 31.76) --
	(157.03, 31.76) --
	(157.09, 29.76) --
	(157.09, 29.76) --
	(157.09, 29.76) --
	(157.17, 28.20) --
	(157.17, 28.20) --
	(157.17, 28.20) --
	(157.23, 27.96) --
	(157.23, 27.96) --
	(157.23, 27.96) --
	(157.24, 27.90) --
	(157.24, 27.90) --
	(157.24, 27.90) --
	(157.32, 27.84) --
	(157.32, 27.84) --
	(157.32, 27.84) --
	(157.39, 27.87) --
	(157.39, 27.87) --
	(157.39, 27.87) --
	(157.47, 28.00) --
	(157.47, 28.00) --
	(157.47, 28.00) --
	(157.54, 28.09) --
	(157.54, 28.09) --
	(157.54, 28.09) --
	(157.62, 28.30) --
	(157.62, 28.30) --
	(157.62, 28.30) --
	(157.69, 28.71) --
	(157.69, 28.71) --
	(157.69, 28.71) --
	(157.77, 29.20) --
	(157.77, 29.20) --
	(157.77, 29.20) --
	(157.84, 30.07) --
	(157.84, 30.07) --
	(157.84, 30.07) --
	(157.92, 31.43) --
	(157.92, 31.43) --
	(157.92, 31.43) --
	(157.99, 33.03) --
	(157.99, 33.03) --
	(157.99, 33.03) --
	(158.04, 33.51) --
	(158.04, 33.51) --
	(158.04, 33.51) --
	(158.07, 33.88) --
	(158.07, 33.88) --
	(158.07, 33.88) --
	(158.14, 33.26) --
	(158.15, 33.26) --
	(158.15, 33.26) --
	(158.22, 32.39) --
	(158.22, 32.39) --
	(158.22, 32.39) --
	(158.24, 32.21) --
	(158.24, 32.21) --
	(158.24, 32.21) --
	(158.30, 31.83) --
	(158.30, 31.83) --
	(158.30, 31.83) --
	(158.37, 32.03) --
	(158.37, 32.03) --
	(158.37, 32.03) --
	(158.45, 33.08) --
	(158.45, 33.08) --
	(158.45, 33.08) --
	(158.52, 34.12) --
	(158.52, 34.12) --
	(158.52, 34.12) --
	(158.60, 34.74) --
	(158.60, 34.74) --
	(158.60, 34.74) --
	(158.67, 35.44) --
	(158.67, 35.44) --
	(158.67, 35.44) --
	(158.75, 35.94) --
	(158.75, 35.94) --
	(158.75, 35.94) --
	(158.76, 35.56) --
	(158.76, 35.56) --
	(158.76, 35.56) --
	(158.82, 34.50) --
	(158.82, 34.50) --
	(158.82, 34.50) --
	(158.89, 33.05) --
	(158.89, 33.05) --
	(158.89, 33.05) --
	(158.97, 32.87) --
	(158.97, 32.87) --
	(158.97, 32.87) --
	(159.04, 33.90) --
	(159.04, 33.90) --
	(159.04, 33.90) --
	(159.12, 35.20) --
	(159.12, 35.20) --
	(159.12, 35.20) --
	(159.19, 34.44) --
	(159.19, 34.44) --
	(159.19, 34.44) --
	(159.25, 33.72) --
	(159.25, 33.72) --
	(159.25, 33.72) --
	(159.27, 33.45) --
	(159.27, 33.45) --
	(159.27, 33.45) --
	(159.34, 33.82) --
	(159.34, 33.82) --
	(159.34, 33.82) --
	(159.42, 35.06) --
	(159.42, 35.06) --
	(159.42, 35.06) --
	(159.49, 35.48) --
	(159.49, 35.48) --
	(159.49, 35.48) --
	(159.57, 36.19) --
	(159.57, 36.19) --
	(159.57, 36.20) --
	(159.57, 36.29) --
	(159.57, 36.29) --
	(159.57, 36.29) --
	(159.65, 37.88) --
	(159.65, 37.88) --
	(159.65, 37.88) --
	(159.72, 39.83) --
	(159.72, 39.83) --
	(159.72, 39.83) --
	(159.78, 40.74) --
	(159.78, 40.74) --
	(159.78, 40.74) --
	(159.80, 41.06) --
	(159.80, 41.06) --
	(159.80, 41.06) --
	(159.87, 41.39) --
	(159.87, 41.39) --
	(159.87, 41.39) --
	(159.94, 41.26) --
	(159.94, 41.26) --
	(159.94, 41.26) --
	(159.98, 41.00) --
	(159.98, 41.00) --
	(159.98, 41.00) --
	(160.02, 40.68) --
	(160.02, 40.68) --
	(160.02, 40.68) --
	(160.09, 38.41) --
	(160.09, 38.41) --
	(160.09, 38.41) --
	(160.14, 37.01) --
	(160.14, 37.01) --
	(160.14, 37.01) --
	(160.17, 36.06) --
	(160.17, 36.06) --
	(160.17, 36.06) --
	(160.24, 35.01) --
	(160.24, 35.01) --
	(160.24, 35.01) --
	(160.26, 35.02) --
	(160.26, 35.02) --
	(160.26, 35.02) --
	(160.30, 35.06) --
	(160.30, 35.06) --
	(160.30, 35.06) --
	(160.32, 35.08) --
	(160.32, 35.08) --
	(160.32, 35.09) --
	(160.39, 36.99) --
	(160.39, 36.99) --
	(160.39, 36.99) --
	(160.42, 37.57) --
	(160.42, 37.57) --
	(160.42, 37.57) --
	(160.47, 38.54) --
	(160.47, 38.54) --
	(160.47, 38.54) --
	(160.54, 39.83) --
	(160.54, 39.83) --
	(160.54, 39.83) --
	(160.54, 39.85) --
	(160.54, 39.85) --
	(160.54, 39.85) --
	(160.58, 40.29) --
	(160.58, 40.29) --
	(160.58, 40.29) --
	(160.62, 40.68) --
	(160.62, 40.68) --
	(160.62, 40.68) --
	(160.69, 40.93) --
	(160.69, 40.93) --
	(160.69, 40.93) --
	(160.71, 40.93) --
	(160.71, 40.93) --
	(160.71, 40.93) --
	(160.77, 40.94) --
	(160.77, 40.94) --
	(160.77, 40.94) --
	(160.79, 40.95) --
	(160.79, 40.95) --
	(160.79, 40.95) --
	(160.84, 41.01) --
	(160.84, 41.01) --
	(160.84, 41.01) --
	(160.92, 40.90) --
	(160.92, 40.90) --
	(160.92, 40.90) --
	(160.95, 40.27) --
	(160.95, 40.27) --
	(160.95, 40.27) --
	(160.99, 39.26) --
	(160.99, 39.26) --
	(160.99, 39.26) --
	(161.07, 37.11) --
	(161.07, 37.11) --
	(161.07, 37.11) --
	(161.07, 37.11) --
	(161.07, 37.11) --
	(161.07, 37.11) --
	(161.11, 37.15) --
	(161.11, 37.15) --
	(161.11, 37.15) --
	(161.14, 37.18) --
	(161.14, 37.18) --
	(161.14, 37.18) --
	(161.22, 38.24) --
	(161.22, 38.24) --
	(161.22, 38.24) --
	(161.23, 38.44) --
	(161.23, 38.44) --
	(161.23, 38.44) --
	(161.27, 39.08) --
	(161.27, 39.08) --
	(161.27, 39.08) --
	(161.29, 39.45) --
	(161.29, 39.45) --
	(161.29, 39.45) --
	(161.31, 39.43) --
	(161.31, 39.43) --
	(161.31, 39.43) --
	(161.31, 39.43) --
	(161.31, 39.43) --
	(161.31, 39.43) --
	(161.37, 39.39) --
	(161.37, 39.39) --
	(161.37, 39.39) --
	(161.39, 39.31) --
	(161.39, 39.31) --
	(161.39, 39.31) --
	(161.43, 39.17) --
	(161.43, 39.17) --
	(161.43, 39.17) --
	(161.44, 39.14) --
	(161.44, 39.14) --
	(161.44, 39.14) --
	(161.51, 39.33) --
	(161.51, 39.33) --
	(161.51, 39.33) --
	(161.52, 39.34) --
	(161.52, 39.34) --
	(161.52, 39.34) --
	(161.55, 39.17) --
	(161.55, 39.17) --
	(161.55, 39.17) --
	(161.59, 38.99) --
	(161.59, 38.99) --
	(161.59, 38.99) --
	(161.67, 39.66) --
	(161.67, 39.66) --
	(161.67, 39.66) --
	(161.74, 40.40) --
	(161.74, 40.40) --
	(161.74, 40.40) --
	(161.80, 40.76) --
	(161.80, 40.76) --
	(161.80, 40.77) --
	(161.82, 40.90) --
	(161.82, 40.90) --
	(161.82, 40.91) --
	(161.84, 41.09) --
	(161.84, 41.09) --
	(161.84, 41.09) --
	(161.89, 41.59) --
	(161.89, 41.59) --
	(161.89, 41.59) --
	(161.92, 41.77) --
	(161.92, 41.77) --
	(161.92, 41.77) --
	(161.96, 42.03) --
	(161.96, 42.03) --
	(161.96, 42.03) --
	(161.97, 42.09) --
	(161.97, 42.09) --
	(161.97, 42.09) --
	(162.00, 41.90) --
	(162.00, 41.90) --
	(162.00, 41.90) --
	(162.04, 41.65) --
	(162.04, 41.65) --
	(162.04, 41.65) --
	(162.04, 41.63) --
	(162.04, 41.63) --
	(162.04, 41.63) --
	(162.12, 40.75) --
	(162.12, 40.75) --
	(162.12, 40.75) --
	(162.12, 40.73) --
	(162.12, 40.73) --
	(162.12, 40.73) --
	(162.19, 40.32) --
	(162.19, 40.32) --
	(162.19, 40.32) --
	(162.27, 39.25) --
	(162.27, 39.25) --
	(162.27, 39.25) --
	(162.28, 38.93) --
	(162.28, 38.93) --
	(162.28, 38.93) --
	(162.34, 37.73) --
	(162.34, 37.73) --
	(162.34, 37.73) --
	(162.42, 37.90) --
	(162.42, 37.90) --
	(162.42, 37.90) --
	(162.48, 38.19) --
	(162.48, 38.19) --
	(162.48, 38.19) --
	(162.49, 38.22) --
	(162.49, 38.22) --
	(162.49, 38.22) --
	(162.56, 37.92) --
	(162.56, 37.92) --
	(162.56, 37.92) --
	(162.64, 37.25) --
	(162.64, 37.25) --
	(162.64, 37.25) --
	(162.69, 37.29) --
	(162.69, 37.29) --
	(162.69, 37.29) --
	(162.71, 37.32) --
	(162.71, 37.32) --
	(162.71, 37.32) --
	(162.73, 37.46) --
	(162.73, 37.46) --
	(162.73, 37.47) --
	(162.79, 38.27) --
	(162.79, 38.27) --
	(162.79, 38.27) --
	(162.87, 39.73) --
	(162.87, 39.73) --
	(162.87, 39.73) --
	(162.89, 39.95) --
	(162.89, 39.95) --
	(162.89, 39.95) --
	(162.93, 40.34) --
	(162.93, 40.34) --
	(162.93, 40.34) --
	(162.94, 40.45) --
	(162.94, 40.45) --
	(162.94, 40.45) --
	(163.01, 40.74) --
	(163.01, 40.74) --
	(163.01, 40.74) --
	(163.05, 40.99) --
	(163.05, 40.99) --
	(163.05, 40.99) --
	(163.09, 41.27) --
	(163.09, 41.27) --
	(163.09, 41.27) --
	(163.09, 41.27) --
	(163.09, 41.27) --
	(163.09, 41.27) --
	(163.16, 41.59) --
	(163.16, 41.59) --
	(163.16, 41.59) --
	(163.24, 41.80) --
	(163.24, 41.80) --
	(163.24, 41.80) --
	(163.25, 41.88) --
	(163.25, 41.88) --
	(163.25, 41.88) --
	(163.31, 42.21) --
	(163.31, 42.21) --
	(163.31, 42.21) --
	(163.37, 42.39) --
	(163.37, 42.39) --
	(163.37, 42.39) --
	(163.39, 42.44) --
	(163.39, 42.44) --
	(163.39, 42.44) --
	(163.45, 42.31) --
	(163.45, 42.31) --
	(163.45, 42.31) --
	(163.46, 42.29) --
	(163.46, 42.29) --
	(163.46, 42.29) --
	(163.53, 42.26) --
	(163.53, 42.26) --
	(163.53, 42.26) --
	(163.54, 42.26) --
	(163.54, 42.26) --
	(163.54, 42.26) --
	(163.57, 42.12) --
	(163.57, 42.12) --
	(163.57, 42.12) --
	(163.61, 41.97) --
	(163.61, 41.97) --
	(163.61, 41.97) --
	(163.66, 41.73) --
	(163.66, 41.73) --
	(163.66, 41.73) --
	(163.69, 41.56) --
	(163.69, 41.56) --
	(163.69, 41.56) --
	(163.70, 41.56) --
	(163.70, 41.56) --
	(163.70, 41.56) --
	(163.71, 41.56) --
	(163.71, 41.56) --
	(163.71, 41.56) --
	(163.76, 41.56) --
	(163.76, 41.56) --
	(163.76, 41.56) --
	(163.82, 41.69) --
	(163.82, 41.69) --
	(163.82, 41.69) --
	(163.84, 41.74) --
	(163.84, 41.74) --
	(163.84, 41.74) --
	(163.90, 41.82) --
	(163.90, 41.82) --
	(163.90, 41.82) --
	(163.91, 41.84) --
	(163.91, 41.84) --
	(163.91, 41.84) --
	(163.99, 42.05) --
	(163.99, 42.05) --
	(163.99, 42.05) --
	(164.02, 42.00) --
	(164.02, 42.00) --
	(164.02, 42.00) --
	(164.06, 41.94) --
	(164.06, 41.94) --
	(164.06, 41.94) --
	(164.14, 41.36) --
	(164.14, 41.36) --
	(164.14, 41.36) --
	(164.18, 41.22) --
	(164.18, 41.22) --
	(164.18, 41.22) --
	(164.21, 41.12) --
	(164.21, 41.12) --
	(164.21, 41.12) --
	(164.22, 41.11) --
	(164.22, 41.11) --
	(164.22, 41.11) --
	(164.28, 41.07) --
	(164.28, 41.07) --
	(164.28, 41.07) --
	(164.34, 41.04) --
	(164.34, 41.04) --
	(164.34, 41.04) --
	(164.36, 41.03) --
	(164.36, 41.03) --
	(164.36, 41.03) --
	(164.42, 40.71) --
	(164.42, 40.71) --
	(164.42, 40.71) --
	(164.43, 40.65) --
	(164.43, 40.65) --
	(164.43, 40.65) --
	(164.51, 40.89) --
	(164.51, 40.89) --
	(164.51, 40.89) --
	(164.55, 41.18) --
	(164.55, 41.18) --
	(164.55, 41.18) --
	(164.58, 41.48) --
	(164.58, 41.48) --
	(164.58, 41.48) --
	(164.66, 42.14) --
	(164.66, 42.14) --
	(164.66, 42.14) --
	(164.66, 42.16) --
	(164.66, 42.16) --
	(164.66, 42.16) --
	(164.67, 42.17) --
	(164.67, 42.17) --
	(164.67, 42.17) --
	(164.71, 42.32) --
	(164.71, 42.32) --
	(164.71, 42.32) --
	(164.73, 42.42) --
	(164.73, 42.42) --
	(164.73, 42.42) --
	(164.81, 42.40) --
	(164.81, 42.40) --
	(164.81, 42.40) --
	(164.88, 41.92) --
	(164.88, 41.92) --
	(164.88, 41.92) --
	(164.95, 41.58) --
	(164.95, 41.58) --
	(164.95, 41.58) --
	(164.96, 41.54) --
	(164.96, 41.54) --
	(164.96, 41.54) --
	(164.99, 41.50) --
	(164.99, 41.50) --
	(164.99, 41.50) --
	(165.03, 41.44) --
	(165.03, 41.44) --
	(165.03, 41.44) --
	(165.07, 41.33) --
	(165.07, 41.33) --
	(165.07, 41.33) --
	(165.11, 41.23) --
	(165.11, 41.23) --
	(165.11, 41.23) --
	(165.18, 41.11) --
	(165.18, 41.11) --
	(165.18, 41.11) --
	(165.25, 41.21) --
	(165.25, 41.21) --
	(165.25, 41.21) --
	(165.27, 41.24) --
	(165.27, 41.24) --
	(165.27, 41.24) --
	(165.33, 41.35) --
	(165.33, 41.35) --
	(165.33, 41.35) --
	(165.41, 41.23) --
	(165.41, 41.23) --
	(165.41, 41.23) --
	(165.48, 41.03) --
	(165.48, 41.03) --
	(165.48, 41.03) --
	(165.55, 41.18) --
	(165.55, 41.18) --
	(165.55, 41.18) --
	(165.60, 41.29) --
	(165.60, 41.29) --
	(165.60, 41.29) --
	(165.63, 41.39) --
	(165.63, 41.39) --
	(165.63, 41.39) --
	(165.70, 41.40) --
	(165.70, 41.40) --
	(165.70, 41.40) --
	(165.76, 41.39) --
	(165.76, 41.39) --
	(165.76, 41.39) --
	(165.78, 41.39) --
	(165.78, 41.39) --
	(165.78, 41.39) --
	(165.85, 41.29) --
	(165.85, 41.29) --
	(165.85, 41.29) --
	(165.93, 41.10) --
	(165.93, 41.10) --
	(165.93, 41.10) --
	(166.00, 40.93) --
	(166.00, 40.93) --
	(166.00, 40.93) --
	(166.00, 40.93) --
	(166.00, 40.93) --
	(166.00, 40.93) --
	(166.08, 39.74) --
	(166.08, 39.74) --
	(166.08, 39.74) --
	(166.15, 38.35) --
	(166.15, 38.35) --
	(166.15, 38.35) --
	(166.16, 38.31) --
	(166.16, 38.31) --
	(166.16, 38.31) --
	(166.23, 38.10) --
	(166.23, 38.10) --
	(166.23, 38.10) --
	(166.30, 39.29) --
	(166.30, 39.29) --
	(166.30, 39.29) --
	(166.38, 40.31) --
	(166.38, 40.31) --
	(166.38, 40.31) --
	(166.44, 39.90) --
	(166.44, 39.90) --
	(166.44, 39.90) --
	(166.45, 39.87) --
	(166.45, 39.87) --
	(166.45, 39.87) --
	(166.48, 39.18) --
	(166.48, 39.18) --
	(166.48, 39.18) --
	(166.52, 38.40) --
	(166.52, 38.40) --
	(166.52, 38.40) --
	(166.60, 35.33) --
	(166.60, 35.33) --
	(166.60, 35.33) --
	(166.67, 33.88) --
	(166.67, 33.88) --
	(166.67, 33.88) --
	(166.75, 34.10) --
	(166.75, 34.10) --
	(166.75, 34.10) --
	(166.81, 34.33) --
	(166.81, 34.33) --
	(166.81, 34.33) --
	(166.82, 34.38) --
	(166.82, 34.38) --
	(166.82, 34.38) --
	(166.90, 34.98) --
	(166.90, 34.98) --
	(166.90, 34.98) --
	(166.97, 35.89) --
	(166.97, 35.89) --
	(166.97, 35.89) --
	(167.05, 37.17) --
	(167.05, 37.17) --
	(167.05, 37.17) --
	(167.12, 37.73) --
	(167.12, 37.73) --
	(167.12, 37.73) --
	(167.17, 36.59) --
	(167.17, 36.59) --
	(167.17, 36.59) --
	(167.20, 36.06) --
	(167.20, 36.06) --
	(167.20, 36.06) --
	(167.27, 33.45) --
	(167.27, 33.45) --
	(167.27, 33.45) --
	(167.34, 33.04) --
	(167.34, 33.04) --
	(167.34, 33.04) --
	(167.42, 33.44) --
	(167.42, 33.44) --
	(167.42, 33.44) --
	(167.49, 34.80) --
	(167.49, 34.80) --
	(167.49, 34.80) --
	(167.57, 36.39) --
	(167.57, 36.39) --
	(167.57, 36.39) --
	(167.64, 36.63) --
	(167.64, 36.63) --
	(167.64, 36.63) --
	(167.66, 36.61) --
	(167.66, 36.61) --
	(167.66, 36.61) --
	(167.72, 36.53) --
	(167.72, 36.53) --
	(167.72, 36.53) --
	(167.79, 34.84) --
	(167.79, 34.84) --
	(167.79, 34.84) --
	(167.83, 34.29) --
	(167.83, 34.29) --
	(167.83, 34.29) --
	(167.87, 33.67) --
	(167.87, 33.67) --
	(167.87, 33.67) --
	(167.94, 34.00) --
	(167.94, 34.00) --
	(167.94, 34.00) --
	(167.98, 34.50) --
	(167.98, 34.50) --
	(167.98, 34.50) --
	(168.01, 34.92) --
	(168.01, 34.92) --
	(168.01, 34.92) --
	(168.09, 37.11) --
	(168.09, 37.11) --
	(168.09, 37.11) --
	(168.16, 40.15) --
	(168.16, 40.15) --
	(168.16, 40.15) --
	(168.24, 41.39) --
	(168.24, 41.39) --
	(168.24, 41.39) --
	(168.30, 42.31) --
	(168.30, 42.31) --
	(168.30, 42.31) --
	(168.31, 42.44) --
	(168.31, 42.44) --
	(168.31, 42.44) --
	(168.34, 42.50) --
	(168.34, 42.50) --
	(168.34, 42.50) --
	(168.39, 42.58) --
	(168.39, 42.58) --
	(168.39, 42.58) --
	(168.46, 41.43) --
	(168.46, 41.43) --
	(168.46, 41.43) --
	(168.51, 40.96) --
	(168.51, 40.96) --
	(168.51, 40.96) --
	(168.54, 40.64) --
	(168.54, 40.64) --
	(168.54, 40.64) --
	(168.55, 40.69) --
	(168.55, 40.69) --
	(168.55, 40.69) --
	(168.61, 40.96) --
	(168.61, 40.96) --
	(168.61, 40.96) --
	(168.63, 40.99) --
	(168.63, 40.99) --
	(168.63, 40.99) --
	(168.67, 41.06) --
	(168.67, 41.06) --
	(168.67, 41.06) --
	(168.69, 41.10) --
	(168.69, 41.10) --
	(168.69, 41.10) --
	(168.71, 41.04) --
	(168.71, 41.04) --
	(168.71, 41.04) --
	(168.75, 40.94) --
	(168.75, 40.94) --
	(168.75, 40.94) --
	(168.76, 40.90) --
	(168.76, 40.90) --
	(168.76, 40.90) --
	(168.83, 40.20) --
	(168.83, 40.20) --
	(168.83, 40.20) --
	(168.87, 39.29) --
	(168.87, 39.29) --
	(168.87, 39.29) --
	(168.91, 38.26) --
	(168.91, 38.26) --
	(168.91, 38.26) --
	(168.98, 34.93) --
	(168.98, 34.93) --
	(168.98, 34.93) --
	(169.06, 32.40) --
	(169.06, 32.40) --
	(169.06, 32.40) --
	(169.13, 31.63) --
	(169.13, 31.63) --
	(169.13, 31.63) --
	(169.19, 31.67) --
	(169.19, 31.67) --
	(169.19, 31.67) --
	(169.21, 31.68) --
	(169.21, 31.68) --
	(169.21, 31.68) --
	(169.28, 31.89) --
	(169.28, 31.89) --
	(169.28, 31.89) --
	(169.36, 32.70) --
	(169.36, 32.70) --
	(169.36, 32.70) --
	(169.36, 32.74) --
	(169.36, 32.74) --
	(169.36, 32.74) --
	(169.43, 33.79) --
	(169.43, 33.79) --
	(169.43, 33.79) --
	(169.51, 35.03) --
	(169.51, 35.03) --
	(169.51, 35.03) --
	(169.58, 36.25) --
	(169.58, 36.25) --
	(169.58, 36.25) --
	(169.65, 37.06) --
	(169.65, 37.06) --
	(169.65, 37.06) --
	(169.73, 37.76) --
	(169.73, 37.76) --
	(169.73, 37.76) --
	(169.76, 37.95) --
	(169.76, 37.95) --
	(169.76, 37.95) --
	(169.80, 38.20) --
	(169.80, 38.20) --
	(169.80, 38.20) --
	(169.88, 38.05) --
	(169.88, 38.05) --
	(169.88, 38.05) --
	(169.95, 38.22) --
	(169.95, 38.22) --
	(169.95, 38.22) --
	(170.03, 36.41) --
	(170.03, 36.41) --
	(170.03, 36.41) --
	(170.10, 34.30) --
	(170.10, 34.30) --
	(170.10, 34.30) --
	(170.13, 33.99) --
	(170.13, 33.99) --
	(170.13, 33.99) --
	(170.17, 33.44) --
	(170.17, 33.44) --
	(170.17, 33.44) --
	(170.25, 32.63) --
	(170.25, 32.63) --
	(170.25, 32.63) --
	(170.32, 32.88) --
	(170.32, 32.88) --
	(170.32, 32.88) --
	(170.40, 34.29) --
	(170.40, 34.29) --
	(170.40, 34.29) --
	(170.47, 35.77) --
	(170.47, 35.77) --
	(170.47, 35.77) --
	(170.51, 36.41) --
	(170.51, 36.41) --
	(170.51, 36.41) --
	(170.55, 37.05) --
	(170.55, 37.05) --
	(170.55, 37.05) --
	(170.62, 37.93) --
	(170.62, 37.93) --
	(170.62, 37.93) --
	(170.70, 38.37) --
	(170.70, 38.37) --
	(170.70, 38.37) --
	(170.77, 38.74) --
	(170.77, 38.74) --
	(170.77, 38.74) --
	(170.84, 39.45) --
	(170.84, 39.45) --
	(170.84, 39.45) --
	(170.89, 38.82) --
	(170.89, 38.82) --
	(170.89, 38.82) --
	(170.92, 38.50) --
	(170.92, 38.50) --
	(170.92, 38.50) --
	(170.99, 36.19) --
	(170.99, 36.19) --
	(170.99, 36.19) --
	(171.05, 35.53) --
	(171.05, 35.53) --
	(171.05, 35.53) --
	(171.07, 35.38) --
	(171.07, 35.38) --
	(171.07, 35.38) --
	(171.14, 35.78) --
	(171.14, 35.78) --
	(171.14, 35.78) --
	(171.22, 36.75) --
	(171.22, 36.75) --
	(171.22, 36.75) --
	(171.29, 37.56) --
	(171.29, 37.56) --
	(171.29, 37.56) --
	(171.36, 37.85) --
	(171.36, 37.85) --
	(171.36, 37.85) --
	(171.37, 37.89) --
	(171.37, 37.89) --
	(171.37, 37.89) --
	(171.44, 38.18) --
	(171.44, 38.18) --
	(171.44, 38.18) --
	(171.51, 38.14) --
	(171.51, 38.14) --
	(171.51, 38.14) --
	(171.59, 38.99) --
	(171.59, 38.99) --
	(171.59, 38.99) --
	(171.66, 40.12) --
	(171.66, 40.12) --
	(171.66, 40.12) --
	(171.66, 40.14) --
	(171.66, 40.14) --
	(171.66, 40.14) --
	(171.74, 40.09) --
	(171.74, 40.09) --
	(171.74, 40.09) --
	(171.81, 40.29) --
	(171.81, 40.29) --
	(171.81, 40.29) --
	(171.85, 40.65) --
	(171.85, 40.65) --
	(171.85, 40.65) --
	(171.88, 40.94) --
	(171.88, 40.94) --
	(171.88, 40.94) --
	(171.96, 41.16) --
	(171.96, 41.16) --
	(171.96, 41.16) --
	(172.03, 41.17) --
	(172.03, 41.17) --
	(172.03, 41.17) --
	(172.11, 39.41) --
	(172.11, 39.41) --
	(172.11, 39.41) --
	(172.14, 38.57) --
	(172.14, 38.57) --
	(172.14, 38.57) --
	(172.18, 37.47) --
	(172.18, 37.47) --
	(172.18, 37.47) --
	(172.26, 37.80) --
	(172.26, 37.80) --
	(172.26, 37.80) --
	(172.33, 38.64) --
	(172.33, 38.64) --
	(172.33, 38.64) --
	(172.33, 38.66) --
	(172.33, 38.66) --
	(172.33, 38.66) --
	(172.40, 39.92) --
	(172.40, 39.92) --
	(172.40, 39.92) --
	(172.43, 40.20) --
	(172.43, 40.20) --
	(172.43, 40.20) --
	(172.48, 40.84) --
	(172.48, 40.84) --
	(172.48, 40.84) --
	(172.55, 41.28) --
	(172.55, 41.28) --
	(172.55, 41.28) --
	(172.63, 41.31) --
	(172.63, 41.31) --
	(172.63, 41.30) --
	(172.70, 40.17) --
	(172.70, 40.17) --
	(172.70, 40.17) --
	(172.78, 40.68) --
	(172.78, 40.68) --
	(172.78, 40.68) --
	(172.81, 40.96) --
	(172.81, 40.96) --
	(172.81, 40.96) --
	(172.85, 41.28) --
	(172.85, 41.28) --
	(172.85, 41.28) --
	(172.92, 41.31) --
	(172.92, 41.31) --
	(172.92, 41.31) --
	(173.00, 40.43) --
	(173.00, 40.43) --
	(173.00, 40.43) --
	(173.00, 40.43) --
	(173.00, 40.43) --
	(173.00, 40.43) --
	(173.07, 40.53) --
	(173.07, 40.53) --
	(173.07, 40.53) --
	(173.15, 40.96) --
	(173.15, 40.96) --
	(173.15, 40.96) --
	(173.22, 41.69) --
	(173.22, 41.69) --
	(173.22, 41.69) --
	(173.29, 42.71) --
	(173.29, 42.71) --
	(173.29, 42.71) --
	(173.29, 42.80) --
	(173.29, 42.80) --
	(173.29, 42.80) --
	(173.37, 42.80) --
	(173.37, 42.80) --
	(173.37, 42.80) --
	(173.44, 42.80) --
	(173.44, 42.80) --
	(173.44, 42.80) --
	(173.48, 42.55) --
	(173.48, 42.55) --
	(173.48, 42.55) --
	(173.52, 42.29) --
	(173.52, 42.29) --
	(173.52, 42.29) --
	(173.58, 41.65) --
	(173.58, 41.65) --
	(173.58, 41.65) --
	(173.59, 41.48) --
	(173.59, 41.48) --
	(173.59, 41.48) --
	(173.67, 41.83) --
	(173.67, 41.83) --
	(173.67, 41.83) --
	(173.74, 42.42) --
	(173.74, 42.42) --
	(173.74, 42.42) --
	(173.77, 42.52) --
	(173.77, 42.52) --
	(173.77, 42.52) --
	(173.81, 42.70) --
	(173.81, 42.70) --
	(173.81, 42.70) --
	(173.86, 42.44) --
	(173.86, 42.44) --
	(173.86, 42.44) --
	(173.89, 42.31) --
	(173.89, 42.31) --
	(173.89, 42.31) --
	(173.96, 41.61) --
	(173.96, 41.61) --
	(173.96, 41.61) --
	(174.04, 41.85) --
	(174.04, 41.85) --
	(174.04, 41.86) --
	(174.11, 42.36) --
	(174.11, 42.36) --
	(174.11, 42.36) --
	(174.15, 42.48) --
	(174.15, 42.48) --
	(174.15, 42.48) --
	(174.18, 42.58) --
	(174.18, 42.58) --
	(174.18, 42.58) --
	(174.25, 42.27) --
	(174.25, 42.27) --
	(174.25, 42.27) --
	(174.26, 42.21) --
	(174.26, 42.21) --
	(174.26, 42.21) --
	(174.33, 42.09) --
	(174.33, 42.09) --
	(174.33, 42.09) --
	(174.34, 42.06) --
	(174.34, 42.06) --
	(174.34, 42.06) --
	(174.41, 41.82) --
	(174.41, 41.82) --
	(174.41, 41.82) --
	(174.44, 41.77) --
	(174.44, 41.77) --
	(174.44, 41.77) --
	(174.48, 41.69) --
	(174.48, 41.69) --
	(174.48, 41.69) --
	(174.53, 41.69) --
	(174.53, 41.69) --
	(174.53, 41.69) --
	(174.56, 41.69) --
	(174.56, 41.69) --
	(174.56, 41.69) --
	(174.63, 41.63) --
	(174.63, 41.63) --
	(174.63, 41.63) --
	(174.63, 41.63) --
	(174.63, 41.63) --
	(174.63, 41.63) --
	(174.70, 41.30) --
	(174.70, 41.30) --
	(174.70, 41.30) --
	(174.78, 40.87) --
	(174.78, 40.87) --
	(174.78, 40.87) --
	(174.82, 40.36) --
	(174.82, 40.36) --
	(174.82, 40.36) --
	(174.85, 40.02) --
	(174.85, 40.02) --
	(174.85, 40.02) --
	(174.92, 39.15) --
	(174.92, 39.15) --
	(174.92, 39.15) --
	(174.93, 39.05) --
	(174.93, 39.05) --
	(174.93, 39.05) --
	(175.00, 37.85) --
	(175.00, 37.85) --
	(175.00, 37.85) --
	(175.01, 37.73) --
	(175.01, 37.73) --
	(175.01, 37.73) --
	(175.07, 37.18) --
	(175.07, 37.18) --
	(175.07, 37.18) --
	(175.15, 35.38) --
	(175.15, 35.38) --
	(175.15, 35.38) --
	(175.22, 33.84) --
	(175.22, 33.84) --
	(175.22, 33.84) --
	(175.30, 32.92) --
	(175.30, 32.92) --
	(175.30, 32.92) --
	(175.30, 32.91) --
	(175.30, 32.91) --
	(175.30, 32.91) --
	(175.37, 32.59) --
	(175.37, 32.59) --
	(175.37, 32.59) --
	(175.45, 31.74) --
	(175.45, 31.74) --
	(175.45, 31.74) --
	(175.52, 30.97) --
	(175.52, 30.97) --
	(175.52, 30.97) --
	(175.59, 30.71) --
	(175.59, 30.71) --
	(175.59, 30.71) --
	(175.59, 30.70) --
	(175.59, 30.70) --
	(175.59, 30.70) --
	(175.67, 29.68) --
	(175.67, 29.68) --
	(175.67, 29.68) --
	(175.68, 29.47) --
	(175.68, 29.47) --
	(175.68, 29.47) --
	(175.74, 28.81) --
	(175.74, 28.81) --
	(175.74, 28.81) --
	(175.78, 28.66) --
	(175.78, 28.66) --
	(175.78, 28.66) --
	(175.82, 28.53) --
	(175.82, 28.53) --
	(175.82, 28.53) --
	(175.89, 28.48) --
	(175.89, 28.48) --
	(175.89, 28.48) --
	(175.96, 28.83) --
	(175.96, 28.83) --
	(175.96, 28.83) --
	(175.97, 28.94) --
	(175.97, 28.94) --
	(175.97, 28.94) --
	(176.04, 29.84) --
	(176.04, 29.84) --
	(176.04, 29.84) --
	(176.07, 30.40) --
	(176.07, 30.40) --
	(176.07, 30.40) --
	(176.11, 31.23) --
	(176.11, 31.23) --
	(176.11, 31.23) --
	(176.16, 31.27) --
	(176.16, 31.27) --
	(176.16, 31.27) --
	(176.19, 31.28) --
	(176.19, 31.28) --
	(176.19, 31.28) --
	(176.26, 29.42) --
	(176.26, 29.42) --
	(176.26, 29.42) --
	(176.33, 28.10) --
	(176.33, 28.10) --
	(176.33, 28.10) --
	(176.36, 28.06) --
	(176.36, 28.06) --
	(176.36, 28.06) --
	(176.41, 27.96) --
	(176.41, 27.96) --
	(176.41, 27.96) --
	(176.45, 27.95) --
	(176.45, 27.95) --
	(176.45, 27.95) --
	(176.48, 27.94) --
	(176.48, 27.94) --
	(176.48, 27.94) --
	(176.56, 27.92) --
	(176.56, 27.92) --
	(176.56, 27.92) --
	(176.63, 27.80) --
	(176.63, 27.80) --
	(176.63, 27.80) --
	(176.64, 27.78) --
	(176.64, 27.78) --
	(176.64, 27.78) --
	(176.70, 27.71) --
	(176.70, 27.71) --
	(176.70, 27.71) --
	(176.74, 27.66) --
	(176.74, 27.66) --
	(176.74, 27.66) --
	(176.78, 27.61) --
	(176.78, 27.61) --
	(176.78, 27.61) --
	(176.83, 27.58) --
	(176.83, 27.58) --
	(176.83, 27.58) --
	(176.85, 27.57) --
	(176.85, 27.57) --
	(176.85, 27.57) --
	(176.93, 27.62) --
	(176.93, 27.62) --
	(176.93, 27.62) --
	(177.00, 27.78) --
	(177.00, 27.78) --
	(177.00, 27.78) --
	(177.03, 27.81) --
	(177.03, 27.81) --
	(177.03, 27.81) --
	(177.07, 27.87) --
	(177.07, 27.87) --
	(177.07, 27.87) --
	(177.12, 27.91) --
	(177.12, 27.91) --
	(177.12, 27.91) --
	(177.15, 27.94) --
	(177.15, 27.94) --
	(177.15, 27.94) --
	(177.22, 28.20) --
	(177.22, 28.20) --
	(177.22, 28.20) --
	(177.30, 28.35) --
	(177.30, 28.35) --
	(177.30, 28.35) --
	(177.31, 28.37) --
	(177.31, 28.37) --
	(177.31, 28.37) --
	(177.37, 28.41) --
	(177.37, 28.41) --
	(177.37, 28.41) --
	(177.44, 28.14) --
	(177.44, 28.14) --
	(177.44, 28.14) --
	(177.51, 27.86) --
	(177.51, 27.86) --
	(177.51, 27.86) --
	(177.52, 27.80) --
	(177.52, 27.80) --
	(177.52, 27.80) --
	(177.59, 27.69) --
	(177.59, 27.69) --
	(177.59, 27.69) --
	(177.60, 27.69) --
	(177.60, 27.69) --
	(177.60, 27.69) --
	(177.67, 27.69) --
	(177.67, 27.69) --
	(177.67, 27.69) --
	(177.74, 27.77) --
	(177.74, 27.77) --
	(177.74, 27.77) --
	(177.79, 27.85) --
	(177.79, 27.85) --
	(177.79, 27.85) --
	(177.81, 27.88) --
	(177.81, 27.88) --
	(177.81, 27.88) --
	(177.89, 27.96) --
	(177.89, 27.96) --
	(177.89, 27.96) --
	(177.89, 27.96) --
	(177.89, 27.96) --
	(177.89, 27.96) --
	(177.96, 27.97) --
	(177.96, 27.97) --
	(177.96, 27.97) --
	(178.04, 27.97) --
	(178.04, 27.97) --
	(178.04, 27.97) --
	(178.11, 28.09) --
	(178.11, 28.09) --
	(178.11, 28.09) --
	(178.18, 28.07) --
	(178.18, 28.07) --
	(178.18, 28.07) --
	(178.18, 28.07) --
	(178.18, 28.07) --
	(178.18, 28.07) --
	(178.26, 28.04) --
	(178.26, 28.04) --
	(178.26, 28.04) --
	(178.33, 28.14) --
	(178.33, 28.14) --
	(178.33, 28.14) --
	(178.37, 28.26) --
	(178.37, 28.26) --
	(178.37, 28.26) --
	(178.41, 28.38) --
	(178.41, 28.38) --
	(178.41, 28.38) --
	(178.48, 28.34) --
	(178.48, 28.34) --
	(178.48, 28.34) --
	(178.55, 28.19) --
	(178.55, 28.19) --
	(178.55, 28.19) --
	(178.56, 28.18) --
	(178.56, 28.18) --
	(178.56, 28.18) --
	(178.63, 28.05) --
	(178.63, 28.05) --
	(178.63, 28.05) --
	(178.70, 27.92) --
	(178.70, 27.92) --
	(178.70, 27.92) --
	(178.75, 27.95) --
	(178.75, 27.95) --
	(178.75, 27.95) --
	(178.78, 27.96) --
	(178.78, 27.96) --
	(178.78, 27.96) --
	(178.85, 27.97) --
	(178.85, 27.97) --
	(178.85, 27.97) --
	(178.92, 27.98) --
	(178.92, 27.98) --
	(178.92, 27.98) --
	(178.94, 28.05) --
	(178.94, 28.05) --
	(178.94, 28.05) --
	(179.00, 28.22) --
	(179.00, 28.22) --
	(179.00, 28.22) --
	(179.07, 28.36) --
	(179.07, 28.36) --
	(179.07, 28.36) --
	(179.14, 28.00) --
	(179.14, 28.00) --
	(179.14, 28.00) --
	(179.22, 27.85) --
	(179.22, 27.85) --
	(179.22, 27.85) --
	(179.29, 27.96) --
	(179.29, 27.96) --
	(179.29, 27.96) --
	(179.33, 28.14) --
	(179.33, 28.14) --
	(179.33, 28.14) --
	(179.37, 28.35) --
	(179.37, 28.35) --
	(179.37, 28.35) --
	(179.44, 28.70) --
	(179.44, 28.70) --
	(179.44, 28.70) --
	(179.51, 28.94) --
	(179.51, 28.94) --
	(179.51, 28.94) --
	(179.59, 28.57) --
	(179.59, 28.57) --
	(179.59, 28.57) --
	(179.66, 28.24) --
	(179.66, 28.24) --
	(179.66, 28.24) --
	(179.71, 28.12) --
	(179.71, 28.12) --
	(179.71, 28.12) --
	(179.73, 28.06) --
	(179.73, 28.06) --
	(179.73, 28.06) --
	(179.81, 27.98) --
	(179.81, 27.98) --
	(179.81, 27.98) --
	(179.81, 27.98) --
	(179.81, 27.98) --
	(179.81, 27.98) --
	(179.88, 28.04) --
	(179.88, 28.04) --
	(179.88, 28.04) --
	(179.96, 28.11) --
	(179.96, 28.11) --
	(179.96, 28.11) --
	(180.03, 28.22) --
	(180.03, 28.22) --
	(180.03, 28.22) --
	(180.10, 28.21) --
	(180.10, 28.21) --
	(180.10, 28.21) --
	(180.18, 28.15) --
	(180.18, 28.15) --
	(180.18, 28.15) --
	(180.19, 28.17) --
	(180.19, 28.17) --
	(180.19, 28.17) --
	(180.25, 28.30) --
	(180.25, 28.30) --
	(180.25, 28.30) --
	(180.33, 28.55) --
	(180.33, 28.55) --
	(180.33, 28.55) --
	(180.40, 28.49) --
	(180.40, 28.49) --
	(180.40, 28.49) --
	(180.47, 28.21) --
	(180.47, 28.21) --
	(180.47, 28.21) --
	(180.55, 28.14) --
	(180.55, 28.14) --
	(180.55, 28.14) --
	(180.57, 28.13) --
	(180.57, 28.13) --
	(180.57, 28.13) --
	(180.62, 28.11) --
	(180.62, 28.11) --
	(180.62, 28.11) --
	(180.70, 28.15) --
	(180.70, 28.15) --
	(180.70, 28.15) --
	(180.77, 28.14) --
	(180.77, 28.14) --
	(180.77, 28.14) --
	(180.84, 28.06) --
	(180.84, 28.06) --
	(180.84, 28.06) --
	(180.92, 28.05) --
	(180.92, 28.05) --
	(180.92, 28.05) --
	(180.99, 28.16) --
	(180.99, 28.16) --
	(180.99, 28.16) --
	(181.05, 28.36) --
	(181.05, 28.36) --
	(181.05, 28.36) --
	(181.06, 28.40) --
	(181.06, 28.40) --
	(181.06, 28.40) --
	(181.14, 28.78) --
	(181.14, 28.78) --
	(181.14, 28.78) --
	(181.21, 29.10) --
	(181.21, 29.10) --
	(181.21, 29.10) --
	(181.28, 29.18) --
	(181.28, 29.18) --
	(181.28, 29.18) --
	(181.36, 29.68) --
	(181.36, 29.68) --
	(181.36, 29.68) --
	(181.43, 31.30) --
	(181.43, 31.30) --
	(181.43, 31.30) --
	(181.51, 33.20) --
	(181.51, 33.20) --
	(181.51, 33.20) --
	(181.53, 33.74) --
	(181.53, 33.74) --
	(181.53, 33.74) --
	(181.58, 34.86) --
	(181.58, 34.86) --
	(181.58, 34.86) --
	(181.65, 37.59) --
	(181.65, 37.59) --
	(181.65, 37.59) --
	(181.73, 37.84) --
	(181.73, 37.84) --
	(181.73, 37.84) --
	(181.80, 35.66) --
	(181.80, 35.66) --
	(181.80, 35.66) --
	(181.87, 35.90) --
	(181.87, 35.90) --
	(181.87, 35.90) --
	(181.95, 38.83) --
	(181.95, 38.83) --
	(181.95, 38.83) --
	(182.01, 39.82) --
	(182.01, 39.82) --
	(182.01, 39.82) --
	(182.02, 40.04) --
	(182.02, 40.04) --
	(182.02, 40.04) --
	(182.10, 37.40) --
	(182.10, 37.40) --
	(182.10, 37.40) --
	(182.17, 34.04) --
	(182.17, 34.04) --
	(182.17, 34.04) --
	(182.24, 34.80) --
	(182.24, 34.80) --
	(182.24, 34.80) --
	(182.32, 37.00) --
	(182.32, 37.00) --
	(182.32, 37.00) --
	(182.39, 39.90) --
	(182.39, 39.90) --
	(182.39, 39.90) --
	(182.46, 40.88) --
	(182.46, 40.88) --
	(182.46, 40.88) --
	(182.54, 40.16) --
	(182.54, 40.16) --
	(182.54, 40.16) --
	(182.58, 37.45) --
	(182.58, 37.45) --
	(182.58, 37.45) --
	(182.61, 35.83) --
	(182.61, 35.83) --
	(182.61, 35.83) --
	(182.68, 32.81) --
	(182.69, 32.81) --
	(182.69, 32.81) --
	(182.76, 33.86) --
	(182.76, 33.86) --
	(182.76, 33.86) --
	(182.83, 34.35) --
	(182.83, 34.35) --
	(182.83, 34.35) --
	(182.91, 35.66) --
	(182.91, 35.66) --
	(182.91, 35.66) --
	(182.98, 39.52) --
	(182.98, 39.52) --
	(182.98, 39.52) --
	(183.05, 41.20) --
	(183.05, 41.20) --
	(183.05, 41.20) --
	(183.13, 40.95) --
	(183.13, 40.95) --
	(183.13, 40.95) --
	(183.20, 37.20) --
	(183.20, 37.20) --
	(183.20, 37.20) --
	(183.27, 32.69) --
	(183.27, 32.69) --
	(183.27, 32.69) --
	(183.35, 31.56) --
	(183.35, 31.56) --
	(183.35, 31.56) --
	(183.35, 31.59) --
	(183.35, 31.59) --
	(183.35, 31.59) --
	(183.42, 32.53) --
	(183.42, 32.53) --
	(183.42, 32.53) --
	(183.50, 34.39) --
	(183.50, 34.39) --
	(183.50, 34.39) --
	(183.57, 33.64) --
	(183.57, 33.64) --
	(183.57, 33.64) --
	(183.64, 32.43) --
	(183.64, 32.43) --
	(183.64, 32.43) --
	(183.72, 32.92) --
	(183.72, 32.92) --
	(183.72, 32.92) --
	(183.79, 35.49) --
	(183.79, 35.49) --
	(183.79, 35.49) --
	(183.86, 37.25) --
	(183.86, 37.25) --
	(183.86, 37.25) --
	(183.94, 36.45) --
	(183.94, 36.45) --
	(183.94, 36.45) --
	(184.01, 33.64) --
	(184.01, 33.64) --
	(184.01, 33.64) --
	(184.09, 31.44) --
	(184.09, 31.44) --
	(184.09, 31.44) --
	(184.12, 31.21) --
	(184.12, 31.21) --
	(184.12, 31.21) --
	(184.16, 30.93) --
	(184.16, 30.93) --
	(184.16, 30.93) --
	(184.23, 30.73) --
	(184.23, 30.73) --
	(184.23, 30.73) --
	(184.30, 30.62) --
	(184.30, 30.62) --
	(184.30, 30.62) --
	(184.38, 32.21) --
	(184.38, 32.21) --
	(184.38, 32.21) --
	(184.41, 32.79) --
	(184.41, 32.79) --
	(184.41, 32.79) --
	(184.45, 33.81) --
	(184.45, 33.81) --
	(184.45, 33.81) --
	(184.53, 33.99) --
	(184.53, 33.99) --
	(184.53, 33.99) --
	(184.60, 33.74) --
	(184.60, 33.74) --
	(184.60, 33.74) --
	(184.67, 31.39) --
	(184.67, 31.39) --
	(184.67, 31.39) --
	(184.75, 30.41) --
	(184.75, 30.41) --
	(184.75, 30.41) --
	(184.82, 31.41) --
	(184.82, 31.41) --
	(184.82, 31.41) --
	(184.89, 32.16) --
	(184.89, 32.16) --
	(184.89, 32.16) --
	(184.97, 32.38) --
	(184.97, 32.38) --
	(184.97, 32.38) --
	(184.98, 32.43) --
	(184.98, 32.43) --
	(184.98, 32.43) --
	(185.04, 32.64) --
	(185.04, 32.64) --
	(185.04, 32.64) --
	(185.11, 31.80) --
	(185.11, 31.80) --
	(185.11, 31.80) --
	(185.19, 30.49) --
	(185.19, 30.49) --
	(185.19, 30.49) --
	(185.26, 30.46) --
	(185.26, 30.46) --
	(185.26, 30.46) --
	(185.27, 30.54) --
	(185.27, 30.54) --
	(185.27, 30.54) --
	(185.33, 31.30) --
	(185.33, 31.30) --
	(185.33, 31.30) --
	(185.41, 31.75) --
	(185.41, 31.75) --
	(185.41, 31.75) --
	(185.48, 32.90) --
	(185.48, 32.90) --
	(185.48, 32.90) --
	(185.55, 33.23) --
	(185.55, 33.23) --
	(185.55, 33.23) --
	(185.63, 31.54) --
	(185.63, 31.54) --
	(185.63, 31.54) --
	(185.70, 31.77) --
	(185.70, 31.77) --
	(185.70, 31.77) --
	(185.75, 32.80) --
	(185.75, 32.80) --
	(185.75, 32.80) --
	(185.78, 33.46) --
	(185.78, 33.46) --
	(185.78, 33.46) --
	(185.85, 34.38) --
	(185.85, 34.38) --
	(185.85, 34.38) --
	(185.92, 36.31) --
	(185.92, 36.31) --
	(185.92, 36.31) --
	(186.00, 37.97) --
	(186.00, 37.97) --
	(186.00, 37.97) --
	(186.03, 37.72) --
	(186.03, 37.72) --
	(186.03, 37.72) --
	(186.07, 37.48) --
	(186.07, 37.48) --
	(186.07, 37.48) --
	(186.14, 34.44) --
	(186.14, 34.44) --
	(186.14, 34.44) --
	(186.22, 33.95) --
	(186.22, 33.95) --
	(186.22, 33.95) --
	(186.29, 34.77) --
	(186.29, 34.77) --
	(186.29, 34.77) --
	(186.36, 38.49) --
	(186.36, 38.49) --
	(186.36, 38.49) --
	(186.44, 41.51) --
	(186.44, 41.51) --
	(186.44, 41.51) --
	(186.51, 41.67) --
	(186.51, 41.67) --
	(186.51, 41.67) --
	(186.58, 38.75) --
	(186.58, 38.75) --
	(186.58, 38.75) --
	(186.66, 33.03) --
	(186.66, 33.03) --
	(186.66, 33.03) --
	(186.71, 32.86) --
	(186.71, 32.86) --
	(186.71, 32.86) --
	(186.73, 32.77) --
	(186.73, 32.77) --
	(186.73, 32.77) --
	(186.80, 36.71) --
	(186.80, 36.71) --
	(186.80, 36.71) --
	(186.88, 38.98) --
	(186.88, 38.98) --
	(186.88, 38.98) --
	(186.95, 38.02) --
	(186.95, 38.02) --
	(186.95, 38.02) --
	(187.02, 35.64) --
	(187.02, 35.64) --
	(187.02, 35.64) --
	(187.10, 35.63) --
	(187.10, 35.63) --
	(187.10, 35.63) --
	(187.17, 35.94) --
	(187.17, 35.94) --
	(187.17, 35.94) --
	(187.18, 36.21) --
	(187.18, 36.21) --
	(187.18, 36.21) --
	(187.25, 37.42) --
	(187.25, 37.42) --
	(187.25, 37.42) --
	(187.32, 38.44) --
	(187.32, 38.44) --
	(187.32, 38.44) --
	(187.39, 41.04) --
	(187.39, 41.04) --
	(187.39, 41.04) --
	(187.46, 41.44) --
	(187.46, 41.44) --
	(187.46, 41.44) --
	(187.54, 40.85) --
	(187.54, 40.85) --
	(187.54, 40.85) --
	(187.61, 37.58) --
	(187.61, 37.58) --
	(187.61, 37.58) --
	(187.68, 32.91) --
	(187.68, 32.91) --
	(187.68, 32.91) --
	(187.76, 33.19) --
	(187.76, 33.19) --
	(187.76, 33.19) --
	(187.76, 33.21) --
	(187.76, 33.21) --
	(187.76, 33.21) --
	(187.83, 34.84) --
	(187.83, 34.84) --
	(187.83, 34.84) --
	(187.91, 36.96) --
	(187.91, 36.96) --
	(187.91, 36.96) --
	(187.98, 40.13) --
	(187.98, 40.13) --
	(187.98, 40.13) --
	(188.05, 42.59) --
	(188.05, 42.59) --
	(188.05, 42.59) --
	(188.13, 42.80) --
	(188.13, 42.80) --
	(188.13, 42.80) --
	(188.20, 42.38) --
	(188.20, 42.38) --
	(188.20, 42.38) --
	(188.27, 38.97) --
	(188.27, 38.97) --
	(188.27, 38.97) --
	(188.33, 36.82) --
	(188.33, 36.82) --
	(188.33, 36.82) --
	(188.34, 36.46) --
	(188.34, 36.46) --
	(188.34, 36.46) --
	(188.42, 33.37) --
	(188.42, 33.36) --
	(188.42, 33.36) --
	(188.49, 31.05) --
	(188.49, 31.05) --
	(188.49, 31.05) --
	(188.56, 31.36) --
	(188.56, 31.36) --
	(188.56, 31.36) --
	(188.64, 32.68) --
	(188.64, 32.68) --
	(188.64, 32.68) --
	(188.71, 33.37) --
	(188.71, 33.37) --
	(188.71, 33.37) --
	(188.79, 34.33) --
	(188.79, 34.33) --
	(188.79, 34.33) --
	(188.86, 37.24) --
	(188.86, 37.24) --
	(188.86, 37.24) --
	(188.93, 40.80) --
	(188.93, 40.80) --
	(188.93, 40.80) --
	(189.01, 42.73) --
	(189.01, 42.73) --
	(189.01, 42.73) --
	(189.08, 42.80) --
	(189.08, 42.80) --
	(189.08, 42.80) --
	(189.15, 42.80) --
	(189.15, 42.80) --
	(189.15, 42.80) --
	(189.22, 42.80) --
	(189.22, 42.80) --
	(189.22, 42.80) --
	(189.29, 42.80) --
	(189.29, 42.80) --
	(189.29, 42.80) --
	(189.30, 42.80) --
	(189.30, 42.80) --
	(189.30, 42.80) --
	(189.37, 42.80) --
	(189.37, 42.80) --
	(189.37, 42.80) --
	(189.44, 42.80) --
	(189.44, 42.80) --
	(189.44, 42.80) --
	(189.52, 42.80) --
	(189.52, 42.80) --
	(189.52, 42.80) --
	(189.59, 42.80) --
	(189.59, 42.80) --
	(189.59, 42.80) --
	(189.66, 42.80) --
	(189.66, 42.80) --
	(189.66, 42.80) --
	(189.74, 42.80) --
	(189.74, 42.80) --
	(189.74, 42.80) --
	(189.81, 42.80) --
	(189.81, 42.80) --
	(189.81, 42.80) --
	(189.88, 41.91) --
	(189.88, 41.91) --
	(189.88, 41.91) --
	(189.96, 41.56) --
	(189.96, 41.56) --
	(189.96, 41.56) --
	(190.03, 41.35) --
	(190.03, 41.35) --
	(190.03, 41.35) --
	(190.10, 41.48) --
	(190.10, 41.48) --
	(190.10, 41.48) --
	(190.18, 42.07) --
	(190.18, 42.07) --
	(190.18, 42.07) --
	(190.25, 41.99) --
	(190.25, 41.99) --
	(190.25, 41.99) --
	(190.25, 41.99) --
	(190.25, 41.99) --
	(190.25, 41.99) --
	(190.32, 41.88) --
	(190.32, 41.88) --
	(190.32, 41.88) --
	(190.40, 41.07) --
	(190.40, 41.07) --
	(190.40, 41.07) --
	(190.47, 38.82) --
	(190.47, 38.82) --
	(190.47, 38.82) --
	(190.54, 38.73) --
	(190.54, 38.73) --
	(190.54, 38.73) --
	(190.62, 40.59) --
	(190.62, 40.59) --
	(190.62, 40.59) --
	(190.69, 41.81) --
	(190.69, 41.81) --
	(190.69, 41.81) --
	(190.76, 42.09) --
	(190.76, 42.09) --
	(190.76, 42.09) --
	(190.84, 41.50) --
	(190.84, 41.50) --
	(190.84, 41.50) --
	(190.91, 41.80) --
	(190.91, 41.80) --
	(190.91, 41.80) --
	(190.98, 42.80) --
	(190.98, 42.80) --
	(190.98, 42.80) --
	(191.06, 42.80) --
	(191.06, 42.80) --
	(191.06, 42.80) --
	(191.13, 42.80) --
	(191.13, 42.80) --
	(191.13, 42.80) --
	(191.20, 42.80) --
	(191.20, 42.80) --
	(191.20, 42.80) --
	(191.27, 42.80) --
	(191.27, 42.80) --
	(191.27, 42.80) --
	(191.35, 42.80) --
	(191.35, 42.80) --
	(191.35, 42.80) --
	(191.42, 40.87) --
	(191.42, 40.87) --
	(191.42, 40.87) --
	(191.49, 40.35) --
	(191.49, 40.35) --
	(191.49, 40.34) --
	(191.57, 39.05) --
	(191.57, 39.05) --
	(191.57, 39.05) --
	(191.64, 38.47) --
	(191.64, 38.47) --
	(191.64, 38.47) --
	(191.71, 36.81) --
	(191.71, 36.81) --
	(191.71, 36.81) --
	(191.79, 36.74) --
	(191.79, 36.74) --
	(191.79, 36.74) --
	(191.86, 36.15) --
	(191.86, 36.15) --
	(191.86, 36.15) --
	(191.93, 34.57) --
	(191.93, 34.57) --
	(191.93, 34.57) --
	(192.01, 32.74) --
	(192.01, 32.74) --
	(192.01, 32.74) --
	(192.07, 31.09) --
	(192.07, 31.08) --
	(192.07, 31.08) --
	(192.08, 30.89) --
	(192.08, 30.89) --
	(192.08, 30.89) --
	(192.15, 30.37) --
	(192.15, 30.37) --
	(192.15, 30.37) --
	(192.23, 30.46) --
	(192.23, 30.46) --
	(192.23, 30.46) --
	(192.30, 30.63) --
	(192.30, 30.63) --
	(192.30, 30.63) --
	(192.37, 30.46) --
	(192.37, 30.46) --
	(192.37, 30.46) --
	(192.44, 30.11) --
	(192.44, 30.11) --
	(192.44, 30.11) --
	(192.52, 29.91) --
	(192.52, 29.91) --
	(192.52, 29.91) --
	(192.59, 29.89) --
	(192.59, 29.89) --
	(192.59, 29.89) --
	(192.66, 29.96) --
	(192.66, 29.96) --
	(192.66, 29.96) --
	(192.74, 30.27) --
	(192.74, 30.27) --
	(192.74, 30.27) --
	(192.81, 30.88) --
	(192.81, 30.88) --
	(192.81, 30.88) --
	(192.88, 32.04) --
	(192.88, 32.04) --
	(192.88, 32.04) --
	(192.96, 34.58) --
	(192.96, 34.58) --
	(192.96, 34.58) --
	(193.03, 35.93) --
	(193.03, 35.93) --
	(193.03, 35.93) --
	(193.10, 33.65) --
	(193.10, 33.65) --
	(193.10, 33.65) --
	(193.17, 33.23) --
	(193.17, 33.23) --
	(193.17, 33.23) --
	(193.25, 35.49) --
	(193.25, 35.49) --
	(193.25, 35.49) --
	(193.32, 33.15) --
	(193.32, 33.15) --
	(193.32, 33.15) --
	(193.39, 30.75) --
	(193.39, 30.75) --
	(193.39, 30.75) --
	(193.41, 30.86) --
	(193.41, 30.86) --
	(193.41, 30.86) --
	(193.47, 31.18) --
	(193.47, 31.18) --
	(193.47, 31.18) --
	(193.54, 31.59) --
	(193.54, 31.59) --
	(193.54, 31.59) --
	(193.61, 32.50) --
	(193.61, 32.50) --
	(193.61, 32.51) --
	(193.69, 36.90) --
	(193.69, 36.90) --
	(193.69, 36.90) --
	(193.76, 36.76) --
	(193.76, 36.76) --
	(193.76, 36.76) --
	(193.83, 32.12) --
	(193.83, 32.12) --
	(193.83, 32.12) --
	(193.91, 30.45) --
	(193.91, 30.45) --
	(193.91, 30.45) --
	(193.98, 29.53) --
	(193.98, 29.53) --
	(193.98, 29.53) --
	(194.05, 29.52) --
	(194.05, 29.52) --
	(194.05, 29.52) --
	(194.12, 29.44) --
	(194.12, 29.44) --
	(194.12, 29.44) --
	(194.20, 29.26) --
	(194.20, 29.26) --
	(194.20, 29.26) --
	(194.27, 29.41) --
	(194.27, 29.41) --
	(194.27, 29.41) --
	(194.34, 29.96) --
	(194.34, 29.96) --
	(194.34, 29.96) --
	(194.42, 30.65) --
	(194.42, 30.65) --
	(194.42, 30.65) --
	(194.49, 30.05) --
	(194.49, 30.05) --
	(194.49, 30.05) --
	(194.56, 29.22) --
	(194.56, 29.22) --
	(194.56, 29.22) --
	(194.63, 29.08) --
	(194.63, 29.08) --
	(194.63, 29.08) --
	(194.71, 29.16) --
	(194.71, 29.16) --
	(194.71, 29.16) --
	(194.78, 29.09) --
	(194.78, 29.09) --
	(194.78, 29.09) --
	(194.85, 29.08) --
	(194.85, 29.08) --
	(194.85, 29.08) --
	(194.93, 29.06) --
	(194.93, 29.06) --
	(194.93, 29.06) --
	(195.00, 29.10) --
	(195.00, 29.10) --
	(195.00, 29.10) --
	(195.07, 29.39) --
	(195.07, 29.39) --
	(195.07, 29.39) --
	(195.15, 29.78) --
	(195.15, 29.78) --
	(195.15, 29.78) --
	(195.22, 30.12) --
	(195.22, 30.12) --
	(195.22, 30.12) --
	(195.23, 30.07) --
	(195.23, 30.07) --
	(195.23, 30.07) --
	(195.29, 29.92) --
	(195.29, 29.92) --
	(195.29, 29.92) --
	(195.36, 29.60) --
	(195.36, 29.60) --
	(195.36, 29.60) --
	(195.44, 29.59) --
	(195.44, 29.59) --
	(195.44, 29.59) --
	(195.51, 29.83) --
	(195.51, 29.83) --
	(195.51, 29.83) --
	(195.58, 30.28) --
	(195.58, 30.28) --
	(195.58, 30.28) --
	(195.66, 30.86) --
	(195.66, 30.86) --
	(195.66, 30.86) --
	(195.73, 31.80) --
	(195.73, 31.80) --
	(195.73, 31.80) --
	(195.80, 33.98) --
	(195.80, 33.98) --
	(195.80, 33.98) --
	(195.87, 36.80) --
	(195.87, 36.80) --
	(195.87, 36.80) --
	(195.95, 39.04) --
	(195.95, 39.04) --
	(195.95, 39.04) --
	(196.02, 40.35) --
	(196.02, 40.35) --
	(196.02, 40.35) --
	(196.09, 41.11) --
	(196.09, 41.11) --
	(196.09, 41.11) --
	(196.17, 42.78) --
	(196.17, 42.78) --
	(196.17, 42.78) --
	(196.24, 42.80) --
	(196.24, 42.80) --
	(196.24, 42.80) --
	(196.31, 41.19) --
	(196.31, 41.19) --
	(196.31, 41.19) --
	(196.38, 40.76) --
	(196.38, 40.76) --
	(196.38, 40.76) --
	(196.46, 41.63) --
	(196.46, 41.63) --
	(196.46, 41.63) --
	(196.53, 42.80) --
	(196.53, 42.80) --
	(196.53, 42.80) --
	(196.60, 42.20) --
	(196.60, 42.20) --
	(196.60, 42.20) --
	(196.68, 40.40) --
	(196.68, 40.40) --
	(196.68, 40.40) --
	(196.75, 38.61) --
	(196.75, 38.61) --
	(196.75, 38.61) --
	(196.82, 36.93) --
	(196.82, 36.93) --
	(196.82, 36.93) --
	(196.89, 35.71) --
	(196.89, 35.71) --
	(196.89, 35.71) --
	(196.97, 34.99) --
	(196.97, 34.99) --
	(196.97, 34.99) --
	(197.04, 33.64) --
	(197.04, 33.64) --
	(197.04, 33.64) --
	(197.06, 33.34) --
	(197.06, 33.34) --
	(197.06, 33.34) --
	(197.11, 32.21) --
	(197.11, 32.21) --
	(197.11, 32.21) --
	(197.19, 30.21) --
	(197.19, 30.21) --
	(197.19, 30.21) --
	(197.26, 29.78) --
	(197.26, 29.78) --
	(197.26, 29.78) --
	(197.33, 29.50) --
	(197.33, 29.50) --
	(197.33, 29.50) --
	(197.40, 29.35) --
	(197.40, 29.35) --
	(197.40, 29.35) --
	(197.48, 29.29) --
	(197.48, 29.29) --
	(197.48, 29.29) --
	(197.55, 29.31) --
	(197.55, 29.31) --
	(197.55, 29.31) --
	(197.62, 29.49) --
	(197.62, 29.49) --
	(197.62, 29.49) --
	(197.69, 29.60) --
	(197.69, 29.60) --
	(197.69, 29.60) --
	(197.77, 29.29) --
	(197.77, 29.29) --
	(197.77, 29.29) --
	(197.84, 29.15) --
	(197.84, 29.15) --
	(197.84, 29.15) --
	(197.91, 29.62) --
	(197.91, 29.62) --
	(197.91, 29.62) --
	(197.99, 30.76) --
	(197.99, 30.76) --
	(197.99, 30.76) --
	(198.06, 33.31) --
	(198.06, 33.31) --
	(198.06, 33.31) --
	(198.11, 36.61) --
	(198.11, 36.61) --
	(198.11, 36.61) --
	(198.13, 38.09) --
	(198.13, 38.09) --
	(198.13, 38.09) --
	(198.20, 38.81) --
	(198.20, 38.81) --
	(198.20, 38.81) --
	(198.28, 34.87) --
	(198.28, 34.87) --
	(198.28, 34.87) --
	(198.35, 30.69) --
	(198.35, 30.69) --
	(198.35, 30.69) --
	(198.42, 29.29) --
	(198.42, 29.29) --
	(198.42, 29.29) --
	(198.50, 29.16) --
	(198.50, 29.16) --
	(198.50, 29.16) --
	(198.57, 29.09) --
	(198.57, 29.09) --
	(198.57, 29.09) --
	(198.64, 28.73) --
	(198.64, 28.73) --
	(198.64, 28.73) --
	(198.71, 28.69) --
	(198.71, 28.69) --
	(198.71, 28.69) --
	(198.79, 28.92) --
	(198.79, 28.92) --
	(198.79, 28.92) --
	(198.86, 28.84) --
	(198.86, 28.84) --
	(198.86, 28.84) --
	(198.93, 28.52) --
	(198.93, 28.52) --
	(198.93, 28.52) --
	(199.00, 28.37) --
	(199.00, 28.37) --
	(199.00, 28.37) --
	(199.08, 28.37) --
	(199.08, 28.37) --
	(199.08, 28.37) --
	(199.15, 28.40) --
	(199.15, 28.40) --
	(199.15, 28.40) --
	(199.22, 28.36) --
	(199.22, 28.36) --
	(199.22, 28.36) --
	(199.26, 28.32) --
	(199.26, 28.32) --
	(199.26, 28.32) --
	(199.29, 28.29) --
	(199.29, 28.29) --
	(199.29, 28.29) --
	(199.37, 28.27) --
	(199.37, 28.27) --
	(199.37, 28.27) --
	(199.44, 28.29) --
	(199.44, 28.29) --
	(199.44, 28.29) --
	(199.51, 28.46) --
	(199.51, 28.46) --
	(199.51, 28.46) --
	(199.59, 28.58) --
	(199.59, 28.58) --
	(199.59, 28.58) --
	(199.66, 28.78) --
	(199.66, 28.78) --
	(199.66, 28.78) --
	(199.73, 28.67) --
	(199.73, 28.67) --
	(199.73, 28.67) --
	(199.80, 28.41) --
	(199.80, 28.41) --
	(199.80, 28.41) --
	(199.88, 28.38) --
	(199.88, 28.38) --
	(199.88, 28.38) --
	(199.95, 28.52) --
	(199.95, 28.52) --
	(199.95, 28.52) --
	(200.02, 28.70) --
	(200.02, 28.70) --
	(200.02, 28.70) --
	(200.03, 28.70) --
	(200.03, 28.70) --
	(200.03, 28.70) --
	(200.09, 28.66) --
	(200.09, 28.66) --
	(200.09, 28.66) --
	(200.17, 28.61) --
	(200.17, 28.61) --
	(200.17, 28.62) --
	(200.24, 29.02) --
	(200.24, 29.02) --
	(200.24, 29.02) --
	(200.31, 29.36) --
	(200.31, 29.36) --
	(200.31, 29.36) --
	(200.38, 28.80) --
	(200.38, 28.80) --
	(200.38, 28.80) --
	(200.46, 28.24) --
	(200.46, 28.24) --
	(200.46, 28.24) --
	(200.53, 28.14) --
	(200.53, 28.14) --
	(200.53, 28.14) --
	(200.60, 28.14) --
	(200.60, 28.14) --
	(200.60, 28.14) --
	(200.67, 28.13) --
	(200.67, 28.13) --
	(200.67, 28.13) --
	(200.75, 28.20) --
	(200.75, 28.20) --
	(200.75, 28.20) --
	(200.79, 28.36) --
	(200.79, 28.36) --
	(200.79, 28.36) --
	(200.82, 28.45) --
	(200.82, 28.45) --
	(200.82, 28.45) --
	(200.89, 28.72) --
	(200.89, 28.72) --
	(200.89, 28.72) --
	(200.96, 29.23) --
	(200.96, 29.23) --
	(200.96, 29.23) --
	(201.04, 29.11) --
	(201.04, 29.11) --
	(201.04, 29.11) --
	(201.11, 28.53) --
	(201.11, 28.53) --
	(201.11, 28.53) --
	(201.18, 28.52) --
	(201.18, 28.52) --
	(201.18, 28.52) --
	(201.26, 28.45) --
	(201.26, 28.45) --
	(201.26, 28.45) --
	(201.33, 28.28) --
	(201.33, 28.28) --
	(201.33, 28.28) --
	(201.40, 28.20) --
	(201.40, 28.20) --
	(201.40, 28.20) --
	(201.47, 28.31) --
	(201.47, 28.31) --
	(201.47, 28.31) --
	(201.54, 28.57) --
	(201.54, 28.57) --
	(201.54, 28.57) --
	(201.62, 28.66) --
	(201.62, 28.66) --
	(201.62, 28.66) --
	(201.69, 28.31) --
	(201.69, 28.31) --
	(201.69, 28.31) --
	(201.76, 28.13) --
	(201.76, 28.13) --
	(201.76, 28.13) --
	(201.83, 28.24) --
	(201.83, 28.24) --
	(201.83, 28.24) --
	(201.91, 28.52) --
	(201.91, 28.52) --
	(201.91, 28.52) --
	(201.98, 28.91) --
	(201.98, 28.91) --
	(201.98, 28.91) --
	(202.04, 29.69) --
	(202.04, 29.69) --
	(202.04, 29.69) --
	(202.05, 29.87) --
	(202.05, 29.87) --
	(202.05, 29.87) --
	(202.12, 31.90) --
	(202.12, 31.90) --
	(202.12, 31.90) --
	(202.20, 30.99) --
	(202.20, 30.99) --
	(202.20, 30.99) --
	(202.27, 28.78) --
	(202.27, 28.78) --
	(202.27, 28.78) --
	(202.34, 28.21) --
	(202.34, 28.21) --
	(202.34, 28.21) --
	(202.41, 28.13) --
	(202.41, 28.13) --
	(202.41, 28.13) --
	(202.49, 28.14) --
	(202.49, 28.14) --
	(202.49, 28.14) --
	(202.56, 28.12) --
	(202.56, 28.12) --
	(202.56, 28.12) --
	(202.63, 28.16) --
	(202.63, 28.16) --
	(202.63, 28.16) --
	(202.70, 28.24) --
	(202.70, 28.24) --
	(202.70, 28.24) --
	(202.78, 28.19) --
	(202.78, 28.19) --
	(202.78, 28.19) --
	(202.81, 28.23) --
	(202.81, 28.23) --
	(202.81, 28.23) --
	(202.85, 28.29) --
	(202.85, 28.29) --
	(202.85, 28.29) --
	(202.92, 28.74) --
	(202.92, 28.74) --
	(202.92, 28.74) --
	(202.99, 29.74) --
	(202.99, 29.74) --
	(202.99, 29.74) --
	(203.07, 29.51) --
	(203.07, 29.51) --
	(203.07, 29.51) --
	(203.14, 28.46) --
	(203.14, 28.46) --
	(203.14, 28.46) --
	(203.21, 28.16) --
	(203.21, 28.16) --
	(203.21, 28.16) --
	(203.28, 28.15) --
	(203.28, 28.15) --
	(203.28, 28.15) --
	(203.36, 28.42) --
	(203.36, 28.42) --
	(203.36, 28.42) --
	(203.43, 28.70) --
	(203.43, 28.70) --
	(203.43, 28.70) --
	(203.50, 28.89) --
	(203.50, 28.89) --
	(203.50, 28.89) --
	(203.57, 29.60) --
	(203.57, 29.60) --
	(203.57, 29.60) --
	(203.64, 31.03) --
	(203.65, 31.03) --
	(203.65, 31.03) --
	(203.72, 30.39) --
	(203.72, 30.39) --
	(203.72, 30.39) --
	(203.76, 29.54) --
	(203.76, 29.54) --
	(203.76, 29.54) --
	(203.79, 29.05) --
	(203.79, 29.05) --
	(203.79, 29.05) --
	(203.86, 29.09) --
	(203.86, 29.09) --
	(203.86, 29.09) --
	(203.94, 29.60) --
	(203.94, 29.60) --
	(203.94, 29.60) --
	(204.01, 29.87) --
	(204.01, 29.87) --
	(204.01, 29.87) --
	(204.08, 29.19) --
	(204.08, 29.19) --
	(204.08, 29.19) --
	(204.15, 28.42) --
	(204.15, 28.42) --
	(204.15, 28.42) --
	(204.22, 28.25) --
	(204.22, 28.25) --
	(204.22, 28.25) --
	(204.30, 28.08) --
	(204.30, 28.08) --
	(204.30, 28.08) --
	(204.37, 27.96) --
	(204.37, 27.96) --
	(204.37, 27.96) --
	(204.44, 27.98) --
	(204.44, 27.98) --
	(204.44, 27.98) --
	(204.51, 27.99) --
	(204.51, 27.99) --
	(204.51, 27.99) --
	(204.59, 28.01) --
	(204.59, 28.01) --
	(204.59, 28.01) --
	(204.63, 28.12) --
	(204.63, 28.12) --
	(204.63, 28.12) --
	(204.66, 28.21) --
	(204.66, 28.21) --
	(204.66, 28.21) --
	(204.73, 28.54) --
	(204.73, 28.54) --
	(204.73, 28.54) --
	(204.80, 28.85) --
	(204.80, 28.85) --
	(204.80, 28.85) --
	(204.88, 28.45) --
	(204.88, 28.45) --
	(204.88, 28.45) --
	(204.95, 28.17) --
	(204.95, 28.17) --
	(204.95, 28.17) --
	(205.02, 28.44) --
	(205.02, 28.44) --
	(205.02, 28.44) --
	(205.09, 28.49) --
	(205.09, 28.49) --
	(205.09, 28.49) --
	(205.16, 28.21) --
	(205.16, 28.21) --
	(205.16, 28.21) --
	(205.24, 28.20) --
	(205.24, 28.20) --
	(205.24, 28.20) --
	(205.31, 28.66) --
	(205.31, 28.66) --
	(205.31, 28.66) --
	(205.38, 28.76) --
	(205.38, 28.76) --
	(205.38, 28.76) --
	(205.45, 28.25) --
	(205.45, 28.25) --
	(205.45, 28.25) --
	(205.53, 27.91) --
	(205.53, 27.91) --
	(205.53, 27.91) --
	(205.58, 27.89) --
	(205.58, 27.89) --
	(205.58, 27.89) --
	(205.60, 27.88) --
	(205.60, 27.88) --
	(205.60, 27.88) --
	(205.67, 27.92) --
	(205.67, 27.92) --
	(205.67, 27.92) --
	(205.74, 27.88) --
	(205.74, 27.88) --
	(205.74, 27.88) --
	(205.82, 27.80) --
	(205.82, 27.80) --
	(205.82, 27.80) --
	(205.89, 27.78) --
	(205.89, 27.78) --
	(205.89, 27.78) --
	(205.96, 27.74) --
	(205.96, 27.74) --
	(205.96, 27.74) --
	(206.03, 27.79) --
	(206.03, 27.79) --
	(206.03, 27.79) --
	(206.10, 27.92) --
	(206.10, 27.92) --
	(206.10, 27.92) --
	(206.18, 28.33) --
	(206.18, 28.33) --
	(206.18, 28.33) --
	(206.25, 29.38) --
	(206.25, 29.38) --
	(206.25, 29.38) --
	(206.32, 30.24) --
	(206.32, 30.24) --
	(206.32, 30.24) --
	(206.39, 30.19) --
	(206.39, 30.19) --
	(206.39, 30.19) --
	(206.46, 29.41) --
	(206.46, 29.41) --
	(206.46, 29.41) --
	(206.54, 29.07) --
	(206.54, 29.07) --
	(206.54, 29.07) --
	(206.54, 29.06) --
	(206.54, 29.06) --
	(206.54, 29.06) --
	(206.61, 28.98) --
	(206.61, 28.98) --
	(206.61, 28.98) --
	(206.68, 28.85) --
	(206.68, 28.85) --
	(206.68, 28.85) --
	(206.75, 28.82) --
	(206.75, 28.82) --
	(206.75, 28.82) --
	(206.83, 28.66) --
	(206.83, 28.66) --
	(206.83, 28.66) --
	(206.90, 28.55) --
	(206.90, 28.55) --
	(206.90, 28.55) --
	(206.97, 28.06) --
	(206.97, 28.06) --
	(206.97, 28.06) --
	(207.04, 27.73) --
	(207.04, 27.73) --
	(207.04, 27.73) --
	(207.12, 27.67) --
	(207.12, 27.67) --
	(207.12, 27.67) --
	(207.19, 27.75) --
	(207.19, 27.75) --
	(207.19, 27.75) --
	(207.26, 27.73) --
	(207.26, 27.73) --
	(207.26, 27.73) --
	(207.33, 27.70) --
	(207.33, 27.70) --
	(207.33, 27.70) --
	(207.40, 27.72) --
	(207.40, 27.72) --
	(207.40, 27.72) --
	(207.48, 27.69) --
	(207.48, 27.69) --
	(207.48, 27.69) --
	(207.55, 27.66) --
	(207.55, 27.66) --
	(207.55, 27.66) --
	(207.60, 27.66) --
	(207.60, 27.66) --
	(207.60, 27.66) --
	(207.62, 27.66) --
	(207.62, 27.66) --
	(207.62, 27.66) --
	(207.69, 27.72) --
	(207.69, 27.72) --
	(207.69, 27.72) --
	(207.76, 27.99) --
	(207.76, 27.99) --
	(207.76, 27.99) --
	(207.84, 28.97) --
	(207.84, 28.97) --
	(207.84, 28.97) --
	(207.91, 30.30) --
	(207.91, 30.30) --
	(207.91, 30.30) --
	(207.98, 29.52) --
	(207.98, 29.52) --
	(207.98, 29.52) --
	(208.05, 28.07) --
	(208.05, 28.07) --
	(208.05, 28.07) --
	(208.12, 27.60) --
	(208.12, 27.60) --
	(208.12, 27.60) --
	(208.20, 27.52) --
	(208.20, 27.52) --
	(208.20, 27.52) --
	(208.27, 27.53) --
	(208.27, 27.53) --
	(208.27, 27.53) --
	(208.34, 27.53) --
	(208.34, 27.53) --
	(208.34, 27.53) --
	(208.41, 27.50) --
	(208.41, 27.50) --
	(208.41, 27.50) --
	(208.46, 27.51) --
	(208.46, 27.51) --
	(208.46, 27.51) --
	(208.48, 27.52) --
	(208.48, 27.52) --
	(208.48, 27.52) --
	(208.56, 27.59) --
	(208.56, 27.59) --
	(208.56, 27.59) --
	(208.63, 27.85) --
	(208.63, 27.85) --
	(208.63, 27.85) --
	(208.70, 28.67) --
	(208.70, 28.67) --
	(208.70, 28.67) --
	(208.77, 29.84) --
	(208.77, 29.84) --
	(208.77, 29.84) --
	(208.84, 29.71) --
	(208.84, 29.71) --
	(208.84, 29.71) --
	(208.92, 28.53) --
	(208.92, 28.53) --
	(208.92, 28.53) --
	(208.99, 27.89) --
	(208.99, 27.89) --
	(208.99, 27.89) --
	(209.06, 27.82) --
	(209.06, 27.82) --
	(209.06, 27.82) --
	(209.13, 27.80) --
	(209.13, 27.80) --
	(209.13, 27.80) --
	(209.20, 27.72) --
	(209.20, 27.72) --
	(209.20, 27.72) --
	(209.28, 27.64) --
	(209.28, 27.64) --
	(209.28, 27.64) --
	(209.32, 27.66) --
	(209.32, 27.66) --
	(209.32, 27.66) --
	(209.35, 27.67) --
	(209.35, 27.67) --
	(209.35, 27.67) --
	(209.42, 28.00) --
	(209.42, 28.00) --
	(209.42, 28.00) --
	(209.49, 28.38) --
	(209.49, 28.38) --
	(209.49, 28.38) --
	(209.56, 28.12) --
	(209.56, 28.12) --
	(209.56, 28.12) --
	(209.64, 27.66) --
	(209.64, 27.66) --
	(209.64, 27.66) --
	(209.71, 27.52) --
	(209.71, 27.52) --
	(209.71, 27.52) --
	(209.78, 27.51) --
	(209.78, 27.51) --
	(209.78, 27.51) --
	(209.85, 27.46) --
	(209.85, 27.46) --
	(209.85, 27.46) --
	(209.92, 27.46) --
	(209.92, 27.46) --
	(209.92, 27.46) --
	(210.00, 27.44) --
	(210.00, 27.44) --
	(210.00, 27.44) --
	(210.07, 27.45) --
	(210.07, 27.45) --
	(210.07, 27.45) --
	(210.14, 27.49) --
	(210.14, 27.49) --
	(210.14, 27.49) --
	(210.21, 27.45) --
	(210.21, 27.45) --
	(210.21, 27.45) --
	(210.28, 27.38) --
	(210.28, 27.38) --
	(210.28, 27.38) --
	(210.29, 27.37) --
	(210.29, 27.37) --
	(210.29, 27.37) --
	(210.36, 27.32) --
	(210.36, 27.32) --
	(210.36, 27.32) --
	(210.43, 27.32) --
	(210.43, 27.32) --
	(210.43, 27.32) --
	(210.50, 27.36) --
	(210.50, 27.36) --
	(210.50, 27.36) --
	(210.57, 27.47) --
	(210.57, 27.47) --
	(210.57, 27.47) --
	(210.65, 27.54) --
	(210.65, 27.54) --
	(210.65, 27.54) --
	(210.72, 27.54) --
	(210.72, 27.54) --
	(210.72, 27.54) --
	(210.79, 27.83) --
	(210.79, 27.83) --
	(210.79, 27.83) --
	(210.86, 28.30) --
	(210.86, 28.30) --
	(210.86, 28.30) --
	(210.93, 28.05) --
	(210.93, 28.05) --
	(210.93, 28.05) --
	(211.00, 27.60) --
	(211.00, 27.60) --
	(211.00, 27.60) --
	(211.08, 27.47) --
	(211.08, 27.47) --
	(211.08, 27.47) --
	(211.15, 27.74) --
	(211.15, 27.74) --
	(211.15, 27.74) --
	(211.22, 28.95) --
	(211.22, 28.95) --
	(211.22, 28.95) --
	(211.29, 33.04) --
	(211.29, 33.04) --
	(211.29, 33.04) --
	(211.36, 35.68) --
	(211.36, 35.69) --
	(211.36, 35.68) --
	(211.43, 32.80) --
	(211.43, 32.80) --
	(211.43, 32.80) --
	(211.43, 32.61) --
	(211.43, 32.61) --
	(211.43, 32.61) --
	(211.51, 29.25) --
	(211.51, 29.25) --
	(211.51, 29.25) --
	(211.58, 28.02) --
	(211.58, 28.02) --
	(211.58, 28.02) --
	(211.65, 27.59) --
	(211.65, 27.59) --
	(211.65, 27.59) --
	(211.72, 27.43) --
	(211.72, 27.43) --
	(211.72, 27.43) --
	(211.79, 27.37) --
	(211.79, 27.37) --
	(211.79, 27.37) --
	(211.87, 27.34) --
	(211.87, 27.34) --
	(211.87, 27.34) --
	(211.94, 27.33) --
	(211.94, 27.33) --
	(211.94, 27.33) --
	(212.01, 27.37) --
	(212.01, 27.37) --
	(212.01, 27.37) --
	(212.08, 27.32) --
	(212.08, 27.32) --
	(212.08, 27.32) --
	(212.15, 27.26) --
	(212.15, 27.26) --
	(212.15, 27.26) --
	(212.23, 27.22) --
	(212.23, 27.22) --
	(212.23, 27.22) --
	(212.30, 27.21) --
	(212.30, 27.21) --
	(212.30, 27.21) --
	(212.37, 27.23) --
	(212.37, 27.23) --
	(212.37, 27.23) --
	(212.44, 27.20) --
	(212.44, 27.20) --
	(212.44, 27.20) --
	(212.51, 27.17) --
	(212.51, 27.17) --
	(212.51, 27.17) --
	(212.58, 27.18) --
	(212.58, 27.18) --
	(212.58, 27.18) --
	(212.66, 27.18) --
	(212.66, 27.18) --
	(212.66, 27.18) --
	(212.73, 27.18) --
	(212.73, 27.18) --
	(212.73, 27.18) --
	(212.80, 27.17) --
	(212.80, 27.17) --
	(212.80, 27.17) --
	(212.87, 27.17) --
	(212.87, 27.17) --
	(212.87, 27.17) --
	(212.94, 27.18) --
	(212.94, 27.18) --
	(212.94, 27.18) --
	(212.96, 27.19) --
	(212.96, 27.19) --
	(212.96, 27.19) --
	(213.02, 27.20) --
	(213.02, 27.20) --
	(213.02, 27.20) --
	(213.09, 27.19) --
	(213.09, 27.19) --
	(213.09, 27.19) --
	(213.16, 27.16) --
	(213.16, 27.16) --
	(213.16, 27.16) --
	(213.23, 27.17) --
	(213.23, 27.17) --
	(213.23, 27.17) --
	(213.30, 27.16) --
	(213.30, 27.16) --
	(213.30, 27.16) --
	(213.37, 27.16) --
	(213.37, 27.16) --
	(213.37, 27.16) --
	(213.45, 27.22) --
	(213.45, 27.22) --
	(213.45, 27.22) --
	(213.52, 27.26) --
	(213.52, 27.26) --
	(213.52, 27.26) --
	(213.59, 27.25) --
	(213.59, 27.25) --
	(213.59, 27.25) --
	(213.66, 27.22) --
	(213.66, 27.22) --
	(213.66, 27.22) --
	(213.73, 27.26) --
	(213.73, 27.26) --
	(213.73, 27.26) --
	(213.80, 27.24) --
	(213.80, 27.24) --
	(213.80, 27.24) --
	(213.88, 27.15) --
	(213.88, 27.15) --
	(213.88, 27.15) --
	(213.95, 27.12) --
	(213.95, 27.12) --
	(213.95, 27.12) --
	(214.02, 27.13) --
	(214.02, 27.13) --
	(214.02, 27.13) --
	(214.09, 27.10) --
	(214.09, 27.10) --
	(214.09, 27.10) --
	(214.16, 27.12) --
	(214.16, 27.12) --
	(214.16, 27.12) --
	(214.23, 27.13) --
	(214.23, 27.13) --
	(214.23, 27.13) --
	(214.31, 27.13) --
	(214.31, 27.13) --
	(214.31, 27.13) --
	(214.31, 27.13) --
	(214.31, 27.13) --
	(214.31, 27.13) --
	(214.38, 27.19) --
	(214.38, 27.19) --
	(214.38, 27.19) --
	(214.45, 27.45) --
	(214.45, 27.45) --
	(214.45, 27.45) --
	(214.52, 28.40) --
	(214.52, 28.40) --
	(214.52, 28.40) --
	(214.59, 29.23) --
	(214.59, 29.23) --
	(214.59, 29.23) --
	(214.66, 28.67) --
	(214.66, 28.67) --
	(214.66, 28.67) --
	(214.74, 27.78) --
	(214.74, 27.78) --
	(214.74, 27.78) --
	(214.81, 27.34) --
	(214.81, 27.34) --
	(214.81, 27.34) --
	(214.88, 27.16) --
	(214.88, 27.16) --
	(214.88, 27.16) --
	(214.95, 27.10) --
	(214.95, 27.10) --
	(214.95, 27.10) --
	(215.02, 27.08) --
	(215.02, 27.08) --
	(215.02, 27.08) --
	(215.09, 27.08) --
	(215.09, 27.08) --
	(215.09, 27.08) --
	(215.17, 27.13) --
	(215.17, 27.13) --
	(215.17, 27.13) --
	(215.24, 27.11) --
	(215.24, 27.11) --
	(215.24, 27.11) --
	(215.31, 27.04) --
	(215.31, 27.04) --
	(215.31, 27.04) --
	(215.38, 27.02) --
	(215.38, 27.02) --
	(215.38, 27.02) --
	(215.45, 26.98) --
	(215.45, 26.98) --
	(215.45, 26.98) --
	(215.52, 26.97) --
	(215.52, 26.97) --
	(215.52, 26.97) --
	(215.60, 26.99) --
	(215.60, 26.99) --
	(215.60, 26.99) --
	(215.67, 27.05) --
	(215.67, 27.05) --
	(215.67, 27.05) --
	(215.74, 27.09) --
	(215.74, 27.09) --
	(215.74, 27.09) --
	(215.74, 27.08) --
	(215.74, 27.08) --
	(215.74, 27.08) --
	(215.81, 27.04) --
	(215.81, 27.04) --
	(215.81, 27.04) --
	(215.88, 26.98) --
	(215.88, 26.98) --
	(215.88, 26.98) --
	(215.95, 26.98) --
	(215.95, 26.98) --
	(215.95, 26.98) --
	(216.03, 26.96) --
	(216.03, 26.96) --
	(216.03, 26.96) --
	(216.10, 26.96) --
	(216.10, 26.96) --
	(216.10, 26.96) --
	(216.17, 26.96) --
	(216.17, 26.96) --
	(216.17, 26.96) --
	(216.24, 26.98) --
	(216.24, 26.98) --
	(216.24, 26.98) --
	(216.31, 27.04) --
	(216.31, 27.04) --
	(216.31, 27.04) --
	(216.38, 27.06) --
	(216.38, 27.06) --
	(216.38, 27.06) --
	(216.46, 27.03) --
	(216.46, 27.03) --
	(216.46, 27.03) --
	(216.53, 26.97) --
	(216.53, 26.97) --
	(216.53, 26.97) --
	(216.60, 26.95) --
	(216.60, 26.95) --
	(216.60, 26.95) --
	(216.67, 26.95) --
	(216.67, 26.95) --
	(216.67, 26.95) --
	(216.74, 26.92) --
	(216.74, 26.92) --
	(216.74, 26.92) --
	(216.81, 26.91) --
	(216.81, 26.91) --
	(216.81, 26.91) --
	(216.88, 26.92) --
	(216.88, 26.92) --
	(216.88, 26.92) --
	(216.96, 26.93) --
	(216.96, 26.93) --
	(216.96, 26.93) --
	(217.03, 26.90) --
	(217.03, 26.90) --
	(217.03, 26.90) --
	(217.10, 26.91) --
	(217.10, 26.91) --
	(217.10, 26.91) --
	(217.17, 26.93) --
	(217.17, 26.93) --
	(217.17, 26.93) --
	(217.24, 26.92) --
	(217.24, 26.92) --
	(217.24, 26.92) --
	(217.31, 27.07) --
	(217.31, 27.07) --
	(217.31, 27.07) --
	(217.38, 27.38) --
	(217.38, 27.38) --
	(217.38, 27.38) --
	(217.46, 27.44) --
	(217.46, 27.44) --
	(217.46, 27.44) --
	(217.53, 27.28) --
	(217.53, 27.28) --
	(217.53, 27.28) --
	(217.60, 27.12) --
	(217.60, 27.12) --
	(217.60, 27.12) --
	(217.67, 26.95) --
	(217.67, 26.95) --
	(217.67, 26.95) --
	(217.74, 26.87) --
	(217.74, 26.87) --
	(217.74, 26.87) --
	(217.76, 26.86) --
	(217.76, 26.86) --
	(217.76, 26.86) --
	(217.81, 26.83) --
	(217.81, 26.83) --
	(217.81, 26.83) --
	(217.88, 26.84) --
	(217.88, 26.84) --
	(217.88, 26.84) --
	(217.96, 26.87) --
	(217.96, 26.87) --
	(217.96, 26.87) --
	(218.03, 26.89) --
	(218.03, 26.89) --
	(218.03, 26.89) --
	(218.10, 26.90) --
	(218.10, 26.90) --
	(218.10, 26.90) --
	(218.17, 26.94) --
	(218.17, 26.94) --
	(218.17, 26.94) --
	(218.24, 26.91) --
	(218.24, 26.91) --
	(218.24, 26.91) --
	(218.31, 26.88) --
	(218.31, 26.88) --
	(218.31, 26.88) --
	(218.38, 26.88) --
	(218.38, 26.88) --
	(218.38, 26.88) --
	(218.46, 26.85) --
	(218.46, 26.85) --
	(218.46, 26.85) --
	(218.53, 26.82) --
	(218.53, 26.82) --
	(218.53, 26.82) --
	(218.60, 26.83) --
	(218.60, 26.83) --
	(218.60, 26.83) --
	(218.67, 26.90) --
	(218.67, 26.90) --
	(218.67, 26.90) --
	(218.74, 26.93) --
	(218.74, 26.93) --
	(218.74, 26.93) --
	(218.81, 26.85) --
	(218.81, 26.85) --
	(218.81, 26.85) --
	(218.88, 26.81) --
	(218.88, 26.81) --
	(218.88, 26.81) --
	(218.95, 26.86) --
	(218.95, 26.86) --
	(218.95, 26.86) --
	(219.03, 27.07) --
	(219.03, 27.07) --
	(219.03, 27.07) --
	(219.10, 27.26) --
	(219.10, 27.26) --
	(219.10, 27.26) --
	(219.17, 27.09) --
	(219.17, 27.09) --
	(219.17, 27.09) --
	(219.24, 26.88) --
	(219.24, 26.88) --
	(219.24, 26.88) --
	(219.31, 26.86) --
	(219.31, 26.86) --
	(219.31, 26.86) --
	(219.38, 26.98) --
	(219.38, 26.98) --
	(219.38, 26.98) --
	(219.46, 26.97) --
	(219.46, 26.97) --
	(219.46, 26.97) --
	(219.48, 26.93) --
	(219.48, 26.93) --
	(219.48, 26.93) --
	(219.53, 26.86) --
	(219.53, 26.86) --
	(219.53, 26.86) --
	(219.60, 26.79) --
	(219.60, 26.79) --
	(219.60, 26.79) --
	(219.67, 26.79) --
	(219.67, 26.79) --
	(219.67, 26.79) --
	(219.74, 26.79) --
	(219.74, 26.79) --
	(219.74, 26.79) --
	(219.81, 26.79) --
	(219.81, 26.79) --
	(219.81, 26.79) --
	(219.88, 26.84) --
	(219.88, 26.84) --
	(219.88, 26.84) --
	(219.95, 26.83) --
	(219.95, 26.83) --
	(219.95, 26.83) --
	(220.03, 26.76) --
	(220.03, 26.76) --
	(220.03, 26.76) --
	(220.10, 26.71) --
	(220.10, 26.71) --
	(220.10, 26.71) --
	(220.17, 26.67) --
	(220.17, 26.67) --
	(220.17, 26.67) --
	(220.24, 26.68) --
	(220.24, 26.68) --
	(220.24, 26.68) --
	(220.31, 26.68) --
	(220.31, 26.68) --
	(220.31, 26.68) --
	(220.38, 26.62) --
	(220.38, 26.62) --
	(220.38, 26.62) --
	(220.45, 26.63) --
	(220.45, 26.63) --
	(220.45, 26.63) --
	(220.52, 26.64) --
	(220.52, 26.64) --
	(220.52, 26.64) --
	(220.60, 26.63) --
	(220.60, 26.63) --
	(220.60, 26.63) --
	(220.67, 26.66) --
	(220.67, 26.66) --
	(220.67, 26.66) --
	(220.74, 26.68) --
	(220.74, 26.68) --
	(220.74, 26.68) --
	(220.81, 26.68) --
	(220.81, 26.68) --
	(220.81, 26.68) --
	(220.88, 26.65) --
	(220.88, 26.65) --
	(220.88, 26.65) --
	(220.95, 26.64) --
	(220.95, 26.64) --
	(220.95, 26.64) --
	(221.02, 26.66) --
	(221.02, 26.66) --
	(221.02, 26.66) --
	(221.09, 26.65) --
	(221.09, 26.65) --
	(221.09, 26.65) --
	(221.17, 26.60) --
	(221.17, 26.60) --
	(221.17, 26.60) --
	(221.24, 26.61) --
	(221.24, 26.61) --
	(221.24, 26.61) --
	(221.31, 26.59) --
	(221.31, 26.59) --
	(221.31, 26.59) --
	(221.38, 26.57) --
	(221.38, 26.57) --
	(221.38, 26.57) --
	(221.45, 26.56) --
	(221.45, 26.56) --
	(221.45, 26.56) --
	(221.49, 26.56) --
	(221.49, 26.56) --
	(221.49, 26.56) --
	(221.52, 26.56) --
	(221.52, 26.56) --
	(221.52, 26.56) --
	(221.59, 26.56) --
	(221.59, 26.56) --
	(221.59, 26.56) --
	(221.66, 26.57) --
	(221.66, 26.57) --
	(221.66, 26.57) --
	(221.74, 26.58) --
	(221.74, 26.58) --
	(221.74, 26.58) --
	(221.81, 26.58) --
	(221.81, 26.58) --
	(221.81, 26.58) --
	(221.88, 26.58) --
	(221.88, 26.58) --
	(221.88, 26.58) --
	(221.95, 26.56) --
	(221.95, 26.56) --
	(221.95, 26.56) --
	(222.02, 26.56) --
	(222.02, 26.56) --
	(222.02, 26.56) --
	(222.09, 26.56) --
	(222.09, 26.56) --
	(222.09, 26.56) --
	(222.16, 26.56) --
	(222.16, 26.56) --
	(222.16, 26.56) --
	(222.23, 26.59) --
	(222.23, 26.59) --
	(222.23, 26.59) --
	(222.30, 26.68) --
	(222.30, 26.68) --
	(222.30, 26.68) --
	(222.37, 26.68) --
	(222.37, 26.68) --
	(222.37, 26.68) --
	(222.45, 26.64) --
	(222.45, 26.64) --
	(222.45, 26.64) --
	(222.52, 26.64) --
	(222.52, 26.64) --
	(222.52, 26.64) --
	(222.59, 26.65) --
	(222.59, 26.65) --
	(222.59, 26.65) --
	(222.66, 26.71) --
	(222.66, 26.71) --
	(222.66, 26.71) --
	(222.73, 26.71) --
	(222.73, 26.71) --
	(222.73, 26.71) --
	(222.80, 26.69) --
	(222.80, 26.69) --
	(222.80, 26.69) --
	(222.87, 26.64) --
	(222.87, 26.64) --
	(222.87, 26.64) --
	(222.94, 26.58) --
	(222.94, 26.58) --
	(222.94, 26.58) --
	(223.02, 26.57) --
	(223.02, 26.57) --
	(223.02, 26.57) --
	(223.09, 26.56) --
	(223.09, 26.56) --
	(223.09, 26.56) --
	(223.16, 26.61) --
	(223.16, 26.61) --
	(223.16, 26.61) --
	(223.23, 26.72) --
	(223.23, 26.72) --
	(223.23, 26.72) --
	(223.30, 26.71) --
	(223.30, 26.71) --
	(223.30, 26.71) --
	(223.37, 26.63) --
	(223.37, 26.63) --
	(223.37, 26.63) --
	(223.44, 26.57) --
	(223.44, 26.57) --
	(223.44, 26.57) --
	(223.51, 26.54) --
	(223.51, 26.54) --
	(223.51, 26.54) --
	(223.58, 26.57) --
	(223.58, 26.57) --
	(223.58, 26.57) --
	(223.66, 26.59) --
	(223.66, 26.59) --
	(223.66, 26.59) --
	(223.73, 26.56) --
	(223.73, 26.56) --
	(223.73, 26.56) --
	(223.79, 26.53) --
	(223.79, 26.53) --
	(223.79, 26.53) --
	(223.80, 26.53) --
	(223.80, 26.53) --
	(223.80, 26.53) --
	(223.87, 26.52) --
	(223.87, 26.52) --
	(223.87, 26.52) --
	(223.94, 26.58) --
	(223.94, 26.58) --
	(223.94, 26.58) --
	(224.01, 26.75) --
	(224.01, 26.75) --
	(224.01, 26.75) --
	(224.08, 26.79) --
	(224.08, 26.79) --
	(224.08, 26.79) --
	(224.15, 26.70) --
	(224.15, 26.70) --
	(224.15, 26.70) --
	(224.22, 26.75) --
	(224.22, 26.75) --
	(224.22, 26.75) --
	(224.29, 26.93) --
	(224.29, 26.93) --
	(224.29, 26.93) --
	(224.36, 26.88) --
	(224.36, 26.88) --
	(224.36, 26.88) --
	(224.44, 26.69) --
	(224.44, 26.69) --
	(224.44, 26.69) --
	(224.51, 26.55) --
	(224.51, 26.55) --
	(224.51, 26.55) --
	(224.58, 26.53) --
	(224.58, 26.53) --
	(224.58, 26.53) --
	(224.65, 26.53) --
	(224.65, 26.53) --
	(224.65, 26.53) --
	(224.72, 26.50) --
	(224.72, 26.50) --
	(224.72, 26.50) --
	(224.79, 26.49) --
	(224.79, 26.49) --
	(224.79, 26.49) --
	(224.86, 26.48) --
	(224.86, 26.48) --
	(224.86, 26.48) --
	(224.93, 26.46) --
	(224.93, 26.46) --
	(224.93, 26.46) --
	(225.00, 26.47) --
	(225.00, 26.47) --
	(225.00, 26.47) --
	(225.07, 26.45) --
	(225.07, 26.45) --
	(225.07, 26.45) --
	(225.15, 26.47) --
	(225.15, 26.47) --
	(225.15, 26.47) --
	(225.22, 26.48) --
	(225.22, 26.48) --
	(225.22, 26.48) --
	(225.29, 26.46) --
	(225.29, 26.46) --
	(225.29, 26.46) --
	(225.36, 26.45) --
	(225.36, 26.45) --
	(225.36, 26.45) --
	(225.43, 26.51) --
	(225.43, 26.51) --
	(225.43, 26.51) --
	(225.50, 26.58) --
	(225.50, 26.58) --
	(225.50, 26.58) --
	(225.57, 26.57) --
	(225.57, 26.57) --
	(225.57, 26.57) --
	(225.61, 26.52) --
	(225.61, 26.52) --
	(225.61, 26.52) --
	(225.64, 26.50) --
	(225.64, 26.50) --
	(225.64, 26.50) --
	(225.71, 26.44) --
	(225.71, 26.44) --
	(225.71, 26.44) --
	(225.78, 26.43) --
	(225.78, 26.43) --
	(225.78, 26.43) --
	(225.85, 26.42) --
	(225.85, 26.42) --
	(225.85, 26.42) --
	(225.93, 26.42) --
	(225.93, 26.42) --
	(225.93, 26.42) --
	(226.00, 26.43) --
	(226.00, 26.43) --
	(226.00, 26.43) --
	(226.07, 26.42) --
	(226.07, 26.42) --
	(226.07, 26.42) --
	(226.14, 26.43) --
	(226.14, 26.43) --
	(226.14, 26.43) --
	(226.21, 26.41) --
	(226.21, 26.41) --
	(226.21, 26.41) --
	(226.28, 26.40) --
	(226.28, 26.40) --
	(226.28, 26.40) --
	(226.35, 26.39) --
	(226.35, 26.39) --
	(226.35, 26.39) --
	(226.42, 26.39) --
	(226.42, 26.39) --
	(226.42, 26.39) --
	(226.49, 26.38) --
	(226.49, 26.38) --
	(226.49, 26.38) --
	(226.56, 26.40) --
	(226.56, 26.40) --
	(226.56, 26.40) --
	(226.63, 26.44) --
	(226.63, 26.44) --
	(226.63, 26.44) --
	(226.71, 26.64) --
	(226.71, 26.64) --
	(226.71, 26.64) --
	(226.78, 26.91) --
	(226.78, 26.91) --
	(226.78, 26.91) --
	(226.85, 26.90) --
	(226.85, 26.90) --
	(226.85, 26.90) --
	(226.92, 26.70) --
	(226.92, 26.70) --
	(226.92, 26.70) --
	(226.99, 26.53) --
	(226.99, 26.53) --
	(226.99, 26.53) --
	(227.06, 26.44) --
	(227.06, 26.44) --
	(227.06, 26.44) --
	(227.13, 26.41) --
	(227.13, 26.41) --
	(227.13, 26.41) --
	(227.20, 26.42) --
	(227.20, 26.42) --
	(227.20, 26.42) --
	(227.27, 26.55) --
	(227.27, 26.55) --
	(227.27, 26.55) --
	(227.34, 26.68) --
	(227.34, 26.68) --
	(227.34, 26.68) --
	(227.41, 26.64) --
	(227.41, 26.64) --
	(227.41, 26.64) --
	(227.48, 26.54) --
	(227.48, 26.54) --
	(227.48, 26.54) --
	(227.56, 26.45) --
	(227.56, 26.45) --
	(227.56, 26.45) --
	(227.62, 26.40) --
	(227.62, 26.40) --
	(227.62, 26.40) --
	(227.70, 26.39) --
	(227.70, 26.39) --
	(227.70, 26.39) --
	(227.77, 26.42) --
	(227.77, 26.42) --
	(227.77, 26.42) --
	(227.84, 26.45) --
	(227.84, 26.45) --
	(227.84, 26.45) --
	(227.91, 26.44) --
	(227.91, 26.44) --
	(227.91, 26.44) --
	(227.91, 26.44) --
	(227.91, 26.44) --
	(227.91, 26.44) --
	(227.98, 26.40) --
	(227.98, 26.40) --
	(227.98, 26.40) --
	(228.05, 26.39) --
	(228.05, 26.39) --
	(228.05, 26.39) --
	(228.12, 26.40) --
	(228.12, 26.40) --
	(228.12, 26.40) --
	(228.19, 26.38) --
	(228.19, 26.38) --
	(228.19, 26.38) --
	(228.26, 26.33) --
	(228.26, 26.33) --
	(228.26, 26.33) --
	(228.33, 26.32) --
	(228.33, 26.32) --
	(228.33, 26.32) --
	(228.40, 26.31) --
	(228.40, 26.31) --
	(228.40, 26.31) --
	(228.47, 26.30) --
	(228.47, 26.30) --
	(228.47, 26.30) --
	(228.55, 26.30) --
	(228.55, 26.30) --
	(228.55, 26.30) --
	(228.62, 26.35) --
	(228.62, 26.35) --
	(228.62, 26.35) --
	(228.69, 26.36) --
	(228.69, 26.36) --
	(228.69, 26.36) --
	(228.76, 26.34) --
	(228.76, 26.34) --
	(228.76, 26.34) --
	(228.83, 26.29) --
	(228.83, 26.29) --
	(228.83, 26.29) --
	(228.90, 26.28) --
	(228.90, 26.28) --
	(228.90, 26.28) --
	(228.97, 26.28) --
	(228.97, 26.28) --
	(228.97, 26.28) --
	(229.04, 26.27) --
	(229.04, 26.27) --
	(229.04, 26.27) --
	(229.11, 26.28) --
	(229.11, 26.28) --
	(229.11, 26.28) --
	(229.18, 26.27) --
	(229.18, 26.27) --
	(229.18, 26.27) --
	(229.25, 26.27) --
	(229.25, 26.27) --
	(229.25, 26.27) --
	(229.32, 26.27) --
	(229.32, 26.27) --
	(229.32, 26.27) --
	(229.39, 26.28) --
	(229.39, 26.28) --
	(229.39, 26.28) --
	(229.46, 26.32) --
	(229.46, 26.32) --
	(229.46, 26.32) --
	(229.53, 26.33) --
	(229.53, 26.33) --
	(229.53, 26.33) --
	(229.60, 26.30) --
	(229.60, 26.30) --
	(229.60, 26.30) --
	(229.68, 26.27) --
	(229.68, 26.27) --
	(229.68, 26.27) --
	(229.75, 26.24) --
	(229.75, 26.24) --
	(229.75, 26.24) --
	(229.82, 26.23) --
	(229.82, 26.23) --
	(229.82, 26.23) --
	(229.89, 26.23) --
	(229.89, 26.23) --
	(229.89, 26.23) --
	(229.96, 26.21) --
	(229.96, 26.21) --
	(229.96, 26.21) --
	(230.03, 26.22) --
	(230.03, 26.22) --
	(230.03, 26.22) --
	(230.10, 26.21) --
	(230.10, 26.21) --
	(230.10, 26.21) --
	(230.17, 26.21) --
	(230.17, 26.21) --
	(230.17, 26.21) --
	(230.24, 26.21) --
	(230.24, 26.21) --
	(230.24, 26.21) --
	(230.31, 26.21) --
	(230.31, 26.21) --
	(230.31, 26.21) --
	(230.38, 26.20) --
	(230.38, 26.20) --
	(230.38, 26.20) --
	(230.45, 26.19) --
	(230.45, 26.19) --
	(230.45, 26.19) --
	(230.50, 26.20) --
	(230.50, 26.20) --
	(230.50, 26.20) --
	(230.52, 26.20) --
	(230.52, 26.20) --
	(230.52, 26.20) --
	(230.59, 26.19) --
	(230.59, 26.19) --
	(230.59, 26.19) --
	(230.66, 26.20) --
	(230.66, 26.20) --
	(230.66, 26.20) --
	(230.73, 26.20) --
	(230.73, 26.20) --
	(230.73, 26.20) --
	(230.80, 26.20) --
	(230.80, 26.20) --
	(230.80, 26.20) --
	(230.88, 26.19) --
	(230.88, 26.19) --
	(230.88, 26.19) --
	(230.95, 26.18) --
	(230.95, 26.18) --
	(230.95, 26.18) --
	(231.02, 26.20) --
	(231.02, 26.20) --
	(231.02, 26.20) --
	(231.09, 26.23) --
	(231.09, 26.23) --
	(231.09, 26.23) --
	(231.16, 26.27) --
	(231.16, 26.27) --
	(231.16, 26.27) --
	(231.23, 26.32) --
	(231.23, 26.32) --
	(231.23, 26.32) --
	(231.30, 26.33) --
	(231.30, 26.33) --
	(231.30, 26.33) --
	(231.37, 26.30) --
	(231.37, 26.30) --
	(231.37, 26.30) --
	(231.44, 26.26) --
	(231.44, 26.26) --
	(231.44, 26.26) --
	(231.51, 26.21) --
	(231.51, 26.21) --
	(231.51, 26.21) --
	(231.58, 26.18) --
	(231.58, 26.18) --
	(231.58, 26.18) --
	(231.65, 26.18) --
	(231.65, 26.18) --
	(231.65, 26.18) --
	(231.72, 26.16) --
	(231.72, 26.16) --
	(231.72, 26.16) --
	(231.79, 26.17) --
	(231.79, 26.17) --
	(231.79, 26.17) --
	(231.86, 26.17) --
	(231.86, 26.17) --
	(231.86, 26.17) --
	(231.93, 26.15) --
	(231.93, 26.15) --
	(231.93, 26.15) --
	(232.00, 26.16) --
	(232.00, 26.16) --
	(232.00, 26.16) --
	(232.07, 26.19) --
	(232.07, 26.19) --
	(232.07, 26.19) --
	(232.14, 26.20) --
	(232.14, 26.20) --
	(232.14, 26.20) --
	(232.21, 26.23) --
	(232.21, 26.23) --
	(232.21, 26.23) --
	(232.28, 26.22) --
	(232.28, 26.22) --
	(232.28, 26.22) --
	(232.36, 26.19) --
	(232.36, 26.19) --
	(232.36, 26.19) --
	(232.43, 26.17) --
	(232.43, 26.17) --
	(232.43, 26.17) --
	(232.50, 26.14) --
	(232.50, 26.14) --
	(232.50, 26.14) --
	(232.57, 26.15) --
	(232.57, 26.15) --
	(232.57, 26.15) --
	(232.64, 26.16) --
	(232.64, 26.16) --
	(232.64, 26.16) --
	(232.71, 26.16) --
	(232.71, 26.16) --
	(232.71, 26.16) --
	(232.78, 26.19) --
	(232.78, 26.19) --
	(232.78, 26.19) --
	(232.85, 26.28) --
	(232.85, 26.28) --
	(232.85, 26.28) --
	(232.92, 26.35) --
	(232.92, 26.35) --
	(232.92, 26.35) --
	(232.99, 26.37) --
	(232.99, 26.37) --
	(232.99, 26.37) --
	(233.06, 26.27) --
	(233.06, 26.27) --
	(233.06, 26.27) --
	(233.13, 26.20) --
	(233.13, 26.20) --
	(233.13, 26.20) --
	(233.20, 26.16) --
	(233.20, 26.16) --
	(233.20, 26.16) --
	(233.27, 26.13) --
	(233.27, 26.13) --
	(233.27, 26.13) --
	(233.28, 26.13) --
	(233.28, 26.13) --
	(233.28, 26.13) --
	(233.34, 26.13) --
	(233.34, 26.13) --
	(233.34, 26.13) --
	(233.41, 26.13) --
	(233.41, 26.13) --
	(233.41, 26.13) --
	(233.48, 26.13) --
	(233.48, 26.13) --
	(233.48, 26.13) --
	(233.55, 26.15) --
	(233.55, 26.15) --
	(233.55, 26.15) --
	(233.62, 26.15) --
	(233.62, 26.15) --
	(233.62, 26.15) --
	(233.69, 26.15) --
	(233.69, 26.15) --
	(233.69, 26.15) --
	(233.76, 26.14) --
	(233.76, 26.14) --
	(233.76, 26.14) --
	(233.83, 26.12) --
	(233.83, 26.12) --
	(233.83, 26.12) --
	(233.90, 26.11) --
	(233.90, 26.11) --
	(233.90, 26.11) --
	(233.97, 26.12) --
	(233.97, 26.12) --
	(233.97, 26.12) --
	(234.04, 26.10) --
	(234.04, 26.10) --
	(234.04, 26.10) --
	(234.11, 26.11) --
	(234.11, 26.11) --
	(234.11, 26.11) --
	(234.18, 26.10) --
	(234.18, 26.10) --
	(234.18, 26.10) --
	(234.25, 26.10) --
	(234.26, 26.10) --
	(234.26, 26.10) --
	(234.32, 26.10) --
	(234.32, 26.10) --
	(234.32, 26.10) --
	(234.40, 26.10) --
	(234.40, 26.10) --
	(234.40, 26.10) --
	(234.47, 26.09) --
	(234.47, 26.09) --
	(234.47, 26.09) --
	(234.54, 26.09) --
	(234.54, 26.09) --
	(234.54, 26.09) --
	(234.61, 26.09) --
	(234.61, 26.09) --
	(234.61, 26.09) --
	(234.68, 26.10) --
	(234.68, 26.10) --
	(234.68, 26.10) --
	(234.75, 26.10) --
	(234.75, 26.10) --
	(234.75, 26.10) --
	(234.82, 26.10) --
	(234.82, 26.10) --
	(234.82, 26.10) --
	(234.89, 26.09) --
	(234.89, 26.09) --
	(234.89, 26.09) --
	(234.96, 26.10) --
	(234.96, 26.10) --
	(234.96, 26.10) --
	(235.03, 26.08) --
	(235.03, 26.08) --
	(235.03, 26.08) --
	(235.10, 26.09) --
	(235.10, 26.09) --
	(235.10, 26.09) --
	(235.17, 26.10) --
	(235.17, 26.10) --
	(235.17, 26.10) --
	(235.24, 26.11) --
	(235.24, 26.11) --
	(235.24, 26.11) --
	(235.31, 26.11) --
	(235.31, 26.11) --
	(235.31, 26.11) --
	(235.38, 26.11) --
	(235.38, 26.11) --
	(235.38, 26.11) --
	(235.45, 26.11) --
	(235.45, 26.11) --
	(235.45, 26.11) --
	(235.52, 26.12) --
	(235.52, 26.12) --
	(235.52, 26.12) --
	(235.59, 26.11) --
	(235.59, 26.11) --
	(235.59, 26.11) --
	(235.66, 26.14) --
	(235.66, 26.14) --
	(235.66, 26.14) --
	(235.73, 26.17) --
	(235.73, 26.17) --
	(235.73, 26.17) --
	(235.80, 26.18) --
	(235.80, 26.18) --
	(235.80, 26.18) --
	(235.87, 26.19) --
	(235.87, 26.19) --
	(235.87, 26.19) --
	(235.94, 26.14) --
	(235.94, 26.14) --
	(235.94, 26.14) --
	(236.01, 26.11) --
	(236.01, 26.11) --
	(236.01, 26.11) --
	(236.08, 26.10) --
	(236.08, 26.10) --
	(236.08, 26.10) --
	(236.15, 26.09) --
	(236.15, 26.09) --
	(236.15, 26.09) --
	(236.22, 26.08) --
	(236.22, 26.08) --
	(236.22, 26.08) --
	(236.25, 26.08) --
	(236.25, 26.08) --
	(236.25, 26.08) --
	(236.29, 26.08) --
	(236.29, 26.08) --
	(236.29, 26.08) --
	(236.36, 26.07) --
	(236.36, 26.07) --
	(236.36, 26.07) --
	(236.43, 26.06) --
	(236.43, 26.06) --
	(236.43, 26.06) --
	(236.50, 26.07) --
	(236.50, 26.07) --
	(236.50, 26.07) --
	(236.57, 26.05) --
	(236.57, 26.05) --
	(236.57, 26.05) --
	(236.64, 26.06) --
	(236.64, 26.06) --
	(236.64, 26.06) --
	(236.71, 26.06) --
	(236.71, 26.06) --
	(236.71, 26.06) --
	(236.78, 26.05) --
	(236.78, 26.05) --
	(236.78, 26.05) --
	(236.85, 26.04) --
	(236.85, 26.04) --
	(236.85, 26.04) --
	(236.92, 26.06) --
	(236.92, 26.06) --
	(236.92, 26.06) --
	(236.99, 26.04) --
	(236.99, 26.04) --
	(236.99, 26.04) --
	(237.06, 26.05) --
	(237.06, 26.05) --
	(237.06, 26.05) --
	(237.13, 26.04) --
	(237.13, 26.04) --
	(237.13, 26.04) --
	(237.20, 26.05) --
	(237.20, 26.05) --
	(237.20, 26.05) --
	(237.27, 26.07) --
	(237.27, 26.07) --
	(237.27, 26.07) --
	(237.34, 26.07) --
	(237.34, 26.07) --
	(237.34, 26.07) --
	(237.41, 26.12) --
	(237.41, 26.12) --
	(237.41, 26.12) --
	(237.48, 26.17) --
	(237.48, 26.17) --
	(237.48, 26.17) --
	(237.55, 26.20) --
	(237.55, 26.20) --
	(237.55, 26.20) --
	(237.62, 26.16) --
	(237.62, 26.16) --
	(237.62, 26.16) --
	(237.69, 26.11) --
	(237.69, 26.11) --
	(237.69, 26.11) --
	(237.76, 26.08) --
	(237.76, 26.08) --
	(237.76, 26.08) --
	(237.83, 26.06) --
	(237.83, 26.06) --
	(237.83, 26.06) --
	(237.90, 26.04) --
	(237.90, 26.04) --
	(237.90, 26.04) --
	(237.97, 26.05) --
	(237.97, 26.05) --
	(237.97, 26.05) --
	(238.04, 26.05) --
	(238.04, 26.05) --
	(238.04, 26.05) --
	(238.11, 26.04) --
	(238.11, 26.04) --
	(238.11, 26.04) --
	(238.18, 26.05) --
	(238.18, 26.05) --
	(238.18, 26.05) --
	(238.25, 26.07) --
	(238.25, 26.07) --
	(238.25, 26.07) --
	(238.32, 26.07) --
	(238.32, 26.07) --
	(238.32, 26.07) --
	(238.39, 26.11) --
	(238.39, 26.11) --
	(238.39, 26.11) --
	(238.46, 26.16) --
	(238.46, 26.16) --
	(238.46, 26.16) --
	(238.53, 26.18) --
	(238.53, 26.18) --
	(238.53, 26.18) --
	(238.60, 26.15) --
	(238.60, 26.15) --
	(238.60, 26.15) --
	(238.67, 26.08) --
	(238.67, 26.08) --
	(238.67, 26.08) --
	(238.74, 26.06) --
	(238.74, 26.06) --
	(238.74, 26.06) --
	(238.74, 26.06) --
	(238.74, 26.06) --
	(238.74, 26.06) --
	(238.81, 26.04) --
	(238.81, 26.04) --
	(238.81, 26.04) --
	(238.88, 26.03) --
	(238.88, 26.03) --
	(238.88, 26.03) --
	(238.95, 26.03) --
	(238.95, 26.03) --
	(238.95, 26.03) --
	(239.02, 26.02) --
	(239.02, 26.02) --
	(239.02, 26.02) --
	(239.09, 26.02) --
	(239.09, 26.02) --
	(239.09, 26.02) --
	(239.16, 26.01) --
	(239.16, 26.01) --
	(239.16, 26.01) --
	(239.23, 26.01) --
	(239.23, 26.01) --
	(239.23, 26.01) --
	(239.30, 26.02) --
	(239.30, 26.02) --
	(239.30, 26.02) --
	(239.37, 26.02) --
	(239.37, 26.02) --
	(239.37, 26.02) --
	(239.44, 26.01) --
	(239.44, 26.01) --
	(239.44, 26.01) --
	(239.51, 26.02) --
	(239.51, 26.02) --
	(239.51, 26.02) --
	(239.58, 26.02) --
	(239.58, 26.02) --
	(239.58, 26.02) --
	(239.65, 26.01) --
	(239.65, 26.01) --
	(239.65, 26.01) --
	(239.72, 26.03) --
	(239.72, 26.03) --
	(239.72, 26.03) --
	(239.79, 26.05) --
	(239.79, 26.05) --
	(239.79, 26.05) --
	(239.86, 26.05) --
	(239.86, 26.05) --
	(239.86, 26.05) --
	(239.93, 26.04) --
	(239.93, 26.04) --
	(239.93, 26.04) --
	(240.00, 26.03) --
	(240.00, 26.03) --
	(240.00, 26.03) --
	(240.07, 26.01) --
	(240.07, 26.01) --
	(240.07, 26.01) --
	(240.14, 26.01) --
	(240.14, 26.01) --
	(240.14, 26.01) --
	(240.21, 26.02) --
	(240.21, 26.02) --
	(240.21, 26.02) --
	(240.28, 26.01) --
	(240.28, 26.01) --
	(240.28, 26.01) --
	(240.35, 26.02) --
	(240.35, 26.02) --
	(240.35, 26.02) --
	(240.42, 26.02) --
	(240.42, 26.02) --
	(240.42, 26.02) --
	(240.49, 26.03) --
	(240.49, 26.03) --
	(240.49, 26.03) --
	(240.56, 26.04) --
	(240.56, 26.04) --
	(240.56, 26.04) --
	(240.56, 26.04) --
	(240.56, 26.04) --
	(240.56, 26.04) --
	(240.63, 26.03) --
	(240.63, 26.03) --
	(240.63, 26.03) --
	(240.70, 26.02) --
	(240.70, 26.02) --
	(240.70, 26.02) --
	(240.77, 26.01) --
	(240.77, 26.01) --
	(240.77, 26.01) --
	(240.84, 26.00) --
	(240.84, 26.00) --
	(240.84, 26.00) --
	(240.91, 26.01) --
	(240.91, 26.01) --
	(240.91, 26.01) --
	(240.98, 25.99) --
	(240.98, 25.99) --
	(240.98, 25.99) --
	(241.05, 26.01) --
	(241.05, 26.01) --
	(241.05, 26.01) --
	(241.12, 26.00) --
	(241.12, 26.00) --
	(241.12, 26.00) --
	(241.19, 25.99) --
	(241.19, 25.99) --
	(241.19, 25.99) --
	(241.26, 26.00) --
	(241.26, 26.00) --
	(241.26, 26.00) --
	(241.33, 26.00) --
	(241.33, 26.00) --
	(241.33, 26.00) --
	(241.40, 25.99) --
	(241.40, 25.99) --
	(241.40, 25.99) --
	(241.47, 25.99) --
	(241.47, 25.99) --
	(241.47, 25.99) --
	(241.54, 25.99) --
	(241.54, 25.99) --
	(241.54, 25.99) --
	(241.61, 26.00) --
	(241.61, 26.00) --
	(241.61, 26.00) --
	(241.68, 25.99) --
	(241.68, 25.99) --
	(241.68, 25.99) --
	(241.75, 25.99) --
	(241.75, 25.99) --
	(241.75, 25.99) --
	(241.82, 25.99) --
	(241.82, 25.99) --
	(241.82, 25.99) --
	(241.89, 26.00) --
	(241.89, 26.00) --
	(241.89, 26.00) --
	(241.96, 25.99) --
	(241.96, 25.99) --
	(241.96, 25.99) --
	(242.03, 26.00) --
	(242.03, 26.00) --
	(242.03, 26.00) --
	(242.09, 26.00) --
	(242.09, 26.00) --
	(242.09, 26.00) --
	(242.16, 25.99) --
	(242.16, 25.99) --
	(242.16, 25.99) --
	(242.23, 25.99) --
	(242.23, 25.99) --
	(242.23, 25.99) --
	(242.30, 25.99) --
	(242.30, 25.99) --
	(242.30, 25.99) --
	(242.37, 25.98) --
	(242.37, 25.98) --
	(242.37, 25.98) --
	(242.44, 25.99) --
	(242.44, 25.99) --
	(242.44, 25.99) --
	(242.51, 25.99) --
	(242.51, 25.99) --
	(242.51, 25.99) --
	(242.58, 25.98) --
	(242.58, 25.98) --
	(242.58, 25.98) --
	(242.65, 25.99) --
	(242.65, 25.99) --
	(242.65, 25.99) --
	(242.72, 25.99) --
	(242.72, 25.99) --
	(242.72, 25.99) --
	(242.79, 26.01) --
	(242.79, 26.01) --
	(242.79, 26.01) --
	(242.86, 26.05) --
	(242.86, 26.05) --
	(242.86, 26.05) --
	(242.93, 26.07) --
	(242.93, 26.07) --
	(242.93, 26.07) --
	(243.00, 26.05) --
	(243.00, 26.05) --
	(243.00, 26.05) --
	(243.07, 26.05) --
	(243.07, 26.05) --
	(243.07, 26.05) --
	(243.14, 26.02) --
	(243.14, 26.02) --
	(243.14, 26.02) --
	(243.21, 26.03) --
	(243.21, 26.03) --
	(243.21, 26.03) --
	(243.28, 26.11) --
	(243.28, 26.11) --
	(243.28, 26.11) --
	(243.35, 26.18) --
	(243.35, 26.18) --
	(243.35, 26.18) --
	(243.42, 26.13) --
	(243.42, 26.13) --
	(243.42, 26.13) --
	(243.49, 26.06) --
	(243.49, 26.06) --
	(243.49, 26.06) --
	(243.56, 26.04) --
	(243.56, 26.04) --
	(243.56, 26.04) --
	(243.63, 26.02) --
	(243.63, 26.02) --
	(243.63, 26.02) --
	(243.70, 25.99) --
	(243.70, 25.99) --
	(243.70, 25.99) --
	(243.77, 25.98) --
	(243.77, 25.98) --
	(243.77, 25.98) --
	(243.84, 25.98) --
	(243.84, 25.98) --
	(243.84, 25.98) --
	(243.90, 25.97) --
	(243.91, 25.97) --
	(243.91, 25.97) --
	(243.98, 25.97) --
	(243.98, 25.97) --
	(243.98, 25.97) --
	(244.05, 25.97) --
	(244.05, 25.97) --
	(244.05, 25.97) --
	(244.11, 25.97) --
	(244.11, 25.97) --
	(244.11, 25.97) --
	(244.18, 25.97) --
	(244.18, 25.97) --
	(244.18, 25.97) --
	(244.25, 25.97) --
	(244.25, 25.97) --
	(244.25, 25.97) --
	(244.32, 25.97) --
	(244.32, 25.97) --
	(244.32, 25.97) --
	(244.39, 25.98) --
	(244.39, 25.98) --
	(244.39, 25.98) --
	(244.46, 25.99) --
	(244.46, 25.99) --
	(244.46, 25.99) --
	(244.53, 25.98) --
	(244.53, 25.98) --
	(244.53, 25.98) --
	(244.60, 25.99) --
	(244.60, 25.99) --
	(244.60, 25.99) --
	(244.67, 25.97) --
	(244.67, 25.97) --
	(244.67, 25.97) --
	(244.74, 25.97) --
	(244.74, 25.97) --
	(244.74, 25.97) --
	(244.81, 25.97) --
	(244.81, 25.97) --
	(244.81, 25.97) --
	(244.88, 25.97) --
	(244.88, 25.97) --
	(244.88, 25.97) --
	(244.95, 25.97) --
	(244.95, 25.97) --
	(244.95, 25.97) --
	(245.02, 25.98) --
	(245.02, 25.98) --
	(245.02, 25.98) --
	(245.09, 25.97) --
	(245.09, 25.97) --
	(245.09, 25.97) --
	(245.16, 25.98) --
	(245.16, 25.98) --
	(245.16, 25.98) --
	(245.23, 25.96) --
	(245.23, 25.96) --
	(245.23, 25.96) --
	(245.30, 25.97) --
	(245.30, 25.97) --
	(245.30, 25.97) --
	(245.37, 25.98) --
	(245.37, 25.98) --
	(245.37, 25.98) --
	(245.44, 25.99) --
	(245.44, 25.99) --
	(245.44, 25.99) --
	(245.51, 26.02) --
	(245.51, 26.02) --
	(245.51, 26.02) --
	(245.57, 26.03) --
	(245.57, 26.03) --
	(245.57, 26.03) --
	(245.64, 26.00) --
	(245.64, 26.00) --
	(245.64, 26.00) --
	(245.71, 25.99) --
	(245.71, 25.99) --
	(245.71, 25.99) --
	(245.78, 25.98) --
	(245.78, 25.98) --
	(245.78, 25.98) --
	(245.85, 25.98) --
	(245.85, 25.98) --
	(245.85, 25.98) --
	(245.92, 25.99) --
	(245.92, 25.99) --
	(245.92, 25.99) --
	(245.99, 26.00) --
	(245.99, 26.00) --
	(245.99, 26.00) --
	(246.06, 26.00) --
	(246.06, 26.00) --
	(246.06, 26.00) --
	(246.13, 26.00) --
	(246.13, 26.00) --
	(246.13, 26.00) --
	(246.20, 25.99) --
	(246.20, 25.99) --
	(246.20, 25.99) --
	(246.27, 25.98) --
	(246.27, 25.98) --
	(246.27, 25.98) --
	(246.34, 25.98) --
	(246.34, 25.98) --
	(246.34, 25.98) --
	(246.41, 25.96) --
	(246.41, 25.96) --
	(246.41, 25.96) --
	(246.48, 25.97) --
	(246.48, 25.97) --
	(246.48, 25.97) --
	(246.55, 25.96) --
	(246.55, 25.96) --
	(246.55, 25.96) --
	(246.61, 25.97) --
	(246.61, 25.97) --
	(246.61, 25.97) --
	(246.68, 25.97) --
	(246.68, 25.97) --
	(246.68, 25.97) --
	(246.75, 25.98) --
	(246.75, 25.98) --
	(246.75, 25.98) --
	(246.82, 25.96) --
	(246.82, 25.96) --
	(246.82, 25.96) --
	(246.89, 25.97) --
	(246.89, 25.97) --
	(246.89, 25.97) --
	(246.96, 25.95) --
	(246.96, 25.95) --
	(246.96, 25.95) --
	(247.03, 25.96) --
	(247.03, 25.96) --
	(247.03, 25.96) --
	(247.10, 25.96) --
	(247.10, 25.96) --
	(247.10, 25.96) --
	(247.17, 25.95) --
	(247.17, 25.95) --
	(247.17, 25.95) --
	(247.24, 25.95) --
	(247.24, 25.95) --
	(247.24, 25.95) --
	(247.31, 25.95) --
	(247.31, 25.95) --
	(247.31, 25.95) --
	(247.38, 25.95) --
	(247.38, 25.95) --
	(247.38, 25.95) --
	(247.45, 25.95) --
	(247.45, 25.95) --
	(247.45, 25.95) --
	(247.52, 25.95) --
	(247.52, 25.95) --
	(247.52, 25.95) --
	(247.59, 25.96) --
	(247.59, 25.96) --
	(247.59, 25.96) --
	(247.65, 25.96) --
	(247.65, 25.96) --
	(247.65, 25.96) --
	(247.72, 25.94) --
	(247.72, 25.94) --
	(247.72, 25.94) --
	(247.79, 25.95) --
	(247.79, 25.95) --
	(247.79, 25.95) --
	(247.86, 25.96) --
	(247.86, 25.96) --
	(247.86, 25.96) --
	(247.93, 25.94) --
	(247.93, 25.94) --
	(247.93, 25.94) --
	(248.00, 25.95) --
	(248.00, 25.95) --
	(248.00, 25.95) --
	(248.07, 25.95) --
	(248.07, 25.95) --
	(248.07, 25.95) --
	(248.14, 25.95) --
	(248.14, 25.95) --
	(248.14, 25.95) --
	(248.21, 25.95) --
	(248.21, 25.95) --
	(248.21, 25.95) --
	(248.28, 25.96) --
	(248.28, 25.96) --
	(248.28, 25.96) --
	(248.35, 25.98) --
	(248.35, 25.98) --
	(248.35, 25.98) --
	(248.42, 25.97) --
	(248.42, 25.97) --
	(248.42, 25.97) --
	(248.49, 25.97) --
	(248.49, 25.97) --
	(248.49, 25.97) --
	(248.56, 25.96) --
	(248.56, 25.96) --
	(248.56, 25.96) --
	(248.63, 25.97) --
	(248.63, 25.97) --
	(248.63, 25.97) --
	(248.69, 25.99) --
	(248.69, 25.99) --
	(248.69, 25.99) --
	(248.76, 25.99) --
	(248.76, 25.99) --
	(248.76, 25.99) --
	(248.83, 25.97) --
	(248.83, 25.97) --
	(248.83, 25.97) --
	(248.90, 25.96) --
	(248.90, 25.96) --
	(248.90, 25.96) --
	(248.97, 25.96) --
	(248.97, 25.96) --
	(248.97, 25.96) --
	(249.04, 25.96) --
	(249.04, 25.96) --
	(249.04, 25.96) --
	(249.11, 25.97) --
	(249.11, 25.97) --
	(249.11, 25.97) --
	(249.18, 26.03) --
	(249.18, 26.03) --
	(249.18, 26.03) --
	(249.25, 26.22) --
	(249.25, 26.22) --
	(249.25, 26.22) --
	(249.32, 26.36) --
	(249.32, 26.36) --
	(249.32, 26.36) --
	(249.39, 26.25) --
	(249.39, 26.25) --
	(249.39, 26.25) --
	(249.46, 26.12) --
	(249.46, 26.12) --
	(249.46, 26.12) --
	(249.52, 26.07) --
	(249.52, 26.07) --
	(249.52, 26.07) --
	(249.59, 26.02) --
	(249.59, 26.02) --
	(249.59, 26.02) --
	(249.66, 25.99) --
	(249.66, 25.99) --
	(249.66, 25.99) --
	(249.73, 25.98) --
	(249.73, 25.98) --
	(249.73, 25.98) --
	(249.80, 25.97) --
	(249.80, 25.97) --
	(249.80, 25.97) --
	(249.87, 25.96) --
	(249.87, 25.96) --
	(249.87, 25.96) --
	(249.94, 25.97) --
	(249.94, 25.97) --
	(249.94, 25.97) --
	(250.01, 25.96) --
	(250.01, 25.96) --
	(250.01, 25.96) --
	(250.08, 25.97) --
	(250.08, 25.97) --
	(250.08, 25.97) --
	(250.15, 26.02) --
	(250.15, 26.02) --
	(250.15, 26.02) --
	(250.22, 26.05) --
	(250.22, 26.05) --
	(250.22, 26.05) --
	(250.28, 26.02) --
	(250.28, 26.02) --
	(250.28, 26.02) --
	(250.35, 25.99) --
	(250.35, 25.99) --
	(250.35, 25.99) --
	(250.42, 25.98) --
	(250.42, 25.98) --
	(250.42, 25.98) --
	(250.49, 25.97) --
	(250.49, 25.97) --
	(250.49, 25.97) --
	(250.56, 25.95) --
	(250.56, 25.95) --
	(250.56, 25.95) --
	(250.63, 25.95) --
	(250.63, 25.95) --
	(250.63, 25.95) --
	(250.70, 25.95) --
	(250.70, 25.95) --
	(250.70, 25.95) --
	(250.77, 25.94) --
	(250.77, 25.94) --
	(250.77, 25.94) --
	(250.84, 25.94) --
	(250.84, 25.94) --
	(250.84, 25.94) --
	(250.91, 25.94) --
	(250.91, 25.94) --
	(250.91, 25.94) --
	(250.97, 25.94) --
	(250.97, 25.94) --
	(250.97, 25.94) --
	(251.04, 25.94) --
	(251.04, 25.94) --
	(251.04, 25.94) --
	(251.11, 25.94) --
	(251.11, 25.94) --
	(251.11, 25.94) --
	(251.18, 25.94) --
	(251.18, 25.94) --
	(251.18, 25.94) --
	(251.25, 25.94) --
	(251.25, 25.94) --
	(251.25, 25.94) --
	(251.32, 25.93) --
	(251.32, 25.93) --
	(251.32, 25.93) --
	(251.39, 25.94) --
	(251.39, 25.94) --
	(251.39, 25.94) --
	(251.46, 25.94) --
	(251.46, 25.94) --
	(251.46, 25.94) --
	(251.53, 25.93) --
	(251.53, 25.93) --
	(251.53, 25.93) --
	(251.60, 25.94) --
	(251.60, 25.94) --
	(251.60, 25.94) --
	(251.66, 25.94) --
	(251.66, 25.94) --
	(251.66, 25.94) --
	(251.73, 25.93) --
	(251.73, 25.93) --
	(251.73, 25.93) --
	(251.80, 25.94) --
	(251.80, 25.94) --
	(251.80, 25.94) --
	(251.87, 25.94) --
	(251.87, 25.94) --
	(251.87, 25.94) --
	(251.94, 25.93) --
	(251.94, 25.93) --
	(251.94, 25.93) --
	(252.01, 25.94) --
	(252.01, 25.94) --
	(252.01, 25.94) --
	(252.08, 25.93) --
	(252.08, 25.93) --
	(252.08, 25.93) --
	(252.15, 25.93) --
	(252.15, 25.93) --
	(252.15, 25.93) --
	(252.22, 25.94) --
	(252.22, 25.94) --
	(252.22, 25.94) --
	(252.29, 25.93) --
	(252.29, 25.93) --
	(252.29, 25.93) --
	(252.35, 25.94) --
	(252.35, 25.94) --
	(252.35, 25.94) --
	(252.42, 25.94) --
	(252.42, 25.94) --
	(252.42, 25.94) --
	(252.49, 25.93) --
	(252.49, 25.93) --
	(252.49, 25.93) --
	(252.56, 25.94) --
	(252.56, 25.94) --
	(252.56, 25.94) --
	(252.63, 25.94) --
	(252.63, 25.94) --
	(252.63, 25.94) --
	(252.70, 25.92) --
	(252.70, 25.92) --
	(252.70, 25.92) --
	(252.77, 25.93) --
	(252.77, 25.93) --
	(252.77, 25.93) --
	(252.84, 25.93) --
	(252.84, 25.93) --
	(252.84, 25.93) --
	(252.91, 25.92) --
	(252.91, 25.92) --
	(252.91, 25.92) --
	(252.97, 25.94) --
	(252.97, 25.94) --
	(252.97, 25.94) --
	(253.04, 25.93) --
	(253.04, 25.93) --
	(253.04, 25.93) --
	(253.11, 25.93) --
	(253.11, 25.93) --
	(253.11, 25.93) --
	(253.18, 25.94) --
	(253.18, 25.94) --
	(253.18, 25.94) --
	(253.25, 25.92) --
	(253.25, 25.92) --
	(253.25, 25.92) --
	(253.32, 25.93) --
	(253.32, 25.93) --
	(253.32, 25.93) --
	(253.39, 25.93) --
	(253.39, 25.93) --
	(253.39, 25.93) --
	(253.46, 25.93) --
	(253.46, 25.93) --
	(253.46, 25.93) --
	(253.53, 25.93) --
	(253.53, 25.93) --
	(253.53, 25.93) --
	(253.59, 25.93) --
	(253.59, 25.93) --
	(253.59, 25.93) --
	(253.66, 25.94) --
	(253.66, 25.94) --
	(253.66, 25.94) --
	(253.73, 25.95) --
	(253.73, 25.95) --
	(253.73, 25.95) --
	(253.80, 25.95) --
	(253.80, 25.95) --
	(253.80, 25.95) --
	(253.87, 25.94) --
	(253.87, 25.94) --
	(253.87, 25.94) --
	(253.94, 25.94) --
	(253.94, 25.94) --
	(253.94, 25.94) --
	(254.01, 25.94) --
	(254.01, 25.94) --
	(254.01, 25.94) --
	(254.08, 25.93) --
	(254.08, 25.93) --
	(254.08, 25.93) --
	(254.14, 25.93) --
	(254.14, 25.93) --
	(254.14, 25.93) --
	(254.21, 25.92) --
	(254.21, 25.92) --
	(254.21, 25.92) --
	(254.28, 25.93) --
	(254.28, 25.93) --
	(254.28, 25.93) --
	(254.35, 25.91) --
	(254.35, 25.91) --
	(254.35, 25.91) --
	(254.42, 25.92) --
	(254.42, 25.92) --
	(254.42, 25.92) --
	(254.49, 25.93) --
	(254.49, 25.93) --
	(254.49, 25.93) --
	(254.56, 25.92) --
	(254.56, 25.92) --
	(254.56, 25.92) --
	(254.63, 25.92) --
	(254.63, 25.92) --
	(254.63, 25.92) --
	(254.70, 25.93) --
	(254.70, 25.93) --
	(254.70, 25.93) --
	(254.76, 25.91) --
	(254.76, 25.91) --
	(254.76, 25.91) --
	(254.83, 25.91) --
	(254.83, 25.91) --
	(254.83, 25.91) --
	(254.90, 25.93) --
	(254.90, 25.93) --
	(254.90, 25.93) --
	(254.97, 25.92) --
	(254.97, 25.92) --
	(254.97, 25.92) --
	(255.04, 25.92) --
	(255.04, 25.92) --
	(255.04, 25.92) --
	(255.11, 25.91) --
	(255.11, 25.91) --
	(255.11, 25.91) --
	(255.18, 25.92) --
	(255.18, 25.92) --
	(255.18, 25.92) --
	(255.24, 25.92) --
	(255.24, 25.92) --
	(255.24, 25.92) --
	(255.31, 25.92) --
	(255.31, 25.92) --
	(255.31, 25.92) --
	(255.38, 25.92) --
	(255.38, 25.92) --
	(255.38, 25.92) --
	(255.45, 25.93) --
	(255.45, 25.93) --
	(255.45, 25.93) --
	(255.52, 25.92) --
	(255.52, 25.92) --
	(255.52, 25.92) --
	(255.59, 25.92) --
	(255.59, 25.92) --
	(255.59, 25.92) --
	(255.66, 25.91) --
	(255.66, 25.91) --
	(255.66, 25.91) --
	(255.73, 25.91) --
	(255.73, 25.91) --
	(255.73, 25.91) --
	(255.79, 25.92) --
	(255.79, 25.92) --
	(255.79, 25.92) --
	(255.86, 25.91) --
	(255.86, 25.91) --
	(255.86, 25.91) --
	(255.93, 25.91) --
	(255.93, 25.91) --
	(255.93, 25.91) --
	(256.00, 25.92) --
	(256.00, 25.92) --
	(256.00, 25.92) --
	(256.07, 25.91) --
	(256.07, 25.91) --
	(256.07, 25.91) --
	(256.14, 25.91) --
	(256.14, 25.91) --
	(256.14, 25.91) --
	(256.20, 25.92) --
	(256.20, 25.92) --
	(256.20, 25.92) --
	(256.27, 25.91) --
	(256.27, 25.91) --
	(256.27, 25.91) --
	(256.34, 25.92) --
	(256.34, 25.92) --
	(256.34, 25.92) --
	(256.41, 25.93) --
	(256.41, 25.93) --
	(256.41, 25.93) --
	(256.48, 25.91) --
	(256.48, 25.91) --
	(256.48, 25.91) --
	(256.55, 25.92) --
	(256.55, 25.92) --
	(256.55, 25.92) --
	(256.62, 25.91) --
	(256.62, 25.91) --
	(256.62, 25.91) --
	(256.69, 25.92) --
	(256.69, 25.92) --
	(256.69, 25.92) --
	(256.75, 25.92) --
	(256.75, 25.92) --
	(256.75, 25.92) --
	(256.82, 25.92) --
	(256.82, 25.92) --
	(256.82, 25.92) --
	(256.89, 25.92) --
	(256.89, 25.92) --
	(256.89, 25.92) --
	(256.96, 25.92) --
	(256.96, 25.92) --
	(256.96, 25.92) --
	(257.03, 25.92) --
	(257.03, 25.92) --
	(257.03, 25.92) --
	(257.10, 25.91) --
	(257.10, 25.91) --
	(257.10, 25.91) --
	(257.17, 25.92) --
	(257.17, 25.92) --
	(257.17, 25.92) --
	(257.24, 25.92) --
	(257.24, 25.92) --
	(257.24, 25.92) --
	(257.30, 25.92) --
	(257.30, 25.92) --
	(257.30, 25.92) --
	(257.37, 25.92) --
	(257.37, 25.92) --
	(257.37, 25.92) --
	(257.44, 25.91) --
	(257.44, 25.91) --
	(257.44, 25.91) --
	(257.51, 25.92) --
	(257.51, 25.92) --
	(257.51, 25.92) --
	(257.58, 25.91) --
	(257.58, 25.91) --
	(257.58, 25.91) --
	(257.65, 25.92) --
	(257.65, 25.92) --
	(257.65, 25.92) --
	(257.71, 25.92) --
	(257.71, 25.92) --
	(257.71, 25.92) --
	(257.78, 25.92) --
	(257.78, 25.92) --
	(257.78, 25.92) --
	(257.85, 25.92) --
	(257.85, 25.92) --
	(257.85, 25.92) --
	(257.92, 25.92) --
	(257.92, 25.92) --
	(257.92, 25.92) --
	(257.99, 25.91) --
	(257.99, 25.91) --
	(257.99, 25.91) --
	(258.06, 25.92) --
	(258.06, 25.92) --
	(258.06, 25.92) --
	(258.12, 25.92) --
	(258.12, 25.92) --
	(258.12, 25.92) --
	(258.19, 25.92) --
	(258.19, 25.92) --
	(258.19, 25.92) --
	(258.26, 25.92) --
	(258.26, 25.92) --
	(258.26, 25.92) --
	(258.33, 25.91) --
	(258.33, 25.91) --
	(258.33, 25.91) --
	(258.40, 25.92) --
	(258.40, 25.92) --
	(258.40, 25.92) --
	(258.47, 25.92) --
	(258.47, 25.92) --
	(258.47, 25.92) --
	(258.54, 25.92) --
	(258.54, 25.92) --
	(258.54, 25.92) --
	(258.60, 25.92) --
	(258.60, 25.92) --
	(258.60, 25.92) --
	(258.67, 25.91) --
	(258.67, 25.91) --
	(258.67, 25.91) --
	(258.74, 25.92) --
	(258.74, 25.92) --
	(258.74, 25.92) --
	(258.81, 25.94) --
	(258.81, 25.94) --
	(258.81, 25.94) --
	(258.88, 25.96) --
	(258.88, 25.96) --
	(258.88, 25.96) --
	(258.95, 25.96) --
	(258.95, 25.96) --
	(258.95, 25.96) --
	(259.01, 25.95) --
	(259.01, 25.95) --
	(259.01, 25.95) --
	(259.08, 25.93) --
	(259.08, 25.93) --
	(259.08, 25.93) --
	(259.15, 25.93) --
	(259.15, 25.93) --
	(259.15, 25.93) --
	(259.22, 25.93) --
	(259.22, 25.93) --
	(259.22, 25.93) --
	(259.29, 25.91) --
	(259.29, 25.91) --
	(259.29, 25.91) --
	(259.36, 25.92) --
	(259.36, 25.92) --
	(259.36, 25.92) --
	(259.42, 25.92) --
	(259.42, 25.92) --
	(259.42, 25.92) --
	(259.49, 25.91) --
	(259.49, 25.91) --
	(259.49, 25.91) --
	(259.56, 25.91) --
	(259.56, 25.91) --
	(259.56, 25.91) --
	(259.63, 25.91) --
	(259.63, 25.91) --
	(259.63, 25.91) --
	(259.70, 25.91) --
	(259.70, 25.91) --
	(259.70, 25.91) --
	(259.77, 25.92) --
	(259.77, 25.92) --
	(259.77, 25.92) --
	(259.83, 25.91) --
	(259.83, 25.91) --
	(259.83, 25.91) --
	(259.90, 25.92) --
	(259.90, 25.92) --
	(259.90, 25.92) --
	(259.97, 25.92) --
	(259.97, 25.92) --
	(259.97, 25.92) --
	(260.04, 25.93) --
	(260.04, 25.93) --
	(260.04, 25.93) --
	(260.11, 25.97) --
	(260.11, 25.97) --
	(260.11, 25.97) --
	(260.18, 26.00) --
	(260.18, 26.00) --
	(260.18, 26.00) --
	(260.25, 25.98) --
	(260.25, 25.98) --
	(260.25, 25.98) --
	(260.31, 25.96) --
	(260.31, 25.96) --
	(260.31, 25.96) --
	(260.38, 25.95) --
	(260.38, 25.95) --
	(260.38, 25.95) --
	(260.45, 25.93) --
	(260.45, 25.93) --
	(260.45, 25.93) --
	(260.52, 25.94) --
	(260.52, 25.94) --
	(260.52, 25.94) --
	(260.59, 25.92) --
	(260.59, 25.92) --
	(260.59, 25.92) --
	(260.66, 25.92) --
	(260.66, 25.92) --
	(260.66, 25.92) --
	(260.72, 25.92) --
	(260.72, 25.92) --
	(260.72, 25.92) --
	(260.79, 25.91) --
	(260.79, 25.91) --
	(260.79, 25.91) --
	(260.86, 25.92) --
	(260.86, 25.92) --
	(260.86, 25.92) --
	(260.93, 25.92) --
	(260.93, 25.92) --
	(260.93, 25.92) --
	(261.00, 25.91) --
	(261.00, 25.91) --
	(261.00, 25.91) --
	(261.06, 25.92) --
	(261.06, 25.92) --
	(261.06, 25.92) --
	(261.13, 25.91) --
	(261.13, 25.91) --
	(261.13, 25.91) --
	(261.20, 25.91) --
	(261.20, 25.91) --
	(261.20, 25.91) --
	(261.27, 25.92) --
	(261.27, 25.92) --
	(261.27, 25.92) --
	(261.34, 25.90) --
	(261.34, 25.90) --
	(261.34, 25.90) --
	(261.40, 25.91) --
	(261.40, 25.91) --
	(261.40, 25.91) --
	(261.47, 25.92) --
	(261.47, 25.92) --
	(261.47, 25.92) --
	(261.54, 25.91) --
	(261.54, 25.91) --
	(261.54, 25.91) --
	(261.61, 25.91) --
	(261.61, 25.91) --
	(261.61, 25.91) --
	(261.68, 25.92) --
	(261.68, 25.92) --
	(261.68, 25.92) --
	(261.75, 25.91) --
	(261.75, 25.91) --
	(261.75, 25.91) --
	(261.81, 25.92) --
	(261.81, 25.92) --
	(261.81, 25.92) --
	(261.88, 25.91) --
	(261.88, 25.91) --
	(261.88, 25.91) --
	(261.95, 25.90) --
	(261.95, 25.90) --
	(261.95, 25.90) --
	(262.02, 25.91) --
	(262.02, 25.91) --
	(262.02, 25.91) --
	(262.09, 25.91) --
	(262.09, 25.91) --
	(262.09, 25.91) --
	(262.16, 25.91) --
	(262.16, 25.91) --
	(262.16, 25.91) --
	(262.22, 25.92) --
	(262.22, 25.92) --
	(262.22, 25.92) --
	(262.29, 25.91) --
	(262.29, 25.91) --
	(262.29, 25.91) --
	(262.36, 25.92) --
	(262.36, 25.92) --
	(262.36, 25.92) --
	(262.43, 25.92) --
	(262.43, 25.92) --
	(262.43, 25.92) --
	(262.50, 25.91) --
	(262.50, 25.91) --
	(262.50, 25.91) --
	(262.56, 25.91) --
	(262.56, 25.91) --
	(262.56, 25.91) --
	(262.63, 25.92) --
	(262.63, 25.92) --
	(262.63, 25.92) --
	(262.70, 25.91) --
	(262.70, 25.91) --
	(262.70, 25.91) --
	(262.77, 25.91) --
	(262.77, 25.91) --
	(262.77, 25.91) --
	(262.84, 25.91) --
	(262.84, 25.91) --
	(262.84, 25.91) --
	(262.90, 25.91) --
	(262.90, 25.91) --
	(262.90, 25.91) --
	(262.97, 25.91) --
	(262.97, 25.91) --
	(262.97, 25.91) --
	(263.04, 25.91) --
	(263.04, 25.91) --
	(263.04, 25.91) --
	(263.11, 25.92) --
	(263.11, 25.92) --
	(263.11, 25.92) --
	(263.18, 25.92) --
	(263.18, 25.92) --
	(263.18, 25.92) --
	(263.25, 25.91) --
	(263.25, 25.91) --
	(263.25, 25.91) --
	(263.31, 25.91) --
	(263.31, 25.91) --
	(263.31, 25.91) --
	(263.38, 25.92) --
	(263.38, 25.92) --
	(263.38, 25.92) --
	(263.45, 25.91) --
	(263.45, 25.91) --
	(263.45, 25.91) --
	(263.52, 25.91) --
	(263.52, 25.91) --
	(263.52, 25.91) --
	(263.58, 25.91) --
	(263.58, 25.91) --
	(263.58, 25.91) --
	(263.65, 25.91) --
	(263.65, 25.91) --
	(263.65, 25.91) --
	(263.72, 25.92) --
	(263.72, 25.92) --
	(263.72, 25.92) --
	(263.79, 25.91) --
	(263.79, 25.91) --
	(263.79, 25.91) --
	(263.86, 25.92) --
	(263.86, 25.92) --
	(263.86, 25.92) --
	(263.93, 25.92) --
	(263.93, 25.92) --
	(263.93, 25.92) --
	(263.99, 25.91) --
	(263.99, 25.91) --
	(263.99, 25.91) --
	(264.06, 25.92) --
	(264.06, 25.92) --
	(264.06, 25.92) --
	(264.13, 25.92) --
	(264.13, 25.92) --
	(264.13, 25.92) --
	(264.20, 25.91) --
	(264.20, 25.91) --
	(264.20, 25.91) --
	(264.26, 25.91) --
	(264.26, 25.91) --
	(264.26, 25.91) --
	(264.33, 25.91) --
	(264.33, 25.91) --
	(264.33, 25.91) --
	(264.40, 25.92) --
	(264.40, 25.92) --
	(264.40, 25.92) --
	(264.47, 25.92) --
	(264.47, 25.92) --
	(264.47, 25.92) --
	(264.54, 25.91) --
	(264.54, 25.91) --
	(264.54, 25.91) --
	(264.61, 25.92) --
	(264.61, 25.92) --
	(264.61, 25.92) --
	(264.67, 25.92) --
	(264.67, 25.92) --
	(264.67, 25.92) --
	(264.74, 25.92) --
	(264.74, 25.92) --
	(264.74, 25.92) --
	(264.81, 25.91) --
	(264.81, 25.91) --
	(264.81, 25.91) --
	(264.88, 25.92) --
	(264.88, 25.92) --
	(264.88, 25.92) --
	(264.94, 25.91) --
	(264.94, 25.91) --
	(264.94, 25.91) --
	(265.01, 25.92) --
	(265.01, 25.92) --
	(265.01, 25.92) --
	(265.08, 25.91) --
	(265.08, 25.91) --
	(265.08, 25.91) --
	(265.15, 25.91) --
	(265.15, 25.91) --
	(265.15, 25.91) --
	(265.22, 25.91) --
	(265.22, 25.91) --
	(265.22, 25.91) --
	(265.28, 25.90) --
	(265.28, 25.90) --
	(265.28, 25.90) --
	(265.35, 25.91) --
	(265.35, 25.91) --
	(265.35, 25.91) --
	(265.42, 25.92) --
	(265.42, 25.92) --
	(265.42, 25.92) --
	(265.49, 25.90) --
	(265.49, 25.90) --
	(265.49, 25.90) --
	(265.56, 25.91) --
	(265.56, 25.91) --
	(265.56, 25.91) --
	(265.62, 25.92) --
	(265.62, 25.92) --
	(265.62, 25.92) --
	(265.69, 25.91) --
	(265.69, 25.91) --
	(265.69, 25.91) --
	(265.76, 25.91) --
	(265.76, 25.91) --
	(265.76, 25.91) --
	(265.83, 25.91) --
	(265.83, 25.91) --
	(265.83, 25.91) --
	(265.90, 25.91) --
	(265.90, 25.91) --
	(265.90, 25.91) --
	(265.96, 25.91) --
	(265.96, 25.91) --
	(265.96, 25.91) --
	(266.03, 25.91) --
	(266.03, 25.91) --
	(266.03, 25.91) --
	(266.10, 25.91) --
	(266.10, 25.91) --
	(266.10, 25.91) --
	(266.17, 25.92) --
	(266.17, 25.92) --
	(266.17, 25.92) --
	(266.23, 25.91) --
	(266.23, 25.91) --
	(266.23, 25.91) --
	(266.30, 25.91) --
	(266.30, 25.91) --
	(266.30, 25.91) --
	(266.37, 25.93) --
	(266.37, 25.93) --
	(266.37, 25.93) --
	(266.44, 25.91) --
	(266.44, 25.91) --
	(266.44, 25.91) --
	(266.51, 25.94) --
	(266.51, 25.94) --
	(266.51, 25.94) --
	(266.57, 25.96) --
	(266.57, 25.96) --
	(266.57, 25.96) --
	(266.64, 25.99) --
	(266.64, 25.99) --
	(266.64, 25.99) --
	(266.71, 26.06) --
	(266.71, 26.06) --
	(266.71, 26.06) --
	(266.78, 26.08) --
	(266.78, 26.08) --
	(266.78, 26.08) --
	(266.84, 26.07) --
	(266.84, 26.07) --
	(266.84, 26.07) --
	(266.91, 26.04) --
	(266.91, 26.04) --
	(266.91, 26.04) --
	(266.98, 25.99) --
	(266.98, 25.99) --
	(266.98, 25.99) --
	(267.05, 25.98) --
	(267.05, 25.98) --
	(267.05, 25.98) --
	(267.11, 25.96) --
	(267.11, 25.96) --
	(267.11, 25.96) --
	(267.18, 25.93) --
	(267.18, 25.93) --
	(267.18, 25.93) --
	(267.25, 25.93) --
	(267.25, 25.93) --
	(267.25, 25.93) --
	(267.32, 25.92) --
	(267.32, 25.92) --
	(267.32, 25.92) --
	(267.39, 25.92) --
	(267.39, 25.92) --
	(267.39, 25.92) --
	(267.45, 25.91) --
	(267.45, 25.91) --
	(267.45, 25.91) --
	(267.52, 25.91) --
	(267.52, 25.91) --
	(267.52, 25.91) --
	(267.59, 25.92) --
	(267.59, 25.92) --
	(267.59, 25.92) --
	(267.66, 25.92) --
	(267.66, 25.92) --
	(267.66, 25.92) --
	(267.72, 25.90) --
	(267.72, 25.90) --
	(267.72, 25.90) --
	(267.79, 25.91) --
	(267.79, 25.91) --
	(267.79, 25.91) --
	(267.86, 25.91) --
	(267.86, 25.91) --
	(267.86, 25.91) --
	(267.93, 25.91) --
	(267.93, 25.91) --
	(267.93, 25.91) --
	(267.99, 25.91) --
	(267.99, 25.91) --
	(267.99, 25.91) --
	(268.06, 25.91) --
	(268.06, 25.91) --
	(268.06, 25.91) --
	(268.13, 25.91) --
	(268.13, 25.91) --
	(268.13, 25.91) --
	(268.20, 25.91) --
	(268.20, 25.91) --
	(268.20, 25.91) --
	(268.27, 25.90) --
	(268.27, 25.90) --
	(268.27, 25.90) --
	(268.33, 25.91) --
	(268.33, 25.91) --
	(268.33, 25.91) --
	(268.40, 25.92) --
	(268.40, 25.92) --
	(268.40, 25.92) --
	(268.47, 25.91) --
	(268.47, 25.91) --
	(268.47, 25.91) --
	(268.54, 25.92) --
	(268.54, 25.92) --
	(268.54, 25.92) --
	(268.60, 25.91) --
	(268.60, 25.91) --
	(268.60, 25.91) --
	(268.67, 25.91) --
	(268.67, 25.91) --
	(268.67, 25.91) --
	(268.74, 25.90) --
	(268.74, 25.90) --
	(268.74, 25.90) --
	(268.81, 25.91) --
	(268.81, 25.91) --
	(268.81, 25.91) --
	(268.87, 25.90) --
	(268.87, 25.90) --
	(268.87, 25.90) --
	(268.94, 25.92) --
	(268.94, 25.92) --
	(268.94, 25.92) --
	(269.01, 25.90) --
	(269.01, 25.90) --
	(269.01, 25.90) --
	(269.08, 25.91) --
	(269.08, 25.91) --
	(269.08, 25.91) --
	(269.14, 25.92) --
	(269.14, 25.92) --
	(269.14, 25.92) --
	(269.21, 25.91) --
	(269.21, 25.91) --
	(269.21, 25.91) --
	(269.28, 25.91) --
	(269.28, 25.91) --
	(269.28, 25.91) --
	(269.35, 25.91) --
	(269.35, 25.91) --
	(269.35, 25.91) --
	(269.42, 25.91) --
	(269.42, 25.91) --
	(269.42, 25.91) --
	(269.48, 25.91) --
	(269.48, 25.91) --
	(269.48, 25.91) --
	(269.55, 25.91) --
	(269.55, 25.91) --
	(269.55, 25.91) --
	(269.62, 25.91) --
	(269.62, 25.91) --
	(269.62, 25.91) --
	(269.69, 25.91) --
	(269.69, 25.91) --
	(269.69, 25.91) --
	(269.75, 25.91) --
	(269.75, 25.91) --
	(269.75, 25.91) --
	(269.82, 25.90) --
	(269.82, 25.90) --
	(269.82, 25.90) --
	(269.89, 25.91) --
	(269.89, 25.91) --
	(269.89, 25.91) --
	(269.95, 25.91) --
	(269.95, 25.91) --
	(269.95, 25.91) --
	(270.02, 25.91) --
	(270.02, 25.91) --
	(270.02, 25.91) --
	(270.09, 25.91) --
	(270.09, 25.91) --
	(270.09, 25.91) --
	(270.16, 25.90) --
	(270.16, 25.90) --
	(270.16, 25.90) --
	(270.22, 25.91) --
	(270.22, 25.91) --
	(270.22, 25.91) --
	(270.29, 25.91) --
	(270.29, 25.91) --
	(270.29, 25.91) --
	(270.36, 25.91) --
	(270.36, 25.91) --
	(270.36, 25.91) --
	(270.43, 25.91) --
	(270.43, 25.91) --
	(270.43, 25.91) --
	(270.49, 25.91) --
	(270.49, 25.91) --
	(270.49, 25.91) --
	(270.56, 25.91) --
	(270.56, 25.91) --
	(270.56, 25.91) --
	(270.63, 25.91) --
	(270.63, 25.91) --
	(270.63, 25.91) --
	(270.70, 25.91) --
	(270.70, 25.91) --
	(270.70, 25.91) --
	(270.76, 25.91) --
	(270.76, 25.91) --
	(270.76, 25.91) --
	(270.83, 25.92) --
	(270.83, 25.92) --
	(270.83, 25.92) --
	(270.90, 25.91) --
	(270.90, 25.91) --
	(270.90, 25.91) --
	(270.97, 25.91) --
	(270.97, 25.91) --
	(270.97, 25.91) --
	(271.03, 25.91) --
	(271.03, 25.91) --
	(271.03, 25.91) --
	(271.10, 25.90) --
	(271.10, 25.90) --
	(271.10, 25.90) --
	(271.17, 25.91) --
	(271.17, 25.91) --
	(271.17, 25.91) --
	(271.24, 25.91) --
	(271.24, 25.91) --
	(271.24, 25.91) --
	(271.30, 25.91) --
	(271.30, 25.91) --
	(271.30, 25.91) --
	(271.37, 25.91) --
	(271.37, 25.91) --
	(271.37, 25.91) --
	(271.44, 25.90) --
	(271.44, 25.90) --
	(271.44, 25.90) --
	(271.51, 25.91) --
	(271.51, 25.91) --
	(271.51, 25.91) --
	(271.57, 25.92) --
	(271.57, 25.92) --
	(271.57, 25.92) --
	(271.64, 25.90) --
	(271.64, 25.90) --
	(271.64, 25.90) --
	(271.71, 25.90) --
	(271.71, 25.90) --
	(271.71, 25.90) --
	(271.77, 25.92) --
	(271.77, 25.92) --
	(271.77, 25.92) --
	(271.84, 25.91) --
	(271.84, 25.91) --
	(271.84, 25.91) --
	(271.91, 25.91) --
	(271.91, 25.91) --
	(271.91, 25.91) --
	(271.98, 25.92) --
	(271.98, 25.92) --
	(271.98, 25.92) --
	(272.04, 25.92) --
	(272.04, 25.92) --
	(272.04, 25.92) --
	(272.11, 25.92) --
	(272.11, 25.92) --
	(272.11, 25.92) --
	(272.18, 25.91) --
	(272.18, 25.91) --
	(272.18, 25.91) --
	(272.25, 25.91) --
	(272.25, 25.91) --
	(272.25, 25.91) --
	(272.31, 25.92) --
	(272.31, 25.92) --
	(272.31, 25.92) --
	(272.38, 25.91) --
	(272.38, 25.91) --
	(272.38, 25.91) --
	(272.45, 25.91) --
	(272.45, 25.91) --
	(272.45, 25.91) --
	(272.52, 25.91) --
	(272.52, 25.91) --
	(272.52, 25.91) --
	(272.58, 25.91) --
	(272.58, 25.91) --
	(272.58, 25.91) --
	(272.65, 25.91) --
	(272.65, 25.91) --
	(272.65, 25.91) --
	(272.72, 25.91) --
	(272.72, 25.91) --
	(272.72, 25.91) --
	(272.79, 25.91) --
	(272.79, 25.91) --
	(272.79, 25.91) --
	(272.85, 25.92) --
	(272.85, 25.92) --
	(272.85, 25.92) --
	(272.92, 25.91) --
	(272.92, 25.91) --
	(272.92, 25.91) --
	(272.99, 25.92) --
	(272.99, 25.92) --
	(272.99, 25.92) --
	(273.05, 25.91) --
	(273.05, 25.91) --
	(273.05, 25.91) --
	(273.12, 25.91) --
	(273.12, 25.91) --
	(273.12, 25.91) --
	(273.19, 25.91) --
	(273.19, 25.91) --
	(273.19, 25.91) --
	(273.26, 25.92) --
	(273.26, 25.92) --
	(273.26, 25.92) --
	(273.32, 25.91) --
	(273.32, 25.91) --
	(273.32, 25.91) --
	(273.39, 25.91) --
	(273.39, 25.91) --
	(273.39, 25.91) --
	(273.46, 25.92) --
	(273.46, 25.92) --
	(273.46, 25.92) --
	(273.53, 25.91) --
	(273.53, 25.91) --
	(273.53, 25.91) --
	(273.59, 25.91) --
	(273.59, 25.91) --
	(273.59, 25.91) --
	(273.66, 25.92) --
	(273.66, 25.92) --
	(273.66, 25.92) --
	(273.73, 25.91) --
	(273.73, 25.91) --
	(273.73, 25.91) --
	(273.79, 25.92) --
	(273.79, 25.92) --
	(273.79, 25.92) --
	(273.86, 25.90) --
	(273.86, 25.90) --
	(273.86, 25.90) --
	(273.93, 25.91) --
	(273.93, 25.91) --
	(273.93, 25.91) --
	(274.00, 25.92) --
	(274.00, 25.92) --
	(274.00, 25.92) --
	(274.06, 25.91) --
	(274.06, 25.91) --
	(274.06, 25.91) --
	(274.13, 25.91) --
	(274.13, 25.91) --
	(274.13, 25.91) --
	(274.20, 25.92) --
	(274.20, 25.92) --
	(274.20, 25.92) --
	(274.26, 25.90) --
	(274.26, 25.90) --
	(274.26, 25.90) --
	(274.33, 25.91) --
	(274.33, 25.91) --
	(274.33, 25.91) --
	(274.40, 25.91) --
	(274.40, 25.91) --
	(274.40, 25.91) --
	(274.46, 25.92) --
	(274.46, 25.92) --
	(274.46, 25.92) --
	(274.53, 25.91) --
	(274.53, 25.91) --
	(274.53, 25.91) --
	(274.60, 25.90) --
	(274.60, 25.90) --
	(274.60, 25.90) --
	(274.67, 25.92) --
	(274.67, 25.92) --
	(274.67, 25.92) --
	(274.73, 25.92) --
	(274.73, 25.92) --
	(274.73, 25.92) --
	(274.80, 25.91) --
	(274.80, 25.91) --
	(274.80, 25.91) --
	(274.87, 25.92) --
	(274.87, 25.92) --
	(274.87, 25.92) --
	(274.93, 25.92) --
	(274.93, 25.92) --
	(274.93, 25.92) --
	(275.00, 25.91) --
	(275.00, 25.91) --
	(275.00, 25.91) --
	(275.07, 25.91) --
	(275.07, 25.91) --
	(275.07, 25.91) --
	(275.14, 25.91) --
	(275.14, 25.91) --
	(275.14, 25.91) --
	(275.20, 25.91) --
	(275.20, 25.91) --
	(275.20, 25.91) --
	(275.27, 25.91) --
	(275.27, 25.91) --
	(275.27, 25.91) --
	(275.34, 25.91) --
	(275.34, 25.91) --
	(275.34, 25.91) --
	(275.40, 25.92) --
	(275.40, 25.92) --
	(275.40, 25.92) --
	(275.47, 25.92) --
	(275.47, 25.92) --
	(275.47, 25.92) --
	(275.54, 25.91) --
	(275.54, 25.91) --
	(275.54, 25.91) --
	(275.60, 25.92) --
	(275.60, 25.92) --
	(275.61, 25.92) --
	(275.67, 25.92) --
	(275.67, 25.92) --
	(275.67, 25.92) --
	(275.74, 25.90) --
	(275.74, 25.90) --
	(275.74, 25.90) --
	(275.81, 25.92) --
	(275.81, 25.92) --
	(275.81, 25.92) --
	(275.87, 25.91) --
	(275.87, 25.91) --
	(275.87, 25.91) --
	(275.94, 25.92) --
	(275.94, 25.92) --
	(275.94, 25.92) --
	(276.01, 25.91) --
	(276.01, 25.91) --
	(276.01, 25.91) --
	(276.07, 25.91) --
	(276.07, 25.91) --
	(276.07, 25.91) --
	(276.14, 25.91) --
	(276.14, 25.91) --
	(276.14, 25.91) --
	(276.21, 25.92) --
	(276.21, 25.92) --
	(276.21, 25.92) --
	(276.28, 25.91) --
	(276.28, 25.91) --
	(276.28, 25.91) --
	(276.34, 25.91) --
	(276.34, 25.91) --
	(276.34, 25.91) --
	(276.41, 25.92) --
	(276.41, 25.92) --
	(276.41, 25.92) --
	(276.48, 25.92) --
	(276.48, 25.92) --
	(276.48, 25.92) --
	(276.54, 25.92) --
	(276.54, 25.92) --
	(276.54, 25.92) --
	(276.61, 25.92) --
	(276.61, 25.92) --
	(276.61, 25.92) --
	(276.68, 25.92) --
	(276.68, 25.92) --
	(276.68, 25.92) --
	(276.74, 25.92) --
	(276.74, 25.92) --
	(276.74, 25.92) --
	(276.81, 25.92) --
	(276.81, 25.92) --
	(276.81, 25.92) --
	(276.88, 25.92) --
	(276.88, 25.92) --
	(276.88, 25.92) --
	(276.94, 25.92) --
	(276.94, 25.92) --
	(276.94, 25.92) --
	(277.01, 25.92) --
	(277.01, 25.92) --
	(277.01, 25.92) --
	(277.08, 25.92) --
	(277.08, 25.92) --
	(277.08, 25.92) --
	(277.15, 25.92) --
	(277.15, 25.92) --
	(277.15, 25.92) --
	(277.21, 25.91) --
	(277.21, 25.91) --
	(277.21, 25.91) --
	(277.28, 25.92) --
	(277.28, 25.92) --
	(277.28, 25.92) --
	(277.35, 25.92) --
	(277.35, 25.92) --
	(277.35, 25.92) --
	(277.41, 25.92) --
	(277.41, 25.92) --
	(277.41, 25.92) --
	(277.48, 25.93) --
	(277.48, 25.93) --
	(277.48, 25.93) --
	(277.55, 25.92) --
	(277.55, 25.92) --
	(277.55, 25.92) --
	(277.61, 25.92) --
	(277.61, 25.92) --
	(277.61, 25.92) --
	(277.68, 25.92) --
	(277.68, 25.92) --
	(277.68, 25.92) --
	(277.75, 25.92) --
	(277.75, 25.92) --
	(277.75, 25.92) --
	(277.81, 25.93) --
	(277.81, 25.93) --
	(277.81, 25.93) --
	(277.88, 25.92) --
	(277.88, 25.92) --
	(277.88, 25.92) --
	(277.95, 25.91) --
	(277.95, 25.91) --
	(277.95, 25.91) --
	(278.01, 25.92) --
	(278.01, 25.92) --
	(278.01, 25.92) --
	(278.08, 25.91) --
	(278.08, 25.91) --
	(278.08, 25.91) --
	(278.15, 25.92) --
	(278.15, 25.92) --
	(278.15, 25.92) --
	(278.21, 25.92) --
	(278.21, 25.92) --
	(278.21, 25.92) --
	(278.28, 25.91) --
	(278.28, 25.91) --
	(278.28, 25.91) --
	(278.35, 25.92) --
	(278.35, 25.92) --
	(278.35, 25.92) --
	(278.41, 25.92) --
	(278.41, 25.92) --
	(278.41, 25.92) --
	(278.48, 25.92) --
	(278.48, 25.92) --
	(278.48, 25.92) --
	(278.55, 25.92) --
	(278.55, 25.92) --
	(278.55, 25.92) --
	(278.61, 25.92) --
	(278.61, 25.92) --
	(278.61, 25.92) --
	(278.68, 25.92) --
	(278.68, 25.92) --
	(278.68, 25.92) --
	(278.75, 25.92) --
	(278.75, 25.92) --
	(278.75, 25.92) --
	(278.82, 25.93) --
	(278.82, 25.93) --
	(278.82, 25.93) --
	(278.88, 25.92) --
	(278.88, 25.92) --
	(278.88, 25.92) --
	(278.95, 25.93) --
	(278.95, 25.93) --
	(278.95, 25.93) --
	(279.02, 25.91) --
	(279.02, 25.91) --
	(279.02, 25.91) --
	(279.08, 25.92) --
	(279.08, 25.92) --
	(279.08, 25.92) --
	(279.15, 25.93) --
	(279.15, 25.93) --
	(279.15, 25.93) --
	(279.22, 25.92) --
	(279.22, 25.92) --
	(279.22, 25.92) --
	(279.28, 25.93) --
	(279.28, 25.93) --
	(279.28, 25.93) --
	(279.35, 25.93) --
	(279.35, 25.93) --
	(279.35, 25.93) --
	(279.42, 25.92) --
	(279.42, 25.92) --
	(279.42, 25.92) --
	(279.48, 25.92) --
	(279.48, 25.92) --
	(279.48, 25.92) --
	(279.55, 25.92) --
	(279.55, 25.92) --
	(279.55, 25.92) --
	(279.62, 25.92) --
	(279.62, 25.92) --
	(279.62, 25.92) --
	(279.68, 25.93) --
	(279.68, 25.93) --
	(279.68, 25.93) --
	(279.75, 25.92) --
	(279.75, 25.92) --
	(279.75, 25.92) --
	(279.82, 25.92) --
	(279.82, 25.92) --
	(279.82, 25.92) --
	(279.88, 25.92) --
	(279.88, 25.92) --
	(279.88, 25.92) --
	(279.95, 25.92) --
	(279.95, 25.92) --
	(279.95, 25.92) --
	(280.02, 25.93) --
	(280.02, 25.93) --
	(280.02, 25.93) --
	(280.08, 25.93) --
	(280.08, 25.93) --
	(280.08, 25.93) --
	(280.15, 25.92) --
	(280.15, 25.92) --
	(280.15, 25.92) --
	(280.22, 25.93) --
	(280.22, 25.93) --
	(280.22, 25.93) --
	(280.28, 25.93) --
	(280.28, 25.93) --
	(280.28, 25.93) --
	(280.35, 25.92) --
	(280.35, 25.92) --
	(280.35, 25.92) --
	(280.42, 25.93) --
	(280.42, 25.93) --
	(280.42, 25.93) --
	(280.48, 25.92) --
	(280.48, 25.92) --
	(280.48, 25.90) --
	(280.48, 25.90) --
	(280.48, 25.90) --
	(280.42, 25.90) --
	(280.42, 25.90) --
	(280.42, 25.90) --
	(280.35, 25.90) --
	(280.35, 25.90) --
	(280.35, 25.90) --
	(280.28, 25.90) --
	(280.28, 25.90) --
	(280.28, 25.90) --
	(280.22, 25.90) --
	(280.22, 25.90) --
	(280.22, 25.90) --
	(280.15, 25.90) --
	(280.15, 25.90) --
	(280.15, 25.90) --
	(280.08, 25.90) --
	(280.08, 25.90) --
	(280.08, 25.90) --
	(280.02, 25.90) --
	(280.02, 25.90) --
	(280.02, 25.90) --
	(279.95, 25.90) --
	(279.95, 25.90) --
	(279.95, 25.90) --
	(279.88, 25.90) --
	(279.88, 25.90) --
	(279.88, 25.90) --
	(279.82, 25.90) --
	(279.82, 25.90) --
	(279.82, 25.90) --
	(279.75, 25.90) --
	(279.75, 25.90) --
	(279.75, 25.90) --
	(279.68, 25.90) --
	(279.68, 25.90) --
	(279.68, 25.90) --
	(279.62, 25.90) --
	(279.62, 25.90) --
	(279.62, 25.90) --
	(279.55, 25.90) --
	(279.55, 25.90) --
	(279.55, 25.90) --
	(279.48, 25.90) --
	(279.48, 25.90) --
	(279.48, 25.90) --
	(279.42, 25.90) --
	(279.42, 25.90) --
	(279.42, 25.90) --
	(279.35, 25.90) --
	(279.35, 25.90) --
	(279.35, 25.90) --
	(279.28, 25.90) --
	(279.28, 25.90) --
	(279.28, 25.90) --
	(279.22, 25.90) --
	(279.22, 25.90) --
	(279.22, 25.90) --
	(279.15, 25.90) --
	(279.15, 25.90) --
	(279.15, 25.90) --
	(279.08, 25.90) --
	(279.08, 25.90) --
	(279.08, 25.90) --
	(279.02, 25.90) --
	(279.02, 25.90) --
	(279.02, 25.90) --
	(278.95, 25.90) --
	(278.95, 25.90) --
	(278.95, 25.90) --
	(278.88, 25.90) --
	(278.88, 25.90) --
	(278.88, 25.90) --
	(278.82, 25.90) --
	(278.82, 25.90) --
	(278.82, 25.90) --
	(278.75, 25.90) --
	(278.75, 25.90) --
	(278.75, 25.90) --
	(278.68, 25.90) --
	(278.68, 25.90) --
	(278.68, 25.90) --
	(278.61, 25.90) --
	(278.61, 25.90) --
	(278.61, 25.90) --
	(278.55, 25.90) --
	(278.55, 25.90) --
	(278.55, 25.90) --
	(278.48, 25.90) --
	(278.48, 25.90) --
	(278.48, 25.90) --
	(278.41, 25.90) --
	(278.41, 25.90) --
	(278.41, 25.90) --
	(278.35, 25.90) --
	(278.35, 25.90) --
	(278.35, 25.90) --
	(278.28, 25.90) --
	(278.28, 25.90) --
	(278.28, 25.90) --
	(278.21, 25.90) --
	(278.21, 25.90) --
	(278.21, 25.90) --
	(278.15, 25.90) --
	(278.15, 25.90) --
	(278.15, 25.90) --
	(278.08, 25.90) --
	(278.08, 25.90) --
	(278.08, 25.90) --
	(278.01, 25.90) --
	(278.01, 25.90) --
	(278.01, 25.90) --
	(277.95, 25.90) --
	(277.95, 25.90) --
	(277.95, 25.90) --
	(277.88, 25.90) --
	(277.88, 25.90) --
	(277.88, 25.90) --
	(277.81, 25.90) --
	(277.81, 25.90) --
	(277.81, 25.90) --
	(277.75, 25.90) --
	(277.75, 25.90) --
	(277.75, 25.90) --
	(277.68, 25.90) --
	(277.68, 25.90) --
	(277.68, 25.90) --
	(277.61, 25.90) --
	(277.61, 25.90) --
	(277.61, 25.90) --
	(277.55, 25.90) --
	(277.55, 25.90) --
	(277.55, 25.90) --
	(277.48, 25.90) --
	(277.48, 25.90) --
	(277.48, 25.90) --
	(277.41, 25.90) --
	(277.41, 25.90) --
	(277.41, 25.90) --
	(277.35, 25.90) --
	(277.35, 25.90) --
	(277.35, 25.90) --
	(277.28, 25.90) --
	(277.28, 25.90) --
	(277.28, 25.90) --
	(277.21, 25.90) --
	(277.21, 25.90) --
	(277.21, 25.90) --
	(277.15, 25.90) --
	(277.15, 25.90) --
	(277.15, 25.90) --
	(277.08, 25.90) --
	(277.08, 25.90) --
	(277.08, 25.90) --
	(277.01, 25.90) --
	(277.01, 25.90) --
	(277.01, 25.90) --
	(276.94, 25.90) --
	(276.94, 25.90) --
	(276.94, 25.90) --
	(276.88, 25.90) --
	(276.88, 25.90) --
	(276.88, 25.90) --
	(276.81, 25.90) --
	(276.81, 25.90) --
	(276.81, 25.90) --
	(276.74, 25.90) --
	(276.74, 25.90) --
	(276.74, 25.90) --
	(276.68, 25.90) --
	(276.68, 25.90) --
	(276.68, 25.90) --
	(276.61, 25.90) --
	(276.61, 25.90) --
	(276.61, 25.90) --
	(276.54, 25.90) --
	(276.54, 25.90) --
	(276.54, 25.90) --
	(276.48, 25.90) --
	(276.48, 25.90) --
	(276.48, 25.90) --
	(276.41, 25.90) --
	(276.41, 25.90) --
	(276.41, 25.90) --
	(276.34, 25.90) --
	(276.34, 25.90) --
	(276.34, 25.90) --
	(276.28, 25.90) --
	(276.28, 25.90) --
	(276.28, 25.90) --
	(276.21, 25.90) --
	(276.21, 25.90) --
	(276.21, 25.90) --
	(276.14, 25.90) --
	(276.14, 25.90) --
	(276.14, 25.90) --
	(276.07, 25.90) --
	(276.07, 25.90) --
	(276.07, 25.90) --
	(276.01, 25.90) --
	(276.01, 25.90) --
	(276.01, 25.90) --
	(275.94, 25.90) --
	(275.94, 25.90) --
	(275.94, 25.90) --
	(275.87, 25.90) --
	(275.87, 25.90) --
	(275.87, 25.90) --
	(275.81, 25.90) --
	(275.81, 25.90) --
	(275.81, 25.90) --
	(275.74, 25.90) --
	(275.74, 25.90) --
	(275.74, 25.90) --
	(275.67, 25.90) --
	(275.67, 25.90) --
	(275.67, 25.90) --
	(275.61, 25.90) --
	(275.60, 25.90) --
	(275.60, 25.90) --
	(275.54, 25.90) --
	(275.54, 25.90) --
	(275.54, 25.90) --
	(275.47, 25.90) --
	(275.47, 25.90) --
	(275.47, 25.90) --
	(275.40, 25.90) --
	(275.40, 25.90) --
	(275.40, 25.90) --
	(275.34, 25.90) --
	(275.34, 25.90) --
	(275.34, 25.90) --
	(275.27, 25.90) --
	(275.27, 25.90) --
	(275.27, 25.90) --
	(275.20, 25.90) --
	(275.20, 25.90) --
	(275.20, 25.90) --
	(275.14, 25.90) --
	(275.14, 25.90) --
	(275.14, 25.90) --
	(275.07, 25.90) --
	(275.07, 25.90) --
	(275.07, 25.90) --
	(275.00, 25.90) --
	(275.00, 25.90) --
	(275.00, 25.90) --
	(274.93, 25.90) --
	(274.93, 25.90) --
	(274.93, 25.90) --
	(274.87, 25.90) --
	(274.87, 25.90) --
	(274.87, 25.90) --
	(274.80, 25.90) --
	(274.80, 25.90) --
	(274.80, 25.90) --
	(274.73, 25.90) --
	(274.73, 25.90) --
	(274.73, 25.90) --
	(274.67, 25.90) --
	(274.67, 25.90) --
	(274.67, 25.90) --
	(274.60, 25.90) --
	(274.60, 25.90) --
	(274.60, 25.90) --
	(274.53, 25.90) --
	(274.53, 25.90) --
	(274.53, 25.90) --
	(274.46, 25.90) --
	(274.46, 25.90) --
	(274.46, 25.90) --
	(274.40, 25.90) --
	(274.40, 25.90) --
	(274.40, 25.90) --
	(274.33, 25.90) --
	(274.33, 25.90) --
	(274.33, 25.90) --
	(274.26, 25.90) --
	(274.26, 25.90) --
	(274.26, 25.90) --
	(274.20, 25.90) --
	(274.20, 25.90) --
	(274.20, 25.90) --
	(274.13, 25.90) --
	(274.13, 25.90) --
	(274.13, 25.90) --
	(274.06, 25.90) --
	(274.06, 25.90) --
	(274.06, 25.90) --
	(274.00, 25.90) --
	(274.00, 25.90) --
	(274.00, 25.90) --
	(273.93, 25.90) --
	(273.93, 25.90) --
	(273.93, 25.90) --
	(273.86, 25.90) --
	(273.86, 25.90) --
	(273.86, 25.90) --
	(273.79, 25.90) --
	(273.79, 25.90) --
	(273.79, 25.90) --
	(273.73, 25.90) --
	(273.73, 25.90) --
	(273.73, 25.90) --
	(273.66, 25.90) --
	(273.66, 25.90) --
	(273.66, 25.90) --
	(273.59, 25.90) --
	(273.59, 25.90) --
	(273.59, 25.90) --
	(273.53, 25.90) --
	(273.53, 25.90) --
	(273.53, 25.90) --
	(273.46, 25.90) --
	(273.46, 25.90) --
	(273.46, 25.90) --
	(273.39, 25.90) --
	(273.39, 25.90) --
	(273.39, 25.90) --
	(273.32, 25.90) --
	(273.32, 25.90) --
	(273.32, 25.90) --
	(273.26, 25.90) --
	(273.26, 25.90) --
	(273.26, 25.90) --
	(273.19, 25.90) --
	(273.19, 25.90) --
	(273.19, 25.90) --
	(273.12, 25.90) --
	(273.12, 25.90) --
	(273.12, 25.90) --
	(273.05, 25.90) --
	(273.05, 25.90) --
	(273.05, 25.90) --
	(272.99, 25.90) --
	(272.99, 25.90) --
	(272.99, 25.90) --
	(272.92, 25.90) --
	(272.92, 25.90) --
	(272.92, 25.90) --
	(272.85, 25.90) --
	(272.85, 25.90) --
	(272.85, 25.90) --
	(272.79, 25.90) --
	(272.79, 25.90) --
	(272.79, 25.90) --
	(272.72, 25.90) --
	(272.72, 25.90) --
	(272.72, 25.90) --
	(272.65, 25.90) --
	(272.65, 25.90) --
	(272.65, 25.90) --
	(272.58, 25.90) --
	(272.58, 25.90) --
	(272.58, 25.90) --
	(272.52, 25.90) --
	(272.52, 25.90) --
	(272.52, 25.90) --
	(272.45, 25.90) --
	(272.45, 25.90) --
	(272.45, 25.90) --
	(272.38, 25.90) --
	(272.38, 25.90) --
	(272.38, 25.90) --
	(272.31, 25.90) --
	(272.31, 25.90) --
	(272.31, 25.90) --
	(272.25, 25.90) --
	(272.25, 25.90) --
	(272.25, 25.90) --
	(272.18, 25.90) --
	(272.18, 25.90) --
	(272.18, 25.90) --
	(272.11, 25.90) --
	(272.11, 25.90) --
	(272.11, 25.90) --
	(272.04, 25.90) --
	(272.04, 25.90) --
	(272.04, 25.90) --
	(271.98, 25.90) --
	(271.98, 25.90) --
	(271.98, 25.90) --
	(271.91, 25.90) --
	(271.91, 25.90) --
	(271.91, 25.90) --
	(271.84, 25.90) --
	(271.84, 25.90) --
	(271.84, 25.90) --
	(271.77, 25.90) --
	(271.77, 25.90) --
	(271.77, 25.90) --
	(271.71, 25.90) --
	(271.71, 25.90) --
	(271.71, 25.90) --
	(271.64, 25.90) --
	(271.64, 25.90) --
	(271.64, 25.90) --
	(271.57, 25.90) --
	(271.57, 25.90) --
	(271.57, 25.90) --
	(271.51, 25.90) --
	(271.51, 25.90) --
	(271.51, 25.90) --
	(271.44, 25.90) --
	(271.44, 25.90) --
	(271.44, 25.90) --
	(271.37, 25.90) --
	(271.37, 25.90) --
	(271.37, 25.90) --
	(271.30, 25.90) --
	(271.30, 25.90) --
	(271.30, 25.90) --
	(271.24, 25.90) --
	(271.24, 25.90) --
	(271.24, 25.90) --
	(271.17, 25.90) --
	(271.17, 25.90) --
	(271.17, 25.90) --
	(271.10, 25.90) --
	(271.10, 25.90) --
	(271.10, 25.90) --
	(271.03, 25.90) --
	(271.03, 25.90) --
	(271.03, 25.90) --
	(270.97, 25.90) --
	(270.97, 25.90) --
	(270.97, 25.90) --
	(270.90, 25.90) --
	(270.90, 25.90) --
	(270.90, 25.90) --
	(270.83, 25.90) --
	(270.83, 25.90) --
	(270.83, 25.90) --
	(270.76, 25.90) --
	(270.76, 25.90) --
	(270.76, 25.90) --
	(270.70, 25.90) --
	(270.70, 25.90) --
	(270.70, 25.90) --
	(270.63, 25.90) --
	(270.63, 25.90) --
	(270.63, 25.90) --
	(270.56, 25.90) --
	(270.56, 25.90) --
	(270.56, 25.90) --
	(270.49, 25.90) --
	(270.49, 25.90) --
	(270.49, 25.90) --
	(270.43, 25.90) --
	(270.43, 25.90) --
	(270.43, 25.90) --
	(270.36, 25.90) --
	(270.36, 25.90) --
	(270.36, 25.90) --
	(270.29, 25.90) --
	(270.29, 25.90) --
	(270.29, 25.90) --
	(270.22, 25.90) --
	(270.22, 25.90) --
	(270.22, 25.90) --
	(270.16, 25.90) --
	(270.16, 25.90) --
	(270.16, 25.90) --
	(270.09, 25.90) --
	(270.09, 25.90) --
	(270.09, 25.90) --
	(270.02, 25.90) --
	(270.02, 25.90) --
	(270.02, 25.90) --
	(269.95, 25.90) --
	(269.95, 25.90) --
	(269.95, 25.90) --
	(269.89, 25.90) --
	(269.89, 25.90) --
	(269.89, 25.90) --
	(269.82, 25.90) --
	(269.82, 25.90) --
	(269.82, 25.90) --
	(269.75, 25.90) --
	(269.75, 25.90) --
	(269.75, 25.90) --
	(269.69, 25.90) --
	(269.69, 25.90) --
	(269.69, 25.90) --
	(269.62, 25.90) --
	(269.62, 25.90) --
	(269.62, 25.90) --
	(269.55, 25.90) --
	(269.55, 25.90) --
	(269.55, 25.90) --
	(269.48, 25.90) --
	(269.48, 25.90) --
	(269.48, 25.90) --
	(269.42, 25.90) --
	(269.42, 25.90) --
	(269.42, 25.90) --
	(269.35, 25.90) --
	(269.35, 25.90) --
	(269.35, 25.90) --
	(269.28, 25.90) --
	(269.28, 25.90) --
	(269.28, 25.90) --
	(269.21, 25.90) --
	(269.21, 25.90) --
	(269.21, 25.90) --
	(269.14, 25.90) --
	(269.14, 25.90) --
	(269.14, 25.90) --
	(269.08, 25.90) --
	(269.08, 25.90) --
	(269.08, 25.90) --
	(269.01, 25.90) --
	(269.01, 25.90) --
	(269.01, 25.90) --
	(268.94, 25.90) --
	(268.94, 25.90) --
	(268.94, 25.90) --
	(268.87, 25.90) --
	(268.87, 25.90) --
	(268.87, 25.90) --
	(268.81, 25.90) --
	(268.81, 25.90) --
	(268.81, 25.90) --
	(268.74, 25.90) --
	(268.74, 25.90) --
	(268.74, 25.90) --
	(268.67, 25.90) --
	(268.67, 25.90) --
	(268.67, 25.90) --
	(268.60, 25.90) --
	(268.60, 25.90) --
	(268.60, 25.90) --
	(268.54, 25.90) --
	(268.54, 25.90) --
	(268.54, 25.90) --
	(268.47, 25.90) --
	(268.47, 25.90) --
	(268.47, 25.90) --
	(268.40, 25.90) --
	(268.40, 25.90) --
	(268.40, 25.90) --
	(268.33, 25.90) --
	(268.33, 25.90) --
	(268.33, 25.90) --
	(268.27, 25.90) --
	(268.27, 25.90) --
	(268.27, 25.90) --
	(268.20, 25.90) --
	(268.20, 25.90) --
	(268.20, 25.90) --
	(268.13, 25.90) --
	(268.13, 25.90) --
	(268.13, 25.90) --
	(268.06, 25.90) --
	(268.06, 25.90) --
	(268.06, 25.90) --
	(267.99, 25.90) --
	(267.99, 25.90) --
	(267.99, 25.90) --
	(267.93, 25.90) --
	(267.93, 25.90) --
	(267.93, 25.90) --
	(267.86, 25.90) --
	(267.86, 25.90) --
	(267.86, 25.90) --
	(267.79, 25.90) --
	(267.79, 25.90) --
	(267.79, 25.90) --
	(267.72, 25.90) --
	(267.72, 25.90) --
	(267.72, 25.90) --
	(267.66, 25.90) --
	(267.66, 25.90) --
	(267.66, 25.90) --
	(267.59, 25.90) --
	(267.59, 25.90) --
	(267.59, 25.90) --
	(267.52, 25.90) --
	(267.52, 25.90) --
	(267.52, 25.90) --
	(267.45, 25.90) --
	(267.45, 25.90) --
	(267.45, 25.90) --
	(267.39, 25.90) --
	(267.39, 25.90) --
	(267.39, 25.90) --
	(267.32, 25.90) --
	(267.32, 25.90) --
	(267.32, 25.90) --
	(267.25, 25.90) --
	(267.25, 25.90) --
	(267.25, 25.90) --
	(267.18, 25.90) --
	(267.18, 25.90) --
	(267.18, 25.90) --
	(267.11, 25.90) --
	(267.11, 25.90) --
	(267.11, 25.90) --
	(267.05, 25.90) --
	(267.05, 25.90) --
	(267.05, 25.90) --
	(266.98, 25.90) --
	(266.98, 25.90) --
	(266.98, 25.90) --
	(266.91, 25.90) --
	(266.91, 25.90) --
	(266.91, 25.90) --
	(266.84, 25.90) --
	(266.84, 25.90) --
	(266.84, 25.90) --
	(266.78, 25.90) --
	(266.78, 25.90) --
	(266.78, 25.90) --
	(266.71, 25.90) --
	(266.71, 25.90) --
	(266.71, 25.90) --
	(266.64, 25.90) --
	(266.64, 25.90) --
	(266.64, 25.90) --
	(266.57, 25.90) --
	(266.57, 25.90) --
	(266.57, 25.90) --
	(266.51, 25.90) --
	(266.51, 25.90) --
	(266.51, 25.90) --
	(266.44, 25.90) --
	(266.44, 25.90) --
	(266.44, 25.90) --
	(266.37, 25.90) --
	(266.37, 25.90) --
	(266.37, 25.90) --
	(266.30, 25.90) --
	(266.30, 25.90) --
	(266.30, 25.90) --
	(266.23, 25.90) --
	(266.23, 25.90) --
	(266.23, 25.90) --
	(266.17, 25.90) --
	(266.17, 25.90) --
	(266.17, 25.90) --
	(266.10, 25.90) --
	(266.10, 25.90) --
	(266.10, 25.90) --
	(266.03, 25.90) --
	(266.03, 25.90) --
	(266.03, 25.90) --
	(265.96, 25.90) --
	(265.96, 25.90) --
	(265.96, 25.90) --
	(265.90, 25.90) --
	(265.90, 25.90) --
	(265.90, 25.90) --
	(265.83, 25.90) --
	(265.83, 25.90) --
	(265.83, 25.90) --
	(265.76, 25.90) --
	(265.76, 25.90) --
	(265.76, 25.90) --
	(265.69, 25.90) --
	(265.69, 25.90) --
	(265.69, 25.90) --
	(265.62, 25.90) --
	(265.62, 25.90) --
	(265.62, 25.90) --
	(265.56, 25.90) --
	(265.56, 25.90) --
	(265.56, 25.90) --
	(265.49, 25.90) --
	(265.49, 25.90) --
	(265.49, 25.90) --
	(265.42, 25.90) --
	(265.42, 25.90) --
	(265.42, 25.90) --
	(265.35, 25.90) --
	(265.35, 25.90) --
	(265.35, 25.90) --
	(265.28, 25.90) --
	(265.28, 25.90) --
	(265.28, 25.90) --
	(265.22, 25.90) --
	(265.22, 25.90) --
	(265.22, 25.90) --
	(265.15, 25.90) --
	(265.15, 25.90) --
	(265.15, 25.90) --
	(265.08, 25.90) --
	(265.08, 25.90) --
	(265.08, 25.90) --
	(265.01, 25.90) --
	(265.01, 25.90) --
	(265.01, 25.90) --
	(264.94, 25.90) --
	(264.94, 25.90) --
	(264.94, 25.90) --
	(264.88, 25.90) --
	(264.88, 25.90) --
	(264.88, 25.90) --
	(264.81, 25.90) --
	(264.81, 25.90) --
	(264.81, 25.90) --
	(264.74, 25.90) --
	(264.74, 25.90) --
	(264.74, 25.90) --
	(264.67, 25.90) --
	(264.67, 25.90) --
	(264.67, 25.90) --
	(264.61, 25.90) --
	(264.61, 25.90) --
	(264.61, 25.90) --
	(264.54, 25.90) --
	(264.54, 25.90) --
	(264.54, 25.90) --
	(264.47, 25.90) --
	(264.47, 25.90) --
	(264.47, 25.90) --
	(264.40, 25.90) --
	(264.40, 25.90) --
	(264.40, 25.90) --
	(264.33, 25.90) --
	(264.33, 25.90) --
	(264.33, 25.90) --
	(264.26, 25.90) --
	(264.26, 25.90) --
	(264.26, 25.90) --
	(264.20, 25.90) --
	(264.20, 25.90) --
	(264.20, 25.90) --
	(264.13, 25.90) --
	(264.13, 25.90) --
	(264.13, 25.90) --
	(264.06, 25.90) --
	(264.06, 25.90) --
	(264.06, 25.90) --
	(263.99, 25.90) --
	(263.99, 25.90) --
	(263.99, 25.90) --
	(263.93, 25.90) --
	(263.93, 25.90) --
	(263.93, 25.90) --
	(263.86, 25.90) --
	(263.86, 25.90) --
	(263.86, 25.90) --
	(263.79, 25.90) --
	(263.79, 25.90) --
	(263.79, 25.90) --
	(263.72, 25.90) --
	(263.72, 25.90) --
	(263.72, 25.90) --
	(263.65, 25.90) --
	(263.65, 25.90) --
	(263.65, 25.90) --
	(263.58, 25.90) --
	(263.58, 25.90) --
	(263.58, 25.90) --
	(263.52, 25.90) --
	(263.52, 25.90) --
	(263.52, 25.90) --
	(263.45, 25.90) --
	(263.45, 25.90) --
	(263.45, 25.90) --
	(263.38, 25.90) --
	(263.38, 25.90) --
	(263.38, 25.90) --
	(263.31, 25.90) --
	(263.31, 25.90) --
	(263.31, 25.90) --
	(263.25, 25.90) --
	(263.25, 25.90) --
	(263.25, 25.90) --
	(263.18, 25.90) --
	(263.18, 25.90) --
	(263.18, 25.90) --
	(263.11, 25.90) --
	(263.11, 25.90) --
	(263.11, 25.90) --
	(263.04, 25.90) --
	(263.04, 25.90) --
	(263.04, 25.90) --
	(262.97, 25.90) --
	(262.97, 25.90) --
	(262.97, 25.90) --
	(262.90, 25.90) --
	(262.90, 25.90) --
	(262.90, 25.90) --
	(262.84, 25.90) --
	(262.84, 25.90) --
	(262.84, 25.90) --
	(262.77, 25.90) --
	(262.77, 25.90) --
	(262.77, 25.90) --
	(262.70, 25.90) --
	(262.70, 25.90) --
	(262.70, 25.90) --
	(262.63, 25.90) --
	(262.63, 25.90) --
	(262.63, 25.90) --
	(262.56, 25.90) --
	(262.56, 25.90) --
	(262.56, 25.90) --
	(262.50, 25.90) --
	(262.50, 25.90) --
	(262.50, 25.90) --
	(262.43, 25.90) --
	(262.43, 25.90) --
	(262.43, 25.90) --
	(262.36, 25.90) --
	(262.36, 25.90) --
	(262.36, 25.90) --
	(262.29, 25.90) --
	(262.29, 25.90) --
	(262.29, 25.90) --
	(262.22, 25.90) --
	(262.22, 25.90) --
	(262.22, 25.90) --
	(262.16, 25.90) --
	(262.16, 25.90) --
	(262.16, 25.90) --
	(262.09, 25.90) --
	(262.09, 25.90) --
	(262.09, 25.90) --
	(262.02, 25.90) --
	(262.02, 25.90) --
	(262.02, 25.90) --
	(261.95, 25.90) --
	(261.95, 25.90) --
	(261.95, 25.90) --
	(261.88, 25.90) --
	(261.88, 25.90) --
	(261.88, 25.90) --
	(261.81, 25.90) --
	(261.81, 25.90) --
	(261.81, 25.90) --
	(261.75, 25.90) --
	(261.75, 25.90) --
	(261.75, 25.90) --
	(261.68, 25.90) --
	(261.68, 25.90) --
	(261.68, 25.90) --
	(261.61, 25.90) --
	(261.61, 25.90) --
	(261.61, 25.90) --
	(261.54, 25.90) --
	(261.54, 25.90) --
	(261.54, 25.90) --
	(261.47, 25.90) --
	(261.47, 25.90) --
	(261.47, 25.90) --
	(261.40, 25.90) --
	(261.40, 25.90) --
	(261.40, 25.90) --
	(261.34, 25.90) --
	(261.34, 25.90) --
	(261.34, 25.90) --
	(261.27, 25.90) --
	(261.27, 25.90) --
	(261.27, 25.90) --
	(261.20, 25.90) --
	(261.20, 25.90) --
	(261.20, 25.90) --
	(261.13, 25.90) --
	(261.13, 25.90) --
	(261.13, 25.90) --
	(261.06, 25.90) --
	(261.06, 25.90) --
	(261.06, 25.90) --
	(261.00, 25.90) --
	(261.00, 25.90) --
	(261.00, 25.90) --
	(260.93, 25.90) --
	(260.93, 25.90) --
	(260.93, 25.90) --
	(260.86, 25.90) --
	(260.86, 25.90) --
	(260.86, 25.90) --
	(260.79, 25.90) --
	(260.79, 25.90) --
	(260.79, 25.90) --
	(260.72, 25.90) --
	(260.72, 25.90) --
	(260.72, 25.90) --
	(260.66, 25.90) --
	(260.66, 25.90) --
	(260.66, 25.90) --
	(260.59, 25.90) --
	(260.59, 25.90) --
	(260.59, 25.90) --
	(260.52, 25.90) --
	(260.52, 25.90) --
	(260.52, 25.90) --
	(260.45, 25.90) --
	(260.45, 25.90) --
	(260.45, 25.90) --
	(260.38, 25.90) --
	(260.38, 25.90) --
	(260.38, 25.90) --
	(260.31, 25.90) --
	(260.31, 25.90) --
	(260.31, 25.90) --
	(260.25, 25.90) --
	(260.25, 25.90) --
	(260.25, 25.90) --
	(260.18, 25.90) --
	(260.18, 25.90) --
	(260.18, 25.90) --
	(260.11, 25.90) --
	(260.11, 25.90) --
	(260.11, 25.90) --
	(260.04, 25.90) --
	(260.04, 25.90) --
	(260.04, 25.90) --
	(259.97, 25.90) --
	(259.97, 25.90) --
	(259.97, 25.90) --
	(259.90, 25.90) --
	(259.90, 25.90) --
	(259.90, 25.90) --
	(259.83, 25.90) --
	(259.83, 25.90) --
	(259.83, 25.90) --
	(259.77, 25.90) --
	(259.77, 25.90) --
	(259.77, 25.90) --
	(259.70, 25.90) --
	(259.70, 25.90) --
	(259.70, 25.90) --
	(259.63, 25.90) --
	(259.63, 25.90) --
	(259.63, 25.90) --
	(259.56, 25.90) --
	(259.56, 25.90) --
	(259.56, 25.90) --
	(259.49, 25.90) --
	(259.49, 25.90) --
	(259.49, 25.90) --
	(259.42, 25.90) --
	(259.42, 25.90) --
	(259.42, 25.90) --
	(259.36, 25.90) --
	(259.36, 25.90) --
	(259.36, 25.90) --
	(259.29, 25.90) --
	(259.29, 25.90) --
	(259.29, 25.90) --
	(259.22, 25.90) --
	(259.22, 25.90) --
	(259.22, 25.90) --
	(259.15, 25.90) --
	(259.15, 25.90) --
	(259.15, 25.90) --
	(259.08, 25.90) --
	(259.08, 25.90) --
	(259.08, 25.90) --
	(259.01, 25.90) --
	(259.01, 25.90) --
	(259.01, 25.90) --
	(258.95, 25.90) --
	(258.95, 25.90) --
	(258.95, 25.90) --
	(258.88, 25.90) --
	(258.88, 25.90) --
	(258.88, 25.90) --
	(258.81, 25.90) --
	(258.81, 25.90) --
	(258.81, 25.90) --
	(258.74, 25.90) --
	(258.74, 25.90) --
	(258.74, 25.90) --
	(258.67, 25.90) --
	(258.67, 25.90) --
	(258.67, 25.90) --
	(258.60, 25.90) --
	(258.60, 25.90) --
	(258.60, 25.90) --
	(258.54, 25.90) --
	(258.54, 25.90) --
	(258.54, 25.90) --
	(258.47, 25.90) --
	(258.47, 25.90) --
	(258.47, 25.90) --
	(258.40, 25.90) --
	(258.40, 25.90) --
	(258.40, 25.90) --
	(258.33, 25.90) --
	(258.33, 25.90) --
	(258.33, 25.90) --
	(258.26, 25.90) --
	(258.26, 25.90) --
	(258.26, 25.90) --
	(258.19, 25.90) --
	(258.19, 25.90) --
	(258.19, 25.90) --
	(258.12, 25.90) --
	(258.12, 25.90) --
	(258.12, 25.90) --
	(258.06, 25.90) --
	(258.06, 25.90) --
	(258.06, 25.90) --
	(257.99, 25.90) --
	(257.99, 25.90) --
	(257.99, 25.90) --
	(257.92, 25.90) --
	(257.92, 25.90) --
	(257.92, 25.90) --
	(257.85, 25.90) --
	(257.85, 25.90) --
	(257.85, 25.90) --
	(257.78, 25.90) --
	(257.78, 25.90) --
	(257.78, 25.90) --
	(257.71, 25.90) --
	(257.71, 25.90) --
	(257.71, 25.90) --
	(257.65, 25.90) --
	(257.65, 25.90) --
	(257.65, 25.90) --
	(257.58, 25.90) --
	(257.58, 25.90) --
	(257.58, 25.90) --
	(257.51, 25.90) --
	(257.51, 25.90) --
	(257.51, 25.90) --
	(257.44, 25.90) --
	(257.44, 25.90) --
	(257.44, 25.90) --
	(257.37, 25.90) --
	(257.37, 25.90) --
	(257.37, 25.90) --
	(257.30, 25.90) --
	(257.30, 25.90) --
	(257.30, 25.90) --
	(257.24, 25.90) --
	(257.24, 25.90) --
	(257.24, 25.90) --
	(257.17, 25.90) --
	(257.17, 25.90) --
	(257.17, 25.90) --
	(257.10, 25.90) --
	(257.10, 25.90) --
	(257.10, 25.90) --
	(257.03, 25.90) --
	(257.03, 25.90) --
	(257.03, 25.90) --
	(256.96, 25.90) --
	(256.96, 25.90) --
	(256.96, 25.90) --
	(256.89, 25.90) --
	(256.89, 25.90) --
	(256.89, 25.90) --
	(256.82, 25.90) --
	(256.82, 25.90) --
	(256.82, 25.90) --
	(256.75, 25.90) --
	(256.75, 25.90) --
	(256.75, 25.90) --
	(256.69, 25.90) --
	(256.69, 25.90) --
	(256.69, 25.90) --
	(256.62, 25.90) --
	(256.62, 25.90) --
	(256.62, 25.90) --
	(256.55, 25.90) --
	(256.55, 25.90) --
	(256.55, 25.90) --
	(256.48, 25.90) --
	(256.48, 25.90) --
	(256.48, 25.90) --
	(256.41, 25.90) --
	(256.41, 25.90) --
	(256.41, 25.90) --
	(256.34, 25.90) --
	(256.34, 25.90) --
	(256.34, 25.90) --
	(256.27, 25.90) --
	(256.27, 25.90) --
	(256.27, 25.90) --
	(256.20, 25.90) --
	(256.20, 25.90) --
	(256.20, 25.90) --
	(256.14, 25.90) --
	(256.14, 25.90) --
	(256.14, 25.90) --
	(256.07, 25.90) --
	(256.07, 25.90) --
	(256.07, 25.90) --
	(256.00, 25.90) --
	(256.00, 25.90) --
	(256.00, 25.90) --
	(255.93, 25.90) --
	(255.93, 25.90) --
	(255.93, 25.90) --
	(255.86, 25.90) --
	(255.86, 25.90) --
	(255.86, 25.90) --
	(255.79, 25.90) --
	(255.79, 25.90) --
	(255.79, 25.90) --
	(255.73, 25.90) --
	(255.73, 25.90) --
	(255.73, 25.90) --
	(255.66, 25.90) --
	(255.66, 25.90) --
	(255.66, 25.90) --
	(255.59, 25.90) --
	(255.59, 25.90) --
	(255.59, 25.90) --
	(255.52, 25.90) --
	(255.52, 25.90) --
	(255.52, 25.90) --
	(255.45, 25.90) --
	(255.45, 25.90) --
	(255.45, 25.90) --
	(255.38, 25.90) --
	(255.38, 25.90) --
	(255.38, 25.90) --
	(255.31, 25.90) --
	(255.31, 25.90) --
	(255.31, 25.90) --
	(255.24, 25.90) --
	(255.24, 25.90) --
	(255.24, 25.90) --
	(255.18, 25.90) --
	(255.18, 25.90) --
	(255.18, 25.90) --
	(255.11, 25.90) --
	(255.11, 25.90) --
	(255.11, 25.90) --
	(255.04, 25.90) --
	(255.04, 25.90) --
	(255.04, 25.90) --
	(254.97, 25.90) --
	(254.97, 25.90) --
	(254.97, 25.90) --
	(254.90, 25.90) --
	(254.90, 25.90) --
	(254.90, 25.90) --
	(254.83, 25.90) --
	(254.83, 25.90) --
	(254.83, 25.90) --
	(254.76, 25.90) --
	(254.76, 25.90) --
	(254.76, 25.90) --
	(254.70, 25.90) --
	(254.70, 25.90) --
	(254.70, 25.90) --
	(254.63, 25.90) --
	(254.63, 25.90) --
	(254.63, 25.90) --
	(254.56, 25.90) --
	(254.56, 25.90) --
	(254.56, 25.90) --
	(254.49, 25.90) --
	(254.49, 25.90) --
	(254.49, 25.90) --
	(254.42, 25.90) --
	(254.42, 25.90) --
	(254.42, 25.90) --
	(254.35, 25.90) --
	(254.35, 25.90) --
	(254.35, 25.90) --
	(254.28, 25.90) --
	(254.28, 25.90) --
	(254.28, 25.90) --
	(254.21, 25.90) --
	(254.21, 25.90) --
	(254.21, 25.90) --
	(254.14, 25.90) --
	(254.14, 25.90) --
	(254.14, 25.90) --
	(254.08, 25.90) --
	(254.08, 25.90) --
	(254.08, 25.90) --
	(254.01, 25.90) --
	(254.01, 25.90) --
	(254.01, 25.90) --
	(253.94, 25.90) --
	(253.94, 25.90) --
	(253.94, 25.90) --
	(253.87, 25.90) --
	(253.87, 25.90) --
	(253.87, 25.90) --
	(253.80, 25.90) --
	(253.80, 25.90) --
	(253.80, 25.90) --
	(253.73, 25.90) --
	(253.73, 25.90) --
	(253.73, 25.90) --
	(253.66, 25.90) --
	(253.66, 25.90) --
	(253.66, 25.90) --
	(253.59, 25.90) --
	(253.59, 25.90) --
	(253.59, 25.90) --
	(253.53, 25.90) --
	(253.53, 25.90) --
	(253.53, 25.90) --
	(253.46, 25.90) --
	(253.46, 25.90) --
	(253.46, 25.90) --
	(253.39, 25.90) --
	(253.39, 25.90) --
	(253.39, 25.90) --
	(253.32, 25.90) --
	(253.32, 25.90) --
	(253.32, 25.90) --
	(253.25, 25.90) --
	(253.25, 25.90) --
	(253.25, 25.90) --
	(253.18, 25.90) --
	(253.18, 25.90) --
	(253.18, 25.90) --
	(253.11, 25.90) --
	(253.11, 25.90) --
	(253.11, 25.90) --
	(253.04, 25.90) --
	(253.04, 25.90) --
	(253.04, 25.90) --
	(252.97, 25.90) --
	(252.97, 25.90) --
	(252.97, 25.90) --
	(252.91, 25.90) --
	(252.91, 25.90) --
	(252.91, 25.90) --
	(252.84, 25.90) --
	(252.84, 25.90) --
	(252.84, 25.90) --
	(252.77, 25.90) --
	(252.77, 25.90) --
	(252.77, 25.90) --
	(252.70, 25.90) --
	(252.70, 25.90) --
	(252.70, 25.90) --
	(252.63, 25.90) --
	(252.63, 25.90) --
	(252.63, 25.90) --
	(252.56, 25.90) --
	(252.56, 25.90) --
	(252.56, 25.90) --
	(252.49, 25.90) --
	(252.49, 25.90) --
	(252.49, 25.90) --
	(252.42, 25.90) --
	(252.42, 25.90) --
	(252.42, 25.90) --
	(252.35, 25.90) --
	(252.35, 25.90) --
	(252.35, 25.90) --
	(252.29, 25.90) --
	(252.29, 25.90) --
	(252.29, 25.90) --
	(252.22, 25.90) --
	(252.22, 25.90) --
	(252.22, 25.90) --
	(252.15, 25.90) --
	(252.15, 25.90) --
	(252.15, 25.90) --
	(252.08, 25.90) --
	(252.08, 25.90) --
	(252.08, 25.90) --
	(252.01, 25.90) --
	(252.01, 25.90) --
	(252.01, 25.90) --
	(251.94, 25.90) --
	(251.94, 25.90) --
	(251.94, 25.90) --
	(251.87, 25.90) --
	(251.87, 25.90) --
	(251.87, 25.90) --
	(251.80, 25.90) --
	(251.80, 25.90) --
	(251.80, 25.90) --
	(251.73, 25.90) --
	(251.73, 25.90) --
	(251.73, 25.90) --
	(251.66, 25.90) --
	(251.66, 25.90) --
	(251.66, 25.90) --
	(251.60, 25.90) --
	(251.60, 25.90) --
	(251.60, 25.90) --
	(251.53, 25.90) --
	(251.53, 25.90) --
	(251.53, 25.90) --
	(251.46, 25.90) --
	(251.46, 25.90) --
	(251.46, 25.90) --
	(251.39, 25.90) --
	(251.39, 25.90) --
	(251.39, 25.90) --
	(251.32, 25.90) --
	(251.32, 25.90) --
	(251.32, 25.90) --
	(251.25, 25.90) --
	(251.25, 25.90) --
	(251.25, 25.90) --
	(251.18, 25.90) --
	(251.18, 25.90) --
	(251.18, 25.90) --
	(251.11, 25.90) --
	(251.11, 25.90) --
	(251.11, 25.90) --
	(251.04, 25.90) --
	(251.04, 25.90) --
	(251.04, 25.90) --
	(250.97, 25.90) --
	(250.97, 25.90) --
	(250.97, 25.90) --
	(250.91, 25.90) --
	(250.91, 25.90) --
	(250.91, 25.90) --
	(250.84, 25.90) --
	(250.84, 25.90) --
	(250.84, 25.90) --
	(250.77, 25.90) --
	(250.77, 25.90) --
	(250.77, 25.90) --
	(250.70, 25.90) --
	(250.70, 25.90) --
	(250.70, 25.90) --
	(250.63, 25.90) --
	(250.63, 25.90) --
	(250.63, 25.90) --
	(250.56, 25.90) --
	(250.56, 25.90) --
	(250.56, 25.90) --
	(250.49, 25.90) --
	(250.49, 25.90) --
	(250.49, 25.90) --
	(250.42, 25.90) --
	(250.42, 25.90) --
	(250.42, 25.90) --
	(250.35, 25.90) --
	(250.35, 25.90) --
	(250.35, 25.90) --
	(250.28, 25.90) --
	(250.28, 25.90) --
	(250.28, 25.90) --
	(250.22, 25.90) --
	(250.22, 25.90) --
	(250.22, 25.90) --
	(250.15, 25.90) --
	(250.15, 25.90) --
	(250.15, 25.90) --
	(250.08, 25.90) --
	(250.08, 25.90) --
	(250.08, 25.90) --
	(250.01, 25.90) --
	(250.01, 25.90) --
	(250.01, 25.90) --
	(249.94, 25.90) --
	(249.94, 25.90) --
	(249.94, 25.90) --
	(249.87, 25.90) --
	(249.87, 25.90) --
	(249.87, 25.90) --
	(249.80, 25.90) --
	(249.80, 25.90) --
	(249.80, 25.90) --
	(249.73, 25.90) --
	(249.73, 25.90) --
	(249.73, 25.90) --
	(249.66, 25.90) --
	(249.66, 25.90) --
	(249.66, 25.90) --
	(249.59, 25.90) --
	(249.59, 25.90) --
	(249.59, 25.90) --
	(249.52, 25.90) --
	(249.52, 25.90) --
	(249.52, 25.90) --
	(249.46, 25.90) --
	(249.46, 25.90) --
	(249.46, 25.90) --
	(249.39, 25.90) --
	(249.39, 25.90) --
	(249.39, 25.90) --
	(249.32, 25.90) --
	(249.32, 25.90) --
	(249.32, 25.90) --
	(249.25, 25.90) --
	(249.25, 25.90) --
	(249.25, 25.90) --
	(249.18, 25.90) --
	(249.18, 25.90) --
	(249.18, 25.90) --
	(249.11, 25.90) --
	(249.11, 25.90) --
	(249.11, 25.90) --
	(249.04, 25.90) --
	(249.04, 25.90) --
	(249.04, 25.90) --
	(248.97, 25.90) --
	(248.97, 25.90) --
	(248.97, 25.90) --
	(248.90, 25.90) --
	(248.90, 25.90) --
	(248.90, 25.90) --
	(248.83, 25.90) --
	(248.83, 25.90) --
	(248.83, 25.90) --
	(248.76, 25.90) --
	(248.76, 25.90) --
	(248.76, 25.90) --
	(248.69, 25.90) --
	(248.69, 25.90) --
	(248.69, 25.90) --
	(248.63, 25.90) --
	(248.63, 25.90) --
	(248.63, 25.90) --
	(248.56, 25.90) --
	(248.56, 25.90) --
	(248.56, 25.90) --
	(248.49, 25.90) --
	(248.49, 25.90) --
	(248.49, 25.90) --
	(248.42, 25.90) --
	(248.42, 25.90) --
	(248.42, 25.90) --
	(248.35, 25.90) --
	(248.35, 25.90) --
	(248.35, 25.90) --
	(248.28, 25.90) --
	(248.28, 25.90) --
	(248.28, 25.90) --
	(248.21, 25.90) --
	(248.21, 25.90) --
	(248.21, 25.90) --
	(248.14, 25.90) --
	(248.14, 25.90) --
	(248.14, 25.90) --
	(248.07, 25.90) --
	(248.07, 25.90) --
	(248.07, 25.90) --
	(248.00, 25.90) --
	(248.00, 25.90) --
	(248.00, 25.90) --
	(247.93, 25.90) --
	(247.93, 25.90) --
	(247.93, 25.90) --
	(247.86, 25.90) --
	(247.86, 25.90) --
	(247.86, 25.90) --
	(247.79, 25.90) --
	(247.79, 25.90) --
	(247.79, 25.90) --
	(247.72, 25.90) --
	(247.72, 25.90) --
	(247.72, 25.90) --
	(247.65, 25.90) --
	(247.65, 25.90) --
	(247.65, 25.90) --
	(247.59, 25.90) --
	(247.59, 25.90) --
	(247.59, 25.90) --
	(247.52, 25.90) --
	(247.52, 25.90) --
	(247.52, 25.90) --
	(247.45, 25.90) --
	(247.45, 25.90) --
	(247.45, 25.90) --
	(247.38, 25.90) --
	(247.38, 25.90) --
	(247.38, 25.90) --
	(247.31, 25.90) --
	(247.31, 25.90) --
	(247.31, 25.90) --
	(247.24, 25.90) --
	(247.24, 25.90) --
	(247.24, 25.90) --
	(247.17, 25.90) --
	(247.17, 25.90) --
	(247.17, 25.90) --
	(247.10, 25.90) --
	(247.10, 25.90) --
	(247.10, 25.90) --
	(247.03, 25.90) --
	(247.03, 25.90) --
	(247.03, 25.90) --
	(246.96, 25.90) --
	(246.96, 25.90) --
	(246.96, 25.90) --
	(246.89, 25.90) --
	(246.89, 25.90) --
	(246.89, 25.90) --
	(246.82, 25.90) --
	(246.82, 25.90) --
	(246.82, 25.90) --
	(246.75, 25.90) --
	(246.75, 25.90) --
	(246.75, 25.90) --
	(246.68, 25.90) --
	(246.68, 25.90) --
	(246.68, 25.90) --
	(246.61, 25.90) --
	(246.61, 25.90) --
	(246.61, 25.90) --
	(246.55, 25.90) --
	(246.55, 25.90) --
	(246.55, 25.90) --
	(246.48, 25.90) --
	(246.48, 25.90) --
	(246.48, 25.90) --
	(246.41, 25.90) --
	(246.41, 25.90) --
	(246.41, 25.90) --
	(246.34, 25.90) --
	(246.34, 25.90) --
	(246.34, 25.90) --
	(246.27, 25.90) --
	(246.27, 25.90) --
	(246.27, 25.90) --
	(246.20, 25.90) --
	(246.20, 25.90) --
	(246.20, 25.90) --
	(246.13, 25.90) --
	(246.13, 25.90) --
	(246.13, 25.90) --
	(246.06, 25.90) --
	(246.06, 25.90) --
	(246.06, 25.90) --
	(245.99, 25.90) --
	(245.99, 25.90) --
	(245.99, 25.90) --
	(245.92, 25.90) --
	(245.92, 25.90) --
	(245.92, 25.90) --
	(245.85, 25.90) --
	(245.85, 25.90) --
	(245.85, 25.90) --
	(245.78, 25.90) --
	(245.78, 25.90) --
	(245.78, 25.90) --
	(245.71, 25.90) --
	(245.71, 25.90) --
	(245.71, 25.90) --
	(245.64, 25.90) --
	(245.64, 25.90) --
	(245.64, 25.90) --
	(245.57, 25.90) --
	(245.57, 25.90) --
	(245.57, 25.90) --
	(245.51, 25.90) --
	(245.51, 25.90) --
	(245.51, 25.90) --
	(245.44, 25.90) --
	(245.44, 25.90) --
	(245.44, 25.90) --
	(245.37, 25.90) --
	(245.37, 25.90) --
	(245.37, 25.90) --
	(245.30, 25.90) --
	(245.30, 25.90) --
	(245.30, 25.90) --
	(245.23, 25.90) --
	(245.23, 25.90) --
	(245.23, 25.90) --
	(245.16, 25.90) --
	(245.16, 25.90) --
	(245.16, 25.90) --
	(245.09, 25.90) --
	(245.09, 25.90) --
	(245.09, 25.90) --
	(245.02, 25.90) --
	(245.02, 25.90) --
	(245.02, 25.90) --
	(244.95, 25.90) --
	(244.95, 25.90) --
	(244.95, 25.90) --
	(244.88, 25.90) --
	(244.88, 25.90) --
	(244.88, 25.90) --
	(244.81, 25.90) --
	(244.81, 25.90) --
	(244.81, 25.90) --
	(244.74, 25.90) --
	(244.74, 25.90) --
	(244.74, 25.90) --
	(244.67, 25.90) --
	(244.67, 25.90) --
	(244.67, 25.90) --
	(244.60, 25.90) --
	(244.60, 25.90) --
	(244.60, 25.90) --
	(244.53, 25.90) --
	(244.53, 25.90) --
	(244.53, 25.90) --
	(244.46, 25.90) --
	(244.46, 25.90) --
	(244.46, 25.90) --
	(244.39, 25.90) --
	(244.39, 25.90) --
	(244.39, 25.90) --
	(244.32, 25.90) --
	(244.32, 25.90) --
	(244.32, 25.90) --
	(244.25, 25.90) --
	(244.25, 25.90) --
	(244.25, 25.90) --
	(244.18, 25.90) --
	(244.18, 25.90) --
	(244.18, 25.90) --
	(244.11, 25.90) --
	(244.11, 25.90) --
	(244.11, 25.90) --
	(244.05, 25.90) --
	(244.05, 25.90) --
	(244.05, 25.90) --
	(243.98, 25.90) --
	(243.98, 25.90) --
	(243.98, 25.90) --
	(243.91, 25.90) --
	(243.91, 25.90) --
	(243.90, 25.90) --
	(243.84, 25.90) --
	(243.84, 25.90) --
	(243.84, 25.90) --
	(243.77, 25.90) --
	(243.77, 25.90) --
	(243.77, 25.90) --
	(243.70, 25.90) --
	(243.70, 25.90) --
	(243.70, 25.90) --
	(243.63, 25.90) --
	(243.63, 25.90) --
	(243.63, 25.90) --
	(243.56, 25.90) --
	(243.56, 25.90) --
	(243.56, 25.90) --
	(243.49, 25.90) --
	(243.49, 25.90) --
	(243.49, 25.90) --
	(243.42, 25.90) --
	(243.42, 25.90) --
	(243.42, 25.90) --
	(243.35, 25.90) --
	(243.35, 25.90) --
	(243.35, 25.90) --
	(243.28, 25.90) --
	(243.28, 25.90) --
	(243.28, 25.90) --
	(243.21, 25.90) --
	(243.21, 25.90) --
	(243.21, 25.90) --
	(243.14, 25.90) --
	(243.14, 25.90) --
	(243.14, 25.90) --
	(243.07, 25.90) --
	(243.07, 25.90) --
	(243.07, 25.90) --
	(243.00, 25.90) --
	(243.00, 25.90) --
	(243.00, 25.90) --
	(242.93, 25.90) --
	(242.93, 25.90) --
	(242.93, 25.90) --
	(242.86, 25.90) --
	(242.86, 25.90) --
	(242.86, 25.90) --
	(242.79, 25.90) --
	(242.79, 25.90) --
	(242.79, 25.90) --
	(242.72, 25.90) --
	(242.72, 25.90) --
	(242.72, 25.90) --
	(242.65, 25.90) --
	(242.65, 25.90) --
	(242.65, 25.90) --
	(242.58, 25.90) --
	(242.58, 25.90) --
	(242.58, 25.90) --
	(242.51, 25.90) --
	(242.51, 25.90) --
	(242.51, 25.90) --
	(242.44, 25.90) --
	(242.44, 25.90) --
	(242.44, 25.90) --
	(242.37, 25.90) --
	(242.37, 25.90) --
	(242.37, 25.90) --
	(242.30, 25.90) --
	(242.30, 25.90) --
	(242.30, 25.90) --
	(242.23, 25.90) --
	(242.23, 25.90) --
	(242.23, 25.90) --
	(242.16, 25.90) --
	(242.16, 25.90) --
	(242.16, 25.90) --
	(242.09, 25.90) --
	(242.09, 25.90) --
	(242.09, 25.90) --
	(242.03, 25.90) --
	(242.03, 25.90) --
	(242.03, 25.90) --
	(241.96, 25.90) --
	(241.96, 25.90) --
	(241.96, 25.90) --
	(241.89, 25.90) --
	(241.89, 25.90) --
	(241.89, 25.90) --
	(241.82, 25.90) --
	(241.82, 25.90) --
	(241.82, 25.90) --
	(241.75, 25.90) --
	(241.75, 25.90) --
	(241.75, 25.90) --
	(241.68, 25.90) --
	(241.68, 25.90) --
	(241.68, 25.90) --
	(241.61, 25.90) --
	(241.61, 25.90) --
	(241.61, 25.90) --
	(241.54, 25.90) --
	(241.54, 25.90) --
	(241.54, 25.90) --
	(241.47, 25.90) --
	(241.47, 25.90) --
	(241.47, 25.90) --
	(241.40, 25.90) --
	(241.40, 25.90) --
	(241.40, 25.90) --
	(241.33, 25.90) --
	(241.33, 25.90) --
	(241.33, 25.90) --
	(241.26, 25.90) --
	(241.26, 25.90) --
	(241.26, 25.90) --
	(241.19, 25.90) --
	(241.19, 25.90) --
	(241.19, 25.90) --
	(241.12, 25.90) --
	(241.12, 25.90) --
	(241.12, 25.90) --
	(241.05, 25.90) --
	(241.05, 25.90) --
	(241.05, 25.90) --
	(240.98, 25.90) --
	(240.98, 25.90) --
	(240.98, 25.90) --
	(240.91, 25.90) --
	(240.91, 25.90) --
	(240.91, 25.90) --
	(240.84, 25.90) --
	(240.84, 25.90) --
	(240.84, 25.90) --
	(240.77, 25.90) --
	(240.77, 25.90) --
	(240.77, 25.90) --
	(240.70, 25.90) --
	(240.70, 25.90) --
	(240.70, 25.90) --
	(240.63, 25.90) --
	(240.63, 25.90) --
	(240.63, 25.90) --
	(240.56, 25.90) --
	(240.56, 25.90) --
	(240.56, 25.90) --
	(240.56, 25.90) --
	(240.56, 25.90) --
	(240.56, 25.90) --
	(240.49, 25.90) --
	(240.49, 25.90) --
	(240.49, 25.90) --
	(240.42, 25.90) --
	(240.42, 25.90) --
	(240.42, 25.90) --
	(240.35, 25.90) --
	(240.35, 25.90) --
	(240.35, 25.90) --
	(240.28, 25.90) --
	(240.28, 25.90) --
	(240.28, 25.90) --
	(240.21, 25.90) --
	(240.21, 25.90) --
	(240.21, 25.90) --
	(240.14, 25.90) --
	(240.14, 25.90) --
	(240.14, 25.90) --
	(240.07, 25.90) --
	(240.07, 25.90) --
	(240.07, 25.90) --
	(240.00, 25.90) --
	(240.00, 25.90) --
	(240.00, 25.90) --
	(239.93, 25.90) --
	(239.93, 25.90) --
	(239.93, 25.90) --
	(239.86, 25.90) --
	(239.86, 25.90) --
	(239.86, 25.90) --
	(239.79, 25.90) --
	(239.79, 25.90) --
	(239.79, 25.90) --
	(239.72, 25.90) --
	(239.72, 25.90) --
	(239.72, 25.90) --
	(239.65, 25.90) --
	(239.65, 25.90) --
	(239.65, 25.90) --
	(239.58, 25.90) --
	(239.58, 25.90) --
	(239.58, 25.90) --
	(239.51, 25.90) --
	(239.51, 25.90) --
	(239.51, 25.90) --
	(239.44, 25.90) --
	(239.44, 25.90) --
	(239.44, 25.90) --
	(239.37, 25.90) --
	(239.37, 25.90) --
	(239.37, 25.90) --
	(239.30, 25.90) --
	(239.30, 25.90) --
	(239.30, 25.90) --
	(239.23, 25.90) --
	(239.23, 25.90) --
	(239.23, 25.90) --
	(239.16, 25.90) --
	(239.16, 25.90) --
	(239.16, 25.90) --
	(239.09, 25.90) --
	(239.09, 25.90) --
	(239.09, 25.90) --
	(239.02, 25.90) --
	(239.02, 25.90) --
	(239.02, 25.90) --
	(238.95, 25.90) --
	(238.95, 25.90) --
	(238.95, 25.90) --
	(238.88, 25.90) --
	(238.88, 25.90) --
	(238.88, 25.90) --
	(238.81, 25.90) --
	(238.81, 25.90) --
	(238.81, 25.90) --
	(238.74, 25.90) --
	(238.74, 25.90) --
	(238.74, 25.90) --
	(238.74, 25.90) --
	(238.74, 25.90) --
	(238.74, 25.90) --
	(238.67, 25.90) --
	(238.67, 25.90) --
	(238.67, 25.90) --
	(238.60, 25.90) --
	(238.60, 25.90) --
	(238.60, 25.90) --
	(238.53, 25.90) --
	(238.53, 25.90) --
	(238.53, 25.90) --
	(238.46, 25.90) --
	(238.46, 25.90) --
	(238.46, 25.90) --
	(238.39, 25.90) --
	(238.39, 25.90) --
	(238.39, 25.90) --
	(238.32, 25.90) --
	(238.32, 25.90) --
	(238.32, 25.90) --
	(238.25, 25.90) --
	(238.25, 25.90) --
	(238.25, 25.90) --
	(238.18, 25.90) --
	(238.18, 25.90) --
	(238.18, 25.90) --
	(238.11, 25.90) --
	(238.11, 25.90) --
	(238.11, 25.90) --
	(238.04, 25.90) --
	(238.04, 25.90) --
	(238.04, 25.90) --
	(237.97, 25.90) --
	(237.97, 25.90) --
	(237.97, 25.90) --
	(237.90, 25.90) --
	(237.90, 25.90) --
	(237.90, 25.90) --
	(237.83, 25.90) --
	(237.83, 25.90) --
	(237.83, 25.90) --
	(237.76, 25.90) --
	(237.76, 25.90) --
	(237.76, 25.90) --
	(237.69, 25.90) --
	(237.69, 25.90) --
	(237.69, 25.90) --
	(237.62, 25.90) --
	(237.62, 25.90) --
	(237.62, 25.90) --
	(237.55, 25.90) --
	(237.55, 25.90) --
	(237.55, 25.90) --
	(237.48, 25.90) --
	(237.48, 25.90) --
	(237.48, 25.90) --
	(237.41, 25.90) --
	(237.41, 25.90) --
	(237.41, 25.90) --
	(237.34, 25.90) --
	(237.34, 25.90) --
	(237.34, 25.90) --
	(237.27, 25.90) --
	(237.27, 25.90) --
	(237.27, 25.90) --
	(237.20, 25.90) --
	(237.20, 25.90) --
	(237.20, 25.90) --
	(237.13, 25.90) --
	(237.13, 25.90) --
	(237.13, 25.90) --
	(237.06, 25.90) --
	(237.06, 25.90) --
	(237.06, 25.90) --
	(236.99, 25.90) --
	(236.99, 25.90) --
	(236.99, 25.90) --
	(236.92, 25.90) --
	(236.92, 25.90) --
	(236.92, 25.90) --
	(236.85, 25.90) --
	(236.85, 25.90) --
	(236.85, 25.90) --
	(236.78, 25.90) --
	(236.78, 25.90) --
	(236.78, 25.90) --
	(236.71, 25.90) --
	(236.71, 25.90) --
	(236.71, 25.90) --
	(236.64, 25.90) --
	(236.64, 25.90) --
	(236.64, 25.90) --
	(236.57, 25.90) --
	(236.57, 25.90) --
	(236.57, 25.90) --
	(236.50, 25.90) --
	(236.50, 25.90) --
	(236.50, 25.90) --
	(236.43, 25.90) --
	(236.43, 25.90) --
	(236.43, 25.90) --
	(236.36, 25.90) --
	(236.36, 25.90) --
	(236.36, 25.90) --
	(236.29, 25.90) --
	(236.29, 25.90) --
	(236.29, 25.90) --
	(236.25, 25.90) --
	(236.25, 25.90) --
	(236.25, 25.90) --
	(236.22, 25.90) --
	(236.22, 25.90) --
	(236.22, 25.90) --
	(236.15, 25.90) --
	(236.15, 25.90) --
	(236.15, 25.90) --
	(236.08, 25.90) --
	(236.08, 25.90) --
	(236.08, 25.90) --
	(236.01, 25.90) --
	(236.01, 25.90) --
	(236.01, 25.90) --
	(235.94, 25.90) --
	(235.94, 25.90) --
	(235.94, 25.90) --
	(235.87, 25.90) --
	(235.87, 25.90) --
	(235.87, 25.90) --
	(235.80, 25.90) --
	(235.80, 25.90) --
	(235.80, 25.90) --
	(235.73, 25.90) --
	(235.73, 25.90) --
	(235.73, 25.90) --
	(235.66, 25.90) --
	(235.66, 25.90) --
	(235.66, 25.90) --
	(235.59, 25.90) --
	(235.59, 25.90) --
	(235.59, 25.90) --
	(235.52, 25.90) --
	(235.52, 25.90) --
	(235.52, 25.90) --
	(235.45, 25.90) --
	(235.45, 25.90) --
	(235.45, 25.90) --
	(235.38, 25.90) --
	(235.38, 25.90) --
	(235.38, 25.90) --
	(235.31, 25.90) --
	(235.31, 25.90) --
	(235.31, 25.90) --
	(235.24, 25.90) --
	(235.24, 25.90) --
	(235.24, 25.90) --
	(235.17, 25.90) --
	(235.17, 25.90) --
	(235.17, 25.90) --
	(235.10, 25.90) --
	(235.10, 25.90) --
	(235.10, 25.90) --
	(235.03, 25.90) --
	(235.03, 25.90) --
	(235.03, 25.90) --
	(234.96, 25.90) --
	(234.96, 25.90) --
	(234.96, 25.90) --
	(234.89, 25.90) --
	(234.89, 25.90) --
	(234.89, 25.90) --
	(234.82, 25.90) --
	(234.82, 25.90) --
	(234.82, 25.90) --
	(234.75, 25.90) --
	(234.75, 25.90) --
	(234.75, 25.90) --
	(234.68, 25.90) --
	(234.68, 25.90) --
	(234.68, 25.90) --
	(234.61, 25.90) --
	(234.61, 25.90) --
	(234.61, 25.90) --
	(234.54, 25.90) --
	(234.54, 25.90) --
	(234.54, 25.90) --
	(234.47, 25.90) --
	(234.47, 25.90) --
	(234.47, 25.90) --
	(234.40, 25.90) --
	(234.40, 25.90) --
	(234.40, 25.90) --
	(234.32, 25.90) --
	(234.32, 25.90) --
	(234.32, 25.90) --
	(234.26, 25.90) --
	(234.26, 25.90) --
	(234.25, 25.90) --
	(234.18, 25.90) --
	(234.18, 25.90) --
	(234.18, 25.90) --
	(234.11, 25.90) --
	(234.11, 25.90) --
	(234.11, 25.90) --
	(234.04, 25.90) --
	(234.04, 25.90) --
	(234.04, 25.90) --
	(233.97, 25.90) --
	(233.97, 25.90) --
	(233.97, 25.90) --
	(233.90, 25.90) --
	(233.90, 25.90) --
	(233.90, 25.90) --
	(233.83, 25.90) --
	(233.83, 25.90) --
	(233.83, 25.90) --
	(233.76, 25.90) --
	(233.76, 25.90) --
	(233.76, 25.90) --
	(233.69, 25.90) --
	(233.69, 25.90) --
	(233.69, 25.90) --
	(233.62, 25.90) --
	(233.62, 25.90) --
	(233.62, 25.90) --
	(233.55, 25.90) --
	(233.55, 25.90) --
	(233.55, 25.90) --
	(233.48, 25.90) --
	(233.48, 25.90) --
	(233.48, 25.90) --
	(233.41, 25.90) --
	(233.41, 25.90) --
	(233.41, 25.90) --
	(233.34, 25.90) --
	(233.34, 25.90) --
	(233.34, 25.90) --
	(233.28, 25.90) --
	(233.28, 25.90) --
	(233.28, 25.90) --
	(233.27, 25.90) --
	(233.27, 25.90) --
	(233.27, 25.90) --
	(233.20, 25.90) --
	(233.20, 25.90) --
	(233.20, 25.90) --
	(233.13, 25.90) --
	(233.13, 25.90) --
	(233.13, 25.90) --
	(233.06, 25.90) --
	(233.06, 25.90) --
	(233.06, 25.90) --
	(232.99, 25.90) --
	(232.99, 25.90) --
	(232.99, 25.90) --
	(232.92, 25.90) --
	(232.92, 25.90) --
	(232.92, 25.90) --
	(232.85, 25.90) --
	(232.85, 25.90) --
	(232.85, 25.90) --
	(232.78, 25.90) --
	(232.78, 25.90) --
	(232.78, 25.90) --
	(232.71, 25.90) --
	(232.71, 25.90) --
	(232.71, 25.90) --
	(232.64, 25.90) --
	(232.64, 25.90) --
	(232.64, 25.90) --
	(232.57, 25.90) --
	(232.57, 25.90) --
	(232.57, 25.90) --
	(232.50, 25.90) --
	(232.50, 25.90) --
	(232.50, 25.90) --
	(232.43, 25.90) --
	(232.43, 25.90) --
	(232.43, 25.90) --
	(232.36, 25.90) --
	(232.36, 25.90) --
	(232.36, 25.90) --
	(232.28, 25.90) --
	(232.28, 25.90) --
	(232.28, 25.90) --
	(232.21, 25.90) --
	(232.21, 25.90) --
	(232.21, 25.90) --
	(232.14, 25.90) --
	(232.14, 25.90) --
	(232.14, 25.90) --
	(232.07, 25.90) --
	(232.07, 25.90) --
	(232.07, 25.90) --
	(232.00, 25.90) --
	(232.00, 25.90) --
	(232.00, 25.90) --
	(231.93, 25.90) --
	(231.93, 25.90) --
	(231.93, 25.90) --
	(231.86, 25.90) --
	(231.86, 25.90) --
	(231.86, 25.90) --
	(231.79, 25.90) --
	(231.79, 25.90) --
	(231.79, 25.90) --
	(231.72, 25.90) --
	(231.72, 25.90) --
	(231.72, 25.90) --
	(231.65, 25.90) --
	(231.65, 25.90) --
	(231.65, 25.90) --
	(231.58, 25.90) --
	(231.58, 25.90) --
	(231.58, 25.90) --
	(231.51, 25.90) --
	(231.51, 25.90) --
	(231.51, 25.90) --
	(231.44, 25.90) --
	(231.44, 25.90) --
	(231.44, 25.90) --
	(231.37, 25.90) --
	(231.37, 25.90) --
	(231.37, 25.90) --
	(231.30, 25.90) --
	(231.30, 25.90) --
	(231.30, 25.90) --
	(231.23, 25.90) --
	(231.23, 25.90) --
	(231.23, 25.90) --
	(231.16, 25.90) --
	(231.16, 25.90) --
	(231.16, 25.90) --
	(231.09, 25.90) --
	(231.09, 25.90) --
	(231.09, 25.90) --
	(231.02, 25.90) --
	(231.02, 25.90) --
	(231.02, 25.90) --
	(230.95, 25.90) --
	(230.95, 25.90) --
	(230.95, 25.90) --
	(230.88, 25.90) --
	(230.88, 25.90) --
	(230.88, 25.90) --
	(230.80, 25.90) --
	(230.80, 25.90) --
	(230.80, 25.90) --
	(230.73, 25.90) --
	(230.73, 25.90) --
	(230.73, 25.90) --
	(230.66, 25.90) --
	(230.66, 25.90) --
	(230.66, 25.90) --
	(230.59, 25.90) --
	(230.59, 25.90) --
	(230.59, 25.90) --
	(230.52, 25.90) --
	(230.52, 25.90) --
	(230.52, 25.90) --
	(230.50, 25.90) --
	(230.50, 25.90) --
	(230.50, 25.90) --
	(230.45, 25.90) --
	(230.45, 25.90) --
	(230.45, 25.90) --
	(230.38, 25.90) --
	(230.38, 25.90) --
	(230.38, 25.90) --
	(230.31, 25.90) --
	(230.31, 25.90) --
	(230.31, 25.90) --
	(230.24, 25.90) --
	(230.24, 25.90) --
	(230.24, 25.90) --
	(230.17, 25.90) --
	(230.17, 25.90) --
	(230.17, 25.90) --
	(230.10, 25.90) --
	(230.10, 25.90) --
	(230.10, 25.90) --
	(230.03, 25.90) --
	(230.03, 25.90) --
	(230.03, 25.90) --
	(229.96, 25.90) --
	(229.96, 25.90) --
	(229.96, 25.90) --
	(229.89, 25.90) --
	(229.89, 25.90) --
	(229.89, 25.90) --
	(229.82, 25.90) --
	(229.82, 25.90) --
	(229.82, 25.90) --
	(229.75, 25.90) --
	(229.75, 25.90) --
	(229.75, 25.90) --
	(229.68, 25.90) --
	(229.68, 25.90) --
	(229.68, 25.90) --
	(229.60, 25.90) --
	(229.60, 25.90) --
	(229.60, 25.90) --
	(229.53, 25.90) --
	(229.53, 25.90) --
	(229.53, 25.90) --
	(229.46, 25.90) --
	(229.46, 25.90) --
	(229.46, 25.90) --
	(229.39, 25.90) --
	(229.39, 25.90) --
	(229.39, 25.90) --
	(229.32, 25.90) --
	(229.32, 25.90) --
	(229.32, 25.90) --
	(229.25, 25.90) --
	(229.25, 25.90) --
	(229.25, 25.90) --
	(229.18, 25.90) --
	(229.18, 25.90) --
	(229.18, 25.90) --
	(229.11, 25.90) --
	(229.11, 25.90) --
	(229.11, 25.90) --
	(229.04, 25.90) --
	(229.04, 25.90) --
	(229.04, 25.90) --
	(228.97, 25.90) --
	(228.97, 25.90) --
	(228.97, 25.90) --
	(228.90, 25.90) --
	(228.90, 25.90) --
	(228.90, 25.90) --
	(228.83, 25.90) --
	(228.83, 25.90) --
	(228.83, 25.90) --
	(228.76, 25.90) --
	(228.76, 25.90) --
	(228.76, 25.90) --
	(228.69, 25.90) --
	(228.69, 25.90) --
	(228.69, 25.90) --
	(228.62, 25.90) --
	(228.62, 25.90) --
	(228.62, 25.90) --
	(228.55, 25.90) --
	(228.55, 25.90) --
	(228.55, 25.90) --
	(228.47, 25.90) --
	(228.47, 25.90) --
	(228.47, 25.90) --
	(228.40, 25.90) --
	(228.40, 25.90) --
	(228.40, 25.90) --
	(228.33, 25.90) --
	(228.33, 25.90) --
	(228.33, 25.90) --
	(228.26, 25.90) --
	(228.26, 25.90) --
	(228.26, 25.90) --
	(228.19, 25.90) --
	(228.19, 25.90) --
	(228.19, 25.90) --
	(228.12, 25.90) --
	(228.12, 25.90) --
	(228.12, 25.90) --
	(228.05, 25.90) --
	(228.05, 25.90) --
	(228.05, 25.90) --
	(227.98, 25.90) --
	(227.98, 25.90) --
	(227.98, 25.90) --
	(227.91, 25.90) --
	(227.91, 25.90) --
	(227.91, 25.90) --
	(227.91, 25.90) --
	(227.91, 25.90) --
	(227.91, 25.90) --
	(227.84, 25.90) --
	(227.84, 25.90) --
	(227.84, 25.90) --
	(227.77, 25.90) --
	(227.77, 25.90) --
	(227.77, 25.90) --
	(227.70, 25.90) --
	(227.70, 25.90) --
	(227.70, 25.90) --
	(227.62, 25.90) --
	(227.62, 25.90) --
	(227.62, 25.90) --
	(227.56, 25.90) --
	(227.56, 25.90) --
	(227.56, 25.90) --
	(227.48, 25.90) --
	(227.48, 25.90) --
	(227.48, 25.90) --
	(227.41, 25.90) --
	(227.41, 25.90) --
	(227.41, 25.90) --
	(227.34, 25.90) --
	(227.34, 25.90) --
	(227.34, 25.90) --
	(227.27, 25.90) --
	(227.27, 25.90) --
	(227.27, 25.90) --
	(227.20, 25.90) --
	(227.20, 25.90) --
	(227.20, 25.90) --
	(227.13, 25.90) --
	(227.13, 25.90) --
	(227.13, 25.90) --
	(227.06, 25.90) --
	(227.06, 25.90) --
	(227.06, 25.90) --
	(226.99, 25.90) --
	(226.99, 25.90) --
	(226.99, 25.90) --
	(226.92, 25.90) --
	(226.92, 25.90) --
	(226.92, 25.90) --
	(226.85, 25.90) --
	(226.85, 25.90) --
	(226.85, 25.90) --
	(226.78, 25.90) --
	(226.78, 25.90) --
	(226.78, 25.90) --
	(226.71, 25.90) --
	(226.71, 25.90) --
	(226.71, 25.90) --
	(226.63, 25.90) --
	(226.63, 25.90) --
	(226.63, 25.90) --
	(226.56, 25.90) --
	(226.56, 25.90) --
	(226.56, 25.90) --
	(226.49, 25.90) --
	(226.49, 25.90) --
	(226.49, 25.90) --
	(226.42, 25.90) --
	(226.42, 25.90) --
	(226.42, 25.90) --
	(226.35, 25.90) --
	(226.35, 25.90) --
	(226.35, 25.90) --
	(226.28, 25.90) --
	(226.28, 25.90) --
	(226.28, 25.90) --
	(226.21, 25.90) --
	(226.21, 25.90) --
	(226.21, 25.90) --
	(226.14, 25.90) --
	(226.14, 25.90) --
	(226.14, 25.90) --
	(226.07, 25.90) --
	(226.07, 25.90) --
	(226.07, 25.90) --
	(226.00, 25.90) --
	(226.00, 25.90) --
	(226.00, 25.90) --
	(225.93, 25.90) --
	(225.93, 25.90) --
	(225.93, 25.90) --
	(225.85, 25.90) --
	(225.85, 25.90) --
	(225.85, 25.90) --
	(225.78, 25.90) --
	(225.78, 25.90) --
	(225.78, 25.90) --
	(225.71, 25.90) --
	(225.71, 25.90) --
	(225.71, 25.90) --
	(225.64, 25.90) --
	(225.64, 25.90) --
	(225.64, 25.90) --
	(225.61, 25.90) --
	(225.61, 25.90) --
	(225.61, 25.90) --
	(225.57, 25.90) --
	(225.57, 25.90) --
	(225.57, 25.90) --
	(225.50, 25.90) --
	(225.50, 25.90) --
	(225.50, 25.90) --
	(225.43, 25.90) --
	(225.43, 25.90) --
	(225.43, 25.90) --
	(225.36, 25.90) --
	(225.36, 25.90) --
	(225.36, 25.90) --
	(225.29, 25.90) --
	(225.29, 25.90) --
	(225.29, 25.90) --
	(225.22, 25.90) --
	(225.22, 25.90) --
	(225.22, 25.90) --
	(225.15, 25.90) --
	(225.15, 25.90) --
	(225.15, 25.90) --
	(225.07, 25.90) --
	(225.07, 25.90) --
	(225.07, 25.90) --
	(225.00, 25.90) --
	(225.00, 25.90) --
	(225.00, 25.90) --
	(224.93, 25.90) --
	(224.93, 25.90) --
	(224.93, 25.90) --
	(224.86, 25.90) --
	(224.86, 25.90) --
	(224.86, 25.90) --
	(224.79, 25.90) --
	(224.79, 25.90) --
	(224.79, 25.90) --
	(224.72, 25.90) --
	(224.72, 25.90) --
	(224.72, 25.90) --
	(224.65, 25.90) --
	(224.65, 25.90) --
	(224.65, 25.90) --
	(224.58, 25.90) --
	(224.58, 25.90) --
	(224.58, 25.90) --
	(224.51, 25.90) --
	(224.51, 25.90) --
	(224.51, 25.90) --
	(224.44, 25.90) --
	(224.44, 25.90) --
	(224.44, 25.90) --
	(224.36, 25.90) --
	(224.36, 25.90) --
	(224.36, 25.90) --
	(224.29, 25.90) --
	(224.29, 25.90) --
	(224.29, 25.90) --
	(224.22, 25.90) --
	(224.22, 25.90) --
	(224.22, 25.90) --
	(224.15, 25.90) --
	(224.15, 25.90) --
	(224.15, 25.90) --
	(224.08, 25.90) --
	(224.08, 25.90) --
	(224.08, 25.90) --
	(224.01, 25.90) --
	(224.01, 25.90) --
	(224.01, 25.90) --
	(223.94, 25.90) --
	(223.94, 25.90) --
	(223.94, 25.90) --
	(223.87, 25.90) --
	(223.87, 25.90) --
	(223.87, 25.90) --
	(223.80, 25.90) --
	(223.80, 25.90) --
	(223.80, 25.90) --
	(223.79, 25.90) --
	(223.79, 25.90) --
	(223.79, 25.90) --
	(223.73, 25.90) --
	(223.73, 25.90) --
	(223.73, 25.90) --
	(223.66, 25.90) --
	(223.66, 25.90) --
	(223.66, 25.90) --
	(223.58, 25.90) --
	(223.58, 25.90) --
	(223.58, 25.90) --
	(223.51, 25.90) --
	(223.51, 25.90) --
	(223.51, 25.90) --
	(223.44, 25.90) --
	(223.44, 25.90) --
	(223.44, 25.90) --
	(223.37, 25.90) --
	(223.37, 25.90) --
	(223.37, 25.90) --
	(223.30, 25.90) --
	(223.30, 25.90) --
	(223.30, 25.90) --
	(223.23, 25.90) --
	(223.23, 25.90) --
	(223.23, 25.90) --
	(223.16, 25.90) --
	(223.16, 25.90) --
	(223.16, 25.90) --
	(223.09, 25.90) --
	(223.09, 25.90) --
	(223.09, 25.90) --
	(223.02, 25.90) --
	(223.02, 25.90) --
	(223.02, 25.90) --
	(222.94, 25.90) --
	(222.94, 25.90) --
	(222.94, 25.90) --
	(222.87, 25.90) --
	(222.87, 25.90) --
	(222.87, 25.90) --
	(222.80, 25.90) --
	(222.80, 25.90) --
	(222.80, 25.90) --
	(222.73, 25.90) --
	(222.73, 25.90) --
	(222.73, 25.90) --
	(222.66, 25.90) --
	(222.66, 25.90) --
	(222.66, 25.90) --
	(222.59, 25.90) --
	(222.59, 25.90) --
	(222.59, 25.90) --
	(222.52, 25.90) --
	(222.52, 25.90) --
	(222.52, 25.90) --
	(222.45, 25.90) --
	(222.45, 25.90) --
	(222.45, 25.90) --
	(222.37, 25.90) --
	(222.37, 25.90) --
	(222.37, 25.90) --
	(222.30, 25.90) --
	(222.30, 25.90) --
	(222.30, 25.90) --
	(222.23, 25.90) --
	(222.23, 25.90) --
	(222.23, 25.90) --
	(222.16, 25.90) --
	(222.16, 25.90) --
	(222.16, 25.90) --
	(222.09, 25.90) --
	(222.09, 25.90) --
	(222.09, 25.90) --
	(222.02, 25.90) --
	(222.02, 25.90) --
	(222.02, 25.90) --
	(221.95, 25.90) --
	(221.95, 25.90) --
	(221.95, 25.90) --
	(221.88, 25.90) --
	(221.88, 25.90) --
	(221.88, 25.90) --
	(221.81, 25.90) --
	(221.81, 25.90) --
	(221.81, 25.90) --
	(221.74, 25.90) --
	(221.74, 25.90) --
	(221.74, 25.90) --
	(221.66, 25.90) --
	(221.66, 25.90) --
	(221.66, 25.90) --
	(221.59, 25.90) --
	(221.59, 25.90) --
	(221.59, 25.90) --
	(221.52, 25.90) --
	(221.52, 25.90) --
	(221.52, 25.90) --
	(221.49, 25.90) --
	(221.49, 25.90) --
	(221.49, 25.90) --
	(221.45, 25.90) --
	(221.45, 25.90) --
	(221.45, 25.90) --
	(221.38, 25.90) --
	(221.38, 25.90) --
	(221.38, 25.90) --
	(221.31, 25.90) --
	(221.31, 25.90) --
	(221.31, 25.90) --
	(221.24, 25.90) --
	(221.24, 25.90) --
	(221.24, 25.90) --
	(221.17, 25.90) --
	(221.17, 25.90) --
	(221.17, 25.90) --
	(221.09, 25.90) --
	(221.09, 25.90) --
	(221.09, 25.90) --
	(221.02, 25.90) --
	(221.02, 25.90) --
	(221.02, 25.90) --
	(220.95, 25.90) --
	(220.95, 25.90) --
	(220.95, 25.90) --
	(220.88, 25.90) --
	(220.88, 25.90) --
	(220.88, 25.90) --
	(220.81, 25.90) --
	(220.81, 25.90) --
	(220.81, 25.90) --
	(220.74, 25.90) --
	(220.74, 25.90) --
	(220.74, 25.90) --
	(220.67, 25.90) --
	(220.67, 25.90) --
	(220.67, 25.90) --
	(220.60, 25.90) --
	(220.60, 25.90) --
	(220.60, 25.90) --
	(220.52, 25.90) --
	(220.52, 25.90) --
	(220.52, 25.90) --
	(220.45, 25.90) --
	(220.45, 25.90) --
	(220.45, 25.90) --
	(220.38, 25.90) --
	(220.38, 25.90) --
	(220.38, 25.90) --
	(220.31, 25.90) --
	(220.31, 25.90) --
	(220.31, 25.90) --
	(220.24, 25.90) --
	(220.24, 25.90) --
	(220.24, 25.90) --
	(220.17, 25.90) --
	(220.17, 25.90) --
	(220.17, 25.90) --
	(220.10, 25.90) --
	(220.10, 25.90) --
	(220.10, 25.90) --
	(220.03, 25.90) --
	(220.03, 25.90) --
	(220.03, 25.90) --
	(219.95, 25.90) --
	(219.95, 25.90) --
	(219.95, 25.90) --
	(219.88, 25.90) --
	(219.88, 25.90) --
	(219.88, 25.90) --
	(219.81, 25.90) --
	(219.81, 25.90) --
	(219.81, 25.90) --
	(219.74, 25.90) --
	(219.74, 25.90) --
	(219.74, 25.90) --
	(219.67, 25.90) --
	(219.67, 25.90) --
	(219.67, 25.90) --
	(219.60, 25.90) --
	(219.60, 25.90) --
	(219.60, 25.90) --
	(219.53, 25.90) --
	(219.53, 25.90) --
	(219.53, 25.90) --
	(219.48, 25.90) --
	(219.48, 25.90) --
	(219.48, 25.90) --
	(219.46, 25.90) --
	(219.46, 25.90) --
	(219.46, 25.90) --
	(219.38, 25.90) --
	(219.38, 25.90) --
	(219.38, 25.90) --
	(219.31, 25.90) --
	(219.31, 25.90) --
	(219.31, 25.90) --
	(219.24, 25.90) --
	(219.24, 25.90) --
	(219.24, 25.90) --
	(219.17, 25.90) --
	(219.17, 25.90) --
	(219.17, 25.90) --
	(219.10, 25.90) --
	(219.10, 25.90) --
	(219.10, 25.90) --
	(219.03, 25.90) --
	(219.03, 25.90) --
	(219.03, 25.90) --
	(218.95, 25.90) --
	(218.95, 25.90) --
	(218.95, 25.90) --
	(218.88, 25.90) --
	(218.88, 25.90) --
	(218.88, 25.90) --
	(218.81, 25.90) --
	(218.81, 25.90) --
	(218.81, 25.90) --
	(218.74, 25.90) --
	(218.74, 25.90) --
	(218.74, 25.90) --
	(218.67, 25.90) --
	(218.67, 25.90) --
	(218.67, 25.90) --
	(218.60, 25.90) --
	(218.60, 25.90) --
	(218.60, 25.90) --
	(218.53, 25.90) --
	(218.53, 25.90) --
	(218.53, 25.90) --
	(218.46, 25.90) --
	(218.46, 25.90) --
	(218.46, 25.90) --
	(218.38, 25.90) --
	(218.38, 25.90) --
	(218.38, 25.90) --
	(218.31, 25.90) --
	(218.31, 25.90) --
	(218.31, 25.90) --
	(218.24, 25.90) --
	(218.24, 25.90) --
	(218.24, 25.90) --
	(218.17, 25.90) --
	(218.17, 25.90) --
	(218.17, 25.90) --
	(218.10, 25.90) --
	(218.10, 25.90) --
	(218.10, 25.90) --
	(218.03, 25.90) --
	(218.03, 25.90) --
	(218.03, 25.90) --
	(217.96, 25.90) --
	(217.96, 25.90) --
	(217.96, 25.90) --
	(217.88, 25.90) --
	(217.88, 25.90) --
	(217.88, 25.90) --
	(217.81, 25.90) --
	(217.81, 25.90) --
	(217.81, 25.90) --
	(217.76, 25.90) --
	(217.76, 25.90) --
	(217.76, 25.90) --
	(217.74, 25.90) --
	(217.74, 25.90) --
	(217.74, 25.90) --
	(217.67, 25.90) --
	(217.67, 25.90) --
	(217.67, 25.90) --
	(217.60, 25.90) --
	(217.60, 25.90) --
	(217.60, 25.90) --
	(217.53, 25.90) --
	(217.53, 25.90) --
	(217.53, 25.90) --
	(217.46, 25.90) --
	(217.46, 25.90) --
	(217.46, 25.90) --
	(217.38, 25.90) --
	(217.38, 25.90) --
	(217.38, 25.90) --
	(217.31, 25.90) --
	(217.31, 25.90) --
	(217.31, 25.90) --
	(217.24, 25.90) --
	(217.24, 25.90) --
	(217.24, 25.90) --
	(217.17, 25.90) --
	(217.17, 25.90) --
	(217.17, 25.90) --
	(217.10, 25.90) --
	(217.10, 25.90) --
	(217.10, 25.90) --
	(217.03, 25.90) --
	(217.03, 25.90) --
	(217.03, 25.90) --
	(216.96, 25.90) --
	(216.96, 25.90) --
	(216.96, 25.90) --
	(216.88, 25.90) --
	(216.88, 25.90) --
	(216.88, 25.90) --
	(216.81, 25.90) --
	(216.81, 25.90) --
	(216.81, 25.90) --
	(216.74, 25.90) --
	(216.74, 25.90) --
	(216.74, 25.90) --
	(216.67, 25.90) --
	(216.67, 25.90) --
	(216.67, 25.90) --
	(216.60, 25.90) --
	(216.60, 25.90) --
	(216.60, 25.90) --
	(216.53, 25.90) --
	(216.53, 25.90) --
	(216.53, 25.90) --
	(216.46, 25.90) --
	(216.46, 25.90) --
	(216.46, 25.90) --
	(216.38, 25.90) --
	(216.38, 25.90) --
	(216.38, 25.90) --
	(216.31, 25.90) --
	(216.31, 25.90) --
	(216.31, 25.90) --
	(216.24, 25.90) --
	(216.24, 25.90) --
	(216.24, 25.90) --
	(216.17, 25.90) --
	(216.17, 25.90) --
	(216.17, 25.90) --
	(216.10, 25.90) --
	(216.10, 25.90) --
	(216.10, 25.90) --
	(216.03, 25.90) --
	(216.03, 25.90) --
	(216.03, 25.90) --
	(215.95, 25.90) --
	(215.95, 25.90) --
	(215.95, 25.90) --
	(215.88, 25.90) --
	(215.88, 25.90) --
	(215.88, 25.90) --
	(215.81, 25.90) --
	(215.81, 25.90) --
	(215.81, 25.90) --
	(215.74, 25.90) --
	(215.74, 25.90) --
	(215.74, 25.90) --
	(215.74, 25.90) --
	(215.74, 25.90) --
	(215.74, 25.90) --
	(215.67, 25.90) --
	(215.67, 25.90) --
	(215.67, 25.90) --
	(215.60, 25.90) --
	(215.60, 25.90) --
	(215.60, 25.90) --
	(215.52, 25.90) --
	(215.52, 25.90) --
	(215.52, 25.90) --
	(215.45, 25.90) --
	(215.45, 25.90) --
	(215.45, 25.90) --
	(215.38, 25.90) --
	(215.38, 25.90) --
	(215.38, 25.90) --
	(215.31, 25.90) --
	(215.31, 25.90) --
	(215.31, 25.90) --
	(215.24, 25.90) --
	(215.24, 25.90) --
	(215.24, 25.90) --
	(215.17, 25.90) --
	(215.17, 25.90) --
	(215.17, 25.90) --
	(215.09, 25.90) --
	(215.09, 25.90) --
	(215.09, 25.90) --
	(215.02, 25.90) --
	(215.02, 25.90) --
	(215.02, 25.90) --
	(214.95, 25.90) --
	(214.95, 25.90) --
	(214.95, 25.90) --
	(214.88, 25.90) --
	(214.88, 25.90) --
	(214.88, 25.90) --
	(214.81, 25.90) --
	(214.81, 25.90) --
	(214.81, 25.90) --
	(214.74, 25.90) --
	(214.74, 25.90) --
	(214.74, 25.90) --
	(214.66, 25.90) --
	(214.66, 25.90) --
	(214.66, 25.90) --
	(214.59, 25.90) --
	(214.59, 25.90) --
	(214.59, 25.90) --
	(214.52, 25.90) --
	(214.52, 25.90) --
	(214.52, 25.90) --
	(214.45, 25.90) --
	(214.45, 25.90) --
	(214.45, 25.90) --
	(214.38, 25.90) --
	(214.38, 25.90) --
	(214.38, 25.90) --
	(214.31, 25.90) --
	(214.31, 25.90) --
	(214.31, 25.90) --
	(214.31, 25.90) --
	(214.31, 25.90) --
	(214.31, 25.90) --
	(214.23, 25.90) --
	(214.23, 25.90) --
	(214.23, 25.90) --
	(214.16, 25.90) --
	(214.16, 25.90) --
	(214.16, 25.90) --
	(214.09, 25.90) --
	(214.09, 25.90) --
	(214.09, 25.90) --
	(214.02, 25.90) --
	(214.02, 25.90) --
	(214.02, 25.90) --
	(213.95, 25.90) --
	(213.95, 25.90) --
	(213.95, 25.90) --
	(213.88, 25.90) --
	(213.88, 25.90) --
	(213.88, 25.90) --
	(213.80, 25.90) --
	(213.80, 25.90) --
	(213.80, 25.90) --
	(213.73, 25.90) --
	(213.73, 25.90) --
	(213.73, 25.90) --
	(213.66, 25.90) --
	(213.66, 25.90) --
	(213.66, 25.90) --
	(213.59, 25.90) --
	(213.59, 25.90) --
	(213.59, 25.90) --
	(213.52, 25.90) --
	(213.52, 25.90) --
	(213.52, 25.90) --
	(213.45, 25.90) --
	(213.45, 25.90) --
	(213.45, 25.90) --
	(213.37, 25.90) --
	(213.37, 25.90) --
	(213.37, 25.90) --
	(213.30, 25.90) --
	(213.30, 25.90) --
	(213.30, 25.90) --
	(213.23, 25.90) --
	(213.23, 25.90) --
	(213.23, 25.90) --
	(213.16, 25.90) --
	(213.16, 25.90) --
	(213.16, 25.90) --
	(213.09, 25.90) --
	(213.09, 25.90) --
	(213.09, 25.90) --
	(213.02, 25.90) --
	(213.02, 25.90) --
	(213.02, 25.90) --
	(212.96, 25.90) --
	(212.96, 25.90) --
	(212.96, 25.90) --
	(212.94, 25.90) --
	(212.94, 25.90) --
	(212.94, 25.90) --
	(212.87, 25.90) --
	(212.87, 25.90) --
	(212.87, 25.90) --
	(212.80, 25.90) --
	(212.80, 25.90) --
	(212.80, 25.90) --
	(212.73, 25.90) --
	(212.73, 25.90) --
	(212.73, 25.90) --
	(212.66, 25.90) --
	(212.66, 25.90) --
	(212.66, 25.90) --
	(212.58, 25.90) --
	(212.58, 25.90) --
	(212.58, 25.90) --
	(212.51, 25.90) --
	(212.51, 25.90) --
	(212.51, 25.90) --
	(212.44, 25.90) --
	(212.44, 25.90) --
	(212.44, 25.90) --
	(212.37, 25.90) --
	(212.37, 25.90) --
	(212.37, 25.90) --
	(212.30, 25.90) --
	(212.30, 25.90) --
	(212.30, 25.90) --
	(212.23, 25.90) --
	(212.23, 25.90) --
	(212.23, 25.90) --
	(212.15, 25.90) --
	(212.15, 25.90) --
	(212.15, 25.90) --
	(212.08, 25.90) --
	(212.08, 25.90) --
	(212.08, 25.90) --
	(212.01, 25.90) --
	(212.01, 25.90) --
	(212.01, 25.90) --
	(211.94, 25.90) --
	(211.94, 25.90) --
	(211.94, 25.90) --
	(211.87, 25.90) --
	(211.87, 25.90) --
	(211.87, 25.90) --
	(211.79, 25.90) --
	(211.79, 25.90) --
	(211.79, 25.90) --
	(211.72, 25.90) --
	(211.72, 25.90) --
	(211.72, 25.90) --
	(211.65, 25.90) --
	(211.65, 25.90) --
	(211.65, 25.90) --
	(211.58, 25.90) --
	(211.58, 25.90) --
	(211.58, 25.90) --
	(211.51, 25.90) --
	(211.51, 25.90) --
	(211.51, 25.90) --
	(211.43, 25.90) --
	(211.43, 25.90) --
	(211.43, 25.90) --
	(211.43, 25.90) --
	(211.43, 25.90) --
	(211.43, 25.90) --
	(211.36, 25.90) --
	(211.36, 25.90) --
	(211.36, 25.90) --
	(211.29, 25.90) --
	(211.29, 25.90) --
	(211.29, 25.90) --
	(211.22, 25.90) --
	(211.22, 25.90) --
	(211.22, 25.90) --
	(211.15, 25.90) --
	(211.15, 25.90) --
	(211.15, 25.90) --
	(211.08, 25.90) --
	(211.08, 25.90) --
	(211.08, 25.90) --
	(211.00, 25.90) --
	(211.00, 25.90) --
	(211.00, 25.90) --
	(210.93, 25.90) --
	(210.93, 25.90) --
	(210.93, 25.90) --
	(210.86, 25.90) --
	(210.86, 25.90) --
	(210.86, 25.90) --
	(210.79, 25.90) --
	(210.79, 25.90) --
	(210.79, 25.90) --
	(210.72, 25.90) --
	(210.72, 25.90) --
	(210.72, 25.90) --
	(210.65, 25.90) --
	(210.65, 25.90) --
	(210.65, 25.90) --
	(210.57, 25.90) --
	(210.57, 25.90) --
	(210.57, 25.90) --
	(210.50, 25.90) --
	(210.50, 25.90) --
	(210.50, 25.90) --
	(210.43, 25.90) --
	(210.43, 25.90) --
	(210.43, 25.90) --
	(210.36, 25.90) --
	(210.36, 25.90) --
	(210.36, 25.90) --
	(210.29, 25.90) --
	(210.29, 25.90) --
	(210.29, 25.90) --
	(210.28, 25.90) --
	(210.28, 25.90) --
	(210.28, 25.90) --
	(210.21, 25.90) --
	(210.21, 25.90) --
	(210.21, 25.90) --
	(210.14, 25.90) --
	(210.14, 25.90) --
	(210.14, 25.90) --
	(210.07, 25.90) --
	(210.07, 25.90) --
	(210.07, 25.90) --
	(210.00, 25.90) --
	(210.00, 25.90) --
	(210.00, 25.90) --
	(209.92, 25.90) --
	(209.92, 25.90) --
	(209.92, 25.90) --
	(209.85, 25.90) --
	(209.85, 25.90) --
	(209.85, 25.90) --
	(209.78, 25.90) --
	(209.78, 25.90) --
	(209.78, 25.90) --
	(209.71, 25.90) --
	(209.71, 25.90) --
	(209.71, 25.90) --
	(209.64, 25.90) --
	(209.64, 25.90) --
	(209.64, 25.90) --
	(209.56, 25.90) --
	(209.56, 25.90) --
	(209.56, 25.90) --
	(209.49, 25.90) --
	(209.49, 25.90) --
	(209.49, 25.90) --
	(209.42, 25.90) --
	(209.42, 25.90) --
	(209.42, 25.90) --
	(209.35, 25.90) --
	(209.35, 25.90) --
	(209.35, 25.90) --
	(209.32, 25.90) --
	(209.32, 25.90) --
	(209.32, 25.90) --
	(209.28, 25.90) --
	(209.28, 25.90) --
	(209.28, 25.90) --
	(209.20, 25.90) --
	(209.20, 25.90) --
	(209.20, 25.90) --
	(209.13, 25.90) --
	(209.13, 25.90) --
	(209.13, 25.90) --
	(209.06, 25.90) --
	(209.06, 25.90) --
	(209.06, 25.90) --
	(208.99, 25.90) --
	(208.99, 25.90) --
	(208.99, 25.90) --
	(208.92, 25.90) --
	(208.92, 25.90) --
	(208.92, 25.90) --
	(208.84, 25.90) --
	(208.84, 25.90) --
	(208.84, 25.90) --
	(208.77, 25.90) --
	(208.77, 25.90) --
	(208.77, 25.90) --
	(208.70, 25.90) --
	(208.70, 25.90) --
	(208.70, 25.90) --
	(208.63, 25.90) --
	(208.63, 25.90) --
	(208.63, 25.90) --
	(208.56, 25.90) --
	(208.56, 25.90) --
	(208.56, 25.90) --
	(208.48, 25.90) --
	(208.48, 25.90) --
	(208.48, 25.90) --
	(208.46, 25.90) --
	(208.46, 25.90) --
	(208.46, 25.90) --
	(208.41, 25.90) --
	(208.41, 25.90) --
	(208.41, 25.90) --
	(208.34, 25.90) --
	(208.34, 25.90) --
	(208.34, 25.90) --
	(208.27, 25.90) --
	(208.27, 25.90) --
	(208.27, 25.90) --
	(208.20, 25.90) --
	(208.20, 25.90) --
	(208.20, 25.90) --
	(208.12, 25.90) --
	(208.12, 25.90) --
	(208.12, 25.90) --
	(208.05, 25.90) --
	(208.05, 25.90) --
	(208.05, 25.90) --
	(207.98, 25.90) --
	(207.98, 25.90) --
	(207.98, 25.90) --
	(207.91, 25.90) --
	(207.91, 25.90) --
	(207.91, 25.90) --
	(207.84, 25.90) --
	(207.84, 25.90) --
	(207.84, 25.90) --
	(207.76, 25.90) --
	(207.76, 25.90) --
	(207.76, 25.90) --
	(207.69, 25.90) --
	(207.69, 25.90) --
	(207.69, 25.90) --
	(207.62, 25.90) --
	(207.62, 25.90) --
	(207.62, 25.90) --
	(207.60, 25.90) --
	(207.60, 25.90) --
	(207.60, 25.90) --
	(207.55, 25.90) --
	(207.55, 25.90) --
	(207.55, 25.90) --
	(207.48, 25.90) --
	(207.48, 25.90) --
	(207.48, 25.90) --
	(207.40, 25.90) --
	(207.40, 25.90) --
	(207.40, 25.90) --
	(207.33, 25.90) --
	(207.33, 25.90) --
	(207.33, 25.90) --
	(207.26, 25.90) --
	(207.26, 25.90) --
	(207.26, 25.90) --
	(207.19, 25.90) --
	(207.19, 25.90) --
	(207.19, 25.90) --
	(207.12, 25.90) --
	(207.12, 25.90) --
	(207.12, 25.90) --
	(207.04, 25.90) --
	(207.04, 25.90) --
	(207.04, 25.90) --
	(206.97, 25.90) --
	(206.97, 25.90) --
	(206.97, 25.90) --
	(206.90, 25.90) --
	(206.90, 25.90) --
	(206.90, 25.90) --
	(206.83, 25.90) --
	(206.83, 25.90) --
	(206.83, 25.90) --
	(206.75, 25.90) --
	(206.75, 25.90) --
	(206.75, 25.90) --
	(206.68, 25.90) --
	(206.68, 25.90) --
	(206.68, 25.90) --
	(206.61, 25.90) --
	(206.61, 25.90) --
	(206.61, 25.90) --
	(206.54, 25.90) --
	(206.54, 25.90) --
	(206.54, 25.90) --
	(206.54, 25.90) --
	(206.54, 25.90) --
	(206.54, 25.90) --
	(206.46, 25.90) --
	(206.46, 25.90) --
	(206.46, 25.90) --
	(206.39, 25.90) --
	(206.39, 25.90) --
	(206.39, 25.90) --
	(206.32, 25.90) --
	(206.32, 25.90) --
	(206.32, 25.90) --
	(206.25, 25.90) --
	(206.25, 25.90) --
	(206.25, 25.90) --
	(206.18, 25.90) --
	(206.18, 25.90) --
	(206.18, 25.90) --
	(206.10, 25.90) --
	(206.10, 25.90) --
	(206.10, 25.90) --
	(206.03, 25.90) --
	(206.03, 25.90) --
	(206.03, 25.90) --
	(205.96, 25.90) --
	(205.96, 25.90) --
	(205.96, 25.90) --
	(205.89, 25.90) --
	(205.89, 25.90) --
	(205.89, 25.90) --
	(205.82, 25.90) --
	(205.82, 25.90) --
	(205.82, 25.90) --
	(205.74, 25.90) --
	(205.74, 25.90) --
	(205.74, 25.90) --
	(205.67, 25.90) --
	(205.67, 25.90) --
	(205.67, 25.90) --
	(205.60, 25.90) --
	(205.60, 25.90) --
	(205.60, 25.90) --
	(205.58, 25.90) --
	(205.58, 25.90) --
	(205.58, 25.90) --
	(205.53, 25.90) --
	(205.53, 25.90) --
	(205.53, 25.90) --
	(205.45, 25.90) --
	(205.45, 25.90) --
	(205.45, 25.90) --
	(205.38, 25.90) --
	(205.38, 25.90) --
	(205.38, 25.90) --
	(205.31, 25.90) --
	(205.31, 25.90) --
	(205.31, 25.90) --
	(205.24, 25.90) --
	(205.24, 25.90) --
	(205.24, 25.90) --
	(205.16, 25.90) --
	(205.16, 25.90) --
	(205.16, 25.90) --
	(205.09, 25.90) --
	(205.09, 25.90) --
	(205.09, 25.90) --
	(205.02, 25.90) --
	(205.02, 25.90) --
	(205.02, 25.90) --
	(204.95, 25.90) --
	(204.95, 25.90) --
	(204.95, 25.90) --
	(204.88, 25.90) --
	(204.88, 25.90) --
	(204.88, 25.90) --
	(204.80, 25.90) --
	(204.80, 25.90) --
	(204.80, 25.90) --
	(204.73, 25.90) --
	(204.73, 25.90) --
	(204.73, 25.90) --
	(204.66, 25.90) --
	(204.66, 25.90) --
	(204.66, 25.90) --
	(204.63, 25.90) --
	(204.63, 25.90) --
	(204.63, 25.90) --
	(204.59, 25.90) --
	(204.59, 25.90) --
	(204.59, 25.90) --
	(204.51, 25.90) --
	(204.51, 25.90) --
	(204.51, 25.90) --
	(204.44, 25.90) --
	(204.44, 25.90) --
	(204.44, 25.90) --
	(204.37, 25.90) --
	(204.37, 25.90) --
	(204.37, 25.90) --
	(204.30, 25.90) --
	(204.30, 25.90) --
	(204.30, 25.90) --
	(204.22, 25.90) --
	(204.22, 25.90) --
	(204.22, 25.90) --
	(204.15, 25.90) --
	(204.15, 25.90) --
	(204.15, 25.90) --
	(204.08, 25.90) --
	(204.08, 25.90) --
	(204.08, 25.90) --
	(204.01, 25.90) --
	(204.01, 25.90) --
	(204.01, 25.90) --
	(203.94, 25.90) --
	(203.94, 25.90) --
	(203.94, 25.90) --
	(203.86, 25.90) --
	(203.86, 25.90) --
	(203.86, 25.90) --
	(203.79, 25.90) --
	(203.79, 25.90) --
	(203.79, 25.90) --
	(203.76, 25.90) --
	(203.76, 25.90) --
	(203.76, 25.90) --
	(203.72, 25.90) --
	(203.72, 25.90) --
	(203.72, 25.90) --
	(203.65, 25.90) --
	(203.65, 25.90) --
	(203.64, 25.90) --
	(203.57, 25.90) --
	(203.57, 25.90) --
	(203.57, 25.90) --
	(203.50, 25.90) --
	(203.50, 25.90) --
	(203.50, 25.90) --
	(203.43, 25.90) --
	(203.43, 25.90) --
	(203.43, 25.90) --
	(203.36, 25.90) --
	(203.36, 25.90) --
	(203.36, 25.90) --
	(203.28, 25.90) --
	(203.28, 25.90) --
	(203.28, 25.90) --
	(203.21, 25.90) --
	(203.21, 25.90) --
	(203.21, 25.90) --
	(203.14, 25.90) --
	(203.14, 25.90) --
	(203.14, 25.90) --
	(203.07, 25.90) --
	(203.07, 25.90) --
	(203.07, 25.90) --
	(202.99, 25.90) --
	(202.99, 25.90) --
	(202.99, 25.90) --
	(202.92, 25.90) --
	(202.92, 25.90) --
	(202.92, 25.90) --
	(202.85, 25.90) --
	(202.85, 25.90) --
	(202.85, 25.90) --
	(202.81, 25.90) --
	(202.81, 25.90) --
	(202.81, 25.90) --
	(202.78, 25.90) --
	(202.78, 25.90) --
	(202.78, 25.90) --
	(202.70, 25.90) --
	(202.70, 25.90) --
	(202.70, 25.90) --
	(202.63, 25.90) --
	(202.63, 25.90) --
	(202.63, 25.90) --
	(202.56, 25.90) --
	(202.56, 25.90) --
	(202.56, 25.90) --
	(202.49, 25.90) --
	(202.49, 25.90) --
	(202.49, 25.90) --
	(202.41, 25.90) --
	(202.41, 25.90) --
	(202.41, 25.90) --
	(202.34, 25.90) --
	(202.34, 25.90) --
	(202.34, 25.90) --
	(202.27, 25.90) --
	(202.27, 25.90) --
	(202.27, 25.90) --
	(202.20, 25.90) --
	(202.20, 25.90) --
	(202.20, 25.90) --
	(202.12, 25.90) --
	(202.12, 25.90) --
	(202.12, 25.90) --
	(202.05, 25.90) --
	(202.05, 25.90) --
	(202.05, 25.90) --
	(202.04, 25.90) --
	(202.04, 25.90) --
	(202.04, 25.90) --
	(201.98, 25.90) --
	(201.98, 25.90) --
	(201.98, 25.90) --
	(201.91, 25.90) --
	(201.91, 25.90) --
	(201.91, 25.90) --
	(201.83, 25.90) --
	(201.83, 25.90) --
	(201.83, 25.90) --
	(201.76, 25.90) --
	(201.76, 25.90) --
	(201.76, 25.90) --
	(201.69, 25.90) --
	(201.69, 25.90) --
	(201.69, 25.90) --
	(201.62, 25.90) --
	(201.62, 25.90) --
	(201.62, 25.90) --
	(201.54, 25.90) --
	(201.54, 25.90) --
	(201.54, 25.90) --
	(201.47, 25.90) --
	(201.47, 25.90) --
	(201.47, 25.90) --
	(201.40, 25.90) --
	(201.40, 25.90) --
	(201.40, 25.90) --
	(201.33, 25.90) --
	(201.33, 25.90) --
	(201.33, 25.90) --
	(201.26, 25.90) --
	(201.26, 25.90) --
	(201.26, 25.90) --
	(201.18, 25.90) --
	(201.18, 25.90) --
	(201.18, 25.90) --
	(201.11, 25.90) --
	(201.11, 25.90) --
	(201.11, 25.90) --
	(201.04, 25.90) --
	(201.04, 25.90) --
	(201.04, 25.90) --
	(200.96, 25.90) --
	(200.96, 25.90) --
	(200.96, 25.90) --
	(200.89, 25.90) --
	(200.89, 25.90) --
	(200.89, 25.90) --
	(200.82, 25.90) --
	(200.82, 25.90) --
	(200.82, 25.90) --
	(200.79, 25.90) --
	(200.79, 25.90) --
	(200.79, 25.90) --
	(200.75, 25.90) --
	(200.75, 25.90) --
	(200.75, 25.90) --
	(200.67, 25.90) --
	(200.67, 25.90) --
	(200.67, 25.90) --
	(200.60, 25.90) --
	(200.60, 25.90) --
	(200.60, 25.90) --
	(200.53, 25.90) --
	(200.53, 25.90) --
	(200.53, 25.90) --
	(200.46, 25.90) --
	(200.46, 25.90) --
	(200.46, 25.90) --
	(200.38, 25.90) --
	(200.38, 25.90) --
	(200.38, 25.90) --
	(200.31, 25.90) --
	(200.31, 25.90) --
	(200.31, 25.90) --
	(200.24, 25.90) --
	(200.24, 25.90) --
	(200.24, 25.90) --
	(200.17, 25.90) --
	(200.17, 25.90) --
	(200.17, 25.90) --
	(200.09, 25.90) --
	(200.09, 25.90) --
	(200.09, 25.90) --
	(200.03, 25.90) --
	(200.03, 25.90) --
	(200.03, 25.90) --
	(200.02, 25.90) --
	(200.02, 25.90) --
	(200.02, 25.90) --
	(199.95, 25.90) --
	(199.95, 25.90) --
	(199.95, 25.90) --
	(199.88, 25.90) --
	(199.88, 25.90) --
	(199.88, 25.90) --
	(199.80, 25.90) --
	(199.80, 25.90) --
	(199.80, 25.90) --
	(199.73, 25.90) --
	(199.73, 25.90) --
	(199.73, 25.90) --
	(199.66, 25.90) --
	(199.66, 25.90) --
	(199.66, 25.90) --
	(199.59, 25.90) --
	(199.59, 25.90) --
	(199.59, 25.90) --
	(199.51, 25.90) --
	(199.51, 25.90) --
	(199.51, 25.90) --
	(199.44, 25.90) --
	(199.44, 25.90) --
	(199.44, 25.90) --
	(199.37, 25.90) --
	(199.37, 25.90) --
	(199.37, 25.90) --
	(199.29, 25.90) --
	(199.29, 25.90) --
	(199.29, 25.90) --
	(199.26, 25.90) --
	(199.26, 25.90) --
	(199.26, 25.90) --
	(199.22, 25.90) --
	(199.22, 25.90) --
	(199.22, 25.90) --
	(199.15, 25.90) --
	(199.15, 25.90) --
	(199.15, 25.90) --
	(199.08, 25.90) --
	(199.08, 25.90) --
	(199.08, 25.90) --
	(199.00, 25.90) --
	(199.00, 25.90) --
	(199.00, 25.90) --
	(198.93, 25.90) --
	(198.93, 25.90) --
	(198.93, 25.90) --
	(198.86, 25.90) --
	(198.86, 25.90) --
	(198.86, 25.90) --
	(198.79, 25.90) --
	(198.79, 25.90) --
	(198.79, 25.90) --
	(198.71, 25.90) --
	(198.71, 25.90) --
	(198.71, 25.90) --
	(198.64, 25.90) --
	(198.64, 25.90) --
	(198.64, 25.90) --
	(198.57, 25.90) --
	(198.57, 25.90) --
	(198.57, 25.90) --
	(198.50, 25.90) --
	(198.50, 25.90) --
	(198.50, 25.90) --
	(198.42, 25.90) --
	(198.42, 25.90) --
	(198.42, 25.90) --
	(198.35, 25.90) --
	(198.35, 25.90) --
	(198.35, 25.90) --
	(198.28, 25.90) --
	(198.28, 25.90) --
	(198.28, 25.90) --
	(198.20, 25.90) --
	(198.20, 25.90) --
	(198.20, 25.90) --
	(198.13, 25.90) --
	(198.13, 25.90) --
	(198.13, 25.90) --
	(198.11, 25.90) --
	(198.11, 25.90) --
	(198.11, 25.90) --
	(198.06, 25.90) --
	(198.06, 25.90) --
	(198.06, 25.90) --
	(197.99, 25.90) --
	(197.99, 25.90) --
	(197.99, 25.90) --
	(197.91, 25.90) --
	(197.91, 25.90) --
	(197.91, 25.90) --
	(197.84, 25.90) --
	(197.84, 25.90) --
	(197.84, 25.90) --
	(197.77, 25.90) --
	(197.77, 25.90) --
	(197.77, 25.90) --
	(197.69, 25.90) --
	(197.69, 25.90) --
	(197.69, 25.90) --
	(197.62, 25.90) --
	(197.62, 25.90) --
	(197.62, 25.90) --
	(197.55, 25.90) --
	(197.55, 25.90) --
	(197.55, 25.90) --
	(197.48, 25.90) --
	(197.48, 25.90) --
	(197.48, 25.90) --
	(197.40, 25.90) --
	(197.40, 25.90) --
	(197.40, 25.90) --
	(197.33, 25.90) --
	(197.33, 25.90) --
	(197.33, 25.90) --
	(197.26, 25.90) --
	(197.26, 25.90) --
	(197.26, 25.90) --
	(197.19, 25.90) --
	(197.19, 25.90) --
	(197.19, 25.90) --
	(197.11, 25.90) --
	(197.11, 25.90) --
	(197.11, 25.90) --
	(197.06, 25.90) --
	(197.06, 25.90) --
	(197.06, 25.90) --
	(197.04, 25.90) --
	(197.04, 25.90) --
	(197.04, 25.90) --
	(196.97, 25.90) --
	(196.97, 25.90) --
	(196.97, 25.90) --
	(196.89, 25.90) --
	(196.89, 25.90) --
	(196.89, 25.90) --
	(196.82, 25.90) --
	(196.82, 25.90) --
	(196.82, 25.90) --
	(196.75, 25.90) --
	(196.75, 25.90) --
	(196.75, 25.90) --
	(196.68, 25.90) --
	(196.68, 25.90) --
	(196.68, 25.90) --
	(196.60, 25.90) --
	(196.60, 25.90) --
	(196.60, 25.90) --
	(196.53, 25.90) --
	(196.53, 25.90) --
	(196.53, 25.90) --
	(196.46, 25.90) --
	(196.46, 25.90) --
	(196.46, 25.90) --
	(196.38, 25.90) --
	(196.38, 25.90) --
	(196.38, 25.90) --
	(196.31, 25.90) --
	(196.31, 25.90) --
	(196.31, 25.90) --
	(196.24, 25.90) --
	(196.24, 25.90) --
	(196.24, 25.90) --
	(196.17, 25.90) --
	(196.17, 25.90) --
	(196.17, 25.90) --
	(196.09, 25.90) --
	(196.09, 25.90) --
	(196.09, 25.90) --
	(196.02, 25.90) --
	(196.02, 25.90) --
	(196.02, 25.90) --
	(195.95, 25.90) --
	(195.95, 25.90) --
	(195.95, 25.90) --
	(195.87, 25.90) --
	(195.87, 25.90) --
	(195.87, 25.90) --
	(195.80, 25.90) --
	(195.80, 25.90) --
	(195.80, 25.90) --
	(195.73, 25.90) --
	(195.73, 25.90) --
	(195.73, 25.90) --
	(195.66, 25.90) --
	(195.66, 25.90) --
	(195.66, 25.90) --
	(195.58, 25.90) --
	(195.58, 25.90) --
	(195.58, 25.90) --
	(195.51, 25.90) --
	(195.51, 25.90) --
	(195.51, 25.90) --
	(195.44, 25.90) --
	(195.44, 25.90) --
	(195.44, 25.90) --
	(195.36, 25.90) --
	(195.36, 25.90) --
	(195.36, 25.90) --
	(195.29, 25.90) --
	(195.29, 25.90) --
	(195.29, 25.90) --
	(195.23, 25.90) --
	(195.23, 25.90) --
	(195.23, 25.90) --
	(195.22, 25.90) --
	(195.22, 25.90) --
	(195.22, 25.90) --
	(195.15, 25.90) --
	(195.15, 25.90) --
	(195.15, 25.90) --
	(195.07, 25.90) --
	(195.07, 25.90) --
	(195.07, 25.90) --
	(195.00, 25.90) --
	(195.00, 25.90) --
	(195.00, 25.90) --
	(194.93, 25.90) --
	(194.93, 25.90) --
	(194.93, 25.90) --
	(194.85, 25.90) --
	(194.85, 25.90) --
	(194.85, 25.90) --
	(194.78, 25.90) --
	(194.78, 25.90) --
	(194.78, 25.90) --
	(194.71, 25.90) --
	(194.71, 25.90) --
	(194.71, 25.90) --
	(194.63, 25.90) --
	(194.63, 25.90) --
	(194.63, 25.90) --
	(194.56, 25.90) --
	(194.56, 25.90) --
	(194.56, 25.90) --
	(194.49, 25.90) --
	(194.49, 25.90) --
	(194.49, 25.90) --
	(194.42, 25.90) --
	(194.42, 25.90) --
	(194.42, 25.90) --
	(194.34, 25.90) --
	(194.34, 25.90) --
	(194.34, 25.90) --
	(194.27, 25.90) --
	(194.27, 25.90) --
	(194.27, 25.90) --
	(194.20, 25.90) --
	(194.20, 25.90) --
	(194.20, 25.90) --
	(194.12, 25.90) --
	(194.12, 25.90) --
	(194.12, 25.90) --
	(194.05, 25.90) --
	(194.05, 25.90) --
	(194.05, 25.90) --
	(193.98, 25.90) --
	(193.98, 25.90) --
	(193.98, 25.90) --
	(193.91, 25.90) --
	(193.91, 25.90) --
	(193.91, 25.90) --
	(193.83, 25.90) --
	(193.83, 25.90) --
	(193.83, 25.90) --
	(193.76, 25.90) --
	(193.76, 25.90) --
	(193.76, 25.90) --
	(193.69, 25.90) --
	(193.69, 25.90) --
	(193.69, 25.90) --
	(193.61, 25.90) --
	(193.61, 25.90) --
	(193.61, 25.90) --
	(193.54, 25.90) --
	(193.54, 25.90) --
	(193.54, 25.90) --
	(193.47, 25.90) --
	(193.47, 25.90) --
	(193.47, 25.90) --
	(193.41, 25.90) --
	(193.41, 25.90) --
	(193.41, 25.90) --
	(193.39, 25.90) --
	(193.39, 25.90) --
	(193.39, 25.90) --
	(193.32, 25.90) --
	(193.32, 25.90) --
	(193.32, 25.90) --
	(193.25, 25.90) --
	(193.25, 25.90) --
	(193.25, 25.90) --
	(193.17, 25.90) --
	(193.17, 25.90) --
	(193.17, 25.90) --
	(193.10, 25.90) --
	(193.10, 25.90) --
	(193.10, 25.90) --
	(193.03, 25.90) --
	(193.03, 25.90) --
	(193.03, 25.90) --
	(192.96, 25.90) --
	(192.96, 25.90) --
	(192.96, 25.90) --
	(192.88, 25.90) --
	(192.88, 25.90) --
	(192.88, 25.90) --
	(192.81, 25.90) --
	(192.81, 25.90) --
	(192.81, 25.90) --
	(192.74, 25.90) --
	(192.74, 25.90) --
	(192.74, 25.90) --
	(192.66, 25.90) --
	(192.66, 25.90) --
	(192.66, 25.90) --
	(192.59, 25.90) --
	(192.59, 25.90) --
	(192.59, 25.90) --
	(192.52, 25.90) --
	(192.52, 25.90) --
	(192.52, 25.90) --
	(192.44, 25.90) --
	(192.44, 25.90) --
	(192.44, 25.90) --
	(192.37, 25.90) --
	(192.37, 25.90) --
	(192.37, 25.90) --
	(192.30, 25.90) --
	(192.30, 25.90) --
	(192.30, 25.90) --
	(192.23, 25.90) --
	(192.23, 25.90) --
	(192.23, 25.90) --
	(192.15, 25.90) --
	(192.15, 25.90) --
	(192.15, 25.90) --
	(192.08, 25.90) --
	(192.08, 25.90) --
	(192.08, 25.90) --
	(192.07, 25.90) --
	(192.07, 25.90) --
	(192.07, 25.90) --
	(192.01, 25.90) --
	(192.01, 25.90) --
	(192.01, 25.90) --
	(191.93, 25.90) --
	(191.93, 25.90) --
	(191.93, 25.90) --
	(191.86, 25.90) --
	(191.86, 25.90) --
	(191.86, 25.90) --
	(191.79, 25.90) --
	(191.79, 25.90) --
	(191.79, 25.90) --
	(191.71, 25.90) --
	(191.71, 25.90) --
	(191.71, 25.90) --
	(191.64, 25.90) --
	(191.64, 25.90) --
	(191.64, 25.90) --
	(191.57, 25.90) --
	(191.57, 25.90) --
	(191.57, 25.90) --
	(191.49, 25.90) --
	(191.49, 25.90) --
	(191.49, 25.90) --
	(191.42, 25.90) --
	(191.42, 25.90) --
	(191.42, 25.90) --
	(191.35, 25.90) --
	(191.35, 25.90) --
	(191.35, 25.90) --
	(191.27, 25.90) --
	(191.27, 25.90) --
	(191.27, 25.90) --
	(191.20, 25.90) --
	(191.20, 25.90) --
	(191.20, 25.90) --
	(191.13, 25.90) --
	(191.13, 25.90) --
	(191.13, 25.90) --
	(191.06, 25.90) --
	(191.06, 25.90) --
	(191.06, 25.90) --
	(190.98, 25.90) --
	(190.98, 25.90) --
	(190.98, 25.90) --
	(190.91, 25.90) --
	(190.91, 25.90) --
	(190.91, 25.90) --
	(190.84, 25.90) --
	(190.84, 25.90) --
	(190.84, 25.90) --
	(190.76, 25.90) --
	(190.76, 25.90) --
	(190.76, 25.90) --
	(190.69, 25.90) --
	(190.69, 25.90) --
	(190.69, 25.90) --
	(190.62, 25.90) --
	(190.62, 25.90) --
	(190.62, 25.90) --
	(190.54, 25.90) --
	(190.54, 25.90) --
	(190.54, 25.90) --
	(190.47, 25.90) --
	(190.47, 25.90) --
	(190.47, 25.90) --
	(190.40, 25.90) --
	(190.40, 25.90) --
	(190.40, 25.90) --
	(190.32, 25.90) --
	(190.32, 25.90) --
	(190.32, 25.90) --
	(190.25, 25.90) --
	(190.25, 25.90) --
	(190.25, 25.90) --
	(190.25, 25.90) --
	(190.25, 25.90) --
	(190.25, 25.90) --
	(190.18, 25.90) --
	(190.18, 25.90) --
	(190.18, 25.90) --
	(190.10, 25.90) --
	(190.10, 25.90) --
	(190.10, 25.90) --
	(190.03, 25.90) --
	(190.03, 25.90) --
	(190.03, 25.90) --
	(189.96, 25.90) --
	(189.96, 25.90) --
	(189.96, 25.90) --
	(189.88, 25.90) --
	(189.88, 25.90) --
	(189.88, 25.90) --
	(189.81, 25.90) --
	(189.81, 25.90) --
	(189.81, 25.90) --
	(189.74, 25.90) --
	(189.74, 25.90) --
	(189.74, 25.90) --
	(189.66, 25.90) --
	(189.66, 25.90) --
	(189.66, 25.90) --
	(189.59, 25.90) --
	(189.59, 25.90) --
	(189.59, 25.90) --
	(189.52, 25.90) --
	(189.52, 25.90) --
	(189.52, 25.90) --
	(189.44, 25.90) --
	(189.44, 25.90) --
	(189.44, 25.90) --
	(189.37, 25.90) --
	(189.37, 25.90) --
	(189.37, 25.90) --
	(189.30, 25.90) --
	(189.30, 25.90) --
	(189.30, 25.90) --
	(189.29, 25.90) --
	(189.29, 25.90) --
	(189.29, 25.90) --
	(189.22, 25.90) --
	(189.22, 25.90) --
	(189.22, 25.90) --
	(189.15, 25.90) --
	(189.15, 25.90) --
	(189.15, 25.90) --
	(189.08, 25.90) --
	(189.08, 25.90) --
	(189.08, 25.90) --
	(189.01, 25.90) --
	(189.01, 25.90) --
	(189.01, 25.90) --
	(188.93, 25.90) --
	(188.93, 25.90) --
	(188.93, 25.90) --
	(188.86, 25.90) --
	(188.86, 25.90) --
	(188.86, 25.90) --
	(188.79, 25.90) --
	(188.79, 25.90) --
	(188.79, 25.90) --
	(188.71, 25.90) --
	(188.71, 25.90) --
	(188.71, 25.90) --
	(188.64, 25.90) --
	(188.64, 25.90) --
	(188.64, 25.90) --
	(188.56, 25.90) --
	(188.56, 25.90) --
	(188.56, 25.90) --
	(188.49, 25.90) --
	(188.49, 25.90) --
	(188.49, 25.90) --
	(188.42, 25.90) --
	(188.42, 25.90) --
	(188.42, 25.90) --
	(188.34, 25.90) --
	(188.34, 25.90) --
	(188.34, 25.90) --
	(188.33, 25.90) --
	(188.33, 25.90) --
	(188.33, 25.90) --
	(188.27, 25.90) --
	(188.27, 25.90) --
	(188.27, 25.90) --
	(188.20, 25.90) --
	(188.20, 25.90) --
	(188.20, 25.90) --
	(188.13, 25.90) --
	(188.13, 25.90) --
	(188.13, 25.90) --
	(188.05, 25.90) --
	(188.05, 25.90) --
	(188.05, 25.90) --
	(187.98, 25.90) --
	(187.98, 25.90) --
	(187.98, 25.90) --
	(187.91, 25.90) --
	(187.91, 25.90) --
	(187.91, 25.90) --
	(187.83, 25.90) --
	(187.83, 25.90) --
	(187.83, 25.90) --
	(187.76, 25.90) --
	(187.76, 25.90) --
	(187.76, 25.90) --
	(187.76, 25.90) --
	(187.76, 25.90) --
	(187.76, 25.90) --
	(187.68, 25.90) --
	(187.68, 25.90) --
	(187.68, 25.90) --
	(187.61, 25.90) --
	(187.61, 25.90) --
	(187.61, 25.90) --
	(187.54, 25.90) --
	(187.54, 25.90) --
	(187.54, 25.90) --
	(187.46, 25.90) --
	(187.46, 25.90) --
	(187.46, 25.90) --
	(187.39, 25.90) --
	(187.39, 25.90) --
	(187.39, 25.90) --
	(187.32, 25.90) --
	(187.32, 25.90) --
	(187.32, 25.90) --
	(187.25, 25.90) --
	(187.25, 25.90) --
	(187.25, 25.90) --
	(187.18, 25.90) --
	(187.18, 25.90) --
	(187.18, 25.90) --
	(187.17, 25.90) --
	(187.17, 25.90) --
	(187.17, 25.90) --
	(187.10, 25.90) --
	(187.10, 25.90) --
	(187.10, 25.90) --
	(187.02, 25.90) --
	(187.02, 25.90) --
	(187.02, 25.90) --
	(186.95, 25.90) --
	(186.95, 25.90) --
	(186.95, 25.90) --
	(186.88, 25.90) --
	(186.88, 25.90) --
	(186.88, 25.90) --
	(186.80, 25.90) --
	(186.80, 25.90) --
	(186.80, 25.90) --
	(186.73, 25.90) --
	(186.73, 25.90) --
	(186.73, 25.90) --
	(186.71, 25.90) --
	(186.71, 25.90) --
	(186.71, 25.90) --
	(186.66, 25.90) --
	(186.66, 25.90) --
	(186.66, 25.90) --
	(186.58, 25.90) --
	(186.58, 25.90) --
	(186.58, 25.90) --
	(186.51, 25.90) --
	(186.51, 25.90) --
	(186.51, 25.90) --
	(186.44, 25.90) --
	(186.44, 25.90) --
	(186.44, 25.90) --
	(186.36, 25.90) --
	(186.36, 25.90) --
	(186.36, 25.90) --
	(186.29, 25.90) --
	(186.29, 25.90) --
	(186.29, 25.90) --
	(186.22, 25.90) --
	(186.22, 25.90) --
	(186.22, 25.90) --
	(186.14, 25.90) --
	(186.14, 25.90) --
	(186.14, 25.90) --
	(186.07, 25.90) --
	(186.07, 25.90) --
	(186.07, 25.90) --
	(186.03, 25.90) --
	(186.03, 25.90) --
	(186.03, 25.90) --
	(186.00, 25.90) --
	(186.00, 25.90) --
	(186.00, 25.90) --
	(185.92, 25.90) --
	(185.92, 25.90) --
	(185.92, 25.90) --
	(185.85, 25.90) --
	(185.85, 25.90) --
	(185.85, 25.90) --
	(185.78, 25.90) --
	(185.78, 25.90) --
	(185.78, 25.90) --
	(185.75, 25.90) --
	(185.75, 25.90) --
	(185.75, 25.90) --
	(185.70, 25.90) --
	(185.70, 25.90) --
	(185.70, 25.90) --
	(185.63, 25.90) --
	(185.63, 25.90) --
	(185.63, 25.90) --
	(185.55, 25.90) --
	(185.55, 25.90) --
	(185.55, 25.90) --
	(185.48, 25.90) --
	(185.48, 25.90) --
	(185.48, 25.90) --
	(185.41, 25.90) --
	(185.41, 25.90) --
	(185.41, 25.90) --
	(185.33, 25.90) --
	(185.33, 25.90) --
	(185.33, 25.90) --
	(185.27, 25.90) --
	(185.27, 25.90) --
	(185.27, 25.90) --
	(185.26, 25.90) --
	(185.26, 25.90) --
	(185.26, 25.90) --
	(185.19, 25.90) --
	(185.19, 25.90) --
	(185.19, 25.90) --
	(185.11, 25.90) --
	(185.11, 25.90) --
	(185.11, 25.90) --
	(185.04, 25.90) --
	(185.04, 25.90) --
	(185.04, 25.90) --
	(184.98, 25.90) --
	(184.98, 25.90) --
	(184.98, 25.90) --
	(184.97, 25.90) --
	(184.97, 25.90) --
	(184.97, 25.90) --
	(184.89, 25.90) --
	(184.89, 25.90) --
	(184.89, 25.90) --
	(184.82, 25.90) --
	(184.82, 25.90) --
	(184.82, 25.90) --
	(184.75, 25.90) --
	(184.75, 25.90) --
	(184.75, 25.90) --
	(184.67, 25.90) --
	(184.67, 25.90) --
	(184.67, 25.90) --
	(184.60, 25.90) --
	(184.60, 25.90) --
	(184.60, 25.90) --
	(184.53, 25.90) --
	(184.53, 25.90) --
	(184.53, 25.90) --
	(184.45, 25.90) --
	(184.45, 25.90) --
	(184.45, 25.90) --
	(184.41, 25.90) --
	(184.41, 25.90) --
	(184.41, 25.90) --
	(184.38, 25.90) --
	(184.38, 25.90) --
	(184.38, 25.90) --
	(184.30, 25.90) --
	(184.30, 25.90) --
	(184.30, 25.90) --
	(184.23, 25.90) --
	(184.23, 25.90) --
	(184.23, 25.90) --
	(184.16, 25.90) --
	(184.16, 25.90) --
	(184.16, 25.90) --
	(184.12, 25.90) --
	(184.12, 25.90) --
	(184.12, 25.90) --
	(184.09, 25.90) --
	(184.09, 25.90) --
	(184.09, 25.90) --
	(184.01, 25.90) --
	(184.01, 25.90) --
	(184.01, 25.90) --
	(183.94, 25.90) --
	(183.94, 25.90) --
	(183.94, 25.90) --
	(183.86, 25.90) --
	(183.86, 25.90) --
	(183.86, 25.90) --
	(183.79, 25.90) --
	(183.79, 25.90) --
	(183.79, 25.90) --
	(183.72, 25.90) --
	(183.72, 25.90) --
	(183.72, 25.90) --
	(183.64, 25.90) --
	(183.64, 25.90) --
	(183.64, 25.90) --
	(183.57, 25.90) --
	(183.57, 25.90) --
	(183.57, 25.90) --
	(183.50, 25.90) --
	(183.50, 25.90) --
	(183.50, 25.90) --
	(183.42, 25.90) --
	(183.42, 25.90) --
	(183.42, 25.90) --
	(183.35, 25.90) --
	(183.35, 25.90) --
	(183.35, 25.90) --
	(183.35, 25.90) --
	(183.35, 25.90) --
	(183.35, 25.90) --
	(183.27, 25.90) --
	(183.27, 25.90) --
	(183.27, 25.90) --
	(183.20, 25.90) --
	(183.20, 25.90) --
	(183.20, 25.90) --
	(183.13, 25.90) --
	(183.13, 25.90) --
	(183.13, 25.90) --
	(183.05, 25.90) --
	(183.05, 25.90) --
	(183.05, 25.90) --
	(182.98, 25.90) --
	(182.98, 25.90) --
	(182.98, 25.90) --
	(182.91, 25.90) --
	(182.91, 25.90) --
	(182.91, 25.90) --
	(182.83, 25.90) --
	(182.83, 25.90) --
	(182.83, 25.90) --
	(182.76, 25.90) --
	(182.76, 25.90) --
	(182.76, 25.90) --
	(182.69, 25.90) --
	(182.69, 25.90) --
	(182.68, 25.90) --
	(182.61, 25.90) --
	(182.61, 25.90) --
	(182.61, 25.90) --
	(182.58, 25.90) --
	(182.58, 25.90) --
	(182.58, 25.90) --
	(182.54, 25.90) --
	(182.54, 25.90) --
	(182.54, 25.90) --
	(182.46, 25.90) --
	(182.46, 25.90) --
	(182.46, 25.90) --
	(182.39, 25.90) --
	(182.39, 25.90) --
	(182.39, 25.90) --
	(182.32, 25.90) --
	(182.32, 25.90) --
	(182.32, 25.90) --
	(182.24, 25.90) --
	(182.24, 25.90) --
	(182.24, 25.90) --
	(182.17, 25.90) --
	(182.17, 25.90) --
	(182.17, 25.90) --
	(182.10, 25.90) --
	(182.10, 25.90) --
	(182.10, 25.90) --
	(182.02, 25.90) --
	(182.02, 25.90) --
	(182.02, 25.90) --
	(182.01, 25.90) --
	(182.01, 25.90) --
	(182.01, 25.90) --
	(181.95, 25.90) --
	(181.95, 25.90) --
	(181.95, 25.90) --
	(181.87, 25.90) --
	(181.87, 25.90) --
	(181.87, 25.90) --
	(181.80, 25.90) --
	(181.80, 25.90) --
	(181.80, 25.90) --
	(181.73, 25.90) --
	(181.73, 25.90) --
	(181.73, 25.90) --
	(181.65, 25.90) --
	(181.65, 25.90) --
	(181.65, 25.90) --
	(181.58, 25.90) --
	(181.58, 25.90) --
	(181.58, 25.90) --
	(181.53, 25.90) --
	(181.53, 25.90) --
	(181.53, 25.90) --
	(181.51, 25.90) --
	(181.51, 25.90) --
	(181.51, 25.90) --
	(181.43, 25.90) --
	(181.43, 25.90) --
	(181.43, 25.90) --
	(181.36, 25.90) --
	(181.36, 25.90) --
	(181.36, 25.90) --
	(181.28, 25.90) --
	(181.28, 25.90) --
	(181.28, 25.90) --
	(181.21, 25.90) --
	(181.21, 25.90) --
	(181.21, 25.90) --
	(181.14, 25.90) --
	(181.14, 25.90) --
	(181.14, 25.90) --
	(181.06, 25.90) --
	(181.06, 25.90) --
	(181.06, 25.90) --
	(181.05, 25.90) --
	(181.05, 25.90) --
	(181.05, 25.90) --
	(180.99, 25.90) --
	(180.99, 25.90) --
	(180.99, 25.90) --
	(180.92, 25.90) --
	(180.92, 25.90) --
	(180.92, 25.90) --
	(180.84, 25.90) --
	(180.84, 25.90) --
	(180.84, 25.90) --
	(180.77, 25.90) --
	(180.77, 25.90) --
	(180.77, 25.90) --
	(180.70, 25.90) --
	(180.70, 25.90) --
	(180.70, 25.90) --
	(180.62, 25.90) --
	(180.62, 25.90) --
	(180.62, 25.90) --
	(180.57, 25.90) --
	(180.57, 25.90) --
	(180.57, 25.90) --
	(180.55, 25.90) --
	(180.55, 25.90) --
	(180.55, 25.90) --
	(180.47, 25.90) --
	(180.47, 25.90) --
	(180.47, 25.90) --
	(180.40, 25.90) --
	(180.40, 25.90) --
	(180.40, 25.90) --
	(180.33, 25.90) --
	(180.33, 25.90) --
	(180.33, 25.90) --
	(180.25, 25.90) --
	(180.25, 25.90) --
	(180.25, 25.90) --
	(180.19, 25.90) --
	(180.19, 25.90) --
	(180.19, 25.90) --
	(180.18, 25.90) --
	(180.18, 25.90) --
	(180.18, 25.90) --
	(180.10, 25.90) --
	(180.10, 25.90) --
	(180.10, 25.90) --
	(180.03, 25.90) --
	(180.03, 25.90) --
	(180.03, 25.90) --
	(179.96, 25.90) --
	(179.96, 25.90) --
	(179.96, 25.90) --
	(179.88, 25.90) --
	(179.88, 25.90) --
	(179.88, 25.90) --
	(179.81, 25.90) --
	(179.81, 25.90) --
	(179.81, 25.90) --
	(179.81, 25.90) --
	(179.81, 25.90) --
	(179.81, 25.90) --
	(179.73, 25.90) --
	(179.73, 25.90) --
	(179.73, 25.90) --
	(179.71, 25.90) --
	(179.71, 25.90) --
	(179.71, 25.90) --
	(179.66, 25.90) --
	(179.66, 25.90) --
	(179.66, 25.90) --
	(179.59, 25.90) --
	(179.59, 25.90) --
	(179.59, 25.90) --
	(179.51, 25.90) --
	(179.51, 25.90) --
	(179.51, 25.90) --
	(179.44, 25.90) --
	(179.44, 25.90) --
	(179.44, 25.90) --
	(179.37, 25.90) --
	(179.37, 25.90) --
	(179.37, 25.90) --
	(179.33, 25.90) --
	(179.33, 25.90) --
	(179.33, 25.90) --
	(179.29, 25.90) --
	(179.29, 25.90) --
	(179.29, 25.90) --
	(179.22, 25.90) --
	(179.22, 25.90) --
	(179.22, 25.90) --
	(179.14, 25.90) --
	(179.14, 25.90) --
	(179.14, 25.90) --
	(179.07, 25.90) --
	(179.07, 25.90) --
	(179.07, 25.90) --
	(179.00, 25.90) --
	(179.00, 25.90) --
	(179.00, 25.90) --
	(178.94, 25.90) --
	(178.94, 25.90) --
	(178.94, 25.90) --
	(178.92, 25.90) --
	(178.92, 25.90) --
	(178.92, 25.90) --
	(178.85, 25.90) --
	(178.85, 25.90) --
	(178.85, 25.90) --
	(178.78, 25.90) --
	(178.78, 25.90) --
	(178.78, 25.90) --
	(178.75, 25.90) --
	(178.75, 25.90) --
	(178.75, 25.90) --
	(178.70, 25.90) --
	(178.70, 25.90) --
	(178.70, 25.90) --
	(178.63, 25.90) --
	(178.63, 25.90) --
	(178.63, 25.90) --
	(178.56, 25.90) --
	(178.56, 25.90) --
	(178.56, 25.90) --
	(178.55, 25.90) --
	(178.55, 25.90) --
	(178.55, 25.90) --
	(178.48, 25.90) --
	(178.48, 25.90) --
	(178.48, 25.90) --
	(178.41, 25.90) --
	(178.41, 25.90) --
	(178.41, 25.90) --
	(178.37, 25.90) --
	(178.37, 25.90) --
	(178.37, 25.90) --
	(178.33, 25.90) --
	(178.33, 25.90) --
	(178.33, 25.90) --
	(178.26, 25.90) --
	(178.26, 25.90) --
	(178.26, 25.90) --
	(178.18, 25.90) --
	(178.18, 25.90) --
	(178.18, 25.90) --
	(178.18, 25.90) --
	(178.18, 25.90) --
	(178.18, 25.90) --
	(178.11, 25.90) --
	(178.11, 25.90) --
	(178.11, 25.90) --
	(178.04, 25.90) --
	(178.04, 25.90) --
	(178.04, 25.90) --
	(177.96, 25.90) --
	(177.96, 25.90) --
	(177.96, 25.90) --
	(177.89, 25.90) --
	(177.89, 25.90) --
	(177.89, 25.90) --
	(177.89, 25.90) --
	(177.89, 25.90) --
	(177.89, 25.90) --
	(177.81, 25.90) --
	(177.81, 25.90) --
	(177.81, 25.90) --
	(177.79, 25.90) --
	(177.79, 25.90) --
	(177.79, 25.90) --
	(177.74, 25.90) --
	(177.74, 25.90) --
	(177.74, 25.90) --
	(177.67, 25.90) --
	(177.67, 25.90) --
	(177.67, 25.90) --
	(177.60, 25.90) --
	(177.60, 25.90) --
	(177.60, 25.90) --
	(177.59, 25.90) --
	(177.59, 25.90) --
	(177.59, 25.90) --
	(177.52, 25.90) --
	(177.52, 25.90) --
	(177.52, 25.90) --
	(177.51, 25.90) --
	(177.51, 25.90) --
	(177.51, 25.90) --
	(177.44, 25.90) --
	(177.44, 25.90) --
	(177.44, 25.90) --
	(177.37, 25.90) --
	(177.37, 25.90) --
	(177.37, 25.90) --
	(177.31, 25.90) --
	(177.31, 25.90) --
	(177.31, 25.90) --
	(177.30, 25.90) --
	(177.30, 25.90) --
	(177.30, 25.90) --
	(177.22, 25.90) --
	(177.22, 25.90) --
	(177.22, 25.90) --
	(177.15, 25.90) --
	(177.15, 25.90) --
	(177.15, 25.90) --
	(177.12, 25.90) --
	(177.12, 25.90) --
	(177.12, 25.90) --
	(177.07, 25.90) --
	(177.07, 25.90) --
	(177.07, 25.90) --
	(177.03, 25.90) --
	(177.03, 25.90) --
	(177.03, 25.90) --
	(177.00, 25.90) --
	(177.00, 25.90) --
	(177.00, 25.90) --
	(176.93, 25.90) --
	(176.93, 25.90) --
	(176.93, 25.90) --
	(176.85, 25.90) --
	(176.85, 25.90) --
	(176.85, 25.90) --
	(176.83, 25.90) --
	(176.83, 25.90) --
	(176.83, 25.90) --
	(176.78, 25.90) --
	(176.78, 25.90) --
	(176.78, 25.90) --
	(176.74, 25.90) --
	(176.74, 25.90) --
	(176.74, 25.90) --
	(176.70, 25.90) --
	(176.70, 25.90) --
	(176.70, 25.90) --
	(176.64, 25.90) --
	(176.64, 25.90) --
	(176.64, 25.90) --
	(176.63, 25.90) --
	(176.63, 25.90) --
	(176.63, 25.90) --
	(176.56, 25.90) --
	(176.56, 25.90) --
	(176.56, 25.90) --
	(176.48, 25.90) --
	(176.48, 25.90) --
	(176.48, 25.90) --
	(176.45, 25.90) --
	(176.45, 25.90) --
	(176.45, 25.90) --
	(176.41, 25.90) --
	(176.41, 25.90) --
	(176.41, 25.90) --
	(176.36, 25.90) --
	(176.36, 25.90) --
	(176.36, 25.90) --
	(176.33, 25.90) --
	(176.33, 25.90) --
	(176.33, 25.90) --
	(176.26, 25.90) --
	(176.26, 25.90) --
	(176.26, 25.90) --
	(176.19, 25.90) --
	(176.19, 25.90) --
	(176.19, 25.90) --
	(176.16, 25.90) --
	(176.16, 25.90) --
	(176.16, 25.90) --
	(176.11, 25.90) --
	(176.11, 25.90) --
	(176.11, 25.90) --
	(176.07, 25.90) --
	(176.07, 25.90) --
	(176.07, 25.90) --
	(176.04, 25.90) --
	(176.04, 25.90) --
	(176.04, 25.90) --
	(175.97, 25.90) --
	(175.97, 25.90) --
	(175.97, 25.90) --
	(175.96, 25.90) --
	(175.96, 25.90) --
	(175.96, 25.90) --
	(175.89, 25.90) --
	(175.89, 25.90) --
	(175.89, 25.90) --
	(175.82, 25.90) --
	(175.82, 25.90) --
	(175.82, 25.90) --
	(175.78, 25.90) --
	(175.78, 25.90) --
	(175.78, 25.90) --
	(175.74, 25.90) --
	(175.74, 25.90) --
	(175.74, 25.90) --
	(175.68, 25.90) --
	(175.68, 25.90) --
	(175.68, 25.90) --
	(175.67, 25.90) --
	(175.67, 25.90) --
	(175.67, 25.90) --
	(175.59, 25.90) --
	(175.59, 25.90) --
	(175.59, 25.90) --
	(175.59, 25.90) --
	(175.59, 25.90) --
	(175.59, 25.90) --
	(175.52, 25.90) --
	(175.52, 25.90) --
	(175.52, 25.90) --
	(175.45, 25.90) --
	(175.45, 25.90) --
	(175.45, 25.90) --
	(175.37, 25.90) --
	(175.37, 25.90) --
	(175.37, 25.90) --
	(175.30, 25.90) --
	(175.30, 25.90) --
	(175.30, 25.90) --
	(175.30, 25.90) --
	(175.30, 25.90) --
	(175.30, 25.90) --
	(175.22, 25.90) --
	(175.22, 25.90) --
	(175.22, 25.90) --
	(175.15, 25.90) --
	(175.15, 25.90) --
	(175.15, 25.90) --
	(175.07, 25.90) --
	(175.07, 25.90) --
	(175.07, 25.90) --
	(175.01, 25.90) --
	(175.01, 25.90) --
	(175.01, 25.90) --
	(175.00, 25.90) --
	(175.00, 25.90) --
	(175.00, 25.90) --
	(174.93, 25.90) --
	(174.93, 25.90) --
	(174.93, 25.90) --
	(174.92, 25.90) --
	(174.92, 25.90) --
	(174.92, 25.90) --
	(174.85, 25.90) --
	(174.85, 25.90) --
	(174.85, 25.90) --
	(174.82, 25.90) --
	(174.82, 25.90) --
	(174.82, 25.90) --
	(174.78, 25.90) --
	(174.78, 25.90) --
	(174.78, 25.90) --
	(174.70, 25.90) --
	(174.70, 25.90) --
	(174.70, 25.90) --
	(174.63, 25.90) --
	(174.63, 25.90) --
	(174.63, 25.90) --
	(174.63, 25.90) --
	(174.63, 25.90) --
	(174.63, 25.90) --
	(174.56, 25.90) --
	(174.56, 25.90) --
	(174.56, 25.90) --
	(174.53, 25.90) --
	(174.53, 25.90) --
	(174.53, 25.90) --
	(174.48, 25.90) --
	(174.48, 25.90) --
	(174.48, 25.90) --
	(174.44, 25.90) --
	(174.44, 25.90) --
	(174.44, 25.90) --
	(174.41, 25.90) --
	(174.41, 25.90) --
	(174.41, 25.90) --
	(174.34, 25.90) --
	(174.34, 25.90) --
	(174.34, 25.90) --
	(174.33, 25.90) --
	(174.33, 25.90) --
	(174.33, 25.90) --
	(174.26, 25.90) --
	(174.26, 25.90) --
	(174.26, 25.90) --
	(174.25, 25.90) --
	(174.25, 25.90) --
	(174.25, 25.90) --
	(174.18, 25.90) --
	(174.18, 25.90) --
	(174.18, 25.90) --
	(174.15, 25.90) --
	(174.15, 25.90) --
	(174.15, 25.90) --
	(174.11, 25.90) --
	(174.11, 25.90) --
	(174.11, 25.90) --
	(174.04, 25.90) --
	(174.04, 25.90) --
	(174.04, 25.90) --
	(173.96, 25.90) --
	(173.96, 25.90) --
	(173.96, 25.90) --
	(173.89, 25.90) --
	(173.89, 25.90) --
	(173.89, 25.90) --
	(173.86, 25.90) --
	(173.86, 25.90) --
	(173.86, 25.90) --
	(173.81, 25.90) --
	(173.81, 25.90) --
	(173.81, 25.90) --
	(173.77, 25.90) --
	(173.77, 25.90) --
	(173.77, 25.90) --
	(173.74, 25.90) --
	(173.74, 25.90) --
	(173.74, 25.90) --
	(173.67, 25.90) --
	(173.67, 25.90) --
	(173.67, 25.90) --
	(173.59, 25.90) --
	(173.59, 25.90) --
	(173.59, 25.90) --
	(173.58, 25.90) --
	(173.58, 25.90) --
	(173.58, 25.90) --
	(173.52, 25.90) --
	(173.52, 25.90) --
	(173.52, 25.90) --
	(173.48, 25.90) --
	(173.48, 25.90) --
	(173.48, 25.90) --
	(173.44, 25.90) --
	(173.44, 25.90) --
	(173.44, 25.90) --
	(173.37, 25.90) --
	(173.37, 25.90) --
	(173.37, 25.90) --
	(173.29, 25.90) --
	(173.29, 25.90) --
	(173.29, 25.90) --
	(173.29, 25.90) --
	(173.29, 25.90) --
	(173.29, 25.90) --
	(173.22, 25.90) --
	(173.22, 25.90) --
	(173.22, 25.90) --
	(173.15, 25.90) --
	(173.15, 25.90) --
	(173.15, 25.90) --
	(173.07, 25.90) --
	(173.07, 25.90) --
	(173.07, 25.90) --
	(173.00, 25.90) --
	(173.00, 25.90) --
	(173.00, 25.90) --
	(173.00, 25.90) --
	(173.00, 25.90) --
	(173.00, 25.90) --
	(172.92, 25.90) --
	(172.92, 25.90) --
	(172.92, 25.90) --
	(172.85, 25.90) --
	(172.85, 25.90) --
	(172.85, 25.90) --
	(172.81, 25.90) --
	(172.81, 25.90) --
	(172.81, 25.90) --
	(172.78, 25.90) --
	(172.78, 25.90) --
	(172.78, 25.90) --
	(172.70, 25.90) --
	(172.70, 25.90) --
	(172.70, 25.90) --
	(172.63, 25.90) --
	(172.63, 25.90) --
	(172.63, 25.90) --
	(172.55, 25.90) --
	(172.55, 25.90) --
	(172.55, 25.90) --
	(172.48, 25.90) --
	(172.48, 25.90) --
	(172.48, 25.90) --
	(172.43, 25.90) --
	(172.43, 25.90) --
	(172.43, 25.90) --
	(172.40, 25.90) --
	(172.40, 25.90) --
	(172.40, 25.90) --
	(172.33, 25.90) --
	(172.33, 25.90) --
	(172.33, 25.90) --
	(172.33, 25.90) --
	(172.33, 25.90) --
	(172.33, 25.90) --
	(172.26, 25.90) --
	(172.26, 25.90) --
	(172.26, 25.90) --
	(172.18, 25.90) --
	(172.18, 25.90) --
	(172.18, 25.90) --
	(172.14, 25.90) --
	(172.14, 25.90) --
	(172.14, 25.90) --
	(172.11, 25.90) --
	(172.11, 25.90) --
	(172.11, 25.90) --
	(172.03, 25.90) --
	(172.03, 25.90) --
	(172.03, 25.90) --
	(171.96, 25.90) --
	(171.96, 25.90) --
	(171.96, 25.90) --
	(171.88, 25.90) --
	(171.88, 25.90) --
	(171.88, 25.90) --
	(171.85, 25.90) --
	(171.85, 25.90) --
	(171.85, 25.90) --
	(171.81, 25.90) --
	(171.81, 25.90) --
	(171.81, 25.90) --
	(171.74, 25.90) --
	(171.74, 25.90) --
	(171.74, 25.90) --
	(171.66, 25.90) --
	(171.66, 25.90) --
	(171.66, 25.90) --
	(171.66, 25.90) --
	(171.66, 25.90) --
	(171.66, 25.90) --
	(171.59, 25.90) --
	(171.59, 25.90) --
	(171.59, 25.90) --
	(171.51, 25.90) --
	(171.51, 25.90) --
	(171.51, 25.90) --
	(171.44, 25.90) --
	(171.44, 25.90) --
	(171.44, 25.90) --
	(171.37, 25.90) --
	(171.37, 25.90) --
	(171.37, 25.90) --
	(171.36, 25.90) --
	(171.36, 25.90) --
	(171.36, 25.90) --
	(171.29, 25.90) --
	(171.29, 25.90) --
	(171.29, 25.90) --
	(171.22, 25.90) --
	(171.22, 25.90) --
	(171.22, 25.90) --
	(171.14, 25.90) --
	(171.14, 25.90) --
	(171.14, 25.90) --
	(171.07, 25.90) --
	(171.07, 25.90) --
	(171.07, 25.90) --
	(171.05, 25.90) --
	(171.05, 25.90) --
	(171.05, 25.90) --
	(170.99, 25.90) --
	(170.99, 25.90) --
	(170.99, 25.90) --
	(170.92, 25.90) --
	(170.92, 25.90) --
	(170.92, 25.90) --
	(170.89, 25.90) --
	(170.89, 25.90) --
	(170.89, 25.90) --
	(170.84, 25.90) --
	(170.84, 25.90) --
	(170.84, 25.90) --
	(170.77, 25.90) --
	(170.77, 25.90) --
	(170.77, 25.90) --
	(170.70, 25.90) --
	(170.70, 25.90) --
	(170.70, 25.90) --
	(170.62, 25.90) --
	(170.62, 25.90) --
	(170.62, 25.90) --
	(170.55, 25.90) --
	(170.55, 25.90) --
	(170.55, 25.90) --
	(170.51, 25.90) --
	(170.51, 25.90) --
	(170.51, 25.90) --
	(170.47, 25.90) --
	(170.47, 25.90) --
	(170.47, 25.90) --
	(170.40, 25.90) --
	(170.40, 25.90) --
	(170.40, 25.90) --
	(170.32, 25.90) --
	(170.32, 25.90) --
	(170.32, 25.90) --
	(170.25, 25.90) --
	(170.25, 25.90) --
	(170.25, 25.90) --
	(170.17, 25.90) --
	(170.17, 25.90) --
	(170.17, 25.90) --
	(170.13, 25.90) --
	(170.13, 25.90) --
	(170.13, 25.90) --
	(170.10, 25.90) --
	(170.10, 25.90) --
	(170.10, 25.90) --
	(170.03, 25.90) --
	(170.03, 25.90) --
	(170.03, 25.90) --
	(169.95, 25.90) --
	(169.95, 25.90) --
	(169.95, 25.90) --
	(169.88, 25.90) --
	(169.88, 25.90) --
	(169.88, 25.90) --
	(169.80, 25.90) --
	(169.80, 25.90) --
	(169.80, 25.90) --
	(169.76, 25.90) --
	(169.76, 25.90) --
	(169.76, 25.90) --
	(169.73, 25.90) --
	(169.73, 25.90) --
	(169.73, 25.90) --
	(169.65, 25.90) --
	(169.65, 25.90) --
	(169.65, 25.90) --
	(169.58, 25.90) --
	(169.58, 25.90) --
	(169.58, 25.90) --
	(169.51, 25.90) --
	(169.51, 25.90) --
	(169.51, 25.90) --
	(169.43, 25.90) --
	(169.43, 25.90) --
	(169.43, 25.90) --
	(169.36, 25.90) --
	(169.36, 25.90) --
	(169.36, 25.90) --
	(169.36, 25.90) --
	(169.36, 25.90) --
	(169.36, 25.90) --
	(169.28, 25.90) --
	(169.28, 25.90) --
	(169.28, 25.90) --
	(169.21, 25.90) --
	(169.21, 25.90) --
	(169.21, 25.90) --
	(169.19, 25.90) --
	(169.19, 25.90) --
	(169.19, 25.90) --
	(169.13, 25.90) --
	(169.13, 25.90) --
	(169.13, 25.90) --
	(169.06, 25.90) --
	(169.06, 25.90) --
	(169.06, 25.90) --
	(168.98, 25.90) --
	(168.98, 25.90) --
	(168.98, 25.90) --
	(168.91, 25.90) --
	(168.91, 25.90) --
	(168.91, 25.90) --
	(168.87, 25.90) --
	(168.87, 25.90) --
	(168.87, 25.90) --
	(168.83, 25.90) --
	(168.83, 25.90) --
	(168.83, 25.90) --
	(168.76, 25.90) --
	(168.76, 25.90) --
	(168.76, 25.90) --
	(168.75, 25.90) --
	(168.75, 25.90) --
	(168.75, 25.90) --
	(168.71, 25.90) --
	(168.71, 25.90) --
	(168.71, 25.90) --
	(168.69, 25.90) --
	(168.69, 25.90) --
	(168.69, 25.90) --
	(168.67, 25.90) --
	(168.67, 25.90) --
	(168.67, 25.90) --
	(168.63, 25.90) --
	(168.63, 25.90) --
	(168.63, 25.90) --
	(168.61, 25.90) --
	(168.61, 25.90) --
	(168.61, 25.90) --
	(168.55, 25.90) --
	(168.55, 25.90) --
	(168.55, 25.90) --
	(168.54, 25.90) --
	(168.54, 25.90) --
	(168.54, 25.90) --
	(168.51, 25.90) --
	(168.51, 25.90) --
	(168.51, 25.90) --
	(168.46, 25.90) --
	(168.46, 25.90) --
	(168.46, 25.90) --
	(168.39, 25.90) --
	(168.39, 25.90) --
	(168.39, 25.90) --
	(168.34, 25.90) --
	(168.34, 25.90) --
	(168.34, 25.90) --
	(168.31, 25.90) --
	(168.31, 25.90) --
	(168.31, 25.90) --
	(168.30, 25.90) --
	(168.30, 25.90) --
	(168.30, 25.90) --
	(168.24, 25.90) --
	(168.24, 25.90) --
	(168.24, 25.90) --
	(168.16, 25.90) --
	(168.16, 25.90) --
	(168.16, 25.90) --
	(168.09, 25.90) --
	(168.09, 25.90) --
	(168.09, 25.90) --
	(168.01, 25.90) --
	(168.01, 25.90) --
	(168.01, 25.90) --
	(167.98, 25.90) --
	(167.98, 25.90) --
	(167.98, 25.90) --
	(167.94, 25.90) --
	(167.94, 25.90) --
	(167.94, 25.90) --
	(167.87, 25.90) --
	(167.87, 25.90) --
	(167.87, 25.90) --
	(167.83, 25.90) --
	(167.83, 25.90) --
	(167.83, 25.90) --
	(167.79, 25.90) --
	(167.79, 25.90) --
	(167.79, 25.90) --
	(167.72, 25.90) --
	(167.72, 25.90) --
	(167.72, 25.90) --
	(167.66, 25.90) --
	(167.66, 25.90) --
	(167.66, 25.90) --
	(167.64, 25.90) --
	(167.64, 25.90) --
	(167.64, 25.90) --
	(167.57, 25.90) --
	(167.57, 25.90) --
	(167.57, 25.90) --
	(167.49, 25.90) --
	(167.49, 25.90) --
	(167.49, 25.90) --
	(167.42, 25.90) --
	(167.42, 25.90) --
	(167.42, 25.90) --
	(167.34, 25.90) --
	(167.34, 25.90) --
	(167.34, 25.90) --
	(167.27, 25.90) --
	(167.27, 25.90) --
	(167.27, 25.90) --
	(167.20, 25.90) --
	(167.20, 25.90) --
	(167.20, 25.90) --
	(167.17, 25.90) --
	(167.17, 25.90) --
	(167.17, 25.90) --
	(167.12, 25.90) --
	(167.12, 25.90) --
	(167.12, 25.90) --
	(167.05, 25.90) --
	(167.05, 25.90) --
	(167.05, 25.90) --
	(166.97, 25.90) --
	(166.97, 25.90) --
	(166.97, 25.90) --
	(166.90, 25.90) --
	(166.90, 25.90) --
	(166.90, 25.90) --
	(166.82, 25.90) --
	(166.82, 25.90) --
	(166.82, 25.90) --
	(166.81, 25.90) --
	(166.81, 25.90) --
	(166.81, 25.90) --
	(166.75, 25.90) --
	(166.75, 25.90) --
	(166.75, 25.90) --
	(166.67, 25.90) --
	(166.67, 25.90) --
	(166.67, 25.90) --
	(166.60, 25.90) --
	(166.60, 25.90) --
	(166.60, 25.90) --
	(166.52, 25.90) --
	(166.52, 25.90) --
	(166.52, 25.90) --
	(166.48, 25.90) --
	(166.48, 25.90) --
	(166.48, 25.90) --
	(166.45, 25.90) --
	(166.45, 25.90) --
	(166.45, 25.90) --
	(166.44, 25.90) --
	(166.44, 25.90) --
	(166.44, 25.90) --
	(166.38, 25.90) --
	(166.38, 25.90) --
	(166.38, 25.90) --
	(166.30, 25.90) --
	(166.30, 25.90) --
	(166.30, 25.90) --
	(166.23, 25.90) --
	(166.23, 25.90) --
	(166.23, 25.90) --
	(166.16, 25.90) --
	(166.16, 25.90) --
	(166.16, 25.90) --
	(166.15, 25.90) --
	(166.15, 25.90) --
	(166.15, 25.90) --
	(166.08, 25.90) --
	(166.08, 25.90) --
	(166.08, 25.90) --
	(166.00, 25.90) --
	(166.00, 25.90) --
	(166.00, 25.90) --
	(166.00, 25.90) --
	(166.00, 25.90) --
	(166.00, 25.90) --
	(165.93, 25.90) --
	(165.93, 25.90) --
	(165.93, 25.90) --
	(165.85, 25.90) --
	(165.85, 25.90) --
	(165.85, 25.90) --
	(165.78, 25.90) --
	(165.78, 25.90) --
	(165.78, 25.90) --
	(165.76, 25.90) --
	(165.76, 25.90) --
	(165.76, 25.90) --
	(165.70, 25.90) --
	(165.70, 25.90) --
	(165.70, 25.90) --
	(165.63, 25.90) --
	(165.63, 25.90) --
	(165.63, 25.90) --
	(165.60, 25.90) --
	(165.60, 25.90) --
	(165.60, 25.90) --
	(165.55, 25.90) --
	(165.55, 25.90) --
	(165.55, 25.90) --
	(165.48, 25.90) --
	(165.48, 25.90) --
	(165.48, 25.90) --
	(165.41, 25.90) --
	(165.41, 25.90) --
	(165.41, 25.90) --
	(165.33, 25.90) --
	(165.33, 25.90) --
	(165.33, 25.90) --
	(165.27, 25.90) --
	(165.27, 25.90) --
	(165.27, 25.90) --
	(165.25, 25.90) --
	(165.25, 25.90) --
	(165.25, 25.90) --
	(165.18, 25.90) --
	(165.18, 25.90) --
	(165.18, 25.90) --
	(165.11, 25.90) --
	(165.11, 25.90) --
	(165.11, 25.90) --
	(165.07, 25.90) --
	(165.07, 25.90) --
	(165.07, 25.90) --
	(165.03, 25.90) --
	(165.03, 25.90) --
	(165.03, 25.90) --
	(164.99, 25.90) --
	(164.99, 25.90) --
	(164.99, 25.90) --
	(164.96, 25.90) --
	(164.96, 25.90) --
	(164.96, 25.90) --
	(164.95, 25.90) --
	(164.95, 25.90) --
	(164.95, 25.90) --
	(164.88, 25.90) --
	(164.88, 25.90) --
	(164.88, 25.90) --
	(164.81, 25.90) --
	(164.81, 25.90) --
	(164.81, 25.90) --
	(164.73, 25.90) --
	(164.73, 25.90) --
	(164.73, 25.90) --
	(164.71, 25.90) --
	(164.71, 25.90) --
	(164.71, 25.90) --
	(164.67, 25.90) --
	(164.67, 25.90) --
	(164.67, 25.90) --
	(164.66, 25.90) --
	(164.66, 25.90) --
	(164.66, 25.90) --
	(164.66, 25.90) --
	(164.66, 25.90) --
	(164.66, 25.90) --
	(164.58, 25.90) --
	(164.58, 25.90) --
	(164.58, 25.90) --
	(164.55, 25.90) --
	(164.55, 25.90) --
	(164.55, 25.90) --
	(164.51, 25.90) --
	(164.51, 25.90) --
	(164.51, 25.90) --
	(164.43, 25.90) --
	(164.43, 25.90) --
	(164.43, 25.90) --
	(164.42, 25.90) --
	(164.42, 25.90) --
	(164.42, 25.90) --
	(164.36, 25.90) --
	(164.36, 25.90) --
	(164.36, 25.90) --
	(164.34, 25.90) --
	(164.34, 25.90) --
	(164.34, 25.90) --
	(164.28, 25.90) --
	(164.28, 25.90) --
	(164.28, 25.90) --
	(164.22, 25.90) --
	(164.22, 25.90) --
	(164.22, 25.90) --
	(164.21, 25.90) --
	(164.21, 25.90) --
	(164.21, 25.90) --
	(164.18, 25.90) --
	(164.18, 25.90) --
	(164.18, 25.90) --
	(164.14, 25.90) --
	(164.14, 25.90) --
	(164.14, 25.90) --
	(164.06, 25.90) --
	(164.06, 25.90) --
	(164.06, 25.90) --
	(164.02, 25.90) --
	(164.02, 25.90) --
	(164.02, 25.90) --
	(163.99, 25.90) --
	(163.99, 25.90) --
	(163.99, 25.90) --
	(163.91, 25.90) --
	(163.91, 25.90) --
	(163.91, 25.90) --
	(163.90, 25.90) --
	(163.90, 25.90) --
	(163.90, 25.90) --
	(163.84, 25.90) --
	(163.84, 25.90) --
	(163.84, 25.90) --
	(163.82, 25.90) --
	(163.82, 25.90) --
	(163.82, 25.90) --
	(163.76, 25.90) --
	(163.76, 25.90) --
	(163.76, 25.90) --
	(163.71, 25.90) --
	(163.71, 25.90) --
	(163.71, 25.90) --
	(163.70, 25.90) --
	(163.70, 25.90) --
	(163.70, 25.90) --
	(163.69, 25.90) --
	(163.69, 25.90) --
	(163.69, 25.90) --
	(163.66, 25.90) --
	(163.66, 25.90) --
	(163.66, 25.90) --
	(163.61, 25.90) --
	(163.61, 25.90) --
	(163.61, 25.90) --
	(163.57, 25.90) --
	(163.57, 25.90) --
	(163.57, 25.90) --
	(163.54, 25.90) --
	(163.54, 25.90) --
	(163.54, 25.90) --
	(163.53, 25.90) --
	(163.53, 25.90) --
	(163.53, 25.90) --
	(163.46, 25.90) --
	(163.46, 25.90) --
	(163.46, 25.90) --
	(163.45, 25.90) --
	(163.45, 25.90) --
	(163.45, 25.90) --
	(163.39, 25.90) --
	(163.39, 25.90) --
	(163.39, 25.90) --
	(163.37, 25.90) --
	(163.37, 25.90) --
	(163.37, 25.90) --
	(163.31, 25.90) --
	(163.31, 25.90) --
	(163.31, 25.90) --
	(163.25, 25.90) --
	(163.25, 25.90) --
	(163.25, 25.90) --
	(163.24, 25.90) --
	(163.24, 25.90) --
	(163.24, 25.90) --
	(163.16, 25.90) --
	(163.16, 25.90) --
	(163.16, 25.90) --
	(163.09, 25.90) --
	(163.09, 25.90) --
	(163.09, 25.90) --
	(163.09, 25.90) --
	(163.09, 25.90) --
	(163.09, 25.90) --
	(163.05, 25.90) --
	(163.05, 25.90) --
	(163.05, 25.90) --
	(163.01, 25.90) --
	(163.01, 25.90) --
	(163.01, 25.90) --
	(162.94, 25.90) --
	(162.94, 25.90) --
	(162.94, 25.90) --
	(162.93, 25.90) --
	(162.93, 25.90) --
	(162.93, 25.90) --
	(162.89, 25.90) --
	(162.89, 25.90) --
	(162.89, 25.90) --
	(162.87, 25.90) --
	(162.87, 25.90) --
	(162.87, 25.90) --
	(162.79, 25.90) --
	(162.79, 25.90) --
	(162.79, 25.90) --
	(162.73, 25.90) --
	(162.73, 25.90) --
	(162.73, 25.90) --
	(162.71, 25.90) --
	(162.71, 25.90) --
	(162.71, 25.90) --
	(162.69, 25.90) --
	(162.69, 25.90) --
	(162.69, 25.90) --
	(162.64, 25.90) --
	(162.64, 25.90) --
	(162.64, 25.90) --
	(162.56, 25.90) --
	(162.56, 25.90) --
	(162.56, 25.90) --
	(162.49, 25.90) --
	(162.49, 25.90) --
	(162.49, 25.90) --
	(162.48, 25.90) --
	(162.48, 25.90) --
	(162.48, 25.90) --
	(162.42, 25.90) --
	(162.42, 25.90) --
	(162.42, 25.90) --
	(162.34, 25.90) --
	(162.34, 25.90) --
	(162.34, 25.90) --
	(162.28, 25.90) --
	(162.28, 25.90) --
	(162.28, 25.90) --
	(162.27, 25.90) --
	(162.27, 25.90) --
	(162.27, 25.90) --
	(162.19, 25.90) --
	(162.19, 25.90) --
	(162.19, 25.90) --
	(162.12, 25.90) --
	(162.12, 25.90) --
	(162.12, 25.90) --
	(162.12, 25.90) --
	(162.12, 25.90) --
	(162.12, 25.90) --
	(162.04, 25.90) --
	(162.04, 25.90) --
	(162.04, 25.90) --
	(162.04, 25.90) --
	(162.04, 25.90) --
	(162.04, 25.90) --
	(162.00, 25.90) --
	(162.00, 25.90) --
	(162.00, 25.90) --
	(161.97, 25.90) --
	(161.97, 25.90) --
	(161.97, 25.90) --
	(161.96, 25.90) --
	(161.96, 25.90) --
	(161.96, 25.90) --
	(161.92, 25.90) --
	(161.92, 25.90) --
	(161.92, 25.90) --
	(161.89, 25.90) --
	(161.89, 25.90) --
	(161.89, 25.90) --
	(161.84, 25.90) --
	(161.84, 25.90) --
	(161.84, 25.90) --
	(161.82, 25.90) --
	(161.82, 25.90) --
	(161.82, 25.90) --
	(161.80, 25.90) --
	(161.80, 25.90) --
	(161.80, 25.90) --
	(161.74, 25.90) --
	(161.74, 25.90) --
	(161.74, 25.90) --
	(161.67, 25.90) --
	(161.67, 25.90) --
	(161.67, 25.90) --
	(161.59, 25.90) --
	(161.59, 25.90) --
	(161.59, 25.90) --
	(161.55, 25.90) --
	(161.55, 25.90) --
	(161.55, 25.90) --
	(161.52, 25.90) --
	(161.52, 25.90) --
	(161.52, 25.90) --
	(161.51, 25.90) --
	(161.51, 25.90) --
	(161.51, 25.90) --
	(161.44, 25.90) --
	(161.44, 25.90) --
	(161.44, 25.90) --
	(161.43, 25.90) --
	(161.43, 25.90) --
	(161.43, 25.90) --
	(161.39, 25.90) --
	(161.39, 25.90) --
	(161.39, 25.90) --
	(161.37, 25.90) --
	(161.37, 25.90) --
	(161.37, 25.90) --
	(161.31, 25.90) --
	(161.31, 25.90) --
	(161.31, 25.90) --
	(161.31, 25.90) --
	(161.31, 25.90) --
	(161.31, 25.90) --
	(161.29, 25.90) --
	(161.29, 25.90) --
	(161.29, 25.90) --
	(161.27, 25.90) --
	(161.27, 25.90) --
	(161.27, 25.90) --
	(161.23, 25.90) --
	(161.23, 25.90) --
	(161.23, 25.90) --
	(161.22, 25.90) --
	(161.22, 25.90) --
	(161.22, 25.90) --
	(161.14, 25.90) --
	(161.14, 25.90) --
	(161.14, 25.90) --
	(161.11, 25.90) --
	(161.11, 25.90) --
	(161.11, 25.90) --
	(161.07, 25.90) --
	(161.07, 25.90) --
	(161.07, 25.90) --
	(161.07, 25.90) --
	(161.07, 25.90) --
	(161.07, 25.90) --
	(160.99, 25.90) --
	(160.99, 25.90) --
	(160.99, 25.90) --
	(160.95, 25.90) --
	(160.95, 25.90) --
	(160.95, 25.90) --
	(160.92, 25.90) --
	(160.92, 25.90) --
	(160.92, 25.90) --
	(160.84, 25.90) --
	(160.84, 25.90) --
	(160.84, 25.90) --
	(160.79, 25.90) --
	(160.79, 25.90) --
	(160.79, 25.90) --
	(160.77, 25.90) --
	(160.77, 25.90) --
	(160.77, 25.90) --
	(160.71, 25.90) --
	(160.71, 25.90) --
	(160.71, 25.90) --
	(160.69, 25.90) --
	(160.69, 25.90) --
	(160.69, 25.90) --
	(160.62, 25.90) --
	(160.62, 25.90) --
	(160.62, 25.90) --
	(160.58, 25.90) --
	(160.58, 25.90) --
	(160.58, 25.90) --
	(160.54, 25.90) --
	(160.54, 25.90) --
	(160.54, 25.90) --
	(160.54, 25.90) --
	(160.54, 25.90) --
	(160.54, 25.90) --
	(160.47, 25.90) --
	(160.47, 25.90) --
	(160.47, 25.90) --
	(160.42, 25.90) --
	(160.42, 25.90) --
	(160.42, 25.90) --
	(160.39, 25.90) --
	(160.39, 25.90) --
	(160.39, 25.90) --
	(160.32, 25.90) --
	(160.32, 25.90) --
	(160.32, 25.90) --
	(160.30, 25.90) --
	(160.30, 25.90) --
	(160.30, 25.90) --
	(160.26, 25.90) --
	(160.26, 25.90) --
	(160.26, 25.90) --
	(160.24, 25.90) --
	(160.24, 25.90) --
	(160.24, 25.90) --
	(160.17, 25.90) --
	(160.17, 25.90) --
	(160.17, 25.90) --
	(160.14, 25.90) --
	(160.14, 25.90) --
	(160.14, 25.90) --
	(160.09, 25.90) --
	(160.09, 25.90) --
	(160.09, 25.90) --
	(160.02, 25.90) --
	(160.02, 25.90) --
	(160.02, 25.90) --
	(159.98, 25.90) --
	(159.98, 25.90) --
	(159.98, 25.90) --
	(159.94, 25.90) --
	(159.94, 25.90) --
	(159.94, 25.90) --
	(159.87, 25.90) --
	(159.87, 25.90) --
	(159.87, 25.90) --
	(159.80, 25.90) --
	(159.80, 25.90) --
	(159.80, 25.90) --
	(159.78, 25.90) --
	(159.78, 25.90) --
	(159.78, 25.90) --
	(159.72, 25.90) --
	(159.72, 25.90) --
	(159.72, 25.90) --
	(159.65, 25.90) --
	(159.65, 25.90) --
	(159.65, 25.90) --
	(159.57, 25.90) --
	(159.57, 25.90) --
	(159.57, 25.90) --
	(159.57, 25.90) --
	(159.57, 25.90) --
	(159.57, 25.90) --
	(159.49, 25.90) --
	(159.49, 25.90) --
	(159.49, 25.90) --
	(159.42, 25.90) --
	(159.42, 25.90) --
	(159.42, 25.90) --
	(159.34, 25.90) --
	(159.34, 25.90) --
	(159.34, 25.90) --
	(159.27, 25.90) --
	(159.27, 25.90) --
	(159.27, 25.90) --
	(159.25, 25.90) --
	(159.25, 25.90) --
	(159.25, 25.90) --
	(159.19, 25.90) --
	(159.19, 25.90) --
	(159.19, 25.90) --
	(159.12, 25.90) --
	(159.12, 25.90) --
	(159.12, 25.90) --
	(159.04, 25.90) --
	(159.04, 25.90) --
	(159.04, 25.90) --
	(158.97, 25.90) --
	(158.97, 25.90) --
	(158.97, 25.90) --
	(158.89, 25.90) --
	(158.89, 25.90) --
	(158.89, 25.90) --
	(158.82, 25.90) --
	(158.82, 25.90) --
	(158.82, 25.90) --
	(158.76, 25.90) --
	(158.76, 25.90) --
	(158.76, 25.90) --
	(158.75, 25.90) --
	(158.75, 25.90) --
	(158.75, 25.90) --
	(158.67, 25.90) --
	(158.67, 25.90) --
	(158.67, 25.90) --
	(158.60, 25.90) --
	(158.60, 25.90) --
	(158.60, 25.90) --
	(158.52, 25.90) --
	(158.52, 25.90) --
	(158.52, 25.90) --
	(158.45, 25.90) --
	(158.45, 25.90) --
	(158.45, 25.90) --
	(158.37, 25.90) --
	(158.37, 25.90) --
	(158.37, 25.90) --
	(158.30, 25.90) --
	(158.30, 25.90) --
	(158.30, 25.90) --
	(158.24, 25.90) --
	(158.24, 25.90) --
	(158.24, 25.90) --
	(158.22, 25.90) --
	(158.22, 25.90) --
	(158.22, 25.90) --
	(158.15, 25.90) --
	(158.15, 25.90) --
	(158.14, 25.90) --
	(158.07, 25.90) --
	(158.07, 25.90) --
	(158.07, 25.90) --
	(158.04, 25.90) --
	(158.04, 25.90) --
	(158.04, 25.90) --
	(157.99, 25.90) --
	(157.99, 25.90) --
	(157.99, 25.90) --
	(157.92, 25.90) --
	(157.92, 25.90) --
	(157.92, 25.90) --
	(157.84, 25.90) --
	(157.84, 25.90) --
	(157.84, 25.90) --
	(157.77, 25.90) --
	(157.77, 25.90) --
	(157.77, 25.90) --
	(157.69, 25.90) --
	(157.69, 25.90) --
	(157.69, 25.90) --
	(157.62, 25.90) --
	(157.62, 25.90) --
	(157.62, 25.90) --
	(157.54, 25.90) --
	(157.54, 25.90) --
	(157.54, 25.90) --
	(157.47, 25.90) --
	(157.47, 25.90) --
	(157.47, 25.90) --
	(157.39, 25.90) --
	(157.39, 25.90) --
	(157.39, 25.90) --
	(157.32, 25.90) --
	(157.32, 25.90) --
	(157.32, 25.90) --
	(157.24, 25.90) --
	(157.24, 25.90) --
	(157.24, 25.90) --
	(157.23, 25.90) --
	(157.23, 25.90) --
	(157.23, 25.90) --
	(157.17, 25.90) --
	(157.17, 25.90) --
	(157.17, 25.90) --
	(157.09, 25.90) --
	(157.09, 25.90) --
	(157.09, 25.90) --
	(157.03, 25.90) --
	(157.03, 25.90) --
	(157.03, 25.90) --
	(157.02, 25.90) --
	(157.02, 25.90) --
	(157.02, 25.90) --
	(156.94, 25.90) --
	(156.94, 25.90) --
	(156.94, 25.90) --
	(156.87, 25.90) --
	(156.87, 25.90) --
	(156.87, 25.90) --
	(156.87, 25.90) --
	(156.87, 25.90) --
	(156.87, 25.90) --
	(156.79, 25.90) --
	(156.79, 25.90) --
	(156.79, 25.90) --
	(156.72, 25.90) --
	(156.72, 25.90) --
	(156.72, 25.90) --
	(156.64, 25.90) --
	(156.64, 25.90) --
	(156.64, 25.90) --
	(156.57, 25.90) --
	(156.57, 25.90) --
	(156.57, 25.90) --
	(156.50, 25.90) --
	(156.50, 25.90) --
	(156.50, 25.90) --
	(156.49, 25.90) --
	(156.49, 25.90) --
	(156.49, 25.90) --
	(156.42, 25.90) --
	(156.42, 25.90) --
	(156.42, 25.90) --
	(156.34, 25.90) --
	(156.34, 25.90) --
	(156.34, 25.90) --
	(156.27, 25.90) --
	(156.27, 25.90) --
	(156.27, 25.90) --
	(156.23, 25.90) --
	(156.23, 25.90) --
	(156.23, 25.90) --
	(156.19, 25.90) --
	(156.19, 25.90) --
	(156.19, 25.90) --
	(156.12, 25.90) --
	(156.12, 25.90) --
	(156.12, 25.90) --
	(156.04, 25.90) --
	(156.04, 25.90) --
	(156.04, 25.90) --
	(155.97, 25.90) --
	(155.97, 25.90) --
	(155.97, 25.90) --
	(155.90, 25.90) --
	(155.90, 25.90) --
	(155.90, 25.90) --
	(155.89, 25.90) --
	(155.89, 25.90) --
	(155.89, 25.90) --
	(155.82, 25.90) --
	(155.82, 25.90) --
	(155.82, 25.90) --
	(155.74, 25.90) --
	(155.74, 25.90) --
	(155.74, 25.90) --
	(155.67, 25.90) --
	(155.67, 25.90) --
	(155.67, 25.90) --
	(155.59, 25.90) --
	(155.59, 25.90) --
	(155.59, 25.90) --
	(155.52, 25.90) --
	(155.52, 25.90) --
	(155.52, 25.90) --
	(155.44, 25.90) --
	(155.44, 25.90) --
	(155.44, 25.90) --
	(155.37, 25.90) --
	(155.37, 25.90) --
	(155.37, 25.90) --
	(155.29, 25.90) --
	(155.29, 25.90) --
	(155.29, 25.90) --
	(155.22, 25.90) --
	(155.22, 25.90) --
	(155.22, 25.90) --
	(155.14, 25.90) --
	(155.14, 25.90) --
	(155.14, 25.90) --
	(155.06, 25.90) --
	(155.06, 25.90) --
	(155.06, 25.90) --
	(154.99, 25.90) --
	(154.99, 25.90) --
	(154.99, 25.90) --
	(154.91, 25.90) --
	(154.91, 25.90) --
	(154.91, 25.90) --
	(154.84, 25.90) --
	(154.84, 25.90) --
	(154.84, 25.90) --
	(154.76, 25.90) --
	(154.76, 25.90) --
	(154.76, 25.90) --
	(154.69, 25.90) --
	(154.69, 25.90) --
	(154.69, 25.90) --
	(154.64, 25.90) --
	(154.64, 25.90) --
	(154.64, 25.90) --
	(154.61, 25.90) --
	(154.61, 25.90) --
	(154.61, 25.90) --
	(154.54, 25.90) --
	(154.54, 25.90) --
	(154.54, 25.90) --
	(154.46, 25.90) --
	(154.46, 25.90) --
	(154.46, 25.90) --
	(154.39, 25.90) --
	(154.39, 25.90) --
	(154.39, 25.90) --
	(154.31, 25.90) --
	(154.31, 25.90) --
	(154.31, 25.90) --
	(154.24, 25.90) --
	(154.24, 25.90) --
	(154.24, 25.90) --
	(154.16, 25.90) --
	(154.16, 25.90) --
	(154.16, 25.90) --
	(154.09, 25.90) --
	(154.09, 25.90) --
	(154.09, 25.90) --
	(154.01, 25.90) --
	(154.01, 25.90) --
	(154.01, 25.90) --
	(153.95, 25.90) --
	(153.95, 25.90) --
	(153.95, 25.90) --
	(153.94, 25.90) --
	(153.94, 25.90) --
	(153.94, 25.90) --
	(153.86, 25.90) --
	(153.86, 25.90) --
	(153.86, 25.90) --
	(153.79, 25.90) --
	(153.79, 25.90) --
	(153.79, 25.90) --
	(153.71, 25.90) --
	(153.71, 25.90) --
	(153.71, 25.90) --
	(153.67, 25.90) --
	(153.67, 25.90) --
	(153.67, 25.90) --
	(153.63, 25.90) --
	(153.63, 25.90) --
	(153.63, 25.90) --
	(153.59, 25.90) --
	(153.59, 25.90) --
	(153.59, 25.90) --
	(153.56, 25.90) --
	(153.56, 25.90) --
	(153.56, 25.90) --
	(153.48, 25.90) --
	(153.48, 25.90) --
	(153.48, 25.90) --
	(153.47, 25.90) --
	(153.47, 25.90) --
	(153.47, 25.90) --
	(153.41, 25.90) --
	(153.41, 25.90) --
	(153.41, 25.90) --
	(153.39, 25.90) --
	(153.39, 25.90) --
	(153.39, 25.90) --
	(153.35, 25.90) --
	(153.35, 25.90) --
	(153.35, 25.90) --
	(153.33, 25.90) --
	(153.33, 25.90) --
	(153.33, 25.90) --
	(153.26, 25.90) --
	(153.26, 25.90) --
	(153.26, 25.90) --
	(153.26, 25.90) --
	(153.26, 25.90) --
	(153.26, 25.90) --
	(153.18, 25.90) --
	(153.18, 25.90) --
	(153.18, 25.90) --
	(153.11, 25.90) --
	(153.11, 25.90) --
	(153.11, 25.90) --
	(153.11, 25.90) --
	(153.11, 25.90) --
	(153.11, 25.90) --
	(153.03, 25.90) --
	(153.03, 25.90) --
	(153.03, 25.90) --
	(152.96, 25.90) --
	(152.96, 25.90) --
	(152.96, 25.90) --
	(152.88, 25.90) --
	(152.88, 25.90) --
	(152.88, 25.90) --
	(152.81, 25.90) --
	(152.81, 25.90) --
	(152.81, 25.90) --
	(152.73, 25.90) --
	(152.73, 25.90) --
	(152.73, 25.90) --
	(152.66, 25.90) --
	(152.66, 25.90) --
	(152.66, 25.90) --
	(152.58, 25.90) --
	(152.58, 25.90) --
	(152.58, 25.90) --
	(152.51, 25.90) --
	(152.51, 25.90) --
	(152.51, 25.90) --
	(152.43, 25.90) --
	(152.43, 25.90) --
	(152.43, 25.90) --
	(152.35, 25.90) --
	(152.35, 25.90) --
	(152.35, 25.90) --
	(152.28, 25.90) --
	(152.28, 25.90) --
	(152.28, 25.90) --
	(152.20, 25.90) --
	(152.20, 25.90) --
	(152.20, 25.90) --
	(152.13, 25.90) --
	(152.13, 25.90) --
	(152.13, 25.90) --
	(152.05, 25.90) --
	(152.05, 25.90) --
	(152.05, 25.90) --
	(151.98, 25.90) --
	(151.98, 25.90) --
	(151.98, 25.90) --
	(151.90, 25.90) --
	(151.90, 25.90) --
	(151.90, 25.90) --
	(151.83, 25.90) --
	(151.83, 25.90) --
	(151.83, 25.90) --
	(151.75, 25.90) --
	(151.75, 25.90) --
	(151.75, 25.90) --
	(151.68, 25.90) --
	(151.68, 25.90) --
	(151.68, 25.90) --
	(151.60, 25.90) --
	(151.60, 25.90) --
	(151.60, 25.90) --
	(151.53, 25.90) --
	(151.53, 25.90) --
	(151.53, 25.90) --
	(151.45, 25.90) --
	(151.45, 25.90) --
	(151.45, 25.90) --
	(151.37, 25.90) --
	(151.37, 25.90) --
	(151.37, 25.90) --
	(151.30, 25.90) --
	(151.30, 25.90) --
	(151.30, 25.90) --
	(151.22, 25.90) --
	(151.22, 25.90) --
	(151.22, 25.90) --
	(151.15, 25.90) --
	(151.15, 25.90) --
	(151.15, 25.90) --
	(151.07, 25.90) --
	(151.07, 25.90) --
	(151.07, 25.90) --
	(151.00, 25.90) --
	(151.00, 25.90) --
	(151.00, 25.90) --
	(150.92, 25.90) --
	(150.92, 25.90) --
	(150.92, 25.90) --
	(150.85, 25.90) --
	(150.85, 25.90) --
	(150.85, 25.90) --
	(150.77, 25.90) --
	(150.77, 25.90) --
	(150.77, 25.90) --
	(150.70, 25.90) --
	(150.70, 25.90) --
	(150.70, 25.90) --
	(150.62, 25.90) --
	(150.62, 25.90) --
	(150.62, 25.90) --
	(150.55, 25.90) --
	(150.55, 25.90) --
	(150.55, 25.90) --
	(150.47, 25.90) --
	(150.47, 25.90) --
	(150.47, 25.90) --
	(150.39, 25.90) --
	(150.39, 25.90) --
	(150.39, 25.90) --
	(150.32, 25.90) --
	(150.32, 25.90) --
	(150.32, 25.90) --
	(150.24, 25.90) --
	(150.24, 25.90) --
	(150.24, 25.90) --
	(150.17, 25.90) --
	(150.17, 25.90) --
	(150.17, 25.90) --
	(150.09, 25.90) --
	(150.09, 25.90) --
	(150.09, 25.90) --
	(150.02, 25.90) --
	(150.02, 25.90) --
	(150.02, 25.90) --
	(149.94, 25.90) --
	(149.94, 25.90) --
	(149.94, 25.90) --
	(149.87, 25.90) --
	(149.87, 25.90) --
	(149.87, 25.90) --
	(149.79, 25.90) --
	(149.79, 25.90) --
	(149.79, 25.90) --
	(149.72, 25.90) --
	(149.72, 25.90) --
	(149.72, 25.90) --
	(149.64, 25.90) --
	(149.64, 25.90) --
	(149.64, 25.90) --
	(149.56, 25.90) --
	(149.56, 25.90) --
	(149.56, 25.90) --
	(149.49, 25.90) --
	(149.49, 25.90) --
	(149.49, 25.90) --
	(149.41, 25.90) --
	(149.41, 25.90) --
	(149.41, 25.90) --
	(149.34, 25.90) --
	(149.34, 25.90) --
	(149.34, 25.90) --
	(149.26, 25.90) --
	(149.26, 25.90) --
	(149.26, 25.90) --
	(149.19, 25.90) --
	(149.19, 25.90) --
	(149.19, 25.90) --
	(149.11, 25.90) --
	(149.11, 25.90) --
	(149.11, 25.90) --
	(149.04, 25.90) --
	(149.04, 25.90) --
	(149.04, 25.90) --
	(148.96, 25.90) --
	(148.96, 25.90) --
	(148.96, 25.90) --
	(148.88, 25.90) --
	(148.88, 25.90) --
	(148.88, 25.90) --
	(148.81, 25.90) --
	(148.81, 25.90) --
	(148.81, 25.90) --
	(148.73, 25.90) --
	(148.73, 25.90) --
	(148.73, 25.90) --
	(148.66, 25.90) --
	(148.66, 25.90) --
	(148.66, 25.90) --
	(148.58, 25.90) --
	(148.58, 25.90) --
	(148.58, 25.90) --
	(148.51, 25.90) --
	(148.51, 25.90) --
	(148.51, 25.90) --
	(148.43, 25.90) --
	(148.43, 25.90) --
	(148.43, 25.90) --
	(148.36, 25.90) --
	(148.36, 25.90) --
	(148.36, 25.90) --
	(148.28, 25.90) --
	(148.28, 25.90) --
	(148.28, 25.90) --
	(148.20, 25.90) --
	(148.20, 25.90) --
	(148.20, 25.90) --
	(148.13, 25.90) --
	(148.13, 25.90) --
	(148.13, 25.90) --
	(148.05, 25.90) --
	(148.05, 25.90) --
	(148.05, 25.90) --
	(147.98, 25.90) --
	(147.98, 25.90) --
	(147.98, 25.90) --
	(147.90, 25.90) --
	(147.90, 25.90) --
	(147.90, 25.90) --
	(147.83, 25.90) --
	(147.83, 25.90) --
	(147.83, 25.90) --
	(147.75, 25.90) --
	(147.75, 25.90) --
	(147.75, 25.90) --
	(147.68, 25.90) --
	(147.68, 25.90) --
	(147.68, 25.90) --
	(147.60, 25.90) --
	(147.60, 25.90) --
	(147.60, 25.90) --
	(147.52, 25.90) --
	(147.52, 25.90) --
	(147.52, 25.90) --
	(147.45, 25.90) --
	(147.45, 25.90) --
	(147.45, 25.90) --
	(147.37, 25.90) --
	(147.37, 25.90) --
	(147.37, 25.90) --
	(147.30, 25.90) --
	(147.30, 25.90) --
	(147.30, 25.90) --
	(147.22, 25.90) --
	(147.22, 25.90) --
	(147.22, 25.90) --
	(147.15, 25.90) --
	(147.15, 25.90) --
	(147.15, 25.90) --
	(147.07, 25.90) --
	(147.07, 25.90) --
	(147.07, 25.90) --
	(146.99, 25.90) --
	(146.99, 25.90) --
	(146.99, 25.90) --
	(146.92, 25.90) --
	(146.92, 25.90) --
	(146.92, 25.90) --
	(146.84, 25.90) --
	(146.84, 25.90) --
	(146.84, 25.90) --
	(146.77, 25.90) --
	(146.77, 25.90) --
	(146.77, 25.90) --
	(146.69, 25.90) --
	(146.69, 25.90) --
	(146.69, 25.90) --
	(146.62, 25.90) --
	(146.62, 25.90) --
	(146.62, 25.90) --
	(146.54, 25.90) --
	(146.54, 25.90) --
	(146.54, 25.90) --
	(146.47, 25.90) --
	(146.47, 25.90) --
	(146.47, 25.90) --
	(146.39, 25.90) --
	(146.39, 25.90) --
	(146.39, 25.90) --
	(146.32, 25.90) --
	(146.32, 25.90) --
	(146.32, 25.90) --
	(146.24, 25.90) --
	(146.24, 25.90) --
	(146.24, 25.90) --
	(146.16, 25.90) --
	(146.16, 25.90) --
	(146.16, 25.90) --
	(146.09, 25.90) --
	(146.09, 25.90) --
	(146.09, 25.90) --
	(146.01, 25.90) --
	(146.01, 25.90) --
	(146.01, 25.90) --
	(145.94, 25.90) --
	(145.94, 25.90) --
	(145.94, 25.90) --
	(145.86, 25.90) --
	(145.86, 25.90) --
	(145.86, 25.90) --
	(145.79, 25.90) --
	(145.79, 25.90) --
	(145.79, 25.90) --
	(145.71, 25.90) --
	(145.71, 25.90) --
	(145.71, 25.90) --
	(145.63, 25.90) --
	(145.63, 25.90) --
	(145.63, 25.90) --
	(145.56, 25.90) --
	(145.56, 25.90) --
	(145.56, 25.90) --
	(145.48, 25.90) --
	(145.48, 25.90) --
	(145.48, 25.90) --
	(145.41, 25.90) --
	(145.41, 25.90) --
	(145.41, 25.90) --
	(145.33, 25.90) --
	(145.33, 25.90) --
	(145.33, 25.90) --
	(145.26, 25.90) --
	(145.26, 25.90) --
	(145.26, 25.90) --
	(145.18, 25.90) --
	(145.18, 25.90) --
	(145.18, 25.90) --
	(145.10, 25.90) --
	(145.10, 25.90) --
	(145.10, 25.90) --
	(145.03, 25.90) --
	(145.03, 25.90) --
	(145.03, 25.90) --
	(144.95, 25.90) --
	(144.95, 25.90) --
	(144.95, 25.90) --
	(144.88, 25.90) --
	(144.88, 25.90) --
	(144.88, 25.90) --
	(144.80, 25.90) --
	(144.80, 25.90) --
	(144.80, 25.90) --
	(144.72, 25.90) --
	(144.72, 25.90) --
	(144.72, 25.90) --
	(144.65, 25.90) --
	(144.65, 25.90) --
	(144.65, 25.90) --
	(144.57, 25.90) --
	(144.57, 25.90) --
	(144.57, 25.90) --
	(144.50, 25.90) --
	(144.50, 25.90) --
	(144.50, 25.90) --
	(144.42, 25.90) --
	(144.42, 25.90) --
	(144.42, 25.90) --
	(144.35, 25.90) --
	(144.35, 25.90) --
	(144.35, 25.90) --
	(144.27, 25.90) --
	(144.27, 25.90) --
	(144.27, 25.90) --
	(144.19, 25.90) --
	(144.19, 25.90) --
	(144.19, 25.90) --
	(144.12, 25.90) --
	(144.12, 25.90) --
	(144.12, 25.90) --
	(144.04, 25.90) --
	(144.04, 25.90) --
	(144.04, 25.90) --
	(143.97, 25.90) --
	(143.97, 25.90) --
	(143.97, 25.90) --
	(143.89, 25.90) --
	(143.89, 25.90) --
	(143.89, 25.90) --
	(143.82, 25.90) --
	(143.82, 25.90) --
	(143.82, 25.90) --
	(143.74, 25.90) --
	(143.74, 25.90) --
	(143.74, 25.90) --
	(143.66, 25.90) --
	(143.66, 25.90) --
	(143.66, 25.90) --
	(143.59, 25.90) --
	(143.59, 25.90) --
	(143.59, 25.90) --
	(143.51, 25.90) --
	(143.51, 25.90) --
	(143.51, 25.90) --
	(143.44, 25.90) --
	(143.44, 25.90) --
	(143.44, 25.90) --
	(143.36, 25.90) --
	(143.36, 25.90) --
	(143.36, 25.90) --
	(143.29, 25.90) --
	(143.29, 25.90) --
	(143.29, 25.90) --
	(143.21, 25.90) --
	(143.21, 25.90) --
	(143.21, 25.90) --
	(143.13, 25.90) --
	(143.13, 25.90) --
	(143.13, 25.90) --
	(143.06, 25.90) --
	(143.06, 25.90) --
	(143.06, 25.90) --
	(142.98, 25.90) --
	(142.98, 25.90) --
	(142.98, 25.90) --
	(142.91, 25.90) --
	(142.91, 25.90) --
	(142.91, 25.90) --
	(142.83, 25.90) --
	(142.83, 25.90) --
	(142.83, 25.90) --
	(142.75, 25.90) --
	(142.75, 25.90) --
	(142.75, 25.90) --
	(142.68, 25.90) --
	(142.68, 25.90) --
	(142.68, 25.90) --
	(142.60, 25.90) --
	(142.60, 25.90) --
	(142.60, 25.90) --
	(142.53, 25.90) --
	(142.53, 25.90) --
	(142.53, 25.90) --
	(142.45, 25.90) --
	(142.45, 25.90) --
	(142.45, 25.90) --
	(142.37, 25.90) --
	(142.37, 25.90) --
	(142.37, 25.90) --
	(142.30, 25.90) --
	(142.30, 25.90) --
	(142.30, 25.90) --
	(142.22, 25.90) --
	(142.22, 25.90) --
	(142.22, 25.90) --
	(142.15, 25.90) --
	(142.15, 25.90) --
	(142.15, 25.90) --
	(142.07, 25.90) --
	(142.07, 25.90) --
	(142.07, 25.90) --
	(142.00, 25.90) --
	(142.00, 25.90) --
	(142.00, 25.90) --
	(141.92, 25.90) --
	(141.92, 25.90) --
	(141.92, 25.90) --
	(141.84, 25.90) --
	(141.84, 25.90) --
	(141.84, 25.90) --
	(141.77, 25.90) --
	(141.77, 25.90) --
	(141.77, 25.90) --
	(141.69, 25.90) --
	(141.69, 25.90) --
	(141.69, 25.90) --
	(141.62, 25.90) --
	(141.62, 25.90) --
	(141.62, 25.90) --
	(141.54, 25.90) --
	(141.54, 25.90) --
	(141.54, 25.90) --
	(141.46, 25.90) --
	(141.46, 25.90) --
	(141.46, 25.90) --
	(141.39, 25.90) --
	(141.39, 25.90) --
	(141.39, 25.90) --
	(141.31, 25.90) --
	(141.31, 25.90) --
	(141.31, 25.90) --
	(141.24, 25.90) --
	(141.24, 25.90) --
	(141.24, 25.90) --
	(141.16, 25.90) --
	(141.16, 25.90) --
	(141.16, 25.90) --
	(141.08, 25.90) --
	(141.08, 25.90) --
	(141.08, 25.90) --
	(141.01, 25.90) --
	(141.01, 25.90) --
	(141.01, 25.90) --
	(140.93, 25.90) --
	(140.93, 25.90) --
	(140.93, 25.90) --
	(140.86, 25.90) --
	(140.86, 25.90) --
	(140.86, 25.90) --
	(140.78, 25.90) --
	(140.78, 25.90) --
	(140.78, 25.90) --
	(140.71, 25.90) --
	(140.71, 25.90) --
	(140.71, 25.90) --
	(140.63, 25.90) --
	(140.63, 25.90) --
	(140.63, 25.90) --
	(140.55, 25.90) --
	(140.55, 25.90) --
	(140.55, 25.90) --
	(140.48, 25.90) --
	(140.48, 25.90) --
	(140.48, 25.90) --
	(140.40, 25.90) --
	(140.40, 25.90) --
	(140.40, 25.90) --
	(140.33, 25.90) --
	(140.33, 25.90) --
	(140.33, 25.90) --
	(140.25, 25.90) --
	(140.25, 25.90) --
	(140.25, 25.90) --
	(140.17, 25.90) --
	(140.17, 25.90) --
	(140.17, 25.90) --
	(140.10, 25.90) --
	(140.10, 25.90) --
	(140.10, 25.90) --
	(140.02, 25.90) --
	(140.02, 25.90) --
	(140.02, 25.90) --
	(139.95, 25.90) --
	(139.95, 25.90) --
	(139.95, 25.90) --
	(139.87, 25.90) --
	(139.87, 25.90) --
	(139.87, 25.90) --
	(139.79, 25.90) --
	(139.79, 25.90) --
	(139.79, 25.90) --
	(139.72, 25.90) --
	(139.72, 25.90) --
	(139.72, 25.90) --
	(139.64, 25.90) --
	(139.64, 25.90) --
	(139.64, 25.90) --
	(139.57, 25.90) --
	(139.57, 25.90) --
	(139.57, 25.90) --
	(139.49, 25.90) --
	(139.49, 25.90) --
	(139.49, 25.90) --
	(139.42, 25.90) --
	(139.42, 25.90) --
	(139.42, 25.90) --
	(139.34, 25.90) --
	(139.34, 25.90) --
	(139.34, 25.90) --
	(139.26, 25.90) --
	(139.26, 25.90) --
	(139.26, 25.90) --
	(139.19, 25.90) --
	(139.19, 25.90) --
	(139.19, 25.90) --
	(139.11, 25.90) --
	(139.11, 25.90) --
	(139.11, 25.90) --
	(139.03, 25.90) --
	(139.03, 25.90) --
	(139.03, 25.90) --
	(138.96, 25.90) --
	(138.96, 25.90) --
	(138.96, 25.90) --
	(138.88, 25.90) --
	(138.88, 25.90) --
	(138.88, 25.90) --
	(138.81, 25.90) --
	(138.81, 25.90) --
	(138.81, 25.90) --
	(138.73, 25.90) --
	(138.73, 25.90) --
	(138.73, 25.90) --
	(138.65, 25.90) --
	(138.65, 25.90) --
	(138.65, 25.90) --
	(138.58, 25.90) --
	(138.58, 25.90) --
	(138.58, 25.90) --
	(138.50, 25.90) --
	(138.50, 25.90) --
	(138.50, 25.90) --
	(138.43, 25.90) --
	(138.43, 25.90) --
	(138.43, 25.90) --
	(138.35, 25.90) --
	(138.35, 25.90) --
	(138.35, 25.90) --
	(138.27, 25.90) --
	(138.27, 25.90) --
	(138.27, 25.90) --
	(138.20, 25.90) --
	(138.20, 25.90) --
	(138.20, 25.90) --
	(138.12, 25.90) --
	(138.12, 25.90) --
	(138.12, 25.90) --
	(138.05, 25.90) --
	(138.05, 25.90) --
	(138.05, 25.90) --
	(137.97, 25.90) --
	(137.97, 25.90) --
	(137.97, 25.90) --
	(137.89, 25.90) --
	(137.89, 25.90) --
	(137.89, 25.90) --
	(137.82, 25.90) --
	(137.82, 25.90) --
	(137.82, 25.90) --
	(137.74, 25.90) --
	(137.74, 25.90) --
	(137.74, 25.90) --
	(137.66, 25.90) --
	(137.66, 25.90) --
	(137.66, 25.90) --
	(137.59, 25.90) --
	(137.59, 25.90) --
	(137.59, 25.90) --
	(137.51, 25.90) --
	(137.51, 25.90) --
	(137.51, 25.90) --
	(137.44, 25.90) --
	(137.44, 25.90) --
	(137.44, 25.90) --
	(137.36, 25.90) --
	(137.36, 25.90) --
	(137.36, 25.90) --
	(137.29, 25.90) --
	(137.29, 25.90) --
	(137.29, 25.90) --
	(137.21, 25.90) --
	(137.21, 25.90) --
	(137.21, 25.90) --
	(137.13, 25.90) --
	(137.13, 25.90) --
	(137.13, 25.90) --
	(137.06, 25.90) --
	(137.06, 25.90) --
	(137.06, 25.90) --
	(136.98, 25.90) --
	(136.98, 25.90) --
	(136.98, 25.90) --
	(136.90, 25.90) --
	(136.90, 25.90) --
	(136.90, 25.90) --
	(136.83, 25.90) --
	(136.83, 25.90) --
	(136.83, 25.90) --
	(136.75, 25.90) --
	(136.75, 25.90) --
	(136.75, 25.90) --
	(136.68, 25.90) --
	(136.68, 25.90) --
	(136.68, 25.90) --
	(136.60, 25.90) --
	(136.60, 25.90) --
	(136.60, 25.90) --
	(136.52, 25.90) --
	(136.52, 25.90) --
	(136.52, 25.90) --
	(136.45, 25.90) --
	(136.45, 25.90) --
	(136.45, 25.90) --
	(136.37, 25.90) --
	(136.37, 25.90) --
	(136.37, 25.90) --
	(136.30, 25.90) --
	(136.30, 25.90) --
	(136.30, 25.90) --
	(136.22, 25.90) --
	(136.22, 25.90) --
	(136.22, 25.90) --
	(136.14, 25.90) --
	(136.14, 25.90) --
	(136.14, 25.90) --
	(136.07, 25.90) --
	(136.07, 25.90) --
	(136.07, 25.90) --
	(135.99, 25.90) --
	(135.99, 25.90) --
	(135.99, 25.90) --
	(135.91, 25.90) --
	(135.91, 25.90) --
	(135.91, 25.90) --
	(135.84, 25.90) --
	(135.84, 25.90) --
	(135.84, 25.90) --
	(135.76, 25.90) --
	(135.76, 25.90) --
	(135.76, 25.90) --
	(135.69, 25.90) --
	(135.69, 25.90) --
	(135.69, 25.90) --
	(135.61, 25.90) --
	(135.61, 25.90) --
	(135.61, 25.90) --
	(135.53, 25.90) --
	(135.53, 25.90) --
	(135.53, 25.90) --
	(135.46, 25.90) --
	(135.46, 25.90) --
	(135.46, 25.90) --
	(135.38, 25.90) --
	(135.38, 25.90) --
	(135.38, 25.90) --
	(135.31, 25.90) --
	(135.31, 25.90) --
	(135.31, 25.90) --
	(135.23, 25.90) --
	(135.23, 25.90) --
	(135.23, 25.90) --
	(135.15, 25.90) --
	(135.15, 25.90) --
	(135.15, 25.90) --
	(135.08, 25.90) --
	(135.08, 25.90) --
	(135.08, 25.90) --
	(135.00, 25.90) --
	(135.00, 25.90) --
	(135.00, 25.90) --
	(134.92, 25.90) --
	(134.92, 25.90) --
	(134.92, 25.90) --
	(134.85, 25.90) --
	(134.85, 25.90) --
	(134.85, 25.90) --
	(134.77, 25.90) --
	(134.77, 25.90) --
	(134.77, 25.90) --
	(134.70, 25.90) --
	(134.69, 25.90) --
	(134.69, 25.90) --
	(134.62, 25.90) --
	(134.62, 25.90) --
	(134.62, 25.90) --
	(134.54, 25.90) --
	(134.54, 25.90) --
	(134.54, 25.90) --
	(134.47, 25.90) --
	(134.47, 25.90) --
	(134.47, 25.90) --
	(134.39, 25.90) --
	(134.39, 25.90) --
	(134.39, 25.90) --
	(134.31, 25.90) --
	(134.31, 25.90) --
	(134.31, 25.90) --
	(134.24, 25.90) --
	(134.24, 25.90) --
	(134.24, 25.90) --
	(134.16, 25.90) --
	(134.16, 25.90) --
	(134.16, 25.90) --
	(134.08, 25.90) --
	(134.08, 25.90) --
	(134.08, 25.90) --
	(134.01, 25.90) --
	(134.01, 25.90) --
	(134.01, 25.90) --
	(133.93, 25.90) --
	(133.93, 25.90) --
	(133.93, 25.90) --
	(133.86, 25.90) --
	(133.86, 25.90) --
	(133.86, 25.90) --
	(133.78, 25.90) --
	(133.78, 25.90) --
	(133.78, 25.90) --
	(133.70, 25.90) --
	(133.70, 25.90) --
	(133.70, 25.90) --
	(133.63, 25.90) --
	(133.63, 25.90) --
	(133.63, 25.90) --
	(133.55, 25.90) --
	(133.55, 25.90) --
	(133.55, 25.90) --
	(133.48, 25.90) --
	(133.48, 25.90) --
	(133.48, 25.90) --
	cycle;
\definecolor{drawColor}{RGB}{0,191,125}

\path[draw=drawColor,line width= 0.6pt,line join=round] (133.48, 25.92) --
	(133.48, 25.92) --
	(133.55, 26.00) --
	(133.55, 26.00) --
	(133.55, 26.00) --
	(133.63, 26.00) --
	(133.63, 26.00) --
	(133.63, 26.00) --
	(133.70, 26.00) --
	(133.70, 26.00) --
	(133.70, 26.00) --
	(133.78, 26.00) --
	(133.78, 26.00) --
	(133.78, 26.00) --
	(133.86, 26.00) --
	(133.86, 26.00) --
	(133.86, 26.00) --
	(133.93, 26.01) --
	(133.93, 26.01) --
	(133.93, 26.01) --
	(134.01, 26.00) --
	(134.01, 26.00) --
	(134.01, 26.00) --
	(134.08, 26.01) --
	(134.08, 26.01) --
	(134.08, 26.01) --
	(134.16, 26.01) --
	(134.16, 26.01) --
	(134.16, 26.01) --
	(134.24, 26.00) --
	(134.24, 26.00) --
	(134.24, 26.00) --
	(134.31, 26.00) --
	(134.31, 26.00) --
	(134.31, 26.00) --
	(134.39, 26.00) --
	(134.39, 26.00) --
	(134.39, 26.00) --
	(134.47, 26.00) --
	(134.47, 26.00) --
	(134.47, 26.00) --
	(134.54, 26.01) --
	(134.54, 26.01) --
	(134.54, 26.01) --
	(134.62, 26.00) --
	(134.62, 26.00) --
	(134.62, 26.00) --
	(134.69, 26.01) --
	(134.69, 26.01) --
	(134.70, 26.01) --
	(134.77, 26.01) --
	(134.77, 26.01) --
	(134.77, 26.01) --
	(134.85, 26.00) --
	(134.85, 26.00) --
	(134.85, 26.00) --
	(134.92, 26.00) --
	(134.92, 26.00) --
	(134.92, 26.00) --
	(135.00, 26.02) --
	(135.00, 26.02) --
	(135.00, 26.02) --
	(135.08, 26.00) --
	(135.08, 26.00) --
	(135.08, 26.00) --
	(135.15, 26.00) --
	(135.15, 26.00) --
	(135.15, 26.00) --
	(135.23, 26.01) --
	(135.23, 26.01) --
	(135.23, 26.01) --
	(135.31, 26.00) --
	(135.31, 26.00) --
	(135.31, 26.00) --
	(135.38, 26.03) --
	(135.38, 26.03) --
	(135.38, 26.03) --
	(135.46, 26.04) --
	(135.46, 26.04) --
	(135.46, 26.04) --
	(135.53, 26.03) --
	(135.53, 26.03) --
	(135.53, 26.03) --
	(135.61, 26.04) --
	(135.61, 26.04) --
	(135.61, 26.04) --
	(135.69, 26.03) --
	(135.69, 26.03) --
	(135.69, 26.03) --
	(135.76, 26.03) --
	(135.76, 26.03) --
	(135.76, 26.03) --
	(135.84, 26.04) --
	(135.84, 26.04) --
	(135.84, 26.04) --
	(135.91, 26.03) --
	(135.91, 26.03) --
	(135.91, 26.03) --
	(135.99, 26.03) --
	(135.99, 26.03) --
	(135.99, 26.03) --
	(136.07, 26.03) --
	(136.07, 26.03) --
	(136.07, 26.03) --
	(136.14, 26.03) --
	(136.14, 26.03) --
	(136.14, 26.03) --
	(136.22, 26.03) --
	(136.22, 26.03) --
	(136.22, 26.03) --
	(136.30, 26.04) --
	(136.30, 26.04) --
	(136.30, 26.04) --
	(136.37, 26.04) --
	(136.37, 26.04) --
	(136.37, 26.04) --
	(136.45, 26.04) --
	(136.45, 26.04) --
	(136.45, 26.04) --
	(136.52, 26.03) --
	(136.52, 26.03) --
	(136.52, 26.03) --
	(136.60, 26.04) --
	(136.60, 26.04) --
	(136.60, 26.04) --
	(136.68, 26.05) --
	(136.68, 26.05) --
	(136.68, 26.05) --
	(136.75, 26.02) --
	(136.75, 26.02) --
	(136.75, 26.02) --
	(136.83, 26.04) --
	(136.83, 26.04) --
	(136.83, 26.04) --
	(136.90, 26.04) --
	(136.90, 26.04) --
	(136.90, 26.04) --
	(136.98, 26.03) --
	(136.98, 26.03) --
	(136.98, 26.03) --
	(137.06, 26.03) --
	(137.06, 26.03) --
	(137.06, 26.03) --
	(137.13, 26.03) --
	(137.13, 26.03) --
	(137.13, 26.03) --
	(137.21, 26.03) --
	(137.21, 26.03) --
	(137.21, 26.03) --
	(137.29, 26.04) --
	(137.29, 26.04) --
	(137.29, 26.04) --
	(137.36, 26.03) --
	(137.36, 26.03) --
	(137.36, 26.03) --
	(137.44, 26.04) --
	(137.44, 26.04) --
	(137.44, 26.04) --
	(137.51, 26.05) --
	(137.51, 26.05) --
	(137.51, 26.05) --
	(137.59, 26.04) --
	(137.59, 26.04) --
	(137.59, 26.04) --
	(137.66, 26.03) --
	(137.66, 26.03) --
	(137.66, 26.03) --
	(137.74, 26.03) --
	(137.74, 26.03) --
	(137.74, 26.03) --
	(137.82, 26.04) --
	(137.82, 26.04) --
	(137.82, 26.04) --
	(137.89, 26.04) --
	(137.89, 26.04) --
	(137.89, 26.04) --
	(137.97, 26.04) --
	(137.97, 26.04) --
	(137.97, 26.04) --
	(138.05, 26.04) --
	(138.05, 26.04) --
	(138.05, 26.04) --
	(138.12, 26.04) --
	(138.12, 26.04) --
	(138.12, 26.04) --
	(138.20, 26.03) --
	(138.20, 26.03) --
	(138.20, 26.03) --
	(138.27, 26.04) --
	(138.27, 26.04) --
	(138.27, 26.04) --
	(138.35, 26.05) --
	(138.35, 26.05) --
	(138.35, 26.05) --
	(138.43, 26.03) --
	(138.43, 26.03) --
	(138.43, 26.03) --
	(138.50, 26.04) --
	(138.50, 26.04) --
	(138.50, 26.04) --
	(138.58, 26.05) --
	(138.58, 26.05) --
	(138.58, 26.05) --
	(138.65, 26.04) --
	(138.65, 26.04) --
	(138.65, 26.04) --
	(138.73, 26.04) --
	(138.73, 26.04) --
	(138.73, 26.04) --
	(138.81, 26.04) --
	(138.81, 26.04) --
	(138.81, 26.04) --
	(138.88, 26.04) --
	(138.88, 26.04) --
	(138.88, 26.04) --
	(138.96, 26.04) --
	(138.96, 26.04) --
	(138.96, 26.04) --
	(139.03, 26.04) --
	(139.03, 26.04) --
	(139.03, 26.04) --
	(139.11, 26.04) --
	(139.11, 26.04) --
	(139.11, 26.04) --
	(139.19, 26.06) --
	(139.19, 26.06) --
	(139.19, 26.06) --
	(139.26, 26.05) --
	(139.26, 26.05) --
	(139.26, 26.05) --
	(139.34, 26.04) --
	(139.34, 26.04) --
	(139.34, 26.04) --
	(139.42, 26.05) --
	(139.42, 26.05) --
	(139.42, 26.05) --
	(139.49, 26.04) --
	(139.49, 26.04) --
	(139.49, 26.04) --
	(139.57, 26.05) --
	(139.57, 26.05) --
	(139.57, 26.05) --
	(139.64, 26.04) --
	(139.64, 26.04) --
	(139.64, 26.04) --
	(139.72, 26.04) --
	(139.72, 26.04) --
	(139.72, 26.04) --
	(139.79, 26.05) --
	(139.79, 26.05) --
	(139.79, 26.05) --
	(139.87, 26.04) --
	(139.87, 26.04) --
	(139.87, 26.04) --
	(139.95, 26.05) --
	(139.95, 26.05) --
	(139.95, 26.05) --
	(140.02, 26.06) --
	(140.02, 26.06) --
	(140.02, 26.06) --
	(140.10, 26.04) --
	(140.10, 26.04) --
	(140.10, 26.04) --
	(140.17, 26.04) --
	(140.17, 26.04) --
	(140.17, 26.04) --
	(140.25, 26.05) --
	(140.25, 26.05) --
	(140.25, 26.05) --
	(140.33, 26.04) --
	(140.33, 26.04) --
	(140.33, 26.04) --
	(140.40, 26.05) --
	(140.40, 26.05) --
	(140.40, 26.05) --
	(140.48, 26.05) --
	(140.48, 26.05) --
	(140.48, 26.05) --
	(140.55, 26.04) --
	(140.55, 26.04) --
	(140.55, 26.04) --
	(140.63, 26.04) --
	(140.63, 26.04) --
	(140.63, 26.04) --
	(140.71, 26.04) --
	(140.71, 26.04) --
	(140.71, 26.04) --
	(140.78, 26.05) --
	(140.78, 26.05) --
	(140.78, 26.05) --
	(140.86, 26.05) --
	(140.86, 26.05) --
	(140.86, 26.05) --
	(140.93, 26.04) --
	(140.93, 26.04) --
	(140.93, 26.04) --
	(141.01, 26.05) --
	(141.01, 26.05) --
	(141.01, 26.05) --
	(141.08, 26.05) --
	(141.08, 26.05) --
	(141.08, 26.05) --
	(141.16, 26.04) --
	(141.16, 26.04) --
	(141.16, 26.04) --
	(141.24, 26.04) --
	(141.24, 26.04) --
	(141.24, 26.04) --
	(141.31, 26.06) --
	(141.31, 26.06) --
	(141.31, 26.06) --
	(141.39, 26.04) --
	(141.39, 26.04) --
	(141.39, 26.04) --
	(141.46, 26.06) --
	(141.46, 26.06) --
	(141.46, 26.06) --
	(141.54, 26.05) --
	(141.54, 26.05) --
	(141.54, 26.05) --
	(141.62, 26.04) --
	(141.62, 26.04) --
	(141.62, 26.04) --
	(141.69, 26.06) --
	(141.69, 26.06) --
	(141.69, 26.06) --
	(141.77, 26.04) --
	(141.77, 26.04) --
	(141.77, 26.04) --
	(141.84, 26.05) --
	(141.84, 26.05) --
	(141.84, 26.05) --
	(141.92, 26.05) --
	(141.92, 26.05) --
	(141.92, 26.05) --
	(142.00, 26.05) --
	(142.00, 26.05) --
	(142.00, 26.05) --
	(142.07, 26.05) --
	(142.07, 26.05) --
	(142.07, 26.05) --
	(142.15, 26.06) --
	(142.15, 26.06) --
	(142.15, 26.06) --
	(142.22, 26.05) --
	(142.22, 26.05) --
	(142.22, 26.05) --
	(142.30, 26.05) --
	(142.30, 26.05) --
	(142.30, 26.05) --
	(142.37, 26.05) --
	(142.37, 26.05) --
	(142.37, 26.05) --
	(142.45, 26.04) --
	(142.45, 26.04) --
	(142.45, 26.04) --
	(142.53, 26.06) --
	(142.53, 26.06) --
	(142.53, 26.06) --
	(142.60, 26.04) --
	(142.60, 26.04) --
	(142.60, 26.04) --
	(142.68, 26.05) --
	(142.68, 26.05) --
	(142.68, 26.05) --
	(142.75, 26.06) --
	(142.75, 26.06) --
	(142.75, 26.06) --
	(142.83, 26.05) --
	(142.83, 26.05) --
	(142.83, 26.05) --
	(142.91, 26.05) --
	(142.91, 26.05) --
	(142.91, 26.05) --
	(142.98, 26.05) --
	(142.98, 26.05) --
	(142.98, 26.05) --
	(143.06, 26.06) --
	(143.06, 26.06) --
	(143.06, 26.06) --
	(143.13, 26.06) --
	(143.13, 26.06) --
	(143.13, 26.06) --
	(143.21, 26.06) --
	(143.21, 26.06) --
	(143.21, 26.06) --
	(143.29, 26.06) --
	(143.29, 26.06) --
	(143.29, 26.06) --
	(143.36, 26.07) --
	(143.36, 26.07) --
	(143.36, 26.07) --
	(143.44, 26.06) --
	(143.44, 26.06) --
	(143.44, 26.06) --
	(143.51, 26.06) --
	(143.51, 26.06) --
	(143.51, 26.06) --
	(143.59, 26.06) --
	(143.59, 26.06) --
	(143.59, 26.06) --
	(143.66, 26.06) --
	(143.66, 26.06) --
	(143.66, 26.06) --
	(143.74, 26.07) --
	(143.74, 26.07) --
	(143.74, 26.07) --
	(143.82, 26.07) --
	(143.82, 26.07) --
	(143.82, 26.07) --
	(143.89, 26.06) --
	(143.89, 26.06) --
	(143.89, 26.06) --
	(143.97, 26.06) --
	(143.97, 26.06) --
	(143.97, 26.06) --
	(144.04, 26.07) --
	(144.04, 26.07) --
	(144.04, 26.07) --
	(144.12, 26.07) --
	(144.12, 26.07) --
	(144.12, 26.07) --
	(144.19, 26.07) --
	(144.19, 26.07) --
	(144.19, 26.07) --
	(144.27, 26.07) --
	(144.27, 26.07) --
	(144.27, 26.07) --
	(144.35, 26.07) --
	(144.35, 26.07) --
	(144.35, 26.07) --
	(144.42, 26.08) --
	(144.42, 26.08) --
	(144.42, 26.08) --
	(144.50, 26.08) --
	(144.50, 26.08) --
	(144.50, 26.08) --
	(144.57, 26.08) --
	(144.57, 26.08) --
	(144.57, 26.08) --
	(144.65, 26.09) --
	(144.65, 26.09) --
	(144.65, 26.09) --
	(144.72, 26.08) --
	(144.72, 26.08) --
	(144.72, 26.08) --
	(144.80, 26.09) --
	(144.80, 26.09) --
	(144.80, 26.09) --
	(144.88, 26.09) --
	(144.88, 26.09) --
	(144.88, 26.09) --
	(144.95, 26.10) --
	(144.95, 26.10) --
	(144.95, 26.10) --
	(145.03, 26.10) --
	(145.03, 26.10) --
	(145.03, 26.10) --
	(145.10, 26.10) --
	(145.10, 26.10) --
	(145.10, 26.10) --
	(145.18, 26.11) --
	(145.18, 26.11) --
	(145.18, 26.11) --
	(145.26, 26.12) --
	(145.26, 26.12) --
	(145.26, 26.12) --
	(145.33, 26.11) --
	(145.33, 26.11) --
	(145.33, 26.11) --
	(145.41, 26.12) --
	(145.41, 26.12) --
	(145.41, 26.12) --
	(145.48, 26.14) --
	(145.48, 26.14) --
	(145.48, 26.14) --
	(145.56, 26.14) --
	(145.56, 26.14) --
	(145.56, 26.14) --
	(145.63, 26.17) --
	(145.63, 26.17) --
	(145.63, 26.17) --
	(145.71, 26.24) --
	(145.71, 26.24) --
	(145.71, 26.24) --
	(145.79, 26.36) --
	(145.79, 26.36) --
	(145.79, 26.36) --
	(145.86, 26.47) --
	(145.86, 26.47) --
	(145.86, 26.47) --
	(145.94, 26.45) --
	(145.94, 26.45) --
	(145.94, 26.45) --
	(146.01, 26.30) --
	(146.01, 26.30) --
	(146.01, 26.30) --
	(146.09, 26.21) --
	(146.09, 26.21) --
	(146.09, 26.21) --
	(146.16, 26.18) --
	(146.16, 26.18) --
	(146.16, 26.18) --
	(146.24, 26.17) --
	(146.24, 26.17) --
	(146.24, 26.17) --
	(146.32, 26.18) --
	(146.32, 26.18) --
	(146.32, 26.18) --
	(146.39, 26.21) --
	(146.39, 26.21) --
	(146.39, 26.21) --
	(146.47, 26.26) --
	(146.47, 26.26) --
	(146.47, 26.26) --
	(146.54, 26.33) --
	(146.54, 26.33) --
	(146.54, 26.33) --
	(146.62, 26.39) --
	(146.62, 26.39) --
	(146.62, 26.39) --
	(146.69, 26.43) --
	(146.69, 26.43) --
	(146.69, 26.43) --
	(146.77, 26.44) --
	(146.77, 26.44) --
	(146.77, 26.44) --
	(146.84, 26.47) --
	(146.84, 26.47) --
	(146.84, 26.47) --
	(146.92, 26.51) --
	(146.92, 26.51) --
	(146.92, 26.51) --
	(146.99, 26.51) --
	(146.99, 26.51) --
	(146.99, 26.51) --
	(147.07, 26.53) --
	(147.07, 26.53) --
	(147.07, 26.53) --
	(147.15, 26.57) --
	(147.15, 26.57) --
	(147.15, 26.57) --
	(147.22, 26.68) --
	(147.22, 26.68) --
	(147.22, 26.68) --
	(147.30, 26.79) --
	(147.30, 26.79) --
	(147.30, 26.79) --
	(147.37, 26.90) --
	(147.37, 26.90) --
	(147.37, 26.90) --
	(147.45, 26.97) --
	(147.45, 26.97) --
	(147.45, 26.97) --
	(147.52, 26.96) --
	(147.52, 26.96) --
	(147.52, 26.96) --
	(147.60, 26.91) --
	(147.60, 26.91) --
	(147.60, 26.91) --
	(147.68, 26.91) --
	(147.68, 26.91) --
	(147.68, 26.91) --
	(147.75, 26.89) --
	(147.75, 26.89) --
	(147.75, 26.89) --
	(147.83, 26.78) --
	(147.83, 26.78) --
	(147.83, 26.78) --
	(147.90, 26.66) --
	(147.90, 26.66) --
	(147.90, 26.66) --
	(147.98, 26.61) --
	(147.98, 26.61) --
	(147.98, 26.61) --
	(148.05, 26.67) --
	(148.05, 26.67) --
	(148.05, 26.67) --
	(148.13, 26.93) --
	(148.13, 26.93) --
	(148.13, 26.93) --
	(148.20, 27.33) --
	(148.20, 27.33) --
	(148.20, 27.33) --
	(148.28, 27.67) --
	(148.28, 27.67) --
	(148.28, 27.67) --
	(148.36, 27.61) --
	(148.36, 27.61) --
	(148.36, 27.61) --
	(148.43, 27.17) --
	(148.43, 27.17) --
	(148.43, 27.17) --
	(148.51, 26.85) --
	(148.51, 26.85) --
	(148.51, 26.85) --
	(148.58, 26.80) --
	(148.58, 26.80) --
	(148.58, 26.80) --
	(148.66, 26.80) --
	(148.66, 26.80) --
	(148.66, 26.80) --
	(148.73, 26.80) --
	(148.73, 26.80) --
	(148.73, 26.80) --
	(148.81, 26.77) --
	(148.81, 26.77) --
	(148.81, 26.77) --
	(148.88, 26.77) --
	(148.88, 26.77) --
	(148.88, 26.77) --
	(148.96, 26.78) --
	(148.96, 26.78) --
	(148.96, 26.78) --
	(149.04, 26.74) --
	(149.04, 26.74) --
	(149.04, 26.74) --
	(149.11, 26.65) --
	(149.11, 26.65) --
	(149.11, 26.65) --
	(149.19, 26.63) --
	(149.19, 26.63) --
	(149.19, 26.63) --
	(149.26, 26.66) --
	(149.26, 26.66) --
	(149.26, 26.66) --
	(149.34, 26.70) --
	(149.34, 26.70) --
	(149.34, 26.70) --
	(149.41, 26.72) --
	(149.41, 26.72) --
	(149.41, 26.72) --
	(149.49, 26.69) --
	(149.49, 26.69) --
	(149.49, 26.69) --
	(149.56, 26.86) --
	(149.56, 26.86) --
	(149.56, 26.86) --
	(149.64, 27.26) --
	(149.64, 27.26) --
	(149.64, 27.26) --
	(149.72, 27.68) --
	(149.72, 27.68) --
	(149.72, 27.68) --
	(149.79, 27.76) --
	(149.79, 27.76) --
	(149.79, 27.76) --
	(149.87, 27.34) --
	(149.87, 27.34) --
	(149.87, 27.34) --
	(149.94, 27.06) --
	(149.94, 27.06) --
	(149.94, 27.06) --
	(150.02, 27.43) --
	(150.02, 27.43) --
	(150.02, 27.43) --
	(150.09, 28.09) --
	(150.09, 28.09) --
	(150.09, 28.09) --
	(150.17, 28.57) --
	(150.17, 28.57) --
	(150.17, 28.57) --
	(150.24, 28.46) --
	(150.24, 28.46) --
	(150.24, 28.46) --
	(150.32, 27.91) --
	(150.32, 27.91) --
	(150.32, 27.91) --
	(150.39, 27.56) --
	(150.39, 27.56) --
	(150.39, 27.56) --
	(150.47, 27.51) --
	(150.47, 27.51) --
	(150.47, 27.51) --
	(150.55, 27.57) --
	(150.55, 27.57) --
	(150.55, 27.57) --
	(150.62, 27.74) --
	(150.62, 27.74) --
	(150.62, 27.74) --
	(150.70, 28.10) --
	(150.70, 28.10) --
	(150.70, 28.10) --
	(150.77, 29.17) --
	(150.77, 29.17) --
	(150.77, 29.17) --
	(150.85, 30.87) --
	(150.85, 30.87) --
	(150.85, 30.87) --
	(150.92, 32.27) --
	(150.92, 32.27) --
	(150.92, 32.27) --
	(151.00, 32.13) --
	(151.00, 32.13) --
	(151.00, 32.13) --
	(151.07, 30.11) --
	(151.07, 30.11) --
	(151.07, 30.11) --
	(151.15, 28.76) --
	(151.15, 28.76) --
	(151.15, 28.76) --
	(151.22, 28.50) --
	(151.22, 28.50) --
	(151.22, 28.50) --
	(151.30, 28.58) --
	(151.30, 28.58) --
	(151.30, 28.58) --
	(151.37, 28.84) --
	(151.37, 28.84) --
	(151.37, 28.84) --
	(151.45, 29.31) --
	(151.45, 29.31) --
	(151.45, 29.31) --
	(151.53, 29.72) --
	(151.53, 29.72) --
	(151.53, 29.72) --
	(151.60, 30.66) --
	(151.60, 30.66) --
	(151.60, 30.66) --
	(151.68, 33.12) --
	(151.68, 33.12) --
	(151.68, 33.12) --
	(151.75, 35.81) --
	(151.75, 35.81) --
	(151.75, 35.81) --
	(151.83, 36.67) --
	(151.83, 36.67) --
	(151.83, 36.67) --
	(151.90, 35.78) --
	(151.90, 35.78) --
	(151.90, 35.78) --
	(151.98, 31.87) --
	(151.98, 31.87) --
	(151.98, 31.87) --
	(152.05, 30.01) --
	(152.05, 30.01) --
	(152.05, 30.01) --
	(152.13, 29.62) --
	(152.13, 29.62) --
	(152.13, 29.62) --
	(152.20, 29.44) --
	(152.20, 29.44) --
	(152.20, 29.44) --
	(152.28, 29.25) --
	(152.28, 29.25) --
	(152.28, 29.25) --
	(152.35, 29.19) --
	(152.35, 29.18) --
	(152.35, 29.19) --
	(152.43, 29.31) --
	(152.43, 29.31) --
	(152.43, 29.31) --
	(152.51, 29.00) --
	(152.51, 29.00) --
	(152.51, 29.00) --
	(152.58, 28.58) --
	(152.58, 28.58) --
	(152.58, 28.58) --
	(152.66, 28.36) --
	(152.66, 28.36) --
	(152.66, 28.36) --
	(152.73, 28.21) --
	(152.73, 28.21) --
	(152.73, 28.21) --
	(152.81, 28.11) --
	(152.81, 28.11) --
	(152.81, 28.11) --
	(152.88, 28.19) --
	(152.88, 28.19) --
	(152.88, 28.19) --
	(152.96, 28.48) --
	(152.96, 28.48) --
	(152.96, 28.48) --
	(153.03, 28.77) --
	(153.03, 28.77) --
	(153.03, 28.77) --
	(153.11, 28.82) --
	(153.11, 28.82) --
	(153.11, 28.82) --
	(153.11, 28.82) --
	(153.11, 28.82) --
	(153.11, 28.82) --
	(153.18, 28.54) --
	(153.18, 28.54) --
	(153.18, 28.54) --
	(153.26, 29.07) --
	(153.26, 29.07) --
	(153.26, 29.07) --
	(153.26, 29.11) --
	(153.26, 29.11) --
	(153.26, 29.12) --
	(153.33, 32.91) --
	(153.33, 32.91) --
	(153.33, 32.91) --
	(153.35, 33.69) --
	(153.35, 33.69) --
	(153.35, 33.69) --
	(153.39, 35.92) --
	(153.39, 35.92) --
	(153.39, 35.92) --
	(153.41, 37.01) --
	(153.41, 37.01) --
	(153.41, 37.01) --
	(153.47, 38.45) --
	(153.47, 38.45) --
	(153.47, 38.45) --
	(153.48, 38.80) --
	(153.48, 38.81) --
	(153.48, 38.81) --
	(153.56, 39.16) --
	(153.56, 39.16) --
	(153.56, 39.16) --
	(153.59, 38.89) --
	(153.59, 38.89) --
	(153.59, 38.89) --
	(153.63, 38.52) --
	(153.63, 38.52) --
	(153.63, 38.52) --
	(153.67, 37.16) --
	(153.67, 37.15) --
	(153.67, 37.15) --
	(153.71, 35.73) --
	(153.71, 35.73) --
	(153.71, 35.73) --
	(153.79, 30.24) --
	(153.79, 30.24) --
	(153.79, 30.24) --
	(153.86, 28.70) --
	(153.86, 28.70) --
	(153.86, 28.70) --
	(153.94, 28.23) --
	(153.94, 28.23) --
	(153.94, 28.23) --
	(153.95, 28.18) --
	(153.95, 28.18) --
	(153.95, 28.18) --
	(154.01, 28.01) --
	(154.01, 28.01) --
	(154.01, 28.01) --
	(154.09, 27.80) --
	(154.09, 27.80) --
	(154.09, 27.80) --
	(154.16, 27.68) --
	(154.16, 27.68) --
	(154.16, 27.68) --
	(154.24, 27.67) --
	(154.24, 27.67) --
	(154.24, 27.67) --
	(154.31, 27.86) --
	(154.31, 27.86) --
	(154.31, 27.86) --
	(154.39, 28.10) --
	(154.39, 28.10) --
	(154.39, 28.10) --
	(154.46, 28.35) --
	(154.46, 28.35) --
	(154.46, 28.35) --
	(154.54, 28.69) --
	(154.54, 28.69) --
	(154.54, 28.69) --
	(154.61, 29.00) --
	(154.61, 29.00) --
	(154.61, 29.00) --
	(154.64, 29.05) --
	(154.64, 29.05) --
	(154.64, 29.05) --
	(154.69, 29.15) --
	(154.69, 29.15) --
	(154.69, 29.15) --
	(154.76, 29.30) --
	(154.76, 29.30) --
	(154.76, 29.30) --
	(154.84, 29.91) --
	(154.84, 29.91) --
	(154.84, 29.91) --
	(154.91, 30.77) --
	(154.91, 30.77) --
	(154.91, 30.77) --
	(154.99, 31.39) --
	(154.99, 31.39) --
	(154.99, 31.39) --
	(155.06, 31.69) --
	(155.06, 31.69) --
	(155.06, 31.69) --
	(155.14, 31.70) --
	(155.14, 31.70) --
	(155.14, 31.70) --
	(155.22, 30.90) --
	(155.22, 30.90) --
	(155.22, 30.90) --
	(155.29, 29.68) --
	(155.29, 29.68) --
	(155.29, 29.68) --
	(155.37, 28.93) --
	(155.37, 28.92) --
	(155.37, 28.92) --
	(155.44, 28.57) --
	(155.44, 28.57) --
	(155.44, 28.57) --
	(155.52, 28.30) --
	(155.52, 28.30) --
	(155.52, 28.30) --
	(155.59, 28.00) --
	(155.59, 28.00) --
	(155.59, 28.00) --
	(155.67, 27.93) --
	(155.67, 27.93) --
	(155.67, 27.93) --
	(155.74, 27.97) --
	(155.74, 27.97) --
	(155.74, 27.97) --
	(155.82, 27.98) --
	(155.82, 27.98) --
	(155.82, 27.98) --
	(155.89, 27.98) --
	(155.89, 27.98) --
	(155.89, 27.98) --
	(155.90, 27.99) --
	(155.90, 27.99) --
	(155.90, 27.99) --
	(155.97, 28.02) --
	(155.97, 28.02) --
	(155.97, 28.02) --
	(156.04, 28.21) --
	(156.04, 28.21) --
	(156.04, 28.21) --
	(156.12, 28.34) --
	(156.12, 28.34) --
	(156.12, 28.34) --
	(156.19, 28.22) --
	(156.19, 28.22) --
	(156.19, 28.22) --
	(156.23, 28.19) --
	(156.23, 28.19) --
	(156.23, 28.19) --
	(156.27, 28.17) --
	(156.27, 28.17) --
	(156.27, 28.17) --
	(156.34, 28.33) --
	(156.34, 28.33) --
	(156.34, 28.33) --
	(156.42, 28.55) --
	(156.42, 28.55) --
	(156.42, 28.55) --
	(156.49, 28.30) --
	(156.49, 28.30) --
	(156.49, 28.30) --
	(156.50, 28.26) --
	(156.50, 28.26) --
	(156.50, 28.26) --
	(156.57, 27.91) --
	(156.57, 27.91) --
	(156.57, 27.91) --
	(156.64, 27.86) --
	(156.64, 27.86) --
	(156.64, 27.86) --
	(156.72, 28.13) --
	(156.72, 28.13) --
	(156.72, 28.13) --
	(156.79, 29.12) --
	(156.79, 29.12) --
	(156.79, 29.12) --
	(156.87, 30.60) --
	(156.87, 30.60) --
	(156.87, 30.60) --
	(156.87, 30.66) --
	(156.87, 30.66) --
	(156.87, 30.66) --
	(156.94, 32.00) --
	(156.94, 32.00) --
	(156.94, 32.00) --
	(157.02, 31.98) --
	(157.02, 31.98) --
	(157.02, 31.98) --
	(157.03, 31.76) --
	(157.03, 31.76) --
	(157.03, 31.76) --
	(157.09, 29.76) --
	(157.09, 29.76) --
	(157.09, 29.76) --
	(157.17, 28.20) --
	(157.17, 28.20) --
	(157.17, 28.20) --
	(157.23, 27.96) --
	(157.23, 27.96) --
	(157.23, 27.96) --
	(157.24, 27.90) --
	(157.24, 27.90) --
	(157.24, 27.90) --
	(157.32, 27.84) --
	(157.32, 27.84) --
	(157.32, 27.84) --
	(157.39, 27.87) --
	(157.39, 27.87) --
	(157.39, 27.87) --
	(157.47, 28.00) --
	(157.47, 28.00) --
	(157.47, 28.00) --
	(157.54, 28.09) --
	(157.54, 28.09) --
	(157.54, 28.09) --
	(157.62, 28.30) --
	(157.62, 28.30) --
	(157.62, 28.30) --
	(157.69, 28.71) --
	(157.69, 28.71) --
	(157.69, 28.71) --
	(157.77, 29.20) --
	(157.77, 29.20) --
	(157.77, 29.20) --
	(157.84, 30.07) --
	(157.84, 30.07) --
	(157.84, 30.07) --
	(157.92, 31.43) --
	(157.92, 31.43) --
	(157.92, 31.43) --
	(157.99, 33.03) --
	(157.99, 33.03) --
	(157.99, 33.03) --
	(158.04, 33.51) --
	(158.04, 33.51) --
	(158.04, 33.51) --
	(158.07, 33.88) --
	(158.07, 33.88) --
	(158.07, 33.88) --
	(158.14, 33.26) --
	(158.15, 33.26) --
	(158.15, 33.26) --
	(158.22, 32.39) --
	(158.22, 32.39) --
	(158.22, 32.39) --
	(158.24, 32.21) --
	(158.24, 32.21) --
	(158.24, 32.21) --
	(158.30, 31.83) --
	(158.30, 31.83) --
	(158.30, 31.83) --
	(158.37, 32.03) --
	(158.37, 32.03) --
	(158.37, 32.03) --
	(158.45, 33.08) --
	(158.45, 33.08) --
	(158.45, 33.08) --
	(158.52, 34.12) --
	(158.52, 34.12) --
	(158.52, 34.12) --
	(158.60, 34.74) --
	(158.60, 34.74) --
	(158.60, 34.74) --
	(158.67, 35.44) --
	(158.67, 35.44) --
	(158.67, 35.44) --
	(158.75, 35.94) --
	(158.75, 35.94) --
	(158.75, 35.94) --
	(158.76, 35.56) --
	(158.76, 35.56) --
	(158.76, 35.56) --
	(158.82, 34.50) --
	(158.82, 34.50) --
	(158.82, 34.50) --
	(158.89, 33.05) --
	(158.89, 33.05) --
	(158.89, 33.05) --
	(158.97, 32.87) --
	(158.97, 32.87) --
	(158.97, 32.87) --
	(159.04, 33.90) --
	(159.04, 33.90) --
	(159.04, 33.90) --
	(159.12, 35.20) --
	(159.12, 35.20) --
	(159.12, 35.20) --
	(159.19, 34.44) --
	(159.19, 34.44) --
	(159.19, 34.44) --
	(159.25, 33.72) --
	(159.25, 33.72) --
	(159.25, 33.72) --
	(159.27, 33.45) --
	(159.27, 33.45) --
	(159.27, 33.45) --
	(159.34, 33.82) --
	(159.34, 33.82) --
	(159.34, 33.82) --
	(159.42, 35.06) --
	(159.42, 35.06) --
	(159.42, 35.06) --
	(159.49, 35.48) --
	(159.49, 35.48) --
	(159.49, 35.48) --
	(159.57, 36.19) --
	(159.57, 36.19) --
	(159.57, 36.20) --
	(159.57, 36.29) --
	(159.57, 36.29) --
	(159.57, 36.29) --
	(159.65, 37.88) --
	(159.65, 37.88) --
	(159.65, 37.88) --
	(159.72, 39.83) --
	(159.72, 39.83) --
	(159.72, 39.83) --
	(159.78, 40.74) --
	(159.78, 40.74) --
	(159.78, 40.74) --
	(159.80, 41.06) --
	(159.80, 41.06) --
	(159.80, 41.06) --
	(159.87, 41.39) --
	(159.87, 41.39) --
	(159.87, 41.39) --
	(159.94, 41.26) --
	(159.94, 41.26) --
	(159.94, 41.26) --
	(159.98, 41.00) --
	(159.98, 41.00) --
	(159.98, 41.00) --
	(160.02, 40.68) --
	(160.02, 40.68) --
	(160.02, 40.68) --
	(160.09, 38.41) --
	(160.09, 38.41) --
	(160.09, 38.41) --
	(160.14, 37.01) --
	(160.14, 37.01) --
	(160.14, 37.01) --
	(160.17, 36.06) --
	(160.17, 36.06) --
	(160.17, 36.06) --
	(160.24, 35.01) --
	(160.24, 35.01) --
	(160.24, 35.01) --
	(160.26, 35.02) --
	(160.26, 35.02) --
	(160.26, 35.02) --
	(160.30, 35.06) --
	(160.30, 35.06) --
	(160.30, 35.06) --
	(160.32, 35.08) --
	(160.32, 35.08) --
	(160.32, 35.09) --
	(160.39, 36.99) --
	(160.39, 36.99) --
	(160.39, 36.99) --
	(160.42, 37.57) --
	(160.42, 37.57) --
	(160.42, 37.57) --
	(160.47, 38.54) --
	(160.47, 38.54) --
	(160.47, 38.54) --
	(160.54, 39.83) --
	(160.54, 39.83) --
	(160.54, 39.83) --
	(160.54, 39.85) --
	(160.54, 39.85) --
	(160.54, 39.85) --
	(160.58, 40.29) --
	(160.58, 40.29) --
	(160.58, 40.29) --
	(160.62, 40.68) --
	(160.62, 40.68) --
	(160.62, 40.68) --
	(160.69, 40.93) --
	(160.69, 40.93) --
	(160.69, 40.93) --
	(160.71, 40.93) --
	(160.71, 40.93) --
	(160.71, 40.93) --
	(160.77, 40.94) --
	(160.77, 40.94) --
	(160.77, 40.94) --
	(160.79, 40.95) --
	(160.79, 40.95) --
	(160.79, 40.95) --
	(160.84, 41.01) --
	(160.84, 41.01) --
	(160.84, 41.01) --
	(160.92, 40.90) --
	(160.92, 40.90) --
	(160.92, 40.90) --
	(160.95, 40.27) --
	(160.95, 40.27) --
	(160.95, 40.27) --
	(160.99, 39.26) --
	(160.99, 39.26) --
	(160.99, 39.26) --
	(161.07, 37.11) --
	(161.07, 37.11) --
	(161.07, 37.11) --
	(161.07, 37.11) --
	(161.07, 37.11) --
	(161.07, 37.11) --
	(161.11, 37.15) --
	(161.11, 37.15) --
	(161.11, 37.15) --
	(161.14, 37.18) --
	(161.14, 37.18) --
	(161.14, 37.18) --
	(161.22, 38.24) --
	(161.22, 38.24) --
	(161.22, 38.24) --
	(161.23, 38.44) --
	(161.23, 38.44) --
	(161.23, 38.44) --
	(161.27, 39.08) --
	(161.27, 39.08) --
	(161.27, 39.08) --
	(161.29, 39.45) --
	(161.29, 39.45) --
	(161.29, 39.45) --
	(161.31, 39.43) --
	(161.31, 39.43) --
	(161.31, 39.43) --
	(161.31, 39.43) --
	(161.31, 39.43) --
	(161.31, 39.43) --
	(161.37, 39.39) --
	(161.37, 39.39) --
	(161.37, 39.39) --
	(161.39, 39.31) --
	(161.39, 39.31) --
	(161.39, 39.31) --
	(161.43, 39.17) --
	(161.43, 39.17) --
	(161.43, 39.17) --
	(161.44, 39.14) --
	(161.44, 39.14) --
	(161.44, 39.14) --
	(161.51, 39.33) --
	(161.51, 39.33) --
	(161.51, 39.33) --
	(161.52, 39.34) --
	(161.52, 39.34) --
	(161.52, 39.34) --
	(161.55, 39.17) --
	(161.55, 39.17) --
	(161.55, 39.17) --
	(161.59, 38.99) --
	(161.59, 38.99) --
	(161.59, 38.99) --
	(161.67, 39.66) --
	(161.67, 39.66) --
	(161.67, 39.66) --
	(161.74, 40.40) --
	(161.74, 40.40) --
	(161.74, 40.40) --
	(161.80, 40.76) --
	(161.80, 40.76) --
	(161.80, 40.77) --
	(161.82, 40.90) --
	(161.82, 40.90) --
	(161.82, 40.91) --
	(161.84, 41.09) --
	(161.84, 41.09) --
	(161.84, 41.09) --
	(161.89, 41.59) --
	(161.89, 41.59) --
	(161.89, 41.59) --
	(161.92, 41.77) --
	(161.92, 41.77) --
	(161.92, 41.77) --
	(161.96, 42.03) --
	(161.96, 42.03) --
	(161.96, 42.03) --
	(161.97, 42.09) --
	(161.97, 42.09) --
	(161.97, 42.09) --
	(162.00, 41.90) --
	(162.00, 41.90) --
	(162.00, 41.90) --
	(162.04, 41.65) --
	(162.04, 41.65) --
	(162.04, 41.65) --
	(162.04, 41.63) --
	(162.04, 41.63) --
	(162.04, 41.63) --
	(162.12, 40.75) --
	(162.12, 40.75) --
	(162.12, 40.75) --
	(162.12, 40.73) --
	(162.12, 40.73) --
	(162.12, 40.73) --
	(162.19, 40.32) --
	(162.19, 40.32) --
	(162.19, 40.32) --
	(162.27, 39.25) --
	(162.27, 39.25) --
	(162.27, 39.25) --
	(162.28, 38.93) --
	(162.28, 38.93) --
	(162.28, 38.93) --
	(162.34, 37.73) --
	(162.34, 37.73) --
	(162.34, 37.73) --
	(162.42, 37.90) --
	(162.42, 37.90) --
	(162.42, 37.90) --
	(162.48, 38.19) --
	(162.48, 38.19) --
	(162.48, 38.19) --
	(162.49, 38.22) --
	(162.49, 38.22) --
	(162.49, 38.22) --
	(162.56, 37.92) --
	(162.56, 37.92) --
	(162.56, 37.92) --
	(162.64, 37.25) --
	(162.64, 37.25) --
	(162.64, 37.25) --
	(162.69, 37.29) --
	(162.69, 37.29) --
	(162.69, 37.29) --
	(162.71, 37.32) --
	(162.71, 37.32) --
	(162.71, 37.32) --
	(162.73, 37.46) --
	(162.73, 37.46) --
	(162.73, 37.47) --
	(162.79, 38.27) --
	(162.79, 38.27) --
	(162.79, 38.27) --
	(162.87, 39.73) --
	(162.87, 39.73) --
	(162.87, 39.73) --
	(162.89, 39.95) --
	(162.89, 39.95) --
	(162.89, 39.95) --
	(162.93, 40.34) --
	(162.93, 40.34) --
	(162.93, 40.34) --
	(162.94, 40.45) --
	(162.94, 40.45) --
	(162.94, 40.45) --
	(163.01, 40.74) --
	(163.01, 40.74) --
	(163.01, 40.74) --
	(163.05, 40.99) --
	(163.05, 40.99) --
	(163.05, 40.99) --
	(163.09, 41.27) --
	(163.09, 41.27) --
	(163.09, 41.27) --
	(163.09, 41.27) --
	(163.09, 41.27) --
	(163.09, 41.27) --
	(163.16, 41.59) --
	(163.16, 41.59) --
	(163.16, 41.59) --
	(163.24, 41.80) --
	(163.24, 41.80) --
	(163.24, 41.80) --
	(163.25, 41.88) --
	(163.25, 41.88) --
	(163.25, 41.88) --
	(163.31, 42.21) --
	(163.31, 42.21) --
	(163.31, 42.21) --
	(163.37, 42.39) --
	(163.37, 42.39) --
	(163.37, 42.39) --
	(163.39, 42.44) --
	(163.39, 42.44) --
	(163.39, 42.44) --
	(163.45, 42.31) --
	(163.45, 42.31) --
	(163.45, 42.31) --
	(163.46, 42.29) --
	(163.46, 42.29) --
	(163.46, 42.29) --
	(163.53, 42.26) --
	(163.53, 42.26) --
	(163.53, 42.26) --
	(163.54, 42.26) --
	(163.54, 42.26) --
	(163.54, 42.26) --
	(163.57, 42.12) --
	(163.57, 42.12) --
	(163.57, 42.12) --
	(163.61, 41.97) --
	(163.61, 41.97) --
	(163.61, 41.97) --
	(163.66, 41.73) --
	(163.66, 41.73) --
	(163.66, 41.73) --
	(163.69, 41.56) --
	(163.69, 41.56) --
	(163.69, 41.56) --
	(163.70, 41.56) --
	(163.70, 41.56) --
	(163.70, 41.56) --
	(163.71, 41.56) --
	(163.71, 41.56) --
	(163.71, 41.56) --
	(163.76, 41.56) --
	(163.76, 41.56) --
	(163.76, 41.56) --
	(163.82, 41.69) --
	(163.82, 41.69) --
	(163.82, 41.69) --
	(163.84, 41.74) --
	(163.84, 41.74) --
	(163.84, 41.74) --
	(163.90, 41.82) --
	(163.90, 41.82) --
	(163.90, 41.82) --
	(163.91, 41.84) --
	(163.91, 41.84) --
	(163.91, 41.84) --
	(163.99, 42.05) --
	(163.99, 42.05) --
	(163.99, 42.05) --
	(164.02, 42.00) --
	(164.02, 42.00) --
	(164.02, 42.00) --
	(164.06, 41.94) --
	(164.06, 41.94) --
	(164.06, 41.94) --
	(164.14, 41.36) --
	(164.14, 41.36) --
	(164.14, 41.36) --
	(164.18, 41.22) --
	(164.18, 41.22) --
	(164.18, 41.22) --
	(164.21, 41.12) --
	(164.21, 41.12) --
	(164.21, 41.12) --
	(164.22, 41.11) --
	(164.22, 41.11) --
	(164.22, 41.11) --
	(164.28, 41.07) --
	(164.28, 41.07) --
	(164.28, 41.07) --
	(164.34, 41.04) --
	(164.34, 41.04) --
	(164.34, 41.04) --
	(164.36, 41.03) --
	(164.36, 41.03) --
	(164.36, 41.03) --
	(164.42, 40.71) --
	(164.42, 40.71) --
	(164.42, 40.71) --
	(164.43, 40.65) --
	(164.43, 40.65) --
	(164.43, 40.65) --
	(164.51, 40.89) --
	(164.51, 40.89) --
	(164.51, 40.89) --
	(164.55, 41.18) --
	(164.55, 41.18) --
	(164.55, 41.18) --
	(164.58, 41.48) --
	(164.58, 41.48) --
	(164.58, 41.48) --
	(164.66, 42.14) --
	(164.66, 42.14) --
	(164.66, 42.14) --
	(164.66, 42.16) --
	(164.66, 42.16) --
	(164.66, 42.16) --
	(164.67, 42.17) --
	(164.67, 42.17) --
	(164.67, 42.17) --
	(164.71, 42.32) --
	(164.71, 42.32) --
	(164.71, 42.32) --
	(164.73, 42.42) --
	(164.73, 42.42) --
	(164.73, 42.42) --
	(164.81, 42.40) --
	(164.81, 42.40) --
	(164.81, 42.40) --
	(164.88, 41.92) --
	(164.88, 41.92) --
	(164.88, 41.92) --
	(164.95, 41.58) --
	(164.95, 41.58) --
	(164.95, 41.58) --
	(164.96, 41.54) --
	(164.96, 41.54) --
	(164.96, 41.54) --
	(164.99, 41.50) --
	(164.99, 41.50) --
	(164.99, 41.50) --
	(165.03, 41.44) --
	(165.03, 41.44) --
	(165.03, 41.44) --
	(165.07, 41.33) --
	(165.07, 41.33) --
	(165.07, 41.33) --
	(165.11, 41.23) --
	(165.11, 41.23) --
	(165.11, 41.23) --
	(165.18, 41.11) --
	(165.18, 41.11) --
	(165.18, 41.11) --
	(165.25, 41.21) --
	(165.25, 41.21) --
	(165.25, 41.21) --
	(165.27, 41.24) --
	(165.27, 41.24) --
	(165.27, 41.24) --
	(165.33, 41.35) --
	(165.33, 41.35) --
	(165.33, 41.35) --
	(165.41, 41.23) --
	(165.41, 41.23) --
	(165.41, 41.23) --
	(165.48, 41.03) --
	(165.48, 41.03) --
	(165.48, 41.03) --
	(165.55, 41.18) --
	(165.55, 41.18) --
	(165.55, 41.18) --
	(165.60, 41.29) --
	(165.60, 41.29) --
	(165.60, 41.29) --
	(165.63, 41.39) --
	(165.63, 41.39) --
	(165.63, 41.39) --
	(165.70, 41.40) --
	(165.70, 41.40) --
	(165.70, 41.40) --
	(165.76, 41.39) --
	(165.76, 41.39) --
	(165.76, 41.39) --
	(165.78, 41.39) --
	(165.78, 41.39) --
	(165.78, 41.39) --
	(165.85, 41.29) --
	(165.85, 41.29) --
	(165.85, 41.29) --
	(165.93, 41.10) --
	(165.93, 41.10) --
	(165.93, 41.10) --
	(166.00, 40.93) --
	(166.00, 40.93) --
	(166.00, 40.93) --
	(166.00, 40.93) --
	(166.00, 40.93) --
	(166.00, 40.93) --
	(166.08, 39.74) --
	(166.08, 39.74) --
	(166.08, 39.74) --
	(166.15, 38.35) --
	(166.15, 38.35) --
	(166.15, 38.35) --
	(166.16, 38.31) --
	(166.16, 38.31) --
	(166.16, 38.31) --
	(166.23, 38.10) --
	(166.23, 38.10) --
	(166.23, 38.10) --
	(166.30, 39.29) --
	(166.30, 39.29) --
	(166.30, 39.29) --
	(166.38, 40.31) --
	(166.38, 40.31) --
	(166.38, 40.31) --
	(166.44, 39.90) --
	(166.44, 39.90) --
	(166.44, 39.90) --
	(166.45, 39.87) --
	(166.45, 39.87) --
	(166.45, 39.87) --
	(166.48, 39.18) --
	(166.48, 39.18) --
	(166.48, 39.18) --
	(166.52, 38.40) --
	(166.52, 38.40) --
	(166.52, 38.40) --
	(166.60, 35.33) --
	(166.60, 35.33) --
	(166.60, 35.33) --
	(166.67, 33.88) --
	(166.67, 33.88) --
	(166.67, 33.88) --
	(166.75, 34.10) --
	(166.75, 34.10) --
	(166.75, 34.10) --
	(166.81, 34.33) --
	(166.81, 34.33) --
	(166.81, 34.33) --
	(166.82, 34.38) --
	(166.82, 34.38) --
	(166.82, 34.38) --
	(166.90, 34.98) --
	(166.90, 34.98) --
	(166.90, 34.98) --
	(166.97, 35.89) --
	(166.97, 35.89) --
	(166.97, 35.89) --
	(167.05, 37.17) --
	(167.05, 37.17) --
	(167.05, 37.17) --
	(167.12, 37.73) --
	(167.12, 37.73) --
	(167.12, 37.73) --
	(167.17, 36.59) --
	(167.17, 36.59) --
	(167.17, 36.59) --
	(167.20, 36.06) --
	(167.20, 36.06) --
	(167.20, 36.06) --
	(167.27, 33.45) --
	(167.27, 33.45) --
	(167.27, 33.45) --
	(167.34, 33.04) --
	(167.34, 33.04) --
	(167.34, 33.04) --
	(167.42, 33.44) --
	(167.42, 33.44) --
	(167.42, 33.44) --
	(167.49, 34.80) --
	(167.49, 34.80) --
	(167.49, 34.80) --
	(167.57, 36.39) --
	(167.57, 36.39) --
	(167.57, 36.39) --
	(167.64, 36.63) --
	(167.64, 36.63) --
	(167.64, 36.63) --
	(167.66, 36.61) --
	(167.66, 36.61) --
	(167.66, 36.61) --
	(167.72, 36.53) --
	(167.72, 36.53) --
	(167.72, 36.53) --
	(167.79, 34.84) --
	(167.79, 34.84) --
	(167.79, 34.84) --
	(167.83, 34.29) --
	(167.83, 34.29) --
	(167.83, 34.29) --
	(167.87, 33.67) --
	(167.87, 33.67) --
	(167.87, 33.67) --
	(167.94, 34.00) --
	(167.94, 34.00) --
	(167.94, 34.00) --
	(167.98, 34.50) --
	(167.98, 34.50) --
	(167.98, 34.50) --
	(168.01, 34.92) --
	(168.01, 34.92) --
	(168.01, 34.92) --
	(168.09, 37.11) --
	(168.09, 37.11) --
	(168.09, 37.11) --
	(168.16, 40.15) --
	(168.16, 40.15) --
	(168.16, 40.15) --
	(168.24, 41.39) --
	(168.24, 41.39) --
	(168.24, 41.39) --
	(168.30, 42.31) --
	(168.30, 42.31) --
	(168.30, 42.31) --
	(168.31, 42.44) --
	(168.31, 42.44) --
	(168.31, 42.44) --
	(168.34, 42.50) --
	(168.34, 42.50) --
	(168.34, 42.50) --
	(168.39, 42.58) --
	(168.39, 42.58) --
	(168.39, 42.58) --
	(168.46, 41.43) --
	(168.46, 41.43) --
	(168.46, 41.43) --
	(168.51, 40.96) --
	(168.51, 40.96) --
	(168.51, 40.96) --
	(168.54, 40.64) --
	(168.54, 40.64) --
	(168.54, 40.64) --
	(168.55, 40.69) --
	(168.55, 40.69) --
	(168.55, 40.69) --
	(168.61, 40.96) --
	(168.61, 40.96) --
	(168.61, 40.96) --
	(168.63, 40.99) --
	(168.63, 40.99) --
	(168.63, 40.99) --
	(168.67, 41.06) --
	(168.67, 41.06) --
	(168.67, 41.06) --
	(168.69, 41.10) --
	(168.69, 41.10) --
	(168.69, 41.10) --
	(168.71, 41.04) --
	(168.71, 41.04) --
	(168.71, 41.04) --
	(168.75, 40.94) --
	(168.75, 40.94) --
	(168.75, 40.94) --
	(168.76, 40.90) --
	(168.76, 40.90) --
	(168.76, 40.90) --
	(168.83, 40.20) --
	(168.83, 40.20) --
	(168.83, 40.20) --
	(168.87, 39.29) --
	(168.87, 39.29) --
	(168.87, 39.29) --
	(168.91, 38.26) --
	(168.91, 38.26) --
	(168.91, 38.26) --
	(168.98, 34.93) --
	(168.98, 34.93) --
	(168.98, 34.93) --
	(169.06, 32.40) --
	(169.06, 32.40) --
	(169.06, 32.40) --
	(169.13, 31.63) --
	(169.13, 31.63) --
	(169.13, 31.63) --
	(169.19, 31.67) --
	(169.19, 31.67) --
	(169.19, 31.67) --
	(169.21, 31.68) --
	(169.21, 31.68) --
	(169.21, 31.68) --
	(169.28, 31.89) --
	(169.28, 31.89) --
	(169.28, 31.89) --
	(169.36, 32.70) --
	(169.36, 32.70) --
	(169.36, 32.70) --
	(169.36, 32.74) --
	(169.36, 32.74) --
	(169.36, 32.74) --
	(169.43, 33.79) --
	(169.43, 33.79) --
	(169.43, 33.79) --
	(169.51, 35.03) --
	(169.51, 35.03) --
	(169.51, 35.03) --
	(169.58, 36.25) --
	(169.58, 36.25) --
	(169.58, 36.25) --
	(169.65, 37.06) --
	(169.65, 37.06) --
	(169.65, 37.06) --
	(169.73, 37.76) --
	(169.73, 37.76) --
	(169.73, 37.76) --
	(169.76, 37.95) --
	(169.76, 37.95) --
	(169.76, 37.95) --
	(169.80, 38.20) --
	(169.80, 38.20) --
	(169.80, 38.20) --
	(169.88, 38.05) --
	(169.88, 38.05) --
	(169.88, 38.05) --
	(169.95, 38.22) --
	(169.95, 38.22) --
	(169.95, 38.22) --
	(170.03, 36.41) --
	(170.03, 36.41) --
	(170.03, 36.41) --
	(170.10, 34.30) --
	(170.10, 34.30) --
	(170.10, 34.30) --
	(170.13, 33.99) --
	(170.13, 33.99) --
	(170.13, 33.99) --
	(170.17, 33.44) --
	(170.17, 33.44) --
	(170.17, 33.44) --
	(170.25, 32.63) --
	(170.25, 32.63) --
	(170.25, 32.63) --
	(170.32, 32.88) --
	(170.32, 32.88) --
	(170.32, 32.88) --
	(170.40, 34.29) --
	(170.40, 34.29) --
	(170.40, 34.29) --
	(170.47, 35.77) --
	(170.47, 35.77) --
	(170.47, 35.77) --
	(170.51, 36.41) --
	(170.51, 36.41) --
	(170.51, 36.41) --
	(170.55, 37.05) --
	(170.55, 37.05) --
	(170.55, 37.05) --
	(170.62, 37.93) --
	(170.62, 37.93) --
	(170.62, 37.93) --
	(170.70, 38.37) --
	(170.70, 38.37) --
	(170.70, 38.37) --
	(170.77, 38.74) --
	(170.77, 38.74) --
	(170.77, 38.74) --
	(170.84, 39.45) --
	(170.84, 39.45) --
	(170.84, 39.45) --
	(170.89, 38.82) --
	(170.89, 38.82) --
	(170.89, 38.82) --
	(170.92, 38.50) --
	(170.92, 38.50) --
	(170.92, 38.50) --
	(170.99, 36.19) --
	(170.99, 36.19) --
	(170.99, 36.19) --
	(171.05, 35.53) --
	(171.05, 35.53) --
	(171.05, 35.53) --
	(171.07, 35.38) --
	(171.07, 35.38) --
	(171.07, 35.38) --
	(171.14, 35.78) --
	(171.14, 35.78) --
	(171.14, 35.78) --
	(171.22, 36.75) --
	(171.22, 36.75) --
	(171.22, 36.75) --
	(171.29, 37.56) --
	(171.29, 37.56) --
	(171.29, 37.56) --
	(171.36, 37.85) --
	(171.36, 37.85) --
	(171.36, 37.85) --
	(171.37, 37.89) --
	(171.37, 37.89) --
	(171.37, 37.89) --
	(171.44, 38.18) --
	(171.44, 38.18) --
	(171.44, 38.18) --
	(171.51, 38.14) --
	(171.51, 38.14) --
	(171.51, 38.14) --
	(171.59, 38.99) --
	(171.59, 38.99) --
	(171.59, 38.99) --
	(171.66, 40.12) --
	(171.66, 40.12) --
	(171.66, 40.12) --
	(171.66, 40.14) --
	(171.66, 40.14) --
	(171.66, 40.14) --
	(171.74, 40.09) --
	(171.74, 40.09) --
	(171.74, 40.09) --
	(171.81, 40.29) --
	(171.81, 40.29) --
	(171.81, 40.29) --
	(171.85, 40.65) --
	(171.85, 40.65) --
	(171.85, 40.65) --
	(171.88, 40.94) --
	(171.88, 40.94) --
	(171.88, 40.94) --
	(171.96, 41.16) --
	(171.96, 41.16) --
	(171.96, 41.16) --
	(172.03, 41.17) --
	(172.03, 41.17) --
	(172.03, 41.17) --
	(172.11, 39.41) --
	(172.11, 39.41) --
	(172.11, 39.41) --
	(172.14, 38.57) --
	(172.14, 38.57) --
	(172.14, 38.57) --
	(172.18, 37.47) --
	(172.18, 37.47) --
	(172.18, 37.47) --
	(172.26, 37.80) --
	(172.26, 37.80) --
	(172.26, 37.80) --
	(172.33, 38.64) --
	(172.33, 38.64) --
	(172.33, 38.64) --
	(172.33, 38.66) --
	(172.33, 38.66) --
	(172.33, 38.66) --
	(172.40, 39.92) --
	(172.40, 39.92) --
	(172.40, 39.92) --
	(172.43, 40.20) --
	(172.43, 40.20) --
	(172.43, 40.20) --
	(172.48, 40.84) --
	(172.48, 40.84) --
	(172.48, 40.84) --
	(172.55, 41.28) --
	(172.55, 41.28) --
	(172.55, 41.28) --
	(172.63, 41.31) --
	(172.63, 41.31) --
	(172.63, 41.30) --
	(172.70, 40.17) --
	(172.70, 40.17) --
	(172.70, 40.17) --
	(172.78, 40.68) --
	(172.78, 40.68) --
	(172.78, 40.68) --
	(172.81, 40.96) --
	(172.81, 40.96) --
	(172.81, 40.96) --
	(172.85, 41.28) --
	(172.85, 41.28) --
	(172.85, 41.28) --
	(172.92, 41.31) --
	(172.92, 41.31) --
	(172.92, 41.31) --
	(173.00, 40.43) --
	(173.00, 40.43) --
	(173.00, 40.43) --
	(173.00, 40.43) --
	(173.00, 40.43) --
	(173.00, 40.43) --
	(173.07, 40.53) --
	(173.07, 40.53) --
	(173.07, 40.53) --
	(173.15, 40.96) --
	(173.15, 40.96) --
	(173.15, 40.96) --
	(173.22, 41.69) --
	(173.22, 41.69) --
	(173.22, 41.69) --
	(173.29, 42.71) --
	(173.29, 42.71) --
	(173.29, 42.71) --
	(173.29, 42.80) --
	(173.29, 42.80) --
	(173.29, 42.80) --
	(173.37, 42.80) --
	(173.37, 42.80) --
	(173.37, 42.80) --
	(173.44, 42.80) --
	(173.44, 42.80) --
	(173.44, 42.80) --
	(173.48, 42.55) --
	(173.48, 42.55) --
	(173.48, 42.55) --
	(173.52, 42.29) --
	(173.52, 42.29) --
	(173.52, 42.29) --
	(173.58, 41.65) --
	(173.58, 41.65) --
	(173.58, 41.65) --
	(173.59, 41.48) --
	(173.59, 41.48) --
	(173.59, 41.48) --
	(173.67, 41.83) --
	(173.67, 41.83) --
	(173.67, 41.83) --
	(173.74, 42.42) --
	(173.74, 42.42) --
	(173.74, 42.42) --
	(173.77, 42.52) --
	(173.77, 42.52) --
	(173.77, 42.52) --
	(173.81, 42.70) --
	(173.81, 42.70) --
	(173.81, 42.70) --
	(173.86, 42.44) --
	(173.86, 42.44) --
	(173.86, 42.44) --
	(173.89, 42.31) --
	(173.89, 42.31) --
	(173.89, 42.31) --
	(173.96, 41.61) --
	(173.96, 41.61) --
	(173.96, 41.61) --
	(174.04, 41.85) --
	(174.04, 41.85) --
	(174.04, 41.86) --
	(174.11, 42.36) --
	(174.11, 42.36) --
	(174.11, 42.36) --
	(174.15, 42.48) --
	(174.15, 42.48) --
	(174.15, 42.48) --
	(174.18, 42.58) --
	(174.18, 42.58) --
	(174.18, 42.58) --
	(174.25, 42.27) --
	(174.25, 42.27) --
	(174.25, 42.27) --
	(174.26, 42.21) --
	(174.26, 42.21) --
	(174.26, 42.21) --
	(174.33, 42.09) --
	(174.33, 42.09) --
	(174.33, 42.09) --
	(174.34, 42.06) --
	(174.34, 42.06) --
	(174.34, 42.06) --
	(174.41, 41.82) --
	(174.41, 41.82) --
	(174.41, 41.82) --
	(174.44, 41.77) --
	(174.44, 41.77) --
	(174.44, 41.77) --
	(174.48, 41.69) --
	(174.48, 41.69) --
	(174.48, 41.69) --
	(174.53, 41.69) --
	(174.53, 41.69) --
	(174.53, 41.69) --
	(174.56, 41.69) --
	(174.56, 41.69) --
	(174.56, 41.69) --
	(174.63, 41.63) --
	(174.63, 41.63) --
	(174.63, 41.63) --
	(174.63, 41.63) --
	(174.63, 41.63) --
	(174.63, 41.63) --
	(174.70, 41.30) --
	(174.70, 41.30) --
	(174.70, 41.30) --
	(174.78, 40.87) --
	(174.78, 40.87) --
	(174.78, 40.87) --
	(174.82, 40.36) --
	(174.82, 40.36) --
	(174.82, 40.36) --
	(174.85, 40.02) --
	(174.85, 40.02) --
	(174.85, 40.02) --
	(174.92, 39.15) --
	(174.92, 39.15) --
	(174.92, 39.15) --
	(174.93, 39.05) --
	(174.93, 39.05) --
	(174.93, 39.05) --
	(175.00, 37.85) --
	(175.00, 37.85) --
	(175.00, 37.85) --
	(175.01, 37.73) --
	(175.01, 37.73) --
	(175.01, 37.73) --
	(175.07, 37.18) --
	(175.07, 37.18) --
	(175.07, 37.18) --
	(175.15, 35.38) --
	(175.15, 35.38) --
	(175.15, 35.38) --
	(175.22, 33.84) --
	(175.22, 33.84) --
	(175.22, 33.84) --
	(175.30, 32.92) --
	(175.30, 32.92) --
	(175.30, 32.92) --
	(175.30, 32.91) --
	(175.30, 32.91) --
	(175.30, 32.91) --
	(175.37, 32.59) --
	(175.37, 32.59) --
	(175.37, 32.59) --
	(175.45, 31.74) --
	(175.45, 31.74) --
	(175.45, 31.74) --
	(175.52, 30.97) --
	(175.52, 30.97) --
	(175.52, 30.97) --
	(175.59, 30.71) --
	(175.59, 30.71) --
	(175.59, 30.71) --
	(175.59, 30.70) --
	(175.59, 30.70) --
	(175.59, 30.70) --
	(175.67, 29.68) --
	(175.67, 29.68) --
	(175.67, 29.68) --
	(175.68, 29.47) --
	(175.68, 29.47) --
	(175.68, 29.47) --
	(175.74, 28.81) --
	(175.74, 28.81) --
	(175.74, 28.81) --
	(175.78, 28.66) --
	(175.78, 28.66) --
	(175.78, 28.66) --
	(175.82, 28.53) --
	(175.82, 28.53) --
	(175.82, 28.53) --
	(175.89, 28.48) --
	(175.89, 28.48) --
	(175.89, 28.48) --
	(175.96, 28.83) --
	(175.96, 28.83) --
	(175.96, 28.83) --
	(175.97, 28.94) --
	(175.97, 28.94) --
	(175.97, 28.94) --
	(176.04, 29.84) --
	(176.04, 29.84) --
	(176.04, 29.84) --
	(176.07, 30.40) --
	(176.07, 30.40) --
	(176.07, 30.40) --
	(176.11, 31.23) --
	(176.11, 31.23) --
	(176.11, 31.23) --
	(176.16, 31.27) --
	(176.16, 31.27) --
	(176.16, 31.27) --
	(176.19, 31.28) --
	(176.19, 31.28) --
	(176.19, 31.28) --
	(176.26, 29.42) --
	(176.26, 29.42) --
	(176.26, 29.42) --
	(176.33, 28.10) --
	(176.33, 28.10) --
	(176.33, 28.10) --
	(176.36, 28.06) --
	(176.36, 28.06) --
	(176.36, 28.06) --
	(176.41, 27.96) --
	(176.41, 27.96) --
	(176.41, 27.96) --
	(176.45, 27.95) --
	(176.45, 27.95) --
	(176.45, 27.95) --
	(176.48, 27.94) --
	(176.48, 27.94) --
	(176.48, 27.94) --
	(176.56, 27.92) --
	(176.56, 27.92) --
	(176.56, 27.92) --
	(176.63, 27.80) --
	(176.63, 27.80) --
	(176.63, 27.80) --
	(176.64, 27.78) --
	(176.64, 27.78) --
	(176.64, 27.78) --
	(176.70, 27.71) --
	(176.70, 27.71) --
	(176.70, 27.71) --
	(176.74, 27.66) --
	(176.74, 27.66) --
	(176.74, 27.66) --
	(176.78, 27.61) --
	(176.78, 27.61) --
	(176.78, 27.61) --
	(176.83, 27.58) --
	(176.83, 27.58) --
	(176.83, 27.58) --
	(176.85, 27.57) --
	(176.85, 27.57) --
	(176.85, 27.57) --
	(176.93, 27.62) --
	(176.93, 27.62) --
	(176.93, 27.62) --
	(177.00, 27.78) --
	(177.00, 27.78) --
	(177.00, 27.78) --
	(177.03, 27.81) --
	(177.03, 27.81) --
	(177.03, 27.81) --
	(177.07, 27.87) --
	(177.07, 27.87) --
	(177.07, 27.87) --
	(177.12, 27.91) --
	(177.12, 27.91) --
	(177.12, 27.91) --
	(177.15, 27.94) --
	(177.15, 27.94) --
	(177.15, 27.94) --
	(177.22, 28.20) --
	(177.22, 28.20) --
	(177.22, 28.20) --
	(177.30, 28.35) --
	(177.30, 28.35) --
	(177.30, 28.35) --
	(177.31, 28.37) --
	(177.31, 28.37) --
	(177.31, 28.37) --
	(177.37, 28.41) --
	(177.37, 28.41) --
	(177.37, 28.41) --
	(177.44, 28.14) --
	(177.44, 28.14) --
	(177.44, 28.14) --
	(177.51, 27.86) --
	(177.51, 27.86) --
	(177.51, 27.86) --
	(177.52, 27.80) --
	(177.52, 27.80) --
	(177.52, 27.80) --
	(177.59, 27.69) --
	(177.59, 27.69) --
	(177.59, 27.69) --
	(177.60, 27.69) --
	(177.60, 27.69) --
	(177.60, 27.69) --
	(177.67, 27.69) --
	(177.67, 27.69) --
	(177.67, 27.69) --
	(177.74, 27.77) --
	(177.74, 27.77) --
	(177.74, 27.77) --
	(177.79, 27.85) --
	(177.79, 27.85) --
	(177.79, 27.85) --
	(177.81, 27.88) --
	(177.81, 27.88) --
	(177.81, 27.88) --
	(177.89, 27.96) --
	(177.89, 27.96) --
	(177.89, 27.96) --
	(177.89, 27.96) --
	(177.89, 27.96) --
	(177.89, 27.96) --
	(177.96, 27.97) --
	(177.96, 27.97) --
	(177.96, 27.97) --
	(178.04, 27.97) --
	(178.04, 27.97) --
	(178.04, 27.97) --
	(178.11, 28.09) --
	(178.11, 28.09) --
	(178.11, 28.09) --
	(178.18, 28.07) --
	(178.18, 28.07) --
	(178.18, 28.07) --
	(178.18, 28.07) --
	(178.18, 28.07) --
	(178.18, 28.07) --
	(178.26, 28.04) --
	(178.26, 28.04) --
	(178.26, 28.04) --
	(178.33, 28.14) --
	(178.33, 28.14) --
	(178.33, 28.14) --
	(178.37, 28.26) --
	(178.37, 28.26) --
	(178.37, 28.26) --
	(178.41, 28.38) --
	(178.41, 28.38) --
	(178.41, 28.38) --
	(178.48, 28.34) --
	(178.48, 28.34) --
	(178.48, 28.34) --
	(178.55, 28.19) --
	(178.55, 28.19) --
	(178.55, 28.19) --
	(178.56, 28.18) --
	(178.56, 28.18) --
	(178.56, 28.18) --
	(178.63, 28.05) --
	(178.63, 28.05) --
	(178.63, 28.05) --
	(178.70, 27.92) --
	(178.70, 27.92) --
	(178.70, 27.92) --
	(178.75, 27.95) --
	(178.75, 27.95) --
	(178.75, 27.95) --
	(178.78, 27.96) --
	(178.78, 27.96) --
	(178.78, 27.96) --
	(178.85, 27.97) --
	(178.85, 27.97) --
	(178.85, 27.97) --
	(178.92, 27.98) --
	(178.92, 27.98) --
	(178.92, 27.98) --
	(178.94, 28.05) --
	(178.94, 28.05) --
	(178.94, 28.05) --
	(179.00, 28.22) --
	(179.00, 28.22) --
	(179.00, 28.22) --
	(179.07, 28.36) --
	(179.07, 28.36) --
	(179.07, 28.36) --
	(179.14, 28.00) --
	(179.14, 28.00) --
	(179.14, 28.00) --
	(179.22, 27.85) --
	(179.22, 27.85) --
	(179.22, 27.85) --
	(179.29, 27.96) --
	(179.29, 27.96) --
	(179.29, 27.96) --
	(179.33, 28.14) --
	(179.33, 28.14) --
	(179.33, 28.14) --
	(179.37, 28.35) --
	(179.37, 28.35) --
	(179.37, 28.35) --
	(179.44, 28.70) --
	(179.44, 28.70) --
	(179.44, 28.70) --
	(179.51, 28.94) --
	(179.51, 28.94) --
	(179.51, 28.94) --
	(179.59, 28.57) --
	(179.59, 28.57) --
	(179.59, 28.57) --
	(179.66, 28.24) --
	(179.66, 28.24) --
	(179.66, 28.24) --
	(179.71, 28.12) --
	(179.71, 28.12) --
	(179.71, 28.12) --
	(179.73, 28.06) --
	(179.73, 28.06) --
	(179.73, 28.06) --
	(179.81, 27.98) --
	(179.81, 27.98) --
	(179.81, 27.98) --
	(179.81, 27.98) --
	(179.81, 27.98) --
	(179.81, 27.98) --
	(179.88, 28.04) --
	(179.88, 28.04) --
	(179.88, 28.04) --
	(179.96, 28.11) --
	(179.96, 28.11) --
	(179.96, 28.11) --
	(180.03, 28.22) --
	(180.03, 28.22) --
	(180.03, 28.22) --
	(180.10, 28.21) --
	(180.10, 28.21) --
	(180.10, 28.21) --
	(180.18, 28.15) --
	(180.18, 28.15) --
	(180.18, 28.15) --
	(180.19, 28.17) --
	(180.19, 28.17) --
	(180.19, 28.17) --
	(180.25, 28.30) --
	(180.25, 28.30) --
	(180.25, 28.30) --
	(180.33, 28.55) --
	(180.33, 28.55) --
	(180.33, 28.55) --
	(180.40, 28.49) --
	(180.40, 28.49) --
	(180.40, 28.49) --
	(180.47, 28.21) --
	(180.47, 28.21) --
	(180.47, 28.21) --
	(180.55, 28.14) --
	(180.55, 28.14) --
	(180.55, 28.14) --
	(180.57, 28.13) --
	(180.57, 28.13) --
	(180.57, 28.13) --
	(180.62, 28.11) --
	(180.62, 28.11) --
	(180.62, 28.11) --
	(180.70, 28.15) --
	(180.70, 28.15) --
	(180.70, 28.15) --
	(180.77, 28.14) --
	(180.77, 28.14) --
	(180.77, 28.14) --
	(180.84, 28.06) --
	(180.84, 28.06) --
	(180.84, 28.06) --
	(180.92, 28.05) --
	(180.92, 28.05) --
	(180.92, 28.05) --
	(180.99, 28.16) --
	(180.99, 28.16) --
	(180.99, 28.16) --
	(181.05, 28.36) --
	(181.05, 28.36) --
	(181.05, 28.36) --
	(181.06, 28.40) --
	(181.06, 28.40) --
	(181.06, 28.40) --
	(181.14, 28.78) --
	(181.14, 28.78) --
	(181.14, 28.78) --
	(181.21, 29.10) --
	(181.21, 29.10) --
	(181.21, 29.10) --
	(181.28, 29.18) --
	(181.28, 29.18) --
	(181.28, 29.18) --
	(181.36, 29.68) --
	(181.36, 29.68) --
	(181.36, 29.68) --
	(181.43, 31.30) --
	(181.43, 31.30) --
	(181.43, 31.30) --
	(181.51, 33.20) --
	(181.51, 33.20) --
	(181.51, 33.20) --
	(181.53, 33.74) --
	(181.53, 33.74) --
	(181.53, 33.74) --
	(181.58, 34.86) --
	(181.58, 34.86) --
	(181.58, 34.86) --
	(181.65, 37.59) --
	(181.65, 37.59) --
	(181.65, 37.59) --
	(181.73, 37.84) --
	(181.73, 37.84) --
	(181.73, 37.84) --
	(181.80, 35.66) --
	(181.80, 35.66) --
	(181.80, 35.66) --
	(181.87, 35.90) --
	(181.87, 35.90) --
	(181.87, 35.90) --
	(181.95, 38.83) --
	(181.95, 38.83) --
	(181.95, 38.83) --
	(182.01, 39.82) --
	(182.01, 39.82) --
	(182.01, 39.82) --
	(182.02, 40.04) --
	(182.02, 40.04) --
	(182.02, 40.04) --
	(182.10, 37.40) --
	(182.10, 37.40) --
	(182.10, 37.40) --
	(182.17, 34.04) --
	(182.17, 34.04) --
	(182.17, 34.04) --
	(182.24, 34.80) --
	(182.24, 34.80) --
	(182.24, 34.80) --
	(182.32, 37.00) --
	(182.32, 37.00) --
	(182.32, 37.00) --
	(182.39, 39.90) --
	(182.39, 39.90) --
	(182.39, 39.90) --
	(182.46, 40.88) --
	(182.46, 40.88) --
	(182.46, 40.88) --
	(182.54, 40.16) --
	(182.54, 40.16) --
	(182.54, 40.16) --
	(182.58, 37.45) --
	(182.58, 37.45) --
	(182.58, 37.45) --
	(182.61, 35.83) --
	(182.61, 35.83) --
	(182.61, 35.83) --
	(182.68, 32.81) --
	(182.69, 32.81) --
	(182.69, 32.81) --
	(182.76, 33.86) --
	(182.76, 33.86) --
	(182.76, 33.86) --
	(182.83, 34.35) --
	(182.83, 34.35) --
	(182.83, 34.35) --
	(182.91, 35.66) --
	(182.91, 35.66) --
	(182.91, 35.66) --
	(182.98, 39.52) --
	(182.98, 39.52) --
	(182.98, 39.52) --
	(183.05, 41.20) --
	(183.05, 41.20) --
	(183.05, 41.20) --
	(183.13, 40.95) --
	(183.13, 40.95) --
	(183.13, 40.95) --
	(183.20, 37.20) --
	(183.20, 37.20) --
	(183.20, 37.20) --
	(183.27, 32.69) --
	(183.27, 32.69) --
	(183.27, 32.69) --
	(183.35, 31.56) --
	(183.35, 31.56) --
	(183.35, 31.56) --
	(183.35, 31.59) --
	(183.35, 31.59) --
	(183.35, 31.59) --
	(183.42, 32.53) --
	(183.42, 32.53) --
	(183.42, 32.53) --
	(183.50, 34.39) --
	(183.50, 34.39) --
	(183.50, 34.39) --
	(183.57, 33.64) --
	(183.57, 33.64) --
	(183.57, 33.64) --
	(183.64, 32.43) --
	(183.64, 32.43) --
	(183.64, 32.43) --
	(183.72, 32.92) --
	(183.72, 32.92) --
	(183.72, 32.92) --
	(183.79, 35.49) --
	(183.79, 35.49) --
	(183.79, 35.49) --
	(183.86, 37.25) --
	(183.86, 37.25) --
	(183.86, 37.25) --
	(183.94, 36.45) --
	(183.94, 36.45) --
	(183.94, 36.45) --
	(184.01, 33.64) --
	(184.01, 33.64) --
	(184.01, 33.64) --
	(184.09, 31.44) --
	(184.09, 31.44) --
	(184.09, 31.44) --
	(184.12, 31.21) --
	(184.12, 31.21) --
	(184.12, 31.21) --
	(184.16, 30.93) --
	(184.16, 30.93) --
	(184.16, 30.93) --
	(184.23, 30.73) --
	(184.23, 30.73) --
	(184.23, 30.73) --
	(184.30, 30.62) --
	(184.30, 30.62) --
	(184.30, 30.62) --
	(184.38, 32.21) --
	(184.38, 32.21) --
	(184.38, 32.21) --
	(184.41, 32.79) --
	(184.41, 32.79) --
	(184.41, 32.79) --
	(184.45, 33.81) --
	(184.45, 33.81) --
	(184.45, 33.81) --
	(184.53, 33.99) --
	(184.53, 33.99) --
	(184.53, 33.99) --
	(184.60, 33.74) --
	(184.60, 33.74) --
	(184.60, 33.74) --
	(184.67, 31.39) --
	(184.67, 31.39) --
	(184.67, 31.39) --
	(184.75, 30.41) --
	(184.75, 30.41) --
	(184.75, 30.41) --
	(184.82, 31.41) --
	(184.82, 31.41) --
	(184.82, 31.41) --
	(184.89, 32.16) --
	(184.89, 32.16) --
	(184.89, 32.16) --
	(184.97, 32.38) --
	(184.97, 32.38) --
	(184.97, 32.38) --
	(184.98, 32.43) --
	(184.98, 32.43) --
	(184.98, 32.43) --
	(185.04, 32.64) --
	(185.04, 32.64) --
	(185.04, 32.64) --
	(185.11, 31.80) --
	(185.11, 31.80) --
	(185.11, 31.80) --
	(185.19, 30.49) --
	(185.19, 30.49) --
	(185.19, 30.49) --
	(185.26, 30.46) --
	(185.26, 30.46) --
	(185.26, 30.46) --
	(185.27, 30.54) --
	(185.27, 30.54) --
	(185.27, 30.54) --
	(185.33, 31.30) --
	(185.33, 31.30) --
	(185.33, 31.30) --
	(185.41, 31.75) --
	(185.41, 31.75) --
	(185.41, 31.75) --
	(185.48, 32.90) --
	(185.48, 32.90) --
	(185.48, 32.90) --
	(185.55, 33.23) --
	(185.55, 33.23) --
	(185.55, 33.23) --
	(185.63, 31.54) --
	(185.63, 31.54) --
	(185.63, 31.54) --
	(185.70, 31.77) --
	(185.70, 31.77) --
	(185.70, 31.77) --
	(185.75, 32.80) --
	(185.75, 32.80) --
	(185.75, 32.80) --
	(185.78, 33.46) --
	(185.78, 33.46) --
	(185.78, 33.46) --
	(185.85, 34.38) --
	(185.85, 34.38) --
	(185.85, 34.38) --
	(185.92, 36.31) --
	(185.92, 36.31) --
	(185.92, 36.31) --
	(186.00, 37.97) --
	(186.00, 37.97) --
	(186.00, 37.97) --
	(186.03, 37.72) --
	(186.03, 37.72) --
	(186.03, 37.72) --
	(186.07, 37.48) --
	(186.07, 37.48) --
	(186.07, 37.48) --
	(186.14, 34.44) --
	(186.14, 34.44) --
	(186.14, 34.44) --
	(186.22, 33.95) --
	(186.22, 33.95) --
	(186.22, 33.95) --
	(186.29, 34.77) --
	(186.29, 34.77) --
	(186.29, 34.77) --
	(186.36, 38.49) --
	(186.36, 38.49) --
	(186.36, 38.49) --
	(186.44, 41.51) --
	(186.44, 41.51) --
	(186.44, 41.51) --
	(186.51, 41.67) --
	(186.51, 41.67) --
	(186.51, 41.67) --
	(186.58, 38.75) --
	(186.58, 38.75) --
	(186.58, 38.75) --
	(186.66, 33.03) --
	(186.66, 33.03) --
	(186.66, 33.03) --
	(186.71, 32.86) --
	(186.71, 32.86) --
	(186.71, 32.86) --
	(186.73, 32.77) --
	(186.73, 32.77) --
	(186.73, 32.77) --
	(186.80, 36.71) --
	(186.80, 36.71) --
	(186.80, 36.71) --
	(186.88, 38.98) --
	(186.88, 38.98) --
	(186.88, 38.98) --
	(186.95, 38.02) --
	(186.95, 38.02) --
	(186.95, 38.02) --
	(187.02, 35.64) --
	(187.02, 35.64) --
	(187.02, 35.64) --
	(187.10, 35.63) --
	(187.10, 35.63) --
	(187.10, 35.63) --
	(187.17, 35.94) --
	(187.17, 35.94) --
	(187.17, 35.94) --
	(187.18, 36.21) --
	(187.18, 36.21) --
	(187.18, 36.21) --
	(187.25, 37.42) --
	(187.25, 37.42) --
	(187.25, 37.42) --
	(187.32, 38.44) --
	(187.32, 38.44) --
	(187.32, 38.44) --
	(187.39, 41.04) --
	(187.39, 41.04) --
	(187.39, 41.04) --
	(187.46, 41.44) --
	(187.46, 41.44) --
	(187.46, 41.44) --
	(187.54, 40.85) --
	(187.54, 40.85) --
	(187.54, 40.85) --
	(187.61, 37.58) --
	(187.61, 37.58) --
	(187.61, 37.58) --
	(187.68, 32.91) --
	(187.68, 32.91) --
	(187.68, 32.91) --
	(187.76, 33.19) --
	(187.76, 33.19) --
	(187.76, 33.19) --
	(187.76, 33.21) --
	(187.76, 33.21) --
	(187.76, 33.21) --
	(187.83, 34.84) --
	(187.83, 34.84) --
	(187.83, 34.84) --
	(187.91, 36.96) --
	(187.91, 36.96) --
	(187.91, 36.96) --
	(187.98, 40.13) --
	(187.98, 40.13) --
	(187.98, 40.13) --
	(188.05, 42.59) --
	(188.05, 42.59) --
	(188.05, 42.59) --
	(188.13, 42.80) --
	(188.13, 42.80) --
	(188.13, 42.80) --
	(188.20, 42.38) --
	(188.20, 42.38) --
	(188.20, 42.38) --
	(188.27, 38.97) --
	(188.27, 38.97) --
	(188.27, 38.97) --
	(188.33, 36.82) --
	(188.33, 36.82) --
	(188.33, 36.82) --
	(188.34, 36.46) --
	(188.34, 36.46) --
	(188.34, 36.46) --
	(188.42, 33.37) --
	(188.42, 33.36) --
	(188.42, 33.36) --
	(188.49, 31.05) --
	(188.49, 31.05) --
	(188.49, 31.05) --
	(188.56, 31.36) --
	(188.56, 31.36) --
	(188.56, 31.36) --
	(188.64, 32.68) --
	(188.64, 32.68) --
	(188.64, 32.68) --
	(188.71, 33.37) --
	(188.71, 33.37) --
	(188.71, 33.37) --
	(188.79, 34.33) --
	(188.79, 34.33) --
	(188.79, 34.33) --
	(188.86, 37.24) --
	(188.86, 37.24) --
	(188.86, 37.24) --
	(188.93, 40.80) --
	(188.93, 40.80) --
	(188.93, 40.80) --
	(189.01, 42.73) --
	(189.01, 42.73) --
	(189.01, 42.73) --
	(189.08, 42.80) --
	(189.08, 42.80) --
	(189.08, 42.80) --
	(189.15, 42.80) --
	(189.15, 42.80) --
	(189.15, 42.80) --
	(189.22, 42.80) --
	(189.22, 42.80) --
	(189.22, 42.80) --
	(189.29, 42.80) --
	(189.29, 42.80) --
	(189.29, 42.80) --
	(189.30, 42.80) --
	(189.30, 42.80) --
	(189.30, 42.80) --
	(189.37, 42.80) --
	(189.37, 42.80) --
	(189.37, 42.80) --
	(189.44, 42.80) --
	(189.44, 42.80) --
	(189.44, 42.80) --
	(189.52, 42.80) --
	(189.52, 42.80) --
	(189.52, 42.80) --
	(189.59, 42.80) --
	(189.59, 42.80) --
	(189.59, 42.80) --
	(189.66, 42.80) --
	(189.66, 42.80) --
	(189.66, 42.80) --
	(189.74, 42.80) --
	(189.74, 42.80) --
	(189.74, 42.80) --
	(189.81, 42.80) --
	(189.81, 42.80) --
	(189.81, 42.80) --
	(189.88, 41.91) --
	(189.88, 41.91) --
	(189.88, 41.91) --
	(189.96, 41.56) --
	(189.96, 41.56) --
	(189.96, 41.56) --
	(190.03, 41.35) --
	(190.03, 41.35) --
	(190.03, 41.35) --
	(190.10, 41.48) --
	(190.10, 41.48) --
	(190.10, 41.48) --
	(190.18, 42.07) --
	(190.18, 42.07) --
	(190.18, 42.07) --
	(190.25, 41.99) --
	(190.25, 41.99) --
	(190.25, 41.99) --
	(190.25, 41.99) --
	(190.25, 41.99) --
	(190.25, 41.99) --
	(190.32, 41.88) --
	(190.32, 41.88) --
	(190.32, 41.88) --
	(190.40, 41.07) --
	(190.40, 41.07) --
	(190.40, 41.07) --
	(190.47, 38.82) --
	(190.47, 38.82) --
	(190.47, 38.82) --
	(190.54, 38.73) --
	(190.54, 38.73) --
	(190.54, 38.73) --
	(190.62, 40.59) --
	(190.62, 40.59) --
	(190.62, 40.59) --
	(190.69, 41.81) --
	(190.69, 41.81) --
	(190.69, 41.81) --
	(190.76, 42.09) --
	(190.76, 42.09) --
	(190.76, 42.09) --
	(190.84, 41.50) --
	(190.84, 41.50) --
	(190.84, 41.50) --
	(190.91, 41.80) --
	(190.91, 41.80) --
	(190.91, 41.80) --
	(190.98, 42.80) --
	(190.98, 42.80) --
	(190.98, 42.80) --
	(191.06, 42.80) --
	(191.06, 42.80) --
	(191.06, 42.80) --
	(191.13, 42.80) --
	(191.13, 42.80) --
	(191.13, 42.80) --
	(191.20, 42.80) --
	(191.20, 42.80) --
	(191.20, 42.80) --
	(191.27, 42.80) --
	(191.27, 42.80) --
	(191.27, 42.80) --
	(191.35, 42.80) --
	(191.35, 42.80) --
	(191.35, 42.80) --
	(191.42, 40.87) --
	(191.42, 40.87) --
	(191.42, 40.87) --
	(191.49, 40.35) --
	(191.49, 40.35) --
	(191.49, 40.34) --
	(191.57, 39.05) --
	(191.57, 39.05) --
	(191.57, 39.05) --
	(191.64, 38.47) --
	(191.64, 38.47) --
	(191.64, 38.47) --
	(191.71, 36.81) --
	(191.71, 36.81) --
	(191.71, 36.81) --
	(191.79, 36.74) --
	(191.79, 36.74) --
	(191.79, 36.74) --
	(191.86, 36.15) --
	(191.86, 36.15) --
	(191.86, 36.15) --
	(191.93, 34.57) --
	(191.93, 34.57) --
	(191.93, 34.57) --
	(192.01, 32.74) --
	(192.01, 32.74) --
	(192.01, 32.74) --
	(192.07, 31.09) --
	(192.07, 31.08) --
	(192.07, 31.08) --
	(192.08, 30.89) --
	(192.08, 30.89) --
	(192.08, 30.89) --
	(192.15, 30.37) --
	(192.15, 30.37) --
	(192.15, 30.37) --
	(192.23, 30.46) --
	(192.23, 30.46) --
	(192.23, 30.46) --
	(192.30, 30.63) --
	(192.30, 30.63) --
	(192.30, 30.63) --
	(192.37, 30.46) --
	(192.37, 30.46) --
	(192.37, 30.46) --
	(192.44, 30.11) --
	(192.44, 30.11) --
	(192.44, 30.11) --
	(192.52, 29.91) --
	(192.52, 29.91) --
	(192.52, 29.91) --
	(192.59, 29.89) --
	(192.59, 29.89) --
	(192.59, 29.89) --
	(192.66, 29.96) --
	(192.66, 29.96) --
	(192.66, 29.96) --
	(192.74, 30.27) --
	(192.74, 30.27) --
	(192.74, 30.27) --
	(192.81, 30.88) --
	(192.81, 30.88) --
	(192.81, 30.88) --
	(192.88, 32.04) --
	(192.88, 32.04) --
	(192.88, 32.04) --
	(192.96, 34.58) --
	(192.96, 34.58) --
	(192.96, 34.58) --
	(193.03, 35.93) --
	(193.03, 35.93) --
	(193.03, 35.93) --
	(193.10, 33.65) --
	(193.10, 33.65) --
	(193.10, 33.65) --
	(193.17, 33.23) --
	(193.17, 33.23) --
	(193.17, 33.23) --
	(193.25, 35.49) --
	(193.25, 35.49) --
	(193.25, 35.49) --
	(193.32, 33.15) --
	(193.32, 33.15) --
	(193.32, 33.15) --
	(193.39, 30.75) --
	(193.39, 30.75) --
	(193.39, 30.75) --
	(193.41, 30.86) --
	(193.41, 30.86) --
	(193.41, 30.86) --
	(193.47, 31.18) --
	(193.47, 31.18) --
	(193.47, 31.18) --
	(193.54, 31.59) --
	(193.54, 31.59) --
	(193.54, 31.59) --
	(193.61, 32.50) --
	(193.61, 32.50) --
	(193.61, 32.51) --
	(193.69, 36.90) --
	(193.69, 36.90) --
	(193.69, 36.90) --
	(193.76, 36.76) --
	(193.76, 36.76) --
	(193.76, 36.76) --
	(193.83, 32.12) --
	(193.83, 32.12) --
	(193.83, 32.12) --
	(193.91, 30.45) --
	(193.91, 30.45) --
	(193.91, 30.45) --
	(193.98, 29.53) --
	(193.98, 29.53) --
	(193.98, 29.53) --
	(194.05, 29.52) --
	(194.05, 29.52) --
	(194.05, 29.52) --
	(194.12, 29.44) --
	(194.12, 29.44) --
	(194.12, 29.44) --
	(194.20, 29.26) --
	(194.20, 29.26) --
	(194.20, 29.26) --
	(194.27, 29.41) --
	(194.27, 29.41) --
	(194.27, 29.41) --
	(194.34, 29.96) --
	(194.34, 29.96) --
	(194.34, 29.96) --
	(194.42, 30.65) --
	(194.42, 30.65) --
	(194.42, 30.65) --
	(194.49, 30.05) --
	(194.49, 30.05) --
	(194.49, 30.05) --
	(194.56, 29.22) --
	(194.56, 29.22) --
	(194.56, 29.22) --
	(194.63, 29.08) --
	(194.63, 29.08) --
	(194.63, 29.08) --
	(194.71, 29.16) --
	(194.71, 29.16) --
	(194.71, 29.16) --
	(194.78, 29.09) --
	(194.78, 29.09) --
	(194.78, 29.09) --
	(194.85, 29.08) --
	(194.85, 29.08) --
	(194.85, 29.08) --
	(194.93, 29.06) --
	(194.93, 29.06) --
	(194.93, 29.06) --
	(195.00, 29.10) --
	(195.00, 29.10) --
	(195.00, 29.10) --
	(195.07, 29.39) --
	(195.07, 29.39) --
	(195.07, 29.39) --
	(195.15, 29.78) --
	(195.15, 29.78) --
	(195.15, 29.78) --
	(195.22, 30.12) --
	(195.22, 30.12) --
	(195.22, 30.12) --
	(195.23, 30.07) --
	(195.23, 30.07) --
	(195.23, 30.07) --
	(195.29, 29.92) --
	(195.29, 29.92) --
	(195.29, 29.92) --
	(195.36, 29.60) --
	(195.36, 29.60) --
	(195.36, 29.60) --
	(195.44, 29.59) --
	(195.44, 29.59) --
	(195.44, 29.59) --
	(195.51, 29.83) --
	(195.51, 29.83) --
	(195.51, 29.83) --
	(195.58, 30.28) --
	(195.58, 30.28) --
	(195.58, 30.28) --
	(195.66, 30.86) --
	(195.66, 30.86) --
	(195.66, 30.86) --
	(195.73, 31.80) --
	(195.73, 31.80) --
	(195.73, 31.80) --
	(195.80, 33.98) --
	(195.80, 33.98) --
	(195.80, 33.98) --
	(195.87, 36.80) --
	(195.87, 36.80) --
	(195.87, 36.80) --
	(195.95, 39.04) --
	(195.95, 39.04) --
	(195.95, 39.04) --
	(196.02, 40.35) --
	(196.02, 40.35) --
	(196.02, 40.35) --
	(196.09, 41.11) --
	(196.09, 41.11) --
	(196.09, 41.11) --
	(196.17, 42.78) --
	(196.17, 42.78) --
	(196.17, 42.78) --
	(196.24, 42.80) --
	(196.24, 42.80) --
	(196.24, 42.80) --
	(196.31, 41.19) --
	(196.31, 41.19) --
	(196.31, 41.19) --
	(196.38, 40.76) --
	(196.38, 40.76) --
	(196.38, 40.76) --
	(196.46, 41.63) --
	(196.46, 41.63) --
	(196.46, 41.63) --
	(196.53, 42.80) --
	(196.53, 42.80) --
	(196.53, 42.80) --
	(196.60, 42.20) --
	(196.60, 42.20) --
	(196.60, 42.20) --
	(196.68, 40.40) --
	(196.68, 40.40) --
	(196.68, 40.40) --
	(196.75, 38.61) --
	(196.75, 38.61) --
	(196.75, 38.61) --
	(196.82, 36.93) --
	(196.82, 36.93) --
	(196.82, 36.93) --
	(196.89, 35.71) --
	(196.89, 35.71) --
	(196.89, 35.71) --
	(196.97, 34.99) --
	(196.97, 34.99) --
	(196.97, 34.99) --
	(197.04, 33.64) --
	(197.04, 33.64) --
	(197.04, 33.64) --
	(197.06, 33.34) --
	(197.06, 33.34) --
	(197.06, 33.34) --
	(197.11, 32.21) --
	(197.11, 32.21) --
	(197.11, 32.21) --
	(197.19, 30.21) --
	(197.19, 30.21) --
	(197.19, 30.21) --
	(197.26, 29.78) --
	(197.26, 29.78) --
	(197.26, 29.78) --
	(197.33, 29.50) --
	(197.33, 29.50) --
	(197.33, 29.50) --
	(197.40, 29.35) --
	(197.40, 29.35) --
	(197.40, 29.35) --
	(197.48, 29.29) --
	(197.48, 29.29) --
	(197.48, 29.29) --
	(197.55, 29.31) --
	(197.55, 29.31) --
	(197.55, 29.31) --
	(197.62, 29.49) --
	(197.62, 29.49) --
	(197.62, 29.49) --
	(197.69, 29.60) --
	(197.69, 29.60) --
	(197.69, 29.60) --
	(197.77, 29.29) --
	(197.77, 29.29) --
	(197.77, 29.29) --
	(197.84, 29.15) --
	(197.84, 29.15) --
	(197.84, 29.15) --
	(197.91, 29.62) --
	(197.91, 29.62) --
	(197.91, 29.62) --
	(197.99, 30.76) --
	(197.99, 30.76) --
	(197.99, 30.76) --
	(198.06, 33.31) --
	(198.06, 33.31) --
	(198.06, 33.31) --
	(198.11, 36.61) --
	(198.11, 36.61) --
	(198.11, 36.61) --
	(198.13, 38.09) --
	(198.13, 38.09) --
	(198.13, 38.09) --
	(198.20, 38.81) --
	(198.20, 38.81) --
	(198.20, 38.81) --
	(198.28, 34.87) --
	(198.28, 34.87) --
	(198.28, 34.87) --
	(198.35, 30.69) --
	(198.35, 30.69) --
	(198.35, 30.69) --
	(198.42, 29.29) --
	(198.42, 29.29) --
	(198.42, 29.29) --
	(198.50, 29.16) --
	(198.50, 29.16) --
	(198.50, 29.16) --
	(198.57, 29.09) --
	(198.57, 29.09) --
	(198.57, 29.09) --
	(198.64, 28.73) --
	(198.64, 28.73) --
	(198.64, 28.73) --
	(198.71, 28.69) --
	(198.71, 28.69) --
	(198.71, 28.69) --
	(198.79, 28.92) --
	(198.79, 28.92) --
	(198.79, 28.92) --
	(198.86, 28.84) --
	(198.86, 28.84) --
	(198.86, 28.84) --
	(198.93, 28.52) --
	(198.93, 28.52) --
	(198.93, 28.52) --
	(199.00, 28.37) --
	(199.00, 28.37) --
	(199.00, 28.37) --
	(199.08, 28.37) --
	(199.08, 28.37) --
	(199.08, 28.37) --
	(199.15, 28.40) --
	(199.15, 28.40) --
	(199.15, 28.40) --
	(199.22, 28.36) --
	(199.22, 28.36) --
	(199.22, 28.36) --
	(199.26, 28.32) --
	(199.26, 28.32) --
	(199.26, 28.32) --
	(199.29, 28.29) --
	(199.29, 28.29) --
	(199.29, 28.29) --
	(199.37, 28.27) --
	(199.37, 28.27) --
	(199.37, 28.27) --
	(199.44, 28.29) --
	(199.44, 28.29) --
	(199.44, 28.29) --
	(199.51, 28.46) --
	(199.51, 28.46) --
	(199.51, 28.46) --
	(199.59, 28.58) --
	(199.59, 28.58) --
	(199.59, 28.58) --
	(199.66, 28.78) --
	(199.66, 28.78) --
	(199.66, 28.78) --
	(199.73, 28.67) --
	(199.73, 28.67) --
	(199.73, 28.67) --
	(199.80, 28.41) --
	(199.80, 28.41) --
	(199.80, 28.41) --
	(199.88, 28.38) --
	(199.88, 28.38) --
	(199.88, 28.38) --
	(199.95, 28.52) --
	(199.95, 28.52) --
	(199.95, 28.52) --
	(200.02, 28.70) --
	(200.02, 28.70) --
	(200.02, 28.70) --
	(200.03, 28.70) --
	(200.03, 28.70) --
	(200.03, 28.70) --
	(200.09, 28.66) --
	(200.09, 28.66) --
	(200.09, 28.66) --
	(200.17, 28.61) --
	(200.17, 28.61) --
	(200.17, 28.62) --
	(200.24, 29.02) --
	(200.24, 29.02) --
	(200.24, 29.02) --
	(200.31, 29.36) --
	(200.31, 29.36) --
	(200.31, 29.36) --
	(200.38, 28.80) --
	(200.38, 28.80) --
	(200.38, 28.80) --
	(200.46, 28.24) --
	(200.46, 28.24) --
	(200.46, 28.24) --
	(200.53, 28.14) --
	(200.53, 28.14) --
	(200.53, 28.14) --
	(200.60, 28.14) --
	(200.60, 28.14) --
	(200.60, 28.14) --
	(200.67, 28.13) --
	(200.67, 28.13) --
	(200.67, 28.13) --
	(200.75, 28.20) --
	(200.75, 28.20) --
	(200.75, 28.20) --
	(200.79, 28.36) --
	(200.79, 28.36) --
	(200.79, 28.36) --
	(200.82, 28.45) --
	(200.82, 28.45) --
	(200.82, 28.45) --
	(200.89, 28.72) --
	(200.89, 28.72) --
	(200.89, 28.72) --
	(200.96, 29.23) --
	(200.96, 29.23) --
	(200.96, 29.23) --
	(201.04, 29.11) --
	(201.04, 29.11) --
	(201.04, 29.11) --
	(201.11, 28.53) --
	(201.11, 28.53) --
	(201.11, 28.53) --
	(201.18, 28.52) --
	(201.18, 28.52) --
	(201.18, 28.52) --
	(201.26, 28.45) --
	(201.26, 28.45) --
	(201.26, 28.45) --
	(201.33, 28.28) --
	(201.33, 28.28) --
	(201.33, 28.28) --
	(201.40, 28.20) --
	(201.40, 28.20) --
	(201.40, 28.20) --
	(201.47, 28.31) --
	(201.47, 28.31) --
	(201.47, 28.31) --
	(201.54, 28.57) --
	(201.54, 28.57) --
	(201.54, 28.57) --
	(201.62, 28.66) --
	(201.62, 28.66) --
	(201.62, 28.66) --
	(201.69, 28.31) --
	(201.69, 28.31) --
	(201.69, 28.31) --
	(201.76, 28.13) --
	(201.76, 28.13) --
	(201.76, 28.13) --
	(201.83, 28.24) --
	(201.83, 28.24) --
	(201.83, 28.24) --
	(201.91, 28.52) --
	(201.91, 28.52) --
	(201.91, 28.52) --
	(201.98, 28.91) --
	(201.98, 28.91) --
	(201.98, 28.91) --
	(202.04, 29.69) --
	(202.04, 29.69) --
	(202.04, 29.69) --
	(202.05, 29.87) --
	(202.05, 29.87) --
	(202.05, 29.87) --
	(202.12, 31.90) --
	(202.12, 31.90) --
	(202.12, 31.90) --
	(202.20, 30.99) --
	(202.20, 30.99) --
	(202.20, 30.99) --
	(202.27, 28.78) --
	(202.27, 28.78) --
	(202.27, 28.78) --
	(202.34, 28.21) --
	(202.34, 28.21) --
	(202.34, 28.21) --
	(202.41, 28.13) --
	(202.41, 28.13) --
	(202.41, 28.13) --
	(202.49, 28.14) --
	(202.49, 28.14) --
	(202.49, 28.14) --
	(202.56, 28.12) --
	(202.56, 28.12) --
	(202.56, 28.12) --
	(202.63, 28.16) --
	(202.63, 28.16) --
	(202.63, 28.16) --
	(202.70, 28.24) --
	(202.70, 28.24) --
	(202.70, 28.24) --
	(202.78, 28.19) --
	(202.78, 28.19) --
	(202.78, 28.19) --
	(202.81, 28.23) --
	(202.81, 28.23) --
	(202.81, 28.23) --
	(202.85, 28.29) --
	(202.85, 28.29) --
	(202.85, 28.29) --
	(202.92, 28.74) --
	(202.92, 28.74) --
	(202.92, 28.74) --
	(202.99, 29.74) --
	(202.99, 29.74) --
	(202.99, 29.74) --
	(203.07, 29.51) --
	(203.07, 29.51) --
	(203.07, 29.51) --
	(203.14, 28.46) --
	(203.14, 28.46) --
	(203.14, 28.46) --
	(203.21, 28.16) --
	(203.21, 28.16) --
	(203.21, 28.16) --
	(203.28, 28.15) --
	(203.28, 28.15) --
	(203.28, 28.15) --
	(203.36, 28.42) --
	(203.36, 28.42) --
	(203.36, 28.42) --
	(203.43, 28.70) --
	(203.43, 28.70) --
	(203.43, 28.70) --
	(203.50, 28.89) --
	(203.50, 28.89) --
	(203.50, 28.89) --
	(203.57, 29.60) --
	(203.57, 29.60) --
	(203.57, 29.60) --
	(203.64, 31.03) --
	(203.65, 31.03) --
	(203.65, 31.03) --
	(203.72, 30.39) --
	(203.72, 30.39) --
	(203.72, 30.39) --
	(203.76, 29.54) --
	(203.76, 29.54) --
	(203.76, 29.54) --
	(203.79, 29.05) --
	(203.79, 29.05) --
	(203.79, 29.05) --
	(203.86, 29.09) --
	(203.86, 29.09) --
	(203.86, 29.09) --
	(203.94, 29.60) --
	(203.94, 29.60) --
	(203.94, 29.60) --
	(204.01, 29.87) --
	(204.01, 29.87) --
	(204.01, 29.87) --
	(204.08, 29.19) --
	(204.08, 29.19) --
	(204.08, 29.19) --
	(204.15, 28.42) --
	(204.15, 28.42) --
	(204.15, 28.42) --
	(204.22, 28.25) --
	(204.22, 28.25) --
	(204.22, 28.25) --
	(204.30, 28.08) --
	(204.30, 28.08) --
	(204.30, 28.08) --
	(204.37, 27.96) --
	(204.37, 27.96) --
	(204.37, 27.96) --
	(204.44, 27.98) --
	(204.44, 27.98) --
	(204.44, 27.98) --
	(204.51, 27.99) --
	(204.51, 27.99) --
	(204.51, 27.99) --
	(204.59, 28.01) --
	(204.59, 28.01) --
	(204.59, 28.01) --
	(204.63, 28.12) --
	(204.63, 28.12) --
	(204.63, 28.12) --
	(204.66, 28.21) --
	(204.66, 28.21) --
	(204.66, 28.21) --
	(204.73, 28.54) --
	(204.73, 28.54) --
	(204.73, 28.54) --
	(204.80, 28.85) --
	(204.80, 28.85) --
	(204.80, 28.85) --
	(204.88, 28.45) --
	(204.88, 28.45) --
	(204.88, 28.45) --
	(204.95, 28.17) --
	(204.95, 28.17) --
	(204.95, 28.17) --
	(205.02, 28.44) --
	(205.02, 28.44) --
	(205.02, 28.44) --
	(205.09, 28.49) --
	(205.09, 28.49) --
	(205.09, 28.49) --
	(205.16, 28.21) --
	(205.16, 28.21) --
	(205.16, 28.21) --
	(205.24, 28.20) --
	(205.24, 28.20) --
	(205.24, 28.20) --
	(205.31, 28.66) --
	(205.31, 28.66) --
	(205.31, 28.66) --
	(205.38, 28.76) --
	(205.38, 28.76) --
	(205.38, 28.76) --
	(205.45, 28.25) --
	(205.45, 28.25) --
	(205.45, 28.25) --
	(205.53, 27.91) --
	(205.53, 27.91) --
	(205.53, 27.91) --
	(205.58, 27.89) --
	(205.58, 27.89) --
	(205.58, 27.89) --
	(205.60, 27.88) --
	(205.60, 27.88) --
	(205.60, 27.88) --
	(205.67, 27.92) --
	(205.67, 27.92) --
	(205.67, 27.92) --
	(205.74, 27.88) --
	(205.74, 27.88) --
	(205.74, 27.88) --
	(205.82, 27.80) --
	(205.82, 27.80) --
	(205.82, 27.80) --
	(205.89, 27.78) --
	(205.89, 27.78) --
	(205.89, 27.78) --
	(205.96, 27.74) --
	(205.96, 27.74) --
	(205.96, 27.74) --
	(206.03, 27.79) --
	(206.03, 27.79) --
	(206.03, 27.79) --
	(206.10, 27.92) --
	(206.10, 27.92) --
	(206.10, 27.92) --
	(206.18, 28.33) --
	(206.18, 28.33) --
	(206.18, 28.33) --
	(206.25, 29.38) --
	(206.25, 29.38) --
	(206.25, 29.38) --
	(206.32, 30.24) --
	(206.32, 30.24) --
	(206.32, 30.24) --
	(206.39, 30.19) --
	(206.39, 30.19) --
	(206.39, 30.19) --
	(206.46, 29.41) --
	(206.46, 29.41) --
	(206.46, 29.41) --
	(206.54, 29.07) --
	(206.54, 29.07) --
	(206.54, 29.07) --
	(206.54, 29.06) --
	(206.54, 29.06) --
	(206.54, 29.06) --
	(206.61, 28.98) --
	(206.61, 28.98) --
	(206.61, 28.98) --
	(206.68, 28.85) --
	(206.68, 28.85) --
	(206.68, 28.85) --
	(206.75, 28.82) --
	(206.75, 28.82) --
	(206.75, 28.82) --
	(206.83, 28.66) --
	(206.83, 28.66) --
	(206.83, 28.66) --
	(206.90, 28.55) --
	(206.90, 28.55) --
	(206.90, 28.55) --
	(206.97, 28.06) --
	(206.97, 28.06) --
	(206.97, 28.06) --
	(207.04, 27.73) --
	(207.04, 27.73) --
	(207.04, 27.73) --
	(207.12, 27.67) --
	(207.12, 27.67) --
	(207.12, 27.67) --
	(207.19, 27.75) --
	(207.19, 27.75) --
	(207.19, 27.75) --
	(207.26, 27.73) --
	(207.26, 27.73) --
	(207.26, 27.73) --
	(207.33, 27.70) --
	(207.33, 27.70) --
	(207.33, 27.70) --
	(207.40, 27.72) --
	(207.40, 27.72) --
	(207.40, 27.72) --
	(207.48, 27.69) --
	(207.48, 27.69) --
	(207.48, 27.69) --
	(207.55, 27.66) --
	(207.55, 27.66) --
	(207.55, 27.66) --
	(207.60, 27.66) --
	(207.60, 27.66) --
	(207.60, 27.66) --
	(207.62, 27.66) --
	(207.62, 27.66) --
	(207.62, 27.66) --
	(207.69, 27.72) --
	(207.69, 27.72) --
	(207.69, 27.72) --
	(207.76, 27.99) --
	(207.76, 27.99) --
	(207.76, 27.99) --
	(207.84, 28.97) --
	(207.84, 28.97) --
	(207.84, 28.97) --
	(207.91, 30.30) --
	(207.91, 30.30) --
	(207.91, 30.30) --
	(207.98, 29.52) --
	(207.98, 29.52) --
	(207.98, 29.52) --
	(208.05, 28.07) --
	(208.05, 28.07) --
	(208.05, 28.07) --
	(208.12, 27.60) --
	(208.12, 27.60) --
	(208.12, 27.60) --
	(208.20, 27.52) --
	(208.20, 27.52) --
	(208.20, 27.52) --
	(208.27, 27.53) --
	(208.27, 27.53) --
	(208.27, 27.53) --
	(208.34, 27.53) --
	(208.34, 27.53) --
	(208.34, 27.53) --
	(208.41, 27.50) --
	(208.41, 27.50) --
	(208.41, 27.50) --
	(208.46, 27.51) --
	(208.46, 27.51) --
	(208.46, 27.51) --
	(208.48, 27.52) --
	(208.48, 27.52) --
	(208.48, 27.52) --
	(208.56, 27.59) --
	(208.56, 27.59) --
	(208.56, 27.59) --
	(208.63, 27.85) --
	(208.63, 27.85) --
	(208.63, 27.85) --
	(208.70, 28.67) --
	(208.70, 28.67) --
	(208.70, 28.67) --
	(208.77, 29.84) --
	(208.77, 29.84) --
	(208.77, 29.84) --
	(208.84, 29.71) --
	(208.84, 29.71) --
	(208.84, 29.71) --
	(208.92, 28.53) --
	(208.92, 28.53) --
	(208.92, 28.53) --
	(208.99, 27.89) --
	(208.99, 27.89) --
	(208.99, 27.89) --
	(209.06, 27.82) --
	(209.06, 27.82) --
	(209.06, 27.82) --
	(209.13, 27.80) --
	(209.13, 27.80) --
	(209.13, 27.80) --
	(209.20, 27.72) --
	(209.20, 27.72) --
	(209.20, 27.72) --
	(209.28, 27.64) --
	(209.28, 27.64) --
	(209.28, 27.64) --
	(209.32, 27.66) --
	(209.32, 27.66) --
	(209.32, 27.66) --
	(209.35, 27.67) --
	(209.35, 27.67) --
	(209.35, 27.67) --
	(209.42, 28.00) --
	(209.42, 28.00) --
	(209.42, 28.00) --
	(209.49, 28.38) --
	(209.49, 28.38) --
	(209.49, 28.38) --
	(209.56, 28.12) --
	(209.56, 28.12) --
	(209.56, 28.12) --
	(209.64, 27.66) --
	(209.64, 27.66) --
	(209.64, 27.66) --
	(209.71, 27.52) --
	(209.71, 27.52) --
	(209.71, 27.52) --
	(209.78, 27.51) --
	(209.78, 27.51) --
	(209.78, 27.51) --
	(209.85, 27.46) --
	(209.85, 27.46) --
	(209.85, 27.46) --
	(209.92, 27.46) --
	(209.92, 27.46) --
	(209.92, 27.46) --
	(210.00, 27.44) --
	(210.00, 27.44) --
	(210.00, 27.44) --
	(210.07, 27.45) --
	(210.07, 27.45) --
	(210.07, 27.45) --
	(210.14, 27.49) --
	(210.14, 27.49) --
	(210.14, 27.49) --
	(210.21, 27.45) --
	(210.21, 27.45) --
	(210.21, 27.45) --
	(210.28, 27.38) --
	(210.28, 27.38) --
	(210.28, 27.38) --
	(210.29, 27.37) --
	(210.29, 27.37) --
	(210.29, 27.37) --
	(210.36, 27.32) --
	(210.36, 27.32) --
	(210.36, 27.32) --
	(210.43, 27.32) --
	(210.43, 27.32) --
	(210.43, 27.32) --
	(210.50, 27.36) --
	(210.50, 27.36) --
	(210.50, 27.36) --
	(210.57, 27.47) --
	(210.57, 27.47) --
	(210.57, 27.47) --
	(210.65, 27.54) --
	(210.65, 27.54) --
	(210.65, 27.54) --
	(210.72, 27.54) --
	(210.72, 27.54) --
	(210.72, 27.54) --
	(210.79, 27.83) --
	(210.79, 27.83) --
	(210.79, 27.83) --
	(210.86, 28.30) --
	(210.86, 28.30) --
	(210.86, 28.30) --
	(210.93, 28.05) --
	(210.93, 28.05) --
	(210.93, 28.05) --
	(211.00, 27.60) --
	(211.00, 27.60) --
	(211.00, 27.60) --
	(211.08, 27.47) --
	(211.08, 27.47) --
	(211.08, 27.47) --
	(211.15, 27.74) --
	(211.15, 27.74) --
	(211.15, 27.74) --
	(211.22, 28.95) --
	(211.22, 28.95) --
	(211.22, 28.95) --
	(211.29, 33.04) --
	(211.29, 33.04) --
	(211.29, 33.04) --
	(211.36, 35.68) --
	(211.36, 35.69) --
	(211.36, 35.68) --
	(211.43, 32.80) --
	(211.43, 32.80) --
	(211.43, 32.80) --
	(211.43, 32.61) --
	(211.43, 32.61) --
	(211.43, 32.61) --
	(211.51, 29.25) --
	(211.51, 29.25) --
	(211.51, 29.25) --
	(211.58, 28.02) --
	(211.58, 28.02) --
	(211.58, 28.02) --
	(211.65, 27.59) --
	(211.65, 27.59) --
	(211.65, 27.59) --
	(211.72, 27.43) --
	(211.72, 27.43) --
	(211.72, 27.43) --
	(211.79, 27.37) --
	(211.79, 27.37) --
	(211.79, 27.37) --
	(211.87, 27.34) --
	(211.87, 27.34) --
	(211.87, 27.34) --
	(211.94, 27.33) --
	(211.94, 27.33) --
	(211.94, 27.33) --
	(212.01, 27.37) --
	(212.01, 27.37) --
	(212.01, 27.37) --
	(212.08, 27.32) --
	(212.08, 27.32) --
	(212.08, 27.32) --
	(212.15, 27.26) --
	(212.15, 27.26) --
	(212.15, 27.26) --
	(212.23, 27.22) --
	(212.23, 27.22) --
	(212.23, 27.22) --
	(212.30, 27.21) --
	(212.30, 27.21) --
	(212.30, 27.21) --
	(212.37, 27.23) --
	(212.37, 27.23) --
	(212.37, 27.23) --
	(212.44, 27.20) --
	(212.44, 27.20) --
	(212.44, 27.20) --
	(212.51, 27.17) --
	(212.51, 27.17) --
	(212.51, 27.17) --
	(212.58, 27.18) --
	(212.58, 27.18) --
	(212.58, 27.18) --
	(212.66, 27.18) --
	(212.66, 27.18) --
	(212.66, 27.18) --
	(212.73, 27.18) --
	(212.73, 27.18) --
	(212.73, 27.18) --
	(212.80, 27.17) --
	(212.80, 27.17) --
	(212.80, 27.17) --
	(212.87, 27.17) --
	(212.87, 27.17) --
	(212.87, 27.17) --
	(212.94, 27.18) --
	(212.94, 27.18) --
	(212.94, 27.18) --
	(212.96, 27.19) --
	(212.96, 27.19) --
	(212.96, 27.19) --
	(213.02, 27.20) --
	(213.02, 27.20) --
	(213.02, 27.20) --
	(213.09, 27.19) --
	(213.09, 27.19) --
	(213.09, 27.19) --
	(213.16, 27.16) --
	(213.16, 27.16) --
	(213.16, 27.16) --
	(213.23, 27.17) --
	(213.23, 27.17) --
	(213.23, 27.17) --
	(213.30, 27.16) --
	(213.30, 27.16) --
	(213.30, 27.16) --
	(213.37, 27.16) --
	(213.37, 27.16) --
	(213.37, 27.16) --
	(213.45, 27.22) --
	(213.45, 27.22) --
	(213.45, 27.22) --
	(213.52, 27.26) --
	(213.52, 27.26) --
	(213.52, 27.26) --
	(213.59, 27.25) --
	(213.59, 27.25) --
	(213.59, 27.25) --
	(213.66, 27.22) --
	(213.66, 27.22) --
	(213.66, 27.22) --
	(213.73, 27.26) --
	(213.73, 27.26) --
	(213.73, 27.26) --
	(213.80, 27.24) --
	(213.80, 27.24) --
	(213.80, 27.24) --
	(213.88, 27.15) --
	(213.88, 27.15) --
	(213.88, 27.15) --
	(213.95, 27.12) --
	(213.95, 27.12) --
	(213.95, 27.12) --
	(214.02, 27.13) --
	(214.02, 27.13) --
	(214.02, 27.13) --
	(214.09, 27.10) --
	(214.09, 27.10) --
	(214.09, 27.10) --
	(214.16, 27.12) --
	(214.16, 27.12) --
	(214.16, 27.12) --
	(214.23, 27.13) --
	(214.23, 27.13) --
	(214.23, 27.13) --
	(214.31, 27.13) --
	(214.31, 27.13) --
	(214.31, 27.13) --
	(214.31, 27.13) --
	(214.31, 27.13) --
	(214.31, 27.13) --
	(214.38, 27.19) --
	(214.38, 27.19) --
	(214.38, 27.19) --
	(214.45, 27.45) --
	(214.45, 27.45) --
	(214.45, 27.45) --
	(214.52, 28.40) --
	(214.52, 28.40) --
	(214.52, 28.40) --
	(214.59, 29.23) --
	(214.59, 29.23) --
	(214.59, 29.23) --
	(214.66, 28.67) --
	(214.66, 28.67) --
	(214.66, 28.67) --
	(214.74, 27.78) --
	(214.74, 27.78) --
	(214.74, 27.78) --
	(214.81, 27.34) --
	(214.81, 27.34) --
	(214.81, 27.34) --
	(214.88, 27.16) --
	(214.88, 27.16) --
	(214.88, 27.16) --
	(214.95, 27.10) --
	(214.95, 27.10) --
	(214.95, 27.10) --
	(215.02, 27.08) --
	(215.02, 27.08) --
	(215.02, 27.08) --
	(215.09, 27.08) --
	(215.09, 27.08) --
	(215.09, 27.08) --
	(215.17, 27.13) --
	(215.17, 27.13) --
	(215.17, 27.13) --
	(215.24, 27.11) --
	(215.24, 27.11) --
	(215.24, 27.11) --
	(215.31, 27.04) --
	(215.31, 27.04) --
	(215.31, 27.04) --
	(215.38, 27.02) --
	(215.38, 27.02) --
	(215.38, 27.02) --
	(215.45, 26.98) --
	(215.45, 26.98) --
	(215.45, 26.98) --
	(215.52, 26.97) --
	(215.52, 26.97) --
	(215.52, 26.97) --
	(215.60, 26.99) --
	(215.60, 26.99) --
	(215.60, 26.99) --
	(215.67, 27.05) --
	(215.67, 27.05) --
	(215.67, 27.05) --
	(215.74, 27.09) --
	(215.74, 27.09) --
	(215.74, 27.09) --
	(215.74, 27.08) --
	(215.74, 27.08) --
	(215.74, 27.08) --
	(215.81, 27.04) --
	(215.81, 27.04) --
	(215.81, 27.04) --
	(215.88, 26.98) --
	(215.88, 26.98) --
	(215.88, 26.98) --
	(215.95, 26.98) --
	(215.95, 26.98) --
	(215.95, 26.98) --
	(216.03, 26.96) --
	(216.03, 26.96) --
	(216.03, 26.96) --
	(216.10, 26.96) --
	(216.10, 26.96) --
	(216.10, 26.96) --
	(216.17, 26.96) --
	(216.17, 26.96) --
	(216.17, 26.96) --
	(216.24, 26.98) --
	(216.24, 26.98) --
	(216.24, 26.98) --
	(216.31, 27.04) --
	(216.31, 27.04) --
	(216.31, 27.04) --
	(216.38, 27.06) --
	(216.38, 27.06) --
	(216.38, 27.06) --
	(216.46, 27.03) --
	(216.46, 27.03) --
	(216.46, 27.03) --
	(216.53, 26.97) --
	(216.53, 26.97) --
	(216.53, 26.97) --
	(216.60, 26.95) --
	(216.60, 26.95) --
	(216.60, 26.95) --
	(216.67, 26.95) --
	(216.67, 26.95) --
	(216.67, 26.95) --
	(216.74, 26.92) --
	(216.74, 26.92) --
	(216.74, 26.92) --
	(216.81, 26.91) --
	(216.81, 26.91) --
	(216.81, 26.91) --
	(216.88, 26.92) --
	(216.88, 26.92) --
	(216.88, 26.92) --
	(216.96, 26.93) --
	(216.96, 26.93) --
	(216.96, 26.93) --
	(217.03, 26.90) --
	(217.03, 26.90) --
	(217.03, 26.90) --
	(217.10, 26.91) --
	(217.10, 26.91) --
	(217.10, 26.91) --
	(217.17, 26.93) --
	(217.17, 26.93) --
	(217.17, 26.93) --
	(217.24, 26.92) --
	(217.24, 26.92) --
	(217.24, 26.92) --
	(217.31, 27.07) --
	(217.31, 27.07) --
	(217.31, 27.07) --
	(217.38, 27.38) --
	(217.38, 27.38) --
	(217.38, 27.38) --
	(217.46, 27.44) --
	(217.46, 27.44) --
	(217.46, 27.44) --
	(217.53, 27.28) --
	(217.53, 27.28) --
	(217.53, 27.28) --
	(217.60, 27.12) --
	(217.60, 27.12) --
	(217.60, 27.12) --
	(217.67, 26.95) --
	(217.67, 26.95) --
	(217.67, 26.95) --
	(217.74, 26.87) --
	(217.74, 26.87) --
	(217.74, 26.87) --
	(217.76, 26.86) --
	(217.76, 26.86) --
	(217.76, 26.86) --
	(217.81, 26.83) --
	(217.81, 26.83) --
	(217.81, 26.83) --
	(217.88, 26.84) --
	(217.88, 26.84) --
	(217.88, 26.84) --
	(217.96, 26.87) --
	(217.96, 26.87) --
	(217.96, 26.87) --
	(218.03, 26.89) --
	(218.03, 26.89) --
	(218.03, 26.89) --
	(218.10, 26.90) --
	(218.10, 26.90) --
	(218.10, 26.90) --
	(218.17, 26.94) --
	(218.17, 26.94) --
	(218.17, 26.94) --
	(218.24, 26.91) --
	(218.24, 26.91) --
	(218.24, 26.91) --
	(218.31, 26.88) --
	(218.31, 26.88) --
	(218.31, 26.88) --
	(218.38, 26.88) --
	(218.38, 26.88) --
	(218.38, 26.88) --
	(218.46, 26.85) --
	(218.46, 26.85) --
	(218.46, 26.85) --
	(218.53, 26.82) --
	(218.53, 26.82) --
	(218.53, 26.82) --
	(218.60, 26.83) --
	(218.60, 26.83) --
	(218.60, 26.83) --
	(218.67, 26.90) --
	(218.67, 26.90) --
	(218.67, 26.90) --
	(218.74, 26.93) --
	(218.74, 26.93) --
	(218.74, 26.93) --
	(218.81, 26.85) --
	(218.81, 26.85) --
	(218.81, 26.85) --
	(218.88, 26.81) --
	(218.88, 26.81) --
	(218.88, 26.81) --
	(218.95, 26.86) --
	(218.95, 26.86) --
	(218.95, 26.86) --
	(219.03, 27.07) --
	(219.03, 27.07) --
	(219.03, 27.07) --
	(219.10, 27.26) --
	(219.10, 27.26) --
	(219.10, 27.26) --
	(219.17, 27.09) --
	(219.17, 27.09) --
	(219.17, 27.09) --
	(219.24, 26.88) --
	(219.24, 26.88) --
	(219.24, 26.88) --
	(219.31, 26.86) --
	(219.31, 26.86) --
	(219.31, 26.86) --
	(219.38, 26.98) --
	(219.38, 26.98) --
	(219.38, 26.98) --
	(219.46, 26.97) --
	(219.46, 26.97) --
	(219.46, 26.97) --
	(219.48, 26.93) --
	(219.48, 26.93) --
	(219.48, 26.93) --
	(219.53, 26.86) --
	(219.53, 26.86) --
	(219.53, 26.86) --
	(219.60, 26.79) --
	(219.60, 26.79) --
	(219.60, 26.79) --
	(219.67, 26.79) --
	(219.67, 26.79) --
	(219.67, 26.79) --
	(219.74, 26.79) --
	(219.74, 26.79) --
	(219.74, 26.79) --
	(219.81, 26.79) --
	(219.81, 26.79) --
	(219.81, 26.79) --
	(219.88, 26.84) --
	(219.88, 26.84) --
	(219.88, 26.84) --
	(219.95, 26.83) --
	(219.95, 26.83) --
	(219.95, 26.83) --
	(220.03, 26.76) --
	(220.03, 26.76) --
	(220.03, 26.76) --
	(220.10, 26.71) --
	(220.10, 26.71) --
	(220.10, 26.71) --
	(220.17, 26.67) --
	(220.17, 26.67) --
	(220.17, 26.67) --
	(220.24, 26.68) --
	(220.24, 26.68) --
	(220.24, 26.68) --
	(220.31, 26.68) --
	(220.31, 26.68) --
	(220.31, 26.68) --
	(220.38, 26.62) --
	(220.38, 26.62) --
	(220.38, 26.62) --
	(220.45, 26.63) --
	(220.45, 26.63) --
	(220.45, 26.63) --
	(220.52, 26.64) --
	(220.52, 26.64) --
	(220.52, 26.64) --
	(220.60, 26.63) --
	(220.60, 26.63) --
	(220.60, 26.63) --
	(220.67, 26.66) --
	(220.67, 26.66) --
	(220.67, 26.66) --
	(220.74, 26.68) --
	(220.74, 26.68) --
	(220.74, 26.68) --
	(220.81, 26.68) --
	(220.81, 26.68) --
	(220.81, 26.68) --
	(220.88, 26.65) --
	(220.88, 26.65) --
	(220.88, 26.65) --
	(220.95, 26.64) --
	(220.95, 26.64) --
	(220.95, 26.64) --
	(221.02, 26.66) --
	(221.02, 26.66) --
	(221.02, 26.66) --
	(221.09, 26.65) --
	(221.09, 26.65) --
	(221.09, 26.65) --
	(221.17, 26.60) --
	(221.17, 26.60) --
	(221.17, 26.60) --
	(221.24, 26.61) --
	(221.24, 26.61) --
	(221.24, 26.61) --
	(221.31, 26.59) --
	(221.31, 26.59) --
	(221.31, 26.59) --
	(221.38, 26.57) --
	(221.38, 26.57) --
	(221.38, 26.57) --
	(221.45, 26.56) --
	(221.45, 26.56) --
	(221.45, 26.56) --
	(221.49, 26.56) --
	(221.49, 26.56) --
	(221.49, 26.56) --
	(221.52, 26.56) --
	(221.52, 26.56) --
	(221.52, 26.56) --
	(221.59, 26.56) --
	(221.59, 26.56) --
	(221.59, 26.56) --
	(221.66, 26.57) --
	(221.66, 26.57) --
	(221.66, 26.57) --
	(221.74, 26.58) --
	(221.74, 26.58) --
	(221.74, 26.58) --
	(221.81, 26.58) --
	(221.81, 26.58) --
	(221.81, 26.58) --
	(221.88, 26.58) --
	(221.88, 26.58) --
	(221.88, 26.58) --
	(221.95, 26.56) --
	(221.95, 26.56) --
	(221.95, 26.56) --
	(222.02, 26.56) --
	(222.02, 26.56) --
	(222.02, 26.56) --
	(222.09, 26.56) --
	(222.09, 26.56) --
	(222.09, 26.56) --
	(222.16, 26.56) --
	(222.16, 26.56) --
	(222.16, 26.56) --
	(222.23, 26.59) --
	(222.23, 26.59) --
	(222.23, 26.59) --
	(222.30, 26.68) --
	(222.30, 26.68) --
	(222.30, 26.68) --
	(222.37, 26.68) --
	(222.37, 26.68) --
	(222.37, 26.68) --
	(222.45, 26.64) --
	(222.45, 26.64) --
	(222.45, 26.64) --
	(222.52, 26.64) --
	(222.52, 26.64) --
	(222.52, 26.64) --
	(222.59, 26.65) --
	(222.59, 26.65) --
	(222.59, 26.65) --
	(222.66, 26.71) --
	(222.66, 26.71) --
	(222.66, 26.71) --
	(222.73, 26.71) --
	(222.73, 26.71) --
	(222.73, 26.71) --
	(222.80, 26.69) --
	(222.80, 26.69) --
	(222.80, 26.69) --
	(222.87, 26.64) --
	(222.87, 26.64) --
	(222.87, 26.64) --
	(222.94, 26.58) --
	(222.94, 26.58) --
	(222.94, 26.58) --
	(223.02, 26.57) --
	(223.02, 26.57) --
	(223.02, 26.57) --
	(223.09, 26.56) --
	(223.09, 26.56) --
	(223.09, 26.56) --
	(223.16, 26.61) --
	(223.16, 26.61) --
	(223.16, 26.61) --
	(223.23, 26.72) --
	(223.23, 26.72) --
	(223.23, 26.72) --
	(223.30, 26.71) --
	(223.30, 26.71) --
	(223.30, 26.71) --
	(223.37, 26.63) --
	(223.37, 26.63) --
	(223.37, 26.63) --
	(223.44, 26.57) --
	(223.44, 26.57) --
	(223.44, 26.57) --
	(223.51, 26.54) --
	(223.51, 26.54) --
	(223.51, 26.54) --
	(223.58, 26.57) --
	(223.58, 26.57) --
	(223.58, 26.57) --
	(223.66, 26.59) --
	(223.66, 26.59) --
	(223.66, 26.59) --
	(223.73, 26.56) --
	(223.73, 26.56) --
	(223.73, 26.56) --
	(223.79, 26.53) --
	(223.79, 26.53) --
	(223.79, 26.53) --
	(223.80, 26.53) --
	(223.80, 26.53) --
	(223.80, 26.53) --
	(223.87, 26.52) --
	(223.87, 26.52) --
	(223.87, 26.52) --
	(223.94, 26.58) --
	(223.94, 26.58) --
	(223.94, 26.58) --
	(224.01, 26.75) --
	(224.01, 26.75) --
	(224.01, 26.75) --
	(224.08, 26.79) --
	(224.08, 26.79) --
	(224.08, 26.79) --
	(224.15, 26.70) --
	(224.15, 26.70) --
	(224.15, 26.70) --
	(224.22, 26.75) --
	(224.22, 26.75) --
	(224.22, 26.75) --
	(224.29, 26.93) --
	(224.29, 26.93) --
	(224.29, 26.93) --
	(224.36, 26.88) --
	(224.36, 26.88) --
	(224.36, 26.88) --
	(224.44, 26.69) --
	(224.44, 26.69) --
	(224.44, 26.69) --
	(224.51, 26.55) --
	(224.51, 26.55) --
	(224.51, 26.55) --
	(224.58, 26.53) --
	(224.58, 26.53) --
	(224.58, 26.53) --
	(224.65, 26.53) --
	(224.65, 26.53) --
	(224.65, 26.53) --
	(224.72, 26.50) --
	(224.72, 26.50) --
	(224.72, 26.50) --
	(224.79, 26.49) --
	(224.79, 26.49) --
	(224.79, 26.49) --
	(224.86, 26.48) --
	(224.86, 26.48) --
	(224.86, 26.48) --
	(224.93, 26.46) --
	(224.93, 26.46) --
	(224.93, 26.46) --
	(225.00, 26.47) --
	(225.00, 26.47) --
	(225.00, 26.47) --
	(225.07, 26.45) --
	(225.07, 26.45) --
	(225.07, 26.45) --
	(225.15, 26.47) --
	(225.15, 26.47) --
	(225.15, 26.47) --
	(225.22, 26.48) --
	(225.22, 26.48) --
	(225.22, 26.48) --
	(225.29, 26.46) --
	(225.29, 26.46) --
	(225.29, 26.46) --
	(225.36, 26.45) --
	(225.36, 26.45) --
	(225.36, 26.45) --
	(225.43, 26.51) --
	(225.43, 26.51) --
	(225.43, 26.51) --
	(225.50, 26.58) --
	(225.50, 26.58) --
	(225.50, 26.58) --
	(225.57, 26.57) --
	(225.57, 26.57) --
	(225.57, 26.57) --
	(225.61, 26.52) --
	(225.61, 26.52) --
	(225.61, 26.52) --
	(225.64, 26.50) --
	(225.64, 26.50) --
	(225.64, 26.50) --
	(225.71, 26.44) --
	(225.71, 26.44) --
	(225.71, 26.44) --
	(225.78, 26.43) --
	(225.78, 26.43) --
	(225.78, 26.43) --
	(225.85, 26.42) --
	(225.85, 26.42) --
	(225.85, 26.42) --
	(225.93, 26.42) --
	(225.93, 26.42) --
	(225.93, 26.42) --
	(226.00, 26.43) --
	(226.00, 26.43) --
	(226.00, 26.43) --
	(226.07, 26.42) --
	(226.07, 26.42) --
	(226.07, 26.42) --
	(226.14, 26.43) --
	(226.14, 26.43) --
	(226.14, 26.43) --
	(226.21, 26.41) --
	(226.21, 26.41) --
	(226.21, 26.41) --
	(226.28, 26.40) --
	(226.28, 26.40) --
	(226.28, 26.40) --
	(226.35, 26.39) --
	(226.35, 26.39) --
	(226.35, 26.39) --
	(226.42, 26.39) --
	(226.42, 26.39) --
	(226.42, 26.39) --
	(226.49, 26.38) --
	(226.49, 26.38) --
	(226.49, 26.38) --
	(226.56, 26.40) --
	(226.56, 26.40) --
	(226.56, 26.40) --
	(226.63, 26.44) --
	(226.63, 26.44) --
	(226.63, 26.44) --
	(226.71, 26.64) --
	(226.71, 26.64) --
	(226.71, 26.64) --
	(226.78, 26.91) --
	(226.78, 26.91) --
	(226.78, 26.91) --
	(226.85, 26.90) --
	(226.85, 26.90) --
	(226.85, 26.90) --
	(226.92, 26.70) --
	(226.92, 26.70) --
	(226.92, 26.70) --
	(226.99, 26.53) --
	(226.99, 26.53) --
	(226.99, 26.53) --
	(227.06, 26.44) --
	(227.06, 26.44) --
	(227.06, 26.44) --
	(227.13, 26.41) --
	(227.13, 26.41) --
	(227.13, 26.41) --
	(227.20, 26.42) --
	(227.20, 26.42) --
	(227.20, 26.42) --
	(227.27, 26.55) --
	(227.27, 26.55) --
	(227.27, 26.55) --
	(227.34, 26.68) --
	(227.34, 26.68) --
	(227.34, 26.68) --
	(227.41, 26.64) --
	(227.41, 26.64) --
	(227.41, 26.64) --
	(227.48, 26.54) --
	(227.48, 26.54) --
	(227.48, 26.54) --
	(227.56, 26.45) --
	(227.56, 26.45) --
	(227.56, 26.45) --
	(227.62, 26.40) --
	(227.62, 26.40) --
	(227.62, 26.40) --
	(227.70, 26.39) --
	(227.70, 26.39) --
	(227.70, 26.39) --
	(227.77, 26.42) --
	(227.77, 26.42) --
	(227.77, 26.42) --
	(227.84, 26.45) --
	(227.84, 26.45) --
	(227.84, 26.45) --
	(227.91, 26.44) --
	(227.91, 26.44) --
	(227.91, 26.44) --
	(227.91, 26.44) --
	(227.91, 26.44) --
	(227.91, 26.44) --
	(227.98, 26.40) --
	(227.98, 26.40) --
	(227.98, 26.40) --
	(228.05, 26.39) --
	(228.05, 26.39) --
	(228.05, 26.39) --
	(228.12, 26.40) --
	(228.12, 26.40) --
	(228.12, 26.40) --
	(228.19, 26.38) --
	(228.19, 26.38) --
	(228.19, 26.38) --
	(228.26, 26.33) --
	(228.26, 26.33) --
	(228.26, 26.33) --
	(228.33, 26.32) --
	(228.33, 26.32) --
	(228.33, 26.32) --
	(228.40, 26.31) --
	(228.40, 26.31) --
	(228.40, 26.31) --
	(228.47, 26.30) --
	(228.47, 26.30) --
	(228.47, 26.30) --
	(228.55, 26.30) --
	(228.55, 26.30) --
	(228.55, 26.30) --
	(228.62, 26.35) --
	(228.62, 26.35) --
	(228.62, 26.35) --
	(228.69, 26.36) --
	(228.69, 26.36) --
	(228.69, 26.36) --
	(228.76, 26.34) --
	(228.76, 26.34) --
	(228.76, 26.34) --
	(228.83, 26.29) --
	(228.83, 26.29) --
	(228.83, 26.29) --
	(228.90, 26.28) --
	(228.90, 26.28) --
	(228.90, 26.28) --
	(228.97, 26.28) --
	(228.97, 26.28) --
	(228.97, 26.28) --
	(229.04, 26.27) --
	(229.04, 26.27) --
	(229.04, 26.27) --
	(229.11, 26.28) --
	(229.11, 26.28) --
	(229.11, 26.28) --
	(229.18, 26.27) --
	(229.18, 26.27) --
	(229.18, 26.27) --
	(229.25, 26.27) --
	(229.25, 26.27) --
	(229.25, 26.27) --
	(229.32, 26.27) --
	(229.32, 26.27) --
	(229.32, 26.27) --
	(229.39, 26.28) --
	(229.39, 26.28) --
	(229.39, 26.28) --
	(229.46, 26.32) --
	(229.46, 26.32) --
	(229.46, 26.32) --
	(229.53, 26.33) --
	(229.53, 26.33) --
	(229.53, 26.33) --
	(229.60, 26.30) --
	(229.60, 26.30) --
	(229.60, 26.30) --
	(229.68, 26.27) --
	(229.68, 26.27) --
	(229.68, 26.27) --
	(229.75, 26.24) --
	(229.75, 26.24) --
	(229.75, 26.24) --
	(229.82, 26.23) --
	(229.82, 26.23) --
	(229.82, 26.23) --
	(229.89, 26.23) --
	(229.89, 26.23) --
	(229.89, 26.23) --
	(229.96, 26.21) --
	(229.96, 26.21) --
	(229.96, 26.21) --
	(230.03, 26.22) --
	(230.03, 26.22) --
	(230.03, 26.22) --
	(230.10, 26.21) --
	(230.10, 26.21) --
	(230.10, 26.21) --
	(230.17, 26.21) --
	(230.17, 26.21) --
	(230.17, 26.21) --
	(230.24, 26.21) --
	(230.24, 26.21) --
	(230.24, 26.21) --
	(230.31, 26.21) --
	(230.31, 26.21) --
	(230.31, 26.21) --
	(230.38, 26.20) --
	(230.38, 26.20) --
	(230.38, 26.20) --
	(230.45, 26.19) --
	(230.45, 26.19) --
	(230.45, 26.19) --
	(230.50, 26.20) --
	(230.50, 26.20) --
	(230.50, 26.20) --
	(230.52, 26.20) --
	(230.52, 26.20) --
	(230.52, 26.20) --
	(230.59, 26.19) --
	(230.59, 26.19) --
	(230.59, 26.19) --
	(230.66, 26.20) --
	(230.66, 26.20) --
	(230.66, 26.20) --
	(230.73, 26.20) --
	(230.73, 26.20) --
	(230.73, 26.20) --
	(230.80, 26.20) --
	(230.80, 26.20) --
	(230.80, 26.20) --
	(230.88, 26.19) --
	(230.88, 26.19) --
	(230.88, 26.19) --
	(230.95, 26.18) --
	(230.95, 26.18) --
	(230.95, 26.18) --
	(231.02, 26.20) --
	(231.02, 26.20) --
	(231.02, 26.20) --
	(231.09, 26.23) --
	(231.09, 26.23) --
	(231.09, 26.23) --
	(231.16, 26.27) --
	(231.16, 26.27) --
	(231.16, 26.27) --
	(231.23, 26.32) --
	(231.23, 26.32) --
	(231.23, 26.32) --
	(231.30, 26.33) --
	(231.30, 26.33) --
	(231.30, 26.33) --
	(231.37, 26.30) --
	(231.37, 26.30) --
	(231.37, 26.30) --
	(231.44, 26.26) --
	(231.44, 26.26) --
	(231.44, 26.26) --
	(231.51, 26.21) --
	(231.51, 26.21) --
	(231.51, 26.21) --
	(231.58, 26.18) --
	(231.58, 26.18) --
	(231.58, 26.18) --
	(231.65, 26.18) --
	(231.65, 26.18) --
	(231.65, 26.18) --
	(231.72, 26.16) --
	(231.72, 26.16) --
	(231.72, 26.16) --
	(231.79, 26.17) --
	(231.79, 26.17) --
	(231.79, 26.17) --
	(231.86, 26.17) --
	(231.86, 26.17) --
	(231.86, 26.17) --
	(231.93, 26.15) --
	(231.93, 26.15) --
	(231.93, 26.15) --
	(232.00, 26.16) --
	(232.00, 26.16) --
	(232.00, 26.16) --
	(232.07, 26.19) --
	(232.07, 26.19) --
	(232.07, 26.19) --
	(232.14, 26.20) --
	(232.14, 26.20) --
	(232.14, 26.20) --
	(232.21, 26.23) --
	(232.21, 26.23) --
	(232.21, 26.23) --
	(232.28, 26.22) --
	(232.28, 26.22) --
	(232.28, 26.22) --
	(232.36, 26.19) --
	(232.36, 26.19) --
	(232.36, 26.19) --
	(232.43, 26.17) --
	(232.43, 26.17) --
	(232.43, 26.17) --
	(232.50, 26.14) --
	(232.50, 26.14) --
	(232.50, 26.14) --
	(232.57, 26.15) --
	(232.57, 26.15) --
	(232.57, 26.15) --
	(232.64, 26.16) --
	(232.64, 26.16) --
	(232.64, 26.16) --
	(232.71, 26.16) --
	(232.71, 26.16) --
	(232.71, 26.16) --
	(232.78, 26.19) --
	(232.78, 26.19) --
	(232.78, 26.19) --
	(232.85, 26.28) --
	(232.85, 26.28) --
	(232.85, 26.28) --
	(232.92, 26.35) --
	(232.92, 26.35) --
	(232.92, 26.35) --
	(232.99, 26.37) --
	(232.99, 26.37) --
	(232.99, 26.37) --
	(233.06, 26.27) --
	(233.06, 26.27) --
	(233.06, 26.27) --
	(233.13, 26.20) --
	(233.13, 26.20) --
	(233.13, 26.20) --
	(233.20, 26.16) --
	(233.20, 26.16) --
	(233.20, 26.16) --
	(233.27, 26.13) --
	(233.27, 26.13) --
	(233.27, 26.13) --
	(233.28, 26.13) --
	(233.28, 26.13) --
	(233.28, 26.13) --
	(233.34, 26.13) --
	(233.34, 26.13) --
	(233.34, 26.13) --
	(233.41, 26.13) --
	(233.41, 26.13) --
	(233.41, 26.13) --
	(233.48, 26.13) --
	(233.48, 26.13) --
	(233.48, 26.13) --
	(233.55, 26.15) --
	(233.55, 26.15) --
	(233.55, 26.15) --
	(233.62, 26.15) --
	(233.62, 26.15) --
	(233.62, 26.15) --
	(233.69, 26.15) --
	(233.69, 26.15) --
	(233.69, 26.15) --
	(233.76, 26.14) --
	(233.76, 26.14) --
	(233.76, 26.14) --
	(233.83, 26.12) --
	(233.83, 26.12) --
	(233.83, 26.12) --
	(233.90, 26.11) --
	(233.90, 26.11) --
	(233.90, 26.11) --
	(233.97, 26.12) --
	(233.97, 26.12) --
	(233.97, 26.12) --
	(234.04, 26.10) --
	(234.04, 26.10) --
	(234.04, 26.10) --
	(234.11, 26.11) --
	(234.11, 26.11) --
	(234.11, 26.11) --
	(234.18, 26.10) --
	(234.18, 26.10) --
	(234.18, 26.10) --
	(234.25, 26.10) --
	(234.26, 26.10) --
	(234.26, 26.10) --
	(234.32, 26.10) --
	(234.32, 26.10) --
	(234.32, 26.10) --
	(234.40, 26.10) --
	(234.40, 26.10) --
	(234.40, 26.10) --
	(234.47, 26.09) --
	(234.47, 26.09) --
	(234.47, 26.09) --
	(234.54, 26.09) --
	(234.54, 26.09) --
	(234.54, 26.09) --
	(234.61, 26.09) --
	(234.61, 26.09) --
	(234.61, 26.09) --
	(234.68, 26.10) --
	(234.68, 26.10) --
	(234.68, 26.10) --
	(234.75, 26.10) --
	(234.75, 26.10) --
	(234.75, 26.10) --
	(234.82, 26.10) --
	(234.82, 26.10) --
	(234.82, 26.10) --
	(234.89, 26.09) --
	(234.89, 26.09) --
	(234.89, 26.09) --
	(234.96, 26.10) --
	(234.96, 26.10) --
	(234.96, 26.10) --
	(235.03, 26.08) --
	(235.03, 26.08) --
	(235.03, 26.08) --
	(235.10, 26.09) --
	(235.10, 26.09) --
	(235.10, 26.09) --
	(235.17, 26.10) --
	(235.17, 26.10) --
	(235.17, 26.10) --
	(235.24, 26.11) --
	(235.24, 26.11) --
	(235.24, 26.11) --
	(235.31, 26.11) --
	(235.31, 26.11) --
	(235.31, 26.11) --
	(235.38, 26.11) --
	(235.38, 26.11) --
	(235.38, 26.11) --
	(235.45, 26.11) --
	(235.45, 26.11) --
	(235.45, 26.11) --
	(235.52, 26.12) --
	(235.52, 26.12) --
	(235.52, 26.12) --
	(235.59, 26.11) --
	(235.59, 26.11) --
	(235.59, 26.11) --
	(235.66, 26.14) --
	(235.66, 26.14) --
	(235.66, 26.14) --
	(235.73, 26.17) --
	(235.73, 26.17) --
	(235.73, 26.17) --
	(235.80, 26.18) --
	(235.80, 26.18) --
	(235.80, 26.18) --
	(235.87, 26.19) --
	(235.87, 26.19) --
	(235.87, 26.19) --
	(235.94, 26.14) --
	(235.94, 26.14) --
	(235.94, 26.14) --
	(236.01, 26.11) --
	(236.01, 26.11) --
	(236.01, 26.11) --
	(236.08, 26.10) --
	(236.08, 26.10) --
	(236.08, 26.10) --
	(236.15, 26.09) --
	(236.15, 26.09) --
	(236.15, 26.09) --
	(236.22, 26.08) --
	(236.22, 26.08) --
	(236.22, 26.08) --
	(236.25, 26.08) --
	(236.25, 26.08) --
	(236.25, 26.08) --
	(236.29, 26.08) --
	(236.29, 26.08) --
	(236.29, 26.08) --
	(236.36, 26.07) --
	(236.36, 26.07) --
	(236.36, 26.07) --
	(236.43, 26.06) --
	(236.43, 26.06) --
	(236.43, 26.06) --
	(236.50, 26.07) --
	(236.50, 26.07) --
	(236.50, 26.07) --
	(236.57, 26.05) --
	(236.57, 26.05) --
	(236.57, 26.05) --
	(236.64, 26.06) --
	(236.64, 26.06) --
	(236.64, 26.06) --
	(236.71, 26.06) --
	(236.71, 26.06) --
	(236.71, 26.06) --
	(236.78, 26.05) --
	(236.78, 26.05) --
	(236.78, 26.05) --
	(236.85, 26.04) --
	(236.85, 26.04) --
	(236.85, 26.04) --
	(236.92, 26.06) --
	(236.92, 26.06) --
	(236.92, 26.06) --
	(236.99, 26.04) --
	(236.99, 26.04) --
	(236.99, 26.04) --
	(237.06, 26.05) --
	(237.06, 26.05) --
	(237.06, 26.05) --
	(237.13, 26.04) --
	(237.13, 26.04) --
	(237.13, 26.04) --
	(237.20, 26.05) --
	(237.20, 26.05) --
	(237.20, 26.05) --
	(237.27, 26.07) --
	(237.27, 26.07) --
	(237.27, 26.07) --
	(237.34, 26.07) --
	(237.34, 26.07) --
	(237.34, 26.07) --
	(237.41, 26.12) --
	(237.41, 26.12) --
	(237.41, 26.12) --
	(237.48, 26.17) --
	(237.48, 26.17) --
	(237.48, 26.17) --
	(237.55, 26.20) --
	(237.55, 26.20) --
	(237.55, 26.20) --
	(237.62, 26.16) --
	(237.62, 26.16) --
	(237.62, 26.16) --
	(237.69, 26.11) --
	(237.69, 26.11) --
	(237.69, 26.11) --
	(237.76, 26.08) --
	(237.76, 26.08) --
	(237.76, 26.08) --
	(237.83, 26.06) --
	(237.83, 26.06) --
	(237.83, 26.06) --
	(237.90, 26.04) --
	(237.90, 26.04) --
	(237.90, 26.04) --
	(237.97, 26.05) --
	(237.97, 26.05) --
	(237.97, 26.05) --
	(238.04, 26.05) --
	(238.04, 26.05) --
	(238.04, 26.05) --
	(238.11, 26.04) --
	(238.11, 26.04) --
	(238.11, 26.04) --
	(238.18, 26.05) --
	(238.18, 26.05) --
	(238.18, 26.05) --
	(238.25, 26.07) --
	(238.25, 26.07) --
	(238.25, 26.07) --
	(238.32, 26.07) --
	(238.32, 26.07) --
	(238.32, 26.07) --
	(238.39, 26.11) --
	(238.39, 26.11) --
	(238.39, 26.11) --
	(238.46, 26.16) --
	(238.46, 26.16) --
	(238.46, 26.16) --
	(238.53, 26.18) --
	(238.53, 26.18) --
	(238.53, 26.18) --
	(238.60, 26.15) --
	(238.60, 26.15) --
	(238.60, 26.15) --
	(238.67, 26.08) --
	(238.67, 26.08) --
	(238.67, 26.08) --
	(238.74, 26.06) --
	(238.74, 26.06) --
	(238.74, 26.06) --
	(238.74, 26.06) --
	(238.74, 26.06) --
	(238.74, 26.06) --
	(238.81, 26.04) --
	(238.81, 26.04) --
	(238.81, 26.04) --
	(238.88, 26.03) --
	(238.88, 26.03) --
	(238.88, 26.03) --
	(238.95, 26.03) --
	(238.95, 26.03) --
	(238.95, 26.03) --
	(239.02, 26.02) --
	(239.02, 26.02) --
	(239.02, 26.02) --
	(239.09, 26.02) --
	(239.09, 26.02) --
	(239.09, 26.02) --
	(239.16, 26.01) --
	(239.16, 26.01) --
	(239.16, 26.01) --
	(239.23, 26.01) --
	(239.23, 26.01) --
	(239.23, 26.01) --
	(239.30, 26.02) --
	(239.30, 26.02) --
	(239.30, 26.02) --
	(239.37, 26.02) --
	(239.37, 26.02) --
	(239.37, 26.02) --
	(239.44, 26.01) --
	(239.44, 26.01) --
	(239.44, 26.01) --
	(239.51, 26.02) --
	(239.51, 26.02) --
	(239.51, 26.02) --
	(239.58, 26.02) --
	(239.58, 26.02) --
	(239.58, 26.02) --
	(239.65, 26.01) --
	(239.65, 26.01) --
	(239.65, 26.01) --
	(239.72, 26.03) --
	(239.72, 26.03) --
	(239.72, 26.03) --
	(239.79, 26.05) --
	(239.79, 26.05) --
	(239.79, 26.05) --
	(239.86, 26.05) --
	(239.86, 26.05) --
	(239.86, 26.05) --
	(239.93, 26.04) --
	(239.93, 26.04) --
	(239.93, 26.04) --
	(240.00, 26.03) --
	(240.00, 26.03) --
	(240.00, 26.03) --
	(240.07, 26.01) --
	(240.07, 26.01) --
	(240.07, 26.01) --
	(240.14, 26.01) --
	(240.14, 26.01) --
	(240.14, 26.01) --
	(240.21, 26.02) --
	(240.21, 26.02) --
	(240.21, 26.02) --
	(240.28, 26.01) --
	(240.28, 26.01) --
	(240.28, 26.01) --
	(240.35, 26.02) --
	(240.35, 26.02) --
	(240.35, 26.02) --
	(240.42, 26.02) --
	(240.42, 26.02) --
	(240.42, 26.02) --
	(240.49, 26.03) --
	(240.49, 26.03) --
	(240.49, 26.03) --
	(240.56, 26.04) --
	(240.56, 26.04) --
	(240.56, 26.04) --
	(240.56, 26.04) --
	(240.56, 26.04) --
	(240.56, 26.04) --
	(240.63, 26.03) --
	(240.63, 26.03) --
	(240.63, 26.03) --
	(240.70, 26.02) --
	(240.70, 26.02) --
	(240.70, 26.02) --
	(240.77, 26.01) --
	(240.77, 26.01) --
	(240.77, 26.01) --
	(240.84, 26.00) --
	(240.84, 26.00) --
	(240.84, 26.00) --
	(240.91, 26.01) --
	(240.91, 26.01) --
	(240.91, 26.01) --
	(240.98, 25.99) --
	(240.98, 25.99) --
	(240.98, 25.99) --
	(241.05, 26.01) --
	(241.05, 26.01) --
	(241.05, 26.01) --
	(241.12, 26.00) --
	(241.12, 26.00) --
	(241.12, 26.00) --
	(241.19, 25.99) --
	(241.19, 25.99) --
	(241.19, 25.99) --
	(241.26, 26.00) --
	(241.26, 26.00) --
	(241.26, 26.00) --
	(241.33, 26.00) --
	(241.33, 26.00) --
	(241.33, 26.00) --
	(241.40, 25.99) --
	(241.40, 25.99) --
	(241.40, 25.99) --
	(241.47, 25.99) --
	(241.47, 25.99) --
	(241.47, 25.99) --
	(241.54, 25.99) --
	(241.54, 25.99) --
	(241.54, 25.99) --
	(241.61, 26.00) --
	(241.61, 26.00) --
	(241.61, 26.00) --
	(241.68, 25.99) --
	(241.68, 25.99) --
	(241.68, 25.99) --
	(241.75, 25.99) --
	(241.75, 25.99) --
	(241.75, 25.99) --
	(241.82, 25.99) --
	(241.82, 25.99) --
	(241.82, 25.99) --
	(241.89, 26.00) --
	(241.89, 26.00) --
	(241.89, 26.00) --
	(241.96, 25.99) --
	(241.96, 25.99) --
	(241.96, 25.99) --
	(242.03, 26.00) --
	(242.03, 26.00) --
	(242.03, 26.00) --
	(242.09, 26.00) --
	(242.09, 26.00) --
	(242.09, 26.00) --
	(242.16, 25.99) --
	(242.16, 25.99) --
	(242.16, 25.99) --
	(242.23, 25.99) --
	(242.23, 25.99) --
	(242.23, 25.99) --
	(242.30, 25.99) --
	(242.30, 25.99) --
	(242.30, 25.99) --
	(242.37, 25.98) --
	(242.37, 25.98) --
	(242.37, 25.98) --
	(242.44, 25.99) --
	(242.44, 25.99) --
	(242.44, 25.99) --
	(242.51, 25.99) --
	(242.51, 25.99) --
	(242.51, 25.99) --
	(242.58, 25.98) --
	(242.58, 25.98) --
	(242.58, 25.98) --
	(242.65, 25.99) --
	(242.65, 25.99) --
	(242.65, 25.99) --
	(242.72, 25.99) --
	(242.72, 25.99) --
	(242.72, 25.99) --
	(242.79, 26.01) --
	(242.79, 26.01) --
	(242.79, 26.01) --
	(242.86, 26.05) --
	(242.86, 26.05) --
	(242.86, 26.05) --
	(242.93, 26.07) --
	(242.93, 26.07) --
	(242.93, 26.07) --
	(243.00, 26.05) --
	(243.00, 26.05) --
	(243.00, 26.05) --
	(243.07, 26.05) --
	(243.07, 26.05) --
	(243.07, 26.05) --
	(243.14, 26.02) --
	(243.14, 26.02) --
	(243.14, 26.02) --
	(243.21, 26.03) --
	(243.21, 26.03) --
	(243.21, 26.03) --
	(243.28, 26.11) --
	(243.28, 26.11) --
	(243.28, 26.11) --
	(243.35, 26.18) --
	(243.35, 26.18) --
	(243.35, 26.18) --
	(243.42, 26.13) --
	(243.42, 26.13) --
	(243.42, 26.13) --
	(243.49, 26.06) --
	(243.49, 26.06) --
	(243.49, 26.06) --
	(243.56, 26.04) --
	(243.56, 26.04) --
	(243.56, 26.04) --
	(243.63, 26.02) --
	(243.63, 26.02) --
	(243.63, 26.02) --
	(243.70, 25.99) --
	(243.70, 25.99) --
	(243.70, 25.99) --
	(243.77, 25.98) --
	(243.77, 25.98) --
	(243.77, 25.98) --
	(243.84, 25.98) --
	(243.84, 25.98) --
	(243.84, 25.98) --
	(243.90, 25.97) --
	(243.91, 25.97) --
	(243.91, 25.97) --
	(243.98, 25.97) --
	(243.98, 25.97) --
	(243.98, 25.97) --
	(244.05, 25.97) --
	(244.05, 25.97) --
	(244.05, 25.97) --
	(244.11, 25.97) --
	(244.11, 25.97) --
	(244.11, 25.97) --
	(244.18, 25.97) --
	(244.18, 25.97) --
	(244.18, 25.97) --
	(244.25, 25.97) --
	(244.25, 25.97) --
	(244.25, 25.97) --
	(244.32, 25.97) --
	(244.32, 25.97) --
	(244.32, 25.97) --
	(244.39, 25.98) --
	(244.39, 25.98) --
	(244.39, 25.98) --
	(244.46, 25.99) --
	(244.46, 25.99) --
	(244.46, 25.99) --
	(244.53, 25.98) --
	(244.53, 25.98) --
	(244.53, 25.98) --
	(244.60, 25.99) --
	(244.60, 25.99) --
	(244.60, 25.99) --
	(244.67, 25.97) --
	(244.67, 25.97) --
	(244.67, 25.97) --
	(244.74, 25.97) --
	(244.74, 25.97) --
	(244.74, 25.97) --
	(244.81, 25.97) --
	(244.81, 25.97) --
	(244.81, 25.97) --
	(244.88, 25.97) --
	(244.88, 25.97) --
	(244.88, 25.97) --
	(244.95, 25.97) --
	(244.95, 25.97) --
	(244.95, 25.97) --
	(245.02, 25.98) --
	(245.02, 25.98) --
	(245.02, 25.98) --
	(245.09, 25.97) --
	(245.09, 25.97) --
	(245.09, 25.97) --
	(245.16, 25.98) --
	(245.16, 25.98) --
	(245.16, 25.98) --
	(245.23, 25.96) --
	(245.23, 25.96) --
	(245.23, 25.96) --
	(245.30, 25.97) --
	(245.30, 25.97) --
	(245.30, 25.97) --
	(245.37, 25.98) --
	(245.37, 25.98) --
	(245.37, 25.98) --
	(245.44, 25.99) --
	(245.44, 25.99) --
	(245.44, 25.99) --
	(245.51, 26.02) --
	(245.51, 26.02) --
	(245.51, 26.02) --
	(245.57, 26.03) --
	(245.57, 26.03) --
	(245.57, 26.03) --
	(245.64, 26.00) --
	(245.64, 26.00) --
	(245.64, 26.00) --
	(245.71, 25.99) --
	(245.71, 25.99) --
	(245.71, 25.99) --
	(245.78, 25.98) --
	(245.78, 25.98) --
	(245.78, 25.98) --
	(245.85, 25.98) --
	(245.85, 25.98) --
	(245.85, 25.98) --
	(245.92, 25.99) --
	(245.92, 25.99) --
	(245.92, 25.99) --
	(245.99, 26.00) --
	(245.99, 26.00) --
	(245.99, 26.00) --
	(246.06, 26.00) --
	(246.06, 26.00) --
	(246.06, 26.00) --
	(246.13, 26.00) --
	(246.13, 26.00) --
	(246.13, 26.00) --
	(246.20, 25.99) --
	(246.20, 25.99) --
	(246.20, 25.99) --
	(246.27, 25.98) --
	(246.27, 25.98) --
	(246.27, 25.98) --
	(246.34, 25.98) --
	(246.34, 25.98) --
	(246.34, 25.98) --
	(246.41, 25.96) --
	(246.41, 25.96) --
	(246.41, 25.96) --
	(246.48, 25.97) --
	(246.48, 25.97) --
	(246.48, 25.97) --
	(246.55, 25.96) --
	(246.55, 25.96) --
	(246.55, 25.96) --
	(246.61, 25.97) --
	(246.61, 25.97) --
	(246.61, 25.97) --
	(246.68, 25.97) --
	(246.68, 25.97) --
	(246.68, 25.97) --
	(246.75, 25.98) --
	(246.75, 25.98) --
	(246.75, 25.98) --
	(246.82, 25.96) --
	(246.82, 25.96) --
	(246.82, 25.96) --
	(246.89, 25.97) --
	(246.89, 25.97) --
	(246.89, 25.97) --
	(246.96, 25.95) --
	(246.96, 25.95) --
	(246.96, 25.95) --
	(247.03, 25.96) --
	(247.03, 25.96) --
	(247.03, 25.96) --
	(247.10, 25.96) --
	(247.10, 25.96) --
	(247.10, 25.96) --
	(247.17, 25.95) --
	(247.17, 25.95) --
	(247.17, 25.95) --
	(247.24, 25.95) --
	(247.24, 25.95) --
	(247.24, 25.95) --
	(247.31, 25.95) --
	(247.31, 25.95) --
	(247.31, 25.95) --
	(247.38, 25.95) --
	(247.38, 25.95) --
	(247.38, 25.95) --
	(247.45, 25.95) --
	(247.45, 25.95) --
	(247.45, 25.95) --
	(247.52, 25.95) --
	(247.52, 25.95) --
	(247.52, 25.95) --
	(247.59, 25.96) --
	(247.59, 25.96) --
	(247.59, 25.96) --
	(247.65, 25.96) --
	(247.65, 25.96) --
	(247.65, 25.96) --
	(247.72, 25.94) --
	(247.72, 25.94) --
	(247.72, 25.94) --
	(247.79, 25.95) --
	(247.79, 25.95) --
	(247.79, 25.95) --
	(247.86, 25.96) --
	(247.86, 25.96) --
	(247.86, 25.96) --
	(247.93, 25.94) --
	(247.93, 25.94) --
	(247.93, 25.94) --
	(248.00, 25.95) --
	(248.00, 25.95) --
	(248.00, 25.95) --
	(248.07, 25.95) --
	(248.07, 25.95) --
	(248.07, 25.95) --
	(248.14, 25.95) --
	(248.14, 25.95) --
	(248.14, 25.95) --
	(248.21, 25.95) --
	(248.21, 25.95) --
	(248.21, 25.95) --
	(248.28, 25.96) --
	(248.28, 25.96) --
	(248.28, 25.96) --
	(248.35, 25.98) --
	(248.35, 25.98) --
	(248.35, 25.98) --
	(248.42, 25.97) --
	(248.42, 25.97) --
	(248.42, 25.97) --
	(248.49, 25.97) --
	(248.49, 25.97) --
	(248.49, 25.97) --
	(248.56, 25.96) --
	(248.56, 25.96) --
	(248.56, 25.96) --
	(248.63, 25.97) --
	(248.63, 25.97) --
	(248.63, 25.97) --
	(248.69, 25.99) --
	(248.69, 25.99) --
	(248.69, 25.99) --
	(248.76, 25.99) --
	(248.76, 25.99) --
	(248.76, 25.99) --
	(248.83, 25.97) --
	(248.83, 25.97) --
	(248.83, 25.97) --
	(248.90, 25.96) --
	(248.90, 25.96) --
	(248.90, 25.96) --
	(248.97, 25.96) --
	(248.97, 25.96) --
	(248.97, 25.96) --
	(249.04, 25.96) --
	(249.04, 25.96) --
	(249.04, 25.96) --
	(249.11, 25.97) --
	(249.11, 25.97) --
	(249.11, 25.97) --
	(249.18, 26.03) --
	(249.18, 26.03) --
	(249.18, 26.03) --
	(249.25, 26.22) --
	(249.25, 26.22) --
	(249.25, 26.22) --
	(249.32, 26.36) --
	(249.32, 26.36) --
	(249.32, 26.36) --
	(249.39, 26.25) --
	(249.39, 26.25) --
	(249.39, 26.25) --
	(249.46, 26.12) --
	(249.46, 26.12) --
	(249.46, 26.12) --
	(249.52, 26.07) --
	(249.52, 26.07) --
	(249.52, 26.07) --
	(249.59, 26.02) --
	(249.59, 26.02) --
	(249.59, 26.02) --
	(249.66, 25.99) --
	(249.66, 25.99) --
	(249.66, 25.99) --
	(249.73, 25.98) --
	(249.73, 25.98) --
	(249.73, 25.98) --
	(249.80, 25.97) --
	(249.80, 25.97) --
	(249.80, 25.97) --
	(249.87, 25.96) --
	(249.87, 25.96) --
	(249.87, 25.96) --
	(249.94, 25.97) --
	(249.94, 25.97) --
	(249.94, 25.97) --
	(250.01, 25.96) --
	(250.01, 25.96) --
	(250.01, 25.96) --
	(250.08, 25.97) --
	(250.08, 25.97) --
	(250.08, 25.97) --
	(250.15, 26.02) --
	(250.15, 26.02) --
	(250.15, 26.02) --
	(250.22, 26.05) --
	(250.22, 26.05) --
	(250.22, 26.05) --
	(250.28, 26.02) --
	(250.28, 26.02) --
	(250.28, 26.02) --
	(250.35, 25.99) --
	(250.35, 25.99) --
	(250.35, 25.99) --
	(250.42, 25.98) --
	(250.42, 25.98) --
	(250.42, 25.98) --
	(250.49, 25.97) --
	(250.49, 25.97) --
	(250.49, 25.97) --
	(250.56, 25.95) --
	(250.56, 25.95) --
	(250.56, 25.95) --
	(250.63, 25.95) --
	(250.63, 25.95) --
	(250.63, 25.95) --
	(250.70, 25.95) --
	(250.70, 25.95) --
	(250.70, 25.95) --
	(250.77, 25.94) --
	(250.77, 25.94) --
	(250.77, 25.94) --
	(250.84, 25.94) --
	(250.84, 25.94) --
	(250.84, 25.94) --
	(250.91, 25.94) --
	(250.91, 25.94) --
	(250.91, 25.94) --
	(250.97, 25.94) --
	(250.97, 25.94) --
	(250.97, 25.94) --
	(251.04, 25.94) --
	(251.04, 25.94) --
	(251.04, 25.94) --
	(251.11, 25.94) --
	(251.11, 25.94) --
	(251.11, 25.94) --
	(251.18, 25.94) --
	(251.18, 25.94) --
	(251.18, 25.94) --
	(251.25, 25.94) --
	(251.25, 25.94) --
	(251.25, 25.94) --
	(251.32, 25.93) --
	(251.32, 25.93) --
	(251.32, 25.93) --
	(251.39, 25.94) --
	(251.39, 25.94) --
	(251.39, 25.94) --
	(251.46, 25.94) --
	(251.46, 25.94) --
	(251.46, 25.94) --
	(251.53, 25.93) --
	(251.53, 25.93) --
	(251.53, 25.93) --
	(251.60, 25.94) --
	(251.60, 25.94) --
	(251.60, 25.94) --
	(251.66, 25.94) --
	(251.66, 25.94) --
	(251.66, 25.94) --
	(251.73, 25.93) --
	(251.73, 25.93) --
	(251.73, 25.93) --
	(251.80, 25.94) --
	(251.80, 25.94) --
	(251.80, 25.94) --
	(251.87, 25.94) --
	(251.87, 25.94) --
	(251.87, 25.94) --
	(251.94, 25.93) --
	(251.94, 25.93) --
	(251.94, 25.93) --
	(252.01, 25.94) --
	(252.01, 25.94) --
	(252.01, 25.94) --
	(252.08, 25.93) --
	(252.08, 25.93) --
	(252.08, 25.93) --
	(252.15, 25.93) --
	(252.15, 25.93) --
	(252.15, 25.93) --
	(252.22, 25.94) --
	(252.22, 25.94) --
	(252.22, 25.94) --
	(252.29, 25.93) --
	(252.29, 25.93) --
	(252.29, 25.93) --
	(252.35, 25.94) --
	(252.35, 25.94) --
	(252.35, 25.94) --
	(252.42, 25.94) --
	(252.42, 25.94) --
	(252.42, 25.94) --
	(252.49, 25.93) --
	(252.49, 25.93) --
	(252.49, 25.93) --
	(252.56, 25.94) --
	(252.56, 25.94) --
	(252.56, 25.94) --
	(252.63, 25.94) --
	(252.63, 25.94) --
	(252.63, 25.94) --
	(252.70, 25.92) --
	(252.70, 25.92) --
	(252.70, 25.92) --
	(252.77, 25.93) --
	(252.77, 25.93) --
	(252.77, 25.93) --
	(252.84, 25.93) --
	(252.84, 25.93) --
	(252.84, 25.93) --
	(252.91, 25.92) --
	(252.91, 25.92) --
	(252.91, 25.92) --
	(252.97, 25.94) --
	(252.97, 25.94) --
	(252.97, 25.94) --
	(253.04, 25.93) --
	(253.04, 25.93) --
	(253.04, 25.93) --
	(253.11, 25.93) --
	(253.11, 25.93) --
	(253.11, 25.93) --
	(253.18, 25.94) --
	(253.18, 25.94) --
	(253.18, 25.94) --
	(253.25, 25.92) --
	(253.25, 25.92) --
	(253.25, 25.92) --
	(253.32, 25.93) --
	(253.32, 25.93) --
	(253.32, 25.93) --
	(253.39, 25.93) --
	(253.39, 25.93) --
	(253.39, 25.93) --
	(253.46, 25.93) --
	(253.46, 25.93) --
	(253.46, 25.93) --
	(253.53, 25.93) --
	(253.53, 25.93) --
	(253.53, 25.93) --
	(253.59, 25.93) --
	(253.59, 25.93) --
	(253.59, 25.93) --
	(253.66, 25.94) --
	(253.66, 25.94) --
	(253.66, 25.94) --
	(253.73, 25.95) --
	(253.73, 25.95) --
	(253.73, 25.95) --
	(253.80, 25.95) --
	(253.80, 25.95) --
	(253.80, 25.95) --
	(253.87, 25.94) --
	(253.87, 25.94) --
	(253.87, 25.94) --
	(253.94, 25.94) --
	(253.94, 25.94) --
	(253.94, 25.94) --
	(254.01, 25.94) --
	(254.01, 25.94) --
	(254.01, 25.94) --
	(254.08, 25.93) --
	(254.08, 25.93) --
	(254.08, 25.93) --
	(254.14, 25.93) --
	(254.14, 25.93) --
	(254.14, 25.93) --
	(254.21, 25.92) --
	(254.21, 25.92) --
	(254.21, 25.92) --
	(254.28, 25.93) --
	(254.28, 25.93) --
	(254.28, 25.93) --
	(254.35, 25.91) --
	(254.35, 25.91) --
	(254.35, 25.91) --
	(254.42, 25.92) --
	(254.42, 25.92) --
	(254.42, 25.92) --
	(254.49, 25.93) --
	(254.49, 25.93) --
	(254.49, 25.93) --
	(254.56, 25.92) --
	(254.56, 25.92) --
	(254.56, 25.92) --
	(254.63, 25.92) --
	(254.63, 25.92) --
	(254.63, 25.92) --
	(254.70, 25.93) --
	(254.70, 25.93) --
	(254.70, 25.93) --
	(254.76, 25.91) --
	(254.76, 25.91) --
	(254.76, 25.91) --
	(254.83, 25.91) --
	(254.83, 25.91) --
	(254.83, 25.91) --
	(254.90, 25.93) --
	(254.90, 25.93) --
	(254.90, 25.93) --
	(254.97, 25.92) --
	(254.97, 25.92) --
	(254.97, 25.92) --
	(255.04, 25.92) --
	(255.04, 25.92) --
	(255.04, 25.92) --
	(255.11, 25.91) --
	(255.11, 25.91) --
	(255.11, 25.91) --
	(255.18, 25.92) --
	(255.18, 25.92) --
	(255.18, 25.92) --
	(255.24, 25.92) --
	(255.24, 25.92) --
	(255.24, 25.92) --
	(255.31, 25.92) --
	(255.31, 25.92) --
	(255.31, 25.92) --
	(255.38, 25.92) --
	(255.38, 25.92) --
	(255.38, 25.92) --
	(255.45, 25.93) --
	(255.45, 25.93) --
	(255.45, 25.93) --
	(255.52, 25.92) --
	(255.52, 25.92) --
	(255.52, 25.92) --
	(255.59, 25.92) --
	(255.59, 25.92) --
	(255.59, 25.92) --
	(255.66, 25.91) --
	(255.66, 25.91) --
	(255.66, 25.91) --
	(255.73, 25.91) --
	(255.73, 25.91) --
	(255.73, 25.91) --
	(255.79, 25.92) --
	(255.79, 25.92) --
	(255.79, 25.92) --
	(255.86, 25.91) --
	(255.86, 25.91) --
	(255.86, 25.91) --
	(255.93, 25.91) --
	(255.93, 25.91) --
	(255.93, 25.91) --
	(256.00, 25.92) --
	(256.00, 25.92) --
	(256.00, 25.92) --
	(256.07, 25.91) --
	(256.07, 25.91) --
	(256.07, 25.91) --
	(256.14, 25.91) --
	(256.14, 25.91) --
	(256.14, 25.91) --
	(256.20, 25.92) --
	(256.20, 25.92) --
	(256.20, 25.92) --
	(256.27, 25.91) --
	(256.27, 25.91) --
	(256.27, 25.91) --
	(256.34, 25.92) --
	(256.34, 25.92) --
	(256.34, 25.92) --
	(256.41, 25.93) --
	(256.41, 25.93) --
	(256.41, 25.93) --
	(256.48, 25.91) --
	(256.48, 25.91) --
	(256.48, 25.91) --
	(256.55, 25.92) --
	(256.55, 25.92) --
	(256.55, 25.92) --
	(256.62, 25.91) --
	(256.62, 25.91) --
	(256.62, 25.91) --
	(256.69, 25.92) --
	(256.69, 25.92) --
	(256.69, 25.92) --
	(256.75, 25.92) --
	(256.75, 25.92) --
	(256.75, 25.92) --
	(256.82, 25.92) --
	(256.82, 25.92) --
	(256.82, 25.92) --
	(256.89, 25.92) --
	(256.89, 25.92) --
	(256.89, 25.92) --
	(256.96, 25.92) --
	(256.96, 25.92) --
	(256.96, 25.92) --
	(257.03, 25.92) --
	(257.03, 25.92) --
	(257.03, 25.92) --
	(257.10, 25.91) --
	(257.10, 25.91) --
	(257.10, 25.91) --
	(257.17, 25.92) --
	(257.17, 25.92) --
	(257.17, 25.92) --
	(257.24, 25.92) --
	(257.24, 25.92) --
	(257.24, 25.92) --
	(257.30, 25.92) --
	(257.30, 25.92) --
	(257.30, 25.92) --
	(257.37, 25.92) --
	(257.37, 25.92) --
	(257.37, 25.92) --
	(257.44, 25.91) --
	(257.44, 25.91) --
	(257.44, 25.91) --
	(257.51, 25.92) --
	(257.51, 25.92) --
	(257.51, 25.92) --
	(257.58, 25.91) --
	(257.58, 25.91) --
	(257.58, 25.91) --
	(257.65, 25.92) --
	(257.65, 25.92) --
	(257.65, 25.92) --
	(257.71, 25.92) --
	(257.71, 25.92) --
	(257.71, 25.92) --
	(257.78, 25.92) --
	(257.78, 25.92) --
	(257.78, 25.92) --
	(257.85, 25.92) --
	(257.85, 25.92) --
	(257.85, 25.92) --
	(257.92, 25.92) --
	(257.92, 25.92) --
	(257.92, 25.92) --
	(257.99, 25.91) --
	(257.99, 25.91) --
	(257.99, 25.91) --
	(258.06, 25.92) --
	(258.06, 25.92) --
	(258.06, 25.92) --
	(258.12, 25.92) --
	(258.12, 25.92) --
	(258.12, 25.92) --
	(258.19, 25.92) --
	(258.19, 25.92) --
	(258.19, 25.92) --
	(258.26, 25.92) --
	(258.26, 25.92) --
	(258.26, 25.92) --
	(258.33, 25.91) --
	(258.33, 25.91) --
	(258.33, 25.91) --
	(258.40, 25.92) --
	(258.40, 25.92) --
	(258.40, 25.92) --
	(258.47, 25.92) --
	(258.47, 25.92) --
	(258.47, 25.92) --
	(258.54, 25.92) --
	(258.54, 25.92) --
	(258.54, 25.92) --
	(258.60, 25.92) --
	(258.60, 25.92) --
	(258.60, 25.92) --
	(258.67, 25.91) --
	(258.67, 25.91) --
	(258.67, 25.91) --
	(258.74, 25.92) --
	(258.74, 25.92) --
	(258.74, 25.92) --
	(258.81, 25.94) --
	(258.81, 25.94) --
	(258.81, 25.94) --
	(258.88, 25.96) --
	(258.88, 25.96) --
	(258.88, 25.96) --
	(258.95, 25.96) --
	(258.95, 25.96) --
	(258.95, 25.96) --
	(259.01, 25.95) --
	(259.01, 25.95) --
	(259.01, 25.95) --
	(259.08, 25.93) --
	(259.08, 25.93) --
	(259.08, 25.93) --
	(259.15, 25.93) --
	(259.15, 25.93) --
	(259.15, 25.93) --
	(259.22, 25.93) --
	(259.22, 25.93) --
	(259.22, 25.93) --
	(259.29, 25.91) --
	(259.29, 25.91) --
	(259.29, 25.91) --
	(259.36, 25.92) --
	(259.36, 25.92) --
	(259.36, 25.92) --
	(259.42, 25.92) --
	(259.42, 25.92) --
	(259.42, 25.92) --
	(259.49, 25.91) --
	(259.49, 25.91) --
	(259.49, 25.91) --
	(259.56, 25.91) --
	(259.56, 25.91) --
	(259.56, 25.91) --
	(259.63, 25.91) --
	(259.63, 25.91) --
	(259.63, 25.91) --
	(259.70, 25.91) --
	(259.70, 25.91) --
	(259.70, 25.91) --
	(259.77, 25.92) --
	(259.77, 25.92) --
	(259.77, 25.92) --
	(259.83, 25.91) --
	(259.83, 25.91) --
	(259.83, 25.91) --
	(259.90, 25.92) --
	(259.90, 25.92) --
	(259.90, 25.92) --
	(259.97, 25.92) --
	(259.97, 25.92) --
	(259.97, 25.92) --
	(260.04, 25.93) --
	(260.04, 25.93) --
	(260.04, 25.93) --
	(260.11, 25.97) --
	(260.11, 25.97) --
	(260.11, 25.97) --
	(260.18, 26.00) --
	(260.18, 26.00) --
	(260.18, 26.00) --
	(260.25, 25.98) --
	(260.25, 25.98) --
	(260.25, 25.98) --
	(260.31, 25.96) --
	(260.31, 25.96) --
	(260.31, 25.96) --
	(260.38, 25.95) --
	(260.38, 25.95) --
	(260.38, 25.95) --
	(260.45, 25.93) --
	(260.45, 25.93) --
	(260.45, 25.93) --
	(260.52, 25.94) --
	(260.52, 25.94) --
	(260.52, 25.94) --
	(260.59, 25.92) --
	(260.59, 25.92) --
	(260.59, 25.92) --
	(260.66, 25.92) --
	(260.66, 25.92) --
	(260.66, 25.92) --
	(260.72, 25.92) --
	(260.72, 25.92) --
	(260.72, 25.92) --
	(260.79, 25.91) --
	(260.79, 25.91) --
	(260.79, 25.91) --
	(260.86, 25.92) --
	(260.86, 25.92) --
	(260.86, 25.92) --
	(260.93, 25.92) --
	(260.93, 25.92) --
	(260.93, 25.92) --
	(261.00, 25.91) --
	(261.00, 25.91) --
	(261.00, 25.91) --
	(261.06, 25.92) --
	(261.06, 25.92) --
	(261.06, 25.92) --
	(261.13, 25.91) --
	(261.13, 25.91) --
	(261.13, 25.91) --
	(261.20, 25.91) --
	(261.20, 25.91) --
	(261.20, 25.91) --
	(261.27, 25.92) --
	(261.27, 25.92) --
	(261.27, 25.92) --
	(261.34, 25.90) --
	(261.34, 25.90) --
	(261.34, 25.90) --
	(261.40, 25.91) --
	(261.40, 25.91) --
	(261.40, 25.91) --
	(261.47, 25.92) --
	(261.47, 25.92) --
	(261.47, 25.92) --
	(261.54, 25.91) --
	(261.54, 25.91) --
	(261.54, 25.91) --
	(261.61, 25.91) --
	(261.61, 25.91) --
	(261.61, 25.91) --
	(261.68, 25.92) --
	(261.68, 25.92) --
	(261.68, 25.92) --
	(261.75, 25.91) --
	(261.75, 25.91) --
	(261.75, 25.91) --
	(261.81, 25.92) --
	(261.81, 25.92) --
	(261.81, 25.92) --
	(261.88, 25.91) --
	(261.88, 25.91) --
	(261.88, 25.91) --
	(261.95, 25.90) --
	(261.95, 25.90) --
	(261.95, 25.90) --
	(262.02, 25.91) --
	(262.02, 25.91) --
	(262.02, 25.91) --
	(262.09, 25.91) --
	(262.09, 25.91) --
	(262.09, 25.91) --
	(262.16, 25.91) --
	(262.16, 25.91) --
	(262.16, 25.91) --
	(262.22, 25.92) --
	(262.22, 25.92) --
	(262.22, 25.92) --
	(262.29, 25.91) --
	(262.29, 25.91) --
	(262.29, 25.91) --
	(262.36, 25.92) --
	(262.36, 25.92) --
	(262.36, 25.92) --
	(262.43, 25.92) --
	(262.43, 25.92) --
	(262.43, 25.92) --
	(262.50, 25.91) --
	(262.50, 25.91) --
	(262.50, 25.91) --
	(262.56, 25.91) --
	(262.56, 25.91) --
	(262.56, 25.91) --
	(262.63, 25.92) --
	(262.63, 25.92) --
	(262.63, 25.92) --
	(262.70, 25.91) --
	(262.70, 25.91) --
	(262.70, 25.91) --
	(262.77, 25.91) --
	(262.77, 25.91) --
	(262.77, 25.91) --
	(262.84, 25.91) --
	(262.84, 25.91) --
	(262.84, 25.91) --
	(262.90, 25.91) --
	(262.90, 25.91) --
	(262.90, 25.91) --
	(262.97, 25.91) --
	(262.97, 25.91) --
	(262.97, 25.91) --
	(263.04, 25.91) --
	(263.04, 25.91) --
	(263.04, 25.91) --
	(263.11, 25.92) --
	(263.11, 25.92) --
	(263.11, 25.92) --
	(263.18, 25.92) --
	(263.18, 25.92) --
	(263.18, 25.92) --
	(263.25, 25.91) --
	(263.25, 25.91) --
	(263.25, 25.91) --
	(263.31, 25.91) --
	(263.31, 25.91) --
	(263.31, 25.91) --
	(263.38, 25.92) --
	(263.38, 25.92) --
	(263.38, 25.92) --
	(263.45, 25.91) --
	(263.45, 25.91) --
	(263.45, 25.91) --
	(263.52, 25.91) --
	(263.52, 25.91) --
	(263.52, 25.91) --
	(263.58, 25.91) --
	(263.58, 25.91) --
	(263.58, 25.91) --
	(263.65, 25.91) --
	(263.65, 25.91) --
	(263.65, 25.91) --
	(263.72, 25.92) --
	(263.72, 25.92) --
	(263.72, 25.92) --
	(263.79, 25.91) --
	(263.79, 25.91) --
	(263.79, 25.91) --
	(263.86, 25.92) --
	(263.86, 25.92) --
	(263.86, 25.92) --
	(263.93, 25.92) --
	(263.93, 25.92) --
	(263.93, 25.92) --
	(263.99, 25.91) --
	(263.99, 25.91) --
	(263.99, 25.91) --
	(264.06, 25.92) --
	(264.06, 25.92) --
	(264.06, 25.92) --
	(264.13, 25.92) --
	(264.13, 25.92) --
	(264.13, 25.92) --
	(264.20, 25.91) --
	(264.20, 25.91) --
	(264.20, 25.91) --
	(264.26, 25.91) --
	(264.26, 25.91) --
	(264.26, 25.91) --
	(264.33, 25.91) --
	(264.33, 25.91) --
	(264.33, 25.91) --
	(264.40, 25.92) --
	(264.40, 25.92) --
	(264.40, 25.92) --
	(264.47, 25.92) --
	(264.47, 25.92) --
	(264.47, 25.92) --
	(264.54, 25.91) --
	(264.54, 25.91) --
	(264.54, 25.91) --
	(264.61, 25.92) --
	(264.61, 25.92) --
	(264.61, 25.92) --
	(264.67, 25.92) --
	(264.67, 25.92) --
	(264.67, 25.92) --
	(264.74, 25.92) --
	(264.74, 25.92) --
	(264.74, 25.92) --
	(264.81, 25.91) --
	(264.81, 25.91) --
	(264.81, 25.91) --
	(264.88, 25.92) --
	(264.88, 25.92) --
	(264.88, 25.92) --
	(264.94, 25.91) --
	(264.94, 25.91) --
	(264.94, 25.91) --
	(265.01, 25.92) --
	(265.01, 25.92) --
	(265.01, 25.92) --
	(265.08, 25.91) --
	(265.08, 25.91) --
	(265.08, 25.91) --
	(265.15, 25.91) --
	(265.15, 25.91) --
	(265.15, 25.91) --
	(265.22, 25.91) --
	(265.22, 25.91) --
	(265.22, 25.91) --
	(265.28, 25.90) --
	(265.28, 25.90) --
	(265.28, 25.90) --
	(265.35, 25.91) --
	(265.35, 25.91) --
	(265.35, 25.91) --
	(265.42, 25.92) --
	(265.42, 25.92) --
	(265.42, 25.92) --
	(265.49, 25.90) --
	(265.49, 25.90) --
	(265.49, 25.90) --
	(265.56, 25.91) --
	(265.56, 25.91) --
	(265.56, 25.91) --
	(265.62, 25.92) --
	(265.62, 25.92) --
	(265.62, 25.92) --
	(265.69, 25.91) --
	(265.69, 25.91) --
	(265.69, 25.91) --
	(265.76, 25.91) --
	(265.76, 25.91) --
	(265.76, 25.91) --
	(265.83, 25.91) --
	(265.83, 25.91) --
	(265.83, 25.91) --
	(265.90, 25.91) --
	(265.90, 25.91) --
	(265.90, 25.91) --
	(265.96, 25.91) --
	(265.96, 25.91) --
	(265.96, 25.91) --
	(266.03, 25.91) --
	(266.03, 25.91) --
	(266.03, 25.91) --
	(266.10, 25.91) --
	(266.10, 25.91) --
	(266.10, 25.91) --
	(266.17, 25.92) --
	(266.17, 25.92) --
	(266.17, 25.92) --
	(266.23, 25.91) --
	(266.23, 25.91) --
	(266.23, 25.91) --
	(266.30, 25.91) --
	(266.30, 25.91) --
	(266.30, 25.91) --
	(266.37, 25.93) --
	(266.37, 25.93) --
	(266.37, 25.93) --
	(266.44, 25.91) --
	(266.44, 25.91) --
	(266.44, 25.91) --
	(266.51, 25.94) --
	(266.51, 25.94) --
	(266.51, 25.94) --
	(266.57, 25.96) --
	(266.57, 25.96) --
	(266.57, 25.96) --
	(266.64, 25.99) --
	(266.64, 25.99) --
	(266.64, 25.99) --
	(266.71, 26.06) --
	(266.71, 26.06) --
	(266.71, 26.06) --
	(266.78, 26.08) --
	(266.78, 26.08) --
	(266.78, 26.08) --
	(266.84, 26.07) --
	(266.84, 26.07) --
	(266.84, 26.07) --
	(266.91, 26.04) --
	(266.91, 26.04) --
	(266.91, 26.04) --
	(266.98, 25.99) --
	(266.98, 25.99) --
	(266.98, 25.99) --
	(267.05, 25.98) --
	(267.05, 25.98) --
	(267.05, 25.98) --
	(267.11, 25.96) --
	(267.11, 25.96) --
	(267.11, 25.96) --
	(267.18, 25.93) --
	(267.18, 25.93) --
	(267.18, 25.93) --
	(267.25, 25.93) --
	(267.25, 25.93) --
	(267.25, 25.93) --
	(267.32, 25.92) --
	(267.32, 25.92) --
	(267.32, 25.92) --
	(267.39, 25.92) --
	(267.39, 25.92) --
	(267.39, 25.92) --
	(267.45, 25.91) --
	(267.45, 25.91) --
	(267.45, 25.91) --
	(267.52, 25.91) --
	(267.52, 25.91) --
	(267.52, 25.91) --
	(267.59, 25.92) --
	(267.59, 25.92) --
	(267.59, 25.92) --
	(267.66, 25.92) --
	(267.66, 25.92) --
	(267.66, 25.92) --
	(267.72, 25.90) --
	(267.72, 25.90) --
	(267.72, 25.90) --
	(267.79, 25.91) --
	(267.79, 25.91) --
	(267.79, 25.91) --
	(267.86, 25.91) --
	(267.86, 25.91) --
	(267.86, 25.91) --
	(267.93, 25.91) --
	(267.93, 25.91) --
	(267.93, 25.91) --
	(267.99, 25.91) --
	(267.99, 25.91) --
	(267.99, 25.91) --
	(268.06, 25.91) --
	(268.06, 25.91) --
	(268.06, 25.91) --
	(268.13, 25.91) --
	(268.13, 25.91) --
	(268.13, 25.91) --
	(268.20, 25.91) --
	(268.20, 25.91) --
	(268.20, 25.91) --
	(268.27, 25.90) --
	(268.27, 25.90) --
	(268.27, 25.90) --
	(268.33, 25.91) --
	(268.33, 25.91) --
	(268.33, 25.91) --
	(268.40, 25.92) --
	(268.40, 25.92) --
	(268.40, 25.92) --
	(268.47, 25.91) --
	(268.47, 25.91) --
	(268.47, 25.91) --
	(268.54, 25.92) --
	(268.54, 25.92) --
	(268.54, 25.92) --
	(268.60, 25.91) --
	(268.60, 25.91) --
	(268.60, 25.91) --
	(268.67, 25.91) --
	(268.67, 25.91) --
	(268.67, 25.91) --
	(268.74, 25.90) --
	(268.74, 25.90) --
	(268.74, 25.90) --
	(268.81, 25.91) --
	(268.81, 25.91) --
	(268.81, 25.91) --
	(268.87, 25.90) --
	(268.87, 25.90) --
	(268.87, 25.90) --
	(268.94, 25.92) --
	(268.94, 25.92) --
	(268.94, 25.92) --
	(269.01, 25.90) --
	(269.01, 25.90) --
	(269.01, 25.90) --
	(269.08, 25.91) --
	(269.08, 25.91) --
	(269.08, 25.91) --
	(269.14, 25.92) --
	(269.14, 25.92) --
	(269.14, 25.92) --
	(269.21, 25.91) --
	(269.21, 25.91) --
	(269.21, 25.91) --
	(269.28, 25.91) --
	(269.28, 25.91) --
	(269.28, 25.91) --
	(269.35, 25.91) --
	(269.35, 25.91) --
	(269.35, 25.91) --
	(269.42, 25.91) --
	(269.42, 25.91) --
	(269.42, 25.91) --
	(269.48, 25.91) --
	(269.48, 25.91) --
	(269.48, 25.91) --
	(269.55, 25.91) --
	(269.55, 25.91) --
	(269.55, 25.91) --
	(269.62, 25.91) --
	(269.62, 25.91) --
	(269.62, 25.91) --
	(269.69, 25.91) --
	(269.69, 25.91) --
	(269.69, 25.91) --
	(269.75, 25.91) --
	(269.75, 25.91) --
	(269.75, 25.91) --
	(269.82, 25.90) --
	(269.82, 25.90) --
	(269.82, 25.90) --
	(269.89, 25.91) --
	(269.89, 25.91) --
	(269.89, 25.91) --
	(269.95, 25.91) --
	(269.95, 25.91) --
	(269.95, 25.91) --
	(270.02, 25.91) --
	(270.02, 25.91) --
	(270.02, 25.91) --
	(270.09, 25.91) --
	(270.09, 25.91) --
	(270.09, 25.91) --
	(270.16, 25.90) --
	(270.16, 25.90) --
	(270.16, 25.90) --
	(270.22, 25.91) --
	(270.22, 25.91) --
	(270.22, 25.91) --
	(270.29, 25.91) --
	(270.29, 25.91) --
	(270.29, 25.91) --
	(270.36, 25.91) --
	(270.36, 25.91) --
	(270.36, 25.91) --
	(270.43, 25.91) --
	(270.43, 25.91) --
	(270.43, 25.91) --
	(270.49, 25.91) --
	(270.49, 25.91) --
	(270.49, 25.91) --
	(270.56, 25.91) --
	(270.56, 25.91) --
	(270.56, 25.91) --
	(270.63, 25.91) --
	(270.63, 25.91) --
	(270.63, 25.91) --
	(270.70, 25.91) --
	(270.70, 25.91) --
	(270.70, 25.91) --
	(270.76, 25.91) --
	(270.76, 25.91) --
	(270.76, 25.91) --
	(270.83, 25.92) --
	(270.83, 25.92) --
	(270.83, 25.92) --
	(270.90, 25.91) --
	(270.90, 25.91) --
	(270.90, 25.91) --
	(270.97, 25.91) --
	(270.97, 25.91) --
	(270.97, 25.91) --
	(271.03, 25.91) --
	(271.03, 25.91) --
	(271.03, 25.91) --
	(271.10, 25.90) --
	(271.10, 25.90) --
	(271.10, 25.90) --
	(271.17, 25.91) --
	(271.17, 25.91) --
	(271.17, 25.91) --
	(271.24, 25.91) --
	(271.24, 25.91) --
	(271.24, 25.91) --
	(271.30, 25.91) --
	(271.30, 25.91) --
	(271.30, 25.91) --
	(271.37, 25.91) --
	(271.37, 25.91) --
	(271.37, 25.91) --
	(271.44, 25.90) --
	(271.44, 25.90) --
	(271.44, 25.90) --
	(271.51, 25.91) --
	(271.51, 25.91) --
	(271.51, 25.91) --
	(271.57, 25.92) --
	(271.57, 25.92) --
	(271.57, 25.92) --
	(271.64, 25.90) --
	(271.64, 25.90) --
	(271.64, 25.90) --
	(271.71, 25.90) --
	(271.71, 25.90) --
	(271.71, 25.90) --
	(271.77, 25.92) --
	(271.77, 25.92) --
	(271.77, 25.92) --
	(271.84, 25.91) --
	(271.84, 25.91) --
	(271.84, 25.91) --
	(271.91, 25.91) --
	(271.91, 25.91) --
	(271.91, 25.91) --
	(271.98, 25.92) --
	(271.98, 25.92) --
	(271.98, 25.92) --
	(272.04, 25.92) --
	(272.04, 25.92) --
	(272.04, 25.92) --
	(272.11, 25.92) --
	(272.11, 25.92) --
	(272.11, 25.92) --
	(272.18, 25.91) --
	(272.18, 25.91) --
	(272.18, 25.91) --
	(272.25, 25.91) --
	(272.25, 25.91) --
	(272.25, 25.91) --
	(272.31, 25.92) --
	(272.31, 25.92) --
	(272.31, 25.92) --
	(272.38, 25.91) --
	(272.38, 25.91) --
	(272.38, 25.91) --
	(272.45, 25.91) --
	(272.45, 25.91) --
	(272.45, 25.91) --
	(272.52, 25.91) --
	(272.52, 25.91) --
	(272.52, 25.91) --
	(272.58, 25.91) --
	(272.58, 25.91) --
	(272.58, 25.91) --
	(272.65, 25.91) --
	(272.65, 25.91) --
	(272.65, 25.91) --
	(272.72, 25.91) --
	(272.72, 25.91) --
	(272.72, 25.91) --
	(272.79, 25.91) --
	(272.79, 25.91) --
	(272.79, 25.91) --
	(272.85, 25.92) --
	(272.85, 25.92) --
	(272.85, 25.92) --
	(272.92, 25.91) --
	(272.92, 25.91) --
	(272.92, 25.91) --
	(272.99, 25.92) --
	(272.99, 25.92) --
	(272.99, 25.92) --
	(273.05, 25.91) --
	(273.05, 25.91) --
	(273.05, 25.91) --
	(273.12, 25.91) --
	(273.12, 25.91) --
	(273.12, 25.91) --
	(273.19, 25.91) --
	(273.19, 25.91) --
	(273.19, 25.91) --
	(273.26, 25.92) --
	(273.26, 25.92) --
	(273.26, 25.92) --
	(273.32, 25.91) --
	(273.32, 25.91) --
	(273.32, 25.91) --
	(273.39, 25.91) --
	(273.39, 25.91) --
	(273.39, 25.91) --
	(273.46, 25.92) --
	(273.46, 25.92) --
	(273.46, 25.92) --
	(273.53, 25.91) --
	(273.53, 25.91) --
	(273.53, 25.91) --
	(273.59, 25.91) --
	(273.59, 25.91) --
	(273.59, 25.91) --
	(273.66, 25.92) --
	(273.66, 25.92) --
	(273.66, 25.92) --
	(273.73, 25.91) --
	(273.73, 25.91) --
	(273.73, 25.91) --
	(273.79, 25.92) --
	(273.79, 25.92) --
	(273.79, 25.92) --
	(273.86, 25.90) --
	(273.86, 25.90) --
	(273.86, 25.90) --
	(273.93, 25.91) --
	(273.93, 25.91) --
	(273.93, 25.91) --
	(274.00, 25.92) --
	(274.00, 25.92) --
	(274.00, 25.92) --
	(274.06, 25.91) --
	(274.06, 25.91) --
	(274.06, 25.91) --
	(274.13, 25.91) --
	(274.13, 25.91) --
	(274.13, 25.91) --
	(274.20, 25.92) --
	(274.20, 25.92) --
	(274.20, 25.92) --
	(274.26, 25.90) --
	(274.26, 25.90) --
	(274.26, 25.90) --
	(274.33, 25.91) --
	(274.33, 25.91) --
	(274.33, 25.91) --
	(274.40, 25.91) --
	(274.40, 25.91) --
	(274.40, 25.91) --
	(274.46, 25.92) --
	(274.46, 25.92) --
	(274.46, 25.92) --
	(274.53, 25.91) --
	(274.53, 25.91) --
	(274.53, 25.91) --
	(274.60, 25.90) --
	(274.60, 25.90) --
	(274.60, 25.90) --
	(274.67, 25.92) --
	(274.67, 25.92) --
	(274.67, 25.92) --
	(274.73, 25.92) --
	(274.73, 25.92) --
	(274.73, 25.92) --
	(274.80, 25.91) --
	(274.80, 25.91) --
	(274.80, 25.91) --
	(274.87, 25.92) --
	(274.87, 25.92) --
	(274.87, 25.92) --
	(274.93, 25.92) --
	(274.93, 25.92) --
	(274.93, 25.92) --
	(275.00, 25.91) --
	(275.00, 25.91) --
	(275.00, 25.91) --
	(275.07, 25.91) --
	(275.07, 25.91) --
	(275.07, 25.91) --
	(275.14, 25.91) --
	(275.14, 25.91) --
	(275.14, 25.91) --
	(275.20, 25.91) --
	(275.20, 25.91) --
	(275.20, 25.91) --
	(275.27, 25.91) --
	(275.27, 25.91) --
	(275.27, 25.91) --
	(275.34, 25.91) --
	(275.34, 25.91) --
	(275.34, 25.91) --
	(275.40, 25.92) --
	(275.40, 25.92) --
	(275.40, 25.92) --
	(275.47, 25.92) --
	(275.47, 25.92) --
	(275.47, 25.92) --
	(275.54, 25.91) --
	(275.54, 25.91) --
	(275.54, 25.91) --
	(275.60, 25.92) --
	(275.60, 25.92) --
	(275.61, 25.92) --
	(275.67, 25.92) --
	(275.67, 25.92) --
	(275.67, 25.92) --
	(275.74, 25.90) --
	(275.74, 25.90) --
	(275.74, 25.90) --
	(275.81, 25.92) --
	(275.81, 25.92) --
	(275.81, 25.92) --
	(275.87, 25.91) --
	(275.87, 25.91) --
	(275.87, 25.91) --
	(275.94, 25.92) --
	(275.94, 25.92) --
	(275.94, 25.92) --
	(276.01, 25.91) --
	(276.01, 25.91) --
	(276.01, 25.91) --
	(276.07, 25.91) --
	(276.07, 25.91) --
	(276.07, 25.91) --
	(276.14, 25.91) --
	(276.14, 25.91) --
	(276.14, 25.91) --
	(276.21, 25.92) --
	(276.21, 25.92) --
	(276.21, 25.92) --
	(276.28, 25.91) --
	(276.28, 25.91) --
	(276.28, 25.91) --
	(276.34, 25.91) --
	(276.34, 25.91) --
	(276.34, 25.91) --
	(276.41, 25.92) --
	(276.41, 25.92) --
	(276.41, 25.92) --
	(276.48, 25.92) --
	(276.48, 25.92) --
	(276.48, 25.92) --
	(276.54, 25.92) --
	(276.54, 25.92) --
	(276.54, 25.92) --
	(276.61, 25.92) --
	(276.61, 25.92) --
	(276.61, 25.92) --
	(276.68, 25.92) --
	(276.68, 25.92) --
	(276.68, 25.92) --
	(276.74, 25.92) --
	(276.74, 25.92) --
	(276.74, 25.92) --
	(276.81, 25.92) --
	(276.81, 25.92) --
	(276.81, 25.92) --
	(276.88, 25.92) --
	(276.88, 25.92) --
	(276.88, 25.92) --
	(276.94, 25.92) --
	(276.94, 25.92) --
	(276.94, 25.92) --
	(277.01, 25.92) --
	(277.01, 25.92) --
	(277.01, 25.92) --
	(277.08, 25.92) --
	(277.08, 25.92) --
	(277.08, 25.92) --
	(277.15, 25.92) --
	(277.15, 25.92) --
	(277.15, 25.92) --
	(277.21, 25.91) --
	(277.21, 25.91) --
	(277.21, 25.91) --
	(277.28, 25.92) --
	(277.28, 25.92) --
	(277.28, 25.92) --
	(277.35, 25.92) --
	(277.35, 25.92) --
	(277.35, 25.92) --
	(277.41, 25.92) --
	(277.41, 25.92) --
	(277.41, 25.92) --
	(277.48, 25.93) --
	(277.48, 25.93) --
	(277.48, 25.93) --
	(277.55, 25.92) --
	(277.55, 25.92) --
	(277.55, 25.92) --
	(277.61, 25.92) --
	(277.61, 25.92) --
	(277.61, 25.92) --
	(277.68, 25.92) --
	(277.68, 25.92) --
	(277.68, 25.92) --
	(277.75, 25.92) --
	(277.75, 25.92) --
	(277.75, 25.92) --
	(277.81, 25.93) --
	(277.81, 25.93) --
	(277.81, 25.93) --
	(277.88, 25.92) --
	(277.88, 25.92) --
	(277.88, 25.92) --
	(277.95, 25.91) --
	(277.95, 25.91) --
	(277.95, 25.91) --
	(278.01, 25.92) --
	(278.01, 25.92) --
	(278.01, 25.92) --
	(278.08, 25.91) --
	(278.08, 25.91) --
	(278.08, 25.91) --
	(278.15, 25.92) --
	(278.15, 25.92) --
	(278.15, 25.92) --
	(278.21, 25.92) --
	(278.21, 25.92) --
	(278.21, 25.92) --
	(278.28, 25.91) --
	(278.28, 25.91) --
	(278.28, 25.91) --
	(278.35, 25.92) --
	(278.35, 25.92) --
	(278.35, 25.92) --
	(278.41, 25.92) --
	(278.41, 25.92) --
	(278.41, 25.92) --
	(278.48, 25.92) --
	(278.48, 25.92) --
	(278.48, 25.92) --
	(278.55, 25.92) --
	(278.55, 25.92) --
	(278.55, 25.92) --
	(278.61, 25.92) --
	(278.61, 25.92) --
	(278.61, 25.92) --
	(278.68, 25.92) --
	(278.68, 25.92) --
	(278.68, 25.92) --
	(278.75, 25.92) --
	(278.75, 25.92) --
	(278.75, 25.92) --
	(278.82, 25.93) --
	(278.82, 25.93) --
	(278.82, 25.93) --
	(278.88, 25.92) --
	(278.88, 25.92) --
	(278.88, 25.92) --
	(278.95, 25.93) --
	(278.95, 25.93) --
	(278.95, 25.93) --
	(279.02, 25.91) --
	(279.02, 25.91) --
	(279.02, 25.91) --
	(279.08, 25.92) --
	(279.08, 25.92) --
	(279.08, 25.92) --
	(279.15, 25.93) --
	(279.15, 25.93) --
	(279.15, 25.93) --
	(279.22, 25.92) --
	(279.22, 25.92) --
	(279.22, 25.92) --
	(279.28, 25.93) --
	(279.28, 25.93) --
	(279.28, 25.93) --
	(279.35, 25.93) --
	(279.35, 25.93) --
	(279.35, 25.93) --
	(279.42, 25.92) --
	(279.42, 25.92) --
	(279.42, 25.92) --
	(279.48, 25.92) --
	(279.48, 25.92) --
	(279.48, 25.92) --
	(279.55, 25.92) --
	(279.55, 25.92) --
	(279.55, 25.92) --
	(279.62, 25.92) --
	(279.62, 25.92) --
	(279.62, 25.92) --
	(279.68, 25.93) --
	(279.68, 25.93) --
	(279.68, 25.93) --
	(279.75, 25.92) --
	(279.75, 25.92) --
	(279.75, 25.92) --
	(279.82, 25.92) --
	(279.82, 25.92) --
	(279.82, 25.92) --
	(279.88, 25.92) --
	(279.88, 25.92) --
	(279.88, 25.92) --
	(279.95, 25.92) --
	(279.95, 25.92) --
	(279.95, 25.92) --
	(280.02, 25.93) --
	(280.02, 25.93) --
	(280.02, 25.93) --
	(280.08, 25.93) --
	(280.08, 25.93) --
	(280.08, 25.93) --
	(280.15, 25.92) --
	(280.15, 25.92) --
	(280.15, 25.92) --
	(280.22, 25.93) --
	(280.22, 25.93) --
	(280.22, 25.93) --
	(280.28, 25.93) --
	(280.28, 25.93) --
	(280.28, 25.93) --
	(280.35, 25.92) --
	(280.35, 25.92) --
	(280.35, 25.92) --
	(280.42, 25.93) --
	(280.42, 25.93) --
	(280.42, 25.93) --
	(280.48, 25.92) --
	(280.48, 25.92);

\node[text=drawColor,anchor=base west,inner sep=0pt, outer sep=0pt, scale=  0.57] at (163.55, 50.62) {Mercury arc};
\definecolor{drawColor}{gray}{0.20}

\path[draw=drawColor,line width= 0.5pt,line join=round,line cap=round] (145.79, 25.90) rectangle (270.13, 61.21);
\end{scope}
\begin{scope}
\path[clip] (  0.00,  0.00) rectangle (274.63,105.51);
\definecolor{drawColor}{gray}{0.20}

\path[draw=drawColor,line width= 0.5pt,line join=round] ( 38.75, 23.65) --
	( 38.75, 25.90);

\path[draw=drawColor,line width= 0.5pt,line join=round] ( 71.04, 23.65) --
	( 71.04, 25.90);

\path[draw=drawColor,line width= 0.5pt,line join=round] (103.34, 23.65) --
	(103.34, 25.90);

\path[draw=drawColor,line width= 0.5pt,line join=round] (135.64, 23.65) --
	(135.64, 25.90);
\end{scope}
\begin{scope}
\path[clip] (  0.00,  0.00) rectangle (274.63,105.51);
\definecolor{drawColor}{RGB}{0,0,0}

\node[text=drawColor,anchor=base,inner sep=0pt, outer sep=0pt, scale=  0.81] at ( 38.75, 16.27) {400};

\node[text=drawColor,anchor=base,inner sep=0pt, outer sep=0pt, scale=  0.81] at ( 71.04, 16.27) {600};

\node[text=drawColor,anchor=base,inner sep=0pt, outer sep=0pt, scale=  0.81] at (103.34, 16.27) {800};

\node[text=drawColor,anchor=base,inner sep=0pt, outer sep=0pt, scale=  0.81] at (135.64, 16.27) {1000};
\end{scope}
\begin{scope}
\path[clip] (  0.00,  0.00) rectangle (274.63,105.51);
\definecolor{drawColor}{gray}{0.20}

\path[draw=drawColor,line width= 0.5pt,line join=round] (167.59, 23.65) --
	(167.59, 25.90);

\path[draw=drawColor,line width= 0.5pt,line join=round] (199.88, 23.65) --
	(199.88, 25.90);

\path[draw=drawColor,line width= 0.5pt,line join=round] (232.18, 23.65) --
	(232.18, 25.90);

\path[draw=drawColor,line width= 0.5pt,line join=round] (264.47, 23.65) --
	(264.47, 25.90);
\end{scope}
\begin{scope}
\path[clip] (  0.00,  0.00) rectangle (274.63,105.51);
\definecolor{drawColor}{RGB}{0,0,0}

\node[text=drawColor,anchor=base,inner sep=0pt, outer sep=0pt, scale=  0.81] at (167.59, 16.27) {400};

\node[text=drawColor,anchor=base,inner sep=0pt, outer sep=0pt, scale=  0.81] at (199.88, 16.27) {600};

\node[text=drawColor,anchor=base,inner sep=0pt, outer sep=0pt, scale=  0.81] at (232.18, 16.27) {800};

\node[text=drawColor,anchor=base,inner sep=0pt, outer sep=0pt, scale=  0.81] at (264.47, 16.27) {1000};
\end{scope}
\begin{scope}
\path[clip] (  0.00,  0.00) rectangle (274.63,105.51);
\definecolor{drawColor}{RGB}{0,0,0}

\node[text=drawColor,anchor=base,inner sep=0pt, outer sep=0pt, scale=  0.90] at (143.54,  6.25) {$\lambda/\unit{\nm}$};
\end{scope}
\begin{scope}
\path[clip] (  0.00,  0.00) rectangle (274.63,105.51);
\definecolor{drawColor}{RGB}{0,0,0}

\node[text=drawColor,rotate= 90.00,anchor=base,inner sep=0pt, outer sep=0pt, scale=  0.90] at ( 10.70, 63.46) {$I$/au};
\end{scope}
\end{tikzpicture}

\end{document}
